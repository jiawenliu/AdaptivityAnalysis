\documentclass[a4paper,11pt]{article}

\usepackage{mathpartir}
\usepackage{amsmath,amsthm,amsfonts}
\usepackage{ amssymb }
\usepackage{color}
\usepackage{algorithm}
\usepackage{algorithmic}
\usepackage{microtype}


\newcommand{\defeq}{\mathrel{\doteq}}

\newcommand{\lzero}{0}

\newcommand{\kw}[1]{\mathtt{#1}}

\newcommand{\expr}{e}
\newcommand{\vall}{w}
\newcommand{\valr}{v}
\newcommand{\eif}{\kw{if}}
\newcommand{\eapp}{\;}
\newcommand{\eprojl}{\kw{fst}}
\newcommand{\eprojr}{\kw{snd}}
%\newcommand{\eprov}[1]{\eta_{#1}}
\newcommand{\etrue}{\kw{true}}
\newcommand{\efalse}{\kw{false}}
\newcommand{\econst}{c}
\newcommand{\eop}{\delta}
\newcommand{\efix}{\mathop{\kw{fix}}}
%\newcommand{\labelA}{\ell}

\newcommand{\tr}{T}
\newcommand{\trift}{\eif^{\kw{t}}}
\newcommand{\triff}{\eif^{\kw{f}}}
\newcommand{\trprojl}{\eprojl}
\newcommand{\trprojr}{\eprojr}
\newcommand{\trtrue}{\etrue}
\newcommand{\trfalse}{\efalse}
\newcommand{\trconst}{\econst}
\newcommand{\trop}{\eop}
\newcommand{\trfix}{\efix}
\newcommand{\trapp}[5]{#1 \; #2 \mathrel{\triangleright} {\efix #3(#4).#5}}

\newcommand{\adap}{\kw{adap}}
\newcommand{\ddep}[1]{\kw{depth}_{#1}}
\newcommand{\nat}{\mathbb{N}}
\newcommand{\natb}{\nat_{\bot}}
\newcommand{\natbi}{\natb^\infty}
\newcommand{\nnatA}{n}
\newcommand{\nnatB}{m}
\newcommand{\nnatbA}{s}
\newcommand{\nnatbB}{t}
\newcommand{\nnatbiA}{q}
\newcommand{\nnatbiB}{r}

\newcommand{\type}{\tau}
\newcommand{\tbase}{\kw{b}}
\newcommand{\tbool}{\kw{bool}}
\newcommand{\tarr}[5]{#1; #3 \xrightarrow{#4; \, #5} #2}
\newcommand{\env}{\theta}

\newcommand{\bigstep}{\mathrel{\Downarrow}}

\newcommand{\dmap}{\rho}
\newcommand{\dmapb}{\bot_\dmap}
\newcommand{\supp}{\kw{supp}}
\newcommand{\dom}{\kw{dom}}

\newcommand{\tvdash}[1]{\vdash_{#1}}

%Packages
\usepackage[T1]{fontenc}
\usepackage{fourier} 
\usepackage[english]{babel} 
\usepackage{amsmath,amsfonts} 
\usepackage{amsthm} 
\usepackage{color}   %May be necessary if you want to color links
\usepackage{hyperref}
\usepackage{lscape}
\usepackage{geometry}
\usepackage{amsmath}
\usepackage{algorithm}
\usepackage{algorithmic}
\usepackage{amssymb}
\usepackage{amsfonts}
\usepackage{times}
\usepackage{bm}
\usepackage{ stmaryrd }
\SetSymbolFont{stmry}{bold}{U}{stmry}{m}{n}

\usepackage{ amssymb }
\usepackage{ textcomp }
\usepackage[normalem]{ulem}
% For derivation rules
\usepackage{mathpartir}
\usepackage{color}
\usepackage{a4wide}
\usepackage{caption}
\usepackage{subcaption}
\usepackage{mathpartir}
\usepackage{amsmath,amsfonts}
\usepackage{ amssymb }
\usepackage{color}
\usepackage{algorithm}
\usepackage{algorithmic}
\usepackage{microtype}
\usepackage{eucal}
\usepackage{url}
\usepackage{xspace}
\usepackage{array}
\usepackage{listings}

\usepackage{tikz}
\usetikzlibrary{shapes.geometric}
\usetikzlibrary{arrows.meta,arrows}
\usetikzlibrary{decorations.text}
% % % % 


\usepackage{multirow}


%%%%%%%%%%%%%%%%%%%%%%%%%%%%%%%%%%%%%%%%%%%%%%%%%%%%%Packages And Definitions For Listing the Code%%%%%%%%%%%%%%%%%%%%%%%%%%%%%%%%%%%%%%%%%%%%%%%%%%%%%%%%%%%%%%%%%%%%%%%%
\usepackage{listings}
\usepackage{xcolor}

\definecolor{codegreen}{rgb}{0,0.6,0}
\definecolor{codegray}{rgb}{0.5,0.5,0.5}
\definecolor{codepurple}{rgb}{0.58,0,0.82}
\definecolor{backcolour}{rgb}{0.95,0.95,0.92}

\lstdefinestyle{mystyle}{
    backgroundcolor=\color{backcolour},   
    commentstyle=\color{codegreen},
    keywordstyle=\color{magenta},
    numberstyle=\tiny\color{codegray},
    stringstyle=\color{codepurple},
    basicstyle=\ttfamily\footnotesize,
    breakatwhitespace=false,         
    breaklines=true,                 
    captionpos=b,                    
    keepspaces=true,                 
    numbers=left,                    
    numbersep=5pt,                  
    showspaces=false,                
    showstringspaces=false,
    showtabs=false,                  
    tabsize=2
}

\lstset{style=mystyle}
\newcommand{\aexpr}{a}
\newcommand{\bexpr}{b}
\newcommand{\cmd}{c}
\newcommand{\node}{N}
\newcommand{\assign}[2]{ \mathrel{ #1  \leftarrow #2 } }
\newcommand{\fassign}[3]{ \mathrel{ #1  \leftarrow^{#3}  \delta^{#3}(
    #2 ) } }
\newcommand{\impif}[3]{\mathrel{\eif \eapp #1\eapp \ethen \eapp #2 \eapp
    \eelse \eapp #3 }}
\newcommand{\impwhile}[2]{\mathrel{ \kw{while} (#1) \eapp #2 } }
\newcommand{\labl}{l}

\let\originalleft\left
\let\originalright\right
\renewcommand{\left}{\mathopen{}\mathclose\bgroup\originalleft}
\renewcommand{\right}{\aftergroup\egroup\originalright}

\theoremstyle{definition}

\title{Clairvoyant Semantics for Adaptivity Analysis}

\author{}

\date{}

\begin{document}

\maketitle



\[\begin{array}{llll}
    \mbox{Pure Expr.} & \expr & ::= & \econst ~|~ x ~|~ \expr \eapp \expr 
                                      ~|~ \lambda x. \expr  ~|~
                                      \epair{\expr}{\expr} ~|~
                                      \eprojl\, \expr ~|~ \eprojr\, \expr\\ 
                      & & &  \etrue ~|~ \efalse ~|~ \eif (\expr, \expr , \expr)  ~|~ \emonadic{m} \\
    % 
    \mbox{Monadic Expr.} & \mexpr & ::= & \mreturn(\expr) ~|~ \mlet{x}{\expr}{\mexpr} ~|~ \eop(\mexpr)\\
    %
    \mbox{Value.} & v & ::= & \econst ~|~ \lambda x. \expr ~|~ \pair{\valr_1}{\valr_2} ~|~  \etrue ~|~ \efalse  ~|~ \emonadic{m}\\
    %
    \mbox{Types} & \type &::=  &  \tbase ~|~ \type \to \type ~|~ \tmonad(\type)  \\
  \end{array}\]



\begin{definition}
  Let $\vdash \mexpr : \tmonad(\type)$, its \emph{adaptivity} is defined as:
  $$
  \min\{\adapt ~|~ \exists \expr',\valr. \mexpr\bigstepf{\adapt} \expr' \land \expr' \bigstep \valr \}
  $$
\end{definition}


\begin{figure}[h]
  \begin{mathpar}
    \inferrule{}{
     \tctx, x:\type \vdash x :  \type
    }~\textbf{ST-AX}
    %
    \and
    %
    \inferrule{
    }{
      \tctx \vdash c: \tbase
    }~\textbf{ST-CST}   
    \and
    %
    \inferrule{
      \tctx, x: \type_1 \vdash \expr : \type_2
    }{
     \tctx \vdash \lambda x. \expr  :\type_1 \to \type_2
    }~\textbf{ST-LAM}
    \and
    %
    \inferrule{
       \tctx \vdash\expr_1:  \type_1 \to \type_2      \\
      \tctx \vdash\expr_2 : \type_1
    }{
      \tctx \vdash ( \expr_1 \eapp \expr_2) : \type_2
    }~\textbf{ST-app}
    %
    \\
    ....
  \end{mathpar}
  \caption{Simple Types - pure rules}
  \label{fig:simple-types-pure}
\end{figure}

 \begin{figure}[h]
  \begin{mathpar}
    \inferrule{
      \tctx \vdash \expr :  \type
    }{
     \tctx \vdash \return(\expr) : \tmonad( \type)
    }~\textbf{ST-RET}
    %
    \and
    %
    \inferrule{
      \tctx \vdash \expr :  \tmonad(\type_1)
      \\
      \tctx, x: \type_1 \vdash \mexpr :  \tmonad(\type_2)
    }{
      \tctx \vdash \mlet{\expr}{x}{\mexpr} : \tmonad(\type_2)
    }~\textbf{ST-LET}
    %
    \and
    %
    \inferrule{
      \tctx \vdash \expr  :  \tmonad({\tt row})
      \\
      \hat{\delta}:{\tt row} \to [0,1]
   }{  \tctx \vdash \delta(\expr) : \tmonad([0,1])
    }~\textbf{ST-DELTA}
  \end{mathpar}
  \caption{Simple Types - monadic rules}
  \label{fig:simple-types-monadic}
\end{figure}


\begin{figure}[h]
  \begin{mathpar}
    \inferrule{
    }{
     \valr \bigstep \valr } ~\textsf{S-VAL}
   \and
   %
   \and
  %
  \inferrule{
    \expr_1 \bigstep \lambda x.\expr  \\
     \expr_2 \bigstep \valr_2  \\
   \expr[\valr_2/x] \bigstep \valr_3
  }{
     \expr_1 \eapp \expr_2 \bigstep \valr_3
   }~\textsf{S-APP}
    %
  \and
 %
 \inferrule{
  \expr_1 \bigstep  \etrue
   \\
   \expr_2  \bigstep \valr_2
  }{
   \env, \eif (\expr_1 ,\expr_2 , \expr_3) \bigstep \valr_2
  }~\textsf{S-IFT}
\and
 \inferrule{
  \env, \expr_1 \bigstep{\adapt_1} \env_1, \etrue
   \\
   \expr_3  \bigstep \valr_3
  }{
   \eif (\expr_1 ,\expr_2 , \expr_3) \bigstep  \valr_3
  }~\textsf{S-IFF}
  % %
 %
  \and
  %
  \inferrule{
     \expr_1  \bigstep \valr_1 \\
     \expr_2  \bigstep \valr_2 \\
  }{
   \pair{\expr_1}{\expr_2} \bigstep
   \pair{\valr_1}{\valr_2}
 }~\textsf{S-PROD}
 %
  \and
  %
  \inferrule{
    \expr  \bigstep (\valr_1, \valr_2) 
  }{
   \eprojl{\, \expr} \bigstep \valr_1
 }~\textsf{S-PL}
 %
  \and
  %
  \inferrule{
    \expr  \bigstep (\valr_1, \valr_2) 
  }{
   \eprojr{(\expr)} \bigstep \valr_2
 }~\textsf{S-PR}
\end{mathpar}
  \caption{Big-step semantics, pure  rules.}
  \label{fig:semantics-pure}
\end{figure}


\begin{figure}[h]
  \begin{mathpar}
    \inferrule{
    }{
     \return(\expr) \bigstepf{0}{0} \expr } ~\textsf{FS-RET}
   \and
   %
  \inferrule{
    \expr\bigstep \emonadic{\mexpr_1}  \\
     \mexpr_1\bigstepf{\adapt_1} \expr_1  \\
     \mexpr[\expr_1/x] \bigstepf{\adapt_2} \expr_2  \\
  }{
     \mlet{x}{\expr}{\mexpr} \bigstepf{\adapt_1+\adapt_2} \expr_2
   }~\textsf{S-APP-A}
    %
 \and
   %
  \inferrule{
     \mexpr\bigstepf{\adapt_2} \expr_2  \\
   }{
     \mlet{x}{\expr_1}{\mexpr} \bigstepf{\adapt_2} \expr_2
   }~\textsf{S-APP-Z}
   \and
  \inferrule{
    \mexpr \bigstepf{\adapt} \expr  \\
    \expr \bigstep \valr\\
    \hat{\eop}(\valr)=\valr'
  }{
     \eop(\mexpr)  \bigstepf{\adapt+1} \valr'
   }~\textsf{S-APP}
    %
\end{mathpar}
  \caption{Big-step semantics, forcing rules.}
  \label{fig:semantics-forcing}
\end{figure}



\begin{figure}[h]
  \begin{mathpar}
    \inferrule{}{
     \tctx, x:!_1\type \vdash x :  \type
    }~\textbf{ST-AX}
    %
    \and
    %
    \inferrule{
    }{
      \tctx \vdash c: \tbase
    }~\textbf{ST-CST}   
    \and
    %
    \inferrule{
      \tctx, x: !_i \type_1 \vdash \expr : \type_2
    }{
     \tctx \vdash \lambda x. \expr  :!_i \type_1 \to \type_2
    }~\textbf{ST-LAM}
    \and
    %
    \inferrule{
       \tctx_1 \vdash\expr_1:  !_i\type_1 \to \type_2      \\
      \tctx_2 \vdash\expr_2 : \type_1
    }{
      \tctx_1+i*\tctx_2 \vdash ( \expr_1 \eapp \expr_2) : \type_2
    }~\textbf{ST-app}
    %
    \\
    ....
  \end{mathpar}
  \caption{Simple Types - pure rules}
  \label{fig:simple-types-pure}
\end{figure}

 \begin{figure}[h]
  \begin{mathpar}
    \inferrule{
      \tctx \vdash \expr :  \type
    }{
     \tctx \vdash \return(\expr) : \tmonad( \type)
    }~\textbf{ST-RET}
    %
    \and
    %
    \inferrule{
      \tctx \vdash \expr :  \tmonad(\type_1)
      \\
      \tctx, x: \type_1 \vdash \mexpr :  \tmonad(\type_2)
    }{
      \tctx \vdash \mlet{\expr}{x}{\mexpr} : \tmonad(\type_2)
    }~\textbf{ST-LET}
    %
    \and
    %
    \inferrule{
      \tctx \vdash \expr  :  \tmonad({\tt row})
      \\
      \hat{\delta}:{\tt row} \to [0,1]
   }{  \tctx \vdash \delta(\expr) : \tmonad([0,1])
    }~\textbf{ST-DELTA}
  \end{mathpar}
  \caption{Simple Types - monadic rules}
  \label{fig:simple-types-monadic}
\end{figure}


  
\end{document}



