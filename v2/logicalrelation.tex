Our type system is sound in the following sense.

\begin{thm}[Soundness]\label{thm:soundness}
If $\tvdash{\nnatA} \expr: \type$ and $\cdot, \expr \bigstep \valr,
\tr$ then $\adap(\tr) \leq \nnatA$.
\end{thm}

To prove this theorem, we build a logical relation on types. As usual,
the relation consists of a value relation and an expression
relation. We first show a non-step-indexed version of the relation in
Figure~\ref{fig:lr:non-step}. This relation is well-founded but
cannot, as usual, be used directly to prove the soundness of the
$\efix$ typing rule. We show this relation here purely for exposition
without the clutter of step-indices. We will step-index the relation
shortly.

\begin{figure}
  \begin{mathpar}
    \begin{array}{lll}
      \lrv{\tbool} & = & \{\etrue, \efalse\} \\
      %
      \lrv{\tbase} & = & \{\econst ~|~ \econst: \tbase \} \\
      %
      \lrv{\type_1 \times \type_2} & = & \{(\valr_1, \valr_2) ~|~ \valr_1 \in \lrv{\type_1} \conj \valr_2 \in \lrv{\type_2} \}\\
      %
      \lrv{\tarr{\type_1}{\type_2}{\nnatbiA}{\dmap}{\nnatA}} & = &
      \{(\efix f(x).\expr, \env) ~|~ \forall \valr \in \lrv{\type_1}.\\
      & & 
      ~~~~~~(\env[x \mapsto \valr, f \mapsto (\efix f(x).\expr, \env)], \expr) \in \lre{\dmap[x: \nnatbiA, f: \infty]}{\nnatA}{\type_2}\} \\
      %
      \\
      %
      \lre{\dmap}{\nnatA}{\type} & = & \{ (\env, \expr) ~|~ \forall \valr\eapp  \tr.\eapp  (\env, \expr \bigstep \valr, \tr) \\
      & & ~~~~~~~~~~~~~~~~~\Rightarrow (\adap(\tr) \leq \nnatA \conj \\
      & & ~~~~~~~~~~~~~~~~~~~~~~~\forall x \in \mbox{Vars}.\eapp  \ddep{x}(\tr) \leq \dmap(x) \conj \\
      & & ~~~~~~~~~~~~~~~~~~~~~~~\valr \in \lrv{\type})
      \}
    \end{array}
  \end{mathpar}
  \caption{Logical relation without step-indexing}
  \label{fig:lr:non-step}
\end{figure}

The value relation $\lrv{\type}$, as usual, defines which
\emph{closed} values are semantically in the type $\type$. The only
interesting case of this relation is the case for the arrow type. The
expression relation $\lre{\dmap}{\nnatA}{\type}$ is more
interesting. It is indexed by a depth map and an
adaptivity. Importantly, this relation contains sets of configurations
$(\env, \expr)$ where the evaluation of $(\env, \expr)$ produces a
value in the value relation at $\type$, the adaptivity of the trace is
no more than $\nnatA$ and for any variable $x$, $\ddep{x}$ of the
trace is no more than $\dmap(x)$.

The value relation can be lifted to contexts in the usual way. Say
that $\env \in \lrv{\Gamma}$ when $\dom(\env) \supseteq \dom(\Gamma)$
and for every $x \in \dom(\Gamma)$, $\env(x) \in \lrv{\Gamma(x)}$.

The fundamental theorem we would \emph{like} to prove is the
following. Of course, this cannot be proven until we step-index the
relation, but this should give an idea of the connection between the
type system and the logical relation.

\begin{prop}[Fundamental theorem]
\label{prop:fund}
  If $\Gamma; \dmap \tvdash{\nnatA} \expr: \type$ and $ \env \in
  \lrv{\Gamma}$, then $(\env, \expr) \in \lre{\dmap}{\nnatA}{\type}$.
\end{prop}


If this proposition were provable, then Theorem~\ref{thm:soundness}
would follow from it immediately.

\paragraph{Step-indexed logical relation}
To step-index the relation we need to have a notion of the number of
reduction steps. One way to do this is to change the operational
semantics to count the number of reduction steps. However, this is
unnecessary since the number of reduction steps in a derivation is
exactly the size of the derivation's trace. So, we just ``index'' the
relation on the size of the derivation's trace. The same technique has
been used previously~\cite{cicek15}.

\begin{defn}[Trace size]
  The size of a trace is defined as the number of
  $\trapp{\tr_1}{\tr_2}{f}{x}{\tr_3}$, $\trift(\tr_b, \tr_t)$,
  $\triff(\tr_b, \tr_f)$ and $\trop(\tr)$ constructors in
  it. Basically, we count all elimination constructors, but not
  introduction constructors.\footnote{Deepak's note: The purpose of
    counting this way is to make sure that the trace of the evaluation
    of any \emph{value} has size $0$. This may be required in the
    proof of the fundamental theorem, but I am not sure.}

\end{defn}

\begin{defn}[Trace size]
  We use $\size{\tr}$ to denote the size of the trace $\tr$.
\end{defn}
\begin{figure}
  \framebox{$\size{\tr}: \mbox{Traces} \to \nat$}
  \begin{mathpar}
    \begin{array}{lcl}
      \size{x} & = & 0 \\
      %
      \size{\trapp{\tr_1}{\tr_2}{f}{x}{\tr_3}} & = &
      \size{\tr_1} + \size{\tr_2} + \size{\tr_3} + 1 \\
      %
      \size{\trfix f(x).e} & = & 0 \\
      %
      \size{(\tr_1, \tr_2)} & = & \size{\tr_1} + \size{\tr_2} \\
      %
      \size{\trprojl(\tr)} & = & \size{\tr} +1 \\
      %
      \size{\trprojr(\tr)} & = & \size{\tr} + 1 \\
      %
      \size{\trtrue} & = & 0 \\
      %
      \size{\trfalse} & = & 0 \\
      %
      \size{\trift(\tr_b, \tr_t)} & = & \size{\tr_b} + \size{\tr_t} + 1 \\
      %
      \size{\triff(\tr_b, \tr_f)} & = & \size{\tr_b} + \size{\tr_f} + 1 \\
      %
      \size{\trconst} & = & 0 \\
      %
      \size{\trop(\tr)} & = & \size{\tr} + 1 \\
      %
      \size{\trnil} & = & 0 \\
      %
      \size{\trcons(\tr_1,\tr_2)} & = & \size{\tr_1} + \size{\tr_2} \\
      % 
      \size{ \trlet(x, \tr_1, \tr_2) } & = & \size{\tr_1} + \size{\tr_2}\\
      %
      \boxed{\color{red} \size{\triapp{\tr_1}{\tr_2}}} & = & \size{\tr_1} + \size{\tr_2}\\
      %
      \boxed{\color{red} \size{\eilam \expr}} & = & 0\\
      \end{array}
  \end{mathpar}
  \caption{Size of a trace}
  \label{fig:size}
\end{figure}



The step-indexed relation is shown in Figure~\ref{fig:lr:step}. Now,
the value relation $\lrv{\type}$ contains pairs of step-indices
$\stepiA$ and closed values. The expression relation contains pairs of
step-indices and configurations $(\env, \expr)$. The relation
basically mirrors the non-step-indexed relation of
Figure~\ref{fig:lr:non-step}, with step indexes added in the
completely standard way.


\begin{figure}
  \begin{mathpar}
    \begin{array}{lll}
      \lrv{\tbool} & = & \{(\stepiA, \etrue) ~|~ \stepiA \in \nat\} \cup
      \{ (\stepiA, \efalse) ~|~ \stepiA \in \nat\} \\
      %
      \lrv{\tbase} & = & \{(\stepiA, \econst) ~|~ \stepiA \in \nat \conj \econst: \tbase \} \\
      %
      \lrv{\type_1 \times \type_2} & = & \{(\stepiA, (\valr_1, \valr_2)) ~|~ (\stepiA, \valr_1) \in \lrv{\type_1} \conj (\stepiA, \valr_2) \in \lrv{\type_2} \}\\
      %
      \lrv{\tarr{\type_1}{\type_2}{\nnatbiA}{\dmap}{\nnatA}} & = &
      \{(\stepiA, (\efix f(x).\expr, \env)) ~|~ \forall \stepiB < \stepiA.\eapp  \forall (\stepiB, \valr) \in \lrv{\type_1}.\\
      & & 
      ~~~~~~(\stepiB, (\env[x \mapsto \valr, f \mapsto (\efix f(x).\expr, \env)], \expr)) \in \lre{\dmap[x: \nnatbiA, f: \infty]}{\nnatA}{\type_2}\} \\
      %
     \boxed{ \lrv{\tlist{\type}}  } & = & \{  (\stepiA, \enil) ~|~ \stepiA \in
                                \nat \} \cup \{  (\stepiA,
                                \econs(\valr_1,\valr_2) ) ~|~
                                (\stepiA, \valr_1) \in \lrv{\type}
                                \land (\stepiA, \valr_2) \in \lrv{\tlist{\type}} \}
      \\
      %
      \\ 
      %
      \lre{\dmap}{\nnatA}{\type} & = & \{ (\stepiA, (\env, \expr)) ~|~ \forall \valr\eapp  \tr\eapp  \stepiB.\eapp  (\env, \expr \bigstep \valr, \tr) \conj (\size{\tr} = \stepiB) \conj (\stepiB \leq \stepiA) \\
      & & ~~~~~~~~~~~~~~~~~~~~~~~~~\Rightarrow (\adap(\tr) \leq \nnatA \conj \\
      & & ~~~~~~~~~~~~~~~~~~~~~~~~~~~~~~~\forall x \in \mbox{Vars}.\eapp  \ddep{x}(\tr) \leq \dmap(x) \conj \\
      & & ~~~~~~~~~~~~~~~~~~~~~~~~~~~~~~~((\stepiA - \stepiB,  \valr) \in \lrv{\type})
      \}\\
      %	
      \boxed{\color{red} \lrv{\tint}} & = & \{(\stepiA, i) ~|~ \stepiA \in \nat \conj i : \tint \}\\
      %
      \boxed{\color{red} \lrv{\treal}} & = & \{(\stepiA, r) ~|~ \stepiA \in \nat \conj r : \treal \}\\
      %  
      \boxed{\color{red} \lrv{\tbox{\type}}} & = & \{(\stepiA, \valr) ~|~ \stepiA \in \nat \conj (\stepiA,\valr) \in \lrv{\type} \conj \eop \notin \valr \}\\
      %
      \boxed{\color{red} \lrv{\tforall{\dmap}{\nnatA}{i} \type} } & = & \{(\stepiA, (\eilam \expr, \env)) ~|~ \stepiA \in \nat \conj \forall I. ~  \tvdash{} I ::S, (\stepiA, \expr) \in \lre{\dmap}{\nnatA[I/i]}{\type[I/i]} \}\\
      %
      \boxed{\color{red} \lrv{ \tint[I] } } & = & \{(\stepiA, n) ~|~ \stepiA \in \nat \conj n = I \}\\
  \end{array}
  \end{mathpar}
  \caption{Logical relation with step-indexing}
  \label{fig:lr:step}
\end{figure}

We say that $(\stepiA, \env) \in \lrv{\Gamma}$ when $\dom(\env)
\supseteq \dom(\Gamma)$ and for every $x \in \dom(\Gamma)$, $(\stepiA,
\env(x)) \in \lrv{\Gamma(x)}$.

{\color{red} We say that $\ienv \in \lrv{\Delta} $ when $\dom(\ienv)
  \supseteq \dom(\Delta) $ and for every $x \in \dom(\Delta) $, $\cdot \vdash \ienv(x) :: \Delta(x)$ }
