
\documentclass[a4paper,11pt]{article}

\usepackage{mathpartir}
\usepackage{amsmath,amsthm,amsfonts}
\usepackage{color}
\usepackage{algorithm}
\usepackage{algorithmic}

\newcommand{\defeq}{\mathrel{\doteq}}

\newcommand{\lzero}{0}

\newcommand{\kw}[1]{\mathtt{#1}}

\newcommand{\expr}{e}
\newcommand{\vall}{w}
\newcommand{\valr}{v}
\newcommand{\eif}{\kw{if}}
\newcommand{\eapp}{\;}
\newcommand{\eprojl}{\kw{fst}}
\newcommand{\eprojr}{\kw{snd}}
%\newcommand{\eprov}[1]{\eta_{#1}}
\newcommand{\etrue}{\kw{true}}
\newcommand{\efalse}{\kw{false}}
\newcommand{\econst}{c}
\newcommand{\eop}{\delta}
\newcommand{\efix}{\mathop{\kw{fix}}}
%\newcommand{\labelA}{\ell}

\newcommand{\tr}{T}
\newcommand{\trift}{\eif^{\kw{t}}}
\newcommand{\triff}{\eif^{\kw{f}}}
\newcommand{\trprojl}{\eprojl}
\newcommand{\trprojr}{\eprojr}
\newcommand{\trtrue}{\etrue}
\newcommand{\trfalse}{\efalse}
\newcommand{\trconst}{\econst}
\newcommand{\trop}{\eop}
\newcommand{\trfix}{\efix}
\newcommand{\trapp}[5]{#1 \; #2 \mathrel{\triangleright} {\efix #3(#4).#5}}

\newcommand{\adap}{\kw{adap}}
\newcommand{\ddep}[1]{\kw{depth}_{#1}}
\newcommand{\nat}{\mathbb{N}}
\newcommand{\natb}{\nat_{\bot}}
\newcommand{\natbi}{\natb^\infty}
\newcommand{\nnatA}{n}
\newcommand{\nnatB}{m}
\newcommand{\nnatbA}{s}
\newcommand{\nnatbB}{t}
\newcommand{\nnatbiA}{q}
\newcommand{\nnatbiB}{r}

\newcommand{\type}{\tau}
\newcommand{\tbase}{\kw{b}}
\newcommand{\tbool}{\kw{bool}}
\newcommand{\tarr}[5]{#1; #3 \xrightarrow{#4; \, #5} #2}
\newcommand{\env}{\theta}

\newcommand{\bigstep}{\mathrel{\Downarrow}}

\newcommand{\dmap}{\rho}
\newcommand{\dmapb}{\bot_\dmap}
\newcommand{\supp}{\kw{supp}}
\newcommand{\dom}{\kw{dom}}

\newcommand{\tvdash}[1]{\vdash_{#1}}


\theoremstyle{definition}
\newtheorem{thm}{Theorem}
\newtheorem{lem}[thm]{Lemma}
\newtheorem{cor}[thm]{Corollary}
\newtheorem{prop}[thm]{Proposition}
\newtheorem{defn}[thm]{Definition}

\title{Adaptivity analysis}

\author{}

\date{January 19, 2019}

\begin{document}

\maketitle


\[\begin{array}{llll}
\mbox{Term} & \term & ::= &     \eilam \expr  ~|~  \expr \eapp []  ~|~
                            \epack \expr ~|~ \eunpack \expr \eas x
                            \ein \expr \\
\mbox{Index Term} & \idx, \nnatA & ::= &     i ~|~ n ~|~ \idx_1 + \idx_2 ~|~  \idx_1
                                 - \idx_2 ~|~ \smax{\idx_1}{\idx_2}\\
  \mbox{Sort} & S & ::= & \natbi \\
  \mbox{Type} & \type & ::= & \tforall{i} \type  ~|~ \texists{i} \type 
\end{array}\]


\begin{figure}
  \begin{mathpar}
    \inferrule{
      \Delta, i; \Gamma ;\dmap \tvdash{\nnatA} \expr: \type
    }{
     \Delta;  \Gamma; \dmap \tvdash{\nnatA}    \eilam \expr    :  \tforall{i} \type 
    }
    %
    \and
    %
  \inferrule{
      \Delta; \Gamma ;\dmap \tvdash{\nnatA} \expr: \tforall{i} \type
      \\
       \Delta \tvdash{}  I ::  S 
    }{
     \Delta;  \Gamma; \dmap \tvdash{\nnatA[I/i]}    \expr \eapp []   :
     \type[ I/i]
    }
  \end{mathpar}
  \caption{typing rules - monad}
  \label{fig:type-rules-monad}
\end{figure}


\newpage
\begin{figure}

Multi-rounds:
\[
\begin{array}{l}
 \efix \eapp  \mathsf{multiRound}(\_). \lambda sc. \lambda scc. \lambda
  I. \lambda j. \lambda k. \lambda D. \lambda N. \lambda C. \\
 \hspace{.2cm} \eif   \big (   (j < k)  ,  \\
  \hspace{.2cm} \elet \eapp p = \uniform \eapp 0 \eapp 1 \ein \\
  \hspace{0.4cm} \elet \eapp q = \lambda x. \bernoulli \eapp p \ein \\
 \hspace{0.4cm} \elet \eapp qc = \lambda c. \bernoulli \eapp p \ein \\
 \hspace{0.4cm} \elet \eapp a = \eop (q)  \ein \\
 \hspace{0.8cm} \elet \eapp sc' =  \mathsf{updtSC} \eapp () \eapp sc  \eapp a \eapp p
 \eapp q \eapp I \eapp  \eapp 0 \eapp  N
  \eapp  \ein \\
\hspace{0.8cm} \elet \eapp scc' =  \mathsf{updtSCC} \eapp () \eapp scc' \eapp a \eapp p
 \eapp qc \eapp  \eapp 0 \eapp  C \ein \\
\hspace{0.8cm} \elet \eapp maxScc =  \mathsf{foldl} \eapp (\lambda acc. \lambda a. \eif ( acc < a, a, acc)) \eapp 0 \eapp scc' \ein \\
\hspace{0.8cm} \elet \eapp I' =  \mathsf{updtI}  \eapp () \eapp maxScc \eapp sc
  \eapp 0 \eapp N  \ein \\
  \hspace{0.8cm} \elet \eapp D' =  D \setminus I' \ein \\
  \hspace{1.2cm} \mathsf{multiRound} ()  \eapp sc' \eapp scc' \eapp I'
  \eapp (j+1) \eapp  k \eapp D' \eapp N \eapp C\\ 
\hspace{0.2cm}   ,     D  \big ) \\
 
\end{array}
\]

UpdtSC
\[
\begin{array}{l}
 \mathsf{updtSC} = \efix \eapp  \mathsf{f}(\_). \lambda sc. \lambda a. \lambda
  p. \lambda q.  \lambda I. \lambda i. \lambda N. \\
 \hspace{.2cm} \eif   \Big (   (i < N)  ,  \\
 \hspace{0.4cm}  \eif \big( ( i \eapp \mathsf{in} I  ) ,       \\
 \hspace{0.8cm} \elet \eapp x =( \mathsf{nth} \eapp sc \eapp i) + (a-p)*(q
  \eapp i - p)  \ein \\
 \hspace{0.8cm} \elet \eapp sc' =  \mathsf{updt} \eapp sc \eapp i
  \eapp x \ein \\
  \hspace{1.2cm} \mathsf{f}  \eapp () \eapp sc' \eapp a \eapp p
 \eapp q \eapp I \eapp  \eapp (i+1) \eapp  N  \\ 
\hspace{0.4cm}  ,  \mathsf{f}  \eapp () \eapp sc \eapp a \eapp p
 \eapp q \eapp I \eapp  \eapp (i+1) \eapp  N \big )  \\ 
\hspace{0.2cm}   ,  sc  \Big ) \\
 
\end{array}
\]

UpdtSCC
\[
\begin{array}{l}
 \mathsf{updtSCC} = \efix \eapp  \mathsf{f}(\_). \lambda scc. \lambda a. \lambda
  p. \lambda qc.  \lambda i. \lambda C. \\
 \hspace{.2cm} \eif   \Big (   (i < C)  ,  \\
 \hspace{0.8cm} \elet \eapp x =( \mathsf{nth} \eapp scc \eapp i) + (a-p)*(qc
  \eapp i - p)  \ein \\
 \hspace{0.8cm} \elet \eapp scc' =  \mathsf{updt} \eapp scc \eapp i
  \eapp x \ein \\
  \hspace{1.2cm} \mathsf{f}  \eapp () \eapp scc' \eapp a \eapp p
 \eapp qc   \eapp (i+1) \eapp  C  \\ 
\hspace{0.2cm}   ,  scc  \Big ) \\
 
\end{array}
\]


UpdtI
\[
\begin{array}{l}
 \mathsf{updtI} = \efix \eapp  \mathsf{f}(\_). \lambda maxScc. \lambda sc. \lambda
  i. \lambda N. \\
 \hspace{.2cm} \eif   \Big (   (i < N)  ,  \\
\hspace{0.4cm}  \eif \big( ( ( \mathsf{nth} \eapp scc \eapp i)  >  maxScc  ) ,       \\
 \hspace{0.8cm}   i :: ( \mathsf{f}  \eapp () \eapp maxScc \eapp sc
  \eapp (i+1) \eapp N  )\\
 \hspace{0.8cm} , \mathsf{f}  \eapp () \eapp maxScc \eapp sc
  \eapp (i+1) \eapp N  \big )  \\
\hspace{0.2cm}   ,  [] \Big ) \\
\end{array}
\]
\end{figure}

\begin{figure}


UpdtSC:
\begin{mathpar}
      \inferrule*[right = FIX]
      {
        \inferrule*[right = FIX...]
        {
          \inferrule*[right = IF]
          {
            \inferrule*[right = BOOL]
            {
              \empty
            }
            {
             i:\tint , I : \tlist{\tint} , \Gamma \tvdash{} {i
               <  N : \tbool}
            }
            \\
            \inferrule*[right = IF]
            {
              \dots
            }
            {
              f:., \Gamma; \tvdash{}  \eif (i \in I)\cdots : \tlist{\treal }
            }
            \and
            \inferrule*[right = VAR]
            {
            }
            {
              f:., sc:. ,\Gamma \tvdash{} {sc : \tlist{\treal}}
            }
            \\
            \and
          }
          {
          f: ., sc: \tlist{\treal}, a:\tint, i:\tint \dots \Gamma
            \tvdash{ } \eif (i < N)  \cdots : \tlist{\treal}
          }
        }
        {
          f: \tunit \rightarrow \dots
          \tlist{\treal}, \Gamma \tvdash{} {\lambda
            sc. \dots \lambda N.
            \eif \cdots :  \tlist{\treal} \rightarrow \dots \tlist{\treal}   }
        }
      }
      {
       \Gamma \tvdash{} \efix \eapp  f( \_ ). \lambda sc. \lambda
        a. \dots \lambda N. \eif \Big ( (i <N), \dots, sc \Big ) : \tunit
        \rightarrow \tlist{\treal} \rightarrow \dots \rightarrow \tlist{\treal} 
      }

   \inferrule*[ right = IF ]
   {
     \inferrule
     {
     }
     {
       \Gamma \tvdash{ }. f \eapp () \eapp sc' \dots \eapp N : \tlist{\treal}
      }
     \and
     \inferrule
     {
     \dots
     }
     {
      \Gamma\tvdash{}  \elet x = \dots \ein \elet sc' = \dots
      \ein f \eapp () \eapp sc' \dots : \tlist{\treal}
    }
     \\
     i :\tint , I : \tlist{\tint},\Gamma \tvdash{} i \in I : \tbool
   }
   { \Gamma \tvdash{}  \eif \big(  (i \in I), \elet x = \dots  ,  f \eapp ()
     \eapp sc \dots N \big) : \tlist{\treal }    }

    
    \inferrule*[right = LET]
    {
    \inferrule*[right = LET]
        {
        \inferrule*[right = APP]
            {
            }
            {
                \Gamma \tvdash{ } \mathsf{updt} \eapp sc \eapp
                i \eapp x : \tlist{\treal}
            }
            \and
        \inferrule*[right = APP]
            {
                \dots
            }
            {
                \Gamma \tvdash{} f () \eapp sc \dots \eapp  (i+1) \eapp  N : \tlist{\treal }
            }
            \\
            {  }
            \and
            {}
        }
        {
            x: \treal ,\Gamma \tvdash{} 
             \elet sc' = \mathsf{updt} \eapp sc \eapp i \eapp x \ein \dots : \tlist{\treal}
        }
      \\
      \inferrule
        {
          \dots
        }
        {
            \Gamma \tvdash{} ( \mathsf{nth} \eapp sc \eapp i  )+ (a-p)*(q \eapp i -
      p) : \treal
        }
    }
    {
      \tvdash{} 
      \elet x =( \mathsf{nth} \eapp sc \eapp i  )+ (a-p)*(q \eapp i -
      p) \ein \dots: \tlist{\treal }
    }
    \end{mathpar}
\end{figure}



\end{document}
