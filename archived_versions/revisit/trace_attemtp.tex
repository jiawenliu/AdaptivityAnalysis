
\paragraph{Traces}
A trace $\tr$ is a representation of the big-step derivation of an
expression's evaluation.

The \emph{adaptivity} of a trace $\tr$, $\adap(\tr)$, which
means the maximum number of nested $\eop$s in $\tr$.

The \emph{depth of variable $x$} in trace
$\tr$, written $\ddep{x}(\tr)$, which is the maximum number of
$\eop{}$s in any path leading from the root of $\tr$ to an occurence
of $x$ (at a leaf),.% Technically, $\adap: \mbox{Traces} \to \nat$ and
% $\ddep{x}: \mbox{Traces} \to \natb$. If $x$ does not appear free in
% $\tr$, $\ddep{x}(\tr)$ is $\bot$.


\[\begin{array}{llll}
\mbox{Expr.} & \expr & ::= & x ~|~ \expr_1 \eapp \expr_2 ~|~ {\efix f(x:\type).\expr}
 ~|~ (\expr_1, \expr_2) ~|~ \eprojl(\expr) ~|~ \eprojr(\expr) ~| \\
%
& & & \eif(\expr_1, \expr_2, \expr_3) ~|~
\econst ~|~ \eop(\expr)  ~|~  {\elet  x:q = \expr_1 \ein \expr_2 } ~|~ \enil ~|~  \econs (
      \expr_1, \expr_2) \\
%
\mbox{Value} & \valr & ::= & \econst ~|~
(\efix f(x:\type).\expr, \env) ~|~ (\valr_1, \valr_2) 
    ~|~ \enil ~|~ \econs (\valr_1, \valr_2) | \\
%
\mbox{Environment} & \env & ::= & x_1 \mapsto \valr_1, \ldots, x_n \mapsto \valr_n
\end{array}\]


%%%%%%%%%%%%%%%%%%%%%%%%%%%%%%%%%%%%%%%%%%%%%%%%%%%%%

%%%%%%%%%%%%%%%%%%%%%%%%%%%%%%%%%%%%%%%%%%%%%%%%%%%%%



%
\[\begin{array}{llll}
\mbox{Trace} & \tr & ::= & {(x, \env)} ~|~ \trapp{\tr_1}{\tr_2}{f}{x}{\tr_3} ~|~
{ (\trfix f(x:\type).e, \env) } ~|~ (\tr_1, \tr_2) ~|~ \trprojl(\tr) ~|\\ 
%
& & & \trprojr(\tr) ~|~ \trtrue ~|~ \trfalse ~|~ \trift(\tr_b, \tr_t)
~|~ \triff(\tr_b, \tr_f) ~|~ \trconst ~|~ \trop(\tr) \\
\end{array}\]



\begin{figure}[b]
\begin{mathpar}
   { \inferrule{ }{\env, x \bigstepT \env(x), (x, \env ) }  }
  %
  \and
  %
  \inferrule{ }{\env, \econst \bigstepT \econst, \trconst}
  % %
 %  \and
 %  %
 %  \inferrule{ }{\env, \etrue \bigstep \etrue, \trtrue}
 %  %
 %  \and
 %  %
 %  \inferrule{ }{\env, \efalse \bigstep \efalse, \trfalse}
 %  %
 %  \and
 %  { \inferrule{  \env, \expr \bigstep \econst, \tr }{\env, \bernoulli \eapp \expr \bigstep \econst,
 %      \bernoulli (\tr)
 %    } }
 %  \and
 % \inferrule{ \env, \expr_1 \bigstep \econst, \tr_1 \\ \env, \expr_2 \bigstep \econst, \tr_2  }{\env, \uniform \eapp \expr_1 \eapp
 %      \expr_2\bigstep \econst, \uniform(\tr_1,\tr_2)  } 
 %  \and
  %
  { \inferrule{
  }{
    \env, \efix f(x:\type). \expr \bigstepT (\efix f(:\type).\expr, \env),
    (\trfix f(x:\type).\expr, \env)
  }
}
  %
  \and
  %
  \inferrule{
    \env, \expr_1 \bigstepT \valr_1, \tr_1 \\
    { \valr_1 = (\efix f(x:\type).\expr, \env')} \\\\
    \env, \expr_2 \bigstepT \valr_2, \tr_2 \\
    \env'[f \mapsto \valr_1, x \mapsto \valr_2], \expr \bigstepT \valr, \tr
  }{
    \env, \expr_1 \eapp \expr_2 \bigstepT \valr, \trapp{\tr_1}{\tr_2}{f}{x}{\tr}
  }
  %
  \and
  %
  \inferrule{
    \env, \expr_1 \bigstepT \valr_1, \tr_1 \\
    \env, \expr_2 \bigstepT \valr_2, \tr_2
  }{
    \env, (\expr_1, \expr_2) \bigstepT (\valr_1, \valr_2), (\tr_1, \tr_2)
  }
  %
  \and
  %
  \inferrule{
    \env, \expr \bigstepT (\valr_1, \valr_2), \tr
  }{
    \env, \eprojl(\expr) \bigstepT \valr_1, \trprojl(\tr)
  }
  %
  \and
  %
  \inferrule{
    \env, \expr \bigstepT (\valr_1, \valr_2), \tr
  }{
    \env, \eprojr(\expr) \bigstepT \valr_2, \trprojr(\tr)
  }
  %
  \and
  %
  \inferrule{
    \env, \expr \bigstepT \etrue, \tr \\
    \env, \expr_1 \bigstepT \valr, \tr_1
  }{
    \env, \eif(\expr, \expr_1, \expr_2) \bigstepT \valr, \trift(\tr, \tr_1)
  }
  %
  \and
  %
  \inferrule{
    \env, \expr \bigstepT \efalse, \tr \\
    \env, \expr_2 \bigstepT \valr, \tr_2
  }{
    \env, \eif(\expr, \expr_1, \expr_2) \bigstepT \valr, \triff(\tr, \tr_2)
  }
  %
  \and
  %
  \inferrule{
    \env, \expr \bigstepT \valr, \tr \\
    \eop{}(\valr) = \valr'
  }{
    \env, \eop(\expr) \bigstepT \valr', \trop(\tr)
  }
% %
% \and
% %
%   \inferrule{
% }
% { \env, \enil \bigstep \enil, \trnil }
% %
% \and
% %
% \inferrule{
% \env, \expr_1 \bigstep \valr_1, \tr_1 \\
% \env, \expr_2 \bigstep \valr_2, \tr_2
% }
% { \env, \econs (\expr_1, \expr_2)  \bigstep \econs (\valr_1, \valr_2),
%   \trcons(\tr_1, \tr_2)
% }
% %
% \and
% %
% \inferrule{
%   \env, \expr_1 \bigstep \valr_1, \tr_1 \\
%   \env[x \mapsto \valr_1] , \expr_2 \bigstep \valr, \tr_2
% }
% {\env, \elet x;q = \expr_1 \ein \expr_2 \bigstep \valr, \trlet (x,
%   \tr_1, \tr_2) }
% %
% \\\\
% %
% \boxed{\color{red}
% \inferrule
% {
%   \empty
% }
% {
%   \env, \eilam \expr \bigstep (\eilam \expr, \env), (\eilam \expr , \env)
% }
% }
% %
% \and
% %
% \boxed{\color{red}
% \inferrule{
%   \env, \expr \bigstep (\eilam \expr', \env'), \tr_1 \\
%   \env, \expr' \bigstep \valr, \tr_2
% }
% {\env, \expr [] \bigstep \valr, \triapp{\tr_1}{\tr_2} }

% }
\end{mathpar}
  \caption{Big-step semantics with provenance}
  \label{fig:big-step}
\end{figure}


\begin{figure}
  \framebox{$\adap: \mbox{Traces} \to \nat$}
  \begin{mathpar}
    \begin{array}{lcl}
       { \adap( (x,\env) )} & = & 0 \\
      %
      \adap(\trapp{\tr_1}{\tr_2}{f}{x}{\tr_3}) & = &
      \adap(\tr_1) + \max (\adap(\tr_3), \adap(\tr_2) + \ddep{x}(\tr_3))\\
      %
       {\adap( (\trfix f(x:\type).\expr, \env)  ) } & = & 0 \\
      %
      \adap((\tr_1, \tr_2)) & = & \max(\adap(\tr_1), \adap(\tr_2)) \\
      %
      \adap(\trprojl(\tr)) & = & \adap(\tr) \\
      %
      \adap(\trprojr(\tr)) & = & \adap(\tr) \\
      %
      \adap(\trtrue) & = & 0 \\
      %
      \adap(\trfalse) & = & 0 \\
      %
      \adap(\trift(\tr_b, \tr_t)) & = & \adap(\tr_b) + \adap(\tr_t) \\
      %
      \adap(\triff(\tr_b, \tr_f)) & = & \adap(\tr_b) + \adap(\tr_f) \\
      %
      \adap(\trconst) & = & 0 \\
      %
      \adap(\trop(\tr)) & = & { 1 + \adap(\tr) } \\
           & &       {  +  \textsf{MAX}_{\valr \in \type} \Big(
                              \max \big(\adap(\tr_3 (\valr) ),
                              \ddep{x}(\tr_3(\valr)) \big) \Big) } \\
      &\mathsf{where}&  { \valr_1 = (\efix f(x: \type). \expr, \env ) =
                       \mathsf{extract}(\tr) } \\
 & & { \conj  \env[f \mapsto
                       \valr_1, x \mapsto \valr], \expr \bigstep
                       \valr', \tr_3(\valr) } \\ 
      \end{array}
  \end{mathpar}
  %
  \framebox{$\ddep{x}: \mbox{Traces} \to \natb$}
  \begin{mathpar}
    \begin{array}{lcl}
       { \ddep{x}( ( y, \env )) } & = &
      \left\lbrace
      \begin{array}{ll}
        0 & \mbox{if } x = y \\
        \bot & \mbox{if } x \neq y
      \end{array}
      \right.\\
      %
      \ddep{x}(\trapp{\tr_1}{\tr_2}{f}{y}{\tr_3}) & = & \max(\ddep{x}(\tr_1), \\
      & & \adap(\tr_1) + \max(\ddep{x}(\tr_3), \ddep{x}(\tr_2) + \ddep{y}(\tr_3))) \\
      %
      { \ddep{x}(  (\trfix f(y:\type).\expr,\env)  )  }& = & \bot \\
      %
      \ddep{x}((\tr_1, \tr_2)) & = & \max(\ddep{x}(\tr_1), \ddep{x}(\tr_2)) \\
      %
      \ddep{x}(\trprojl(\tr)) & = & \ddep{x}(\tr) \\
      %
      \ddep{x}(\trprojr(\tr)) & = & \ddep{x}(\tr) \\
      %
      \ddep{x}(\trtrue) & = & \bot \\
      %
      \ddep{x}(\trfalse) & = & \bot \\
      %
      \ddep{x}(\trift(\tr_b, \tr_t)) & = & \max(\ddep{x}(\tr_b), \adap(\tr_b) + \ddep{x}(\tr_t)) \\
      %
      \ddep{x}(\trift(\tr_b, \tr_f)) & = & \max(\ddep{x}(\tr_b), \adap(\tr_b) + \ddep{x}(\tr_f)) \\
      %
      \ddep{x}(\trconst) & = & \bot \\
      %
      \ddep{x}(\trop(\tr)) & = & 1 +  \max(\ddep{x}(\tr),  \\
      & &  \adap(\tr) + \textsf{MAX}_{\valr \in \type} \Big(
          \max(\ddep{x}(\tr_3(\valr)), \bot )   \Big ) ) \\  
 &\mathsf{where}&  { \valr_1 = (\efix f(x: \type). \expr, \env ) =
                       \mathsf{extract}(\tr) } \\
 & & { \conj  \env[f \mapsto
                       \valr_1, x \mapsto \valr], \expr \bigstep
                       \valr', \tr_3(\valr) } \\ 
    \end{array}
  \end{mathpar}
  \caption{Adaptivity of a trace and depth of variable $x$ in a trace}
  \label{fig:adap}
\end{figure}

\subsection{Challenge (Couterexample) in this setting}

\begin{mainitem}
\item[1] adaptivity is not precise. The definition of the max number
  of nested $\delta$s is not very accurate, especially the way it handles the
  application. Another way of understanding adaptivity is not only the
  occurence of $\delta$, but the times the program accesses the
  database ($\delta$).  For instance, $\lambda
  x. ( \eif \, (x < 10 ) \ethen \, x \eelse x  ) \, \delta(v) $ and $
   ( \eif \, ( \delta(v) < 10 ) \ethen \, \delta(v) \eelse \delta(v)  )
   $. should distinguish the two definition of adaptivity. \\
\item[2] This operational semantics has trouble to give the reasonable trace to
  nested lambda. Consider $\lambda x. \lambda y. (x y)$.  and its corresponding 
  application.  consider $ (  \lambda x. \lambda y. (x y) ) \;  \delta(v)  \; \delta(v) $ , its trace equals to 0 \\
\item[3] The typing system is not consistent for $\alpha$ renaming.
  \end{mainitem}