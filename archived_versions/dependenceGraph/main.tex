\documentclass[a4paper,11pt]{article}

\usepackage{mathpartir}
\usepackage{amsmath,amsthm,amsfonts}
\usepackage{ amssymb }
\usepackage{color}
\usepackage{algorithm}
\usepackage{algorithmic}
\usepackage{microtype}


%%%%%%%%%%%%%%%%%%%%%%%%%%%%%%%%%%%%%%%%%%%%%%%%%%%%%%%%%%%%%%%%%%%%%%%%%%%%%%%%%%%%%%%%%%%%%%%%%%%%%%%%%%%%%%%%%%%%%%%%%%%%%%%%%%%%%%%%%
%%%%%%%%%%%%%%%%%%%%%%%%%%%%%%%%%%%%%%%%%%%%%%%%%%%%% COMMANDS FOR GENERAL PAPER WRITING %%%%%%%%%%%%%%%%%%%%%%%%%%%%%%%%%%%%%%%%%%%%%%%%%%%%
%%%%%%%%%%%%%%%%%%%%%%%%%%%%%%%%%%%%%%%%%%%%%%%%%%%%%%%%%%%%%%%%%%%%%%%%%%%%%%%%%%%%%%%%%%%%%%%%%%%%%%%%%%%%%%%%%%%%%%%%%%%%%%%%%%%%%%%%%


%%%%%%%%%%%%%%%%%%%%%%%%%%%% Theorem, Definition and Proof
\newtheorem{lem}{Lemma}[section]
\newtheorem{thm}{Theorem}[section]
\newtheorem{defn}{Definition}
\newtheorem{coro}{Corollary}[thm]

\newtheorem{lemma}{Lemma}
\newtheorem{corollary}{Corollary}
\newtheorem*{theorem}{Theorem}
\renewcommand\thelemma{\unskip}
\renewcommand\thecorollary{\unskip}

% \theoremstyle{remark}
% \newtheorem*{lemma}{Lemma}

\newtheorem{example}{Example}[section]

\newcommand{\completeness}[1]{{\color{blue}\textbf{[[ #1 ]]}}}
\newcommand{\caseL}[1]{\item \textbf{case: #1}\newline}
\newcommand{\subcaseL}[1]{\item \textbf{sub-case: #1}\newline}
\newcommand{\subsubcaseL}[1]{\item \textbf{subsub-case: #1}\newline}
\newcommand{\subsubsubcaseL}[1]{\item \textbf{subsubsub-case: \boldmath{#1}}\newline}

\newcommand{\blue}[1]{{\tiny \color{blue}{ #1 }}}


\let\originalleft\left
\let\originalright\right
\renewcommand{\left}{\mathopen{}\mathclose\bgroup\originalleft}
\renewcommand{\right}{\aftergroup\egroup\originalright}
\newcommand{\ts}[1]{ \llparenthesis {#1} \rrparenthesis }

\theoremstyle{definition}

\newtheorem{case}{Case}
\newtheorem{subcase}{Case}
\numberwithin{subcase}{case}
\newtheorem{subsubcase}{Case}
\numberwithin{subsubcase}{subcase}

\newtheorem{subsubsubcase}{Case}
\numberwithin{subsubsubcase}{subsubcase}
\newcommand{\st}{~.~}
\newcommand{\sthat}{~.~}

%%%%COLORS
\definecolor{periwinkle}{rgb}{0.8, 0.8, 1.0}
\definecolor{powderblue}{rgb}{0.69, 0.88, 0.9}
\definecolor{sandstorm}{rgb}{0.93, 0.84, 0.25}
\definecolor{trueblue}{rgb}{0.0, 0.45, 0.81}

\newlength\Origarrayrulewidth
% horizontal rule equivalent to \cline but with 2pt width
\newcommand{\Cline}[1]{%
 \noalign{\global\setlength\Origarrayrulewidth{\arrayrulewidth}}%
 \noalign{\global\setlength\arrayrulewidth{2pt}}\cline{#1}%
 \noalign{\global\setlength\arrayrulewidth{\Origarrayrulewidth}}%
}

% draw a vertical rule of width 2pt on both sides of a cell
\newcommand\Thickvrule[1]{%
  \multicolumn{1}{!{\vrule width 2pt}c!{\vrule width 2pt}}{#1}%
}

% draw a vertical rule of width 2pt on the left side of a cell
\newcommand\Thickvrulel[1]{%
  \multicolumn{1}{!{\vrule width 2pt}c|}{#1}%
}

% draw a vertical rule of width 2pt on the right side of a cell
\newcommand\Thickvruler[1]{%
  \multicolumn{1}{|c!{\vrule width 2pt}}{#1}%
}

\newenvironment{subproof}[1][\proofname]{%
  \renewcommand{\qedsymbol}{$\blacksquare$}%
  \begin{proof}[#1]%
}{%
  \end{proof}%
}
%%%%%%%%%%%%%%%%%%%%%%%%%%%%%%% Fonts Definition %%%%%%%%%%%%%%%%%%%%%%%%%%%
\newcommand{\omitthis}[1]{}

% Misc.
\newcommand{\etal}{\textit{et al.}}
\newcommand{\bump}{\hspace{3.5pt}}

% Text fonts
\newcommand{\tbf}[1]{\textbf{#1}}

% Math fonts
\newcommand{\mbb}[1]{\mathbb{#1}}
\newcommand{\mbf}[1]{\mathbf{#1}}
\newcommand{\mrm}[1]{\mathrm{#1}}
\newcommand{\mtt}[1]{\mathtt{#1}}
\newcommand{\mcal}[1]{\mathcal{#1}}
\newcommand{\mfrak}[1]{\mathfrak{#1}}
\newcommand{\msf}[1]{\mathsf{#1}}
\newcommand{\mscr}[1]{\mathscr{#1}}

\newcommand{\diam}{{\color{red}\diamond}}
\newcommand{\dagg}{{\color{blue}\dagger}}
\let\oldstar\star
\renewcommand{\star}{\oldstar}

\newcommand{\im}[1]{\ensuremath{#1}}

\newcommand{\kw}[1]{\im{\mathtt{#1}}}
\newcommand{\set}[1]{\im{\{{#1}\}}}

\newcommand{\mmax}{\ensuremath{\mathsf{max}}}

%%%%%%%%%%%%%%%%%%%%%%%%%%%%%%% Blocks Definition %%%%%%%%%%%%%%%%%%%%%%%%%%%
\tikzstyle{decision} = [diamond, draw, fill=blue!20, 
    text width=4.5em, text badly centered, node distance=3cm, inner sep=0pt]
% \tikzstyle{block} = [rectangle, draw, fill=blue!20, 
%     text width=5em, text centered, rounded corners, minimum height=4em]
\tikzstyle{block} = [draw, very thick, fill=white, rectangle, 
    minimum height=2.5em, minimum width=6em, text centered]

\tikzstyle{line} = [draw, -latex']
\tikzstyle{cloud} = [draw, ellipse,fill=red!20, node distance=3cm,    minimum height=2em]

\tikzstyle{vecArrow} = [thick, decoration={markings,mark=at position
  1 with {\arrow[semithick]{open triangle 60}}},
  double distance=1.4pt, shorten >= 5.5pt,
  preaction = {decorate},
  postaction = {draw,line width=1.4pt, white,shorten >= 4.5pt}]
\tikzstyle{innerWhite} = [semithick, white,line width=1.4pt, shorten >= 4.5pt]



%%%%%%%%%%%%%%%%%%%%%%%%%%%%%%%%%%%%%%%%%%%%%%%%%%%%%%%%%%%%%%%%%%%%%%%%%%%%%%%%%%%%%%%%%%%%%%%%%%%%%%%%%%%%%%%%%%%%%%%%%%%%%%%%%%%%%%%%%%%%%%%%%%%%%%%%%%
%%%%%%%%%%%%%%%%%%%%%%%%%%%%%%%%%%%%%%%%%%%%%%%%%%%%%%%%%%%% Introduction %%%%%%%%%%%%%%%%%%%%%%%%%%%%%%%%%%%%%%%%%%%%%%%%%%%%%%%%%%%%%%%%
%%%%%%%%%%%%%%%%%%%%%%%%%%%%%%%%%%%%%%%%%%%%%%%%%%%%%%%%%%%%%%%%%%%%%%%%%%%%%%%%%%%%%%%%%%%%%%%%%%%%%%%%%%%%%%%%%%%%%%%%%%%%%%%%%%%%%%%%%%%%%%%%%%%%%%%%%%

\newcommand{\dist}{P}
\newcommand{\mech}{M}
\newcommand{\univ}{\mathcal{X}}
\newcommand{\anyl}{A}
\newcommand{\qrounds}{r}
\newcommand{\answer}{a}
\newcommand{\sample}{X}
\newcommand{\ex}[2]{{\ifx&#1& \mathbb{E} \else \underset{#1}{\mathbb{E}} \fi \left[#2\right]}}
\newcommand{\pr}[2]{{\ifx&#1& \mathbb{P} \else \underset{#1}{\mathbb{P}} \fi \left[#2\right]}}
\newcommand{\var}[2]{{\ifx&#1& \mathrm{Var} \else \underset{#1}{\mathrm{Var}} \fi \left[#2\right]}}
\newcommand{\eps}{\varepsilon}
\newcommand{\from}{:}
\newcommand{\sep}{ \ | \ }


%%%%%%%%%%%%%%%%%%%%%%%%%%%%%%%%%%%%%%%%%%%%%%%%%%%%%%%%%%%%%%%%%%%%%%%%%%%%%%%%%%%%%%%%%%%%%%%%%%%%%%%%%%%%%%%%%%%%%%%%%%%%%%%%%%%%%%%%%%%%%%%%%%%%%%%%%%
%%%%%%%%%%%%%%%%%%%%%%%%%%%%%%%%%%%%%%%%%%%%%%%%%%%%%%%%%%%% Query While Language %%%%%%%%%%%%%%%%%%%%%%%%%%%%%%%%%%%%%%%%%%%%%%%%%%%%%%%%%%%%%%%%
%%%%%%%%%%%%%%%%%%%%%%%%%%%%%%%%%%%%%%%%%%%%%%%%%%%%%%%%%%%%%%%%%%%%%%%%%%%%%%%%%%%%%%%%%%%%%%%%%%%%%%%%%%%%%%%%%%%%%%%%%%%%%%%%%%%%%%%%%%%%%%%%%%%%%%%%%%
% Language
\newcommand{\command}{c}
%Label
\newcommand{\lin}{\kw{in}}
\newcommand{\lex}{\kw{ex}}
% expression
\newcommand{\expr}{e}
\newcommand{\aexpr}{a}
\newcommand{\bexpr}{b}
\newcommand{\sexpr}{\ssa{\expr} }
\newcommand{\qexpr}{\psi}
\newcommand{\qval}{\alpha}
\newcommand{\query}{{\tt query}}
\newcommand{\eif}{\;\kw{if}\;}
\newcommand{\ethen}{\kw{\;then\;}}
\newcommand{\eelse}{\kw{\;else\;}} 
\newcommand{\eapp}{\;}
\newcommand{\eprojl}{\kw{fst}}
\newcommand{\eprojr}{\kw{snd}}
\newcommand{\eifvar}{\kw{ifvar}}
\newcommand{\ewhile}{\;\kw{while}\;}
\newcommand{\bop}{\;*\;}
\newcommand{\uop}{\;\circ\;}
\newcommand{\eskip}{\kw{skip}}
\newcommand{\edo}{\;\kw{do}\;}
% More unary expression operators:
\newcommand{\esign}{~\kw{sign}~}
\newcommand{\elog}{~\kw{log}~}


\newcommand{\emax}{~\kw{max}~}
\newcommand{\emin}{~\kw{min}~}


%%%%%%%%%% Extended
\newcommand{\efun}{~\kw{fun}~}
\newcommand{\ecall}{~\kw{call}~}


% Domains
\newcommand{\qdom}{\mathcal{QD}}
\newcommand{\memdom}{\mathcal{M}}
\newcommand{\dbdom}{\mathcal{DB}}
\newcommand{\cdom}{\mathcal{C}}
\newcommand{\ldom}{\mathcal{L}}

\newcommand{\emap}{~\kw{map}~}
\newcommand{\efilter}{~\kw{filter}~}

%configuration
\newcommand{\config}[1]{\langle #1 \rangle}
\newcommand{\ematch}{\kw{match}}
\newcommand{\clabel}[1]{\left[ #1 \right]}

\newcommand{\etrue}{\kw{true}}
\newcommand{\efalse}{\kw{false}}
\newcommand{\econst}{c}
\newcommand{\eop}{\delta}
\newcommand{\efix}{\mathop{\kw{fix}}}
\newcommand{\elet}{\mathop{\kw{let}}}
\newcommand{\ein}{\mathop{ \kw{in}} }
\newcommand{\eas}{\mathop{ \kw{as}} }
\newcommand{\enil}{\kw{nil}}
\newcommand{\econs}{\mathop{\kw{cons}}}
\newcommand{\term}{t}
\newcommand{\return}{\kw{return}}
\newcommand{\bernoulli}{\kw{bernoulli}}
\newcommand{\uniform}{\kw{uniform}}
\newcommand{\app}[2]{\mathrel{ {#1} \, {#2} }}


% Operational Semantics
\newcommand{\env}{\rho}
\newcommand{\rname}[1]{\textsf{\small{#1}}}
\newcommand{\aarrow}{\Downarrow_a}
\newcommand{\barrow}{\Downarrow_b}
\newcommand{\earrow}{\Downarrow_e}
\newcommand{\qarrow}{\Downarrow_q}
\newcommand{\cmd}{c}
\newcommand{\node}{N}
\newcommand{\assign}[2]{ \mathrel{ #1  \leftarrow #2 } }


%%%%%%%%%%%%%%%%%%%%%%%%%%%%%%%%%%%%%%%%%%%%%%%%%%%%%%%%%%%%%%%%%%%%%%%% Trace and Events %%%%%%%%%%%%%%%%%%%%%%%%%%%%%%%%%%%%%%%%
%%%%%%%%%%%%%%%%%%%%%%%%%%%%%%%%%%%%%%%%%%%%%%%%%%%%%%%%%%%%%%%%%%%%%% Trace 
%%%%%%%% annotated query
\newcommand{\aq}{\kw{aq}}
\newcommand{\qtrace}{\kw{qt}}
%annotated variables
\newcommand{\av}{\kw{av}}
\newcommand{\vtrace}{\kw{\tau}}
\newcommand{\ostrace}{{\kw{\tau}}}
\newcommand{\posttrace}{{\kw{\tau}}}



%%%%%%%%%%%%%%%%%traces
\newcommand{\tdom}{\mathcal{T}^{+\infty}}
\newcommand{\trace}{\kw{\tau}}
\newcommand{\ftdom}{\mathcal{T}^{+}}
\newcommand{\inftdom}{\mathcal{T}^{\infty}}

% \newcommand{\vcounter}{\kw{\zeta}}
\newcommand{\vcounter}{\kw{cnt}}

\newcommand{\postevent}{{\kw{\epsilon}}}


\newcommand{\event}{\kw{\epsilon}}
\newcommand{\eventset}{\mathcal{E}}
\newcommand{\eventin}{\in_{\kw{e}}}
\newcommand{\eventeq}{=_{\kw{e}}}
\newcommand{\eventneq}{\neq_{\kw{e}}}
\newcommand{\eventgeq}{\geq_{\kw{e}}}
\newcommand{\eventlt}{<_{\kw{e}}}
\newcommand{\eventleq}{\leq_{\kw{e}}}
\newcommand{\eventdep}{\mathsf{DEP_{\kw{e}}}}
\newcommand{\asn}{\kw{{asn}}}
\newcommand{\test}{\kw{{test}}}
\newcommand{\ctl}{\kw{{ctl}}}

\newcommand{\sig}{\kw{sig}}
\newcommand{\sigeq}{=_{\sig}}
\newcommand{\signeq}{\neq_{\sig}}
\newcommand{\notsigin}{\notin_{\sig}}
\newcommand{\sigin}{\in_{\sig}}
\newcommand{\sigdiff}{\kw{Diff}_{\sig}}
\newcommand{\action}{\kw{act}}
\newcommand{\diff}{\kw{Diff}}
\newcommand{\seq}{\kw{seq}}
\newcommand{\sdiff}{\kw{Diff}_{\seq}}

\newcommand{\tracecat}{{\scriptscriptstyle ++}}
\newcommand{\traceadd}{{\small ::}}

\newcommand{\ism}{\kw{ism}}
\newcommand{\ismdiff}{\kw{Diff}_{\sig}}
\newcommand{\ismeq}{=_{\ism}}
\newcommand{\ismneq}{\neq_{\ism}}
\newcommand{\notismin}{\notin_{\ism}}
\newcommand{\ismin}{\in_{\ism}}

%operations on the trace and Annotated Query
\newcommand{\projl}[1]{\kw{\pi_{l}(#1)}}
\newcommand{\projr}[1]{\kw{\pi_{r}(#1)}}

% operations on annotated query, i.e., aq
\newcommand{\aqin}{\in_{\kw{aq}}}
\newcommand{\aqeq}{=_{\kw{aq}}}
\newcommand{\aqneq}{\neq_{\kw{aq}}}
\newcommand{\aqgeq}{\geq_{\kw{aq}}}

% operations on annotated variables, i.e., av
\newcommand{\avin}{\in_{\kw{av}}}
\newcommand{\aveq}{=_{\kw{av}}}
\newcommand{\avneq}{\neq_{\kw{av}}}
\newcommand{\avgeq}{\geq_{\kw{av}}}
\newcommand{\avlt}{<_{\kw{av}}}

% adaptivity
\newcommand{\adap}{\kw{adap}}
\newcommand{\ddep}[1]{\kw{depth}_{#1}}
\newcommand{\nat}{\mathbb{N}}
\newcommand{\natb}{\nat_{\bot}}
\newcommand{\natbi}{\natb^\infty}
\newcommand{\nnatA}{Z}
\newcommand{\nnatB}{m}
\newcommand{\nnatbA}{s}
\newcommand{\nnatbB}{t}
\newcommand{\nnatbiA}{q}
\newcommand{\nnatbiB}{r}


%%%%%%%%%%%%%%%%%%%%%%%%%%%%%%%%%%%%%%%%%%%%%%%%%%%%%%%%%%%%%%%%%%%%%%%%%%%%%%%%%%%%%%%%%%%%%%%%%%%%%%%%%%%%%%%%%%%%%%%%%%%%%%%%%%%%%%%%%%%%%%%%%%%%%%%%%%
%%%%%%%%%%%%%%%%%%%%%%%%%%%%%%%%%%%%%%%%%%%%%%%%%%%%%%%%%%%% Semantic Definition %%%%%%%%%%%%%%%%%%%%%%%%%%%%%%%%%%%%%%%%%%%%%%%%%%%%%%%%%%%%%%%%
%%%%%%%%%%%%%%%%%%%%%%%%%%%%%%%%%%%%%%%%%%%%%%%%%%%%%%%%%%%%%%%%%%%%%%%%%%%%%%%%%%%%%%%%%%%%%%%%%%%%%%%%%%%%%%%%%%%%%%%%%%%%%%%%%%%%%%%%%%%%%%%%%%%%%%%%%%

%%%%%%%%%%%%%%%%%%%%%%%%%%%%%%%%% Semantics Based Dependency, Graph and Adaptivity 
\newcommand{\paths}{\mathcal{PATH}}
\newcommand{\walks}{\mathcal{WK}}
\newcommand{\progwalks}{\mathcal{WK}^{\kw{est}}}

\newcommand{\len}{\kw{len}}
\newcommand{\lvar}{\mathbb{LV}}
\newcommand{\avar}{\mathbb{AV}}
\newcommand{\qvar}{\mathbb{QV}}
\newcommand{\qdep}{\mathsf{DEP_{q}}}
\newcommand{\vardep}{\mathsf{DEP_{var}}}
\newcommand{\avdep}{\mathsf{DEP_{\av}}}
\newcommand{\finitewalk}{\kw{fw}}
\newcommand{\pfinitewalk}{\kw{fwp}}
\newcommand{\dep}{\mathsf{DEP}}

\newcommand{\llabel}{\iota}
\newcommand{\entry}{\kw{entry}}
\newcommand{\tlabel}{\mathbb{TL}}

\newcommand{\traceG}{\kw{{G_{trace}}}}
\newcommand{\traceV}{\kw{{V_{trace}}}}
\newcommand{\traceE}{\kw{{E_{trace}}}}
\newcommand{\traceF}{\kw{{Q_{trace}}}}
\newcommand{\traceW}{\kw{{W_{trace}}}}

%%%%%%%%%%%%%%%%%%%%%%%%%%%%%%%%%%%%%%%%%%%%%%%%%%%%%%%%%%%%%%%%%%%%%%%%%%%%%%%%%%%%%%%%%%%%%%%%%%%%%%%%%%%%%%%%%%%%%%%%%%%%%%%%%%%%%%%%%%%%%%%%%%%%%%%%%%
%%%%%%%%%%%%%%%%%%%%%%%%%%%%%%%%%%%%%%%%%%%%%%%%%%%%%%%%%%%% Static Program Analysis %%%%%%%%%%%%%%%%%%%%%%%%%%%%%%%%%%%%%%%%%%%%%%%%%%%%%%%%%%%%%%%%
%%%%%%%%%%%%%%%%%%%%%%%%%%%%%%%%%%%%%%%%%%%%%%%%%%%%%%%%%%%%%%%%%%%%%%%%%%%%%%%%%%%%%%%%%%%%%%%%%%%%%%%%%%%%%%%%%%%%%%%%%%%%%%%%%%%%%%%%%%%%%%%%%%%%%%%%%%

%Static Adaptivity Definition:
\newcommand{\flowsto}{\kw{flowsTo}}
\newcommand{\live}{\kw{RD}}

%Analysis Algorithms and Graphs
\newcommand{\weight}{\mathsf{W}}
\newcommand{\green}[1]{{ \color{green} #1 } }

\newcommand{\func}[2]{\mathsf{AD}(#1) \to (#2)}
\newcommand{\varCol}{\bf{VetxCol}}
\newcommand{\graphGen}{\bf{FlowGen}}
\newcommand{\progG}{\kw{{G_{est}}}}
\newcommand{\progV}{\kw{{V_{est}}}}
\newcommand{\progE}{\kw{{E_{est}}}}
\newcommand{\progF}{\kw{{Q_{est}}}}
\newcommand{\progW}{\kw{{W_{est}}}}
\newcommand{\progA}{{\kw{A_{est}}}}

\newcommand{\midG}{\kw{{G_{mid}}}}
\newcommand{\midV}{\kw{{V_{mid}}}}
\newcommand{\midE}{\kw{{E_{mid}}}}
\newcommand{\midF}{\kw{{Q_{mid}}}}



\newcommand{\sccgraph}{\kw{G^{SCC}}}
\newcommand{\sccG}{\kw{{G_{scc}}}}
\newcommand{\sccV}{\kw{{V_{scc}}}}
\newcommand{\sccE}{\kw{{E_{scc}}}}
\newcommand{\sccF}{\kw{{Q_{scc}}}}
\newcommand{\sccW}{\kw{{W_{scc}}}}


\newcommand{\visit}{\kw{visit}}

\newcommand{\flag}{\kw{F}}
\newcommand{\Mtrix}{\kw{M}}
\newcommand{\rMtrix}{\kw{RM}}
\newcommand{\lMtrix}{\kw{LM}}
\newcommand{\vertxs}{\kw{V}}
\newcommand{\qvertxs}{\kw{QV}}
\newcommand{\qflag}{\kw{Q}}
\newcommand{\edges}{\kw{E}}
\newcommand{\weights}{\kw{W}}
\newcommand{\qlen}{\len^{\tt q}}
\newcommand{\pwalks}{\mathcal{WK}_{\kw{p}}}

\newcommand{\rb}{\mathsf{RechBound}}
\newcommand{\pathsearch}{\mathsf{AdaptSearch}}

%program abstraction
\newcommand{\abst}[1]{\kw{abs}{#1}}
\newcommand{\absexpr}{\abst{\kw{expr}}}
% \newcommand{\absevent}{\stackrel{\scriptscriptstyle{\alpha}}{\event{}}}
\newcommand{\absevent}{\hat{\event{}}}
\newcommand{\absfinal}{\abst{\kw{final}}}
\newcommand{\absinit}{\abst{\kw{init}}}
\newcommand{\absflow}{\abst{\kw{trace}}}
\newcommand{\absG}{\abst{\kw{G}}}
\newcommand{\absV}{\abst{\kw{V}}}
\newcommand{\absE}{\abst{\kw{E}}}
\newcommand{\absF}{\abst{\kw{F}}}
\newcommand{\absW}{\abst{\kw{W}}}
\newcommand{\locbound}{\kw{locb}}
\newcommand{\absdom}{\mathcal{ADOM}}

\newcommand{\inpvar}{\vardom_{\kw{in}}}
\newcommand{\grdvar}{\vardom_{\kw{guard}}}
\newcommand{\vardom}{\mathcal{V}}
\newcommand{\inpvardom}{\mathcal{V}_{\kw{\lin}}}
\newcommand{\scexprdom}{\mathcal{A}_{\kw{sc}}}
\newcommand{\inpexprdom}{\mathcal{A}_{\lin}}
\newcommand{\scvardom}{\mathcal{SC}}
\newcommand{\valuedom}{\mathcal{VAL}}
\newcommand{\scexpr}{{\kw{A_{sc}}}}


\newcommand{\absclr}{{\kw{TB}}}
\newcommand{\varinvar}{{\kw{VB}}}
\newcommand{\init}{\kw{init}}
\newcommand{\constdom}{\mathcal{SC}}
\newcommand{\dcdom}{\mathcal{DC}}
\newcommand{\reset}{\kw{re}}
\newcommand{\resetchain}{\kw{rechain}}
\newcommand{\inc}{\kw{inc}}
\newcommand{\dec}{\kw{dec}}
\newcommand{\booldom}{\mathcal{B}}
\newcommand{\incrs}{\kw{increase}}




%%%%%%%%%%%%%%%%%%%%%%%%%%%%%%%%%%%%%%%%%%%%%%%%%%%%%%%%%%%%%%%%%%%%%%%%%%%%%%%%%%%%%%%%%%%%%%%%%%%%%%%%%%%%%%%%%%%%%%%%%%%%%%%%%%%%%%%%%%%%%%%%%%%%%%%%%%
%%%%%%%%%%%%%%%%%%%%%%%%%%%%%%%%%%%%%%%%%%%%%%%%%%%%%%%%%%%%%%%%%%%%%%% author comments in draft mode %%%%%%%%%%%%%%%%%%%%%%%%%%%%%%%%%%%%%%%%%%%%%%%%%%%%%%%%%%%%%%%%%%%%%%
%%%%%%%%%%%%%%%%%%%%%%%%%%%%%%%%%%%%%%%%%%%%%%%%%%%%%%%%%%%%%%%%%%%%%%%%%%%%%%%%%%%%%%%%%%%%%%%%%%%%%%%%%%%%%%%%%%%%%%%%%%%%%%%%%%%%%%%%%%%%%%%%%%%%%%%%%%

\newif\ifdraft
%\draftfalse
\drafttrue

\newcommand{\todo}[1]{{\color{red}\textbf{[[ #1 ]]}}}
% \newcommand{\todo}[1]{#1}
\newcommand{\todomath}[1]{{\scriptstyle \color{red}\mathbf{[[ #1 ]]}}}


\ifdraft
% Jiawen
\newcommand{\jl}[1]{\textcolor[rgb]{.00,0.00,1.00}{[JL: #1]}}
% \newcommand{\jl}[1]{#1}
\newcommand{\jlside}[1]{\marginpar{\tiny \sf \textcolor[rgb]{.00,0.80,0.00}{[jl: #1]}}}
% Deepak
\newcommand{\dg}[1]{\textcolor[rgb]{.00,0.80,0.00}{[DG: #1]}}
\newcommand{\dgside}[1]{\marginpar{\tiny \sf \textcolor[rgb]{.00,0.80,0.00}{[DG: #1]}}}
% Marco
\newcommand{\mg}[1]{\textcolor[rgb]{.80,0.00,0.00}{[MG: #1]}}
\newcommand{\mgside}[1]{\marginpar{\tiny \sf \textcolor[rgb]{.80,0.00,0.00}{[MG: #1]}}}
% Weihao
\newcommand{\wq}[1]{\textcolor[rgb]{.00,0.80,0.00}{[WQ: #1]}}
\newcommand{\wqside}[1]{\marginpar{\tiny \sf \textcolor[rgb]{.00,0.80,0.00}{[WQ: #1]}}}
\else
\newcommand{\mg}[1]{}
\newcommand{\mgside}[1]{}
\newcommand{\wq}[1]{}
\newcommand{\wqside}[1]{}
\newcommand{\rname}[1]{$\textbf{#1}$}
\fi

\newcommand{\highlight}[1]{\textcolor[rgb]{.0,0.0,1.0}{ #1}}
% \newcommand{\highlight}[1]{{ #1}}

\newcommand{\mg}[1]{\textcolor[rgb]{.90,0.00,0.00}{[MG: #1]}}
\newcommand{\dg}[1]{\textcolor[rgb]{0.00,0.5,0.5}{[DG: #1]}}
\newcommand{\wq}[1]{\textcolor[rgb]{.50,0.0,0.7}{ #1}}
\newcommand{\aexpr}{a}
\newcommand{\bexpr}{b}
\newcommand{\cmd}{c}
\newcommand{\node}{N}
\newcommand{\assign}[2]{ \mathrel{ #1  \leftarrow #2 } }
\newcommand{\fassign}[3]{ \mathrel{ #1  \leftarrow^{#3}  \delta^{#3}(
    #2 ) } }
\newcommand{\ethen}{\kw{then}}
\newcommand{\eelse}{\kw{else}}
\newcommand{\impif}[3]{\mathrel{\eif \eapp #1\eapp \ethen \eapp #2 \eapp
    \eelse \eapp #3 }}
\newcommand{\impwhile}[2]{\mathrel{ \kw{while} (#1) \eapp #2 } }
\newcommand{\labl}{l}

\let\originalleft\left
\let\originalright\right
\renewcommand{\left}{\mathopen{}\mathclose\bgroup\originalleft}
\renewcommand{\right}{\aftergroup\egroup\originalright}

\theoremstyle{definition}
\newtheorem{thm}{Theorem}
\newtheorem{lem}[thm]{Lemma}
\newtheorem{cor}[thm]{Corollary}
\newtheorem{prop}[thm]{Proposition}
\newtheorem{defn}[thm]{Definition}

\title{Adaptivity analysis}

\author{}

\date{}

\begin{document}

\maketitle

% \begin{abstract}
% An adaptive data analysis is based on multiple queries over a data set, in which some queries rely on the results of some other queries. The error of each query is usually controllable and bound independently, but the error can propagate through the chain of different queries and bring to high generalization error. To address this issue, data analysts are adopting different mechanisms in their algorithms, such as Gaussian mechanism, etc. To utilize these mechanisms in the best way one needs to understand the depth of chain of queries that one can generate in a data analysis. In this work, we define a programming language which can provide, through its type system, an upper bound on the adaptivity  depth (the length of the longest chain of queries) of a program implementing an adaptive data analysis. We show how this language can help to analyze the generalization error of two data analyses with different adaptivity structures.
% \end{abstract}


% \section{Everything Else}

% \paragraph{Adaptivity}
% Adaptivity is a measure of the nesting depth of a mechanism. To
% represent this depth, we use extended natural numbers. Define $\natb =
% \nat \cup \{\bot\}$, where $\bot$ is a special symbol and $\natbi =
% \natb \cup \{\infty\}$. We use $\nnatA, \nnatB$ to range over $\nat$,
% $\nnatbA, \nnatbB$ to range over $\natb$, and $\nnatbiA, \nnatbiB$ to
% range over $\natbi$.

% The functions $\max$ and $+$, and the order $\leq$ on natural numbers
% extend to $\natbi$ in the natural way:
% \[\begin{array}{lcl}
% \max(\bot, \nnatbiA) & = & \nnatbiA \\
% \max(\nnatbiA, \bot) & = & \nnatbiA \\
% \max(\infty, \nnatbiA) & = & \infty \\
% \max(\nnatbiA, \infty) & = & \infty \\
% \\
% %
% \bot + \nnatbiA & = & \bot \\
% \nnatbiA + \bot & = & \bot \\
% \infty + \nnatbiA & = & \infty ~~~~ \mbox{if } \nnatbiA \neq \bot \\
% \nnatbiA + \infty & = & \infty ~~~~ \mbox{if } \nnatbiA \neq \bot \\
% \\
% %
% \bot \leq \nnatbiA \\
% \nnatbiA \leq \infty
% \end{array}
% \]
% One can think of $\bot$ as $-\infty$, with the special proviso that,
% here, $-\infty + \infty$ is specifically defined to be $-\infty$.

% \paragraph{Language}
% Expressions are shown below. $\econst$ denotes constants (of some base
% type $\tbase$, which may, for example, be reals or rational
% numbers). $\eop$ represents a primitive operation (such as a
% mechanism), which determines adaptivity. For simplicity, we assume
% that $\eop$ can only have type $\tbase \to \tbool$. We make
% environments explicit in closures. This is needed for the tracing
% semantics later.
\[\begin{array}{llll}
\mbox{AExpr.} & \aexpr & ::= & n~|~ x ~|~ \aexpr_1 + \aexpr_2  ~|~  \aexpr_1 -
                            \aexpr_2 ~|~ \aexpr_1 * \aexpr_2 \\
%
\mbox{BExpr} & \bexpr & ::= & v ~|~  \aexpr_1 < \aexpr_2 ~|~ \aexpr_1
                              = \aexpr_2 ~|~ \neg \bexpr ~|~ \bexpr_1
                              \land \bexpr_2 ~|~ \bexpr_1 \lor \bexpr_2
\\
  \mbox{Command} & \cmd& ::= & Skip ~|~ \cmd_1 ; \cmd_2 ~|~ \impif{\bexpr}{\cmd_1}{\cmd_2}
                               ~|~             \impwhile{\bexpr}{\cmd}  \\
              &&& ~|~\assign{x}{\aexpr} ~|~ \fassign{x}{\aexpr}{l}  
    \\
    \mbox{Label}  & \labl & \in &  \mathbb{Z} \\
    \mbox{Trace} & T & ::= &   \{  [(x_1,\labl_1),\dots, (x_i,
                             \labl_i)],  \dots,  [(y_1,\labl_1),\dots, (y_i,
                             \labl_i)]  \}    \\
\mbox{Environment} & \env & ::= & x_1 \mapsto (n_1, T_1),
                                  \ldots, x_n \mapsto
                                  (\valr_n,T_n) \\
    \mbox{Node}  & \node & ::=&  Empty ~|~D_1(x, T) ~|~ D_2(x,T) ~|~ IFT(T_b, \node) ~|~
                                IFF(T_b, \node) ~|~ W (T_b, \node)
                                ~|~ \node_1 ; \node_2\\
                                  \mbox{Trace} & T & \in  & Set <
                                                            List<Var
                                                            \times Label> >
\end{array}\]





%%%%%%%%%%%%%%%%%%%%%%%%%%%%%%%%%%%%%%%%%%%%%%%%%%%% sementics

%%%%%%%%%%%%%%%%%%%%%%%%%%%%%%%%%%%%%%%%%%%%%%%%%%%%% 

\begin{figure}
  \boxed{ \env, \aexpr \bigstep{T} n  }
  \begin{mathpar}
   \inferrule{ \env(x) =  (n, T)  }{\env , x \bigstep{T} n
   }~\textsf{var}
   %
   \and
   %
   \inferrule{  }{\env , n \bigstep{()} n
   }~\textsf{const}
  \and
  %
  \inferrule{
    \env, \aexpr_1 \bigstep{T_1} n_1\\
    \env, \aexpr_2 \bigstep{T_2} n_2
  }{\env, \aexpr_1 + \aexpr_2 \bigstep{ T_1 \cup T_2 }
     n_1 + n_2}~\textsf{sum} 
\end{mathpar}
\boxed{\env, \bexpr \bigstep{T} v }
\begin{mathpar}
   \inferrule{
    \env, \aexpr_1 \bigstep{T_1} n_1\\
    \env, \aexpr_2 \bigstep{T_2} n_2
  }{\env, \aexpr_1 + \aexpr_2 \bigstep{ T_1 \cup T_2 }
    n_1 < n_2}~\textsf{les}
  %
  \and
  %
  \inferrule{
    \env, \bexpr_1 \bigstep{T_1} v_1\\
    \env, \bexpr_2 \bigstep{T_2} v_2
  }{\env, \bexpr_1 \land \bexpr_2 \bigstep{ T_1 \cup T_2 }
    v_1 \land v_2}~\textsf{land}
\end{mathpar}
\boxed{ \env, \cmd \bigstep{\node} \env'  }
\begin{mathpar}
  \inferrule{ }{\env, Skip \bigstep{} \env
  }~\textsf{skip}
  %
  \and
  %
  \inferrule{
    \env, \cmd_1 \bigstep{\node_1} \env_1\\
    \env_1, \cmd_2 \bigstep{\node_2} \env_2
  }{\env, \cmd_1 ; \cmd_2 \bigstep{ \node_1 ; \node_2 }
    \env_2}~\textsf{Seq}
  %
  \and
  %
  \inferrule{
    \env, \aexpr \bigstep{ T } n 
  }{\env, \assign{x}{\aexpr} \bigstep{ D_1(x, T)  }
    \env[x \to (n, T)]  }~\textsf{Assign}
  %
  \and
  %
  \inferrule{
    \env, \aexpr \bigstep{ T } n \\
    \eop(n) = n_1
    }{\env, \fassign{x}{\aexpr}{\labl} \bigstep{ D_2(x,T+(x,l))}
    \env[x \to (n_1, T +(x,l) )] }~\textsf{FAssign}
  %
  \and
  %
  \inferrule{
    \env, \bexpr \bigstep{ T_b } false \\
    \env, \cmd_2 \bigstep{\node_2} \env_2
    }{\env, \impif{\bexpr}{\cmd_1}{\cmd_2}  \bigstep{ IFF(T_b, \node_2)}
    \env_2 }~\textsf{IFF}
 %
  \and
  %
  \inferrule{
    \env, \bexpr \bigstep{ T_b } true \\
    \env, \cmd_1 \bigstep{\node_1} \env_1
    }{\env, \impif{\bexpr}{\cmd_1}{\cmd_2}  \bigstep{ IFT(T_b, \node_1)}
      \env_1 }~\textsf{IFT}
    %
    \and
    %
    \inferrule{
    \env, \bexpr \bigstep{ T_b } false 
    }{\env, \impwhile{\bexpr}{\cmd}  \bigstep{ WF(T_b)}
      \env }~\textsf{WHILEF}
    %
    \and
    %
    \inferrule{
      \env, \bexpr \bigstep{ T_b } true\\
      \env, \cmd \bigstep{\node_1} \env_1  \\
      \env_1, \impwhile{\bexpr}{\cmd} \bigstep{\node_2} \env_2
    }{\env, \impwhile{\bexpr}{\cmd}  \bigstep{ WT(T_b,\node_1, \node_2 )}
      \env_2 }~\textsf{WHILET}
 \\\\
  \begin{array}{lll}
    \{ \} + (x,l) & \triangleq & \{ (x,l)  \}   \\
     \{  [(x_1, l_1), \dots, (x_i, l_i)] , \dots, [(y_1, l_1'), \dots, (y_i,
    l_i')]  \} + (x,l)   & \triangleq & 
    \\
     \{  [(x_1, l_1), \dots, (x_i,
                                       l_i),(x,l)] , \dots, [ (y_1,
                                        l_1'), \dots, (y_i, l_i'),
    (x,l) ]\}  & &
  \end{array}
\end{mathpar}
  \caption{Big-step semantics}
  \label{fig:semantics1}
\end{figure}





%%%%%%%%%%%%%%%%%%%%%%%%%%%%%%%%%%%%%%%%%%%%%%%%%%%%%

%%%%%%%%%%%%%%%%%%%%%%%%%%%%%%%%%%%%%%%%%%%%%%%%%%%%%


% \[
% \begin{array}{llll}
%  % \mbox{Index Term} & \idx, \nnatA & ::= &     i ~|~ n ~|~ \idx_1 + \idx_2 ~|~  \idx_1
%  %                                  - \idx_2 ~|~ \smax{\idx_1}{\idx_2}\\
% %                                  \mbox{Sort} & S & ::= & \nat \\
%   \mbox{Linear type} & \ltype &::=  &  \type \lto \type ~|~ \tbase \\
%   \mbox{Nonlinear Type} & \type & ::= & \bang{\idx} \ltype   \\
% \end{array}
% \]

% \begin{figure}
%   \begin{mathpar}
%     \inferrule{
%     }{
%       \ictx \tctx , x: \bang{\nnatA}\ltype, \Gamma' \tvdash{\nnatA} x: \bang{\nnatA}\ltype
%     }~\textbf{Ax}
%     %
%     \and
%     %
%     \inferrule{
%     }{
%       \ictx \Gamma \tvdash{\nnatA} c : \bang{\nnatA}\tbase 
%     }~\textbf{const}
%     %
%     % \and
%     % %
%     % \inferrule{
%     % }{
%     %   \ictx \Gamma \tvdash{\nnatA} \evec : \bang{\nnatA}\tbase 
%     % }~\textbf{Dict}
%     %
%     \and
%     %
%     \inferrule{
%       \ictx \Gamma, x: \type_1
%       \tvdash{\nnatA }
%       \expr: \type_2
%     }{
%       \ictx k+\Gamma \tvdash{k+\nnatA} \lambda x. \expr : \bang{k}  ( \type_1
%       \lto \type_2)
%     }~\textbf{lambda}
%     \and
%     %
%     \inferrule{
%       \ictx \Gamma_1  \tvdash{\nnatA_1} \expr_1:  \bang{0} ( \type_1
%       \lto \type_2      ) \\
%       \ictx \Gamma_2 \tvdash{\nnatA_2} \expr_2: \type_1 
%     }{
%       \ictx \max (\Gamma_1, \Gamma_2 ) \tvdash{\max( \nnatA_1,\nnatA_2) } \expr_1 \eapp \expr_2 : \type_2
%     }~\textbf{app}
%     %
%     \and
%     %
%     \inferrule{
%       \ictx \Gamma \tvdash{\nnatA} \expr: \bang{k} \ltype 
%     }{
%       \ictx \Gamma' ,1+\Gamma  \tvdash{1+\nnatA} \delta(\expr): \bang{k} \ltype 
%     }~\textbf{delta}
%      %
%     \and
%     %
%     \inferrule{
%       \ictx \Gamma'  \tvdash{\nnatA'} \expr: \type' \\
%       \Gamma' \leqslant \Gamma \\
%       \nnatA' \leq \nnatA\\
%       \sub{\type'}{\type} \\
%       \ictx \Gamma \tvdash{\nnatA} \expr: \bang{k} \ltype 
%     }{
%       \ictx \Gamma  \tvdash{\nnatA} \expr: \type 
%     }~\textbf{subtype}
%       %
%     \and
%     %
%     \inferrule{
%       \ictx \Gamma, y: \type', x: \type ,\Gamma'  \tvdash{\nnatA} \expr: \type 
%     }{
%       \ictx \Gamma, x: \type, y: \type' ,\Gamma'  \tvdash{\nnatA} \expr: \type 
%     }~\textbf{exchange}
%     \\\\
%     \boxed{
%  \inferrule{
%       \ictx \Gamma, x: \type_1
%       \tvdash{\nnatA }
%       \expr: \type_2
%     }{
%       \ictx k+\Gamma \tvdash{k} \lambda x. \expr : \bang{k}  ( \type_1
%       \lto^{\nnatA} \type_2)
%     }~\textbf{lambda}
%     \and
%     %
%     \inferrule{
%       \ictx \Gamma  \tvdash{\nnatA_1} \expr_1:  \bang{0} ( \type_1
%       \lto^{\nnatA} \type_2      ) \\
%       \ictx \Gamma \tvdash{\nnatA_2} \expr_2: \type_1 
%     }{
%       \ictx \Gamma  \tvdash{ \nnatA_1 + \max( \nnatA,\nnatA_2) } \expr_1 \eapp \expr_2 : \type_2
%     }~\textbf{app}
%     }
%     \\\\
% \begin{array}{lll}
%    k+\bang{r} \ltype  &\triangleq  &  \bang{k+r} \ltype  \\
%   k + \emptyset   & \triangleq & \emptyset   \\
%   k + ( [x : \type], \Gamma) & \triangleq &  [x : k+\type], k+\Gamma   
%   \\
%   \max(\bang{k_1} \ltype, \bang{k_2} \ltype) & \triangleq& \bang{ \max(k_1,
%                                                     k_2) } \ltype \\
%   \max(\Gamma, \emptyset) & \triangleq & \Gamma \\
%   \max(\emptyset, \Gamma) & \triangleq & \Gamma \\
%   \max\Big(  ([x : \type ],\Gamma),  ([x: \type'],\Delta)  \Big) & \triangleq
%                             & [x: \max(\type, \type')], \max(\Gamma,
%                               \Delta )\\
%   \sub{\Gamma}{\Delta} & \triangleq &  \dom(\Gamma) = \dom(\Delta)
%                                       \land \forall x \in
%                                       \dom(\Gamma), \sub{\Delta(x)}{\Gamma(x)}  
% \end{array}
%   \end{mathpar}
%   \caption{Typing rules, first version}
%   \label{fig:type-rules1}
% \end{figure}

% \begin{figure}
%   \begin{mathpar}
%     \inferrule{
%       k_1 \leq k \\
%       \sub{\ltype}{\ltype_1}
%     }{
%       \sub{\bang{k} \ltype}{\bang{k_1} \ltype_1}
%     }~\textsf{bang}
%     %
%     \and
%     %
%      \inferrule{
%         \sub{\type_1}{\type}   \\
%       \sub{\type'}{\type_1'}
%     }{
%       \sub{\type \lto \type' }{\type_1 \lto \ltype_1'}
%     }~\textsf{arrow}
%     %
%     \and
%     %
%     \inferrule{
%     }{
%     \sub{\tbase}{\tbase}
%     }~\textsf{base}
%   \end{mathpar}
%   \caption{subtyping}
%  \end{figure}

%  \clearpage

%  \begin{thm}[Weaking]
%   \label{sub}
%   \begin{enumerate} 
%    \item If $ \Gamma,x : \type' \tvdash{ \nnatA} \expr : \type $ and $
%   x \not \in \fv{\expr}  $ ,  then  $ \Gamma \tvdash{ \nnatA} \expr : \type $.
%   \end{enumerate}
% \end{thm}

% \begin{thm}[Value Adaptivity]
%   \label{sub}
%   \begin{enumerate} 
%    \item for all type $\bang{k} \ltype$,  exist value $\valr$, then  $
%      \empty \tvdash{ k} \valr : \bang{k} \ltype $.
%   \end{enumerate}
% \end{thm}

% \begin{thm}[Substitution]
%   \label{sub}
%   \begin{enumerate} 
%    \item If $ \Gamma,x : \type' \tvdash{ \nnatA} \expr : \type $ and $
%   \empty \tvdash{\nnatA'} \valr : \type'  $ , then  $\Gamma
%   \tvdash{\max(\nnatA,\nnatA' )} \expr[\valr/x]  : \type$. 
%   \end{enumerate}
% \end{thm}

% \begin{proof}
%   By induction on the typing derivation.\\
% \caseL{
%   $   \inferrule{
%     }{
%       \ictx \tctx , x: \bang{\nnatA}\ltype \tvdash{\nnatA} x: \bang{\nnatA}\ltype
%     }~\textbf{Ax}  $
%   }
% Assume $\empty \tvdash{\nnatA'} \valr : \bang{\nnatA}\ltype $, TS:  $\Gamma
%   \tvdash{\max(\nnatA,\nnatA' )} x[\valr/x]  : \type$. proved by
%   subtype rule on the assumption.
% \caseL{
%  $   \inferrule{
%     }{
%       \ictx \tctx ,y:\type', x: \bang{\nnatA}\ltype \tvdash{\nnatA} x: \bang{\nnatA}\ltype
%     }~\textbf{Ax2}  $
%   }
%   Assume $\empty \tvdash{\nnatA'} \valr : \bang{\nnatA}\ltype $, TS:
%   $\Gamma,   x: \bang{\nnatA}\ltype
%   \tvdash{\max(\nnatA,\nnatA' )} x[\valr/y]  : \type$. proved by rule
%   AX and then subtype.
%   \caseL{
%    \inferrule{
%       \ictx \Gamma, x: \type_1, y:\type'
%       \tvdash{\nnatA }
%       \expr: \type_2
%     }{
%       \ictx k+\Gamma, y: k + \type' \tvdash{k+\nnatA} \lambda x. \expr : \bang{k}  ( \type_1
%       \lto \type_2)
%     }~\textbf{lambda}
%   }
%    Assume $\empty \tvdash{k+\nnatA'} \valr : k+\type' $, TS:
%   $k+\Gamma
%   \tvdash{\max(k+\nnatA,k+\nnatA' )} (\lambda x. \expr)[\valr/y]  : \type$. From the
%   Lemma~\ref{para-dec} on the assumption, we know: $\empty
%   \tvdash{\nnatA'} \valr : \type' ~(1)$.\\
%   By Induction hypothesis on the premise, we get: $ \Gamma, x:\type_1
%   \tvdash{\max( \nnatA, \nnatA' )}
%       \expr[\valr/y]: \type_2 ~(2)$. By rule lambda, we conclude that
%       $k+\Gamma \tvdash{ k+ ( \max(\nnatA,\nnatA ) }
%       \lambda x.\expr[\valr/y]: \type_2 $.
%       \caseL{
%       \inferrule{
%       \ictx \Gamma_1,x:\type'  \tvdash{\nnatA_1} \expr_1:  \bang{0} ( \type_1
%       \lto \type_2      ) \\
%       \ictx \Gamma_2 ,x: \type'', \tvdash{\nnatA_2} \expr_2: \type_1 
%     }{
%       \ictx \max (\Gamma_1, \Gamma_2 ), x:\max(\type',\type'') \tvdash{\max( \nnatA_1,\nnatA_2) } \expr_1 \eapp \expr_2 : \type_2
%     }~\textbf{app}
%   }
%   Assume $\empty \tvdash{\nnatA'} \valr : \max(\type',\type'')$, TS: $\max (\Gamma_1, \Gamma_2 )
%   \tvdash{\max(\nnatA_1,\nnatA_2, \nnatA' )} (\expr_1 \eapp
%   \expr_2)[\valr/x]  : \type_2$. From the definition of $\max(\type',
%   \type'')$, we know that $\type'$ and $\type''$ have similar
%   form. Let us assume $\type'= \bang{k_1} \ltype$ and $\type'' =
%   \bang{k_2} \ltype$ so that $\max(\type',\type'') = \bang{\max(k_1,k_2)}
%   \ltype$.\\
%   From the Lemma~\ref{para-dec} on the assumption, we have $\empty
%   \tvdash{\nnatA' - (\max(k_1,k_2)-k_1) } \valr : \bang{k_1}
%   \ltype~(1)$ and $\empty
%   \tvdash{\nnatA' - (\max(k_1,k_2)-k_2) } \valr : \bang{k_2}
%   \ltype~(2)$.\\ By induction hypothesis on $(1)$ and $(2)$ respctively,
%   we know that:  $ \Gamma_1  \tvdash{ \max( \nnatA_1, \nnatA' - (\max(k_1,k_2)-k_1) ) } \expr_1[\valr/x]:  \bang{0} ( \type_1
%   \lto \type_2   ) ~(3)$  and $ \Gamma_2  \tvdash{\max(\nnatA_2 ,
%     \nnatA' - (\max(k_1,k_2)-k_2)   )} \expr_2[\valr/x]: \type_1 ~(4)$.  By the
%   rule app and $(3)$, $(4)$, we conclude that $$\max (\Gamma_1, \Gamma_2 )
%   \tvdash{\max(  \max( \nnatA_1, \nnatA' - (\max(k_1,k_2)-k_1) )  , \max(\nnatA_2 ,
%     \nnatA' - (\max(k_1,k_2)-k_2)   )  )} \expr_1[\valr/x] \eapp
%   \expr_2[\valr/x]  : \type_2 ~(5).$$
%   Because $\max(\nnatA' - (\max(k_1,k_2)-k_1) ) , \nnatA' -
%   (\max(k_1,k_2)-k_2)   ) \leq \nnatA' $, by subtype, we raise the
%   adaptivity to  $\max(\nnatA_1,\nnatA_2, \nnatA' ) $ from $(5)$.
%    \caseL{
%       \inferrule{
%       \ictx \Gamma_1,x:\type'  \tvdash{\nnatA_1} \expr_1:  \bang{0} ( \type_1
%       \lto \type_2      ) \\
%       \ictx \Gamma_2  \tvdash{\nnatA_2} \expr_2: \type_1 
%     }{
%       \ictx \max (\Gamma_1, \Gamma_2 ), x:\type' \tvdash{\max( \nnatA_1,\nnatA_2) } \expr_1 \eapp \expr_2 : \type_2
%     }~\textbf{app2}
%   }
%   It is another case for application when x only appear in the first
%   premise. In this case, $\expr_2[\valr/x] = \expr_2$. Another case
%   when variable x only appears in the second premise can be proved in
%   a similar way.\\
%   Assume $\empty \tvdash{\nnatA'} \valr :\type'$. TS:$\max (\Gamma_1, \Gamma_2 )
%   \tvdash{\max(\nnatA_1,\nnatA_2, \nnatA' )} (\expr_1 \eapp
%   \expr_2)[\valr/x]  : \type_2$.  By Induction Hypothesis on the first
%   premise using the assumption, we get: $\Gamma_1
%   \tvdash{\max(\nnatA_1, \nnatA')} \expr_1[\valr/x]:  \bang{0} ( \type_1
%       \lto \type_2  )  ~(1)$. By the rule app using (1) and the second
%       premise, we conclude that $$ \max (\Gamma_1, \Gamma_2 )
%       \tvdash{\max( \max(\nnatA_1,\nnatA'),\nnatA_2) }
%       \expr_1[\valr/x] \eapp \expr_2 : \type_2$$
%       \caseL{
%  \inferrule{
%       \ictx \Gamma, x:\type' \tvdash{\nnatA} \expr: \bang{k} \ltype 
%     }{
%       \ictx \Gamma' ,1+\Gamma, x:1+\type'  \tvdash{1+\nnatA} \delta(\expr): \bang{k} \ltype 
%     }~\textbf{delta}
%   }
%   Assume $\empty \tvdash{\nnatA'+1} \valr : 1+\type' $, TS: $ \Gamma'
%   ,1+\Gamma \tvdash{\max(1+\nnatA, 1+\nnatA')} \delta(\expr)
%   [\valr/x]: \bang{k} \ltype $.
%   By Lemma~\ref{para-dec} on the assumption, we have $\empty
%   \tvdash{\nnatA'} \valr : \type'~(1) $. By IH on the first premise
%   along with (1), we have: $\Gamma \tvdash{\max(\nnatA, \nnatA')}
%   \expr[\valr/x]: \bang{k} \ltype~ (2)$.
%    By the rule delta using (2), we conclude that $\Gamma' ,1+\Gamma  \tvdash{1+(\nnatA,\nnatA')} \delta(\expr[\valr/x]): \bang{k} \ltype$.
% \end{proof}



% \begin{proof}
%   By Induction on the typing derivation.
%   \caseL{
%      $   \inferrule{
%     }{
%       \ictx \tctx , x: \bang{\nnatA}\ltype \tvdash{\nnatA} x: \bang{\nnatA}\ltype
%     }~\textbf{Ax}  $
%   }
%   Assume $\env= \Big( \env_1, [x \to (\valr,\adapt
%   )] , \Big) \vDash (\tctx , x: \bang{\nnatA}\ltype  )$ where $\env_1 \vDash \Gamma$. We know that $
%   \empty \tvdash{\adapt} \valr : \bang{\nnatA}\ltype $.
%   From the evaluation rule var, we know $\env , x \bigstep{\adapt} \valr,
%   \env  $.
%   TS:  $ \adapt + adap(\valr, \env)  \leq  \nnatA +
%   F(\env) \implies \adapt + 0 \leq \nnatA + \max( \adapt, F(\env_1))
%   $.It is trivially true.
% \caseL{
%   $
%     \inferrule{
%       \ictx \Gamma, x: \type_1
%       \tvdash{\nnatA }
%       \expr: \type_2
%     }{
%       \ictx k+\Gamma \tvdash{k+\nnatA} \lambda x. \expr : \bang{k}  ( \type_1
%       \lto \type_2)
%     }~\textbf{lambda}
%   $
% }
% Choose $\env \vDash  (k+\Gamma)$ so that $\forall x_i \in
% (\Gamma). \env(x_1) =(\valr_i, \adapt_i ) \land \empty
% \tvdash{\adapt_i } \valr_i: k+\Gamma(x_i) $.  By the evaluation rule
% we know $\env, \lambda x. \expr \bigstep{0}
%                                        \lambda x.\expr, \env $, TS: $0
%                                        + \adap(\lambda x.\expr, \env)
%                                        \leq  k+\nnatA + F(\env)$, which is trivially
%                                        true because $ \adap(\lambda
%                                        x.\expr, \env) \leq F(\env) $.
                                       
% \caseL{
%     $  \inferrule{
%       \ictx \Gamma_1  \tvdash{\nnatA_1} \expr_1:  \bang{0} ( \type_1
%       \lto \type_2      ) \\
%       \ictx \Gamma_2 \tvdash{\nnatA_2} \expr_2: \type_1 
%     }{
%       \ictx \max (\Gamma_1, \Gamma_2 ) \tvdash{\max( \nnatA_1,\nnatA_2) } \expr_1 \eapp \expr_2 : \type_2
%     }~\textbf{app}  $
%   }
%   Choose $\env = [x_i \to (\valr_i,0)] $ for all $x_i$ in
%   $\dom(\max(\Gamma_1,\Gamma_2))$
%   so that  $\empty \tvdash{\nnatA_i} \valr_i  : (\max(\Gamma_1,
%   \Gamma_2)(x_i) $.
%   From the definition, we know that $\env \vDash \Gamma_1$ and $\env
%   \vDash \Gamma_2$. Because $\expr_1$ has the arrow type and will be
%   evaluated to a function, assume exists $\env_1$ so that $\env,
%   \expr_1 \bigstep{\adapt_1} \lambda x.\expr , \env_1 $.  By induction
%   hypothesis on the first premise, we know that: $\adapt_1 +
%   \adap(\lambda x. \expr, \env_1) \leq \nnatA_1 + F(\env,
%   \Gamma_1)~(1)$.Assume exists $\env_2$ so that $\expr_2$ is evaluated
%   to an arbitrary value $\valr_2$ : $ \env, \expr_2 \bigstep{\adapt_2}
%   \valr_2 , \env_2$, by induction hypothesis, we conclude that :  $\adapt_2 +
%   \adap(\valr , \env_2) \leq \nnatA_2 + F(\env,
%   \Gamma_2)~(2)$.
                            


% \[
% \inferrule{
%     \env, \expr_1 \bigstep{\adapt_1} \lambda x.\expr , \env_1 \\
%     \env, \expr_2 \bigstep{\adapt_2} \valr_2 , \env_2 \\
%     (\env_1 \uplus \env_2)[ x  \to (\valr_2,   \adapt_2  ) ], \expr
%     \bigstep{\adapt_3} \valr, \env_3
%   }{
%     \env, \expr_1 \eapp \expr_2 \bigstep{\adapt_1+\adapt_3} \valr, \env_3
%   }~\textsf{app}
% \]
%  \end{proof} 


\end{document}



