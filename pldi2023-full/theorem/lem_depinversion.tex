\begin{lem}[The One-Step Event Dependency Inversion]
	\label{lem:flowsto_soundness_emptytrace}
	For every $ c \in \cdom, D \in \dbdom$ and two assignment events $\event_1, \event_2 \in \eventset^{\asn}$,
	if $\eventdep(\event_1, \event_2, [\event_1; \event_2],  c, D) $,
	then, $\flowsto(\pi_1(\event_1)^{\pi_2(\event_1)}, \pi_1(\event_2)^{\pi_2(\event_2)}, c)$.
	%
	\[
	\begin{array}{l}
		\forall \event_1, \event_2 \in \eventset^{\asn}, c \in \cdom, D \in \dbdom 
		\st 
		\eventdep(\event_1, \event_2, [\event_1; \event_2],  c, D) 
		\\ \quad 
		\implies 
		\flowsto(\pi_1(\event_1)^{\pi_2(\event_1)}, \pi_1(\event_2)^{\pi_2(\event_2)}, c)
	\end{array}
	\]
\end{lem}
Proof Summary:
\\
1. case of (the labelled unique assignment command associated to the $\event_2$ 
is executed but the value assigned to the variable in this event is changed in second execution)
\\
show x directly used by the assignment of the second event
\\
2.(the labelled unique assignment command associated to the $\event_2$ isn't executed in second execution)
\\
show x is directly used by the boolean expression for a conditional command and second event shows in the body of that conditional command 
%
\begin{proof}
	By the Definition~\ref{def:event_dep} for $\eventdep(\event_1, \event_2, [\event_1; \event_2], c, D)$, 
	we know there are 2 cases:
	%
	\caseL{1}
		\textbf{(the labelled unique assignment command associated to the $\event_2$ 
		is executed but the value assigned to the variable in this event is changed in second execution).}
	\begin{subproof}[Proof of the Basecase: Case 1]
%
\label{pf:eventdep_base_val}
We have the following by the definition $\eventdep(\event_1, \event_2, [\event_1; \event_2], c, D)$ for case 1:
\begin{equation}
  \label{eq:eventdep_def_base_val}
  \exists \vtrace_0,
    \vtrace_1, \vtrace' \in \mathcal{T},\event_1' \in \eventset^{\asn}, \event_2' \in \eventset, {c}_1, {c}_2  \in \cdom  \st
    \diff(\event_1, \event_1') \land
      \left(
      \begin{array}{ll}   
     & \config{{c}, \vtrace_0} \rightarrow^{*} 
    \config{{c}_1, \vtrace_1 \tracecat [\event_1]}  \rightarrow^{*} 
      \config{{c}_2,  \vtrace_1 \tracecat [\event_1; \event_2] } 
      % 
     \\ 
     \bigwedge &
      \config{{c}_1, \vtrace_1 \tracecat [\event_1']}  \rightarrow^{*} 
      \config{{c}_2,  \vtrace_1 \tracecat[ \event_1'] \tracecat \vtrace' \tracecat [\event_2'] } 
    \\
    \bigwedge & 
    \diff(\event_2,\event_2') \land 
    \vcounter(\vtrace) ~ \pi_2(\event_2)
    = 
    \vcounter(\vtrace') ~ \pi_2(\event_2)\\
    \end{array}
    \right)
  \end{equation}
  %
Let $\vtrace_0,
\vtrace_1, \vtrace' \in \mathcal{T}, \event_2' \in \eventset, \event_1' \in \eventset^{\asn}, {c}_1, {c}_2$ be the traces, events and commands satisfying the executions,
by Inversion Lemma~\ref{lem:inv_event} on 
$\event_1$, $\event_2$, we have the following instance of the first execution in Eq.~\ref{eq:eventdep_def_base_val},
 %
%
 %
\begin{equation}
\label{eq:valdep_inv1}
  \begin{array}{l}   
\config{{c}, \vtrace_0} \rightarrow^{*} 
\config{[\assign{x_1}{\expr_1 / \query(\qexpr_1)}]^{\pi_2(\event_1)} ; {c}_1, 
\vtrace_1}  
\rightarrow^\rname{assn/query}
 \config{c_1, \vtrace_1 \tracecat [\event_1]} \\
  \qquad \rightarrow^{*} 
  \config{[\assign{{x}_2}{\expr_2 / \query(\qexpr_2)}]^{l_2};{c}_2, 
  \vtrace_1 \tracecat [\event_1]} 
  \rightarrow^\rname{assn/query} 
  \config{{c}_2,  \vtrace_1 \tracecat [\event_1; \event_2]} 
  % 
\end{array}
\footnote{
$\assign{x}{\expr / \query(\qexpr)}$ denotes variable $x$ is assigned by either an expression $\expr$ or query $\query(\qexpr)$
}
\end{equation}
%
% \wqside{Some typo in equation 4, but I can follow:-)}
% \jl{thanks}
, where $x_1 = \pi_1(\event_1)$, $l_1 = {\pi_2(\event_1)}$, $x_2 = \pi_1(\event_2)$, $l_2 = \pi_2(\event_2)$, 
and $\expr_1 / \qexpr_1$, $\expr_2 / \qexpr_2$ are the expressions of the assignment commands 
associated to the $\event_1$ and $\event_2$ from  Lemma~\ref{lem:inv_event}.
\\
%
By $\diff(\event_2,\event_2')$ and the command label consistency,
we also have the instance of second execution in Eq.~\ref{eq:eventdep_def_base_val} as follows:
% \[
% \config{{c}_1, \vtrace_1 \tracecat [\event_1']}  \rightarrow^{*} 
%   \config{{c}_2',  \vtrace_1 \tracecat [\event_1']\cdot \vtrace_2' \tracecat [\event_2'] } 
%   \], 
% we know there exists $\expr_2'$ or $\qexpr_2'$ and following execution instance,
%  \[
%   \begin{array}{l}   
%   \config{c_1, \vtrace_1 \tracecat [\event_1']} 
%   \rightarrow^{*} 
%   \config{[\assign{{x}_2'}{\expr_2' / \query(\qexpr_2')}]^{l_2'} ; {c}_2', \vtrace_1 \tracecat [\event_1']\tracecat \vtrace'} 
%   \rightarrow^\rname{assn/query} 
%   \config{{c}_2',   \vtrace_1 \tracecat [\event_1'] \tracecat \vtrace' \tracecat [\event_2']} 
%   % 
% \end{array}
%  \]
%  , where  $x_2' = \pi_1(\event_2')$ and $l_2' = \pi_2(\event_2')$.
% %
% Unfolding $\diff(\event_2,\event_2')$, we have:
% \[
%   x_2 = x_2' \land l_2 = l_2' 
% \] 
% %
% Since each command in $c$ has a unique label, we have $\expr_2' = \expr_2$, $\qexpr_2 = \qexpr_2'$, and following execution instance:
\begin{equation}
\label{eq:valdep_inv2}
  \config{c_1, \vtrace_1 \tracecat [\event_1']} 
  \rightarrow^{*} 
  \config{[\assign{{x}_2}{\expr_2 / \query(\qexpr_2)}]^{l_2} ; {c}_2, \vtrace_1 \tracecat [\event_1']\cdot \vtrace_2'} 
  \rightarrow^\rname{assn/query} 
  \config{{c}_2,  \vtrace_1 \tracecat [\event_1']\cdot \vtrace_2' \tracecat [\event_2']} 
\end{equation}
%
From Eq.~\ref{eq:eventdep_def_base_val}, we also have
\begin{equation}
\label{eq:valdep_invn}
  \vcounter(\vtrace') l_2 = \vcounter( [] ) l_2 = 0
\end{equation}
%
%
% By Induction on the operational semantics rules on following execution from Eq.~\ref{eq:valdep_inv1}: 
% \wqside{Surprised we do induction here:-)}
 %
%  \[\config{c_1, \vtrace_1 \tracecat [\event_1]}
%   \rightarrow^{*} 
%   \config{[\assign{{x}_2}{\expr_2 / \query(\qexpr_2)}]^{l_2};{c}_2, 
%   \vtrace_1 \tracecat [\event_1]} 
% \]
 %
%
By Inversion Lemma~\ref{lem:inv_skip} and the execution in Eq.~\ref{eq:valdep_inv1}, we know:
 \[
 c_1 =_c 
 [\eskip]{}^*;[\assign{{x}_2}{\expr_2 / \query(\qexpr_2)}]^{l_2};{c}_2
 	\footnote{$([\eskip];){}^*$ denotes a sequence command only composed of $[\eskip]$ commands.}
 \]
 %
By substituting $c_1$ in Eq.~\ref{eq:valdep_inv2}, the following subproof shows there is only 1 qualified
 instance of the execution in Eq.~\ref{eq:valdep_inv2}.
\begin{subproof}[Subproof]
\label{pf:noiteration_inv2}
There are two possibilities by the law of excluded middle:
\\
$[\assign{{x}_2}{\expr_2 / \query(\qexpr_2)}]^{l_2} \in_c c_2$ 
\\
or $[\assign{{x}_2}{\expr_2 / \query(\qexpr_2)}]^{l_2} \notin_c c_2$.
%
\begin{enumerate}
\item{$[\assign{{x}_2}{\expr_2 / \query(\qexpr_2)}]^{l_2}\notin_c c_2$}
\\
In this case, we have the following execution instance:
%
\footnote{$\rightarrow^{\rname{skip}^*}$ denotes the rule applied on 
every evaluation step of this execution is the $\rname{skip}$ rule.}
 %
  \[
  \config{c_1, \vtrace_1 \tracecat [\event_1']} 
  \rightarrow^{\rname{skip}^*} 
  \config{[\assign{{x}_2}{\expr_2 / \query(\qexpr_2)}]^{l_2} ; {c}_2, \vtrace_1 \tracecat [\event_1']} 
  \rightarrow^\rname{assn/query} 
  \config{{c}_2,  \vtrace_0 \tracecat \vtrace_1 \tracecat [\event_1'; \event_2']} 
 \]
%
\item{$[\assign{{x}_2}{\expr_2 / \query(\qexpr_2)}]^{l_2} \in_c c_2$}
\\
By Inversion Lemma~\ref{lem:inv_while}, 
we have a $\ewhile$ conditional command
 $(\ewhile [b_w]^l_w \edo c_w)$ in $c_2$, where
% $(\ewhile [b_w]^l_w \edo c_w) \in_c c_2$ and 
$[\assign{{x}_2}{\expr_2 / \query(\qexpr_2)}]^{l_2} \in_c c_w$.
% \\
Then, we have the following possible execution instances
 %
  \[
  \config{c_1, \vtrace_1 \tracecat [\event_1']} 
  \rightarrow^{\rname{skip}^*} 
  \config{[\assign{{x}_2}{\expr_2 / \query(\qexpr_2)}]^{l_2} ; {c}_2, \vtrace_1 \tracecat [\event_1']} 
  \rightarrow^\rname{assn/query} 
  \config{{c}_2,  \vtrace_1 \tracecat [\event_1']\tracecat [\event_2']} 
 \]
%
  \[
  \begin{array}{l}
  \config{c_1, \vtrace_1 \tracecat [\event_1']} 
  \rightarrow^{\rname{skip}^*} 
  \config{[\assign{{x}_2}{\expr_2 / \query(\qexpr_2)}]^{l_2} ; {c}_2, \vtrace_1 \tracecat [\event_1']} 
  \rightarrow^\rname{assn/query} 
  \config{{c}_2,  \vtrace_1 \tracecat [\event_1']\tracecat [(x_2, l_2,  v_2')]} 
  \\ \qquad
  \rightarrow^{*} 
  \config{[\assign{{x}_2}{\expr_2 / \query(\qexpr_2)}]^{l_2} ; {c}_2, 
  \vtrace_1 \tracecat [\event_1']\tracecat [(x_2, l_2,  v_2')] \tracecat \trace_3} 
  \\ \qquad
  \rightarrow^\rname{assn/query} 
  \config{{c}_2,  \vtrace_1 \tracecat [\event_1']\tracecat [(x_2, l_2,  v_2')] \tracecat \trace_3 \tracecat [\event_2']} 
 \end{array}
 \]
\[
  \cdots
\] 
, where each execution instance iterates the conditional command 
$(\ewhile [b_w]^l_w \edo c_w)$ in $c_2$, $0, 1$ or more times.
%
\\
%
For each execution instance, we have the corresponding instance of $\trace'$ as follows:
\\
$\trace'  = [] $
\\
$\trace' = [(x_2, l_2,  v_2')] \tracecat \trace_3 $
%
\\
$\cdots$
%
\\
%
By Eq.~\ref{eq:valdep_invn} where $\vcounter(\trace') l_2 = 0$,
%
we know only the first execution instance with 0 iteration of $\ewhile$ command in $c_2$ satisfies this restriction, 
i.e., $\trace' = []$.
%
\end{enumerate}
In conclusion, we have the only qualified execution instance as follows where $\trace' = []$.
  \[
    \config{c_1, \vtrace_1 \tracecat [\event_1']} 
    \rightarrow^{\rname{skip}^*} 
    \config{[\assign{{x}_2}{\expr_2 / \query(\qexpr_2)}]^{l_2} ; {c}_2, \vtrace_1 \tracecat [\event_1']} 
    \rightarrow^\rname{assn/query} 
    \config{{c}_2,  \vtrace_1 \tracecat [\event_1']\tracecat [\event_2']} 
 \]
\end{subproof}
%
Then we know by the environment definition,
$\env$ obtains different values only for variable $x_1$ 
from trace $\vtrace_1 \tracecat [\event_1]$ and 
$\vtrace_1 \tracecat [\event_1']$, i.e.,
\[
  \forall z^r \in \lvar_c \setminus \{x_1^{l_1}\} ,
  \env(\vtrace_1 \tracecat [\event_1]) (z) =  
  \env(\vtrace_1 \tracecat [\event_1']) (z)
\]
%
By {Inversion Lemma~\ref{lem:inv_expr}} of arithmetic expression evaluation, we have
\[
  x_1 \in VAR(\expr_2 / \qexpr_2) 
\]
Since $\llabel(\vtrace_1 \tracecat [\event_1]) x_1 = l_1$, 
% by Inversion Lemma~\ref{lem:inv_live} we know $x_1^{l_1} \in \live^{l_2}(c)$.
by Inversion Lemma~\ref{lem:inv_live} we know $x_1^{l_1} \in \live(l_2, c)$.
%
\\
%
By $\flowsto$ definition, we have:
%
\[
\flowsto(x_1^{l_1}, {x}_2^{l_2}, c)
\]
i.e.,
%
\[
\flowsto(\pi_1(\event_1)^{\pi_2(\event_1)}, \pi_1(\event_2)^{\pi_2(\event_2)}, c)
 \]
%
This case is proved.
\end{subproof}
	% \end{case}
	\caseL{2}
		\textbf{(the labelled unique assignment command associated to the $\event_2$ isn't executed in second execution).}
		\\
		Proof Summary:
		\\
		1. Let $\event_b$ be the testing event,
		in the same way of case 1, we get:
		 $\pi_1(\event_1) \in VAR(\pi_1(\event_b)) 
		 \land 
		%  l_1 \in \live^{l_b}$
		\pi_1(\event_1)^{l_1} \in \live(l_b, c)$
		%
		 \\
		 2. By Lemma~\ref{lem:ctldep_inv}, we know:
		   $\forall z \in VAR(\pi_1(\event_b)) \st \exists i \in \mathbb{N} \st
		 \flowsto(z^i, \pi_1(\event)^{\pi_2(\event)}, c)$
		 %
		 \\
		 3. By $\flowsto$ definition we have:
		   $\flowsto(\pi_1(\event_1)^{\pi_2(\event_1)}, \pi_1(\event_2)^{\pi_2(\event_2)}, c)$
		\begin{subproof}[Proof of the Basecase: Case 2]
%
\label{pf:eventdep_base_ctl}
We have the following by the definition $\eventdep(\event_1, \event_2, [\event_1; \event_2], c, D)$ of case 2:
\begin{equation}
  \label{eq:eventdep_def_base_ctl}
  \begin{array}{ll}   
    & \exists \vtrace_0,
    \vtrace_1, \vtrace', \vtrace_3, \vtrace_3' \in \mathcal{T},\event_1' \in \eventset^{\asn}, {c}_1, {c}_2  \in \cdom, 
    \event_b \in \eventset^{\test}
    \st
    \\ 
   &   \qquad    \diff(\event_1, \event_1') 
\land
   \Big(
  \config{{c}, \vtrace_0} \rightarrow^{*} 
      \config{{c}_1, \vtrace_1 \tracecat [\event_1]}  \rightarrow^{*} 
      \config{c_2,  \trace_1 \tracecat [\event_1;\event_b] \tracecat  \trace_3} 
    \\   
   & \qquad \bigwedge 
    \config{{c}_1, \vtrace_1 \tracecat [\event_1']}  \rightarrow^{*} 
    \config{c_2,  \vtrace_1 \tracecat [\event_1'] \tracecat \trace' \tracecat [(\neg \event_b)] \tracecat \trace_3'} 
    \\
    & \qquad \bigwedge  \tlabel_{\trace_3} \cap \tlabel_{\trace_3'} = \emptyset
     \land \vcounter(\trace') ~  \pi_2(\event_b) = \vcounter(\trace) ~  \pi_2(\event_b)
      \land \event_2 \eventin \trace_3
    \land \event_2 \not\eventin \trace_3'
   \Big)
 \end{array}
  \end{equation}
  %
%
Let $\vtrace_0,
\vtrace_1, \vtrace', \vtrace_3, \vtrace_3' \in \mathcal{T}, 
\event_2' \in \eventset, \event_1' \in \eventset^{\asn}, \event_b, {c}_1, {c}_2$ be the traces, events and commands satisfying the executions,
by Inversion Lemma~\ref{lem:inv_event} on 
$\event_1$, $\event_2$, and $\event_b$,
we have the following instance of the first execution in Eq.~\ref{eq:eventdep_def_base_ctl},
 %
%
% Let $\event_{ih} = (b, l_b, n_b, v_b)$, by Eq.~\ref{eq:ctldep_inv1} and {Inversion Lemma~\ref{lem:inv_test}}, we have:
\begin{equation}
\label{eq:ctldep_inv1}
  \begin{array}{l}   
\config{{c}, \vtrace_0} \rightarrow^{*} 
\config{[\assign{{x}_1}{\expr_1 / \query(\qexpr_1)}]^{l_1} ; {c}_1, \vtrace_1}  
\rightarrow^{assn/query}
 \config{c_1, \vtrace_1 \tracecat [\event_1]} 
 \\
  \qquad \rightarrow^{*} 
  \config{\eif ([b]^{l_b}, c_t, c_f) / \ewhile [b]^{l_b} \edo c_w;{c}_3', 
  \vtrace_1 \tracecat [\event_1]} 
  \\
  \qquad 
   \rightarrow^{\rname{if-b / while-b}} 
  \config{(c_t;c_3' / c_f;c_3') /(c_3' / c_w; \ewhile [b]^{l_b} \edo c_w;{c}_3'), 
  \trace_1 \tracecat [\event_1;\event_b]} 
  \\
  \qquad   \rightarrow^{*} 
  \config{c_3, 
  \trace_1 \tracecat [\event_1;\event_b] \tracecat  \trace_3}
  % 
\end{array}
\end{equation}
%  %
% \begin{equation}
% \label{eq:ctldep_inv1}
%   \begin{array}{l}   
% \config{{c}, \vtrace_0} \rightarrow^{*} 
% \config{[\assign{x_1}{\expr_1 / \query(\qexpr_1)}]^{\pi_2(\event_1)} ; {c}_1
% \footnote{
% $\assign{x}{\expr / \query(\qexpr)}$ denotes variable $x$ is assigned by either an expression $\expr$ or query $\query(\qexpr)$
% }, 
% \vtrace_1}  
% \rightarrow^\rname{assn/query}
%  \config{c_1, \vtrace_1 \tracecat [\event_1]} \\
%   \qquad \rightarrow^{*} 
%   \config{[\assign{{x}_2}{\expr_2 / \query(\qexpr_2)}]^{l_2};{c}_2, 
%   \vtrace_1 \tracecat [\event_1]} 
%   \rightarrow^\rname{assn/query} 
%   \config{{c}_2,  \vtrace_1 \tracecat [\event_1; \event_2]} 
%   % 
% \end{array}
% \end{equation}
% %
% % \wqside{Some typo in equation 4, but I can follow:-)}
% % \jl{thanks}
, where $x_1 = \pi_1(\event_1)$, $l_1 = {\pi_2(\event_1)}$, 
% $x_2 = \pi_1(\event_2)$, $l_2 = \pi_2(\event_2)$, 
and $\eif ([b]^{l_b}, c_t, c_f) / \ewhile [b]^{l_b} \edo c_w$ 
is the conditional command of the assignment commands 
associated to the $\event_b$ from Inversion Lemma~\ref{lem:inv_event} of testing event.
\\
%
By the command label consistency,
we also have the instance of second execution in Eq.~\ref{eq:eventdep_def_base_ctl} as follows:
% \[
% \config{{c}_1, \vtrace_1 \tracecat [\event_1']}  \rightarrow^{*} 
%   \config{{c}_2',  \vtrace_1 \tracecat [\event_1']\cdot \vtrace_2' \tracecat [\event_2'] } 
%   \], 
% we know there exists $\expr_2'$ or $\qexpr_2'$ and following execution instance,
%  \[
%   \begin{array}{l}   
%   \config{c_1, \vtrace_1 \tracecat [\event_1']} 
%   \rightarrow^{*} 
%   \config{[\assign{{x}_2'}{\expr_2' / \query(\qexpr_2')}]^{l_2'} ; {c}_2', \vtrace_1 \tracecat [\event_1']\tracecat \vtrace'} 
%   \rightarrow^\rname{assn/query} 
%   \config{{c}_2',   \vtrace_1 \tracecat [\event_1'] \tracecat \vtrace' \tracecat [\event_2']} 
%   % 
% \end{array}
%  \]
%  , where  $x_2' = \pi_1(\event_2')$ and $l_2' = \pi_2(\event_2')$.
% %
% Unfolding $\diff(\event_2,\event_2')$, we have:
% \[
%   x_2 = x_2' \land l_2 = l_2' 
% \] 
% %
% Since each command in $c$ has a unique label, we have $\expr_2' = \expr_2$, $\qexpr_2 = \qexpr_2'$, and following execution instance:
\begin{equation}
\label{eq:ctldep_inv2}
\begin{array}{l}   
  \config{{c}, \vtrace_0} \rightarrow^{*} 
  \config{[\assign{{x}_1}{\expr_1 / \query(\qexpr_1)}]^{l_1} ; {c}_1, \vtrace_1}  
  \rightarrow^{assn/query}
   \config{c_1, \vtrace_1 \tracecat [\event_1]} 
   \\
    \qquad \rightarrow^{*} 
    \config{\eif ([b]^{l_b}, c_t, c_f) / \ewhile [b]^{l_b} \edo c_w;{c}_3', 
    \vtrace_1 \tracecat [\event_1] \tracecat \trace'} 
    \\
    \qquad 
     \rightarrow^{\rname{if-b / while-b}} 
    \config{(c_f;c_3' / c_t;c_3') /(c_w; \ewhile [b]^{l_b} \edo c_w;{c}_3' / c_3'), 
    \trace_1 \tracecat [\event_1]  \tracecat \trace' \tracecat [\neg \event_b]} 
    \\
    \qquad   \rightarrow^{*} 
    \config{c_3, 
    \trace_1 \tracecat [\event_1]  \tracecat \trace' \tracecat [\neg \event_b] \tracecat  \trace_3'}
    % 
  \end{array}
\end{equation}
%
From Eq.~\ref{eq:eventdep_def_base_ctl}, we also have
  $\vcounter(\vtrace') l_b = \vcounter( [] ) l_b = 0$.
\\
%
%
By the same proof steps from case 1 in Subproof~\ref{pf:eventdep_base_val}, we have
\[
  x_1 \in VAR(b)  \land x_1^{l_1} \in \live(l_b, c)
\]
%
By Lemma~\ref{lem:ctldep_inv}, we also know:
\[
  \forall z \in VAR(\pi_1(\event_b)) \st \exists i \in \mathbb{N} \st
\flowsto(z^i, \pi_1(\event)^{\pi_2(\event)}, c)
\]
%
Then by $\flowsto$ definition, we have $\flowsto(x_1^{l_1}, {x}_2^{l_2}, c)$
%
i.e.,
%
\[
\flowsto(\pi_1(\event_1)^{\pi_2(\event_1)}, \pi_1(\event_2)^{\pi_2(\event_2)}, c)
 \]
%
This case is proved.
\end{subproof}
\end{proof}
%
\begin{lem}[Control Dependency Inversion]
	\label{lem:ctldep_inv}
	For every $c \in \cdom$, $D \in \dbdom, \trace \in \mathcal{T}$ and two assignment events $\event_1, \event_2 \in \eventset^{\asn} $, if they are in the second case of the \emph{Event May-Dependency} relation from Definition.~\ref{def:event_dep},
	$\eventdep(\event, \event, c, \trace, D)$ as Eq.~\ref{eq:ctlflowsto_inv},
	then for all  $z \in VAR(\pi_1(\event_b))$ there exists a label $i \in \mathbb{N}$ such that 
	$\flowsto(z^i, \pi_1(\event)^{\pi_2(\event)}, c)$
	\begin{equation}
		\label{eq:ctlflowsto_inv},		
		\begin{array}{l}
			\forall D \in \dbdom , c \in \cdom, \trace \in \mathcal{T},
			\event_1, \event_2 \in \eventset^{\asn} \st 
			\\ 
			\exists \vtrace_0,
			\vtrace_1, \vtrace', \vtrace_3, \vtrace_3' \in \mathcal{T},\event_1' \in \eventset^{\asn}, {c}_1, {c}_2  \in \cdom, 
			\event_b \in \eventset^{\test},
			\trace_{ih} \in \mathcal{T} \st 
		\trace = [\event_1] \tracecat \trace_{ih} \tracecat [\event_2]
		\\ \quad \implies	  
			  \config{{c}, \vtrace_0} \rightarrow^{*} 
				\config{{c}_1, \vtrace_1 \tracecat [\event_1]}  \rightarrow^{*} 
				\config{c_2,  \vtrace_1 \tracecat [\event_1] \tracecat \trace \tracecat [\event_b] \tracecat  \trace_3} 
			  \\ \qquad \land
			  \config{{c}_1, \vtrace_1 \tracecat [\event_1']}  \rightarrow^{*} 
			  \config{c_2,  \vtrace_1 \tracecat [\event_1'] \tracecat \trace' \tracecat [(\neg \event_b)] \tracecat \trace_3'} 
			  \\ \qquad \land
			\tlabel_{\trace_3} \cap \tlabel_{\trace_3'} = \emptyset
			   \land \vcounter(\trace') ~  \pi_2(\event_b) = \vcounter(\trace) ~  \pi_2(\event_b)
				\land \event_2 \eventin \trace_3
			  \land \event_2 \not\eventin \trace_3'
		\\ \quad \implies	
		\forall z \in VAR(\pi_1(\event_b)) \st 
		\exists l \in \mathbb{N} \st 
		\flowsto(z^l, \pi_1(\event_2)^{\pi_2(\event_2)},c)
	\end{array}
\end{equation}
	\end{lem}
	Proof Summary:
	\\
	Proving by using the Inversion Lemmas~\ref{lem:inv_expr}, \ref{lem:inv_expr_gnl}, 
	\ref{lem:inv_event}, and \ref{lem:inv_live}, and the \emph{Event May-Dependency} definition of the second case.
%
	\begin{proof}
		Take arbitrary $D \in \dbdom , c \in \cdom, \trace \in \mathcal{T},
		\event_1, \event_2 \in \eventset^{\asn} $,
%
let $\vtrace_0,
\vtrace_1, \vtrace', \vtrace_3, \vtrace_3' \in \mathcal{T}, 
\event_2' \in \eventset, \event_1' \in \eventset^{\asn}, \event_b, {c}_1, {c}_2$ be the traces, 
events and commands satisfying the executions,
by Inversion Lemma~\ref{lem:inv_event} on 
$\event_2$, and $\event_b$,
we have the following instance of the first execution in Eq.~\ref{eq:ctlflowsto_inv},
 %
%
% Let $\event_{ih} = (b, l_b, n_b, v_b)$, by Eq.~\ref{eq:ctldep_inv1} and {Inversion Lemma~\ref{lem:inv_test}}, we have:
\begin{equation}
% \label{eq:ctldep_inv1}
  \begin{array}{l}   
\config{{c}, \vtrace_0} \rightarrow^{*} 
% \config{[\assign{{x}_1}{\expr_1 / \query(\qexpr_1)}]^{l_1} ; {c}_1, \vtrace_1}  
% \rightarrow^{\rname{assn/query}}
%  \config{c_1, \vtrace_1 \tracecat [\event_1]} 
%  \\ \qquad 
%  \rightarrow^{*} 
  \config{\eif ([b]^{l_b}, c_t, c_f) / \ewhile [b]^{l_b} \edo c_w;{c}_3', 
  \vtrace_1 \tracecat [\event_1] \tracecat \trace} 
  \\
  \qquad 
   \rightarrow^{\rname{if-b / while-b}} 
  \config{(c_t;c_3' / c_f;c_3') /(c_w; \ewhile [b]^{l_b} \edo c_w;c_3'/[\eskip]; c_3'), 
  \trace_1 \tracecat [\event_1] \tracecat \trace \tracecat [\event_b]} 
  \\
  \qquad  \rightarrow^{*} 
  \config{[\assign{{x}_2}{\expr_2 / \query(\qexpr_2)}]^{l_2}; c_{3b}', 
  \trace_1 \tracecat [\event_1] \tracecat \trace \tracecat [\event_b] \tracecat  \trace_{3a}}
  \\ \qquad \rightarrow^{\rname{assn/query}}
  \config{ c_{3b}', 
  \trace_1 \tracecat [\event_1] \tracecat \trace \tracecat [\event_b] \tracecat  \trace_{3a} \tracecat [\event_2]}
  \rightarrow^{*} 
  \config{c_3, 
  \trace_1 \tracecat [\event_1] \tracecat \trace \tracecat [\event_b] \tracecat  \trace_{3a} \tracecat [\event_2] \tracecat \trace_{3b}}
  % 
\end{array}
\end{equation}
%  %
, where $\trace_3 = \trace_{3a} \tracecat [\event_2] \tracecat \trace_{3b}$,
% $x_1 = \pi_1(\event_1)$, $l_1 = {\pi_2(\event_1)}$, 
$x_2 = \pi_1(\event_2)$, $l_2 = \pi_2(\event_2)$, 
and $\eif ([b]^{l_b}, c_t, c_f) / \ewhile [b]^{l_b} \edo c_w$ 
is the conditional command of the assignment commands associated to the 
$\event_b$ from Inversion Lemma~\ref{lem:inv_event} of testing event.
\\
% \[\begin{array}{l}   
% 	  \config{\eif ([b]^{l_b}, c_t, c_f) / \ewhile [b]^{l_b} \edo c_w;{c}_3', 
% 	  \vtrace_1 \tracecat [\event_1] \tracecat \trace} 
% 	  \\
% 	  \qquad 
% 	   \rightarrow^{\rname{if-b / while-b}} 
% 	  \config{(c_t;c_3' / c_f;c_3') /(c_3' / c_w; \ewhile [b]^{l_b} \edo c_w;{c}_3'), 
% 	  \trace_1 \tracecat [\event_1] \tracecat \trace \tracecat [\event_b]} 
% 	  % 
% 	\end{array}
% 	\]
% $\config{\eif ([b]^{l_b}, c_t, c_f) / \ewhile [b]^{l_b} \edo c_w;{c}_3', 
% \vtrace_1 \tracecat [\event_1] \tracecat \trace} 
%  \rightarrow^{\rname{if-b / while-b}} 
% \config{(c_t;c_3' / c_f;c_3') /(c_3' / c_w; \ewhile [b]^{l_b} \edo c_w;{c}_3'), 
% \trace_1 \tracecat [\event_1] \tracecat \trace \tracecat [\event_b]} 
% $
The notation $(c_t;c_3' / c_f;c_3') /(c_w; \ewhile [b]^{l_b} \edo c_w;c_3' / [\eskip]; c_3')$ represents:
\\
In case of $\eif ([b]^{l_b}, c_t, c_f)$, if $\pi_3(\event_b) = \etrue$, we have the evaluation:
$$
\config{\eif ([b]^{l_b}, c_t, c_f) ;{c}_3', 
\vtrace_1 \tracecat [\event_1] \tracecat \trace} 
 \rightarrow^{\rname{if-b}} 
\config{c_t;c_3' 
\trace_1 \tracecat [\event_1] \tracecat \trace \tracecat [\event_b]} 
$$
%
The same for case of $\eif ([b]^{l_b}, c_t, c_f)$ with $\pi_3(\event_b) = \efalse$,
and case of $\ewhile [b]^{l_b} \edo c_w$ with $\pi_3(\event_b) = \etrue$ and $\pi_3(\event_b) = \efalse$.
%
\\
By the command label consistency,
we also have the instance of second execution as follows:
\begin{equation}
\label{eq:ctldep_inv2}
\begin{array}{l}   
  \config{{c}, \vtrace_0} \rightarrow^{*} 
%   \config{[\assign{{x}_1}{\expr_1 / \query(\qexpr_1)}]^{l_1} ; {c}_1, \vtrace_1}  
%   \rightarrow^{\rname{\rname{assn/query}}}
%    \config{c_1, \vtrace_1 \tracecat [\event_1]} 
%    \\
%     \qquad \rightarrow^{*} 
    \config{\eif ([b]^{l_b}, c_t, c_f) / \ewhile [b]^{l_b} \edo c_w;{c}_3', 
    \vtrace_1 \tracecat [\event_1] \tracecat \trace'} 
    \\
    \qquad 
     \rightarrow^{\rname{if-b / while-b}} 
    \config{(c_f;c_3' / c_t;c_3') /([\eskip]; c_3' / c_w; \ewhile [b]^{l_b} \edo c_w;{c}_3' ), 
    \trace_1 \tracecat [\event_1]  \tracecat \trace' \tracecat [\neg \event_b]} 
    \\
    \qquad   \rightarrow^{*} 
    \config{c_3, 
    \trace_1 \tracecat [\event_1]  \tracecat \trace' \tracecat [\neg \event_b] \tracecat  \trace_3'}
    % 
  \end{array}
\end{equation}
%
By the label consistency, and $\tlabel_{\trace_3} \cap \tlabel_{\trace_3'} = \emptyset$, 
% i.e., 
% $ \trace_{3a} \tracecat [\event_2] \tracecat \trace_{3b} \cap \tlabel_{\trace_3'} = \emptyset$
we know $\trace_3$ and $\trace_3'$ doesn't contain any event of evaluating the commands in $c_3'$.
Otherwise, $\tlabel_{\trace_3} \cap \tlabel_{\trace_3'} \neq \emptyset$, which is a contradiction.
\\
Since $\trace_3= \trace_{3a} \tracecat [\event_2] \tracecat \trace_{3b} $, we know $\event_2$ doesn't comes from evaluating
of $c_3'$, i.e.,:
\\
In the case of $\eif ([b]^{l_b}, c_t, c_f)$, $\event_2$ comes from the evaluation of $c_t$ or $c_f$,
i.e., $[\assign{{x}_2}{\expr_2 / \query(\qexpr_2)}]^{l_2} \in_c c_t$ or $c_f$;
\\
and in the case of $\ewhile [b]^{l_b} \edo c_w$, $\event_2$ comes from the evaluation of $c_w$,
i.e., $[\assign{{x}_2}{\expr_2 / \query(\qexpr_2)}]^{l_2} \in_c c_w$.
\\
In both of the two cases, we know $\forall z \in VAR(\pi_1(\event_b)) $ there is a label $l \in \mathbb{N}$ for this variable,
and by the $\flowsto$ definition, 
$\flowsto(z^l, \pi_1(\event_2)^{\pi_2(\event_2)},c)$.
\\
This lemma is proved.
	\end{proof}
	%
	% \begin{lem}[One Step Dependency Inversion]
	% 	\label{lem:onestepdep_inv}
	% For all $ c \in \cdom, D \in \dbdom, x^i \in \lvar_c$, and $\event_y \in \eventset^{\asn}$, 
	% if $x^i \in VAR(\expr_y)$, 
	% or there exists $\event_b \in \eventset^{\test}$ such that 
	% $x^i \in VAR(\pi_1(\event_b)$ and 
	% $\eventdep^{\ctl}(\event_b, \event_y, c, D)$, then $\flowsto(x^i, \pi_1(\event_y)^{\pi_2(\event_y)}, c)$.
	% %
	% 	\[
	% 	\begin{array}{l}
	% 		\forall c \in \cdom, D \in \dbdom, x^i \in \lvar_c, \event_y \in \eventset^{\asn}
	% 		\st
	% 		\\ \quad
	% 		(x^i \in VAR(\expr_y)\lor 
	% 		(\exists \event_b \in \eventset^{\test} \st x^i \in VAR(\pi_1(\event_b)) 
	% 		\land \eventdep^{\ctl}(\event_b, \event_y, c, D)))
	% 		\implies \flowsto(x^i, \pi_1(\event_y)^{\pi_2(\event_y)}, c)
	% 	\end{array}
	% \]
	% \end{lem}
	% \begin{proof}
	% 	proving by using the Inversion Lemmas~\ref{lem:inv_expr_gnl}, ~\ref{lem:inv_expr},
	% 	\ref{lem:inv_event}, and \ref{lem:inv_live}, 
	% 	and Control Dependency Inversion Lemmas~\ref{lem:ctldep_inv}.
	% \end{proof}
	%
	%
\begin{lem}[The Multiple-Steps Event Dependency Inversion]
	\label{lem:depevents_exist}
For every $D \in \dbdom , c \in \cdom, \trace \in \mathcal{T}$, and two assignment events 
$\event_1, \event_2 \in \eventset^{\asn}$,
if the trace $trace$ has the form $\trace = [\event_1] \tracecat \trace' \tracecat [\event_2]$ with $\trace' \in \mathcal{T}$, and $\eventdep(\event_1, \event_2, \trace, c, D)$
then $\flowsto(\pi_1(\event_1)^{\pi_2(\event_1)}, \pi_1(\event_2)^{\pi_2(\event_2)}, c) $,
or otherwise there exists
$\event \in \trace'$ such that
$\left( 		
   \eventdep(\event_1, \event, \trace[\event_1:\event], c, D)
\land 
\flowsto(\pi_1(\event)^{\pi_2(\event)}, \pi_1(\event_2)^{\pi_2(\event_2)}, c) 
\right)$.
%
	\[
	\begin{array}{l}
		\forall D \in \dbdom , c \in \cdom, \trace \in \mathcal{T} \st \forall \event_1, \event_2 \in \eventset^{\asn} \st
		 \exists \trace' \in \mathcal{T} \st \trace = [\event_1] \tracecat \trace' \tracecat [\event_2]
		\implies
		\eventdep(\event_1, \event_2, \trace, c, D) 
		\\ \quad 
		\implies 
		\flowsto(\pi_1(\event_1)^{\pi_2(\event_1)}, \pi_1(\event_2)^{\pi_2(\event_2)}, c) 
		\\ \qquad \quad \lor
		\exists \event \in \trace' \st 
		\left( 		
			\eventdep(\event_1, \event, \trace[\event_1:\event], c, D)
		\land 
		\flowsto(\pi_1(\event)^{\pi_2(\event)}, \pi_1(\event_2)^{\pi_2(\event_2)}, c) 
	\right) 
		% \\ \qquad \qquad \lor
		% \flowsto(\pi_1(\event_1)^{\pi_2(\event_1)}, \pi_1(\event_2)^{\pi_2(\event_2)}, c) 
	\end{array}
	\]
\end{lem}
Proof Summary: 
\\
Proving by using Lemma~\ref{lem:inv_indepevents}, Lemma~\ref{lem:ctldep_inv}, and the Inversion Lemmas~\ref{lem:inv_expr}, \ref{lem:inv_expr_gnl},
\ref{lem:inv_event}, and \ref{lem:inv_live}
and showing a contradiction.
\begin{proof}
	Taking arbitrary 
	$ D \in \dbdom , c \in \cdom, \trace \in \mathcal{T} $ and two events 
	$\event_1, \event_2 \in \eventset^{\asn}$, where $\trace$ has the form 
	$\trace = [\event_1] \tracecat \trace' \tracecat [\event_2]$ 
	for some 
	$\trace' \in \mathcal{T}$ and $\eventdep(\event_1, \event_2, \trace, c, D)$
	\\ 
	Assume 
	\[
		\begin{array}{l}
	\neg \flowsto(\pi_1(\event_1)^{\pi_2(\event_1)}, \pi_1(\event_2)^{\pi_2(\event_2)}, c) ~ (1)
	\\ \quad 
	\land 
	\forall \event \in \trace' \st 
	\left( 		
		\neg \eventdep(\event_1, \event, \trace[\event_1:\event], c, D)
	\lor 
		\neg \flowsto(\pi_1(\event)^{\pi_2(\event)}, \pi_1(\event_2)^{\pi_2(\event_2)}, c) 
	\right) ~ (2)
	\end{array}
	\]
	Then, by Lemma~\ref{lem:inv_indepevents} and $(2)$, we know 
	$$\flowsto(\pi_1(\event_1)^{\pi_2(\event_1)}, \pi_1(\event_2)^{\pi_2(\event_2)}, c)$$
	, which is contradict to $(1)$.
	\\
	This Lemma is proved.
\end{proof}
%
%
%
%
\begin{lem}[Independent Events Doesn't Block $\flowsto$ ]
		\label{lem:inv_indepevents}
		For every $D \in \dbdom , c \in \cdom, \trace \in \mathcal{T}$, one assignment events 
		$\event_1\in \eventset^{\asn}$, and another event $\event_2 \in \eventset$,
		if the trace $\trace$ has the form $\trace = [\event_1] \tracecat \trace' \tracecat [\event_2]$ with $\trace' \in \mathcal{T}$, 
		and $\eventdep(\event_1, \event_2, \trace, c, D)$,
		then the following two conclusions hold when $\event_2$ is an assignment event and a testing event respectively.
	\begin{itemize}
		\item
		If $\event_2 \in \eventset^{\asn}$,
		% For every $D \in \dbdom , c \in \cdom, \trace \in \mathcal{T}$, and two assignment events 
		% $\event_1, \event_2 \in \eventset^{\asn}$,
		% if the trace $\trace$ has the form $\trace = [\event_1] \tracecat \trace' \tracecat [\event_2]$ with $\trace' \in \mathcal{T}$, and $\eventdep(\event_1, \event_2, \trace, c, D)$,
		then for every $\event \in \trace'$, if it either doesn't have the \emph{Event May-Dependency} relation on $\event_1$, 
		or $\pi_1(\event)^{\pi_2(\event)}$ doesn't have the $\flowsto$ relation with $ \pi_1(\event_2)^{\pi_2(\event_2)}$,
		then the labelled variable $\pi_1(\event_1)^{\pi_2(\event_1)}$ directly flows to the other one $\pi_1(\event_2)^{\pi_2(\event_2)}$, 
		i.e., $\flowsto(\pi_1(\event_1)^{\pi_2(\event_1)}, \pi_1(\event_2)^{\pi_2(\event_2)}, c)$.
		%
		\[
		\begin{array}{l}
			\forall D \in \dbdom , c \in \cdom, \trace \in \mathcal{T} \st \forall \event_1, \event_2 \in \eventset^{\asn} \st
			 \exists \trace' \in \mathcal{T} \st \trace = [\event_1] \tracecat \trace' \tracecat [\event_2]
			\implies
			\eventdep(\event_1, \event_2, \trace, c, D) 
			\\ \quad 
			\implies 
			\left( \forall \event \in \trace' \st \neg \eventdep(\event_1, \event, \trace[\event_1:\event], c, D)
			\lor \neg \flowsto(\pi_1(\event)^{\pi_2(\event)}, \pi_1(\event_2)^{\pi_2(\event_2)}, c) 
			\right) 
			\\ \quad 
			\implies 
			\flowsto(\pi_1(\event_1)^{\pi_2(\event_1)}, \pi_1(\event_2)^{\pi_2(\event_2)}, c)
		\end{array}
		\]
		\item
If $\event_2 \in \eventset^{\test}$, 
then for every $\event \in \trace'$, if it either doesn't have the \emph{Event May-Dependency} relations on $\event_1$,
or $\pi_1(\event) \notin VAR(\pi_1(\event_2)) $,
then 
$\pi_1(\event_1) \in VAR(\pi_1(\event_2))$, and $ {\pi_2(\event_1)} = \llabel(\trace)$
%
\[
\begin{array}{l}
	\forall D \in \dbdom , c \in \cdom, \trace \in \mathcal{T} \st \forall \event_1,\in \eventset^{\asn}, \event_2 \in \eventset^{\test} \st
	 \exists \trace' \in \mathcal{T} \st \trace = [\event_1] \tracecat \trace' \tracecat [\event_2]
	\implies
	\eventdep(\event_1, \event_2, \trace, c, D) 
	\\ \quad 
	\implies 
	\left( \forall \event \in \trace' \st 
	\neg \eventdep(\event_1, \event, \trace[\event_1:\event], c, D)
	\lor  \pi_1(\event) \notin VAR(\pi_1(\event_2))
	\right) 
	\\ \quad 
	\implies 
	\pi_1(\event_1) \in VAR(\pi_1(\event_2)) \land {\pi_2(\event_1)} = \llabel(\trace)
\end{array}
\]
\end{itemize}
\end{lem}
%
\begin{proof}
Taking arbitrary $D \in \dbdom , c \in \cdom$, and an assignment events $\event_1 \in \eventset^{\asn}$ and another event 
$\event_2\in \eventset$.
\\
Without loss of generalization, 
taking arbitrary trace has the form $\trace = [\event_1; \cdots; \event_2]$ for arbitrary $\trace_2 \in \mathcal{T}$,
 then we know $\exists \trace' \in \mathcal{T} \st \trace = [\event_1] \tracecat \trace' \tracecat [\event_2]$, let $\trace_2$ be this $\trace'$.
%
\caseL{$\event_2 \in \eventset^{\asn}$}
%
By the definition of $\eventdep(\event_1, \event_2, \trace, c, D)$, 
taking $ \event_1', \event_2' \in \eventset^{\asn},
\trace_2' \in \mathcal{T}, c_1, c_2 \in \cdom$ as the events, traces and commands satisfying the definition,
 we have following two executions:
% \[
%   \exists \event_1', \event_2' \in \eventset^{\asn},
%   \trace_2' \in \mathcal{T}, c_1, c_2 \in \cdom \st
% \]
%
\[
\begin{array}{l}
\config{c, \trace_0} \rightarrow^{*}
\config{c_1, \trace_1 \tracecat [\event_1]} \rightarrow^{*} \config{c_2, \trace_1 \tracecat [\event_1] \tracecat \trace_2 \tracecat [\event_2]} 
\\ \quad
% \land
\config{c_1, \trace_1 \tracecat [\event_1']} \rightarrow^{*} \config{c_2, \trace_1 \tracecat [\event_1'] \tracecat \trace_2' \tracecat [\event_2']} 
\end{array}
\]
%
%
By inversion Lemma.~\ref{lem:inv_event} on $\event_2$ and $\event_2'$ in the two executions
and $\diff(\event_2, \event_2)$,
 we have the following two execution instances:
\[
\config{c_1, \trace_1 \tracecat [\event_1]} \rightarrow^{*} \config{[\assign{\pi_1(\event_2)}{\expr_2 / \query(\qexpr_2)}]{}^{\pi_2(\event_2)};c_2, \trace_1 \tracecat [\event_1] \tracecat \trace_2} 
\rightarrow^\rname{asn / query} \config{c_2, \trace_1 \tracecat [\event_1] \tracecat \trace_2 \tracecat [\event_2]}  
\]
%
\[
\config{c_1, \trace_1 \tracecat [\event_1']} \rightarrow^{*} \config{[\assign{\pi_1(\event_2)}{\expr_2 / \query(\qexpr_2)}]{}^{\pi_2(\event_2)};c_2, \trace_1 \tracecat [\event_1'] \tracecat \trace_2'} 
\rightarrow^\rname{asn / query} \config{c_2, \trace_1 \tracecat [\event_1'] \tracecat \trace_2' \tracecat [\event_2']}  
\]
, where $\expr_2 / \qexpr_2$ is the expression of the assignment command associated to the $\event_2$ and $\event_2'$ by the Inversion Lemma.~\ref{lem:inv_event}.
\\
Taking arbitrary $\event_z \in \trace_2$, we know 
$\neg \eventdep(\event_1, \event, \trace[\event_1:\event_z], c, D)
\lor  \pi_1(\event_z) \notin VAR(\expr_2 / \qexpr_2)$.
\\
In case of $\neg \eventdep(\event_1, \event, \trace[\event_1:\event_z], c, D)$,
by Definition~\ref{def:event_dep}, we know $\event_z \in \trace_2'$ and 
\[
	\env(\trace_1 \tracecat \trace[\event_1:\event_z]) \pi_1(\event_z) = \env(\trace_1 \tracecat \trace[\event_1':\event_z]) \pi_1(\event_z)
	\]
%
In case of $ \pi_1(\event_z) \notin VAR(\expr_2 / \qexpr_2)$, by Inversion Lemma~\ref{lem:inv_expr}
 of arithmetic and query expression cases, we know:
%
\[
	\forall x^i \in \lvar, \trace, \trace' \in \mathcal{T}, v, v' \st
	\Big( \forall z^j \in \lvar / \{\pi_1(\event_z)^{\pi_2(\event_z)} \} \st 
	\env(\trace) z = \env(\trace') z \Big) \land 
	\config{\trace, \expr_2 / \qexpr_2} \aarrow v \land \config{\trace', \expr_2} \aarrow  v' \implies v = v'
	\]
	\[
		\forall x^i \in \lvar, \trace, \trace' \in \mathcal{T}, \qval, \qval' \st
		\Big( \forall z^j \in \lvar / \{\pi_1(\event_z)^{\pi_2(\event_z)} \} \st 
		\env(\trace) z = \env(\trace') z \Big) \land 
		\config{\trace, \qexpr_2} \qarrow \qval \land \config{\trace', \qexpr_2} \qarrow \qval' \implies \qval =_{q} \qval'
		\]
for $\expr_2$ or $\qexpr_2$ respectively.
\\
Let $\kw{use}_{\trace_2}$ a subset of the events in $\trace_2$, satisfying: 
\[
	\begin{array}{l}
		\forall \event \in \eventset^{\asn} \st 
	\event \in \kw{use}_{\trace_2} \Longleftrightarrow 
	\event \in \trace_2 \land
	\pi_1(\event) \in VAR(\expr_2 / \qexpr_2)
\end{array}		
\]
Then we also know for every $\event_z \in \kw{use}_{\trace_2}$, 
$\neg \eventdep(\event_1, \event_z, \trace[\event_1:\event_z], c, D)$, i.e.,:
\[
	\forall z^l \in \lvar \setminus 
	\big( 
		( \lvar_{\trace_2} \setminus \lvar_{\kw{use}_{\trace_2}}) \cup \{\pi_1(\event_1)^{\pi_2(\event_1)}\} \big)
	\st
	\env(\trace_1 \tracecat [\event_1] \tracecat \trace_2) z = \env(\trace_1 \tracecat [\event_1'] \tracecat \trace_2') z
	~ (1)
\]
 and
 \\ 
$
	\forall z^l \in \lvar \setminus ( \lvar_{\trace_2} \setminus \lvar_{\kw{use}_{\trace_2}}), 
	\trace, \trace' \in \mathcal{T}, v, v' \st 
	\env(\trace) z = \env(\trace') z 
	\land 
	\config{\trace, \expr_2} \aarrow v 
	\land 
	\config{\trace', \expr_2} \aarrow v'
	\implies 
	v = v' 
	~ (2a)
$;
\\
$
	\forall z^l \in \lvar \setminus ( \lvar_{\trace_2} \setminus \lvar_{\kw{use}_{\trace_2}}), 
	\trace, \trace' \in \mathcal{T}, \qval, \qval' \st 
	\env(\trace) z = \env(\trace') z 
	\land 
	\config{\trace, \qexpr_2} \qarrow \qval 
	\land 
	\config{\trace', \qexpr_2} \qarrow \qval'
	\implies 
	\qval =_q \qval' 
	~ (2q)
$,
\\
where $\lvar_{\trace_2}$ and $\lvar_{\kw{use}_{\trace_2}}$ are
 the sets of labelled variables of every event in $\trace_2$ and $\kw{use}_{\trace_2}$ respectively .
% \\
% and 
% $
% 	\forall\event_z \in \lvar/\kw{diff}_{\eventset}, 
% 	\trace, \trace' \in \mathcal{T}, v, v' \st 
% 	\env(\trace) \pi_1(\event_z) = \env(\trace') \pi_1(\event_z) 
% 	\land 
% 	\config{\trace, \expr_2} \aarrow v 
% 	\land 
% 	\config{\trace', \expr_2} \aarrow v'
% 	\implies 
% 	v = v' 
% $
% \\
% By the label consistency, we know 
% \[
% 	\forall \event_z \in \kw{use}_{\trace_2} \st
% 	\env(\trace_1 \tracecat [\event_1] \tracecat \trace_2) \pi_1(\event_z) 
% 	= \env(\trace_1 \tracecat [\event_1'] \tracecat \trace_2']) \pi_1(\event_z)
% \]
% \[
% 	\forall\event_z \in \trace_2 \setminus \kw{use}_{\eventset}, 
% 	\trace, \trace' \in \mathcal{T}, v, v' \st 
% 	% \env(\trace) \pi_1(\event_z) = \env(\trace') \pi_1(\event_z) 
% 	% \land 
% 	\config{\trace, \expr_2} \aarrow v 
% 	\land 
% 	\config{\trace', \expr_2} \aarrow v'
% 	\implies 
% 	v = v' 
% \]
% Let $\diff_{\eventset}$ be a subset of the events in $\trace_2$, satisfying: 
% % \todo{refine the notation}
% \[
% 	\begin{array}{l}
% 		\forall \event_z \in \eventset^{\asn} \st 
% 	\event_z \in \diff_{\eventset} \Leftrightarrow 
% 	\exists \trace_2^h, \trace'^h_2, \trace_2^t, \trace'^t_2, \event_z' \in \trace_2' \st 
% 	\trace_2 = \trace_2^h \tracecat [\event_z] \tracecat \trace_2^t
% 	\\ \quad
% 	\land 
% 	\trace_2' = \trace'^h_2 \tracecat [\event_z'] \tracecat \trace'^t_2
% 	\land 
% 	\diff(\event_z, \event_z')
% 	\land 
% 	\vcounter(\trace_2^h) \pi_1(\event_z) = \vcounter(\trace'^h_2)(\event_z)
% \end{array}		
% \]
%
% Then we know for all $\event_z \in  \diff_{\eventset}$,
%  $\eventdep(\event_1, \event_z, \trace[\event_1:\event_z], c, D)$;
% \\
% and $\forall z^j \in (\lvar \setminus (\mathbb{LV}_{\diff_{\eventset}} \cup\{\pi_1(\event_1)^{\pi_2(\event_2)}\}) ) \st 
% \env(\trace_1 \tracecat [\event_1] \tracecat \trace_2) z = \env(\trace_1 \tracecat [\event_1'] \tracecat \trace_2') z $,
% \\
% where $\mathbb{LV}_{\diff_{\eventset}}$ is the set of labelled variables of every event in $\diff_{\eventset}$.
% We also know for all $\event_z \in  \diff_{\eventset}$, $\pi_1(\event_z) \notin VAR(\pi_1(\event_2))$ and follows:
% $\neg \eventdep(\event_z, \event_2, \trace[\event_z:\event_2], c, D) $.
% \\
% Then by $\eventdep$ definition, we know 
% $\forall \event_z', \event_2'' \in \eventset^{\asn}, \trace_z',\trace_z,  \trace_1' \in \mathcal{T}, c_z \in \cdom$ satisfying following two executions, we have $\event_2 \eventeq \event_2''$.
% \[
% 	\begin{array}{l}
% 		\config{c, \trace_0} \rightarrow^{*}
% 		\config{c_z, \trace_1' \tracecat [\event_z]} \rightarrow^{*} \config{c_2, \trace_1' \tracecat [\event_z] \tracecat \trace_z \tracecat [\event_2]} 
% 		\\ \quad
% 		\land
% 		\config{c_z, \trace_1' \tracecat [\event_z']} \rightarrow^{*} \config{c_2, \trace_1' \tracecat [\event_z'] \tracecat \trace_z' \tracecat [\event_2']'} 
% 		\end{array}		
% \]
%
% Then we know 
% \todo{type correctness}
% \[
% 	\forall z^j \in (\lvar \setminus \mathbb{LV}_{\diff_{\eventset}} ), \trace, \trace' \in \mathcal{T}, v, v' \st 
% 	\env(\trace) z = \env(\trace') z 
% 	\land 
% 	\config{\trace, \expr_2} \aarrow v 
% 	\land 
% 	\config{\trace', \expr_2} \aarrow v'
% 	\land 
% 	v = v' 
% \]
%
Since $\diff(\event_2, \event_2')$, we also know:
%
\[	
\config{\trace_1 \tracecat [\event_1] \tracecat \trace_2, \expr_2} \aarrow \pi_3(\event_2)
\land 
\config{\trace_1 \tracecat [\event_1'] \tracecat \trace_2', \expr_2} \aarrow \pi_3(\event_2') 
\land 
\pi_3(\event_2) \neq \pi_3(\event_2')
\]
%
% By construction of $\diff_{\eventset}$, we also have:
% \\
% $\forall z^j \in (\lvar \setminus \diff_{\eventset} \cup\{\event_1\} ) \st 
% \env(\trace_1 \tracecat [\event_1] \tracecat \trace_2) z = \env(\trace_1 \tracecat [\event_1'] \tracecat \trace_2') z $.
% \\
% By $\forall \event \in (\trace[\event_1:\event_y]\setminus \{\event_1, \event_y\}) \st
%   \neg \eventdep (\event_1, \event, c, D) \lor \neg \eventdep (\event, \event_y, c, D)$,
We know $\event_1$ is the only cause of the difference in $\event_y$ and $\event_y'$ when evaluating 
$[\assign{\pi_1(\event_2)}{\expr_2 / \query(\qexpr_2)}]{}^{\pi_2(\event_2)}$.
%
\\
By inversion Lemma.~\ref{lem:inv_expr_gnl} of arithmetic and query expression cases, 
given the two traces
$\trace_1 \tracecat [\event_1'] \tracecat \trace_2'$ and 
$\trace_1 \tracecat [\event_1'] \tracecat \trace_2'$ 
satisfying this lemma by $(1)$, $(2a)$ and $(2q)$, 
we know
\[
  \pi_1(\event_1) \in VAR(\expr_2 / \qexpr_2) \land {\pi_2(\event_1)} = \llabel(\trace_1 \tracecat [\event_1] \tracecat \trace_2) \pi_1(\event_1) 
\]
%
By $\flowsto$ definition:
% \todo{add the Liveness inversion}
\[
  \flowsto(\pi_1(\event_1)^{\pi_2(\event_1)}, \pi_1(\event_y)^{\pi_2(\event_y)}, c)
\]
This case is proved.
%
\caseL{$\event_2 \in \eventset^{\test}$}
\\
By the definition of $\eventdep(\event_1, \event_2, \trace, c, D)$, 
taking $ \event_1' \in \eventset^{\asn},
\trace_2' \in \mathcal{T}, c_1, c_2 \in \cdom$ and $\event_2' \in \eventset^{\test}$
 as the events, traces and commands satisfying the definition,
 we have following two executions:
% \[
%   \exists \event_1', \event_2' \in \eventset^{\asn},
%   \trace_2' \in \mathcal{T}, c_1, c_2 \in \cdom \st
% \]
%
\[
\begin{array}{l}
\config{c, \trace_0} \rightarrow^{*}
\config{c_1, \trace_1 \tracecat [\event_1]} \rightarrow^{*} \config{c_2, \trace_1 \tracecat [\event_1] \tracecat \trace_2 \tracecat [\event_2]} 
\\ \quad
% \land
\config{c_1, \trace_1 \tracecat [\event_1']} \rightarrow^{*} \config{c_2, \trace_1 \tracecat [\event_1'] \tracecat \trace_2' \tracecat [\event_2']} 
\end{array}
\]
%
Taking arbitrary $\event_z \in \trace_2$, we know 
$\neg \eventdep(\event_1, \event, \trace[\event_1:\event_z], c, D)
\lor  \pi_1(\event_z) \notin VAR(\expr_2 / \qexpr_2)$.
\\
Then by the same proof in \textbf{case: $\event_2 \in \eventset^{\asn}$}, and applying the Inversion Lemma~\ref{lem:inv_expr} and \ref{lem:inv_expr_gnl} of the boolean expression case,
we have:
\[ 
	\pi_1(\event_1) \in VAR(\pi_1(\event_2)) \land {\pi_2(\event_1)} = \llabel(\trace)
	\]
	This case is proved.
\end{proof}
%
% \begin{lem}[Arithmetic Inversion]
% \label{lem:inv_a}
% For all {$ x^i \in \lvar$, and $\trace, \trace' \in \mathcal{T}$,  and arithmetic expression $\aexpr$}, if
% $ \forall z^j \in \lvar / \{x^i\} \st 
% \env(\trace) z = \env(\trace') z $ and $\config{\trace, \aexpr} \aarrow v $ and 
% $\config{\trace', \aexpr} \aarrow v' $ with $v' \neq v$, then $ x $ is in the free variables of $\aexpr$, i.e., $x \in VAR(\aexpr)$.
% %
% % \[
% 	% \forall x^i \in \lvar, \trace, \trace' \in \mathcal{T}, \aexpr \st
% % 	\Big( \forall z^j \in \lvar / \{x^i\} \st 
% % 	\env(\trace) z = \env(\trace') z\Big) \land 
% % 	\config{\trace, \expr} \aarrow v \land \config{\trace', \expr} \aarrow v' \land v \neq v'
% % 	\implies x \in VAR(\expr)
% % \]
% \end{lem}
% %
% %
% \begin{lem}[Boolean Inversion]
% \label{lem:inv_b}
% For all {$ x^i \in \lvar$, and $\trace, \trace' \in \mathcal{T}$,  and boolean expression $\bexpr$}, if
% $ \forall z^j \in \lvar / \{x^i\} \st 
% \env(\trace) z = \env(\trace') z $ and $\config{\trace, \bexpr} \barrow v $ and 
% $\config{\trace', \bexpr} \barrow v' $ with $v' \neq v$, then $ x $ is in the free variables of $\bexpr$, i.e., $x \in VAR(\bexpr)$.
% % \[
% % 	text{(\forall x^i \in \lvar, \trace, \trace' \in \mathcal{T}, \bexpr \st )}
% % 	\Big(\forall z^j \in \lvar / \{x^i\} \st
% % 	\env(\trace) z = \env(\trace') z\Big) \land
% % 	\config{\trace, \bexpr} \barrow v \land \config{\trace', \bexpr} \barrow v' \land v \neq v'
% % 	\implies x \in VAR(\bexpr) 
% % \]
% \end{lem}
% %
% \begin{lem}[Query Inversion]
% \label{lem:inv_q}
% For all {$ x^i \in \lvar$, and $\trace, \trace' \in \mathcal{T}$,  and query expression $\qexpr$}, if
% $ \forall z^j \in \lvar / \{x^i\} \st 
% \env(\trace) z = \env(\trace') z $ and $\config{\trace, \qexpr} \qarrow \qval $ and 
% $\config{\trace', \qexpr} \qarrow \qval' $ with $\qval \neq_{q} \qval'$, then $ x $ is in the free variables of $\qexpr$, i.e., $x \in VAR(\qexpr)$.
% % \[
% % 	\forall x^i \in \lvar, \trace, \trace' \in \mathcal{T}, \qexpr \st 
% % 	\left(\forall z^j \in \lvar / \{x^i \} \st
% % 	\env(\trace) z = \env(\trace') z \right) \implies 
% % 	\config{\trace, \qexpr} \qarrow \qval \land \config{\trace', \qexpr} \qarrow \qval' 
% % 	\implies \qval \neq_{q} \qval'
% % 	\implies x \in VAR(\qexpr) 
% % \]
% \end{lem}
%
\begin{lem}[Expression Inversion]
	\label{lem:inv_expr}
	For all {$ x^i \in \lvar$, and $\trace, \trace' \in \mathcal{T}$, and an expression $\expr$} if
	$ \forall z^j \in \lvar / \{x^i\} \st 
	\env(\trace) z = \env(\trace') z $, and if
	\begin{itemize}
		\item $\expr$ is an arithmetic expression $\aexpr$,
		% \\ 
		and $\config{\trace, \aexpr} \aarrow v $ and 
	$\config{\trace', \aexpr} \aarrow v' $ with $v' \neq v$, 
	then $ x $ is in the free variables of $\aexpr$ and $i$ is the latest label for $x$ 
    in $\trace$, i.e., $x \in VAR(\aexpr)$ and $i = \llabel(\trace) x$.
%
	\item $\expr$ is a boolean expression $\bexpr$,
	% \\
  and $\config{\trace, \bexpr} \barrow v $ and 
 $\config{\trace', \bexpr} \barrow v' $ with $v' \neq v$, then $ x $ is in the free variables of $\bexpr$ and $i$ is the latest label for $x$ 
 in $\trace$, i.e., $x \in VAR(\bexpr)$ and $i = \llabel(\trace) x$.
% 
	\item $\expr$ is a query expression $\qexpr$,
	% \\
	and $\config{\trace, \qexpr} \qarrow \qval $ and 
	$\config{\trace', \qexpr} \qarrow \qval' $ with $\qval \neq_{q} \qval'$, then $ x $ is in the free variables of $\qexpr$ and $i$ is the latest label for $x$ 
    in $\trace$, i.e., $x \in VAR(\qexpr)$ and $i = \llabel(\trace) x$.
\end{itemize}	%
	\end{lem}
    Proof Summary:
    \\
    To show $x \in VAR(\aexpr)$, by showing contradiction ($\forall \trace, \trace'$ in second hypothesis  $v = v'$)
     if $x \notin VAR(\aexpr)$.
     \\
    To show $i = \llabel(\trace)$, by showing contradiction ($\forall \trace, \trace'$ in second hypothesis  $v = v'$ ) 
    if $j = \llabel(\trace) x$ and $i \neq j$.
    \begin{proof}
		Take two arbitrary traces $\trace, \trace' \in \mathcal{T}$, and an expression $\expr$ satisfying
		$ \forall z^j \in \lvar / \{x^i\} \st 
		\env(\trace) z = \env(\trace') z $, we have the following three cases.
    \caseL{$\expr$ is an arithmetic expression $\aexpr$}
	We have $\config{\trace, \bexpr} \barrow v $ and 
	$\config{\trace', \bexpr} \barrow v' $ with $v' \neq v$ from the lemma hypothesis.
	\\
	To show $x \in VAR(\qexpr)$ and $i = \llabel(\trace) x$: 
	\\
	Assuming $x \notin VAR(\aexpr)$,
	% by $\config{\trace, \aexpr} \aarrow v $	and $\config{\trace', \aexpr} \aarrow v'$, 
	since
	%  $x \notin VAR(\aexpr)$ and 
	$ \forall z^j \in \lvar / \{x^i\} \st 
		\env(\trace) z = \env(\trace') z $,
	we know $v = v'$, which is contradicted to $v' \neq v$.
	\\
	Then we know $x \in VAR(\qexpr)$.
	\\
	Assuming $j = \llabel(\trace) x \land i \neq j$,
	by 
	% $\config{\trace, \aexpr} \aarrow v $	and $\config{\trace', \aexpr} \aarrow v'$, 
	% since $x \notin VAR(\aexpr)$ and 
	$ \forall z^j \in \lvar / \{x^i\} \st 
		\env(\trace) z = \env(\trace') z $, we know 
		$\env(\trace) x = \env(\trace') x$, i.e., 
	\\
	$\forall z^j \in \lvar \st \env(\trace) z = \env(\trace') z$.
	\\
	Then by the determination of the evaluation, 
	% and 
	% $\config{\trace, \aexpr} \aarrow v $ and $\config{\trace', \aexpr} \aarrow v'$, 
	we know $v = v'$, which is contradicted to $v' \neq v$.
	\\
	Then we know $i = \llabel(\trace) x$.

    \caseL{$\expr$ is a boolean expression $\bexpr$}
	This case is proved trivially in the same way as the case of the arithmetic expression.
	\caseL{$\expr$ is a query expression $\qexpr$}
	This case is proved trivially in the same way as the case of the arithmetic expression.
\end{proof}
	%
	%
	% \begin{lem}[Boolean Inversion]
	% \label{lem:inv_b}
	% For all {$ x^i \in \lvar$, and $\trace, \trace' \in \mathcal{T}$,  and boolean expression $\bexpr$}, if
	% $ \forall z^j \in \lvar / \{x^i\} \st 
	% \env(\trace) z = \env(\trace') z $ and $\config{\trace, \bexpr} \barrow v $ and 
	% $\config{\trace', \bexpr} \barrow v' $ with $v' \neq v$, then $ x $ is in the free variables of $\bexpr$, i.e., $x \in VAR(\bexpr)$.
	% % \[
	% % 	text{(\forall x^i \in \lvar, \trace, \trace' \in \mathcal{T}, \bexpr \st )}
	% % 	\Big(\forall z^j \in \lvar / \{x^i\} \st
	% % 	\env(\trace) z = \env(\trace') z\Big) \land
	% % 	\config{\trace, \bexpr} \barrow v \land \config{\trace', \bexpr} \barrow v' \land v \neq v'
	% % 	\implies x \in VAR(\bexpr) 
	% % \]
	% \end{lem}
	% %
	% \begin{lem}[Query Inversion]
	% \label{lem:inv_q}
	% For all {$ x^i \in \lvar$, and $\trace, \trace' \in \mathcal{T}$,  and query expression $\qexpr$}, if
	% $ \forall z^j \in \lvar / \{x^i\} \st 
	% \env(\trace) z = \env(\trace') z $ and $\config{\trace, \qexpr} \qarrow \qval $ and 
	% $\config{\trace', \qexpr} \qarrow \qval' $ with $\qval \neq_{q} \qval'$, then $ x $ is in the free variables of $\qexpr$, i.e., $x \in VAR(\qexpr)$.
	% % \[
	% % 	\forall x^i \in \lvar, \trace, \trace' \in \mathcal{T}, \qexpr \st 
	% % 	\left(\forall z^j \in \lvar / \{x^i \} \st
	% % 	\env(\trace) z = \env(\trace') z \right) \implies 
	% % 	\config{\trace, \qexpr} \qarrow \qval \land \config{\trace', \qexpr} \qarrow \qval' 
	% % 	\implies \qval \neq_{q} \qval'
	% % 	\implies x \in VAR(\qexpr) 
	% % \]
	% \end{lem}
%
% \begin{lem}[Arithmetic Inversion Generalization]
% 	\label{lem:inv_a_gnl}
% 	% For all {$ x^i \in \lvar$, and $\trace, \trace' \in \mathcal{T}$,  and arithmetic expression $\aexpr$}, if
% 	% $ \forall z^j \in \lvar / \{x^i\} \st 
% 	% \env(\trace) z = \env(\trace') z $ and $\config{\trace, \aexpr} \aarrow v $ and 
% 	% $\config{\trace', \aexpr} \aarrow v' $ with $v' \neq v$, then $ x $ is in the free variables of $\aexpr$, i.e., $x \in VAR(\aexpr)$.
% 	%
% 	For all subset of the labelled variables $\diff \subset \lvar$, and $x^i \in (\lvar \setminus \diff)$,
% 	and an arithmetic expression $\aexpr$,
% 	if for all $z^j \in \lvar \setminus \diff, \trace, \trace' \in \mathcal{T}, v, v'$ such that 
% 	$\env(\trace) z = \env(\trace') z$, and 
% 	$
% 	\config{\trace, \aexpr} \aarrow v$, and $\config{\trace', \aexpr} \aarrow v'$ with $v = v'$;
% 	and for all $z^j \in \lvar / (\diff \cup \{x^i\} )$ 
% 	there exist $\trace, \trace' \in \mathcal{T}, v, v'$ such that 
% 	$\env(\trace) z = \env(\trace') z$, and 
% 	$
% 	\config{\trace, \aexpr} \aarrow v$, and $\config{\trace', \aexpr} \aarrow v'$ with $v \neq v'$,
% 	then $x \in VAR(\aexpr)$ and $i = \llabel(\trace) x$.
% 	\[
% 		\begin{array}{l}
% 		\forall \diff \subset \lvar,  x^i \in (\lvar \setminus \diff), \aexpr \st
% 		\\ \quad
% 		\forall z^j \in \lvar \setminus \diff, \trace, \trace' \in \mathcal{T}, v, v' \st 
% 		\env(\trace) z = \env(\trace') z \land 
% 		\config{\trace, \aexpr} \aarrow v \land \config{\trace', \aexpr} \aarrow v' \land v = v'
% 		\\ \quad
% 		\implies 
% 		\forall z^j \in \lvar / (\diff \cup \{x^i\} ) \st 
% 		 \exists \trace, \trace' \in \mathcal{T}, v, v'\st 
% 		\env(\trace) z = \env(\trace') z \land 
% 		\config{\trace, \aexpr} \aarrow v \land \config{\trace', \aexpr} \aarrow v' \land v \neq v'
% 		\\ \qquad
% 		\implies x \in VAR(\aexpr) \land i = \llabel(\trace) x
% 		\end{array}
% 	\]
% 	\end{lem}
% \begin{proof}.
% 	\\
% 	To show $x \in VAR(\aexpr)$, by showing contradiction ($\forall \trace, \trace'$ in second hypothesis  $v = v'$)
% 	 if $x \notin VAR(\aexpr)$.
% 	 \\
% 	To show $i = \llabel(\trace)$, by showing contradiction ($\forall \trace, \trace'$ in second hypothesis  $v = v'$ ) 
% 	if $j = \llabel(\trace) x$ and $i \neq j$.
% \end{proof}
% 	%
% \begin{lem}[Boolean Inversion Generalization]
% 		\label{lem:inv_b_gnl}
% 		% For all {$ x^i \in \lvar$, and $\trace, \trace' \in \mathcal{T}$,  and arithmetic expression $\aexpr$}, if
% 		% $ \forall z^j \in \lvar / \{x^i\} \st 
% 		% \env(\trace) z = \env(\trace') z $ and $\config{\trace, \aexpr} \aarrow v $ and 
% 		% $\config{\trace', \aexpr} \aarrow v' $ with $v' \neq v$, then $ x $ is in the free variables of $\aexpr$, i.e., $x \in VAR(\aexpr)$.
% 		%
% 		\[
% 			\begin{array}{l}
% 			\forall \diff \subset \lvar,  x^i \in (\lvar \setminus \diff), \bexpr \st
% 			\\ \quad
% 			\forall z^j \in \lvar \setminus \diff, \trace, \trace' \in \mathcal{T}, v, v' \st 
% 			\env(\trace) z = \env(\trace') z \land 
% 			\config{\trace, \bexpr} \barrow v \land \config{\trace', \bexpr} \barrow v' \land v = v'
% 			\\ \quad
% 			\implies 
% 			\forall z^j \in \lvar / (\diff \cup \{x^i\} ) \st 
% 			 \exists \trace, \trace' \in \mathcal{T}, v, v'\st 
% 			\env(\trace) z = \env(\trace') z \land 
% 			\config{\trace, \bexpr} \barrow v \land \config{\trace', \bexpr} \barrow v' \land v \neq v'
% 			\\ \qquad
% 			\implies x \in VAR(\bexpr) \land i = \llabel(\trace)
% 			\end{array}
% 		\]
% \end{lem}%
\begin{lem}[Expression Inversion Generalization]
	\label{lem:inv_expr_gnl}
	For all subset of the labelled variables $\lvar_{\diff} \subset \lvar$
	and an expression $\expr$, if 
	\begin{itemize}
		\item $\expr$ is an arithmetic expression $\aexpr$,
		% \\ 
		and for all $z^j \in \lvar \setminus \lvar_{\diff}$ 
		there exist $\trace, \trace' \in \mathcal{T}, v, v'$ such that 
		$\env(\trace) z = \env(\trace') z$,
		$\config{\trace, \aexpr} \aarrow v$ and 
		$\config{\trace', \aexpr} \aarrow v'$ with $v \neq v'$,
		% and for all $z^j \in \lvar / (\diff \cup \{x^i\} )$ 
		% there exist $\trace, \trace' \in \mathcal{T}, v, v'$ such that 
		% $\env(\trace) z = \env(\trace') z$, and 
		% $
		% \config{\trace, \aexpr} \aarrow v$, and $\config{\trace', \aexpr} \aarrow v'$ with $v \neq v'$,
		then $\exists x^i \in \lvar_{\diff} $ 
		such that $x \in FV(\aexpr)$ and $i = \llabel(\trace) x$.
		\[
			\begin{array}{l}
			\forall \lvar_{\diff}  \subset \lvar,  
			% x^i \in (\lvar \setminus \diff), 
			\aexpr \st
			\\ \quad
			\forall z^j \in \lvar \setminus \lvar_{\diff} \st 
			\exists \trace, \trace' \in \mathcal{T}, v, v' \st 
			\env(\trace) z = \env(\trace') z 
			\land 
			\config{\trace, \aexpr} \aarrow v 
			\land \config{\trace', \aexpr} \aarrow v' \land v \neq v'
			\\ \quad
			% \implies 
			% \forall z^j \in \lvar / (\diff \cup \{x^i\} ) \st 
			% \exists \trace, \trace' \in \mathcal{T}, v, v'\st 
			% \env(\trace) z = \env(\trace') z \land 
			% \config{\trace, \aexpr} \aarrow v \land \config{\trace', \aexpr} \aarrow v' \land v \neq v'
			% \\ \qquad
			\implies \exists x^i \in \lvar_{\diff} \st 
			x \in FV(\aexpr) \land i = \llabel(\trace) x
			\end{array}
		\]
		\todo{rewrite}
	\item $\expr$ is a boolean expression $\bexpr$,
	and for all $z^j \in \lvar \setminus \lvar_{\diff}$ 
	there exist $\trace, \trace' \in \mathcal{T}, v, v'$ such that 
	$\env(\trace) z = \env(\trace') z$,
	$\config{\trace, \bexpr} \barrow v$ and 
	$\config{\trace', \bexpr} \barrow v'$ with $v \neq v'$,
	% and for all $z^j \in \lvar / (\diff \cup \{x^i\} )$ 
	% there exist $\trace, \trace' \in \mathcal{T}, v, v'$ such that 
	% $\env(\trace) z = \env(\trace') z$, and 
	% $
	% \config{\trace, \aexpr} \aarrow v$, and $\config{\trace', \aexpr} \aarrow v'$ with $v \neq v'$,
	then $\exists x^i \in \lvar_{\diff} $ 
	such that $x \in FV(\bexpr)$ and $i = \llabel(\trace) x$.
	\[
		\begin{array}{l}
		\forall \lvar_{\diff}  \subset \lvar,  
		% x^i \in (\lvar \setminus \diff), 
		\aexpr \st
		\\ \quad
		\forall z^j \in \lvar \setminus \lvar_{\diff} \st 
		\exists \trace, \trace' \in \mathcal{T}, v, v' \st 
		\env(\trace) z = \env(\trace') z 
		\land 
		\config{\trace, \bexpr} \barrow v 
		\land \config{\trace', \bexpr} \barrow v' \land v \neq v'
		\\ \quad
		% \implies 
		% \forall z^j \in \lvar / (\diff \cup \{x^i\} ) \st 
		% \exists \trace, \trace' \in \mathcal{T}, v, v'\st 
		% \env(\trace) z = \env(\trace') z \land 
		% \config{\trace, \aexpr} \aarrow v \land \config{\trace', \aexpr} \aarrow v' \land v \neq v'
		% \\ \qquad
		\implies \exists x^i \in \lvar_{\diff} \st 
		x \in FV(\bexpr) \land i = \llabel(\trace) x
		\end{array}
	\]
% 
	\item $\expr$ is a query expression $\qexpr$,
	and for all $z^j \in \lvar \setminus \lvar_{\diff}$ 
	there exist $\trace, \trace' \in \mathcal{T}, \qval, \qval'$ such that 
	$\env(\trace) z = \env(\trace') z$,
	$\config{\trace, \qexpr} \qarrow v$ and 
	$\config{\trace', \qexpr} \qarrow v'$ with $\qval \neq \qval'$,
	% and for all $z^j \in \lvar / (\diff \cup \{x^i\} )$ 
	% there exist $\trace, \trace' \in \mathcal{T}, v, v'$ such that 
	% $\env(\trace) z = \env(\trace') z$, and 
	% $
	% \config{\trace, \aexpr} \aarrow v$, and $\config{\trace', \aexpr} \aarrow v'$ with $v \neq v'$,
	then $\exists x^i \in \lvar_{\diff} $ 
	such that $x \in FV(\qexpr)$ and $i = \llabel(\trace) x$.
	\[
		\begin{array}{l}
		\forall \lvar_{\diff}  \subset \lvar,  
		% x^i \in (\lvar \setminus \diff), 
		\aexpr \st
		\\ \quad
		\forall z^j \in \lvar \setminus \lvar_{\diff} \st 
		\exists \trace, \trace' \in \mathcal{T}, \qval, \qval' \st 
		\env(\trace) z = \env(\trace') z 
		\land 
		\config{\trace, \qexpr} \qarrow \qval 
		\land \config{\trace', \qexpr} \qarrow \qval' \land \qval \neq \qval'
		\\ \quad
		% \implies 
		% \forall z^j \in \lvar / (\diff \cup \{x^i\} ) \st 
		% \exists \trace, \trace' \in \mathcal{T}, v, v'\st 
		% \env(\trace) z = \env(\trace') z \land 
		% \config{\trace, \aexpr} \aarrow v \land \config{\trace', \aexpr} \aarrow v' \land v \neq v'
		% \\ \qquad
		\implies \exists x^i \in \lvar_{\diff} \st 
		x \in FV(\qexpr) \land i = \llabel(\trace) x
		\end{array}
	\]
	\end{itemize}
	\end{lem}
	\begin{proof}
		Proof by showing contradiction and Applying Lemma~\ref{lem:inv_expr}.
	\end{proof}
	%
\begin{lem}[Expression Inversion Generalization-II]
	\label{lem:inv_expr_gnl_II}
	For all subset of the labelled variables $\diff \subset \lvar$, and $x^i \in (\lvar \setminus \diff)$,
	and an expression $\expr$, if 
	\begin{itemize}
		\item $\expr$ is an arithmetic expression $\aexpr$,
		% \\ 
		and for all $z^j \in \lvar \setminus \diff, \trace, \trace' \in \mathcal{T}, v, v'$ such that 
		$\env(\trace) z = \env(\trace') z$, and 
		$
		\config{\trace, \aexpr} \aarrow v$, and $\config{\trace', \aexpr} \aarrow v'$ with $v = v'$;
		and for all $z^j \in \lvar / (\diff \cup \{x^i\} )$ 
		there exist $\trace, \trace' \in \mathcal{T}, v, v'$ such that 
		$\env(\trace) z = \env(\trace') z$, and 
		$
		\config{\trace, \aexpr} \aarrow v$, and $\config{\trace', \aexpr} \aarrow v'$ with $v \neq v'$,
		then $x \in VAR(\aexpr)$ and $i = \llabel(\trace) x$.
		\[
			\begin{array}{l}
			\forall \diff \subset \lvar,  x^i \in (\lvar \setminus \diff), \aexpr \st
			\\ \quad
			\forall z^j \in \lvar \setminus \diff, \trace, \trace' \in \mathcal{T}, v, v' \st 
			\env(\trace) z = \env(\trace') z \land 
			\config{\trace, \aexpr} \aarrow v \land \config{\trace', \aexpr} \aarrow v' \land v = v'
			\\ \quad
			\implies 
			\forall z^j \in \lvar / (\diff \cup \{x^i\} ) \st 
			\exists \trace, \trace' \in \mathcal{T}, v, v'\st 
			\env(\trace) z = \env(\trace') z \land 
			\config{\trace, \aexpr} \aarrow v \land \config{\trace', \aexpr} \aarrow v' \land v \neq v'
			\\ \qquad
			\implies x \in VAR(\aexpr) \land i = \llabel(\trace) x
			\end{array}
		\]
	\item $\expr$ is a boolean expression $\bexpr$,
	and for all $ z^j \in \lvar \setminus \diff, \trace, \trace' \in \mathcal{T}, v, v'$ such that 
	$ \env(\trace) z = \env(\trace') z \land 
	\config{\trace, \bexpr} \barrow v \land \config{\trace', \bexpr} \barrow v' \land v = v'$;
	and for all
	$ z^j \in \lvar / (\diff \cup \{x^i\} ) \st 
	 \exists \trace, \trace' \in \mathcal{T}, v, v'\st 
	\env(\trace) z = \env(\trace') z \land 
	\config{\trace, \bexpr} \barrow v \land \config{\trace', \bexpr} \barrow v' \land v \neq v'$
	then 
	 $x \in VAR(\bexpr) \land i = \llabel(\trace) x$
	% \\
	\[
		\begin{array}{l}
		\forall \diff \subset \lvar,  x^i \in (\lvar \setminus \diff), \bexpr \st
		\\ \quad
		\forall z^j \in \lvar \setminus \diff, \trace, \trace' \in \mathcal{T}, v, v' \st 
		\env(\trace) z = \env(\trace') z \land 
		\config{\trace, \bexpr} \barrow v \land \config{\trace', \bexpr} \barrow v' \land v = v'
		\\ \quad
		\implies 
		\forall z^j \in \lvar / (\diff \cup \{x^i\} ) \st 
		 \exists \trace, \trace' \in \mathcal{T}, v, v'\st 
		\env(\trace) z = \env(\trace') z \land 
		\config{\trace, \bexpr} \barrow v \land \config{\trace', \bexpr} \barrow v' \land v \neq v'
		\\ \qquad
		\implies x \in VAR(\bexpr) \land i = \llabel(\trace) x
		\end{array}
	\]
% 
	\item $\expr$ is a query expression $\qexpr$,
	and for all $\diff \subset \lvar,  x^i \in (\lvar \setminus \diff), \qexpr$ such that 
	for all $ z^j \in \lvar \setminus \diff, \trace, \trace' \in \mathcal{T}, \qval, \qval' \st 
 \env(\trace) z = \env(\trace') z \land 
 \config{\trace, \qexpr} \qarrow \qval \land \config{\trace', \qexpr} \qarrow \qval' \land \qval =_q \qval'$;
 and for all 
	$ z^j \in \lvar / (\diff \cup \{x^i\} ) \st 
  \exists \trace, \trace' \in \mathcal{T}, \qval, \qval'\st 
 \env(\trace) z = \env(\trace') z \land 
 \config{\trace, \qexpr} \qarrow \qval \land \config{\trace', \qexpr} \qarrow \qval' \land \qval \neq_{q} \qval'$,
 then  $x \in VAR(\qexpr) \land i = \llabel(\trace) x$.
	% \\
	\[
		\begin{array}{l}
		\forall \diff \subset \lvar,  x^i \in (\lvar \setminus \diff), \qexpr \st
		\\ \quad
		\forall z^j \in \lvar \setminus \diff, \trace, \trace' \in \mathcal{T}, \qval, \qval' \st 
		\env(\trace) z = \env(\trace') z \land 
		\config{\trace, \qexpr} \qarrow \qval \land \config{\trace', \qexpr} \qarrow \qval' \land \qval =_q \qval'
		\\ \quad
		\implies 
		\forall z^j \in \lvar / (\diff \cup \{x^i\} ) \st 
		 \exists \trace, \trace' \in \mathcal{T}, \qval, \qval'\st 
		\env(\trace) z = \env(\trace') z \land 
		\config{\trace, \qexpr} \qarrow \qval \land \config{\trace', \qexpr} \qarrow \qval' \land \qval \neq_{q} \qval'
		\\ \qquad
		\implies x \in VAR(\qexpr) \land i = \llabel(\trace) x
		\end{array}
	\]
	\end{itemize}
	\end{lem}
	%
Proof Summary: 
\\
To show $x \in VAR(\aexpr)$, by showing contradiction ($\forall \trace, \trace'$ in second hypothesis  $v = v'$)
 if $x \notin VAR(\aexpr)$.
 \\
To show $i = \llabel(\trace)$, by showing contradiction ($\forall \trace, \trace'$ in second hypothesis  $v = v'$ ) 
if $j = \llabel(\trace) x$ and $i \neq j$.
\begin{proof}
	Taking an arbitrary expression $\expr$,
	 we have the following three cases.
\caseL{$\expr$ is an arithmetic expression $\aexpr$}
Taking an arbitrary set of labelled variables 
	$\diff \subset \lvar$, $x^i \in (\lvar \setminus \diff)$ satisfies:
	\\
	$\forall z^j \in \lvar \setminus \diff, \trace, \trace' \in \mathcal{T}, v, v' \st 
	\env(\trace) z = \env(\trace') z \land 
	\config{\trace, \aexpr} \aarrow v \land \config{\trace', \aexpr} \aarrow v' \land v = v' ~ (1)
	$
	\\
	and 
	$\forall z^j \in \lvar \setminus (\diff \cup \{x^i\} ) \st 
	\exists \trace, \trace' \in \mathcal{T}, v, v'\st 
	\env(\trace) z = \env(\trace') z \land 
	\config{\trace, \aexpr} \aarrow v \land \config{\trace', \aexpr} \aarrow v' \land v \neq v' ~ (2) 
	$,
	\\
	Let $\trace, \trace' \in \mathcal{T}, v, v'$ be the two traces and values satisfies hypothesis $(2)$.
	\\
	To show: $x \in VAR(\aexpr) \land i = \llabel(\trace) x$:
	\\
% \[
% 	\begin{array}{l}
% 	\forall \diff \subset \lvar,  x^i \in (\lvar \setminus \diff), \aexpr \st
% 	\\ \quad
% 	\forall z^j \in \lvar \setminus \diff, \trace, \trace' \in \mathcal{T}, v, v' \st 
% 	\env(\trace) z = \env(\trace') z \land 
% 	\config{\trace, \aexpr} \aarrow v \land \config{\trace', \aexpr} \aarrow v' \land v = v'
% 	\\ \quad
% 	\implies 
% 	\forall z^j \in \lvar / (\diff \cup \{x^i\} ) \st 
% 	\exists \trace, \trace' \in \mathcal{T}, v, v'\st 
% 	\env(\trace) z = \env(\trace') z \land 
% 	\config{\trace, \aexpr} \aarrow v \land \config{\trace', \aexpr} \aarrow v' \land v \neq v'
% 	\\ \qquad
% 	\implies x \in VAR(\aexpr) \land i = \llabel(\trace) x
% 	\end{array}
% \] 
Assuming $x \notin VAR(\aexpr)$, we know from the Inversion Lemma~\ref{lem:inv_expr} of the arithmetic expression case,
\\
$\forall z^j \in \lvar \setminus \{x^i\}, \trace, \trace' \in \mathcal{T}, v, v' \st 
\env(\trace) z = \env(\trace') z \land 
\config{\trace, \aexpr} \aarrow v \land \config{\trace', \aexpr} \aarrow v' \land v = v'$.
\\
Then with the hypothesis $(1)$, we know:
\\
$\forall z^j \in \lvar \setminus (\diff \cup \{x^i\} ), \trace, \trace' \in \mathcal{T}, v, v'\st 
\env(\trace) z = \env(\trace') z \land 
\config{\trace, \aexpr} \aarrow v \land \config{\trace', \aexpr} \aarrow v' \land v = v'$
\\
This is contradicted to the hypothesis $(2)$.
\\
Then we know $x \in VAR(\expr)$.
\\
Assuming $j = \llabel(\trace) x \land i \neq j$,
by hypothesis $(2)$ where 
% $\config{\trace, \aexpr} \aarrow v $	and $\config{\trace', \aexpr} \aarrow v'$, 
% since $x \notin VAR(\aexpr)$ and 
$ \forall z^j \in \lvar \setminus (\diff \cup \{x^i\} )  \st\env(\trace) z = \env(\trace') z $, 
we know $\env(\trace) x = \env(\trace') x$, i.e., 
\\
$\forall z^j \in  \lvar \setminus (\diff  ) \st \env(\trace) z = \env(\trace') z$.
\\
Then we have $v' = v$ by hypothesis $(1)$, which is contradicted to $v' \neq v$.
\\
Then we know $i = \llabel(\trace) x $.

\caseL{$\expr$ is a boolean expression $\bexpr$}
This case is proved trivially in the same way as the case of the arithmetic expression.
\caseL{$\expr$ is a query expression $\qexpr$}
This case is proved trivially in the same way as the case of the arithmetic expression.
\end{proof}
	%
% \begin{lem}[Boolean Inversion Generalization]
% 		\label{lem:inv_b_gnl}
% 		% For all {$ x^i \in \lvar$, and $\trace, \trace' \in \mathcal{T}$,  and arithmetic expression $\aexpr$}, if
% 		% $ \forall z^j \in \lvar / \{x^i\} \st 
% 		% \env(\trace) z = \env(\trace') z $ and $\config{\trace, \aexpr} \aarrow v $ and 
% 		% $\config{\trace', \aexpr} \aarrow v' $ with $v' \neq v$, then $ x $ is in the free variables of $\aexpr$, i.e., $x \in VAR(\aexpr)$.
% 		%
% 		\[
% 			\begin{array}{l}
% 			\forall \diff \subset \lvar,  x^i \in (\lvar \setminus \diff), \bexpr \st
% 			\\ \quad
% 			\forall z^j \in \lvar \setminus \diff, \trace, \trace' \in \mathcal{T}, v, v' \st 
% 			\env(\trace) z = \env(\trace') z \land 
% 			\config{\trace, \bexpr} \barrow v \land \config{\trace', \bexpr} \barrow v' \land v = v'
% 			\\ \quad
% 			\implies 
% 			\forall z^j \in \lvar / (\diff \cup \{x^i\} ) \st 
% 			 \exists \trace, \trace' \in \mathcal{T}, v, v'\st 
% 			\env(\trace) z = \env(\trace') z \land 
% 			\config{\trace, \bexpr} \barrow v \land \config{\trace', \bexpr} \barrow v' \land v \neq v'
% 			\\ \qquad
% 			\implies x \in VAR(\bexpr) \land i = \llabel(\trace)
% 			\end{array}
% 		\]
% \end{lem}%
%
% \begin{lem}[Assignment Inversion].
% \label{lem:inv_a}
% \[
% 	\forall x \in \lvar, \expr \st 
% 	\Big( \exists \trace, \trace' \st \forall z^i \in \lvar / \{x^l\} \st 
% 	\env(\trace) z = \env(\trace') z \st 
% 	\config{\trace, \expr} \aarrow v \land \config{\trace', \expr} \aarrow v' \land v  \neq v'
% 	\implies x \in VAR(\expr) \land x^l \in \Big)
% \]
% \end{lem}
%
% \begin{lem}[Assignment Event Inversion]
% \label{lem:inv_asn}
% \[
% \begin{array}{l}
% 	\forall \trace_0 \in \mathcal{T}, c \in \cdom,
% 	\event \in \eventset^{\asn} \st
% 	\config{c, \trace_0} \rightarrow^* \config{\eskip, \trace_0 \tracecat \trace_1} \implies
% 	(\event \eventin \trace_1 \land	x = \pi_1(\event) \land l = \pi_2(\event))
% 	\\ 
% 	\implies 
% 	\big( 
% 		\exists \trace_1' \in \mathcal{T}, \expr, c' \st
% 		\config{c, \trace_0} \rightarrow^* \config{ [\assign{x}{\expr}]^l;c', \trace_0  \tracecat  \trace'} \rightarrow^\rname{assn}
% 		\config{c', \trace_0 \tracecat \trace_1'\tracecat [\event]} \rightarrow^{*}
% 		\config{\eskip, \trace_0  \tracecat  \trace_1}
% 	\big)
% 	\\ \qquad \lor
% 	\big( 
% 		\exists \trace_1' \in \mathcal{T}, \qexpr, c' \st
% 		\config{c, \trace_0} \rightarrow^* \config{ [\assign{x}{\query(\qexpr)}]^l;c', \trace_0 \tracecat \trace_1'} \rightarrow^{query}
% 		\config{c', \trace_0  \tracecat  \trace_1' \tracecat [\event] } \rightarrow^{*}
% 		\config{ \eskip, \trace_0\trace_1}
% 	\big)
% \end{array}
% \]
% %
% \end{lem}
% %
% \begin{lem}[Testing Event Inversion]
% \label{lem:inv_test}
% \[
% \begin{array}{l}
% 	\forall c\in \cdom, \trace_0 \in \mathcal{T}, \event = (b, l, n, v) \in \eventset^{\test} \st
% 	 \config{c, \trace_0} \rightarrow^* \config{\eskip, \trace_0 \tracecat \trace_1}
% 	\implies  \event \eventin \trace_1 \\
% 	\implies 
% 	\big( 
% 		\exists \trace_1' \in \mathcal{T}, \bexpr, c', c_t, c_f, c'' \in \cdom \st
% 		\config{c, \trace_0} \rightarrow^* \config{\eif ([b]^l, c_t, c_f);c', \trace_0 \tracecat \trace_1'} \rightarrow^{if-b}
% 		\config{c'', \trace_0 \tracecat \trace_1'\tracecat [\event] } \rightarrow^{*}
% 		\config{\eskip, \trace_0 \tracecat \trace_1} 
% 	\big)
% 	\\ \qquad \lor
% 	\big( 
% 		\exists \trace_1' \in \mathcal{T}, \bexpr, c', c_w, c'' \in \cdom \st
% 		\config{c, \trace_0} \rightarrow^* \config{ \ewhile([b]^l, c_w);c', \trace_0 \tracecat  \trace_1'} \rightarrow^{while-b}
% 		\config{c'', \trace_0 \tracecat \trace_1'\tracecat [\event] } \rightarrow^{*}
% 		\config{\eskip, \trace_0  \tracecat \trace_1}
% 	\big)
% \end{array}
% \]
% \end{lem}
\begin{lem}[Event Inversion]
\label{lem:inv_event}
For all $c\in \cdom, \trace_0 \in \mathcal{T}, \event \in \eventset$such that 
$\config{c, \trace_0} \rightarrow^* \config{\eskip, \trace_0 \tracecat \trace_1}$, 
and $\event \eventin \trace_1$, if 
\begin{itemize}
	\item $\event \in \eventset^{\asn}$, then either
	\begin{itemize}
	 \item there exists $\trace_1' \in \mathcal{T}, c' \in \cdom, \expr$ such that
\[
\begin{array}{l}
	% \forall \trace_0 \in \mathcal{T}, c \in \cdom,
	% \event \in \eventset^{\asn} \st
	% \config{c, \trace_0} \rightarrow^* \config{\eskip, \trace_0 \tracecat \trace_1} \implies
	% (\event \eventin \trace_1 \land	x = \pi_1(\event) \land l = \pi_2(\event))
	% \\ 
	% \implies 
	% \big( 
		% \exists \trace_1' \in \mathcal{T}, \expr, c' \st
		\config{c, \trace_0} \rightarrow^* \config{ [\assign{x}{\expr}]^l;c', \trace_0  \tracecat  \trace'} \rightarrow^\rname{assn}
		\config{c', \trace_0 \tracecat \trace_1'\tracecat [\event]} \rightarrow^{*}
		\config{\eskip, \trace_0  \tracecat  \trace_1}
	% \big)
	% \\  \lor
	% \big( 
	% 	\exists \trace_1' \in \mathcal{T}, \qexpr, c' \st
	% 	\config{c, \trace_0} \rightarrow^* \config{ [\assign{x}{\query(\qexpr)}]^l;c', \trace_0 \tracecat \trace_1'} \rightarrow^{query}
	% 	\config{c', \trace_0  \tracecat  \trace_1' \tracecat [\event] } \rightarrow^{*}
	% 	\config{ \eskip, \trace_0\trace_1}
	% \big)
\end{array}
\]
\item or there exists $\trace_1' \in \mathcal{T}, c' \in \cdom, \qexpr$ such that 
\[
\begin{array}{l}
	% \forall \trace_0 \in \mathcal{T}, c \in \cdom,
	% \event \in \eventset^{\asn} \st
	% \config{c, \trace_0} \rightarrow^* \config{\eskip, \trace_0 \tracecat \trace_1} \implies
	% (\event \eventin \trace_1 \land	x = \pi_1(\event) \land l = \pi_2(\event))
	% \\ 
	% \implies 
	% \big( 
	% 	\exists \trace_1' \in \mathcal{T}, \expr, c' \st
	% 	\config{c, \trace_0} \rightarrow^* \config{ [\assign{x}{\expr}]^l;c', \trace_0  \tracecat  \trace'} \rightarrow^\rname{assn}
	% 	\config{c', \trace_0 \tracecat \trace_1'\tracecat [\event]} \rightarrow^{*}
	% 	\config{\eskip, \trace_0  \tracecat  \trace_1}
	% \big)
	% \\  \lor
	% \big( 
		% \exists \trace_1' \in \mathcal{T}, \qexpr, c' \st
		\config{c, \trace_0} \rightarrow^* \config{ [\assign{x}{\query(\qexpr)}]^l;c', \trace_0 \tracecat \trace_1'} \rightarrow^{query}
		\config{c', \trace_0  \tracecat  \trace_1' \tracecat [\event] } \rightarrow^{*}
		\config{ \eskip, \trace_0 \tracecat \trace_1}
	% \big)
\end{array}
\]
\end{itemize}

\item $\event\in \eventset^{\test}$ then either 
\begin{itemize}
\item there exists $\trace_1' \in \mathcal{T}, c', c_t, c_f, c'' \in \cdom, \bexpr$ such that
\[
\begin{array}{l}
	% \big( 
	% 	\exists \trace_1' \in \mathcal{T}, c', c_t, c_f, c'' \in \cdom, \bexpr \st
		\config{c, \trace_0} \rightarrow^* \config{\eif ([b]^l, c_t, c_f);c', \trace_0 \tracecat \trace_1'} \rightarrow^{if-b}
		\config{c'', \trace_0 \tracecat \trace_1'\tracecat [\event] } \rightarrow^{*}
		\config{\eskip, \trace_0 \tracecat \trace_1} 
	% \big)
	% \\  \lor
	% \big( 
	% 	\exists \trace_1' \in \mathcal{T}, \bexpr, c', c_w, c'' \in \cdom \st
	% 	\config{c, \trace_0} \rightarrow^* \config{ \ewhile([b]^l, c_w);c', \trace_0 \tracecat  \trace_1'} \rightarrow^{while-b}
	% 	\config{c'', \trace_0 \tracecat \trace_1'\tracecat [\event] } \rightarrow^{*}
	% 	\config{\eskip, \trace_0  \tracecat \trace_1}
	% \big)
\end{array}
\]
\item or there exists $ \trace_1' \in \mathcal{T}, c', c_w, c'' \in \cdom, \bexpr$ such that 
\[
% \begin{array}{l}
% 	\big( 
% 		\exists \trace_1' \in \mathcal{T}, \bexpr, c', c_t, c_f, c'' \in \cdom \st
% 		\config{c, \trace_0} \rightarrow^* \config{\eif ([b]^l, c_t, c_f);c', \trace_0 \tracecat \trace_1'} \rightarrow^{if-b}
% 		\config{c'', \trace_0 \tracecat \trace_1'\tracecat [\event] } \rightarrow^{*}
% 		\config{\eskip, \trace_0 \tracecat \trace_1} 
% 	\big)
% 	\\  \lor
% 	\big( 
% 		\exists \trace_1' \in \mathcal{T}, \bexpr, c', c_w, c'' \in \cdom \st
		\config{c, \trace_0} \rightarrow^* \config{ \ewhile([b]^l, c_w);c', \trace_0 \tracecat  \trace_1'} \rightarrow^{while-b}
		\config{c'', \trace_0 \tracecat \trace_1'\tracecat [\event] } \rightarrow^{*}
		\config{\eskip, \trace_0  \tracecat \trace_1}
% 	\big)
% \end{array}
\]
\end{itemize}
\end{itemize}
%
\end{lem}
Proof Summary: trivially by induction on $c$ and enumerate all operational semantic rules.
\begin{proof}
	Take arbitrary $\trace_0 \in \mathcal{T}$, by induction on $c$, we have following cases:
		\caseL{$c = [\assign{x}{\expr}]^l$}
		By the evaluation rule $\rname{assn}$, we have
		$
		{
		\config{[\assign{{x}}{\aexpr}]^{l},  \trace } 
		\xrightarrow{} 
		\config{\eskip, \trace \tracecat [({x}, l, v) ]}
		}$.
		\\
		Then we know $\trace_1 = [({x}, l, v)]$ and there is only 1 event $(x, l, v) \in \trace_1$.
		\\
		Then we have $\trace_1' = []$ and $c' = \eskip$ satisfying
		\\
		$\config{c, \trace_0} \rightarrow^* \config{ [\assign{x}{\expr}]^l;c', \trace_0  \tracecat  \trace'} \rightarrow^\rname{assn}
		\config{c', \trace_0 \tracecat \trace_1'\tracecat [\event]} \rightarrow^{*}
		\config{\eskip, \trace_0  \tracecat  \trace_1}$.
		\\
		This case is proved.
		\caseL{$c = [\assign{x}{\query(\qexpr)}]^l$}
		This case is proved trivially in the same way as \textbf{case: $c = [\assign{x}{\expr}]^l$}.
		\caseL{$c = c_{s1};c_{s2}$}
		This case is proved trivially by the induction hypothesis on $c_{s1}$ and $c_{s2}$ separately, we have this case proved.
		\caseL{$\ewhile [b]^{l} \edo c$}
		If the rule applied to is $\rname{while-t}$, we have:
		\\
		$\config{{\ewhile [b]^{l} \edo c_w, \trace}}
			\xrightarrow{} 
			\config{{
			c_w; \ewhile [b]^{l} \edo c_w,
			\trace \tracecat [(b, l, \etrue)]}}
			\xrightarrow{*} 
			\config{{
			\eskip,
			\trace \tracecat \trace_1}}
		$,
		\\
		%
		$(b, l, \etrue) \in \event^{\test}$ and $(b, l, \etrue) \in \trace_1$.
		\\
		Let $\trace' = []$, $c' = \eskip$ and $c'' = c_w; \ewhile [b]^{l} \edo c_w$, we know that they satisfy
		\\
		$\config{c, \trace_0} \rightarrow^* \config{ \ewhile([b]^l, c_w);c', \trace_0 \tracecat  \trace_1'} \rightarrow^{while-b}
		\config{c'', \trace_0 \tracecat \trace_1'\tracecat [\event] } \rightarrow^{*}
		\config{\eskip, \trace_0  \tracecat \trace_1}$
		\\
		% And we also have the existence of $l = l_b, b$ and $c_w$, and $\ewhile [b]^{l} \edo c_w \in_c c_2$ and  $c_1 \in c_w$.
		% \\
		% If $c_w$ isn't a sequence command, let $c_1 = c_w$, then we have $c_2 = \ewhile [b]^{l} \edo c_w,  \eskip)$ 
		% and $c_1 \in_c c_2$.
		% \\
		% And we also have the existence of $l = l_b, b$ and $c_w$, and $\ewhile [b]^{l} \edo c_w \in_c c_2$ and  $c_1 \in c_w$.
		% \\
		This case is proved.
		\\
		If the rule applied to is $\rname{while-f}$, we have
		\\
		$
		{
			\config{{\ewhile [b]^{l} \edo c_w, \trace}}
			\xrightarrow{}^\rname{while-f}
			\config{{
			\eskip,
			\trace \tracecat [((b, l, \efalse))]}}
		}$,
		$(b, l, \etrue) \in \event^{\test}$, and $(b, l, \etrue) \in \trace_1$.
		\\
		Let $\trace' = []$, $c' = \eskip$ and $c'' = \eskip$, we know that they satisfy
		\\
		$\config{c, \trace_0} \rightarrow^* \config{ \ewhile([b]^l, c_w);c', \trace_0 \tracecat  \trace_1'} 
		\rightarrow^\rname{while-f}
		\config{c'', \trace_0 \tracecat \trace_1'\tracecat [(b, l, \efalse)] } \rightarrow^{*}
		\config{\eskip, \trace_0  \tracecat \trace_1}$
		\\
		This case is proved.
		\caseL{$\eif([b]^l, c_t, c_f)$}
		This case is proved in the same way as \textbf{case: $c = [\assign{x}{\query(\qexpr)}]^l$}.
	\end{proof}
%
% \begin{lem}[Testing Event Inversion]
% \label{lem:inv_test}
% \[
% \begin{array}{l}
% 	\forall c\in \cdom, \trace_0 \in \mathcal{T}, \event = (b, l, n, v) \in \eventset^{\test} \st
% 	 \config{c, \trace_0} \rightarrow^* \config{\eskip, \trace_0 \tracecat \trace_1}
% 	\implies  \event \eventin \trace_1 \\
% 	\implies 
% 	\big( 
% 		\exists \trace_1' \in \mathcal{T}, \bexpr, c', c_t, c_f, c'' \in \cdom \st
% 		\config{c, \trace_0} \rightarrow^* \config{\eif ([b]^l, c_t, c_f);c', \trace_0 \tracecat \trace_1'} \rightarrow^{if-b}
% 		\config{c'', \trace_0 \tracecat \trace_1'\tracecat [\event] } \rightarrow^{*}
% 		\config{\eskip, \trace_0 \tracecat \trace_1} 
% 	\big)
% 	\\ \qquad \lor
% 	\big( 
% 		\exists \trace_1' \in \mathcal{T}, \bexpr, c', c_w, c'' \in \cdom \st
% 		\config{c, \trace_0} \rightarrow^* \config{ \ewhile([b]^l, c_w);c', \trace_0 \tracecat  \trace_1'} \rightarrow^{while-b}
% 		\config{c'', \trace_0 \tracecat \trace_1'\tracecat [\event] } \rightarrow^{*}
% 		\config{\eskip, \trace_0  \tracecat \trace_1}
% 	\big)
% \end{array}
% \]
% \end{lem}
%
% \begin{lem}[Control Dependency -> Exists Testing Event]
% \label{lem:inv_ctltotest}
% \[
% 	\forall \event_1, \event_2 \in \eventset, c \st 
% 	\eventdep^{\ctl}(\event_1, \event_2, c, D)
% 	\implies
% 	\exists \event_b \in \eventset^{\test}, \trace_2 \in \mathcal{T} \st \eventdep(\event_1, \event_b, \trace_2, c, D)
% \]
% \end{lem}

% \begin{lem}[Control Dependency -> Event 2 in the Body Command of the Testing Event]
% \label{lem:inv_ctltoevent2}
% \[
% \begin{array}{l}
% 	\forall \event_1, \event_2 = (x_2, l_2, n_2, v_2) \in \eventset, c \st 
% 	\eventdep^{\ctl}(\event_1, \event_2, c, D)\\
% 	\implies
% 	\exists \event_b = (b, l, n, v) \in \eventset^{\test}, \expr_2 \st \eventdep(\event_1, \event_b, c, D)\\
% 	\quad \land \Big(
% 	\exists c_t, c_f \st (\eif ([b]{}^l, c_t, c_f)) \in_{c} c \land ([\assign{x_2}{\expr_2}]^{l_2}) \in_c c_t;c_f \\
% 	\qquad \lor\exists c_w \st (\ewhile [b]{}^l \edo c_w) \in_{c} c \land ([\assign{x_2}{\expr_2}]^{l_2}) \in_c c_w
% 	\Big)
% \end{array}
% \]
% \end{lem}
%
\begin{lem}[Reachable Varibale Inversion]
\label{lem:inv_live}
For all $c \in \cdom \trace, \trace' \in \mathcal{T} $, if 
$\config{c, \trace} \xrightarrow{}^* \config{c', \trace'}$,
and for all $x^l \in \lvar_c$ such that 
% $\llabel(\trace') x = l $, then $(x^l \in \live^{\entry_{c'}}(c))$.
$\llabel(\trace') x = l $, then $x^l \in \live(\absinit(c), c)$.
%
\[
	\forall c \in \cdom , \trace, \trace' \in \mathcal{T} \st
	\config{c, \trace} \xrightarrow{}^* \config{c', \trace'}
	\implies
	% \forall x^l \in \lvar_c \st \llabel(\trace') x = l \implies (x^l \in \live^{\entry_{c'}}(c))
	\forall x^l \in \lvar_c \st \llabel(\trace') x = l \implies x^l \in \live(\absinit(c), c)
\]
\end{lem}
Proof Summary: 
If a variable with the label which is the latest one in the trace,
Then by the environment definition, the value associated to this labelled variable is read from the trace.
\\
Then this labelled variable must be reachable at the point of $\entry_{c'}$, i.e., 
% $x^l \in \live^{\entry_{c'}}(c)$.
$x^l \in \live(\absinit(c), c)$.
\begin{proof}
	Take arbitrary $c \in \cdom , \trace, \trace' \in \mathcal{T}$ satisfying 
	$\config{c, \trace} \xrightarrow{}^* \config{c', \trace'}$, 
	and an arbitrary $x^l \in \lvar_c$ satisfying $\llabel(\trace') x = l$.
	\\
	By definition of $\llabel$, we know $\trace'$ has the form $\trace'_{a} \tracecat [(x, l, v)] \tracecat \trace_{b}'$
	for some $\trace'_{a} , \trace_{b}' \in \mathcal{T}$ and $v$.
	\\
	And the variable $x$ doesn't show up in all the events in $\trace_b'$.
%
\\
	Then, by the environment definition, we know:
	$\env(\trace') x = v$, i.e., $x^l$ is 
	reachable at the point of 
	% $\entry_{c'}$.
	$\absinit(c)$.
	\\
	By the $in(l)$ operator define in Section~\ref{sec:alg_edgegen}, we know $x^l$ is in the $in(\absinit(c)$ for prpgram $c$.
	\\
	% By the $\live$ definition, 
	Since $\live(\absinit(c), c)$ is a stabilized closure of $in(l)$ for $c$,
	we know 
	% $x^l \in \live^{\entry_{c'}}(c)$.
	$x^l \in \live(\absinit(c), c)$.
	\\
	This lemma is proved.
\end{proof}
%
\begin{lem}[While Loop Inversion]
	\label{lem:inv_while}
	For every $\trace, \trace' \in \mathcal{T}, c, c_1, c_2 \in \cdom$ 
	if $ \config{c, \trace} \rightarrow^* \config{c_1; c_2, \trace'}$ and 
	$c_1 \in_c c_2$, 
	then there must exist a $\ewhile$ command in $c_2$ and $c_1$ must shows up in the body of that $\ewhile$ command,
	 i.e., $\exists l \in \mathbb{N}, b \in \mathcal{B}, c_w \in \cdom \st 
	(\ewhile [b]^l \edo c_w) \in_c c_2 \land c_1 \in_c c_w$.
	%
	\[
	\begin{array}{l}
	\forall \trace, \trace' \in \mathcal{T}, c, c_1, c_2 \in \cdom \st
		\\ \quad
		\config{c, \trace} \rightarrow^* \config{c_1; c_2, \trace'}
		\implies
		c_1 \in_c c_2
		\implies
		\exists l \in \mathbb{N}, b \in \mathcal{B}, c_w \in \cdom \st 
		(\ewhile [b]^l \edo c_w) \in_c c_2 \land c_1 \in_c c_w
	\end{array}
	\]
	\end{lem}	
	Proof Summary: trivially by induction on $c$ and enumerate all operational semantic rules.
\begin{proof}
	Take arbitrary $\trace \in \mathcal{T}$, by induction on $c$, we have following cases:
		\caseL{$c = [\assign{x}{\expr}]^l$}
		By the evaluation rule $\rname{assn}$, we have
		$
		{
		\config{[\assign{{x}}{\aexpr}]^{l},  \trace } 
		\xrightarrow{} 
		\config{\eskip, \trace \tracecat [({x}, l, v) ]}
		}$.
		\\
		Since there doesn't exist $c_1, c_2 \in \cdom$ satisfying $\eskip = c_1; c_2$, this theorem is vacuously true.
		\caseL{$c = [\assign{x}{\query(\qexpr)}]^l$}
		By the evaluation rule $\rname{query}$, we have
		$
		{
		\config{[\assign{{x}}{\query(\qexpr)}]^{l},  \trace } 
		\xrightarrow{} 
		\config{\eskip, \trace \tracecat [({x}, l, \qval, v) ]}
		}$.
		\\
		Since there doesn't exist $c_1, c_2 \in \cdom$ satisfying $\eskip = c_1; c_2$, this theorem is vacuously true.
		\caseL{$c = \eif([b]^{l}, c_1, c_2)$}
		By the evaluation rule $\rname{query}$ and $\rname{if-f}$, and the label consistency, we know:
		\\
		for all possible $c_{t1}$ and $c_{t2}$ 
		such that $c_t$ has the form $c_t = c_{t1};c_{t2}$;
		\\
		all possible $c_{f1}$ and $c_{f2}$ 
		such that $c_f$ has the form $c_f = c_{f1};c_{f2}$,
		\\
		$c_{t1} \notin c_{t1}$ and $c_{f1} \notin c_{f2}$.
		\\
		Then this theorem is vacuously true.
		\caseL{$c = c_{s1};c_{s2}$}
		By label consistency, we know for every $c_1' \in_c c_{s1}$, $c_1' \notin c_{s2}$.
		\\
		Then by the induction hypothesis on $c_{s1}$ and $c_{s2}$ separately, we have this case proved.
		\caseL{$\ewhile [b]^{l} \edo c$}
		By rule $\rname{while-t}$, we have:
		\[
			\config{{\ewhile [b]^{l} \edo c_w, \trace}}
			\xrightarrow{} 
			\config{{
			c_w; \ewhile [b]^{l} \edo c_w,  \eskip),
			\trace \tracecat [\event]}}
		\]
		%
		If $c_w$ is a sequence command,
		let $c_1 = c_{w1}$ be the any possible command in this sequence, for all possible $c_{w1}$ and $c_{w2}$ 
		such that $c_w$ has the form $c_w = c_{w1};c_{w2}$.
		\\
		Then we have $c_2 = c_{w2};\ewhile [b]^{l} \edo c_w,  \eskip)$ and $c_1 \in_c c_2$.
		\\
		And we also have the existence of $l = l_b, b$ and $c_w$, and $\ewhile [b]^{l} \edo c_w \in_c c_2$ and  $c_1 \in c_w$.
		\\
		If $c_w$ isn't a sequence command, let $c_1 = c_w$, then we have $c_2 = \ewhile [b]^{l} \edo c_w,  \eskip)$ 
		and $c_1 \in_c c_2$.
		\\
		And we also have the existence of $l = l_b, b$ and $c_w$, and $\ewhile [b]^{l} \edo c_w \in_c c_2$ and  $c_1 \in c_w$.
		\\
		This case is proved.
		\\
		By the evaluation rule $\rname{while-f}$, we have
		$
		{
			\config{{\ewhile [b]^{l}, \edo c_w, \trace}}
			\xrightarrow{} 
			\config{{
			[\eskip]^l ,
			\trace \tracecat [((b, l, \efalse))]}}
		}$.
		\\
		Since there doesn't exist $c_1, c_2 \in \cdom$ satisfying $\eskip = c_1; c_2$, this theorem is vacuously true.
	\end{proof}
%
%
\begin{lem}[Only $\eskip$ Command doesn't Produce Event].
	\label{lem:inv_skip}
	For all trace $\trace\in \mathcal{T}$, and $c, c' \in \cdom$,  
	$\config{c, \trace} \rightarrow \config{c', \trace}$ if and only if $c = [\eskip];c'$. 
	\[
		\forall \trace\in \mathcal{T}, c, c' \in \cdom \st
		\config{c, \trace} \rightarrow \config{c', \trace}
		\Leftrightarrow 
		c = [\eskip];c'
	% \footnote{$([\eskip];){}^*$ denotes a sequence command only composed of $[\eskip]$ commands.}
	\]
	\end{lem}
\begin{proof}
	Proved trivially by induction on $c$ and enumerate all operational semantic rules.
\end{proof}
% \begin{lem}[Independent Events Doesn't Block $\flowsto$ for Testing Event]
% 	\label{lem:inv_indepeventstest}
% 	%
% 	For every $D \in \dbdom , c \in \cdom, \trace \in \mathcal{T}$, an assignment event
% 	$\event_1 \in \eventset^{\asn}$ and a test event $\event_2 \in \eventset^{\test}$,
% 	if the trace $trace$ has the form $\trace = [\event_1] \tracecat \trace' \tracecat [\event_2]$ with $\trace' \in \mathcal{T}$, 
% 	and $\eventdep(\event_1, \event_2, \trace, c, D)$,
% 	and every $\event \in \trace'$ doesn't have the \emph{May-Dependency} relations both on $\event_1$ and to $\event_2$,
% 	then 
% 	$\pi_1(\event_1) \in VAR(\pi_1(\event_2))$, and $ {\pi_2(\event_1)} = \llabel(\trace)$
% 	%
% 	\[
% 	\begin{array}{l}
% 		\forall D \in \dbdom , c \in \cdom, \trace \in \mathcal{T} \st \forall \event_1,\in \eventset^{\asn}, \event_2 \in \eventset^{\test} \st
% 		 \exists \trace' \in \mathcal{T} \st \trace = [\event_1] \tracecat \trace' \tracecat [\event_2]
% 		\implies
% 		\eventdep(\event_1, \event_2, \trace, c, D) 
% 		\\ \quad 
% 		\implies 
% 		\left( \forall \event \in \trace' \st \neg \eventdep(\event_1, \event, \trace[\event_1:\event], c, D)
% 		\lor \neg \eventdep(\event, \event_2, \trace[\event:\event_2], c, D) 
% 		\right) 
% 		\\ \quad 
% 		\implies 
% 		\pi_1(\event_1) \in VAR(\pi_1(\event_2)) \land {\pi_2(\event_1)} = \llabel(\trace)
% 	\end{array}
% 	\]
% \end{lem}
%
% \begin{lem}[Flow Search Algorithm ($\mathcal{A}$) Inversion 1]
% \label{lem:inv_alg1}
% For all $D \in \dbdom , c \in \cdom, \trace \in \mathcal{T}, \event_1, \event_2 \in \eventset^{\asn}$, and a list $l$,
% if $l \in \mathcal{A}(\event_1, \event_2, \trace, c, D)$,
% then l must have the form $[\pi_1(\event_1)^{\pi_2(\event_1)},\ldots, \pi_1(\event_2)^{\pi_2(\event_2)}]$.
% \[
% \begin{array}{l}
%     \forall D \in \dbdom , c \in \cdom, \trace \in \mathcal{T}, l \st \forall \event_1, \event_2 \in \eventset^{\asn} \st
%   \\ \quad 
%  l\in \mathcal{A}(\event_1, \event_2, \trace, c, D)  \implies  l = [\pi_1(\event_1)^{\pi_2(\event_1)},\ldots, \pi_1(\event_2)^{\pi_2(\event_2)}]
% \end{array}
% \]
% \end{lem}
% %
% \begin{proof}
% % Let $l \in \mathcal{A}(\event_1, \event_2, \trace, c, D)$. By definition of $\mathcal{A}$ we have 
% % \[l\in \kw{setmap} 
% % 	% \bigcup\limits_{l \in \kw{dfs}(\trace, c, D) \land l = \event_1 :: l'}
% % 	\left(\emap 
% % 		(\efun  \event \to \pi_1(\event)^{\pi_2(\event)})	
% % 	(\efilter 
% % 		(\efun \event \to  \event \in \eventset^{\asn})) \right)
% % 	S
% % \]
% % for $S=\kw{setfilter}
% % 		(\efun l \to \exists l' \st l = \event_1 :: l' ++ [\event_2]) ~ \kw{dfs}(\trace, c, D)$.
% % So, in particular by definition of setmap there is a list $l_1\in S$ such that 
% % \[
% % 	% \bigcup\limits_{l \in \kw{dfs}(\trace, c, D) \land l = \event_1 :: l'}
% % \emap 
% % 		(\efun  \event \to \pi_1(\event)^{\pi_2(\event)})	
% % 	(\efilter 
% % 		(\efun \event \to  \event \in \eventset^{\asn}))
% % 	l_1 = l
% % \]
% % \[
% % l = [\event_1, \cdots, \event_2]
% % \]
% % \\
% Let  $l \in \mathcal{A}(\event_1, \event_2, \trace, c, D)$,
% by definition of $\mathcal{A}$, we know 
% %
% $$l\in \kw{setmap} 
% 	% \bigcup\limits_{l \in \kw{dfs}(\trace, c, D) \land l = \event_1 :: l'}
% 		\left(\efun l \to ( \emap 
% 		(\efun  \event \to \pi_1(\event)^{\pi_2(\event)})
% 	(\efilter 
% 		(\efun \event \to  \event \in \eventset^{\asn}) ~ l) \right)
% 	~ S,
% $$
% %
% where $S=(\kw{setfilter} ~(\efun l \to l = [\event_1, \cdots, \event_2]) ~ (\kw{dfs} \eapp \trace \eapp c \eapp  D))$.
% \\
% Then, by definition of $\kw{setmap}$, we know $l$ is an output of
% \\
% $\left(\efun l \to ( \emap 
% 		(\efun  \event \to \pi_1(\event)^{\pi_2(\event)})
% 	(\efilter 
% 		(\efun \event \to  \event \in \eventset^{\asn}) ~ l) \right)$.
% \\
% Then we know there exists a preimage
% $l_e \in S $
% for $l$ such that 
% $$
% \emap (\efun  \event \to \pi_1(\event)^{\pi_2(\event)}) 
% (\efilter (\efun \event \to  \event \in \eventset^{\asn}) ~ l_e) 
% = l.
% $$
%  %
% Since $l_e \in (\kw{setfilter} ~(\efun l \to l = [\event_1, \cdots, \event_2]) ~ (\kw{dfs} \eapp \trace \eapp c \eapp  D))$,
% \\
% by the $\kw{setfilter}$ function,
% we know only the lists of events in $(\kw{dfs} \eapp \trace \eapp c \eapp  D)$ having the form
% $ [\event_1, \cdots, \event_2] $ are preserved in $S$, i.e.,
% \[
% 	\forall l \in (\kw{setfilter} ~(\efun l \to l = [\event_1, \cdots, \event_2]) ~ \kw{dfs}(\trace, c, D))
% 	\st l = [\event_1, \cdots, \event_2]
% \]
% %
% Then we know $l_e$ also has the same form, 
% i.e., $l_e = [\event_1, \cdots, \event_2]$.
% %
% \\
% Let $l_{ef} = (\efilter (\efun \event \to  \event \in \eventset^{\asn})) ~ l_e$, 
% by $\event_1, \event_2 \in \eventset^{\asn}$, 
% we know $\event_1$ and $\event_2$ are preserved in $l_{ef}$, i.e.,:
% \[
% 	l_{ef} =[\event_1, \cdots, \event_2]
% \]
% %
% Then, by applying the function
% $\emap (\efun  \event \to \pi_1(\event)^{\pi_2(\event)})$ to 
% $l_{ef}$, we have $l$ as follows:
% \[
% 	[\pi_1(\event_1)^{\pi_2(\event_1)}, \cdots, \pi_1(\event_2)^{\pi_2(\event_2)}]
% \]
% %
% %
% This lemma is proved.
% \end{proof}
% %
% \begin{lem}[Flow Search Algorithm ($\mathcal{A}$) Inversion 2]
% \label{lem:inv_alg2}
% For every $\event_1, \event_2 \in \eventset^{\asn}, D \in \dbdom , c \in \cdom$, we have either one of the two following cases:
% \begin{enumerate}
%   \item $\mathcal{A}(\event_1, \event_2,  [\event_1; \event_2], c, D) = 
%   \left\{[\pi_1(\event_1)^{\pi_2(\event_1)}, \pi_1(\event_2)^{\pi_2(\event_2)}] \right \}$ 
%   and $\eventdep(\event_1, \event_2, [\event_1; \event_2], c, D)$.
%   \item  $\mathcal{A}(\event_1, \event_2, [\event_1; \event_2], c, D) = \{\}$ 
%   and $\neg \eventdep(\event_1, \event_2, [\event_1; \event_2] c, D)$;
% \end{enumerate}
% \end{lem}
% % \wq{ Good! I just realize this lemma is only used for case 4 of 5.3.}
% %\jl{Yes!}
% \begin{proof}
% By definition of $A$, we know:
% %
% \[
% 	\begin{array}{l}
% 	\mathcal{A}(\event_1, \event_2, [\event_1; \event_2], c, D)
% 	= 
% 	\kw{setmap} ~
% 	% \bigcup\limits_{l \in \kw{dfs}(\trace, c, D) \land l = \event_1 :: l'}
% 	\\ \qquad \qquad
% 	\left(\efun l \to ( \emap 
% 		(\efun  \event \to \pi_1(\event)^{\pi_2(\event)})
% 	(\efilter 
% 		(\efun \event \to  \event \in \eventset^{\asn}) ~ l) \right)
% 	\\ \qquad \qquad
% 	(\kw{setfilter} ~
% 		(\efun l \to l = [\event_1, \cdots, \event_2]) ~ 
% 		% \left(\left\{[\event_2]\right\} \cup \left\{ \event_1 \stackrel{[\event_1; \event_2]}{\uplus} [\event_2] \right\} \right))
% 		\left(\left\{[\event_2]\right\} \cup \left(  {\uplus} \eapp \event_1 \eapp {[\event_1; \event_2]} \eapp [\event_2] \right) \right))
% 	\end{array}
% \]
% by definition of $ {\uplus} \eapp \event_1 \eapp {[\event_1; \event_2]} \eapp [\event_2]  $, we know 
% \[
% 	\begin{array}{l}
% 	% \event_1 \stackrel{[\event_1; \event_2]}{\uplus} [\event_2]
% 	{\uplus} \eapp \event_1 \eapp {[\event_1; \event_2]} \eapp [\event_2] 
% 	=   
% 	\\ \quad \qquad 	
% 	\ecase \eventdep(\event_1, \event_2, [\event_1; \event_2], c, D)
% 	\to \left\{ [\event_1, \event_2] \right\}
% 	\\ \quad \qquad 	
% 	\ecase \_
% 	\to \left\{ \right\}
% \end{array}
% \]
% %
% By simplification of the $\kw{setfilter}$, $\emap$, $\efilter$ and $\kw{setmap}$ functions, we know
% \\
% in the case of $\eventdep(\event_1, \event_2, [\event_1; \event_2], c, D)$:
% % \\
% $\mathcal{A}(\event_1, \event_2, [\event_1; \event_2], c, D) = 
%   \left\{[\pi_1(\event_1)^{\pi_2(\event_1)}, \pi_1(\event_2)^{\pi_2(\event_2)}] \right \}$
% \\
% (1) is proved.
% \\
% And in the case of $\neg \eventdep(\event_1, \event_2, [\event_1; \event_2], c, D)$: 
% % \\
% $\mathcal{A}(\event_1, \event_2, \cdot  \event_1 \tracecat [\event_2], c, D) = 
%   \left\{ \right \}$
% \\
% (2) is proved.
% \end{proof}
%
%
% \begin{lem}[\todo{Assignment Evaluation Inversion}].
% 	\label{lem:inv_eval_asn}
% 	\[
% 	\begin{array}{l}
% 		\forall x \in \lvar_c, \kw{V_{ptl}} \in \subseteq \lvar_c \expr \st 
% 		\exists \trace, \trace' \in \mathcal{T} \st 
% 		\\ \quad
% 		\forall z \in \lvar_c \setminus (\{x\} \cup \kw{V_{ptl}}) \st 
% 		\env(\trace) z = \env(\trace') z 
% 		\\ \quad \land
% 		\forall \event \in \trace, \event' \in \trace' \st 
% 		\pi_1(\event) \in \kw{V_{ptl}} \land \diff(\event, \event') 
% 		\\ \quad
% 		\implies 
% 		\neg \eventdep(\event, \event_y, \trace[\event:\event_y] ) 
% 		\implies
% 		\config{\trace, [\assign{y}{\expr}]{}^l;c'} \rightarrow^{asn} \config{\trace\cdot \event_y, c'}
% 		\\ \quad
% 		\implies 
% 		\config{\trace', [\assign{y}{\expr}]{}^l;c'} \rightarrow^{asn} \config{\trace'\cdot \event_y',c'}
% 		\land \diff(\event_y, \event_y')
% 		\implies x \in VAR(\expr)
% 	\end{array}
% 	\]
% 	\end{lem}	
%
%
% \todo{Event Dependency Transitivity}
\begin{lem}
	\label{lem:valdep_trans}
(Event Dependency Transitivity)
For every $D \in \dbdom , c \in \cdom, \trace \in \mathcal{T}$, and $\event_1, \event_2, \event_3 \in \eventset^{\asn}, \trace_{12}, \trace_{23} \in \mathcal{T}$,
if $\eventdep(\event_1, \event_2, \trace_{12}, c, D)$
and $\eventdep(\event_2, \event_3, \trace_{23}, c, D) $,
then $\eventdep(\event_1, \event_3, \trace_{12}\tracecat\trace_{23}, c, D)$.
  % An event $\event_2 \in \eventset^{\asn}$ is in the \emph{may-dependency} relation with another
  % event $\event_1 \in \eventset^{\asn}$ in a program ${c}$ with a hidden database $D$, denoted as $\eventdep(\event_1, \event_2, c, D)$,
  % if and only if
  \[
	  \begin{array}{l}
  \forall D \in \dbdom , c \in \cdom, \event_1, \event_2, \event_3 \in \eventset^{\asn}, \trace_{12}, \trace_{23} \in \mathcal{T} \st 
  \eventdep(\event_1, \event_2, \trace_{12}, c, D) 
  \land
  \eventdep(\event_2, \event_3, \trace_{23}, c, D) 
  \\ \quad
  \implies
  \eventdep(\event_1, \event_3, \trace_{12}\tracecat\trace_{23}, c, D)
	  \end{array}
  \]
\end{lem}
%
%
% \begin{lem}(Control Dependency Transitivity)
% \label{lem:ctl_trans}
% For every $D \in \dbdom , c \in \cdom, \event_1, \event_2 \in \eventset^{\test}, \event_3 \in \eventset$
% if $\eventdep^{\ctl}(\event_1, \event_2, c, D)$ and $\eventdep^{\ctl}(\event_2, \event_3, c, D)$,
% then $\eventdep^{\ctl}(\event_1, \event_3, c, D)$.
% %
% \[
%   \forall D \in \dbdom , c \in \cdom, \event_1, \event_2 \in \eventset^{\test}, \event_3 \in \eventset \st
%   \eventdep^{\ctl}(\event_1, \event_2, c, D) 
%   \land \eventdep^{\ctl}(\event_2, \event_3, c, D)
%   \implies \eventdep^{\ctl}(\event_1, \event_3, c, D)
% \]
% \end{lem}
%
% \begin{lem}
% 	\label{lem:eventdep_trans}
% (\emph{Variable May-Dependency} Transitivity)
% 	\[
% 	\forall c \in \cdom, D \in \dbdom , \event_1, \event_2, \event_3 \in \eventset^{\asn}\st 
% 	\eventdep(\event_1, \event_2, c, D) 
% 	\land
% 	\eventdep(\event_2, \event_3, c, D) 
% 	\implies
% 	\eventdep(\event_1, \event_3, c, D)
% 	\]
%   \end{lem}
%
\begin{lem}[\emph{Variable May-Dependency} Transitivity]
	\label{lem:vardep_trans}
For every $c \in \cdom, x^i, y^j, z^l \in \lvar_c$, 
if $\vardep(x^i, y^j, c)$ and 
$\vardep(y^j, z^l, c)$, then $\vardep(x^i, z^l, c)$.
	\[
	\forall c \in \cdom, x^i, y^j, z^l \in \lvar_c \st 
	\vardep(x^i, y^j, c) 
	\land
	\vardep(y^j, z^l, c) 
	\implies
	\vardep(x^i, z^l, c)
	\]
  \end{lem}
  %
%
