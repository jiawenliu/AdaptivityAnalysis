\begin{subproof}[Proof of the Basecase: Case 1]
%
\label{pf:eventdep_base_val}
We have the following by the definition $\eventdep(\event_1, \event_2, [\event_1; \event_2], c, D)$ for case 1:
\begin{equation}
  \label{eq:eventdep_def_base_val}
  \exists \vtrace_0,
    \vtrace_1, \vtrace' \in \mathcal{T},\event_1' \in \eventset^{\asn}, \event_2' \in \eventset, {c}_1, {c}_2  \in \cdom  \st
    \diff(\event_1, \event_1') \land
      \left(
      \begin{array}{ll}   
     & \config{{c}, \vtrace_0} \rightarrow^{*} 
    \config{{c}_1, \vtrace_1 \tracecat [\event_1]}  \rightarrow^{*} 
      \config{{c}_2,  \vtrace_1 \tracecat [\event_1; \event_2] } 
      % 
     \\ 
     \bigwedge &
      \config{{c}_1, \vtrace_1 \tracecat [\event_1']}  \rightarrow^{*} 
      \config{{c}_2,  \vtrace_1 \tracecat[ \event_1'] \tracecat \vtrace' \tracecat [\event_2'] } 
    \\
    \bigwedge & 
    \diff(\event_2,\event_2') \land 
    \vcounter(\vtrace) ~ \pi_2(\event_2)
    = 
    \vcounter(\vtrace') ~ \pi_2(\event_2)\\
    \end{array}
    \right)
  \end{equation}
  %
Let $\vtrace_0,
\vtrace_1, \vtrace' \in \mathcal{T}, \event_2' \in \eventset, \event_1' \in \eventset^{\asn}, {c}_1, {c}_2$ be the traces, events and commands satisfying the executions,
by Inversion Lemma~\ref{lem:inv_event} on 
$\event_1$, $\event_2$, we have the following instance of the first execution in Eq.~\ref{eq:eventdep_def_base_val},
 %
%
 %
\begin{equation}
\label{eq:valdep_inv1}
  \begin{array}{l}   
\config{{c}, \vtrace_0} \rightarrow^{*} 
\config{[\assign{x_1}{\expr_1 / \query(\qexpr_1)}]^{\pi_2(\event_1)} ; {c}_1, 
\vtrace_1}  
\rightarrow^\rname{assn/query}
 \config{c_1, \vtrace_1 \tracecat [\event_1]} \\
  \qquad \rightarrow^{*} 
  \config{[\assign{{x}_2}{\expr_2 / \query(\qexpr_2)}]^{l_2};{c}_2, 
  \vtrace_1 \tracecat [\event_1]} 
  \rightarrow^\rname{assn/query} 
  \config{{c}_2,  \vtrace_1 \tracecat [\event_1; \event_2]} 
  % 
\end{array}
\footnote{
$\assign{x}{\expr / \query(\qexpr)}$ denotes variable $x$ is assigned by either an expression $\expr$ or query $\query(\qexpr)$
}
\end{equation}
%
% \wqside{Some typo in equation 4, but I can follow:-)}
% \jl{thanks}
, where $x_1 = \pi_1(\event_1)$, $l_1 = {\pi_2(\event_1)}$, $x_2 = \pi_1(\event_2)$, $l_2 = \pi_2(\event_2)$, 
and $\expr_1 / \qexpr_1$, $\expr_2 / \qexpr_2$ are the expressions of the assignment commands 
associated to the $\event_1$ and $\event_2$ from  Lemma~\ref{lem:inv_event}.
\\
%
By $\diff(\event_2,\event_2')$ and the command label consistency,
we also have the instance of second execution in Eq.~\ref{eq:eventdep_def_base_val} as follows:
% \[
% \config{{c}_1, \vtrace_1 \tracecat [\event_1']}  \rightarrow^{*} 
%   \config{{c}_2',  \vtrace_1 \tracecat [\event_1']\cdot \vtrace_2' \tracecat [\event_2'] } 
%   \], 
% we know there exists $\expr_2'$ or $\qexpr_2'$ and following execution instance,
%  \[
%   \begin{array}{l}   
%   \config{c_1, \vtrace_1 \tracecat [\event_1']} 
%   \rightarrow^{*} 
%   \config{[\assign{{x}_2'}{\expr_2' / \query(\qexpr_2')}]^{l_2'} ; {c}_2', \vtrace_1 \tracecat [\event_1']\tracecat \vtrace'} 
%   \rightarrow^\rname{assn/query} 
%   \config{{c}_2',   \vtrace_1 \tracecat [\event_1'] \tracecat \vtrace' \tracecat [\event_2']} 
%   % 
% \end{array}
%  \]
%  , where  $x_2' = \pi_1(\event_2')$ and $l_2' = \pi_2(\event_2')$.
% %
% Unfolding $\diff(\event_2,\event_2')$, we have:
% \[
%   x_2 = x_2' \land l_2 = l_2' 
% \] 
% %
% Since each command in $c$ has a unique label, we have $\expr_2' = \expr_2$, $\qexpr_2 = \qexpr_2'$, and following execution instance:
\begin{equation}
\label{eq:valdep_inv2}
  \config{c_1, \vtrace_1 \tracecat [\event_1']} 
  \rightarrow^{*} 
  \config{[\assign{{x}_2}{\expr_2 / \query(\qexpr_2)}]^{l_2} ; {c}_2, \vtrace_1 \tracecat [\event_1']\cdot \vtrace_2'} 
  \rightarrow^\rname{assn/query} 
  \config{{c}_2,  \vtrace_1 \tracecat [\event_1']\cdot \vtrace_2' \tracecat [\event_2']} 
\end{equation}
%
From Eq.~\ref{eq:eventdep_def_base_val}, we also have
\begin{equation}
\label{eq:valdep_invn}
  \vcounter(\vtrace') l_2 = \vcounter( [] ) l_2 = 0
\end{equation}
By Inversion Lemma~\ref{lem:inv_skip} and the execution in Eq.~\ref{eq:valdep_inv1}, we know:
 \[
 c_1 =_c 
 [\eskip]{}^*;[\assign{{x}_2}{\expr_2 / \query(\qexpr_2)}]^{l_2};{c}_2
 	\footnote{$([\eskip];){}^*$ denotes a sequence command only composed of $[\eskip]$ commands.}
 \]
 %
By substituting $c_1$ in Eq.~\ref{eq:valdep_inv2}, the following subproof shows there is only 1 qualified
 instance of the execution in Eq.~\ref{eq:valdep_inv2}.
\begin{subproof}[Subproof]
\label{pf:noiteration_inv2}
There are two possibilities by the law of excluded middle:
\\
$[\assign{{x}_2}{\expr_2 / \query(\qexpr_2)}]^{l_2} \in_c c_2$ 
\\
or $[\assign{{x}_2}{\expr_2 / \query(\qexpr_2)}]^{l_2} \notin_c c_2$.
%
\begin{enumerate}
\item{$[\assign{{x}_2}{\expr_2 / \query(\qexpr_2)}]^{l_2}\notin_c c_2$}
\\
In this case, we have the following execution instance:
%
\footnote{$\rightarrow^{\rname{skip}^*}$ denotes the rule applied on 
every evaluation step of this execution is the $\rname{skip}$ rule.}
 %
  \[
  \config{c_1, \vtrace_1 \tracecat [\event_1']} 
  \rightarrow^{\rname{skip}^*} 
  \config{[\assign{{x}_2}{\expr_2 / \query(\qexpr_2)}]^{l_2} ; {c}_2, \vtrace_1 \tracecat [\event_1']} 
  \rightarrow^\rname{assn/query} 
  \config{{c}_2,  \vtrace_0 \tracecat \vtrace_1 \tracecat [\event_1'; \event_2']} 
 \]
%
\item{$[\assign{{x}_2}{\expr_2 / \query(\qexpr_2)}]^{l_2} \in_c c_2$}
\\
By Inversion Lemma~\ref{lem:inv_while}, 
we have a $\ewhile$ conditional command
 $(\ewhile [b_w]^l_w \edo c_w)$ in $c_2$, where
% $(\ewhile [b_w]^l_w \edo c_w) \in_c c_2$ and 
$[\assign{{x}_2}{\expr_2 / \query(\qexpr_2)}]^{l_2} \in_c c_w$.
% \\
Then, we have the following possible execution instances
 %
  \[
  \config{c_1, \vtrace_1 \tracecat [\event_1']} 
  \rightarrow^{\rname{skip}^*} 
  \config{[\assign{{x}_2}{\expr_2 / \query(\qexpr_2)}]^{l_2} ; {c}_2, \vtrace_1 \tracecat [\event_1']} 
  \rightarrow^\rname{assn/query} 
  \config{{c}_2,  \vtrace_1 \tracecat [\event_1']\tracecat [\event_2']} 
 \]
%
  \[
  \begin{array}{l}
  \config{c_1, \vtrace_1 \tracecat [\event_1']} 
  \rightarrow^{\rname{skip}^*} 
  \config{[\assign{{x}_2}{\expr_2 / \query(\qexpr_2)}]^{l_2} ; {c}_2, \vtrace_1 \tracecat [\event_1']} 
  \rightarrow^\rname{assn/query} 
  \config{{c}_2,  \vtrace_1 \tracecat [\event_1']\tracecat [(x_2, l_2,  v_2')]} 
  \\ \qquad
  \rightarrow^{*} 
  \config{[\assign{{x}_2}{\expr_2 / \query(\qexpr_2)}]^{l_2} ; {c}_2, 
  \vtrace_1 \tracecat [\event_1']\tracecat [(x_2, l_2,  v_2')] \tracecat \trace_3} 
  \\ \qquad
  \rightarrow^\rname{assn/query} 
  \config{{c}_2,  \vtrace_1 \tracecat [\event_1']\tracecat [(x_2, l_2,  v_2')] \tracecat \trace_3 \tracecat [\event_2']} 
 \end{array}
 \]
\[
  \cdots
\] 
, where each execution instance iterates the conditional command 
$(\ewhile [b_w]^l_w \edo c_w)$ in $c_2$, $0, 1$ or more times.
%
\\
%
For each execution instance, we have the corresponding instance of $\trace'$ as follows:
\\
$\trace'  = [] $
\\
$\trace' = [(x_2, l_2,  v_2')] \tracecat \trace_3 $
%
\\
$\cdots$
%
\\
%
By Eq.~\ref{eq:valdep_invn} where $\vcounter(\trace') l_2 = 0$,
%
we know only the first execution instance with 0 iteration of $\ewhile$ command in $c_2$ satisfies this restriction, 
i.e., $\trace' = []$.
%
\end{enumerate}
In conclusion, we have the only qualified execution instance as follows where $\trace' = []$.
  \[
    \config{c_1, \vtrace_1 \tracecat [\event_1']} 
    \rightarrow^{\rname{skip}^*} 
    \config{[\assign{{x}_2}{\expr_2 / \query(\qexpr_2)}]^{l_2} ; {c}_2, \vtrace_1 \tracecat [\event_1']} 
    \rightarrow^\rname{assn/query} 
    \config{{c}_2,  \vtrace_1 \tracecat [\event_1']\tracecat [\event_2']} 
 \]
\end{subproof}
%
Then we know by the environment definition,
$\env$ obtains different values only for variable $x_1$ 
from trace $\vtrace_1 \tracecat [\event_1]$ and 
$\vtrace_1 \tracecat [\event_1']$, i.e.,
\[
  \forall z^r \in \lvar_c \setminus \{x_1^{l_1}\} ,
  \env(\vtrace_1 \tracecat [\event_1]) (z) =  
  \env(\vtrace_1 \tracecat [\event_1']) (z)
\]
%
By {Inversion Lemma~\ref{lem:inv_expr}} of arithmetic expression evaluation, we have
\[
  x_1 \in VAR(\expr_2 / \qexpr_2) 
\]
Since $\llabel(\vtrace_1 \tracecat [\event_1]) x_1 = l_1$, 
% by Inversion Lemma~\ref{lem:inv_live} we know $x_1^{l_1} \in \live^{l_2}(c)$.
by Inversion Lemma~\ref{lem:inv_live} we know $x_1^{l_1} \in \live(l_2, c)$.
%
\\
%
By $\flowsto$ definition, we have:
%
\[
\flowsto(x_1^{l_1}, {x}_2^{l_2}, c)
\]
i.e.,
%
\[
\flowsto(\pi_1(\event_1)^{\pi_2(\event_1)}, \pi_1(\event_2)^{\pi_2(\event_2)}, c)
 \]
%
This case is proved.
\end{subproof}