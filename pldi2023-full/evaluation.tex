\subsection{Implementation Results}
%%%%%%%%%%%%%%%%%%%%%%%%%%%%%%%%%%% Previous Version for Reference %%%%%%%%%%%%%%%%%%%%%%%%%%%%%%%%%%%%%%%%%%%%%%%%%
% We implemented $\THESYSTEM$ as a tool which takes a labeled command as input  
% and outputs two upper bounds on the program adaptivity and the number of query requests respectively.
% This implementation consists of an 
% abstract control flow graph generation,
% edge estimation (as presented in Section~\ref{sec:alg_edgegen}), and weight estimation (as presented in Section~\ref{sec:alg_weightgen}) in Ocaml, 
% and the adaptivity computation algorithm shown in Section~\ref{sec:alg_adaptcompute} in Python.
% The OCaml program takes the labeled command as input and outputs the program-based dependency graph and
% the abstract transition graph,
% feeds into the python program and the python program provides the adaptivity upper bound and the query number as the final output.

% We evaluated this implementation on $17$ example programs with the evaluation results shown in Table~\ref{tb:adapt-imp}.
% In this table,
% the first column is the name of each program.
% For each program $c$, the second column is its intuitive adaptivity rounds,
% the third column is the adaptivity $A(c)(\trace_0)$ w.r.t the input initial trace $\trace_0 \in \mathcal{T}_0(c)$ as definition~\ref{def:trace_adapt}.
% In all these examples, the input variable $k$ specifies the loop iteration numbers.
% Since $A(c)(\trace_0)$ by definition~\ref{def:trace_adapt} will count the execution times of
% query request command in the loop, which is indeed same as  the loop iteration numbers,
% we use $\env(\trace_0) k$ in the third column represent this number, which computes the $k$'s initial value from input initial trace $\trace_0$.
% The fifth column is the output of the $\THESYSTEM$ implementation, which consists of two expressions.
% The first one is the upper bound for adaptivity and the second one is the 
% upper bound for the total number of query requests in the program. And the last column is the performance evaluation w.r.t. the program size.

% For the forth program $\kw{multiRoundsO(k)}$, $\THESYSTEM$ outputs an over-approximated upper bound $1 + 2*k$ for the $A(c)$, which is consistent with our expectation as discussed in Example~\ref{ex:multiRoundsO}. 
% The fifth program is the evaluation results for the example in Example~\ref{ex:multiRoundsS}, where $\THESYSTEM$ outputs the tight bound for $A(c)$ but $A(c)$ is a loose definition of the program's actual adaptivity rounds.
% %
% The first two programs $\kw{twoRounds(k)}$, $ \kw{multiRounds(k)}$ are the same as Figure~\ref{fig:overview-example}(a) and Figure~\ref{fig:multipleRounds}(a).
% The other programs in the table from  $\kw{seq()}$ to $ \kw{nestWhileMPRV(k)}$ are 
% designed for testing the programs under different possible situations.
% They contain control dependency, data value dependency,
% the nested while, dependency through multiple variables, dependency across nested loops. 
% Overall for these examples, our system gives both the accurate adaptivity definition and estimated
% adaptivity upper bound through our formalization and analysis framework $\THESYSTEM$.
% The complete programs are defined below from Example~\ref{ex:twoRoundsComplete} to Example~\ref{ex:nestedWhileMPRV},
% and the implementations are in GitHub.
% \begin {table}[H]
%     \caption{Experimental results of {\THESYSTEM} implementation}
%         \label{tb:adapt-imp}
%         \begin{center}
%         \centering
% {\footnotesize
%         \begin{tabular}{ r | c | c | c | c | c | c }
%         \multirow{2}{*}{Program $c$} & 
%         \multirow{2}{*}{\emph{adaptivity}} 
%          & \multirow{2}{*}{$A(c) (\trace_0)$} 
%          & \multicolumn{2}{c|}{$\THESYSTEM$}
%          & \multicolumn{2}{c}{performance} \\ 
%          \cline{4-7}
%          & & & $\progA(c)$ & $\query$ \# & line of code & time (second) \\
%          \hline \hline
%          $  \kw{twoRounds(k)}$ & $2$ & $2$ & $2$ & $k$ & 8 & 0.014 \\
%          $  \kw{multipleRounds(k)}$ & $k$ & $ \env(\trace_0) k $ & $k$ & $k$  &  10 & 0.017 \\
%          $  \kw{lRGD(k, r)}$ & $k$ & $\env(\trace_0) k$ & $k$ & $2 * \env(\trace_0) k$  &  10 & 0.017  \\
%          $  \kw{multiRoundsO(k)}$ & $1 + k$ & $1 + (\env(\trace_0) k) $  & $1 +2 * k$ & $1 + 2*k$  &  10 & 0.019  \\
%          $  \kw{multiRoundsS(k)}$    & $2$ & $2 + (\env(\trace_0) k) $ & $2 + k$ & $2 + k$  &  9 & 0.017  \\
%          $  \kw{seq()}$ & $4$ & $4$ & $4$ & $4$ & 4 & 0.011 \\ 
%          $  \kw{seqRV()}$ & $4$ & $4$ & $4$ & $4$ & 4 & 0.011\\  
%          $  \kw{ifVD()}$ & $3$ & $3$ & $3$ & $3$ & 5 & 0.012 \\
%          $  \kw{ifCD()}$ & $3$ & $3$ & $3$ & $3$  & 5 & 0.009   \\
%          $  \kw{whileNested(k)}$ & $1+k$ & $1+ (\env(\trace_0) k)$ & $1+k$  &  $1+k$ & 7 & 0.015 \\
%          $  \kw{whileM(k)}$ & $1 + k$ & $1 +2 * \lfloor \frac{\env(\trace_0) k}{2} \rfloor$ & $1 +2 * \lfloor \frac{k}{2} \rfloor$ & $1 + 2 * k$  &  9 & 0.0139  \\
%          $  \kw{whileRV(k)}$ & $1 + 2*k$ & $1 + 2*(\env(\trace_0) k)$ & $1 + 2*k$ & $2 + 3 * k$  &  9 & 0.014  \\
%          $  \kw{whileVCD(k)}$ & $1 + 2*k$ & $1 + 2*(\env(\trace_0) k)$ & $1 + 2 * k$ & $2 + 2 * k$  &  6 & 0.007  \\
%          $  \kw{whileMPVCD(k)}$ & $2 + k$ & $2 + (\env(\trace_0) k)$  & $2 + k$ & $1 + 2 * k$   &   9 & 0.013 \\
%          $  \kw{nestWhileVD(k)}$ & $2 + k^2$ & $2 + (\env(\trace_0) k)^2$  & $2 + k^2$ & $1 + k + k^2$   &  10 & 0.022  \\
%          $  \kw{nestWhileRV(k)}$ & $1 + 2*k$ & $1 + 2*(\env(\trace_0) k)$ & $1 + 2*k$ &  $1 + k + k^2$   &  10 & 0.019  \\
%          $  \kw{nestWhileMR(k)}$ & $1 + k + k^2$ & $1 + (\env(\trace_0) k) + (\env(\trace_0) k)^2$  & $1 + k + k^2$ &  $2 + k + k^2$  & 10 & 0.039  \\
%          $  \kw{nestWhileMPRV(k)}$ & $1 + k + k^2$ & $1 + (\env(\trace_0) k) + (\env(\trace_0) k)^2$  & $1 + k + k^2$ &  $2 + k + k^2$  &  10 & 0.148  \\
%          $  \kw{mRComplete(k)}$ & $k$ & $ \env(\trace_0) k $ & $k$ & $k$   &  30 & 408.998 \\
%          $  \kw{mRComposed(k)}$ & $k$ & $ \env(\trace_0) k $ & $k$ & $k$   &  50 & 22539.638\\
%          $  \kw{tRComposed(k)}$ & $k$ & $ \env(\trace_0) k $ & $k$ & $k$  &  49 & * \\
%          $  \kw{seqComposed(k)}$ & $k$ & $ \env(\trace_0) k $ & $k$ & $k$  &  502 & * \\
%          $  \kw{jumboS(k)}$ & $k$ & $ \env(\trace_0) k $ & $k$ & $k$  &  70 & * \\
%          $  \kw{jumbo(k)}$ & $k$ & $ \env(\trace_0) k $ & $k$ & $k$  &  521 & * \\
%          $  \kw{big(k)}$ & $k$ & $ \env(\trace_0) k $ & $k$ & $k$  &  214 & *
%         \end{tabular}
% }        
% \end{center}
% \end{table}
%
%%%%%%%%%%%%%%%%%%%%%%%%%%%%%%%%%%% Previous Version Above %%%%%%%%%%%%%%%%%%%%%%%%%%%%%%%%%%%%%%%%%%%%%%%%%

\jl{
We implemented $\THESYSTEM$ as a tool which takes a labeled command as input  
and outputs two upper bounds on the program adaptivity and the number of query requests respectively.
This implementation consists of an 
abstract control flow graph generation,
edge estimation (as presented in Section~\ref{sec:alg_edgegen}), and weight estimation (as presented in Section~\ref{sec:alg_weightgen}) in Ocaml, 
and the adaptivity computation algorithm shown in Section~\ref{sec:alg_adaptcompute} in Python.
The OCaml program takes the labeled command as input and outputs the program-based dependency graph and
the abstract transition graph,
feeds into the python program and the python program provides the adaptivity upper bound and the query number as the final output.
}

We evaluated this implementation on $23$ example programs with the evaluation results shown in Table~\ref{tb:adapt-imp}.
In this table,
the first column is the name of each program.
For each program $c$, the second column is its intuitive adaptivity rounds,
% the third column is the adaptivity $A(c)(\trace_0)$ w.r.t the input initial trace $\trace_0 \in \mathcal{T}_0(c)$ as definition~\ref{def:trace_adapt}.
% In all these examples, the input variable $k$ specifies the loop iteration numbers.
% Since $A(c)(\trace_0)$ by definition~\ref{def:trace_adapt} will count the execution times of
% query request command in the loop, which is indeed same as  the loop iteration numbers,
% we use $\env(\trace_0) k$ in the third column represent this number, which computes the $k$'s initial value from input initial trace $\trace_0$.
the third column is the output of the $\THESYSTEM$ implementation, which consists of two expressions.
The first one is the upper bound for adaptivity and the second one is the 
upper bound for the total number of query requests in the program. And the last column is the performance evaluation w.r.t. the program size.

\jl{
The last column is the performance evaluation.
The time contains three parts. The first part is the running time of the Ocaml code, which parses the program and generates the $\progG(c)$.
The second and third parts are the running times of the reachability bound analysis algorithm
and the adaptivity computation algorithm, $\pathsearch(c)$.
}

\jl{
For the forth program $\kw{multiRoundsO(k)}$, $\THESYSTEM$ outputs an over-approximated upper bound $1 + 2*k$ for the $A(c)$, which is consistent with our expectation as discussed in Example~\ref{ex:multiRoundsO}. 
The fifth program is the evaluation results for the example in Example~\ref{ex:multiRoundsS}, where $\THESYSTEM$ outputs the tight bound for $A(c)$ but $A(c)$ is a loose definition of the program's actual adaptivity rounds.
%
The first two programs $\kw{twoRounds(k)}$, $ \kw{multiRounds(k)}$ are the same as Figure~\ref{fig:overview-example}(a) and Figure~\ref{fig:multipleRounds}(a).
The programs in the table from  $\kw{seq()}$ to $ \kw{nestWhileMPRV(k)}$ are 
designed for testing the programs under different possible situations.
They contain control dependency, data value dependency,
the nested while, dependency through multiple variables, dependency across nested loops. 
The last six programs are composed of some programs above in order to test the performance limitation when the input program is large. 
From the evaluation results, the performance bottleneck is the reachability bound analysis algorithm.
By implementing the bound analysis algorithm in Section~\ref{sec:alg_weightgen} (adapted from \cite{sinn2017complexity}), we are unable to evaluate the $\kw{Jumbo}$ in a reasonable time period.
Alternatively, we implement another light reachability bound analysis algorithm and compute the \emph{adaptivity} for
$\kw{jumboS}, \kw{jumbo}$ and $\kw{big}$ effectively.}

Overall for these examples, our system gives both the accurate adaptivity definition and estimated
adaptivity upper bound through our formalization and analysis framework $\THESYSTEM$.
The complete programs are defined below from Example~\ref{ex:twoRoundsComplete} to Example~\ref{ex:nestedWhileMPRV},
and the implementations are in GitHub.
{\tiny
\begin {table}[H]
    \caption{Experimental results of {\THESYSTEM} implementation}
        \label{tb:adapt-imp}
        \begin{center}
        \centering
{\footnotesize
        \begin{tabular}{ r | c | c | c | c | c | c | c  }
        \multirow{3}{*}{Program $c$} & 
        \multirow{3}{*}{\emph{adaptivity}}
         & \multicolumn{2}{c|}{$\THESYSTEM$}
         & \multicolumn{4}{c}{performance} \\ 
         \cline{3-8}
         & & \multirow{2}{*}{$\progA(c)$} & \multirow{2}{*}{$\query$ \#} & \multirow{2}{*}{line of code} & \multicolumn{3}{c}{running time (second)} \\ 
         \cline{6-8}
         & & & &  & Ocaml & Bound Analysis & $\pathsearch$  \\
         \hline \hline
         $  \kw{twoRounds(k)}$ & $2$ &  $2$ & $k$ & 8 & 0.00051 & 0.0017 & 0.00024 \\
         $  \kw{multiRounds(k)}$ & $k$ &  $k$ & $k$  &  10 & 0.001129 & 0.001678 & 0.00019 \\
         $  \kw{lRGD(k, r)}$ & $k$ & $k$ & $2 * \env(\trace_0) k$  &  10 & 0.001457 & 0.0071489 & 0.000207  \\
         $  \kw{mROdd(k)}$ & $1 + k$ &  $1 +2 * k$ & $1 + 2*k$  &  10 & 0.00144 & 0.00602 & 0.00018 \\
         $  \kw{mRSingle(k)}$    & $2$ &  $2 + k$ & $2 + k$  &  9 & 0.000959 & 0.00751 & 0.0001709 \\
         $  \kw{seq()}$ & $4$ & $4$ & $4$ & 4 & 0.001602 & 0.0001568 & 0.00006 \\ 
         $  \kw{seqRV()}$ & $4$ & $4$ &  $4$ & 4 & 0.001115 & 0.0002579 & 0.000053\\  
         $  \kw{ifVD()}$ & $3$ & $3$ &  $3$ & 5 & 0.00099 & 0.00045  & 0.00005 \\
         $  \kw{ifCD()}$ & $3$ & $3$ &   $3$  & 5 & 0.00049 & 0.00034  & 0.00004 \\
         $  \kw{while(k)}$ & $1+k$ &   $1+k$  &  $1+k$ & 7 & & 0.015 \\
         $  \kw{whileM(k)}$ & $1 + k$ &  $1 +2 * \lfloor \frac{k}{2} \rfloor$ & $1 + 2 * k$  &  9 & & 0.0139  \\
         $  \kw{whileRV(k)}$ & $1 + 2*k$ &  $1 + 2*k$ & $2 + 3 * k$  &  9 & & 0.014  \\
         $  \kw{whileVCD(k)}$ & $1 + 2*k$ &   $1 + 2 * k$ & $2 + 2 * k$  &  6 & & 0.007  \\
         $  \kw{whileMPVCD(k)}$ & $2 + k$ &  $2 + k$ & $1 + 2 * k$   &   9 & & & 0.013 \\
         $  \kw{nestWhileVD(k)}$ & $2 + k^2$ &   $2 + k^2$ & $1 + k + k^2$   &  10 &  & & 0.022  \\
         $  \kw{nestWhileRV(k)}$ & $1 + 2*k$ &   $1 + 2*k$ &  $1 + k + k^2$   &  10 & & & 0.019  \\
         $  \kw{nestWhileMR(k)}$ & $1 + k + k^2$ & $1 + k + k^2$ &  $2 + k + k^2$  & 10 & & & 0.039  \\
         $  \kw{nestWhileMPRV(k)}$ & $1 + k + k^2$ &  $1 + k + k^2$ &  $2 + k + k^2$  &  10 & & &  \\
         $  \kw{mRComplete(k)}$ & $k$ & $k$ & $k$   &  30 & & 87.998 & 0.013 \\
         $  \kw{mRComposed(k)}$ & $k$ & $*$ & $*$   &  50 & & 22539.638\\
         $  \kw{tRComposed(k)}$ & $k$ &  $*$ & $*$  &  49 & & * & *\\
         $  \kw{seqComposed(k)}$ & ** & $**$ & $326$  &  502 & 0.1541 & *  & 0.0253629684\\
         $  \kw{jumboS(k)}$ & $**$ & $**$ & $34$  &  70 & & * &  0.0182 \\
         $  \kw{jumbo(k)}$ & $**$ &  $**$ & $310$  &  521 & & * & 0.46\\
         $  \kw{big(k)}$ & $**$ &  $**$ & $124$  &  214 & & * & 0.147
        \end{tabular}
}        
\end{center}
\end{table}
}

 \subsection{The Evaluated Examples}  
 \paragraph{The complete algorithm for the data analysis example with two adaptivity rounds} 
 
\begin{example}[Complete Two Rounds Algorithm]
    \label{ex:twoRoundsComplete}

Below is the complete \emph{two rounds analyst strategy} for random data algorithm. This is instantiated from the
\emph{Custom Adaptive Analyst Strategy}, the Algorithm 5 in \cite{RogersRSSTW20} by setting the adaptive queries indices parameter, $S$ as the last column $\{ k \}$.

\begin{algorithm}
    \caption{The complete \emph{two rounds analyst strategy} for random data}
    \label{alg:twoRound}
    \begin{algorithmic}
    \REQUIRE Mechanism $\mathcal{M}$ with a hidden data set $D \in \{-1,+1\}^{n\times (k+1)} \subset \dbdom$.
    \STATE  {\bf for}\ $j\in [k]$\ {\bf do}.  
    \STATE \qquad {\bf define} $q_j(d)=d(j)\cdot d(k)$ where $d \in \{D(i) ~|~ i = 0, \cdots, n\} \subseteq \{-1,+1\}^{k+1}$.
    \STATE \qquad {\bf let} $a_j=\mathcal{M}(q_j)$ 
    \STATE \qquad \COMMENT{In the line above, $\mathcal{M}$ computes approx. the exp. value  of $q_j$ over $D$. So, $a_j\in [-1,+1]$.}
    \STATE {\bf define} $q_{k}(d)= d(k) \cdot \kw{sign}\big (\sum_{i\in [k]} x(i) \cdot \ln\frac{1+a_i}{1-a_i} \big )$ where $x\in \{-1,+1\}^{k+1}$.
    \STATE\COMMENT{In the line above,  $\kw{sign}(y)=\left \{ \begin{array}{lr} +1 & \kw{if}\ y\geq 0\\ -1 &\kw{otherwise} \end{array} \right . $.}
    \STATE {\bf let} $a_{k+1}=\mathcal{M}(q_{k+1})$
    \STATE\COMMENT{In the line above,  $\mathcal{M}$ computes approx. the exp. value  of $q_{k+1}$ over $X$. So, $a_{k+1}\in [-1,+1]$.}
    \RETURN $a_{k+1}$.
    \ENSURE $a_{k+1}\in [-1,+1]$
    \end{algorithmic}
    \end{algorithm}
    %
%
We also have the complete implementation of the algorithm above in our language below.
\[
    \kw{twoRounds(k)} \triangleq
\begin{array}{l}
       \clabel{ a \leftarrow []}^{1} ; \\
        \clabel{\assign{j}{k} }^{2} ; \\
        \ewhile ~ \clabel{j > 0}^{3} ~ \edo ~ \\
        \Big(
         \clabel{\assign{x}{\query(\chi[k - j]\cdot \chi[k])} }^{4}  ; \\
         \clabel{\assign{j}{j-1}}^{5} ;\\
        \clabel{a \leftarrow x :: a}^{6}       \Big);\\
        \clabel{l \leftarrow (\kw{sign}\big (\sum_{i\in [k]} \chi[i]\times\ln\frac{1+a[i]}{1-a[i]} \big ))}^{7}\\
    \end{array}
\]
%
The evaluation table in Tab.~\ref{tb:adapt-imp} shows that our {\THESYSTEM} works well for this complete implementation.
    \end{example}
 %
 \paragraph{The complete algorithm for the data analysis example with multiple adaptivity rounds} 
 
    \begin{example}[Complete Multiple Round Algorithm]
    %
    Below is the complete \emph{multiple rounds analyst strategy} for random data algorithm. This is instantiated from the
    \emph{Monitor}, the Algorithm 2 in \cite{RogersRSSTW20}.
    We instantiate this algorithm to trace the possible population shows in a database.
    We monitor the score of each person in the database. 
    \begin{algorithm}
    \footnotesize
    \caption{The complete \emph{multiple rounds analyst strategy} for random data}
    \label{alg:multiRound}
    \begin{algorithmic}
    \REQUIRE Mechanism $\mathcal{M}$ with a hidden state $X\in [N]^{n}$ sampled u.a.r., control set size $c$
    \STATE Define control dataset $C = \{0,1, \cdots, c - 1\}$
    \STATE Initialize $Nscore(i) = 0$ for $i \in [N]$, $I = \emptyset$ and $Cscore(C(i)) = 0$ for $i \in [c]$
    \STATE  {\bf for}\ $j\in [k]$\ {\bf do} 
    \STATE \qquad {\bf let} $p=\uniform(0,1)$ 
    \STATE \qquad {\bf define} $q (x) = \bernoulli ( p )$ .
    \STATE \qquad {\bf define} $qc (x) = \bernoulli ( p )$ .
    \STATE \qquad {\bf let} $a = \mathcal{M}(q)$ 
    \STATE \qquad {\bf for}\ $i \in [N]$\ {\bf do}
    \STATE \qquad \qquad $Nscore(i) = Nscore(i) + (a - p)*(q (i) - p)$ if $i \notin I$
    \STATE \qquad {\bf for}\ $i \in [c]$\ {\bf do}
    \STATE \qquad \qquad $Cscore(C(i)) = Cscore(C(i)) + (a - p)*(qc (i) - p)$
    \STATE \qquad {\bf let} $I = \{i | i\in [N] \land Nscore(i) > \max(Cscore)\}$
    \STATE \qquad {\bf let} $D = D \setminus I$ 
    \RETURN $D$.
    \end{algorithmic}
    \end{algorithm}
    %
    We also have the complete implementation of the algorithm above in our language below.
    {\small
    \begin{figure}
        \begin{subfigure}{1.0\textwidth}
        \begin{centering}
        $
    \kw{mRComplete(k, c, N)} \triangleq
    \begin{array}{l}
        \clabel{\assign{j}{N}}^0 ; 
         \clabel{\assign{cs}{0}}^1; 
         \clabel{\assign{ns}{0}}^2;
         \clabel{\assign{I}{0}}^3; 
         \clabel{\assign{w}{k}}^{4} ;\\
         \ewhile ~ \clabel{j > 0}^{5} ~ \edo ~ \\
         \Big(
         \clabel{\assign{j}{j-1}}^{6} ;
         \clabel{\assign{cs}{0 + cs}}^7; 
         \clabel{\assign{ns}{0 + ns}}^8
         \Big); \\
    
         \ewhile ~ \clabel{w > 0}^{9} ~ \edo ~ \\
        \Big(
        \clabel{\assign{w}{w-1}}^{10} ;
        \left[p \leftarrow c \right]^{11}; 
        \left[q \leftarrow c \right]^{12}; 
        \left[ a \leftarrow \query (\chi[I]) \right]^{13};\\
        \clabel{\assign{i}{N}}^{14} ; 
        \ewhile ~ \clabel{i > 0}^{15} ~ \edo ~ \\
        \Big(
        \clabel{\assign{i}{i-1}}^{16} ;
        \clabel{\assign{cs(i)}{cs(i) + (a - p) * (q - p)}}^{17}; \\
        \eif (\clabel{ I < i}^{18}, \clabel{\assign{ns(i)}{{ns(i) + (a - p) * (q - p)}}}^{19},
        \clabel{\assign{ns}{ns(i)}}^{20}    )
        \Big); \\
        \clabel{\assign{i2}{N}}^{21} ; \\
        \ewhile ~ \clabel{i2 > 0}^{22} ~ \edo ~ \\
        \Big(
        \clabel{\assign{i2}{i2-1}}^{23} ;
        \eif (\clabel{ns(i2) > \kw{max}(cs)}^{24}, 
        \clabel{\assign{I}{i + I}}^{25},
        \clabel{\assign{I}{I}}^{26})
        \Big)
        \Big) 
    \end{array}
       $
       \caption{}
        \end{centering}
        \end{subfigure}
        \vspace{-0.3cm}
        \caption{(a) The labeled program implementing the multiple round algorithm (b)The same program in the SSA version}
        \vspace{-0.5cm}
        \label{fig:multiround_complete}
        \end{figure}
    }
    The evaluation table in Tab.~\ref{tb:adapt-imp} shows that our {\THESYSTEM} works well for this complete implementation.
    %
    \end{example}
 %
\paragraph{The complete Linear Regression Algorithm with Gradient Decent Optimization}
%
\begin{example}[Linear Regression Algorithm with Gradient Decent Optimization]
\label{ex:linearregression}
    The linear regression algorithm with gradient decent Optimization works well 
    in our $\THESYSTEM$ as well.
            %   \[
            %   %
            %   \begin{array}{l}
            %   \kw{linearRegression(step, rate)} \triangleq \\
            %          \clabel{ a \leftarrow 0}^{0} ; \\
            %          \clabel{ c \leftarrow 0}^{1} ; \\
            %           \clabel{\assign{j}{\kw{step}} }^{2} ; \\
            %         %   \clabel{\assign{d}{10000000} }^{2} ; \\
            %           \ewhile ~ \clabel{j > 0}^{3} ~ \edo ~ \\
            %           \Big(
            %               \clabel{\assign{da}{\query(-2 * (\chi[1] - (\chi[0]\times a + c)) \times (\chi[0]))} }^{4}  ; \\
            %               \clabel{\assign{dc}{\query(-2 * (\chi[1] - (\chi[0]\times a + c)))} }^{5}  ; \\
            %               \clabel{\assign{a}{a - \kw{rate} * da} }^{6}  ; \\
            %               \clabel{\assign{c}{c - \kw{rate} * dc} }^{7}  ; \\
            %            \clabel{\assign{j}{j-1}}^{8} 
            %         %   \clabel{a \leftarrow x :: a}^{6} 
            %           \Big);
            %       \end{array}
            %   \]
              %
              %
                   %
\begin{figure}
\centering
\begin{subfigure}{0.45\textwidth}
    \centering
    {\small
        \[
        \begin{array}{l}
            \kw{linearRegressionGD(k, rate)} \triangleq \\
                   \clabel{ a \leftarrow 0}^{0} ; 
                   \clabel{ c \leftarrow 0}^{1} ; 
                    \clabel{\assign{j}{\kw{k}} }^{2} ; \\
                  %   \clabel{\assign{d}{10000000} }^{2} ; \\
                    \ewhile ~ \clabel{j > 0}^{3} ~ \edo ~ \\
                    \Big(
                        \clabel{\assign{da}{\query(-2 * (\chi[1] - (\chi[0]\times a + c)) \times (\chi[0]))} }^{4}  ; \\
                        \clabel{\assign{dc}{\query(-2 * (\chi[1] - (\chi[0]\times a + c)))} }^{5}  ; \\
                        \clabel{\assign{a}{a - \kw{rate} * da} }^{6}  ; 
                        \clabel{\assign{c}{c - \kw{rate} * dc} }^{7}  ; \\
                     \clabel{\assign{j}{j-1}}^{8} 
                  %   \clabel{a \leftarrow x :: a}^{6} 
                    \Big);
                \end{array}
        \]
        }
     \caption{}
        \end{subfigure}
      \begin{subfigure}{.45\textwidth}
          \begin{centering}
          \begin{tikzpicture}[scale=\textwidth/20cm,samples=200]
    % Variables Initialization
    \draw[] (-6, 1) circle (0pt) node{{ $a^0: {}^1_{0}$}};
    \draw[] (-6, 4) circle (0pt) node{{ $c^1: {}^{1}_{0}$}};
    % Variables Inside the Loop
       \draw[] (0, 10) circle (0pt) node{{ $da^4: {}^{k}_{1}$}};
       \draw[] (0, 7) circle (0pt) node{{ $dc^5: {}^{k}_{0}$}};
       \draw[] (0, 4) circle (0pt) node{{ $a^6: {}^{k}_{0}$}};
       \draw[] (0, 1) circle (0pt) node{{ $c^7: {}^{k}_{0}$}};
       % Counter Variables
       \draw[] (7, 9) circle (0pt) node {{$j^0: {}^{1}_{0}$}};
       \draw[] (7, 6) circle (0pt) node {{ $j^8: {}^{k}_{0}$}};
       %
       % Value Dependency Edges:
       \draw[ thick, -latex,] (0, 1.5)  -- (0, 3.5) ;
       \draw[ thick, -Straight Barb] (1.8, 4.2) arc (220:-100:1);
       \draw[ thick, -Straight Barb] (7.5, 6.5) arc (150:-150:1);
       \draw[](10, 6) node[] {\highlight{$k$}} ;
       \draw[ thick, -Straight Barb] (1.7, 1.) arc (120:-200:1);
       \draw[](4, 0) node[] {\highlight{$k$}} ;
       \draw[ thick, -latex] (6, 6.5)  -- 
       node [right] {\highlight{$k$}}(6, 8.5) ;
       % Value Dependency Edges on Initial Values:
       \draw[ thick, -latex,] (-2, 1)  -- 
       node [above] {\highlight{$k$}}(-4.5, 1) ;
       \draw[ thick, -latex,] (-2, 4)  -- 
       node [above] {\highlight{$k$}}(-4.5, 4) ;
       %
       \draw[ ultra thick, -latex, densely dotted,] (-1, 1.5)  to  [out=-220,in=220]  
       node [below] {\highlight{$k$}}(-1, 6.5);
       \draw[ ultra thick, -latex, densely dotted,] (-1, 4.5)  to  [out=-220,in=220]  
       node [above] {\highlight{$k$}}(-1, 9.5);
       \draw[ ultra thick, -latex, densely dotted,]  (1, 6.2) to  [out=-60,in=60] 
       node [below] {\highlight{$k$}}(0.5, 1.5) ;
       \draw[ ultra thick, -latex, densely dotted,]  (1.2, 9.2)  to  [out=-50,in=50] 
       node [above] {\highlight{$k$}}(0.5, 4.5);
       % Control Dependency
      %  \draw[ thick,-latex] (1.5, 7)  -- (4, 9) ;
      %  \draw[ thick,-latex] (1.5, 4)  -- (4, 9) ;
       \draw[ thick,-latex] (1.8, 10)  -- 
       node [above] {\highlight{$k$}}(5.5, 6) ;
       \draw[ thick,-latex] (1.8, 7)  -- (5.5, 6) ;
       \draw[ thick,-latex] (1.8, 4)  -- 
       node [above] {\highlight{$k$}}(5.5, 6) ;
       \draw[ thick,-latex] (1.8, 1)  -- 
       node [below] {\highlight{$k$}}(5.5, 6) ;
       \end{tikzpicture}
       \caption{}
          \end{centering}
          \end{subfigure}
          %
        \begin{subfigure}{.8\textwidth}
            \begin{centering}
            \begin{tikzpicture}[scale=\textwidth/20cm,samples=200]
      % Variables Initialization
      \draw[] (-6, 1) circle (0pt) node{{ $a^0: {}^1_{0}$}};
      \draw[] (-6, 4) circle (0pt) node{{ $c^1: {}^{1}_{0}$}};
      % Variables Inside the Loop
         \draw[] (0, 10) circle (0pt) node{{ $da^4: {}^{k}_{1}$}};
         \draw[] (0, 7) circle (0pt) node{{ $dc^5: {}^{k}_{0}$}};
         \draw[] (0, 4) circle (0pt) node{{ $a^6: {}^{k}_{0}$}};
         \draw[] (0, 1) circle (0pt) node{{ $c^7: {}^{k}_{0}$}};
         % Counter Variables
         \draw[] (7, 9) circle (0pt) node {{$j^0: {}^{1}_{0}$}};
         \draw[] (7, 6) circle (0pt) node {{ $j^8: {}^{k}_{0}$}};
         %
         % Value Dependency Edges:
         \draw[ thick, -latex,] (0, 1.5)  -- (0, 3.5) ;
         \draw[ thick, -Straight Barb] (1.8, 4.2) arc (220:-100:1);
         \draw[ thick, -Straight Barb] (7.5, 6.5) arc (150:-150:1);
         \draw[](10, 6) node[] {\highlight{$\trace_0 \to \env(\trace_0) k $}} ;
         \draw[ thick, -Straight Barb] (1.7, 1.) arc (120:-200:1);
         \draw[](4, 0) node[] {\highlight{$\trace_0 \to \env(\trace_0) k $}} ;
         \draw[ thick, -latex] (6, 6.5)  -- 
         node [right] {\highlight{$\trace_0 \to \env(\trace_0) k $}}(6, 8.5) ;
         % Value Dependency Edges on Initial Values:
         \draw[ thick, -latex,] (-2, 1)  -- 
         node [above] {\highlight{$\trace_0 \to \env(\trace_0) k $}}(-4.5, 1) ;
         \draw[ thick, -latex,] (-2, 4)  -- 
         node [above] {\highlight{$\trace_0 \to \env(\trace_0) k $}}(-4.5, 4) ;
         %
         \draw[ ultra thick, -latex, densely dotted,] (-1, 1.5)  to  [out=-220,in=220]  
         node [below] {\highlight{$\trace_0 \to \env(\trace_0) k $}}(-1, 6.5);
         \draw[ ultra thick, -latex, densely dotted,] (-1, 4.5)  to  [out=-220,in=220]  
         node [above] {\highlight{$\trace_0 \to \env(\trace_0) k $}}(-1, 9.5);
         \draw[ ultra thick, -latex, densely dotted,]  (1, 6.2) to  [out=-60,in=60] 
         node [below] {\highlight{$\trace_0 \to \env(\trace_0) k $}}(0.5, 1.5) ;
         \draw[ ultra thick, -latex, densely dotted,]  (1.2, 9.2)  to  [out=-50,in=50] 
         node [above] {\highlight{$\trace_0 \to \env(\trace_0) k $}}(0.5, 4.5);
         % Control Dependency
        %  \draw[ thick,-latex] (1.5, 7)  -- (4, 9) ;
        %  \draw[ thick,-latex] (1.5, 4)  -- (4, 9) ;
         \draw[ thick,-latex] (1.8, 10)  -- 
         node [above] {\highlight{$\trace_0 \to \env(\trace_0) k $}}(5.5, 6) ;
         \draw[ thick,-latex] (1.8, 7)  -- (5.5, 6) ;
         \draw[ thick,-latex] (1.8, 4)  -- 
         node [above] {\highlight{$\trace_0 \to \env(\trace_0) k $}}(5.5, 6) ;
         \draw[ thick,-latex] (1.8, 1)  -- 
         node [below] {\highlight{$\trace_0 \to \env(\trace_0) k $}}(5.5, 6) ;
         \end{tikzpicture}
         \caption{}
            \end{centering}
            \end{subfigure}
    \vspace{-0.5cm}
    \caption{(a) The linear regression algorithm 
    (b) The program-based dependency graph from $\THESYSTEM$
    (c) The execution-based dependency graph.}
    \vspace{-0.5cm}
    \label{fig:linear_regression}
\end{figure}
%
Analysis Result: $ \progA(\kw{linearRegressionGD(k, rate)}) = k$
\end{example} 
%
 
This linear regression algorithm 
% in order to
aims to
model a linear relationship between a dependent variable $y$,
% corresponding to the observed value in the column $\chi[1]$ in database, 
and an independent variable $x$, $y = a \times x + c$, specifically approximating the 
model parameter $a$ and $c$.
In order to have a good approximation on the model parameter 
$a$ and $c$, 
% corresponding to the observed value in the column $\chi[0]$ in database, 
it sends query to a training data set adaptively in every iteration.
This training data set contains two columns (can extend to higher dimensional data sets), first column is used as the observed value for the independent variable $x$,
second column is used as the observed label value for the dependent variable $y$.
This algorithm is written in our {\tt Query While} language in Figure~\ref{fig:linear_regression}(a) as $\kw{linearRegressionGD(k, rate)}$.
% taking the iteration number $\kw{step}$ 

This linear regression algorithm starts from initializing the linear model parameters and the counter variable,
and then goes into the training iterations.
In each iteration, it computes the differential value w.r.t. parameter
$a$ and $c$ respectively,
through requesting two queries, $\query(-2 * (\chi[1] - (\chi[0]\times a + c)) \times (\chi[0]))$ and 
$\query(-2 * (\chi[1] - (\chi[0]\times a + c)))$
at line 4 and 5.
Then, it uses these two differential values stored in variable $da$ and $dc$ to update the linear model parameters $a$ and $c$.
%
Its the program-based dependency graph is shown in Figure~\ref{fig:linear_regression}(b). Its execution-based dependency graph share the same graph, only needs to change the weight, $k$ into $w_k$ and $1$ for $w_1$ as we do in the previous example.
% We omit the detail of how to 
% generate this graph, which is similar to the generation procedure in 
% Example~\ref{alg:multiRound}.
In the execution-based dependency graph, there are multiple walks having the same longest query length.
For example, the walk $c^7 \to dc^6 : \to c^7 \to \cdots \to dc^6$ along the 
dotted arrows, where each vertex is visited $w_k(\trace_0)$ times for an initial trace $\trace_0$.
% By counting the total occurrence time of vertices with annotation $1$ in this walk, we have this program's adaptivity $k$.
There is actually other walks having the same query length $k$, the 
walk $a^7 \to da^6  \to a^7 \to \cdots \to da^6 $ along the 
dotted arrows, where each vertex is visited $w_k(\trace_0)$ times.
% the dotted path corresponds to a finite walk with the longest query length and its adaptivity on this walk is $k$.
But it doesn't affect the adaptivity for this program, which is still the maximal query length $w_k(\trace_0)$ with respect to initial trace $\trace_0$.
Also, $\THESYSTEM$, estimates the adaptivity $k$ for this example. Similarly as the multiple round example, we can show it is a tight bound.
%
 %          
\paragraph*{The complete Programs for Examples from line:6 - 17 in Table.\ref{tb:adapt-imp}}
%
    \begin{example}[The Complete Gradient Decent Optimization Algorithm]
        This example is the gradient decent algorithm example is a generalization of the linear regression on a higher degree data relation.
        It uses gradient decent algorithm to minimize 
        the mean square loss function
        for a two-degree relation
         $y = a_1 \times x_1^2 + a_2 \times x_2 + c$
        on the dataset of two feature columns and one indicator column.
     \[
     %
     \begin{array}{l}
     \kw{gradientDecent(step, rate, t, n)} \triangleq \\
        \clabel{ a_1 \leftarrow 0}^{0} ; \\
        \clabel{ a_2 \leftarrow 0}^{1} ; \\
        \clabel{ c \leftarrow 0}^{2} ; \\
        \clabel{\assign{j}{\kw{step}} }^{3} ; \\
        \ewhile ~ \clabel{j > 0}^{4} ~ \edo ~ \\
      \Big(
          \clabel{\assign{da1}{\query(-2 * (\chi[2] - (\chi[0]^2 \times a_1 + \chi[1] \times a_2 + c)) \times (\chi[0]))} }^{5}  ; \\
          \clabel{\assign{da2}{\query(-2 * (\chi[2] - (\chi[0]^2 \times a_1 + \chi[1] \times a_2 + c)) \times (\chi[1]))} }^{6}  ; \\  \clabel{\assign{dc}{\query(-2 * (\chi[2] - (\chi[0]^2 \times a_1 + \chi[1] \times a_2 + c)))} }^{5}  ; \\
          \clabel{\assign{a_1}{a_1 - \kw{rate} * da1} }^{7}  ; \\
          \clabel{\assign{a_2}{a_2 - \kw{rate} * da2} }^{8}  ; \\
          \clabel{\assign{c}{c - \kw{rate} * dc} }^{9}  ; \\
       \clabel{\assign{j}{j-1}}^{10} 
      \Big);
  \end{array}
     \]
     %
     %
        This approach can be generalized to the regression of a variety of 
        relations in machine learning area.
   %
     \end{example}
%

    \begin{example}[Sequence with Linear Query Dependency]
        \label{ex:seq}
        This example algorithm contains only sequence of four query commands.
        Each of them depends on a previous query.
        The longest dependency depth, i.e., the adaptivity is expectation to be $4$.
        %
        %
        \[
        %
            \kw{seq()} \triangleq 
        \begin{array}{l} 
               \clabel{ \assign{x}{\chi[0]}}^{0} ; 
   \clabel{\assign{y}{\chi[x + 1]} }^{1} ; \\
   \clabel{\assign{z}{\chi[y + 1]}}^{2}; 
    \clabel{\assign{w}{\chi[z + 1]} }^{3}
            \end{array}
        \]
        Evaluation Result: $ \progA( \kw{seq()}) = 4$
        \end{example}
    %
    \begin{example}[Sequence with Query Dependency between Related Variables]
        \label{ex:seqRV}
        %
        This example algorithm contains a sequence of four query commands.
        Each of them depends on one or more of the previous queries.
        The longest dependency depth, i.e., the adaptivity is expectation to be $4$.
        %
        \[
        %
            \kw{seqRV()} \triangleq 
        \begin{array}{l} 
               \clabel{ \assign{x}{\chi[0]}}^{0} ;
   \clabel{\assign{y}{\chi[x + 1]} }^{1} ; \\
   \clabel{\assign{z}{\chi[y + x]}}^{2}; 
    \clabel{\assign{w}{\chi[z + 1] \cdot \chi[y]} }^{3}
            \end{array}
        \]
        Evaluation Result: $ \progA(\kw{seqMultiVar()}) = 4$
    \end{example}
    %
        \begin{example}[If with Data-Value Dependency Separated]
            \label{ex:ifVD}
            This example algorithm contains a $\eif$ command and a query requests
            in each branch.
            Only the query in the first branch depend on the query in the command $0$,
            and the variable in the guard is not assigned by a query request.
            % Each of them depends on one or more of the previous queries.   %
            The longest dependency depth, i.e., the adaptivity is expectation to be $3$.
            \[
            %
            \kw{ifVD}(k) \triangleq 
            \begin{array}{l}
               \quad \clabel{ \assign{z}{\query(\chi[0])}}^{0} ; 
               \quad \clabel{\assign{x}{k / 2} }^{1} ; \\
               \quad \eif(\clabel{x < 0}^2,
               \quad \clabel{\assign{y}{\query(\chi[z])}}^{3},
               \quad \clabel{\assign{y}{\query(\chi[0])}}^{4})
   \end{array}
            \]
            Evaluation Result: $ \progA( \kw{ifVD()}) = 3$
        \end{example}
    
            \begin{example}[If with Data-Control Dependency Overlapped]
   \label{ex:ifCD}
   %
   This example algorithm contains a $\eif$ command and a query requests
   in each branch.
   The variable in the guard is assigned by a query request in command $1$.
   The two queries in the branches depend on the second query in command $1$
   but not depend on the query in the command $0$.
   Even though the variable $x$ isn't used in the query expression in the query $3$ and $4$,
   there are still dependency relation because $x$ is in the guard.
%
The longest dependency depth, i.e., the adaptivity is expectation to be $3$.
   \[
   %
   \kw{ifCD()} \triangleq 
   \begin{array}{l}
\clabel{ \assign{z}{\query(\chi[0])}}^{0} ;
\clabel{\assign{x}{\query(\chi[z])} }^{1} ; \\
\eif(\clabel{x < 0}^{2}, 
\clabel{\assign{y}{\query(\chi[0] + \chi[1])}}^{3}, 
\clabel{\assign{y}\query{(\chi[0])}}^{4})
   \end{array}
   \]
   %
   Evaluation Result: $ \progA( \kw{ifCD()}) = 3$
   \end{example}
    
    
\begin{example}[While with Nested Query Dependency]
\label{ex:whileNested}
This example algorithm contains a simple while loop.
There is one query requests in the loop body at command $3$.
In each iteration, the query request depend on the query result from previous iteration.
The longest dependency depth, i.e., the adaptivity is expectation to be $k$.
%
\[
%
\kw{whileNested}(k) \triangleq
\begin{array}{l}
    \clabel{ \assign{j}{k} }^{0} ; 
    \clabel{ \assign{a}{\query(\chi[0])} }^{1} ; \\
        \ewhile ~ \clabel{j > 0}^{2} ~ \edo ~ \\
        \Big(
         \clabel{\assign{x}{\query(\chi[a]) }}^{3}  ; 
         \clabel{\assign{a}{x + a}}^{4} ;
        \clabel{\assign{j}{j-1}}^{5}       \Big)
    \end{array}
\]
The Evaluation Result: $ \progA(\kw{whileRec}(k)) = 1 + k$
   \end{example}
    %
            \begin{example}[While with Multi-Path Query Dependency]
   \label{ex:whileM}
   %
   This example algorithm contains a simple while loop and a $\eif$ command in the loop body.
% There is one query requests in the loop body at command $3$.
Each branch  has a query request (in the commands $5$ and $6$)
depend on the query at command $1$ and the query at command $7$.
Among the $\frac{k}{2}$ iterations,
% the query at command $7$ depend on the query at line $5$, otherwise not.
 result from previous iteration.
The longest dependency depth, i.e., the adaptivity is expectation to be $1 +2 * \lfloor \frac{k}{2} \rfloor$.
            %
            \[
            %
            \kw{whileM}(k) \triangleq 
            \begin{array}{l}
   \clabel{ \assign{j}{k}}^{0} ; 
   \clabel{ \assign{x}{\query(\chi[0])} }^{1} ; \\
\ewhile ~ \clabel{j > 0}^{2} ~ \edo ~ \\
\Big(
 \clabel{\assign{j}{j-1}}^{3} ;\\
 \eif(\clabel{j \% 2 == 0}^{4}, 
 \clabel{\assign{y}{\chi[x]}}^{5}, 
 \clabel{\assign{w}{\chi[x]}}^{6});\\        
 \clabel{\assign{x}{\query(\chi(\ln(y)))} }^{7} \Big)
   \end{array}
            \]
            The Evaluation Result: $ \progA(\kw{whileM}(k)) = 1 +2 * \lfloor \frac{k}{2} \rfloor $
        \end{example}
    %
            \begin{example}[While with Query Dependency through Related Variables]
   \label{ex:whileRV}
   This example algorithm contains a simple while loop
    and a sequence of three query requests in the loop body.
% There is one query requests in the loop body at command $3$.
In each iteration, every query request depend on one or more
query results from previous iteration.
% the query at command $7$ depend on the query at line $5$, otherwise not.
The longest dependency depth, i.e., the adaptivity is expectation to be $1 +2 * k$.
   \[
   %
   \kw{whileRV}(k) \triangleq 
   \begin{array}{l}
   \clabel{\assign{j}{k} }^{0} ; 
   \clabel{ \assign{x}{\query(\chi[0])}}^{1} ; 
\clabel{ \assign{y}{\query(\chi[1])}}^{2} ; \\
    \ewhile ~ \clabel{j > 0}^{3} ~ \edo ~ \\
    \Big(
     \clabel{\assign{j}{j-1}}^{4} ;
     \clabel{\assign{z}{\query(\chi(x + \ln(y)))} }^{5}  ; 
     \clabel{ \assign{x}{\query(\chi[z])}}^{6} ; 
     \clabel{ \assign{y}{\query(\chi[z])}}^{7} 
    \Big)
\end{array}
   \]
   The Evaluation Result: $ \progA(\kw{whileRV}(k)) = 1 + 2 * k $
            \end{example}
   %
   %
   \begin{example}[While with Query Dependency trhough Control Flow and Data Flow]
\label{ex:whileVCD}
%
This example algorithm contains a simple while loop
and a sequence of three query requests in the loop body.
The variable in the guard is assigned by a query request in command $0$.
In each iteration, the query at $3$ depends on either the query at line $1$, and the query result at line $4$ from the previous iteration.
%  in the branches depend on the second query in command $1$
In each iteration, the query at $4$ depends on either the query at line $0$ and the query at line $3$ in the same iteration.
% Even though the variable $x$ isn't used in the query expression in the query $3$ and $4$,
The longest dependency depth, i.e., the adaptivity is expectation to be $1 +2 * k$.
\[
\kw{whileVCD}() \triangleq
\begin{array}{l}
    \clabel{ \assign{x}{\query(\chi[0])} }^{0} ; 
    \clabel{ \assign{z}{\query(\chi[0])} }^{1} ; \\
        \ewhile ~ \clabel{x > 0}^{2} ~ \edo ~ \\
        \Big(
        \clabel{\assign{x}{\query(\chi(z))} }^{3}  ; 
        \clabel{\assign{z}{\query(\chi(x))}}^{4}
      \Big)
    \end{array}
\]
The Evaluation Result: $ \progA(\kw{whileVCD}(k)) = 1 + 2 * k $
   \end{example}
    %
   \begin{example}[While with Multiple Path Query Dependency Dependency]
\label{ex:whileMPVCD}
%
This example algorithm contains a simple while loop and a $\eif$ command in the loop body.
% There is one query requests in the loop body at command $3$.
Each branch  has a query request (in the commands $5$ and $6$)
depend on either the query at command $1$ or the query at command $7$.
% Among the $\frac{k}{2}$ iterations,
% % the query at command $7$ depend on the query at line $5$, otherwise not.
%  result from previous iteration.
The longest dependency depth, i.e., the adaptivity is expectation to be $2 + k$.
\[
    %
    \kw{whileMPVCD}(k) \triangleq
    \begin{array}{l}
        \clabel{ \assign{x}{\query(k)}}^{0} ; 
        \clabel{\assign{y}{0} }^{1} ; 
            \ewhile ~ \clabel{x > 0}^{2} ~ \edo ~ \\
            \Big(
             \eif(\clabel{y > 0}^{3}, 
             \clabel{\assign{y}{\query(\chi[12])}}^{4}, 
             \clabel{\assign{w}{\query(\chi[9])}}^{5});        
             \\
             \clabel{\assign{x}{x-1}}^{6}\Big);\\
             \clabel{\assign{y}{\query(\chi(\ln(y)))} }^{7} 
        \end{array}
    \]
    The Evaluation Result: $ \progA(\kw{whileMPVCD}(k)) = 2 + k $
\end{example}
   %
\begin{example}[Nested While with Nested Query Dependency]
    \label{ex:nestWhileVD}
    %
    This example algorithm contains two nested while loops.
    The query in the outer loop at line $5$ depends on either the query at line $1$ or
    the query results at line $8$ from the previous iteration of the inner loop.
    The longest dependency depth, i.e., the adaptivity is expectation to be $2 + k^2$.
        %
    \[
    %
    \kw{nestWhileVD}(k) \triangleq 
    \begin{array}{l}
        \clabel{ \assign{i}{k} }^{0} ; 
        \clabel{\assign{x}{\query(\chi[0])}}^{1} ; \\
            \ewhile ~ \clabel{i > 0}^{2} ~ \edo ~ 
            \Big(
             \clabel{\assign{i}{i-1}}^{3} ;
             \clabel{\assign{j}{k}}^{4} ;
             \clabel{\assign{y}{\query(\chi(\ln(x)))} }^{5}  ; \\
             \ewhile ~ \clabel{j > 0}^{6} ~ \edo ~ 
             \Big(
              \clabel{\assign{j}{j-1}}^{7};
              \clabel{\assign{x}{\query(\chi(\ln(x)))} }^{8}
              \Big) \Big)
        \end{array}
    \]
    The Evaluation Result: $ \progA(\kw{nestWhileVD}(k)) = 2 + k^2 $
\end{example}
    
    \begin{example}[Nested While with Query Dependency through Related Variables]
        \label{ex:nestedWhileRV}
        %
        This example algorithm contains two nested while loops, one query in the outer loop, and one query in the inner loop.
        The query in the outer loop at line $8$ depends on only the query result at line $7$
        from the last iteration of the inner loop.
        %  either the query at line $1$ or
        However, the query at line $7$ depends on  either the query at line $1$ 
        the query results at line $8$ from the previous iteration.
        The longest dependency depth, i.e., the adaptivity is expectation to be $1 + 2 * k $.
            %
        \[
        %
            \kw{nestWhileRV}(k) \triangleq 
        \begin{array}{l}
            \clabel{ \assign{i}{k} }^{0} ; 
            \clabel{\assign{x}{\query(\chi[0])}}^{1} ; \\
   \ewhile ~ \clabel{i > 0}^{2} ~ \edo ~ 
   \Big(
    \clabel{\assign{i}{i-1}}^{3} ;
    \clabel{\assign{j}{k}}^{4} ;\\
    \ewhile ~ \clabel{j > 0}^{5} ~ \edo ~ 
    \Big(
     \clabel{\assign{j}{j-1}}^{6};
     \clabel{\assign{y}{\query(\chi(x) + \chi(1))} }^{7}
     \Big); \\
    \clabel{\assign{x}{\query(\chi(\ln(y)))} }^{8}
     \Big)
            \end{array}
        \]
        The Evaluation Result: $ \progA(\kw{nestWhileRV}(k)) = 1 + 2 * k $
    \end{example}
%
   
        \begin{example}[Nested While with Nest Query Dependency and Related Variable Accross Outer and Inner Loop]
            \label{ex:nestedWhileMR}
            %
            This example algorithm contains two nested while loops, one query in the outer loop, and one query in the inner loop as well.
            The two queries depend on both the query results assigned to themselves in previous iteration.
            The longest dependency depth, i.e., the adaptivity is expectation to be $1 + k + k^2 $.
            \[
            %
            \kw{nestWhileMR}(k) \triangleq 
            \begin{array}{l}
                \clabel{\assign{i}{k} }^{0} ; 
                \clabel{ \assign{x}{\query(\chi[0])}}^{1} ; 
                \clabel{ \assign{y}{\query(\chi[1])}}^{2} ; 
                \ewhile ~ \clabel{i > 0}^{3} ~ \edo ~ \\
                \Big(
                \clabel{\assign{i}{i-1}}^{4} ;
                \clabel{\assign{j}{k}}^{5} ;
                \clabel{\assign{y}{\query(\chi(\ln(x) + y))} }^{6}  ; \\
                \ewhile ~ \clabel{j > 0}^{7} ~ \edo ~ 
                \Big(
                \clabel{\assign{j}{j-1}}^{8};
                \clabel{\assign{x}{\query(\chi(\ln(y))+\chi[x])} }^{9}
                \Big) \Big)
            \end{array}
            \]
            The Evaluation Result: 
            $ \progA(\kw{nestWhileMR}(k)) = 1 + k + k^2$
            \\
            Reachability Bound The Evaluation Result: \\
            weight for Variable: j of label 6 is: 0 + 0 + 1 * k * k\\
            weight for Variable: y of label 7 is: 0 + 0 + 1 * k * k\\
            weight for Variable: j of label 4 is: 0 + 1 * k\\
            weight for Variable: i of label 3 is: 0 + 1 * k\\
            weight for Variable: x of label 8 is: 0 + 1 * k\\
            weight for Variable: x of label 1 is: 1\\
            weight for Variable: i of label 0 is: 1\\
            \end{example}
            \begin{example}[Nested While with MultiplePath and Nested Recursive Multiple Variable 
   Data-Value Dependency Across Outer and Inner Loop]
   \label{ex:nestedWhileMPRV}
   %
   We then show a more complex example with nested while command and nested data-flow across the outer and inner while loop through multiple variables.
   This example also contains the if command with data dependency occurred through the if guard.
   The longest dependency depth, i.e., the adaptivity is expectation to be $1 + k + k^2 $.
   %
   \[
   %
   \kw{nestWhileMPRV}(k) \triangleq 
   \begin{array}{l}
\clabel{\assign{i}{k} }^{0} ; 
\clabel{ \assign{x}{\query(\chi[0])}}^{1} ; 
\clabel{ \assign{y}{\query(\chi[1])}}^{2} ; \\
    \ewhile ~ \clabel{i > 0}^{3} ~ \edo ~ 
    \Big(
     \clabel{\assign{i}{i-1}}^{4} ;
     \clabel{\assign{j}{k}}^{5} ;\\
     \eif(\clabel{x > 0}^6, \clabel{\assign{y}{\query(\chi(\ln(x) + y))} }^{7},
     \clabel{\assign{y}{\query(\chi(x))} }^{8} )
      ; \\
     \ewhile ~ \clabel{j > 0}^{9} ~ \edo ~ 
     \Big(
      \clabel{\assign{j}{j-1}}^{10};
      \clabel{\assign{x}{\query(\chi(\ln(y))+\chi[x])} }^{11}
      \Big) \Big)
\end{array}
   \]
   \end{example}
   The Evaluation Result: $ \progA(\kw{nestWhileMPRV}(k)) = 1 + k + k^2$
   \\
   Reachability Bound The Evaluation Result: \\
            weight for Variable: j of label 10 is: 0 + 0 + 1 * k * k \\
   weight for Variable: x of label 11 is: 0 + 0 + 1 * k * k \\
   weight for Variable: y of label 7 is: 0 + 1 * k \\
   weight for Variable: y of label 8 is: 0 + 1 * k \\
   weight for Variable: j of label 5 is: 0 + 1 * k \\
   weight for Variable: i of label 4 is: 0 + 1 * k \\
   weight for Variable: y of label 2 is: 1 \\
   weight for Variable: x of label 1 is: 1 \\
   weight for Variable: i of label 0 is: 1 \\