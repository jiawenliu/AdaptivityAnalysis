\subsection{Definitions for Adaptivity Based on Program Analysis in \THESYSTEM}
In order to give a bound on the program's adaptivity, we first build a
program-based data-dependency graph to {over-}approximate the
trace-based dependency graph.  Then, we define a program-based
adaptivity over this approximated graph, as an upper bound for
$A(c)$.

This program-based graph has a similar topology structure as the one
of the trace-based (semantic) dependency graph. It has the same
vertices and query labels, and approximated edges and weights.  An
approximated edge correspond to a program-based data dependency
relation ($\flowsto$ in Definition~\ref{def:flowsto}) and an approximated
weight corresponds to a reachability bound analysis results from
Definition~\ref{def:transition_closure}.

%
\subsubsection{Program-Based Variable Dependency}
The program-based dependency relation over two labeled variables ($x^i, y^j)$ is defined as a $\flowsto$ relation with respect to the program $c$ as follows.
%
\begin{defn}[Data Flow Relation between Assigned Variables ($\flowsto$)].
\label{def:flowsto}
\\
Given a program  ${c}$,
a variable ${x^i}  \in \lvar_c $ is in the \emph{flows to} relation with another variable ${y^j} \in \lvar_c$, if and only if:
\mg{I cannot even parse the next formula. Why there is a big disjunction on the left? Disjunction is a binary operation, or n-ary if given a set, what is this disjunction between?}\\
\mg{please, remove the underscript $c$ in the exists. It just makes everything mroe difficult to parse.}\\
\mg{The use of $\lor$ is odd. E.g. $\exists {(\expr \lor \qexpr)}$ or
  $ [{\assign{y}{\expr \lor \query(\qexpr)}}]^{j} $. I suggest to write the whole formula instead of using weird shortenings.}\\
\mg{Also, now it is too late to change this but instead of breaking down the definition using the subterm relation and then defining the flowto relation, it would have been better to give just one inductive definition of Flowto - I imagine that this makes also the proof more awkward.}
%
{\footnotesize 
\[
\begin{array}{l}
\flowsto({x^i, y^j, c}) \triangleq 
\\
\left( \bigvee
\begin{array}{l}
(\exists \expr \st \clabel{\assign{y}{\expr}}^j \in_{c} {c} 
\land {x} \in FV(\expr) \land (x^i \in \live(j, c)))
\\
(\exists {\qexpr} \st [\assign{y}{\query({\qexpr})}]^j \in_{c} {c} 
\land x \in FV({\qexpr}) \land (x^i \in \live(j,c))))
\\
\Big(\exists {c_w} \in \cdom, l \in \mathcal{L}, \bexpr \st
	\ewhile [\bexpr]^l \edo {c_w} \in_{c} {c}
	\land \flowsto(x^i, y^j, c_w)
	\\ \qquad	
     \lor 
	\big( \exists {(\expr \lor \qexpr)} \st
	[{\assign{y}{\expr \lor \query(\qexpr)}}]^{j} \in_{c}  {c_w}  \land {x} \in FV(\bexpr) \land x^i \in \live(l, c)
	\big)
	\Big)
\\
\Big(
\exists {c_1}, {c_2} \in \cdom, l \in \mathcal{L}, \bexpr 
\st 
	\eif([\bexpr]^l, {c_1}, {c_2}) \in_{c} {c} \land
	\flowsto(x^i, y^j, c_1) \lor \flowsto(x^i, y^j, c_2)
	\\ \qquad 
	\lor 
	\big( \exists {(\expr \lor \qexpr)} \st
	\land {x} \in FV(\bexpr) \land x^i \in \live(l, c) \land
	([{\assign{y}{\expr \lor \query(\qexpr)}}]^{j} \in_{c}  {c_1}  
	\lor [{\assign{y}{\expr \lor \query(\qexpr)}}]^{j} \in_{c}  {c_2})
	\big)
\Big)
% \\
\end{array}
\right).
\end{array}
\]
}
%
\end{defn}
%
\mg{The next notation is inconsistent with the one used above. Also, this definition should be given before the definition of flowto. From the description I have no cluse what this notion of reachability means. Also, the definition is referred to does not define this notation.}\\
\mg{the definition somehow seems to make sense but until when the or notation is fixed and I don't see the definition of RD, I cannot tell for sure.}
$\live^l(c) \subseteq \lvar_c$,
which is the set of all the reachable variables at location of label $l$ in the program $c$.
For every labelled variable $x^l$ in this set, 
the value assigned to that variable
in the assignment command associated to that label is reachable at the entry point of  executing the command of label $l$.
This is formally defined , formally computed in Definition~\ref{def:feasible_flowsto}
\\
\mg{This description seems inconsistent with the definition. I suggest to use the same variables and terms.}
To understand the $\flowsto$ intuition, 
given a program  ${c}$ with its labelled variables $\lvar_c$, and two variables ${x^i}, y^j  \in \lvar_c $ 
% showing up as $i$-th, $j$-th elements in $\lvar$ 
% (i.e., ${x} = \lvar(i)$ and ${y} = \lvar(j)$),
we say $y^j$ flows to ${x^i}$ in ${c}$ if and only if 
the value of $y^j$ directly or indirectly influence the evaluation of the value of ${x}$ as follows:
%
\begin{itemize}
\item (Explicit Influence) The program ${c}$ contains either 
a command $[\assign{{x}}{\aexpr}]^i$ or $[\assign{{x}}{\query({\qexpr})}]^i$,
such that ${y}$ shows up as a free variable in $\expr$ or ${\qexpr}$.
We use $\flowsto({x^i, y^j, c})$ to denote $y^j$ flows to $x^i$ in ${c}$.
%
\item (Implicit Influence) The program ${c}$ contains either a while loop
command
or if command, 
such that $x$ shows up in the guard
and $y$ shows up in the left hand of an assignment command and this assignment command showing up
 in the body of the while loop, or branches of if command.
\end{itemize}
%
% This is formally defined in \ref{def:flowsto}.
% We use $FV(\expr)$, $FV(\sbexpr)$ and $FV(\qexpr)$ denote the set of free variables in 
% expression $\expr$, boolean expression $\sbexpr$ and query expression $\qexpr$ respectively.
%
%
\mg{I don't understand what this definition of equivalence means. It is not observational equivalence
and it is not syntactic equivalence. What are we trying to capture here? Also, it is equivalence of programs, not of program.}
\begin{defn}[Equivalence of Program]
%
\label{def:aq_prog}
Given 2 programs $c_1$ and $c_2$:
\[
c_1 =_{c} c_2
\triangleq 
\left\{
  \begin{array}{ll} 
    \etrue        
    & c_1 = \eskip \land c_2 = \eskip
    \\ 
    \forall \trace \in \mathcal{T} \st \exists v \in \mathcal{VAL}
    \st \config{ \trace, \expr_1} \aarrow v \land \config{ \trace, \expr_1} \aarrow v     
    & c_1 = \assign{x}{\expr_1} \land c_2 = \assign{x}{\expr_2} 
    \\ 
    \qexpr_1 =_{q} \qexpr_2       
    & c_1 = \assign{x}{\query(\qexpr_1)} \land c_1 = \assign{x}{\query(\qexpr_2)} 
    \\
    c_1^f =_{c} c_2^f \land c_1^t =_{c} c_2^t
    & c_1 = \eif(b, c_1^t, c_1^f) \land c_2 = \eif(b, c_2^t, c_2^f)
    \\ 
    c_1' =_{c} c_2'         
    & c_1 = \ewhile b \edo c_1' \land c_2 = \ewhile b \edo c_2'
    \\ 
    c_1^h =_{c} c_2^h \land c_1^t =_{c} c_2^t
    & c_1 = c_1^h;c_1^t \land c_2 = c_2^h;c_2^t 
  \end{array}
  \right.
\]
%
As usual, we denote by $c_1 \neq_{c} c_2$ the negation of the equivalence.
%
\end{defn}
%
\mg{This definition needs to go before it is used. }
Given 2 programs $c$ and $c'$, we denote by $c' \in_{c} c$  that $c'$ is a sub-program of $c$ defined as follows,
\begin{equation}
c' \in_{c} c \triangleq \exists c_1, c_2, c''. ~ s.t.,~
c =_{c} c_1; c''; c_2 \land c' =_{c} c''
\end{equation} 
%

\subsubsection{Program Analysis Based Dependency Graph}
We give the formal definition for the program-based dependency graph for a program $c$, 
$\progG({c}) = (\vertxs, \edges, \weights, \flag)$ as follows.
\begin{defn}
    [Program-Based Dependency Graph].
    \label{def:prog_graph}
    \\
Given a program ${c}$
its program-based graph 
$\progG({c}) = (\vertxs, \edges, \weights, \qflag)$ is defined as:
{\footnotesize
\[
\begin{array}{rlcl}
\text{Vertices} &
\vertxs & := & \left\{ 
x^l \in \mathcal{LV} 
~ \middle\vert ~
x^l \in \lvar_{c}
\right\}
\\
\text{Directed Edges} &
\edges & := & 
\left\{ 
  ({x}_1^{i}, {x}_2^{j}) \in \mathcal{LV} \times \mathcal{LV}
  ~ \middle\vert ~
  \begin{array}{l}
    {x}_1^{i}, {x}_2^{j} \in \vertxs
	\land
    % \\
    \exists n \in \mathbb{N}, z_1^{r_1}, \cdots, z_n^{r_n} \in \lvar_{{c}} \st 
    n \geq 0 \land
    \\
    \flowsto(x^i,  z_1^{r_1}, c) 
    \land \cdots \land \flowsto(z_n^{r_n}, y^j, c) 
  \end{array}
\right\}
\\
\text{Weights} &
\weights & := &
% \bigcup
% \begin{array}{l}
	\left\{ (x^l, w) \in  \mathcal{LV} \times EXPR(\constdom)
	\mid
	x^l \in \lvar_{{c}} \land w = \absW(l)
	\right\}
% \end{array} 
\\
\text{Query Annotation} &
\qflag & := & 
\left\{(x^l, n)  \in  \mathcal{LV} \times \{0, 1\} 
~ \middle\vert ~
 x^l \in \lvar_{c},
n = 1 \iff x^l \in \qvar_{c} \land n = 0 \iff  x^l \in \qvar_{c} .
\right\}
\end{array}
\] 
}
, where the $\absW(l)$ is the symbolic reachability bound in domain of $EXPR(\constdom)$,
% for the assignment command of label $l$ to which  
the labeled variable $x^l$, 
% is associated, 
computed from the $\THESYSTEM$ algorithm 
in Definition~\ref{def:transition_closure}.
The $EXPR(\constdom)$ is an expression over symbolic constants containing the
input variables and natural number.
\end{defn} 
%
\paragraph{Program-Based Adaptivity ($\progA(c)$)}
%
Given a program ${c}$, we generate its program-based graph 
$\progG({c}) = (\vertxs, \edges, \weights, \qflag)$.
%
Then the adaptivity bound based on program analysis for ${c}$ 
% is the number of query vertices on a finite walk in $\progG({c})$. This finite walk satisfies:
% \begin{itemize}
% \item the number of query vertices on this walk is maximum
% \item the visiting times of each vertex $v$ on this walk is bound by its reachability bound $\weights(v)$.
% \end{itemize}
is computed as the maximum query length over all finite walks in $\walks(\progG({c}))$,
%
% It is formally defined in \ref{def:prog_adapt}.
defined formally as follows.
%
%
\begin{defn}
[{Program-Based Adaptivity}].
\label{def:prog_adapt}
\\
{
Given a program ${c}$ and its program-based graph 
$\progG({c}) = (\vertxs, \edges, \weights, \qflag)$,
%
the program-based adaptivity for $c$ is defined as%
\[
\progA({c}) 
:= \max
\left\{ \qlen(k)\ \mid \  k\in \walks(\progG({c}))\right \}.
\]
}
\end{defn}  
%
%
% {
% \begin{defn}[Variable Flags ($\flag$)].
% \\
% Given a program  ${c}$ with its labelled variables $\lvar$, the $\flag$ is a vector of the same length as $\lvar$, s.t. for each variable ${x}$ showing up as the $i$-th element in $\lvar$ (i.e., ${x} = \lvar(i)$), 
% $\flag(i) \in \{0, 1, 2\}$ is defined as follows:
% %
% %
% \[
% \flag(i) := 
% \left\{
% \begin{array}{ll}
% 2 & 
% {x^l} \in \lvar_{c} \land 
% (\exists {\qexpr}. ~ s.t., ~
% [\assign{{x}}{\query({\qexpr})}]^l \in_{c} {c})
% \\
% 1 &  
% \begin{array}{l}
% {x^l} \in \lvar_{c} \bigwedge \\
% \left(
% \begin{array}{l}
% \big(\exists  ~ {c'}, {\expr}, \sbexpr, l, l'. ~
% 	\ewhile [\sbexpr]^l \edo {c'} \in_{c} {c}
% 	\land 
% 	[{\assign{x}{\expr}}]^{l'} \in_{c}  {c'}
% \big) \bigvee
% \\
% \big(\exists ~ \sbexpr, l, l_1, l_2, {c_1}, {c_2}, {\expr}_1, {\expr}_2. ~
% 	\eif([\sbexpr]^l, {c_1}, {c_2}) \in_{c} {c} \land
% 	([{\assign{x}{\expr_1}}]^{l1} \in_{c} {c_1} \lor 
% 	[{\assign{x}{\expr_2}}]^{l2} \in_{c} {c_2})
% \big)
% \end{array}
% \right)
% \end{array}
% \\
% 0 & \text{o.w.}
% \end{array}
% \right\}. 
% \] 
% %
% \end{defn}
%
% Operations on $\flag$ are defined as follows:
% \begin{equation}
% \begin{array}{llll}
% {\flag_1 \uplus \flag_2}(i) & := &
% \left\{
% \begin{array}{ll}
% k & k = \max{\big\{\flag_1(i), \flag_2(i)\big\}} 
% \land |\flag_1| = |\flag_2|\\
% 0 & o.w.
% \end{array}\right.
% & i = 1, \cdots, |\flag_1|  
% \\
% {\flag \uplus n}(i) & := & 
% \max\big\{ \flag(i), n \big\} 
% & i = 1, \ldots, |\flag|    
% \\
% \left[ n \right]^k (i) & := &  n
% & i = 1, \ldots, k ~ \land ~ |\left[ n \right]^k| = k
% \end{array}
% \end{equation}
%
%
%
% \begin{defn}[Data Flow Matrix ($\Mtrix$)]
% The data flow matrix $\Mtrix$ of a program $c$ is a matrix of size $|\lvar_c| \times |\lvar_c|$ 
% s.t.,
% %
% \[
% \Mtrix(i, j) \triangleq
% \left\{
% \begin{array}{ll}
% 1	&	\flowsto({x^i, y^j, c}) \\
% 0	& o.w.
% \end{array}
% \right., {x^i}, y^j  \in \lvar_c.
% \]
% %
% \end{defn}
% %
% Operations on the data flow matrices are defined as follows:
% %
% \begin{equation}
% \Mtrix_1 ; \Mtrix_2 
% := \Mtrix_2 \cdot \Mtrix_1 + \Mtrix_1 + \Mtrix_2
% \end{equation}
% %
% Consider the same program $c$ as above, its data flow matrix $\Mtrix$ and $\flag$ for the program $c$ is as follows:
% $$
% {c} = 
% \begin{array}{l}
% \left[{\assign {x_1} {\query(0)}}	\right]^1;
% \\
% \left[{\assign {x_2} {x_1 + 1}}		\right]^2;
% \\
% \left[{\assign {x_3} {x_2 + 2}}		\right]^3
% \end{array}
% ~~~~~~~~~~~~
% \Mtrix
% =  \left[ 
% \begin{matrix}
% 0 & 0 & 0 \\
% 1 & 0 & 0 \\
% 1 & 1 & 0 \\
% \end{matrix} \right] ~ , 
% \flag = \left [ \begin{matrix}
% 1 \\
% 0 \\
% 0 \\
% \end{matrix} \right ]
% $$
% %
% % There are two special matrices used for generating the data flow matrix $\Mtrix$ in the analysis algorithm. They are the left matrix $\lMtrix_i$ and right matrix $\mathsf{R_{(e, i)}}$.

% % Given a program  ${c}$ with its labelled variables $\lvar$ of length $N$,
% % the left matrix $\lMtrix_i$ generates a matrix of $1$ column, $N$ rows, 
% % where the $i$-th row is $1$ and all the other rows are $0$.
% % %
% % \begin{defn}[Left Matrix ($\lMtrix_i$)].
% % \\
% % Given a program  ${c}$ with its labelled variables $\lvar$ of size $N$, 
% % the left matrix $\lMtrix_i$ is defined as follows:
% % \[
% % \lMtrix_i(j) : = 
% % \left
% % \{
% % \begin{array}{ll}
% % 1 & j = i \\
% % 0 & o.w.
% % \end{array}
% % \right.,
% % j = 1, \ldots, N.
% % \]
% % \end{defn}
% % %
% % Given a program  ${c}$ with its labelled variables $\lvar$ of length $N$,
% % the right matrix $\rMtrix_{\expr, i}$ generates a matrix of one row and $N$ columns, 
% % where the locations of free variables in $\expr$ is marked as $1$. 
% % %
% % %
% % \begin{defn}[Right Matrix ($\rMtrix_{\expr}$)].
% % \\
% % Given a program  ${c}$ with its labelled variables $\lvar$ of length $N$, 
% % the right matrix $\rMtrix_{\expr}$ is defined as follows:
% % \[
% % \rMtrix_{\expr}(j) : = 
% % \left\{
% % \begin{array}{ll}
% % 1 & {x} \in FV(\expr) 
% % \\
% % 0 & o.w.
% % \end{array}
% % \right.,
% % {x} = \lvar(j) ~ , ~ j = 1, \ldots, N.
% % \]
% % %
% % %
% % \end{defn}
% % %
% % Using the same example program ${c}$ as above with labelled variables $\lvar = [ {x_1 , x_2 , x_3} ] $,
% % the left and right matrices w.r.t. its $2$-nd command 
% % $\left[{\assign {x_2} {x_1 + 1}}\right]^2$  are as follows:
% % \[
% % \lMtrix_1 = \left[ \begin{matrix}
% % 0   \\
% % 1 	 \\
% % 0   \\
% % \end{matrix}   \right ] 
% % ~~~~~~~~~~~~~~
% % \rMtrix_{{x}_1 + 1}
% % = \left[ \begin{matrix} 
% % 1 & 0 & 0 \\
% % \end{matrix}  \right]
% % \]
% %