\subsection{Syntax}
The language we consider is a typed lambda calculus with expressions
for basic data types, recursion, and special constructions to execute
queries. The grammar is defined as follow:

\begin{table}
\[\begin{array}{llll}
  \mbox{Types} & \type & ::= & \tbool ~|~  \treal ~|~  \tint ~|~   \tint[I] ~|~ \type_1 \times
  \type_2 ~|~ \tarr{\type_1}{\type_2}{\nnatbiA}{\dmap}{\nnatA} ~|~
                              \tlist{\type} ~|~ \tbox{\type}~|~ \\
& & &                      \tforall{\dmap}{\nnatA}{i} \type  ~|~ \texists{i} \type\\
\mbox{Index Term} & \idx, \nnatA & ::= &     i ~|~ n ~|~ \idx_1 + \idx_2 ~|~  \idx_1
                                 - \idx_2 ~|~ \smax{\idx_1}{\idx_2}\\
\mbox{Expr.} & \expr & ::= & x ~|~ \econst ~|~ \distr ~|~ \eop(\expr) ~|~ \expr_1 \eapp \expr_2 ~|~ \efix f(x:\tau).\expr
 ~|~ (\expr_1, \expr_2) ~|~ \eprojl(\expr) ~|~ \eprojr(\expr) ~| \\
%
& & & \etrue ~|~ \efalse ~|~ \eif(\expr_1, \expr_2, \expr_3) ~|~ \eilam \expr  ~|~  \expr \eapp []  ~|~
                            \epack \expr ~|~ \eunpack \expr \eas x
                            \ein \expr ~|~
\\
& & &  \enil ~|~  \econs (
      \expr_1, \expr_2) ~|~ 
      \ecase \expr \eof\, []\mapsto\expr~;\econs(x,xs)\mapsto\expr\\
& & &  \elet  x:q = \expr_1 \ein \expr_2\\
%
\mbox{Distr.} & \distr & ::= & \bernoulli \eapp \expr ~|~ \uniform \eapp \expr_1 \eapp \expr_2\\      
\mbox{Values} & \valr & ::= & \etrue ~|~ \efalse ~|~ \econst ~|~
(\efix f(x).\expr, \env) ~|~ (\valr_1, \valr_2) 
    ~|~ \enil ~|~ \econs (\valr_1, \valr_2)\\
\mbox{Environments} & \env & ::= & x_1 \mapsto \valr_1, \ldots, x_n \mapsto \valr_n\\
\end{array}\]
\caption{Syntax of \THESYSTEM.}
\label{tab:syntax}
\end{table}
where by $\econst$ we denote arbitrary constants, $\eop$ represents a primitive operation (such as a
mechanism), which determines adaptivity. For simplicity, we assume
that $\eop$ can only have type $\tbase \to \tbool$. We make
environments explicit in closures. This is needed for the tracing
semantics later.
\subsection{Tracing Operational Semantics}
