


\[
\begin{array}{llll}
 % \mbox{Index Term} & \idx, \nnatA & ::= &     i ~|~ n ~|~ \idx_1 + \idx_2 ~|~  \idx_1
 %                                  - \idx_2 ~|~ \smax{\idx_1}{\idx_2}\\
%                                  \mbox{Sort} & S & ::= & \nat \\
  \mbox{Linear type} & \ltype &::=  &  \type \lto \type ~|~ \tbase ~|~
  \tbool\\
  \mbox{Type} & \type & ::= & \bang{\idx} \ltype   \\
\end{array}
\]

\begin{figure}
  \begin{mathpar}
    \inferrule{
    }{
      \ictx \tctx , x: \bang{\nnatA} \ltype \tvdash{\nnatA} x: \bang{\nnatA}\ltype
    }~\textbf{Ax}
    %
    \and
    %
    \inferrule{
    }{
      \ictx \Gamma \tvdash{\nnatA} c : \bang{\nnatA}\tbase
    }~\textbf{b}
    %
    \and
    %
    \inferrule{
    }{
      \ictx \Gamma \tvdash{\nnatA} \etrue : \bang{\nnatA}\tbool
    }~\textbf{true}
    \and
    %
    \inferrule{
      \ictx \Gamma, x: \type_1
      \tvdash{k}
      \expr: \type_2
    }{
      \ictx r+\Gamma \tvdash{k+r} \lambda x. \expr : \bang{r}  ( \type_1
      \lto \type_2)
    }~\textbf{lambda}
    %
    \and
    %
    \inferrule{
      \ictx \Gamma_1  \tvdash{\nnatA_1} \expr_1:  \bang{0} ( \type_1
      \lto \type_2      )\\
      \ictx \Gamma_2 \tvdash{\nnatA_2} \expr_2: \type_1 
    }{
      \ictx \max (\Gamma_1, \Gamma_2 ) \tvdash{\max( \nnatA_1,\nnatA_2) } \expr_1 \eapp \expr_2 : \type_2
    }~\textbf{app}
    %
    \and
    %
    \inferrule{
      \ictx \Gamma \tvdash{\nnatA} \expr: \bang{k} \ltype 
    }{
      \ictx \Gamma' ,1+\Gamma  \tvdash{1+\nnatA} \delta(\expr): \bang{k} \ltype 
    }~\textbf{delta}
    %
    \and
    %
    \inferrule{
      \ictx \Gamma_1 \tvdash{\nnatA} \expr: \type_1 \\
      \ictx \Gamma_2, x: \type_1 \tvdash{\nnatA'} \expr' : \type
    }{
      \ictx \max(\Gamma_1,\Gamma_2) \tvdash{\max(\nnatA, \nnatA') } \elet x =  \expr \ein \expr': \type
    }~\textbf{let}
    %
    \and
    %
    \inferrule{
      \ictx \Gamma_1 \tvdash{\nnatA_1} \expr_1 : \bang{k} \tbool\\
      \ictx \Gamma_2 \tvdash{\nnatA_2} \expr_2 : \type\\
      \ictx \Gamma_2 \tvdash{\nnatA_2} \expr_3 : \type
    }{
      \ictx \boxed{\max(\Gamma_1, \nnatA_1+ \Gamma_2) } \tvdash{\nnatA_1 + \nnatA_2} \eif \eapp \expr_1 \eapp \expr_2
      \eapp \expr_3 : \type
    }~\textbf{if}
  \end{mathpar}
  \caption{Typing rules, part 1}
  \label{fig:type-rules}
\end{figure}

\begin{thm}[Context Raising]
  If $ \Gamma \tvdash{ \nnatA} \expr : \type $ and $r \in \mathbb{N}$,
  then $ r+\Gamma \tvdash{ r+ \nnatA} \expr : r+ \type $.
\end{thm}

\begin{thm}[Context Weaking]
  If $ \Gamma \tvdash{ \nnatA} \expr : \type $,  $ \Gamma,x : \type' \tvdash{ \nnatA} \expr : \type $.
\end{thm}

\begin{thm}[Context Strengthening]
  If $ \Gamma,x : \type' \tvdash{ \nnatA} \expr : \type $ and $x \not\
  in \fv{\expr} $, then $ \Gamma , x \tvdash{ \nnatA} \expr : \type $.
\end{thm}

\begin{thm}[Context Max]
  If $ \Gamma_1 \tvdash{ \nnatA_1} \expr : \type $ and $ \Gamma_2 \tvdash{ \nnatA_2} \expr : \type $,
  then $\max(\Gamma_1,\Gamma_2) \tvdash{\max( \nnatA_1 , \nnatA_2 )} \expr : \type $.
\end{thm}

\begin{thm}[Adaptivity Monotonicity]
   If $ \Gamma \tvdash{ \nnatA} \expr : \type $ and $\nnatA' >
   \nnatA$, then $ \Gamma \tvdash{ \nnatA'} \expr : \type $.
\end{thm}

\begin{thm}[Substitution]
  \begin{enumerate} 
   \item If $ \Gamma,x : \type' \tvdash{ \nnatA} \expr : \type $ and $
  \Gamma \tvdash{\nnatA'} \valr : \type'  $ and $ x \in \fv{\expr} $,  then  $\Gamma
  \tvdash{\max(\nnatA,\nnatA' )} \expr[\valr/x]  : \type$. 
  \item If $ \Gamma,x : \type' \tvdash{ \nnatA} \expr : \type $ and $
  \Gamma \tvdash{\nnatA'} \valr : \type'  $ and $ x \not \in \fv{\expr} $,  then  $\Gamma
  \tvdash{\nnatA } \expr[\valr/x]  : \type$.
  \end{enumerate}
\end{thm}

\begin{proof}
  By simultaneous induction on the typing derivation.\\
  Proof of Statement (2).\\
  Since we know $x \not \in \fv{\expr}$ so that $\expr[\valr/x]  =
  \expr$. \\
  From the assumption $ \Gamma,x : \type' \tvdash{ \nnatA} \expr :
  \type $, by context strengthening,  we know that $   \Gamma  \tvdash{
    \nnatA} \expr : \type $.\\

  Proof of Statement (1).\\
  
\caseL{
  $  \inferrule{
    }{
      \ictx \tctx , x: \bang{\nnatA} \ltype \tvdash{\nnatA} x: \bang{\nnatA}\ltype
    }~\textbf{Ax}  $
  }
  Assume  $\Gamma \tvdash{\nnatA'} \valr : \bang{\nnatA} \ltype ~(\star) $.\\
  TS:   $\Gamma \tvdash{\max(\nnatA, \nnatA')} x [\valr/x]  :
  \bang{\nnatA} \ltype$.\\
  It is proved from the assumption $(\star)$ by weaking from $\nnatA'$
  to $\max(\nnatA,\nnatA')$.\\

% \caseL{
%   $  \inferrule{
%     }{
%       \ictx \tctx , x: \bang{\nnatA} \ltype, y : \type' \tvdash{\nnatA} x: \bang{\nnatA}\ltype
%     }~\textbf{Ax}  $
%   }
%   Assume  $\Gamma \tvdash{\nnatA'} \valr : \type' ~(\star) $.\\
%   TS:   $\Gamma \tvdash{\max(\nnatA, \nnatA')} x [\valr/y]  :
%   \bang{\nnatA} \ltype$.\\
%   It is proved from the assumption by weaking.\\ 
 

  \caseL{
  $  \inferrule{
      \ictx \Gamma, y: \type',  x: \type_1
      \tvdash{k}
      \expr: \type_2 (\diamond)
    }{
      \ictx r+\Gamma, y: (r+\type') \tvdash{k+r} \lambda x. \expr : \bang{r}  ( \type_1
      \lto \type_2)
    }~\textbf{lambda} $
  }
  Assume  $\Gamma \tvdash{\nnatA'} \valr : \type' ~(\star) $.\\
  By Theorem 2 context raisig, we know: $r+\Gamma \tvdash{\nnatA'+r} \valr : r+\type' ~(\star\star) $\\
  TS:   $r+\Gamma \tvdash{\max(k+r, \nnatA'+r)} \lambda x. \expr [\valr/y]  :
  \bang{r}  ( \type_1 \lto \type_2) $.\\
  We know that $y \in \fv{\expr}$ from the assumption $ y \in \fv{\lambda x.\expr} $.\\
  By IH on $\diamond$ along with $(\star)$, we get:  $\Gamma,x:\type_1 \tvdash{\max(k, \nnatA')}  \expr [\valr/y]  :
  \type_2 ~(1)$.\\
  Using the lambda rule, we conclude: \\
  $r+\Gamma \tvdash{\max(k+r, \nnatA'+r)} \lambda x. \expr [\valr/y]  :
  \bang{r}  ( \type_1 \lto \type_2) $.\\

  
  \caseL{
     $  \inferrule{
      \ictx \Gamma_1 ,x:\type'  \tvdash{\nnatA_1} \expr_1:  \bang{0} ( \type_1
      \lto \type_2   ) ~(\diamond)\\
      \ictx \Gamma_2 ,x:\type' \tvdash{\nnatA_2} \expr_2: \type_1 ~(\clubsuit)
    }{
      \ictx \max (\Gamma_1, \Gamma_2 ),x:\type' \tvdash{\max( \nnatA_1,\nnatA_2) } \expr_1 \eapp \expr_2 : \type_2
    }~\textbf{app} $
  }
  Assume  $\Gamma_1 \tvdash{\nnatA_1'} \valr : \type' ~(1) $.\\
  Assume $\Gamma_2 \tvdash{\nnatA_2'} \valr : \type' ~(2) $.\\
  From Theorem 3, we know : $\max(\Gamma_1,\Gamma_2)
  \tvdash{\max(\nnatA_1', \nnatA_2')} \valr : \type' ~(3) $.\\
   TS:   $\max (\Gamma_1, \Gamma_2 ) \tvdash{ \max(\max(
     \nnatA_1,\nnatA_2) , \max(\nnatA_1', \nnatA_2') ) } (\expr_1
   \eapp \expr_2)[\valr/x] : \type_2$.\\
   There are three situations:
   \begin{enumerate}
   \item $x \in \fv{\expr_1}, x \in \fv{\expr_2}$, \\
     By IH1 on $(\diamond)$ with $(1)$, we get: $ \Gamma_1
\tvdash{\max(\nnatA_1, \nnatA_1' )} \expr_1 [\valr/x]:  \bang{0} ( \type_1
\lto \type_2   )  $.\\
By IH1 on $(\clubsuit)$ and $(2)$, we get: $ \Gamma_2
\tvdash{\max(\nnatA_2, \nnatA_2')}
\expr_2 [\valr/x]: \type_1  $.\\
\item $x \not\in \fv{\expr_1}, x \in \fv{\expr_2}$\\
  By IH2 on $(\diamond)$ with $(1)$, we get: $ \Gamma_1
\tvdash{\nnatA_1} \expr_1 [\valr/x]:  \bang{0} ( \type_1
\lto \type_2 ) $, by Lemma Adaptivity Monotonicity, we know: $\Gamma_1
\tvdash{\max(\nnatA_1, \nnatA_1' )} \expr_1 [\valr/x]:  \bang{0} ( \type_1
\lto \type_2 ) $.\\
By IH1 on $(\clubsuit)$ and $(2)$, we get: $ \Gamma_2
\tvdash{\max(\nnatA_2, \nnatA_2')}
\expr_2 [\valr/x]: \type_1  $.\\
\item $x \in \fv{\expr_1}, x \not\in \fv{\expr_2}$\\
  By IH1 on $(\diamond)$ with $(1)$, we get: $ \Gamma_1
\tvdash{\max(\nnatA_1, \nnatA_1' )} \expr_1 [\valr/x]:  \bang{0} ( \type_1
\lto \type_2   )  $.\\
By IH2 on $(\clubsuit)$ and $(2)$, by Lemma Adaptivity Monotonicty, we get: $ \Gamma_2
\tvdash{\max(\nnatA_2, \nnatA_2')}
\expr_2 [\valr/x]: \type_1  $.\\
     \end{enumerate}
   
By using the rule app, we conclude:\\
  $\max (\Gamma_1, \Gamma_2 ) \tvdash{ \max(\max(
     \nnatA_1,\nnatA_1') , \max(\nnatA_2, \nnatA_2') ) } (\expr_1 \eapp \expr_2)[\valr/x] : \type_2$.\\

  \caseL{
   $   \inferrule{
      \ictx \Gamma , x: \type' \tvdash{\nnatA} \expr: \bang{k} \ltype 
    }{
      \ictx \Gamma' ,1+\Gamma ,x : 1+\type' \tvdash{1+\nnatA} \delta(\expr): \bang{k} \ltype 
    }~\textbf{delta} $
  }
  Assume  $\Gamma  \tvdash{\nnatA'} \valr : \type' ~(1) $.\\
   By Theorem 2 context raisig, we know: $1+\Gamma \tvdash{\nnatA'+1} \valr : 1+\type' ~(2) $\\
   TS:
   $\Gamma' ,1+\Gamma \tvdash{ \max( 1+\nnatA, \nnatA'+1)  }
   \delta(\expr): \bang{k} \ltype $.\\
   We know :$ x \in \fv{\expr}$ from the assumption $ x \in \fv{\delta{\expr}}$.\\
   By IH2 on the premise, we know: $ \Gamma
   \tvdash{\max(\nnatA,\nnatA')} \expr[\valr/x]: \bang{k} \ltype $.\\
   By the rule delta, we get :
      $\Gamma' ,1+\Gamma \tvdash{1+ \max( \nnatA, \nnatA')  } \delta(\expr[\valr/x]): \bang{k} \ltype $
.\\
  
  \caseL{
      $   \inferrule{
      \ictx \Gamma_1, y: \type'  \tvdash{\nnatA} \expr: \type_1 \\
      \ictx \Gamma_2, y: \type',  x: \type_1 \tvdash{\nnatA'} \expr' : \type
    }{
      \ictx \max(\Gamma_1,\Gamma_2), y: \type' \tvdash{\max(\nnatA, \nnatA') } \elet x =  \expr \ein \expr': \type
    }~\textbf{let}  $
  }
Assume  $\Gamma_1 \tvdash{\nnatA_1'} \valr : \type' ~(1) $.\\
  Assume $\Gamma_2 \tvdash{\nnatA_2'} \valr : \type' ~(2) $.\\
  From Theorem 3, we know : $\max(\Gamma_1,\Gamma_2)
  \tvdash{\max(\nnatA_1', \nnatA_2')} \valr : \type' ~(3) $.\\
   TS:   $\max (\Gamma_1, \Gamma_2 ) \tvdash{ \max(\max(
     \nnatA_1',\nnatA_2') , \max(\nnatA', \nnatA) ) } ( \elet x =  \expr \ein \expr' )[\valr/x] : \type_2$.\\
   Similar to the application rule, we have 3 situations.\\
   By IH on the first premise, we know :$\Gamma_1
   \tvdash{\max(\nnatA_1', \nnatA )}  \expr[\valr/y] : \type_1  $. \\
   By IH on the second premise, we know : $ \Gamma_2, x: \type_1
   \tvdash{\max(\nnatA_2', \nnatA'  )} $.\\
   By the rule let, we conclude that: \\
    $\max (\Gamma_1, \Gamma_2 ) \tvdash{ \max(\max(
     \nnatA_1',\nnatA_2') , \max(\nnatA, \nnatA') ) } (\elet x =  \expr \ein \expr')[\valr/y] : \type_2$.\\

  % \caseL{
  % $  \inferrule{
  %     \ictx \Gamma_1 \tvdash{\nnatA_1} \expr_1 : \bang{k} \tbool\\
  %     \ictx \Gamma_2 \tvdash{\nnatA_2} \expr_2 : \type\\
  %     \ictx \Gamma_2 \tvdash{\nnatA_2} \expr_3 : \type
  %   }{
  %     \ictx \max(\Gamma_1, \nnatA_1+ \Gamma_2)  \tvdash{\nnatA_1 + \nnatA_2} \eif \eapp \expr_1 \eapp \expr_2
  %     \eapp \expr_3 : \type
  %   }~\textbf{if} $
  % }

  


  \end{proof} 





\begin{thm}[Adaptivity Soundness theorem]
  If $ \Gamma  \tvdash{\nnatA} \expr : \type$
  and  exists $\env $ satisfies $\Gamma$  and  $\env, \expr \bigstep \valr, \tr $,
  then $ A(\tr) \leq \nnatA  $.\\(
  $\env$ satisfies $\Gamma$ means $\dom(\env) = \dom(\Gamma) \land
  \forall x_i \in \dom(\Gamma).  \exists r_i.$ so that $\empty
  \tvdash{r_i} \env(xi) : \Gamma(x) $  )
\end{thm}%

\begin{thm}[Adaptivity Soundness theorem]
  If $ \Gamma  \tvdash{\nnatA} \expr : \type$
  and  exists $\env $ satisfies $\Gamma$  and  $\env, \expr \bigstep \valr, \tr $,
  then $ \Gamma  \tvdash{\nnatA- A(\tr)} \valr : \type$.\\(
  $\env$ satisfies $\Gamma$ means $\dom(\env) = \dom(\Gamma) \land
  \forall x_i \in \dom(\Gamma).  \exists r_i.$ so that $\empty
  \tvdash{r_i} \env(xi) : \Gamma(x) $  )
\end{thm}%

\begin{proof}
  By indution on the typing derivation.\\


  \caseL{
    \inferrule{
    }{
      \ictx \tctx , x: \bang{\nnatA} \ltype \tvdash{\nnatA} x: \bang{\nnatA}\ltype
    }~\textbf{Ax}
  }
  Assume exists the enviroment $\env$ and $\env(x) =\valr$ and
  $\empty \tvdash{ \nnatA} \valr: \bang{\nnatA}\ltype $. so that
  $\inferrule{ }{\env, x \bigstep \valr, (x, \env ) }  $. \\
  So we conclude that: $ A(  (x, \env)  ) = 0 \leq \nnatA $.
   
  
  \caseL{
    \inferrule{
      \ictx \Gamma, x: \type_1
      \tvdash{k}
      \expr: \type_2
    }{
      \ictx r+\Gamma \tvdash{k+r} \lambda x. \expr : \bang{r}  ( \type_1
      \lto \type_2)
    }~\textbf{lambda}
  }
 Assume exists the enviroment $\env$ and $\env(x) =\valr$ and
  $\empty \tvdash{ \nnatA} \valr: \bang{\nnatA}\ltype $. so that
  $\inferrule{ }{\env, x \bigstep \valr, (x, \env ) }  $. \\

$$\wq{ \inferrule{
  }{
    \env, \lambda x. \expr \bigstep (\lambda x.\expr, \env),
    (\lambda x.\expr, \env)
  }
}$$
  
  \caseL{
     \inferrule{
      \ictx \Gamma_1  \tvdash{\nnatA_1} \expr_1:  \bang{0} ( \type_1
      \lto \type_2      )\\
      \ictx \Gamma_2 \tvdash{\nnatA_2} \expr_2: \type_1 
    }{
      \ictx \max (\Gamma_1, \Gamma_2 ) \tvdash{\max( \nnatA_1,\nnatA_2) } \expr_1 \eapp \expr_2 : \type_2
    }~\textbf{app}
  }
  \[
\inferrule{
    \env, \expr_1 \bigstep \valr_1, \tr_1 \\
    \wq{ \valr_1 = (\efix f(x:\type).\expr, \env')} \\\\
    \env, \expr_2 \bigstep \valr_2, \tr_2 \\
    \env'[f \mapsto \valr_1, x \mapsto \valr_2], \expr \bigstep \valr, \tr
  }{
    \env, \expr_1 \eapp \expr_2 \bigstep \valr, \trapp{\tr_1}{\tr_2}{f}{x}{\tr}
  }
  \]

  
  
  \end{proof}