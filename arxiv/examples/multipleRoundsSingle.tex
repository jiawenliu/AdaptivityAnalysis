\begin{example}[Single Adaptivity Round Example]
    \label{ex:multiRoundsS}
    The program's adaptivity definition in our formal model,
    (in Definition~\ref{def:trace_adapt})
    comes across an over-approximation when capturing the program's intuitive adaptivity rounds.
    It results from the difference between its weight calculation and the \emph{variable may-dependency} definition.
    It occurs when the weight is computed over the traces different from the traces used in 
    witnessing the \emph{variable may-dependency} relation.
    
    The program $\kw{multiRoundsS(k)}$ in Figure~\ref{fig:multiRoundsS}(a) demonstrates this over-approximation.
    It is a variant of the multiple rounds strategy with input $k$.
    In each iteration, the query request $\clabel{\assign{p}{\query(\chi[y]+p)} }^{7}$ is based on value stored in $p$ and $y$ from previous iteration.
    Differ from Example~\ref{ex:multipleRounds},
    only the query answer from the $(k - 2)^{th}$ iteration is used in the query request, $\clabel{\assign{p}{\query(\chi[y]+p)} }^{7}$ of the next $(k - 1)^{th}$ iteration.
    % $\clabel{\assign{p}{\query(\chi[y]+p)} }^{7}$.
    In all the other iterations, $j \neq (k - 2)$, the if-control goes to the first branch
    % Because the execution will reset
    $p$'s value is reset by the constant $0$ in command $ \clabel{\assign{p}{0}}^{9}$.
    % in all the other iterations
    % at line $10$ after this query request.
    In this way, all the query answers stored in $p$ are erased and are not used
    in the query request at the next iteration, except the one at the $(k - 2)^{th}$ iteration.
    Intuitively, when $k \geq 2$, only the $\clabel{\assign{p}{\query(\chi[y]+p)} }^{7}$ in the $(k - 1)^{th}$ iteration
    depends on the query in $(k - 2)^{th}$ iteration and the \emph{adaptivity} round is $2$.
    When $k = 0$, the program does not go into the loop and there is no dependency between any query request.
    When $k = 1$, the program goes to the first branch in the first if-control with guard $\clabel{ k = 1}^{3}$.
    In this case, the second query request $ \clabel{ \assign{y}{\query(z)}}^{4}$ depends on the first one,
    $\clabel{\assign{z}{\query(0)} }^{1}$ and then the next query in the loop, $\clabel{\assign{p}{\query(\chi[y]+p)} }^{7}$ depends on the second one. Intuitively, the adaptivity is $3$.
    % In this sense, the intuitive \emph{adaptivity} rounds for this example is $2$. 
    However, our adaptivity definition fails to realize that there is only a dependency relation 
    between $p^7$ to itself at the $(k - 2)^{th}$ iteration.
    % but not in all the others. 
    As shown in the semantics-based dependency graph in Figure~\ref{fig:multiRoundsS}(b), 
    there is a cycle on $p^7$ representing the existence of the \emph{Variable May-Dependency} from $p^7$ on itself.
    % and the visiting times of labeled variable $p^7$ is 
    % $\lambda \trace_0 \st k$. 
    Weight of this vertex is $\lambda \trace_0 \st \env(\trace_0) k$,
    % The function $\lambda \trace_0 \st \env(\trace_0) k$ 
    which returns the evaluation times of the command $\clabel{\assign{p}{\query(\chi[y]+p)} }^{7}$ during the program execution under the initial trace $\trace_0$.
    % , which is expected to be equal to the loop iteration numbers, i.e., an initial value of input $k$ from the initial trace $\trace_0$.
    Since the command $\clabel{\assign{p}{\query(\chi[y]+p)} }^{7}$  will always be evaluated the same time as the loop iteration numbers, i.e. $k$,
    the weight function returns $\env(\trace_0) k$.
    However, $\env(\trace_0) k$ is the total number that this command is evaluated, rather than the number of the evaluations in which this command depends on other query requests.
    As a result, the walk with the longest query length 
    is
    $p^7 \to \cdots \to p^7 \to y^4 \to z^1 $ with the vertex $p^7$ visited $\env(\trace_0) k$ times, as the dotted arrows. 
    The adaptivity based on this walk
    is $\lambda \trace_0 \st \env(\trace_0) k + 2$,
    which is expected to be $0$ when $k = 0$ and $2$ when $k \geq 2$.
    %  instead of $\max\{0, 2, 3\}$. 
    % Though the $\THESYSTEM$ is able to give us $2 + k$, as an accurate bound w.r.t this definition.
    {
    \begin{figure}
    \centering
    %}
    \quad
    \begin{subfigure}{.8\textwidth}
    \begin{centering}
    {
    $ \begin{array}{l}
    \kw{multiRoundsS(k)} \triangleq \\
    \clabel{ \assign{j}{0}}^{0} ; 
    \clabel{\assign{z}{\query(0)} }^{1} ; 
    \clabel{\assign{p}{0} }^{2} ; \\
    \eif(\clabel{ k = 1}^{3},
    \clabel{ \assign{y}{\query(z)}}^{4},
    \clabel{\eskip}^5);\\
    \ewhile ~ \clabel{j \neq k}^{6} ~ \edo ~ \Big(
    \\
    \qquad \clabel{\assign{p}{\query(\chi[y]+p)} }^{7} ; \\
    \qquad 
    \eif(\clabel{ j \neq k - 2}^{8}, 
    \clabel{ \assign{p}{0}}^{9} ,
    \clabel{\eskip}^{10})  \\ 
    \qquad \clabel{\assign{j}{j + 1}}^{11} ; 
    \Big)
    \end{array}
    $ 
    }
    \caption{}
    \end{centering}
    \end{subfigure}
    \begin{subfigure}{.8\textwidth}
    \begin{centering}
    \begin{tikzpicture}[scale=\textwidth/15cm,samples=200]
    % Variables Initialization
    \draw[] (-5, 2) circle (0pt) node{{ $z^1: {}^{\lambda \trace_0 \st 1}$}};
    \draw[] (-5, 7) circle (0pt) node{{$p^2: {}^{\lambda \trace_0 \st 1}$}};
    \draw[] (-5, 4) circle (0pt) node{{ $y^4: {}^{\lambda \trace_0 \st 1}$}};
    % Variables Inside the Loop
    \draw[] (0, 6) circle (0pt) node{ $p^7: {}^{\lambda \trace_0 \st \env(\trace_0) k}$};
    \draw[] (0, 2) circle (0pt) node{ $p^{10}: {}^{\lambda \trace_0 \st \env(\trace_0) k}$};
    % Counter Variables
    \draw[] (5, 6) circle (0pt) node {$j^0: {}^{\lambda \trace_0 \st 1}$};
    \draw[] (5, 2) circle (0pt) node { $j^8: {}^{\lambda \trace_0 \st \env(\trace_0) k}$};
    %
    % Value Dependency Edges:
    \draw[ thick, -Straight Barb, densely dotted,] (0.8, 7) arc (220:-100:1);
    \draw[ -latex] (-1.5, 5.5) to [out=-130,in=130] (-1.5, 2);
    % Value Dependency Edges on Initial Values:
    \draw[ thick, -latex, densely dotted,] (-5, 3.5) -- (-5, 2.5) ;
    \draw[ -latex,] (-1.5, 5.5) -- (-4, 7) ;
    \draw[ thick, -latex, densely dotted,] (-1.5, 5.5) -- (-4, 4.7) ;
    \draw[ -latex,] (-1.5, 5.5) -- (-4, 2) ;
    %
    % Value Dependency For Control Variables:
    \draw[ -Straight Barb] (6.5, 2.5) arc (150:-150:1);
    % Control Dependency
    \draw[ -latex] (5, 2.5) -- (5, 5.5) ;
    \draw[ -latex] (1.2, 6) -- (3.5, 6) ;
    \draw[ -latex] (1.2, 6) -- (3.5, 2) ;
    \draw[ -latex] (1.5, 1.8) -- (3.5, 2) ; 
    % Edges Produced by Transitivity
    \draw[ -latex] (1.5, 1.8) -- (3.5, 6) ; 
    \end{tikzpicture}
    \caption{}
    \end{centering}
    \end{subfigure}
    \caption{(a) The loop example with single adaptivity rounds.
    (b) The corresponding semantics-based dependency graph.}
    \label{fig:multiRoundsS}
    \end{figure}
    }
    \end{example}