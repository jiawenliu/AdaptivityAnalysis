%
%
\subsection{Syntax}
\[
\begin{array}{llll}
\mbox{Arithmetic Operators} 
& \oplus_a & ::= & + ~|~ - ~|~ \times 
%
~|~ \div ~|~ \max ~|~ \min\\  
\mbox{Boolean Operators} 
& \oplus_b & ::= & \lor ~|~ \land
\\
%
\mbox{Relational Operators} 
& \sim & ::= & < ~|~ \leq ~|~ == 
\\  
%
\mbox{Label} 
& l & \in & \mathbb{N} \cup \{\lin, \lex\} 
\\ 
%
\mbox{Arithmetic Expression} 
& \aexpr & ::= & 
n \in \mathbb{N}^{\infty} ~|~ {x} ~|~ \aexpr \oplus_a \aexpr 
 ~|~ \elog \aexpr  ~|~ \esign \aexpr
\\
%
\mbox{Boolean Expression} & \bexpr & ::= & 
%
\etrue ~|~ \efalse  ~|~ \neg \bexpr
 ~|~ \bexpr \oplus_b \bexpr
%
~|~ \aexpr \sim \aexpr 
\\
%
\mbox{Expression} & \expr & ::= & v ~|~ \aexpr ~|~ \bexpr ~|~ [\expr, \dots, \expr]
\\  
%
\mbox{Value} 
& v & ::= & { n ~|~ \etrue ~|~ \efalse ~|~ [] ~|~ [v, \dots, v]}  
\\
%
\mbox{Query Expression} 
& {\qexpr} & ::= 
& { \qval ~|~ \aexpr ~|~ \qexpr \oplus_a \qexpr ~|~ \chi[\aexpr]} 
\\
%
\mbox{Query Value} & \qval & ::= 
& {n ~|~ \chi[n] ~|~ \qval \oplus_a  \qval ~|~ n \oplus_a  \chi[n]
    ~|~ \chi[n] \oplus_a  n}
    \\
\mbox{Labeled Command} 
& {c} & ::= &   [\assign {{x}}{ {\expr}}]^{l} ~|~  [\assign {{x} } {{\query(\qexpr)}}]^{l}
~|~ {\ewhile [ \bexpr ]^{l} \edo {c} }
\\
&&&
~|~ {c};{c}  
~|~ \eif([\bexpr]{}^l , {c}, {c}) 
~|~ [\eskip]^l\\ 
\mbox{Event} 
& \event & ::= & 
    ({x}, l, v, \bullet) ~|~ ({x}, l, v, \qval)  ~~~~~~~~~~~ \mbox{Assignment Event} \\
&&& ~|~(\bexpr, l, v, \bullet)   ~~~~~~~~~~~~~~~~~~~~~~~~~~~~~~~~~~ \mbox{Testing Event}
\\
\end{array}
\]
We use following notations to represent the set of corresponding terms:
\[
\begin{array}{lll}
\mathcal{V} & : & \mbox{Set of Variables}  
\\ 
%
\mathcal{VAL} & : & \mbox{Set of Values} 
\\ 
%
\mathcal{QVAL} & : & \mbox{Set of Query Values} 
\\ 
%
\cdom & : & \mbox{Set of Commands} 
\\ 
%
\eventset  & : & \mbox{Set of Events}  
\\
%
\eventset^{\asn}  & : & \mbox{Set of Assignment Events}  
\\
%
\eventset^{\test}  & : & \mbox{Set of Testing Events}  
\\
%
\ldom  & : & \mbox{Set of Labels}  
\\
%%
\mathcal{VAL}  & : & \mbox{Set of Labeled Variables}  
\\
%%
\dbdom  & : & \mbox{{Set of Databases}} 
\\
%
{\mathcal{T}} & : & \mbox{Set of Traces}
\\
%
\qdom & : & \mbox{{Domain of Query Results}}\\
\end{array}
\]
%
The labeled variables and assigned variables are set of variables annotated by a label. 
We use  
$\mathcal{LV}$ represents the universe of all the labeled variables and 
$\lvar(c) \in \mathcal{P}(\mathcal{V} \times \mathcal{L}) \subseteq \mathcal{LV}$,
represents the set of labeled variables in program $c$,
defined in Definition~\ref{def:lvar}.
Labeled variables in $c$ is the set of assigned variables.
%
%
\begin{defn}[labelled Variables $\lvar$]
  \label{def:lvar}
  {\footnotesize
  $$
    \lvar(c) \triangleq
    \left\{
    \begin{array}{ll}
        \{{x}^l\}           
        & {c} = [{\assign x e}]^{l} 
        \\
        \{{x}^l\}            
        & {c} = [{\assign x \query(\qexpr)}]^{l} 
        \\
        \lvar(c_1) \cup \lvar(c_2) 
        & {c} = {c_1};{c_2}
        \\
        \lvar(c) \cup \lvar(c_2)
        & {c} =\eif([\bexpr]^{l}, c_1, c_2) 
        \\
        \lvar(c')
        & {c}   = \ewhile ([\bexpr]^{l}, {c}')
  \end{array}
  \right.
  $$
  }
  \end{defn}
  $FV: \expr \to \mathcal{P}(\mathcal{V})$, computes the set of free variables in an expression. To be precise,
  $FV(\aexpr)$, $FV(\bexpr)$ and $FV(\qexpr)$ represent the set of free variables in arithmetic
  expression $\aexpr$, boolean expression $\bexpr$ and query expression $\qexpr$ respectively.
  The free variables
  showing up in $c$, which aren't defined before using, are actually the input variables of this program.
  
% \begin{defn}[Assigned Variables ($\avar : \cdom \to \mathcal{P}(\mathcal{V} \times \mathbb{N})$)]
% \label{def:avar}
% {\footnotesize
% $$ \avar_{c} \triangleq
%   \left\{
%   \begin{array}{ll}
%       \{{x}^l\}                   
%       & {c} = [{\assign x e}]^{l} 
%       \\
%       \{{x}^l\}                   
%       & {c} = [{\assign x \query(\qexpr)}]^{l} 
%       \\
%       \avar_{{c_1}} \cup \avar_{{c_2}}  
%       & {c} = {c_1};{c_2}
%       \\
%       \avar_{{c}} \cup \avar_{{c_2}} 
%       & {c} =\eif([\bexpr]^{l}, c_1, c_2) 
%       \\
%       \avar_{{c}'}
%       & {c}   = \ewhile ([\bexpr]^{l}, {c}')
% \end{array}
% \right.
% $$
% }
% \end{defn}
%

% \begin{defn}[labelled Variables $\lvar$]
% \label{def:lvar}
% {\footnotesize
% $$
%   \lvar_{c} \triangleq
%   \left\{
%   \begin{array}{ll}
%       \{{x}^l\} \cup FV(\expr)^{\lin}                  
%       & {c} = [{\assign x e}]^{l} 
%       \\
%       \{{x}^l\}   \cup FV(\qexpr)^{\lin}                
%       & {c} = [{\assign x \query(\qexpr)}]^{l} 
%       \\
%       \lvar_{{c_1}} \cup \lvar_{{c_2}}  
%       & {c} = {c_1};{c_2}
%       \\
%       \lvar_{{c}} \cup \lvar_{{c_2}} \cup FV(\bexpr)^{\lin}
%       & {c} =\eif([\bexpr]^{l}, c_1, c_2) 
%       \\
%       \lvar_{{c}'} \cup FV(\bexpr)^{\lin}
%       & {c}   = \ewhile ([\bexpr]^{l}, {c}')
% \end{array}
% \right.
% $$
% }
% \end{defn}
%
We also defined the set of query variables for a program $c$,
it is the set of variables set to the result of a query in the program formally in Definition~\ref{def:qvar}.
\begin{defn}[Query Variables ($\qvar: \cdom \to \mathcal{P}(\mathcal{LV})$)] 
  \label{def:qvar}
Given a program $c$, its query variables 
$\qvar(c)$ is the set of variables set to the result of a query in the program.
It is defined as follows:
{\footnotesize
$$
  \qvar(c) \triangleq
  \left\{
  \begin{array}{ll}
      \{\}                  
      & {c} = [{\assign x \expr}]^{l} 
      \\
      \{{x}^l\}                  
      & {c} = [{\assign x \query(\qexpr)}]^{l} 
      \\
      \qvar(c_1) \cup \qvar(c_2)  
      & {c} = {c_1};{c_2}
      \\
      \qvar(c_1) \cup \qvar(c_2) 
      & {c} =\eif([\bexpr]^{l}, c_1, c_2) 
      \\
      \qvar(c')
      & {c}   = \ewhile ([\bexpr]^{l}, {c}')
\end{array}
\right.
$$
}
\end{defn}
%
It is easy to see that a program $c$'s query variables is a subset of 
its labeled variables, $\qvar(c) \subseteq \lvar(c)$.
%
%
Every labeled variable in a program is unique, formally as follows with proof in Appendix~\ref{apdx:lvar_unique}.
\begin{lem}[Uniqueness of the Labeled Variables]
  \label{lem:lvar_unique}
  For every program $c \in \cdom$ and every two labeled variables such that
  $x^i, y^j \in \lvar(c)$, then $x^i \neq y^j$.
  \[
    \forall c \in \cdom, x^i, y^j \in \mathcal{L} \st x^i, y^j \in \lvar(c)\implies x^i \neq y^j.
    \]
\end{lem}
%
%

\subsection{Trace-based Operational Semantics for {\tt Query While} Language}

\subsubsection{Event}
%
%
We have two kinds of events: \emph{assignment events} and \emph{testing events}. 
Each event consists of a quadruple,
and we use $\eventset^{\asn}$ and $\eventset^{\test}$ to denote the set of all assignment events and testing events, respectively.
\begin{center}
$ \begin{array}{lllll}
\mbox{Event} 
& \event & ::= & 
({x}, l, v, \bullet) ~|~ ({x}, l, v, \qval)  & \mbox{Assignment Event} \\
&&& ~|~(\bexpr, l, v, \bullet)  & \mbox{Testing Event}
\\
\end{array}$
\end{center}
An assignment event tracks the execution of an assignment  or a query request and consists of the assigned variable, the label of the command that generates it, the value assigned to the variable, and the normal form of the query expression, $\qval$ if this command is a query request, otherwise a default value $\bullet$.
A testing event tracks the execution of if and while commands and consists of the guard of the command, the label of the command, the result of evaluating the guard, while the last element is $\bullet$.
%

Event projection operators $\pi_i$ projects the $i$th element from an event: 
\\
$\pi_i : 
\eventset \to \mathcal{VAR}\cup \mbox{Boolean Expression}  \cup \mathbb{N} \cup \mathcal{VAL} \cup \mathcal{QVAL} $ 
% \wqside{use b for Boolean expression?}

%
\begin{defn}[Equivalence of Query Expression]
%
\label{def:query_equal}
% \mg{Two} \sout{2} 
Two query expressions $\qexpr_1$, $\qexpr_2$ are equivalent, denoted as $\qexpr_1 =_{q} \qexpr_2$, if and only if
% $$
%  \begin{array}{l} 
%   \exists \qval_1, \qval_2 \in \mathcal{QVAL} \st \forall \trace \in \mathcal{T} \st
%     (\config{\trace,  \qexpr_1} \qarrow \qval_1 \land \config{\trace,  \qexpr_2 } \qarrow \qval_2) 
%     \\
%     \quad \land (\forall D \in \dbdom, r \in D \st 
%     \exists v \in \mathcal{VAL} \st 
%           \config{\trace, \qval_1[r/\chi]} \aarrow v \land \config{\trace,  \qval_2[r/\chi] } \aarrow v)  
%   \end{array}.
% $$
$$
 \begin{array}{l} 
   \forall \trace \in \mathcal{T} \st \exists \qval_1, \qval_2 \in \mathcal{QVAL} \st
    (\config{\trace,  \qexpr_1} \qarrow \qval_1 \land \config{\trace,  \qexpr_2 } \qarrow \qval_2) 
    \\
    \quad \land (\forall D \in \dbdom, r \in D \st 
    \exists v \in \mathcal{VAL} \st 
          \config{\trace, \qval_1[r/\chi]} \aarrow v \land \config{\trace,  \qval_2[r/\chi] } \aarrow v)  
  \end{array}.
$$
% \mg{$$
%  \begin{array}{l} 
%    \forall \trace \in \mathcal{T} \st \exists \qval_1, \qval_2 \in \mathcal{QVAL} \st
%     (\config{\trace,  \qexpr_1} \qarrow \qval_1 \land \config{\trace,  \qexpr_2 } \qarrow \qval_2) 
%     \\
%     \quad \land (\forall D \in \dbdom, r \in D \st 
%     \exists v \in \mathcal{VAL} \st 
%           \config{\trace, \qval_1[r/\chi]} \aarrow v \land \config{\trace,  \qval_2[r/\chi] } \aarrow v)  
%   \end{array}.
% $$
% }
 %
 where $r \in D$ is a row of the hidden database $\chi$, and $\chi$ is in the database domain $D$. 
 As usual, we will denote by $\qexpr_1 \neq_{q} \qexpr_2$  the negation of the equivalence.
% \\ 
% where $r \in D$ is a record in the database domain $D$,
% \mg{is  $FV(\qexpr)$ being defined here? If yes, I suggest to put it in a different place, rather than in the middle of another definition.} 
% $FV(\qexpr)$ is the set of free variables in the query expression $\qexpr$.
% \sout{$\qexpr_1 \neq_{q}^{\trace} \qexpr_2$  is defined vice versa.}
% \mg{As usual, we will denote by $\qexpr_1 \neq_{q}^{\trace} \qexpr_2$  the negation of the equivalence.}
%
\end{defn}
%
% \mg{In the next definition you don’t need the subscript e, it is clear that it is equivalence of events by the fact that the elements on the two sides of = are events. That is also true for query expressions. Also, I am confused by this definition. What happen for two query events?}
% \\
% \jl{The last component of the event is equal based on Query equivalence, $\pi_{4}(\event_1) =_q \pi_{4}(\event_2)$.
% In the previous version, the query expression is in the third component and I defined $v \neq \qexpr$ for all $v$ that isn't a query value.}
% \begin{defn}[Event Equivalence $\eventeq$]
% Two events $\event_1, \event_2 \in \eventset$ \mg{are equivalent, \sout{is in \emph{Equivalence} relation,}} denoted as $\event_1 \eventeq \event_2$ if and only if:
% \[
% \pi_1(\event_1) = \pi_1(\event_2) 
% \land  
% \pi_2(\event_1) = \pi_2(\event_2) 
% \land
% \pi_{3}(\event_1) = \pi_{3}(\event_2)
% \land 
% \pi_{4}(\event_1) =_q \pi_{4}(\event_2)
% \]
% %
% % \sout{The $\event_1 \eventneq \event_2$ is defined as vice versa.}
% % \mg{As usual, we will denote by $\event_1 \eventneq \event_2$  the negation of the equivalence.}
% \end{defn}
\begin{defn}[Event Equivalence]
  Two events $\event_1, \event_2 \in \eventset$ are equivalent, 
  % denoted as $\event_1 \eventeq \event_2$ 
  denoted as $\event_1 = \event_2$ 
  if and only if:
  \[
  \pi_1(\event_1) = \pi_1(\event_2) 
  \land  
  \pi_2(\event_1) = \pi_2(\event_2) 
  \land
  \pi_{3}(\event_1) = \pi_{3}(\event_2)
  \land 
  \pi_{4}(\event_1) =_q \pi_{4}(\event_2)
  \]
  %
  As usual, we will denote by $\event_1 \neq \event_2$  the negation of the equivalence.
\end{defn}

\subsubsection{Trace}
%
\subsection{Trace}
%
% An event $\event \in \eventset$ belongs to a trace $\trace$, i.e., $\event \eventin \trace$ are defined as follows:
% %
% \begin{equation}
%   \event \eventin \trace  
%   \triangleq \left\{
%   \begin{array}{ll} 
%     \etrue                  & \trace =  (\trace' \tracecate \event') \land (\event \eventeq \event') \\
%     \event \eventin \trace' & \trace =  (\trace' \tracecate \event') \land (\event \eventneq \event') \\ 
%     \efalse                 & o.w.
%   \end{array}
%   \right.
% \end{equation}
% %
% A well-formed event $\event \in \eventset$ belongs to a trace $\trace$ in signature, 
% i.e., $\event \sigin \trace$ are defined as follows:
%   %
% \begin{equation}
%   \event \sigin \trace  
%   \triangleq \left\{
%   \begin{array}{ll} 
%     \etrue                  & \trace =  (\trace' \tracecate \event') \land (\event \sigeq \event') \\
%     \event \sigin \trace'   & \trace =  (\trace' \tracecate \event') \land (\event \signeq \event') \\ 
%     \efalse                 & o.w.
%   \end{array}
%   \right.
% \end{equation}
%
%
%
%
% \todo{
% \[
% \mbox{Post-Processed Trace} \qquad \trace \qquad ::= \qquad \tracecate | \trace \tracecate \event
% % %
% \]
% Trace appending: $\tracecate: \mathcal{T} \to \eventset \to \mathcal{T}$
% \[
%   \trace \tracecate \event \triangleq
%   \left\{
%   \begin{array}{ll} 
%     {} \tracecate \event           & \trace =  \tracecate \\
%     \trace \tracecate \event    & \trace =  \trace' \tracecate \event\\ 
%   \end{array}
%   \right.
% \]
%
% \mg{I suggest to use more definition environments throughout this section.}
% \mg{Is ++ a constructor or a defined operation? It is in the grammar
%   but it seems that you are actually defining it here.}
%   \jl{It is a defined operator, I mixed them wrongly.}
  \begin{defn}[Trace Concatenation, $\tracecat: \mathcal{T} \to \mathcal{T} \to \mathcal{T}$]
Given two traces $\trace_1, \trace_2 \in \mathcal{T}$, the trace concatenation operator 
$\tracecat$ is defined as:
% \[
  % \trace_1 \tracecat \trace_2 \triangleq
  % \left\{
  % \begin{array}{ll} 
  %   \trace_1 \tracecat [] \triangleq \trace_1 & 
  %   \trace_1 \tracecat (\trace_2' :: \event) \triangleq  (\trace_1  \tracecat \trace_2')  :: \event 
  % \end{array}
  % \right.
% \]
% \[
  % \trace_1 \tracecat \trace_2 \triangleq
  % \left\{
  % \begin{array}{ll} 
  %   \trace_1 \tracecat [] \triangleq \trace_1 & 
  %   \trace_1 \tracecat (\trace_2' :: \event) \triangleq  (\trace_1  \tracecat \trace_2')  :: \event 
  % \end{array}
  % \right.
% \]
\[
  \trace_1 \tracecat \trace_2 \triangleq
  \left\{
  \begin{array}{ll} 
     \trace_1 & \trace_2 = [] \\
     (\trace_1  \tracecat \trace_2')  :: \event & \trace_2 = \trace_2' :: \event
  \end{array}
  \right.
\]
\end{defn}
%
% \todo{ need to consider the occurrence times }
% \\
% \mg{This definition is not well given. You use a different operator to define it t[:e] which you say it is a shorthand. It cannot be a shorthand because it is used in the definition. I think you need to define two operations, either in sequence or mutually recursive.}
% \jl{I moved this definition from the main paper into the appendix \ref{apdx:flowsto_event_soundness}. Because this operator is only being used in the soundness proof. And I also feel it doesn't worth to spend many lines in the main paper for defining this complex notation.}
% Subtrace: $[ : ] : \mathcal{{T} \to \eventset \to \eventset \to \mathcal{T}}$ 
% \wqside{Confusing, I can not understand the subtraction, it takes a trace, and two events, and this operator is used to subtract these two events?}
% \[
%   \trace[\event_1 : \event_2] \triangleq
%   \left\{
%   \begin{array}{ll} 
%   \trace'[\event_1: \event_2]             & \trace = \event :: \trace' \land \event \eventneq \event_1 \\
%   \event_1 :: \trace'[:\event_2]  & \trace = \event :: \trace' \land \event \eventeq \event_1 \\
%   {[]} & \trace = [] \\
%   \end{array}
%   \right.
% \]
% For any trace $\trace$ and two events $\event_1, \event_2 \in \eventset$,
% $\trace[\event_1 : \event_2]$ takes the subtrace of $\trace$ starting with $\event_1$ and ending with $\event_2$ including $\event_1$ and $\event_2$.
% \\
% We use $\trace[:\event_2] $ as the shorthand of subtrace starting from head and ending with $\event_2$, and similary for $\trace[\event_1:]$.
% \[
%   \trace[:\event] \triangleq
%   \left\{
%   \begin{array}{ll} 
%  \event' :: \trace'[: \event]             & \trace = \event' :: \trace' \land \event' \eventneq \event \\
%   \event'  & \trace = \event' :: \trace' \land \event' \eventeq \event \\
%   {[]}  & \trace = [] 
%   \end{array}
%   \right.
% % \]
% % \[
%   \quad
%   \trace[\event: ] \triangleq
%   \left\{
%   \begin{array}{ll} 
%   \trace'[\event: ]     & \trace =  \event' :: \trace' \land \event \eventneq \event' \\
%   \event' :: \trace'  & \trace = \event' :: \trace' \land \event \eventeq \event' \\
%   {[ ] } & \trace = []
%   \end{array}
%   \right.
% \]
% %
% \mg{why in the next definition you use ( ) while in the previous ones you didn't? They seem like the same cases. And why you use o.w. instead of []?}
% An event $\event \in \eventset$ belongs to a trace $\trace$, i.e., $\event \eventin \trace$ are defined as follows:
% %
% \begin{equation}
%   \event \eventin \trace  
%   \triangleq \left\{
%   \begin{array}{ll} 
%     \etrue                  & \trace =  (\event' :: \trace') \land (\event \eventeq \event')
%                               \\
%     \event \eventin \trace' & \trace =  (\event' :: \trace') \land (\event \eventneq \event') \\ 
%     \efalse                 & o.w.
%   \end{array}
%   \right.
% \end{equation}
\begin{defn}(An Event Belongs to A Trace)
  An event $\event \in \eventset$ belongs to a trace $\trace$, i.e., $\event \in \trace$ are defined as follows:
%
\begin{equation}
  \event \in \trace  
  \triangleq \left\{
  \begin{array}{ll} 
    \etrue                  & \trace =  \trace' :: \event'
     \land \event = \event'
                              \\
    \event \in \trace' & \trace =  \trace' :: \event'
    \land \event \neq \event' \\ 
    \efalse                 & \trace = []
  \end{array}
  \right.
\end{equation}
As usual, we denote by $\event \notin \trace$ that the event $\event$ doesn't belong to the trace $\trace$.
\end{defn}
%
% An event $\event \in \eventset$ belongs to a trace $\trace$ up to value, 
% i.e., $\event \sigin \trace$ are defined as follows:
%   %
% \begin{equation}
%   \event \sigin \trace  
%   \triangleq \left\{
%   \begin{array}{ll} 
%     \etrue                  & \trace =  (\trace' \tracecate \event')                          \land \pi_1(\event_1) = \pi_1(\event_2) 
%                               \land  \pi_2(\event_1) = \pi_2(\event_2)  
%                               % \land \vcounter(\trace \event) = \vcounter()
%                               \\
%     \event \sigin \trace'   & \trace =  (\trace' \tracecate \event') 
%                               \land 
%                               (\pi_1(\event_1) \neq \pi_1(\event_2) 
%                               \lor  \pi_2(\event_1) \neq \pi_2(\event_2)) 
%                               \\ 
%     \efalse                 & o.w.
%   \end{array}
%   \right.
% \end{equation}
%
% % \mg{Why the previous definition used :: and now you switch to ++? Cannot this just be defined using ::? I am trying to anticipate places where a reader might be confused. Also, this definition would be much simpler if we defined event in a more uniform way. Ideally, we want to distinguish three cases, we don't need to distinguish 7 cases.}\\
% % \jl{My bad, I was really too sticky to the convention.
% I though the list appending  $::$ can only append element on the left side.}
% \mg{We introduce a counting operator $\vcounter : \mathcal{T} \to \mathbb{N} \to \mathbb{N}$ whose behavior is defined as follows, \sout{Counter $\vcounter : \mathcal{T} \to \mathbb{N} \to \mathbb{N}$ }}.
% \wq{The operator counter actually provides the number of times a specific label appears in a trace. Only a number, the position of label is ignored.}
% \[
% \begin{array}{lll}
% \vcounter((x, l, v) :: \trace ) l \triangleq \vcounter(\trace) l + 1
% &
% \vcounter((b, l, v):: \trace ) l \triangleq \vcounter(\trace) l + 1
% &
% \vcounter((x, l, \qval, v):: \trace ) l \triangleq \vcounter(\trace) l + 1
% \\
% \vcounter((x, l', v):: \trace ) l \triangleq \vcounter(\trace ) l
% &
% \vcounter((b, l', v):: \trace ) l \triangleq \vcounter(\trace ) l
% &
% \vcounter((x, l', \qval, v):: \trace ) l \triangleq \vcounter(\trace ) l
% \\
% \vcounter({[]}) l \triangleq 0
% &&
% \end{array}
% \]
% \[
% \begin{array}{lll}
% \vcounter(\trace  \tracecat [(x, l, v)] ) l \triangleq \vcounter(\trace) l + 1
% &
% \vcounter(\trace  \tracecat [(b, l, v)] ) l \triangleq \vcounter(\trace) l + 1
% &
% \vcounter(\trace  \tracecat [(x, l, \qval, v)] ) l \triangleq \vcounter(\trace) l + 1
% \\
% \vcounter(\trace  \tracecat [(x, l', v)] ) l \triangleq \vcounter(\trace ) l
% &
% \vcounter(\trace  \tracecat [(b, l', v)] ) l \triangleq \vcounter(\trace ) l
% &
% \vcounter(\trace  \tracecat [(x, l', \qval, v)]) l \triangleq \vcounter(\trace ) l
% \\
% \vcounter({[]}) l \triangleq 0
% &&
% \end{array}
% \]
We introduce a counting operator $\vcounter : \mathcal{T} \to \mathbb{N} \to \mathbb{N}$ whose behavior is defined as follows,
% \[
% \begin{array}{lll}
% \vcounter(\trace :: (x, l, v, \bullet) ) l \triangleq \vcounter(\trace) l + 1
% &
% \vcounter(\trace  ::(b, l, v, \bullet) ) l \triangleq \vcounter(\trace) l + 1
% &
% \vcounter(\trace  :: (x, l, v, \qval) ) l \triangleq \vcounter(\trace) l + 1
% \\
% \vcounter(\trace  :: (x, l', v, \bullet) ) l \triangleq \vcounter(\trace ) l, l' \neq l
% &
% \vcounter(\trace  :: (b, l', v, \bullet) ) l \triangleq \vcounter(\trace ) l, l' \neq l
% &
% \vcounter(\trace  :: (x, l', v, \qval)) l \triangleq \vcounter(\trace ) l, l' \neq l
% \\
% \vcounter({[]}) l \triangleq 0
% &&
% \end{array}
% \]
\[
\begin{array}{ll}
\vcounter(\trace :: (x, l, v, \bullet), l ) \triangleq \vcounter(\trace, l) + 1
&
\vcounter(\trace  ::(b, l, v, \bullet), l) \triangleq \vcounter(\trace, l) + 1
\\
\vcounter(\trace  :: (x, l, v, \qval), l) \triangleq \vcounter(\trace, l) + 1
&
\vcounter(\trace  :: (x, l', v, \bullet), l) \triangleq \vcounter(\trace, l), l' \neq l
\\
\vcounter(\trace  :: (b, l', v, \bullet), l) \triangleq \vcounter(\trace, l), l' \neq l
&
\vcounter(\trace  :: (x, l', v, \qval), l) \triangleq \vcounter(\trace, l), l' \neq l
\\
\vcounter({[]}, l) \triangleq 0
&
\end{array}
\]
%
% The Latest Label $\llabel : \mathcal{T} \to \mathcal{VAR} \to \mathbb{N}$ 
% The label of the latest assignment event which assigns value to variable $x$.
% \[
%   \begin{array}{lll}
% \llabel((x, l, v):: \trace) x \triangleq l
% &
% \llabel((b, l, v)):: \trace x \triangleq \llabel(\trace) x
% &
% \llabel((x, l, \qval, v):: \trace) x \triangleq l
% \\
% \llabel((y, l, v):: \trace) x \triangleq \llabel(\trace ) x
% &
% \llabel((y, l, \qval, v):: \trace) x \triangleq \llabel(\trace ) x
% \\
% \llabel({[]}) x \triangleq \bot
% &&
% \end{array}
% \]
%
% \todo{wording}
% \mg{This wording needs to be fixed. Also notice that the type is wrong, a label is not always returned.}
%  The Latest Label $\llabel : \mathcal{T} \to \mathcal{VAR} \to \mathbb{N}$ 
% The label of the latest assignment event which assigns value to variable $x$.
% \[
%   \begin{array}{lll}
% \llabel(\trace  \tracecat [(x, l, v)]) x \triangleq l
% &
% \llabel(\trace  \tracecat [(b, l, v)]) x \triangleq \llabel(\trace) x
% &
% \llabel(\trace  \tracecat [(x, l, \qval, v)]) x \triangleq l
% \\
% \llabel(\trace  \tracecat [(y, l, v)]) x \triangleq \llabel(\trace ) x
% &
% \llabel(\trace  \tracecat [(y, l, \qval, v)]) x \triangleq \llabel(\trace ) x
% \\
% \llabel({[]}) x \triangleq \bot
% &&
% \end{array}
% \]
We introduce an operator $\llabel : \mathcal{T} \to \mathcal{VAR} \to \ldom \cup \{\bot\}$, which 
takes a trace and a variable and returns the label of the latest assignment event which assigns value to that variable.
Its behavior is defined as follows,
% \begin{defn}[Latest Label]
  \[
    % \begin{array}{lll}
  \llabel(\trace  :: (x, l, \_, \_)) x \triangleq l
  ~~~
  \llabel(\trace  :: (y, l, \_, \_)) x \triangleq \llabel(\trace ) x, y \neq x
  % &
  ~~~
  \llabel(\trace :: (b, l, v, \bullet)) x \triangleq \llabel(\trace) x
  % &
  % \\
  % \llabel(\trace  :: (y, l, v, \bullet)) x \triangleq \llabel(\trace ) x
  % &
  % \llabel(\trace :: (y, l, v, \qval)) x \triangleq \llabel(\trace ) x
  % &
  ~~~
  \llabel({[]}) x \triangleq \bot
  % \end{array}
  \]
% \end{defn}
%
% \mg{This wording needs to be fixed but also the description does not make sense. This operator seems to just collect all the labels in a trace. Again, this definition would be shorter with a more uniform definition of events.}
% The Trace Label Set $\tlabel : \mathcal{T} \to \mathcal{P}{(\mathbb{N})}$ 
% The label of the latest assignment event which assigns value to variable $x$.
% \[
%   \begin{array}{llll}
% \tlabel_{(\trace  \tracecat [(x, l, v)])} \triangleq \{l\} \cup \tlabel_{(\trace )}
% &
% \tlabel_{(\trace  \tracecat [(b, l, v)])} \triangleq \{l\} \cup \tlabel_{(\trace)}
% &
% \tlabel_{(\trace  \tracecat [(x, l, \qval, v)])} \triangleq \{l\} \cup \tlabel_{(\trace)}
% &
% \tlabel_{[]} \triangleq \{\}
% \end{array}
% \]
% \begin{defn}
  The operator $\tlabel : \mathcal{T} \to \mathcal{P}{(\ldom)}$ gives the set of labels in every event belonging to 
  a trace, whoes behavior is defined as follows,
\[
  % \begin{array}{llll}
\tlabel{(\trace  :: (\_, l, \_, \_))} \triangleq \{l\} \cup \tlabel{(\trace )}
~~~
\tlabel({[ ]}) \triangleq \{\}
% \end{array}
\]

If we observe the operational semantics rules, we can find that no rule will shrink the trace. 
So we have the Lemma~\ref{lem:tracenondec} with proof in Appendix~\ref{apdx:tracenondec}, 
specifically the trace has the property that its length never decreases during the program execution.
\begin{lem}
[Trace Non-Decreasing]
\label{lem:tracenondec}
For every program $c \in \cdom$ and traces $\trace, \trace' \in \mathcal{T}$, if 
$\config{c, \trace} \rightarrow^{*} \config{\eskip, \trace'}$,
then there exists a trace $\trace'' \in \mathcal{T}$ with $\trace \tracecat \trace'' = \trace'$
%
$$
\forall \trace, \trace' \in \mathcal{T}, c \st
\config{c, \trace} \rightarrow^{*} \config{\eskip, \trace'} 
\implies \exists \trace'' \in \mathcal{T} \st \trace \tracecat \trace'' = \trace'
$$
\end{lem}

Since the equivalence over two events is defined over the query value equivalence, 
when there is an event belonging to a trace, 
if this event is a query assignment event, 
it is possible that 
the event showing up in this trace has a different form of query value, 
but they are equivalent by Definition~\ref{def:query_equal}.
So we have the following Corollary~\ref{coro:aqintrace} with proof in Appendix~\ref{apdx:aqintrace}.
\begin{coro}
\label{coro:aqintrace}
For every event and a trace $\trace \in \mathcal{T}$,
if $\event \in \trace$, 
then there exist another event $\event' \in \eventset$ and traces $\trace_1, \trace_2 \in \mathcal{T}$
such that $\trace_1 \tracecat [\event'] \tracecat \trace_2 = \trace $
with 
$\event$ and $\event'$ equivalent but may differ in their query value.
\[
  \forall \event \in \eventset, \trace \in \mathcal{T} \st
\event \in \trace \implies \exists \trace_1, \trace_2 \in \mathcal{T}, 
\event' \in \eventset \st (\event \in \event') \land \trace_1 \tracecat [\event'] \tracecat \trace_2 = \trace  
\]
\end{coro}

\subsubsection{Environment} 
$ \env : {\mathcal{T}}  \to \mathcal{V} \to \mathcal{VAL} \cup \{\bot\}$
\[
\begin{array}{lll}
\env(\trace  \traceadd (x, l, v, \bullet)) x \triangleq v
&
\env(\trace \traceadd (y, l, v, \bullet)) x \triangleq \env(\trace) x, y \neq x
&
\env(\trace \traceadd (b, l, v, \bullet)) x \triangleq \env(\trace) x
\\
\env(\trace \traceadd (x, l, v, \qval)) x \triangleq v
&
\env(\trace \traceadd (y, l, v, \qval)) x \triangleq \env(\trace) x, y \neq x
&
\env({[]} ) x \triangleq \bot
\end{array}
\]

\subsubsection{Operational Semantics Rules}
{
\begin{mathpar}
\boxed{ \config{\trace,\aexpr} \aarrow v \, : \, \mbox{Trace  $\times$ Arithmetic Expr $\Rightarrow$ Arithmetic Value} }
\\
% \\
\inferrule{ 
  \empty
}{
 \config{\trace,  n} 
 \aarrow n
}
\and
\inferrule{ 
  \env(\trace) x = v
}{
 \config{\trace,  x} 
 \aarrow v
}
\and
\inferrule{ 
  \config{\trace, \aexpr_1} \aarrow v_1
  \and 
  \config{\trace, \aexpr_2} \aarrow v_2
  \and 
   v_1 \oplus_a v_2 = v
}{
 \config{\trace,  \aexpr_1 \oplus_a \aexpr_2} 
 \aarrow v
}
\and
\inferrule{ 
  \config{\trace, \aexpr} \aarrow v'
  \and 
  \elog v' = v
}{
 \config{\trace,  \elog \aexpr} 
 \aarrow v
}
\and
\inferrule{ 
  \config{\trace, \aexpr} \aarrow v'
  \and 
  \esign v' = v
}{
 \config{\trace,  \esign \aexpr} 
 \aarrow v
}
\\
\boxed{ \config{\trace, \bexpr} \barrow v \, : \, \mbox{Trace $\times$ Boolean Expr $\Rightarrow$ Boolean Value} }
\\
\inferrule{ 
  \empty
}{
 \config{\trace,  \efalse} 
 \barrow \efalse
}
\and 
\inferrule{ 
  \empty
}{
 \config{\trace,  \etrue} 
 \barrow \etrue
}
\and 
\inferrule{ 
  \config{\trace, \bexpr} \barrow v'
  \and 
  \neg v' = v
}{
 \config{\trace,  \neg \bexpr} 
 \barrow v
}
\and 
\inferrule{ 
  \config{\trace, \bexpr_1} \barrow v_1
  \and 
  \config{\trace, \bexpr_2} \barrow v_2
  \and 
   v_1 \oplus_b v_2 = v
}{
 \config{\trace,  \bexpr_1 \oplus_b \bexpr_2} 
 \barrow v
}
\and 
\inferrule{ 
  \config{\trace, \aexpr_1} \aarrow v_1
  \and 
  \config{\trace, \aexpr_2} \aarrow v_2
  \and 
   v_1 \sim v_2 = v
}{
 \config{\trace,  \aexpr_1 \sim \aexpr_2} 
 \barrow v
}
\\
\boxed{ \config{\trace, \expr} \earrow v \, : \, \mbox{Trace $\times$ Expression $\Rightarrow$ Value} }
\\
\inferrule{ 
  \config{\trace, \aexpr} \aarrow v
}{
 \config{\trace,  \aexpr} 
 \earrow v
}
\and
\inferrule{ 
  \config{\trace, \bexpr} \barrow v
}{
 \config{\trace,  \bexpr} 
 \earrow v
}
\and
\inferrule{ 
  \config{\trace, \expr_1} \earrow v_1
  \cdots
  \config{\trace, \expr_n} \earrow v_n
}{
 \config{\trace,  [\expr_1, \cdots, \expr_n]} 
 \earrow [v_1, \cdots, v_n]
}
\and
\inferrule{ 
  \empty
}{
 \config{\trace,  v} 
 \earrow v
}
\\
\boxed{ \config{\trace, \qexpr} \qarrow \qval \, : \, \mbox{Trace  $\times$ Query Expr $\Rightarrow$ Query Value} }
\\
\inferrule{ 
  \config{\trace, \aexpr} \aarrow n
}{
 \config{\trace,  \aexpr} 
 \qarrow n
}
\and
\inferrule{ 
  \config{\trace, \qexpr_1} \qarrow \qval_1
  \and
  \config{\trace, \qexpr_2} \qarrow \qval_2
}{
 \config{\trace,  \qexpr_1 \oplus_a \qexpr_2} 
 \qarrow \qval_1 \oplus_a \qval_2
}
\and
\inferrule{ 
  \config{\trace, \aexpr} \aarrow n
}{
 \config{\trace, \chi[\aexpr]} \qarrow \chi[n]
}
\and
\inferrule{ 
  \empty
}{
 \config{\trace,  \qval} 
 \qarrow \qval
}
 \end{mathpar}
%
The trace based operational semantics rules are defined in Figure \ref{fig:os}.
%
\begin{figure}
{
\begin{mathpar}
\boxed{
\mbox{Command $\times$ Trace}
\xrightarrow{}
\mbox{Command $\times$ Trace}
}
\and
\boxed{\config{{c, \trace}}
\xrightarrow{} 
\config{{c',  \trace'}}
}
\\
\inferrule
{
\empty
}
{
\config{\clabel{\eskip}^l,  \trace } 
\xrightarrow{} 
\config{\clabel{\eskip}^l, \trace}
}
~\textbf{skip}
%
\and
%
\inferrule
{
\event = ({x}, l, v, \bullet)
}
{
\config{[\assign{{x}}{\aexpr}]^{l},  \trace } 
\xrightarrow{} 
\config{\clabel{\eskip}^l, \trace \traceadd \event}
}
~\textbf{assn}
%
\and
%
{
\inferrule
{
 \trace, \qexpr \qarrow \qval
 \and 
\query(\qval) = v
\and 
\event = ({x}, l, v, \qval)
}
{
\config{{[\assign{x}{\query(\qexpr)}]^l, \trace}}
\xrightarrow{} 
\config{{\clabel{\eskip}^l,  \trace \traceadd \event} }
}
~\textbf{query}
}
%
\and
%
\inferrule
{
 \trace, b \barrow \etrue
 \and 
 \event = (b, l, \etrue, \bullet)
}
{
\config{{\ewhile [b]^{l} \edo c, \trace}}
\xrightarrow{} 
\config{{
c; \ewhile [b]^{l} \edo c),
\trace \traceadd \event}}
}
~\textbf{while-t}
%
%
\and
%
\inferrule
{
 \trace, b \barrow \efalse
 \and 
 \event = (b, l, \efalse, \bullet)
}
{
\config{{\ewhile [b]^{l}, \edo c, \trace}}
\xrightarrow{} 
\config{{
  \clabel{\eskip}^l,
\trace \traceadd \event}}
}
~\textbf{while-f}
%
%
\and
%
%
\inferrule
{
\config{{c_1, \trace}}
\xrightarrow{}
\config{{c_1',  \trace'}}
}
{
\config{{c_1; c_2, \trace}} 
\xrightarrow{} 
\config{{c_1'; c_2, \trace'}}
}
~\textbf{seq1}
%
\and
%
\inferrule
{
  \config{{c_2, \trace}}
  \xrightarrow{}
  \config{{c_2',  \trace'}}
}
{
\config{{\clabel{\eskip}^l; c_2, \trace}} \xrightarrow{} \config{{ c_2', \trace'}}
}
~\textbf{seq2}
%
\and
%
%
\inferrule
{
   \trace, b \barrow \etrue
 \and 
 \event = (b, l, \etrue, \bullet)
}
{
 \config{{
\eif([b]^{l}, c_1, c_2), 
\trace}}
\xrightarrow{} 
\config{{c_1, \trace \traceadd \event}}
}
~\textbf{if-t}
%
\and
%
\inferrule
{
 \trace, b \barrow \efalse
 \and 
 \event = (b, l, \efalse, \bullet)
}
{
\config{{\eif([b]^{l}, c_1, c_2), \trace}}
\xrightarrow{} 
\config{{c_2, \trace \traceadd \event}}
}
~\textbf{if-f}
% %
%
%
\end{mathpar}
}
% \end{subfigure}
    \caption{Trace-based Operational Semantics for Language.}
    \label{fig:os}
\end{figure}
%
%
If we observe the operational semantics rules, we can find that no rule will shrink the trace. 
So we have the Lemma~\ref{lem:tracenondec} with proof in Appendix~\ref{apdx:tracenondec}, 
specifically the trace has the property that its length never decreases during the program execution.
\begin{lem}
[Trace Non-Decreasing]
\label{lem:tracenondec}
For every program $c \in \cdom$ and traces $\trace, \trace' \in \mathcal{T}$, if 
$\config{c, \trace} \rightarrow^{*} \config{\eskip, \trace'}$,
then there exists a trace $\trace'' \in \mathcal{T}$ with $\trace \tracecat \trace'' = \trace'$
%
$$
\forall \trace, \trace' \in \mathcal{T}, c \st
\config{c, \trace} \rightarrow^{*} \config{\eskip, \trace'} 
\implies \exists \trace'' \in \mathcal{T} \st \trace \tracecat \trace'' = \trace'
$$
\end{lem}

Since the equivalence over two events is defined over the query value equivalence, 
when there is an event belonging to a trace, 
if this event is a query assignment event, 
it is possible that 
the event showing up in this trace has a different form of query value, 
but they are equivalent by Definition~\ref{def:query_equal}.
So we have the following Corollary~\ref{coro:aqintrace} with proof in Appendix~\ref{apdx:aqintrace}.
\begin{coro}
\label{coro:aqintrace}
For every event and a trace $\trace \in \mathcal{T}$,
if $\event \in \trace$, 
then there exist another event $\event' \in \eventset$ and traces $\trace_1, \trace_2 \in \mathcal{T}$
such that $\trace_1 \tracecat [\event'] \tracecat \trace_2 = \trace $
with 
$\event$ and $\event'$ equivalent but may differ in their query value.
\[
  \forall \event \in \eventset, \trace \in \mathcal{T} \st
\event \in \trace \implies \exists \trace_1, \trace_2 \in \mathcal{T}, 
\event' \in \eventset \st (\event \in \event') \land \trace_1 \tracecat [\event'] \tracecat \trace_2 = \trace  
\]
\end{coro}
%