We have two kinds of events: \emph{assignment events} and \emph{testing events}. 
Each event consists of a quadruple,
and we use $\eventset^{\asn}$ and $\eventset^{\test}$ to denote the set of all assignment events and testing events, respectively.
\begin{center}
$ \begin{array}{lllll}
\mbox{Event} 
& \event & ::= & 
({x}, l, v, \bullet) ~|~ ({x}, l, v, \qval) & \mbox{Assignment Event} \\
&&& ~|~(\bexpr, l, v, \bullet) & \mbox{Testing Event}
\\
\end{array}$
\end{center}
An assignment event tracks the execution of an assignment or a query request and consists of the variable name, the label of the command that generates it, the value assigned to the variable.
The fourth element is the normal form of the query request, $\qval$ if this command is a query request, otherwise a default value $\bullet$.

A testing event tracks the execution of if and while commands and consists of the guard of the command, the label of the command, and the result of evaluating the guard, while the last element is $\bullet$.

The event projection operators $\pi_i$ projects the $i^{th}$ element from an event:
$ \pi_i : 
\eventset \to \vardom \cup \booldom \cup \ldom \cup \valuedom \cup \qdom \cup\{\bullet\} $.
Its computation is as follows,
\\
$\pi_1(x, l, v, \bullet) = x $ \qquad $\pi_1({x}, l, v, \qval) = x$ \qquad $ \pi_1(\bexpr, l, v, \bullet) = \bexpr$
\\
$\pi_2(\_, l, \_, \_ ) = l$
\qquad
$\pi_3(\_, \_, v, \_) = v$
\qquad
$\pi_4(\_, \_, \_, \qval) = \qval$ \qquad $\pi_4(\_, \_, \_, \bullet) = \bullet$