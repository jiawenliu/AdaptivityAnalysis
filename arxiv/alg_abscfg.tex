This section shows how to generate the abstract transition graph $\absG(c)$ of a
program $c$ through constructing its vertices and edges.

An \emph{Abstract Transition Graph}, $\absG(c)$ for a program $c$ is composed of
a vertex set $\absV(c)$ and an edge set $\absE(c)$, $\absG(c) \triangleq (\absV(c), \absE(c))$.
% For a program $c$, this analysis first generates its abstract execution control flow graph notated as follows,
% \[\absG(c) =(\absV(c), \absE(c))\]
%
\\
Every 
vertex $l \in \absV(c)$ is the label of a labeled command in $c$, which is unique.
We also call the unique label as program point.
% corresponds to a program point $l$, which is a unique
% label of a command in this program.
% $\absV(c)$ is the set of $c$'s all program points,
\\
Each edge $(l \xrightarrow{dc} l') \in \absE(c)$ is an abstract transition
between two program points $l, l'$. 
There is an edge from $l$ to $l'$ if and only if
the command with label $l'$ can execute right after the execution of the command with label $l$.
% if and only if there is a control flow between two program points.
Each edge is annotated by a constraint $dc \in \dcdom^{\top}$, which is generated from the command with label $l$.
This constraint describes the abstract execution of the command with $l$. 
%  before the introduction of the edge and weight estimation.  
% We discuss the vertices and edge of the
% abstract transition graph for a program $c$, $\absG(c)$.

\paragraph{Abstract Control Flow Graph Vertices Construction}
\label{sec:alg_abscfg-vertex}
Every 
vertex $l \in \absV(c)$ corresponds to a program point $l$, which is a unique
label of a command in this program.
Concretely,
the vertices of this graph is the set of $c$'s labels with the exit label ${\lex}$ formally as follows,
\[ 
  \absV(c) = \lvar(c)\cup\{{\lex}\}
\]
%  corresponding to a label command in the program.

\paragraph{Abstract Control Flow Graph Edge Construction}
\label{sec:alg_abscfg-edge}

Each edge $(l \xrightarrow{dc} l') \in \absE(c)$ is an abstract transition
between two program points $l, l'$. 
The edge is constructed by the standard control flow analysis satisfying
there is an edge from $l$ to $l'$ if and only if
the command with label $l'$ can execute right after the execution of the command with label $l$.

% if and only if there is a control flow between two program points.
Each edge is annotated by a constraint $dc$ generated from the command with label $l$.
This constraint describes the abstract execution of the command with $l$. 
It is either
the symbol $\top$, 
a boolean expression or 
a \emph{difference constraint} computed according to the algorithm in~\cite{SinnZV17}.
% This step shows how to generate the abstract transition graph $\absG(c)$ of a
% program $c$ through constructing its vertices and edges.
Below we summarize how to compute the constraints and generate the edges in three steps,
\begin{enumerate}
 \item In the first step, we generate the \emph{constraint}
 of the expression for every program's labeled command,
 which is used as the annotation of an edge.

 \item In the second step, we compute the \emph{initial and continuation state} for each command. 
 The initial state is a set that contains the
 program point where this command {starts} executing, 
 and the final state is a set
 that contains the constraint of this command
 and the continuation program points after the execution of this command.

 \item In the third step, we compute the set of edges for the program, $\absflow(c)$.
 Each edge is an \emph{abstract event}, which is a pair of the initial and final states.
\end{enumerate}
%
\paragraph{Constraint Computation}
% In this step, we first show how to compute the constraints for expressions in a program $c$,
% by a program abstraction method adopted from the
% algorithm in Section 6 in~\cite{SinnZV17}.
% \\
Given a program $c$,
every expression in an assignment command or in the guard of a $\eif$ or $\ewhile$ command
is transformed into a constraint as follows.
%
%
\begin{defn}[Difference Constraint, Symbolic Constant]
The difference constraints $DC(\mathcal{V} \cup \constdom)$ is the set of all the inequality of
form $x' \leq y + v$ or $x' \leq v$ where $x \in \mathcal{V} $, 
$y \in \mathcal{V}$ and $v \in \scvardom$.
The \emph{Symbolic Constant} can be a natural number, $\infty$, the input variable of a program, $\inpvardom$ or
$Q_m$ representing the maximum value returned as the query answer. 
We use $\scvardom \subseteq \mathbb{N} \cup \vardom \cup \{ \infty, Q_m \}$ to denote the universe of all \emph{Symbolic Constant},
which is the set of all natural numbers with $\infty$ and programs' input variables.
\end{defn}
An inequality $x' \leq y + v$ describes that the value of $x$ in the current state is
at most the value of $y$ in the previous state plus the symbolic constant $v$.
An inequality $x' \leq v$ describes that the value of $x$ in the current state is
at most the value $v$.

\begin{defn}[Symbolic Expression]
 \label{def:adaptfun-symbolic_expr}
 The \emph{symbolic expression} of a program $\scexpr(c) \subseteq \mathcal{A}$ is the set of all the arithmetic expressions over $\mathbb{N} \cup \inpvar(c) \cup \{\infty, Q_m \}$ for the program $c$.
\end{defn}

\begin{defn}[Constraint]
 The constraint set, $\dcdom^{\top}$ is composed of the \emph{Difference Constraints} $DC(\mathcal{V} \cup \constdom)$, the \emph{Boolean Expressions}, $\booldom$ and $\top$.
 \end{defn}
 

When a difference constrain shows up as an edge annotation, $l \xrightarrow{x' \leq y + v} l'$,
% Then $x'$ 
it denotes that
the value of variable $x$
after executing the command at $l$ is at most
% and the right-hand side describes 
the value of variable $y$ plus $v$ before the execution,
and $l \xrightarrow{x' \leq v} l'$ respectively denotes value of variable $x$
after executing the command at $l$ is at most
the value of the symbolic constant $v$ before the execution.
We have $l \xrightarrow{x' \leq Q_m} l'$ in the case that command $l$ is a query request and the query answer is assigned to variable $x$.

%

The Boolean Expressions $b$ from the set $\booldom$.
$b$ on an edge $l \xrightarrow{b} l'$ describes
that after evaluating the guard with label $l$,
$b$ holds and the command with label $l$ will execute right after.
%

The top constraint, $\top$ denotes true. It is preserved for $\eskip$ command or commands that don't
interfere with any counter variable.


For every expression in each of the label commands, we abstract it into a constraint in three steps via the program abstraction method as below.
It is a context-free computation skeleton overall the variable and value domain. In a specific computation with a given program $c$, these domains will associate with the program $c$.

\begin{defn}[Constraint Computation]
 \label{def:constraint_compute}
 With a program $c$, a boolean expression $\bexpr$ in the guard of a $\eif$ or $\ewhile$ command
 or an expression $\expr$ and a variable $x$
 in an assignment command $\assign{x}{\expr}$ as input.

 We first initialize 
 $\grdvar = \{\}$ as the set of the variable used in the expression of every while or if guard in the program $c$.

 Then the constraint $\absexpr(\bexpr, \_)$ or $\absexpr(x - v, x)$ is computed as follows,
 \[
 \begin{array}{ll} 
 \absexpr(x - v, x) = x' \leq x - v & x \in \grdvar \land v \in \constdom \\
 \absexpr(y + v, x) = x' \leq y + v & x, y \in \grdvar \land v \in \constdom \\
 \absexpr(v, x) = x' \leq v & x \in \grdvar \land v \in (\grdvar \cup \constdom) \\
 \absexpr(y + v, x) = x' \leq y + v, 
 \grdvar = \grdvar \cup \{y\} & x \in \grdvar \land y \notin \grdvar \land v \in \constdom \\
 \absexpr(\qexpr, x) = x' \leq Q_m & x \in \grdvar \land \qexpr \text{ is a query expression} \\
 \absexpr(\bexpr, \_) = \bexpr, \grdvar = \grdvar \cup 
 \kw{FV}(\bexpr) & x \in \grdvar \land \bexpr \text{ is a boolean expression} \\
 \absexpr(\expr, x) = \top & x \notin \grdvar \\
 \end{array}
 \]
 $\absexpr(\expr, x)$ $\grdvar$ is iteratively updating until stabilized over every guard and assignment command in $c$, and $\absexpr(\expr, x)$
 denotes the stabilized result in the following paper.
 \end{defn}
%
$\grdvar \subseteq \vardom$ is the domain for all variables used in the guard expression of every while command overall programs. 
In the fourth case, if a variable $x$, belonging to the set 
$\grdvar$ is updated by a variable $y$, which isn't in this set, 
we add $y$ into the set $\grdvar$ and repeat 
above procedure until $\grdvar$ and $\absexpr(\expr, x)$ is stabilized. 
Specifically,
we handle a normalized expression, $x > 0$
in guards of while loop headers, and 
the counter variable $x$ only increases, decreases, or is reset by 
simple arithmetic expression (mainly multiplication, division, minus, and plus (able to extend to max and min)). 
The counter variable $x$ is generalized into the norm when the boolean expression $x > 0$
in $\ewhile$ doesn't have the form $x > 0$.
The way of normalizing the guards and computing the norms is adopted from the computation step 1 in Section 6.1 in paper \cite{SinnZV17}. 

In the $5^{th}$ case where the expression comes from a query request, we abstract this query expression into $x \leq Q_m$.
By doing this, we ignore the answer of this query in the static analysis just to capture the possible dependency relation pass through query requests.
% \\
\begin{defn}[Universe of The Symbolic Expression ($\scexprdom$)]
 \label{def:symbolic_expr_domain}
 $\scexprdom$ is the universe of all the arithmetic expressions 
over $\constdom$, $\scexprdom \subseteq \mathcal{A}$
\end{defn}
Intuitively, the symbolic expression set is a subset of arithmetic expressions over $\mathbb{N}$
by restricting the variables into the
input variables.
% , 
% i.e., $\scexprdom \subseteq \inpexprdom$.

\paragraph{Abstract Initial and Final State Computation}
The initial state is the
program points before executing this command, which is computed by the standard initial state generation method from control flow analysis.
The final state is a set
that contains the constraint of this command and the program points after the execution of this command.
This set is enriched 
% program's initial and final states 
from the standard control flow analysis.

\begin{itemize}
 \item The \emph{abstract initial state}, $\absinit(c) \in \mathcal{P}(\ldom)$
 for a command, $c$ is the initial program point corresponds to the unique program label of the first executed command in $c$,
 computed as follows,
%
\[
 \begin{array}{ll}
 \absinit(\clabel{\assign{x}{\expr}}{}^l) & = l \\
 \absinit(\clabel{\assign{x}{\query(\qexpr)}}{}^l) & = l \\
 \absinit(\clabel{\eskip}^{l}) & = l \\
 \absinit(\eif \clabel{\bexpr}^{l} \ethen c_1 \eelse c_2) & = l\\
 \absinit(\ewhile \clabel{\bexpr}^{l} \edo (c_w)) & = l \\
 \absinit(c_1 ; c_2) & = \absinit(c_1) \\
 \end{array}
 \]
\item The \emph{abstract final state} of the program $c$, 
$\absfinal(c) \in \mathcal{P}(\ldom \times \dcdom^{\top})$
is a set of pairs, $(l, dc)$ with a
program point (i.e., a label), $l$ as the first component and a constraint, 
$dc$ as the second component.
% Every pair in $\absfinal(c)$ 
The program point $l$ corresponds to the labeled command after the execution of $c$,
and the constraint $dc$ in this pair is computed by $\absexpr$ for the expression in $c$.
% in the first step.
\\
Given a program $c$, its final state, $\absfinal(c)$ is computed as follows,
% $\absfinal(c) \in\mathcal{P}(\ldom \times \dcdom^{\top})$,
% computes the set of Abstract Final State for the command. 
 \[
 \begin{array}{ll}
 \absfinal(\clabel{\assign{x}{\expr}}{}^l) & = \{(l, \absexpr\eapp (\expr, x))\} \\
 \absfinal(\clabel{\assign{x}{\query(\qexpr)}}{}^l) & = \{
 (l, x' \leq 0 + Q_m )\} \\
 \absfinal(\clabel{\eskip}^{l}) 
 & = \{(l, \top)\} \\
 \absfinal(\eif \clabel{\bexpr}^{l} \ethen c_1 \eelse c_2) & = \absfinal(c_1) \cup \absfinal(c_2) \\
 \absfinal(\ewhile \clabel{\bexpr}^{l} \edo (c_w)) & = \{(l, \absexpr(\bexpr, \top))\} \\
 \absfinal(c_1 ; c_2) & = \absfinal(c_2) \\
 \end{array}
 \]
 %
\end{itemize}
 \paragraph{Abstract Event Computation} 
% Each abstract event is an edge between two vertices in the abstract transition graph.
% It is 
 The set of abstract events for a program $c$ is generated by computing the initial state and final state interactively and recursively.
% \begin{itemize}
% \item \emph{Abstract Event}: 
% $\absevent \in $
% $\ldom \times \dcdom^{\top} \times \ldom$
% \item \emph{Abstract Event Computation}: $\absflow \in \cdom \to \mathcal{P}( \ldom \times \dcdom^{\top} \times \ldom )$
% \end{itemize}
 \begin{defn}[Abstract Event]
 \label{def:adaptfun-abs_event}
 An \emph{abstract event},
 $\absevent \in $
 $\ldom \times \dcdom^{\top} \times \ldom$
 is a 
 % pair of the abstract initial state and final state.
 triple where the first and third components are labels,
 second component is a constraint from $\dcdom^{\top}$.
 % the thrid % computed from program's abstract final and initial state, $\absfinal(c)$ and $\absinit(c)$ with formal definition, and algorithm detail in Appendix.
 % the constraint and the third corresponds to a final state.
 \end{defn}
 In an abstract event $(l, dc, l')$ of a program $c$, 
 the first label $l \in \ldom$ corresponds to an initial state of $c$, and 
 the second label $l' \in \ldom$ with the constraint $dc \in \dcdom^{\top}$ correspond to an abstract continuation state of $c$.
 The abstract initial state is a label from $\ldom$.
% The abstract final state is a pair from $\ldom \times \dcdom^{\top}$, 
% where first component is a label from $\ldom$ and the second component is a constraint from $\dcdom^{\top}$.
 %
We abuse the notation $\mathcal{P}(\absevent)$ for the power set of all abstract events.

The set of the abstract events $\absflow(c)$ for a program $c$
% .
% Its type is formally defined 
is computed as follows in Definition~\ref{def:absevent_compute}.
 %
 \begin{defn}[Abstract Event Computation]
 \label{def:absevent_compute}
 $\absflow \in \cdom \to \mathcal{P}( \ldom \times \dcdom^{\top} \times \ldom )$
 \end{defn}
 %
% The \emph{Abstract Execution Trace} for program $c$ is computed as follows.
% \\
 % We now show how to compute the abstract execution trace. 
 We first append a $\eskip$ command with 
% a symbolic label $l_e$, i.e., $\clabel{\eskip}^{l_e}$ at the end of the program $c$, and compute the $\absflow(c) = \absflow'(c')$ for $c'$, where $c' = c;\clabel{\eskip}^{l_e}$ as follows,
the label $\lex$, i.e., $\clabel{\eskip}^{\lex}$ at the end of the program $c$, and construct 
the program $c' = c;\clabel{\eskip}^{\lex}$.
Then, we compute the $\absflow(c) = \absflow'(c')$ for $c'$ as follows,
 %
 {
 \[
 \begin{array}{ll}
 \absflow'(\clabel{\assign{x}{\expr}}{}^l) & = \emptyset \\
 \absflow'(\clabel{\assign{x}{\query(\qexpr)}}{}^l) & = \emptyset \\
 \absflow'([\eskip]^{l}) & = \emptyset \\
 \absflow'(\eif \clabel{\bexpr}^{l} \ethen c_t \eelse c_f) & = \absflow'(c_t) \cup \absflow'(c_f)
 \\ & \quad 
 \cup \left\{(l, \absexpr(\bexpr, \top), \absinit(c_t) ) \right\}
 \\ & \quad 
 \cup \left\{ (l, \absexpr(\neg\bexpr, \top), \absinit(c_f)) \right\} \\
 \absflow'(\ewhile \clabel{\bexpr}^{l} \edo (c_w)) & = \absflow'(c_w) \cup \{(l, \absexpr(\bexpr, \top), \absinit(c_w)) \} 
 \\ & \quad 
 \cup \{(l', dc, l)| (l', dc) \in \absfinal(c_w) \} \\
 \absflow'(c_1 ; c_2) & = \absflow'(c_1) \cup \absflow'(c_2) 
 \\ & \quad 
 \cup \{ (l, dc, \absinit(c_2)) | (l, dc) \in \absfinal(c_1) \} \\
 \end{array}
 \]
 }
 Notice $\absflow'([x := \expr]^{l})$, $\absflow'([x := \query(\qexpr)]^{l})$ and $\absflow'([\eskip]^{l})$ are all empty set. 
 
% \highlight{Theorem Guarantees:}

For every event $\event$ with label $l$, if it is in an execution trace $\trace$ of the program $c$, 
 there is an abstract event in the program's abstract execution trace of form $(l, \_, \_)$, formally below
 with the proof in Appendix~\ref{apdx:abscfg_sound}
 \begin{lem}[Soundness of the Abstract Events]
 \label{lem:abscfg_sound}
 For every program $c$ and
 an execution trace $\trace \in \ftdom$ that is generated w.r.t.
 an initial trace $\vtrace_0 \in \ftdom_0(c)$,
 there is an abstract event $\absevent = (l, \_, \_) \in \absflow(c)$ 
 for every event $\event \in \trace$ having the same label $l$, i.e., $\event = (\_, l, \_, \_)$.
 %
 \[
 \begin{array}{l}
 \forall c \in \cdom, \vtrace_0 \in \ftdom_0(c), \trace \in \ftdom , l, l' \in \absV(c), \event = (\_, l, \_, \_) \in \eventset \st
 \\
 \quad
 \config{{c}, \trace_0} \to^{*} \config{\clabel{\eskip}^{l'}, \trace_0 \tracecat \vtrace} 
 \land \event \in \trace 
 \\
 \qquad \implies \exists \absevent = (l, \_, \_) \in (\ldom\times \dcdom^{\top} \times \ldom) \st 
 \absevent \in \absflow(c)
 \end{array}
 \]
 \end{lem}

For every program point $l$, if it is the label of an assignment command in a program $c$,
there is a unique abstract event in the program's abstract events set $\absevent \in \absflow(c)$ of form $(l, \_, \_)$. 
\begin{lem}[Uniqueness of the Abstract Event]
 \label{lem:abscfg_uniquex}
 For every program $c$ and
 an execution trace $\trace \in \ftdom$ that is generated w.r.t.
 an initial trace $\vtrace_0 \in \ftdom_0(c)$,
 there is a unique abstract event $\absevent = (l, \_, \_) \in \absflow(c)$ 
 for every assignment event $\event \in \eventset^{\asn}$ in the
 execution trace having the label $l$, i.e., $\event = (\_, l, \_, \_)$ and $\event \in \trace$.
%
\[
 \begin{array}{l}
 \forall c \in \cdom, \vtrace_0 \in \ftdom_0(c), \trace \in \ftdom , l, l' \in \absV(c), \event = (\_, l, \_) \in \eventset^{\asn} \st
 \\
 \qquad \config{{c}, \trace_0} \to^{*} \config{\clabel{\eskip}^{l'}, \trace_0 \tracecat \vtrace} 
 \land \event \in \trace 
 \\
 \qquad \implies \exists! \absevent = (l, \_, \_) \in (\ldom\times \dcdom^{\top} \times \ldom) \st 
 \absevent \in \absflow(c)
\end{array}
\]
\end{lem}
This lemma is proved in Appendix~\ref{apdx:abscfg_uniquex}.

 \paragraph{Edge Construction}
For a program $c$, the edges on its abstract transition graph are constructed by computing the set of all its abstract events, $\absflow(c)$ as follows,
 \[
 \absE(c) = \{(l_1, dc, l_2) | (l_1, dc, l_2) \in \absflow(c)\}
 \]
\paragraph{Abstract Transition Graph Construction} 
With the vertices $\absV(c)$ and edges $\absE(c)$ ready, we construct the abstract transition graph, formally in
Definition~\ref{def:abs_cfg}.
%
\begin{defn}[Abstract Transition Graph]
\label{def:abs_cfg}
Given a program $c$, 
its \emph{abstract transition graph} $\absG(c) =(\absV(c), \absE(c))$ is computed as follows,
\\
$\absE(c) = \{(l_1, dc, l_2) | (l_1, dc, l_2) \in \absflow(c)\}$,
\\
$\absV(c) = \lvar(c)\cup\{\lex\}$
\end{defn}


The show again the running example as below, to illustrate the construction of the abstraction transition graphs.
%
The edge $(0 \xrightarrow{a' \leq 0} 1)$ on the top, tells us the command 
$\clabel{\assign{a}{0}}^0$ is executed with a continuation point $1$ such that the
% where the 
command $\clabel{\assign{j}{k}}^1$ will be executed next.
The annotation $a' \leq 0$ is a difference constraint 
computed for
% by abstracting
the expression $0$ in the assignment command $\assign{a}{0}$.
%  from the function $\absexpr(0)$.
It represents that the value of $a$ is less than or equal to $0$ after the
execution of $\clabel{\assign{a}{0}}^0$ and before executing $\clabel{\assign{j}{k}}^1$.
Another example edge $5 \xrightarrow{a' \leq a + x } 2$ describes the execution of
 the command at line $5$
$\clabel{\assign{a}{x + a}}^{5}$.
This edge has difference constraint $a' \leq a+x $.
The $a'$ on the left side of $a' \leq a+x$ represents the value of $a$ after executing this assignment command.
The boolean constraint $j \leq 0 $ on the edge $2 \xrightarrow{j \leq 0} 6$
represents the negation of the testing guard $j > 0$
in the $\ewhile$ command with loop header at line $2$.
The edge from $3$ to $4$ comes from a query request command.
The constraint over this edge, $x' < Q_m$ describes after executing the query request command,
$\clabel{\assign{x}{\query(\chi[j])} }^{3}$, the query request results stored in $x$ is bounded by $Q_m$. 

\begin{figure} 
    \centering
    \begin{subfigure}{.2\textwidth}
    \begin{centering}
    {\small
    $
        \begin{array}{l}
              \clabel{ \assign{a}{0}}^{0} ;   
                \clabel{\assign{j}{k} }^{1} ; \\
                \ewhile ~ \clabel{j > 0}^{2} ~ \edo ~ \\
                \Big(
                 \clabel{\assign{x}{\query(\chi[j])} }^{3}  ; \\
                 \clabel{\assign{j}{j-1}}^{4} ;\\
                \clabel{\assign{a}{x + a}}^{5}       \Big);\\
                \clabel{\assign{l}{\query(\chi[k]*a)} }^{6}\\
            \end{array}
    $
    }
    \caption{}
    \end{centering}
    \end{subfigure}
    \begin{subfigure}{.38\textwidth}
        \begin{centering}
      \begin{tikzpicture}[scale=\textwidth/20cm,samples=200]
      \draw[] (-7, 10) circle (0pt) node{{ $0$}};
      \draw[] (0, 10) circle (0pt) node{{ $1$}};
      \draw[] (0, 7) circle (0pt) node{\textbf{$2$}};
      \draw[] (0, 4) circle (0pt) node{{ $3$}};
      \draw[] (0, 1) circle (0pt) node{{ $4$}};
      \draw[] (-7, 1) circle (0pt) node{{ $5$}};
      % Counter Variables
      \draw[] (6, 7) circle (0pt) node {\textbf{$6$}};
      \draw[] (6, 4) circle (0pt) node {{ $\lex$}};
      %
      % Control Flow Edges:
      \draw[ thick, -latex] (-6, 10)  -- node [above] {$a' \leq 0$}(-0.5, 10);
      \draw[ thick, -latex] (0, 9.5)  -- node [left] {$j' \leq k$} (0, 7.5) ;
      \draw[ thick, -latex] (0, 6.5)  -- node [right] {$j > 0$}  (0, 4.5);
      \draw[ thick, -latex] (0, 3.5)  -- node [right] {$x' \leq Q_m$} (0, 1.5) ;
      \draw[ thick, -latex] (-0.5, 1)  -- node [above] {$j' \leq j - 1$} (-6, 1) ;
      \draw[ thick, -latex] (-6, 1.5)  -- node [left] {$a' \leq x + a$} (-0.5, 7)  ;
      \draw[ thick, -latex] (0.5, 7)  -- node [above] {$ j \leq 0 $}  (5.5, 7);
      \draw[ thick, -latex] (6, 6.5)  -- node [right] {$l' \leq k * a$} (6, 4.5) ;
      \end{tikzpicture}
      \caption{}
        \end{centering}
        \end{subfigure}
        \begin{subfigure}{.38\textwidth}
          \begin{centering}
        %   \todo{abstract-cfg for two round}
        \begin{tikzpicture}[scale=\textwidth/20cm,samples=200]
        \draw[] (-10, 10) circle (0pt) node{{ $0: 1$}};
        \draw[] (0, 10) circle (0pt) node{{ $1: 1$}};
        \draw[] (0, 7) circle (0pt) node{\textbf{$2: k$}};
        \draw[] (0, 4) circle (0pt) node{{ $3: k$}};
        \draw[] (0, 1) circle (0pt) node{{ $4: k$}};
        \draw[] (-10, 1) circle (0pt) node{{ $5: k$}};
        % Counter Variables
        \draw[] (6, 7) circle (0pt) node {\textbf{$6: 1$}};
        \draw[] (6, 4) circle (0pt) node {{ $\lex: 1$}};
        %
        % Control Flow Edges:
      \draw[ thick, -latex] (-8, 10)  -- node [above] {$a' \leq 0$}(-1.5, 10);
      \draw[ thick, -latex] (0, 9.5)  -- node [left] {$j' \leq k$} (0, 7.5) ;
      \draw[ thick, -latex] (0, 6.5)  -- node [right] {$j > 0 $}  (0, 4.5);
      \draw[ thick, -latex] (0, 3.5)  -- node [right] {$x' \leq Q_m$} (0, 1.5) ;
      \draw[ thick, -latex] (-1.5, 1)  -- node [above] {$j' \leq j - 1$} (-8, 1) ;
      \draw[ thick, -latex] (-8, 1.5)  -- node [left] {$a' \leq x + a$} (-1.5, 7)  ;
      \draw[ thick, -latex] (1.5, 7)  -- node [above] {$j \leq 0 $}  (4.5, 7);
      \draw[ thick, -latex] (6, 6.5)  -- node [right] {$l' \leq k * a$} (6, 4.5) ;
        \end{tikzpicture}
        \caption{}
          \end{centering}
          \end{subfigure}
      \caption{(a) The same $\kw{towRounds(k)}$ program as Figure~\ref{fig:overview-example}
      (b) The abstract control flow graph for $\kw{towRounds(k)}$  (c) The abstract control flow graph with the reachability bound for $\kw{towRounds(k)}$.}
      \label{fig:abscfg_tworound}
    \end{figure}