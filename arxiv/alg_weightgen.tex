This section presents the quantitative analysis algorithm, which performs over the same \emph{abstract control flow graph}, $\absG(c)$ of a program $c$ as well. 
As the $\traceW(c)$ defined in Definition~\ref{def:trace_graph}, the weight of every $x^l \in \progV(c)$ is
the execution times of the command with label $l$. 
In this sense, to estimate weight of $x^l$, this step first computes an upper bound, the \emph{reachability-bound}\cite{GulwaniZ10} for every $l \in \absV(c)$
%  in the abstract control flow graph 
on the execution times of the command with label $l$. 
Then, the \emph{reachability-bound} is used to estimate the maximal visiting times of the labeled variable $x^l \in \lvar(c)$
and 
% as well as 
the weight of the vertex $x^l \in \progV(c)$.
The two computation steps are summarized as follows,
\begin{enumerate}
  \item \textbf{Reachability Bound Analysis}
  This analysis estimate for each program point $l in \absV(c)$ a symbolic upper bounds on the execution times of the command with label $l$. 
  These symbolic upper bounds are expressions with the input variables as free variables, hence they correspond to the weight functions in the semantics-based dependency graphs. 
  Our reachability-bound algorithm adapts to our setting ideas from previous work~\cite{ZulegerGSV11,SinnZV14,sinn2017complexity}.
  Specifically, it provides an upper bound on the number of times every command can be executed.
    \item \textbf{Weight Estimation}
    Because
    the vertex in program's $\absG(c)$ shares the same unique label with the vertex in $\progV(c)$, 
    we use the \emph{reachability-bound} on the vertex $l \in \absV(c)$ directly as the weight of the vertex $x^l$ in $\progV(c)$.
\end{enumerate}

\paragraph{Step-1: Reachability Bound Analysis}
This symbolic reachability bound analysis performs over the same \emph{abstract control flow graph}, $\absG(c)$ of a program $c$. 
% through the edges in $\absG(c)$, which correspond to $c$'s abstract transition between labels.
It first computes a \emph{reachability bound} for every edge $l \xrightarrow{dc} l' \in \absE(c)$,
which is a symbolic bound on the maximum execution times of the command with label $l$ of $c$.
%  when executing the program $c$.
Then the \emph{reachability bound} for edge $l \xrightarrow{dc} l'$ is used as the bound on the maximum visiting times of the vertex $l \in \absV(c)$.
It is a sound upper bound on the visiting times
of every label $l \in \absV(c)$, named \emph{reachability-bound}.
The computation steps are 
summarized as follows,
\begin{enumerate}
  \item It first collects three edge sets for each variable,
in which the variable increases, decreases and reset respectively.
\item
Then, for each edge  $l \xrightarrow{dc} l' \in \absE(c)$, it assigns a variable $x$ (or a symbolic constant $v \in \constdom$) if $x$ (or $v$) decreases in $dc$, as this edge's local bound.
\item
It then computes the bound on the maximum value of the local bound for each edge,
and the \emph{reachability-bound} on the execution
times of the corresponding edge recursively.
%  but path-insensitively.
\item The last step uses \emph{reachability-bound} $w$ for edge $l \xrightarrow{dc} l'$ as the bound on the maximum visiting times of the vertex $l \in \absV(c)$ and generates a set $\absW(c)$ contains a pair $(l, w)$ for every $l \in \absV(c)$.
\end{enumerate}

The algorithm in this step is inspired from the Algorithm.2 in paper~\cite{SinnZV14},
% which assigns a variable to each edge on which this variable decrease as its ranking function.
the Algorithm.3 in paper~\cite{ZulegerGSV11},
and the Definition.25 in Section 4 of paper~\cite{sinn2017complexity}.
\begin{itemize}
\item Algorithm.3 in paper~\cite{ZulegerGSV11} assigns a set of variables to each transition in which these variables decrease as the local bound
and estimates the maximum value each variable in this set.
\item Algorithm.2 in paper~\cite{SinnZV14} assigns a variable to each edge on which this variable decrease as its ranking function
and then estimates the maximum value for the ranking function.
\item The Definition.25 in paper~\cite{sinn2017complexity}
assigns each transition with a variable that decreases in this transition, as the local bound and computes the bound similarly.
\end{itemize}
%
The computation steps are as follows,
\begin{enumerate}
\item \textbf{Variable Modifications}
For each variable $x$ in a program $c$, this step computes three edge sets, $\inc(c, x)$, $\dec(c, x)$,
and $\reset(c, x)$ for $x$.
Every edge in a set corresponds to a transition in which $x$ is increased,
%  $\inc(c, x)$,
decreased
% $\dec(c, x)$ and 
or reset
% $\reset(c, x)$, 
respectively.
\\
$\inc: \cdom \to \mathcal{V} \to \mathcal{P}(\absevent) $
is the set of the edges where the variable increase, 
\\
$\inc(c, x) = \{ \absevent | \absevent = (l, l', x' \leq x + v) \land \absevent \in \absflow(c)\}$
\\
$\dec: \mathcal{V} \to \mathcal{P}(\absevent) $
is the set of abstract events where the variable decrease,
\\
$\dec(c, x) = \{\absevent| \absevent = (l, l', x' \leq x - v) \land \absevent \in \absflow(c)\}$
\\
$\reset: \cdom \to \mathcal{V} \to \mathcal{P}(\absevent) $
is the set of the abstract events where the variable is reset,
\\
$\reset(c, x) = \{\absevent| \absevent = (l, l', x' \leq y - v) \land x \neq y \land \absevent \in \absflow(c)\}$
\\
$\resetchain: \cdom \to \mathcal{V} \to \mathcal{P}(\mathcal{P}(\absevent)) $
is the set of the chain of abstract events where the variable is reset through the chain.
\\
In addition to
collect the edge set that $x$ is reset on every edge in this set, i.e., compute the $\reset(c, x)$,
we also compute a set, $\resetchain(c, x)$ contains sequences of edges for $x$
based on the Definition.20 in \cite{sinn2017complexity}.
In each sequence, $(e_0, \cdots, e_m) \in \resetchain(c, x)$
a variable $x_i$ is reset by another variable $x_{i + 1}$ on edge $e_{i}$
and $x_{i + 1}$ is reset on edge $e_{i + 1}$ recursively
for every $i = 0, \cdots, m - 1$.
$x$ is reset on the first edge $e_0$ of every sequence in $\resetchain(c, x)$.
\highlight{Rephrase: Each edge $e_i$ in a sequence $(e_0, \cdots, e_m) \in \resetchain(c, x)$
resets a variable $x_i$ by another variable $x_{i + 1}$ such that $x_{i + 1}$
is reset on edge $e_{i + 1}$ recursively. The first edge $e_0$ of each sequence resets the variable $x$.}
%
\\
In the following steps, $c$ is omitted in $\inc(x)$,
$\dec(x)$ and $\reset(x)$ for concise when the reference of a program $c$ is clear in the context.
\item {Assigning The Local Bound to An Edge}
This step adopts the local bound computation method in Section 4 of \cite{sinn2017complexity}.
It assigns to each edge  $l \xrightarrow{dc} l'\in \absE(c)$ a \emph{local bound} as follows.
\begin{defn}[Local Bound  Generatation]
  \label{def:ranking_gen}
% Given a program $c$ with its abstract transition graph 
% $\absG(c) = (\absV, \absE)$,
For every edge $\absevent$ in the transition graph $\absG(c)$ of a program $c$,
its \emph{local bound}, $\locbound(\absevent, c)$
is the variable that decreases on this edge, computed as follows,
%
\[ 
\begin{array}{ll}
  \locbound(\absevent, c) \triangleq 1 
  & \absevent \notin SCC(\absG(c))
  \\
  \locbound(\absevent, c) \triangleq x
  & \absevent \in SCC(\absG(c)) \land \absevent \in \dec(x) \land  \absevent = (\_, \_ , x' \leq x - v) \\
  \locbound(\absevent, c) \triangleq x
  & \absevent \in SCC(\absG(c)) \land 
  \absevent  \notin \bigcup_{x \in \mathcal{V}} \dec(x)
  \land \absevent \notin SCC(\absG(c) \setminus \dec(x)). \\
  \locbound(\absevent, c) \triangleq \infty  & o.w.
\end{array}
\]
$SCC(\absG(c))$ is the set of all the strong connected components of $\absG(c)$.
\end{defn}
We look at the strongly connected components of $\absG(c)$. 
If the edge does not belong to any strongly connected components, then the local bound is $1$, representing the fact that the edge is not in a loop and so it gets executed at most once.
If the edge belongs to a strongly connected component and one of the variables $x$ in $dc$ decreases, then the local bound is $x$.
Otherwise, if the edge belongs to a strongly connected component and there is a variable $y$ that decreases in the difference constraint of some other edge, and if by removing this other edge, the original edge does not belong anymore to the strongly connected components of $\absG(c)$, then the local bound is $y$.
Otherwise, the local bound is $\infty$. 
Notice that the output is either a symbolic constant in $\constdom$ or a variable that is not an input variable.

\highlight{Soundness Informal}
  The first case is straightforward. 
  For the label $l$ which is not in any while loop, 
  the labeled command with the label $l$ will be 
  evaluated at most once. 
  The second and third cases are guaranteed by the \emph{Discussion on Soundness} in Section 4 in~\cite{sinn2017complexity}.
  We formalized the soundness and proof by Lemma~\ref{lem:local_bound_sound} in Appendix~\ref{apdx:reachability_soundness}.
\item \textbf{Reachability-bound Estimation}
This step aims at determining the \emph{reachability-bound} $\absclr(e, c)$ of every edge $e\in \progE(c)$.
Every bound is a symbolic expression built out of symbols in $\constdom$ and the operations $+, *, \max$.
For every edge, if the local bound of this edge computed at the previous step is a symbol in $\constdom$ then this is already the reachability-bound. 
If instead the local bound of the edge is a variable $y$ which is not an input variable, this step will eliminate it and replace it with a symbolic expression.
In order to do this, this steps will compute two quantities: first, it will recursively sum the reachability-bounds of all the edges whose difference constraint may increment the variable $y$, plus the corresponding increment;
second, it will recursively sum the reachability-bounds of all the edges whose difference constraint may reset the variable $y$ to a (symbolic) expression that doesn't depend on it, multiplied by the maximal value of this symbolic expression. The sum of these two quantities provides the symbolic expression that is an upper bound on the number of times the edge can be reached.

To compute these two quantities we use two mutually recursive procedures in a path-insensitive manner.

\highlight{Notations:}
\\ 
Bound on the maximum value of $\locbound(\absevent, c)$ assigned to edge $\absevent \in \absE(c)$:
% , the 
$ \varinvar: (\mathcal{V} \cup \constdom  \times \cdom) \to \mathcal{A}_{\lin}$
\\
the bound on the iteration times of each corresponding edge, $\absclr(\absevent, c)$:
$\absclr: (\absevent \times \cdom) \to \mathcal{A}_{\lin}$

\highlight{Computations:}

The first one estimates the upper bound, $\varinvar(x, c) \in \mathcal{A}_{\lin}$
on the maximum value for each local bound  $x \in  \mathcal{V} \cup \constdom$.
For a program $c$, the \emph{local bound} of a
$\varinvar(\locbound(\absevent, c)) \in \mathcal{A}_{\lin}$ is 
the bound on the maximum value of the local bound 
assigned to the edge $\absevent \in \absE(c)$, formally in Definition~\ref{def:ranking_bound} and~\ref{def:edge_pathinsensitivebound}.
\begin{defn}[Local Bound Computation]
  \label{def:ranking_bound}
For a program $c$ and an edge $\absevent \in \absE(c)$,
the \emph{local bound}, $\varinvar(\locbound(\absevent, c), c)$ for the local bound $\locbound(\absevent, c)$
of this edge
is computed as follows,
  \[ 
\begin{array}{lll}
  \varinvar(x, c) & \triangleq x & x \in \constdom \\
  \varinvar(x, c) & \triangleq \incrs(x, c) + 
  \max(\{\varinvar(y, c) + v ~\mid~ (l, x' \leq y + v, l') \in \reset(x)\}) & x \notin \constdom
\end{array}
\]
%
$\incrs(x, c) \triangleq \sum\limits_{\absevent \in \inc(x)}\{\absclr(\absevent, c) \times v ~\mid~ 
\absevent = (l, x' \leq x + v, l')\}$
\end{defn}

\begin{defn}[Transition Bound]
  \label{def:edge_pathinsensitivebound}
  For a program $c$ and an edge $\absevent \in \absE(c)$, the \emph{path-insensitive transition bound},
  $\absclr(\absevent, c) \in \mathcal{A}_{\lin}$ 
for this edge is
computed as follows,
\[ 
\begin{array}{lll}
  \absclr(\absevent, c) 
  & \triangleq \varinvar(\locbound(\absevent, c), c)   \qquad \qquad  \text{if} \quad  \locbound(\absevent, c) \in \constdom & \\
  \absclr(\absevent, c) 
  & \triangleq \incrs(x, c) 
   + 
  \sum\limits_{\absevent' \in \reset(x, c) \land \absevent' = (l, x \leq y + v, l') }
  \Big( \absclr(\absevent', c) \times \big( \varinvar(y, c) + v \big) \Big)
  & \\
  &  \text{if} \quad  \locbound(\absevent, c) = x \land x \notin \constdom & ,
\end{array}
  \]
\end{defn}

Then we construct the set of reachability bound $w$ for every program point $l$, as $\absW(c)$.
For each pair $(l, w) \in \absW(c)$, $w = \sum\left\{  \absclr(\absevent, c) \middle\vert \absevent = (l, \_, \_) \right\}$.

\highlight{Theorem Guarantee}
For a program $c$ and an edge $\absevent \in \absE(c)$,
$\absclr(\absevent)$ is a sound upper bound
on the execution times of this transition by paper~\cite{sinn2017complexity}.
The soundness theorem is attached in Theorem~\ref{thm:transition_bound_sound} in Appendix~\ref{apdx:reachability_soundness}.
%
\begin{thm}[Soundness of the Transition Bound]
  \label{thm:transition_bound_sound}
For each program ${c}$ and an edge $\absevent = (l, \_, \_) \in \absG(c)$, if $l$ is the label of an assignment command,
%  label $l \in \lvar(c)$,
then its \emph{path-insensitive transition bound} $\absclr(\absevent, c)$ 
 is a sound upper bound on 
%  execution-based reachability bound $w^t$ 
the execution times of this assignment command in $c$.
  \[
    \begin{array}{l}
      \forall c \in \cdom, l \in \lvar(c),\trace_0 \in \mathcal{T}_0(c), 
      \trace \in \mathcal{T}, v \in \mathbb{N}
       \st 
       \config{{c}, \trace_0} \to^{*} \config{\eskip, \trace_0\tracecat\vtrace} 
       \land \config{\absclr(\absevent, c), \trace_0} \aarrow v
       \land
      \vcounter(\trace, l) \leq v
    \end{array}
    \]
\end{thm}
%
\paragraph*{Example}
We perform the symbolic reachability bound analysis on the abstract control flow graph in Figure~\ref{fig:abscfg_tworound}(b) and compute the result in Figure~\ref{fig:abscfg_tworound}(c).
We would like to generate the closure of every edge, which is an equality relation between variables.  Solving this closure gives us the reachability bound for this edge. With all the bound for all the edges in the abstract control flow graph, we can calculate the weight for every vertex in this graph. For example, we show the closure generated for the edge 
$4 \xrightarrow{j' \leq j - 1} 5$, 
$\absclr(4 \xrightarrow{j' \leq j - 1} 5) = \varinvar(j)$. The invariant for variable $j$, $\varinvar(j)$ used here is 
$\varinvar(j) = k * \absclr(1 \xrightarrow{i' \leq k} 2)$, 
which is generated by all the difference constraints involving $j$ in the graph.
Notice the $k$ in $\varinvar(j)$ comes from considering both difference constraint $j' \leq k$ from edge
$1 \xrightarrow{j' \leq k} 2$ and $j'\leq j - 1$ from $4 \xrightarrow{j' \leq j - 1} 5$, which intuitively reflects the while loop whose counter is set to $k$ at the beginning and decreases by 1 at each iteration. 
With all the closures for all the edges of the abstract control flow graph, we can solve them to obtains the reachability bound of every edge. We decide the weight for every vertex in the abstract control flow graph by using the bound of the edges which head out from this vertex, by taking the max of the bound from these involving edges. For instance,   
By the constraint on the edge $4 \xrightarrow{j' \leq j - 1} 5$, we get bound $k$ for this edge.
Then, we assign vertex $4$ by reachability bound $k$, as in Figure~\ref{fig:abscfg_tworound}(c). 
Another interesting vertex is $2$, which has more than one edge heading out from it,$2 \xrightarrow{j \geq 0} 3$ and 
$2 \xrightarrow{j \leq 0} 6$. For the weight for vertex $2$, 
we choose the max between the bound $k$ from $2 \xrightarrow{j \geq 0} 3$ and $1$ from $2 \xrightarrow{j \leq 0} 6$.
The same way for the rest weights' computation.
We use $\absW(c)$ for the set of weights we just computed 
for each label in the abstract control flow graph of $c$.
%
The same way for the rest weights' computation.
\end{enumerate}

\paragraph{Vertex Weight Computation}
Using the reachability-bound $\absclr(e, c)$ for every edge $e = (l, dc, l')$ we can provide a bound on the visiting times of each vertex $x^l \in \absV(c)$.
The weight $\progW(c)$
for each vertex $x^l \in \lvar(c)$ maps to a symbolic expression over $\constdom$.
$\progW(c) \in \mathcal{P}(\mathcal{LV} \times \mathcal{A}_{\lin})$ is formally computed
as follows.
\highlight{
% :
% \\
 \[\progW(c) \triangleq
   \left\{ (x^l, w) 
\mid
x^l \in \progV(c) \land (l, w) \in \absW(c)
\right\}.
\]
}
%
Notice that $w \in \mathcal{A}_{\lin}$ is a symbolic arithmetic expression over symbols in $\constdom$. In particular, it may contain the input variables and so it may effectively be used as a function of the input - and capture loop bounds in terms of these inputs.
 
We prove that this 
symbolic expression for $x^l \in \progV(c)$ is a sound upper bound of 
the weight for the same vertex $x^l$ in Program's semantics-based dependency graph by Lemma~\ref{lem:weights_map}
in Appendix~\ref{apdx:adapt_soundness}, and Theorem~\ref{thm:rb_soundness} in Appendix~\ref{apdx:weight_soundness}.

\begin{thm}[Soundness of the Reachability Bounds Estimation]
  \label{thm:addweight_soundness}
Let  ${c}$ be a program and $\progW(c)$ be its estimated weight set.
Then, for each $(x^l, w) \in \progW(c),\vtrace_0 \in \mathcal{T}_0(c), \trace \in \mathcal{T},
v \in \mathbb{N}$ we have:
  %
\[
 \text{ if }
\config{{c}, \trace_0} \to^{*} \config{\eskip, \trace_0 \tracecat\vtrace} 
\land 
\config{\vtrace_0, w} \earrow v
\land
% \right\} 
\text{ then }
\vcounter(\vtrace, l) \leq v
\]
\end{thm}
Notice that in this theorem, the evaluation $\config{\vtrace_0, w} \earrow v$ is needed in order to obtain a concrete value $v$ from the symbolic weight $w$ by specifying a value for the input variables through $\vtrace_0$.


\paragraph*{Example}
Going back to the
quantitative dependency graph for two-round example in
Figure~\ref{fig:kadaptwhile_alg}(c), which we aim to estimate.
%
Every vertex from $\progV(c)$ in this graph corresponds to a labeled variable, for example $a^5$,
and this label $5$ is also a vertex $5$ in the abstract control flow graph in Figure~\ref{fig:abscfg_tworound}(b).
%
% we infer the value bound for $j$, which is $k$.
% Then the reachability bound for this abstract transition, (i.e., edge $4, j \leq j - 1, 5$) 
Then, it is straight forward, 
that the reachability bound for the label $5$, 
is also the maximum visiting times bound of the labeled variable $a^5$.
So, we estimate the visiting time for  labeled variable $a^5$ in estimated dependency graph in Figrue~\ref{fig:abscfg_tworound}(c) as $k$ as well.
%
The same way for the rest weights' computation.