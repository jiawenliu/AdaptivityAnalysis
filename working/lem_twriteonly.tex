\todo{
\begin{lem}[Trace is Written Only]
\label{lem:twriteonly}
Given a program $c$ with starting trace $t_1$ and $t_2$,
for arbitrary starting memory $m$ and while map $w$,
if there exist evaluations
$$\config{m, c, t_1, w} \to^{*} \config{m_1', \eskip, t_1', w_1'}$$
% 
$$\config{m, c, t_2, w} \to^{*} \config{m_2', \eskip, t_2', w_2'}$$
%
then:
%
\[
	m_1' = m_2' \land w_1' = w_2'
\]
\end{lem}
%
\begin{subproof}[Proof of Lemma~\ref{lem:tunique}]
%
By induction on program $c$, we have following cases:
%
\caseL{ $[\assign x \expr]^{l}$}
%
From the evaluation rule \rname{assn1} and rule \rname{assn2}, we know there exist evaluations
$$
\config{m, [\assign x \expr]^{l}, t_1, w} \to^{*} \config{m[v/x], \eskip, t_1, w}
$$
%
\[
\config{m, [\assign x \expr]^{l}, t_2, w} \to^{*} \config{m[v/x], \eskip, t_2, w}
\]
%
for starting trace $t_1$ and $t_2$ respectively, where $\config{m, \expr} \aarrow^{*} v$.
%
Since the resulting memories and while maps in the 2 evaluations are still $m$ and $w$, this case is proved.
%
\caseL{$[\assign{x}{\query(\qexpr)}]^{l}$}
%
By repeatedly applying the operational semantics rule \rname{query-e}, we know there exist evaluations
%
\[
\config{m, [\assign{x}{\query(\qexpr)}]^{l}, t_1, w} \to^{*} \config{m, [\assign{x}{\query(\qval)}]^{l}, t_1, w}
\]
%
\[
\config{m, [\assign{x}{\query(\qexpr)}]^{l}, t_2, w} \to^{*} \config{m, [\assign{x}{\query(\qval)}]^{l}, t_2, w}
\]
%
for starting trace $t_1$ and $t_2$ respectively, where $\config{m, \qexpr} \qarrow^{*} \qval$.
%
\\
%
According to the evaluation rule \rname{query-v}, we then have the following 2 transitions with starting trace $t_1$ and $t_2$ respectively:
%
\[
\config{m, [\assign{x}{\query(\qval)}]^{l}, t_1, w} \to \config{m[\qval/x], \eskip, t_1 ++ [(\qval, l, w)], w}
\]
%
\[
\config{m, [\assign{x}{\query(\qval)}]^{l}, t_2, w} \to \config{m[\qval/x], \eskip, t_2 ++ [(\qval, l, w)], w}
\]
%
The resulting memories in the two evaluations are both $m[\qval/x]$, and resulting while map are $w$.
\\
%
This case is proved.
%%
\caseL{$\eif([b]^l, c_1, c_2)$}
%
According to the evaluation rule \rname{if}, we know there exist 2 evaluations for $t_1$ and $t_2$ respectively by repeatedly applying the rule \rname{if}: 
%
\[
\config{m, \eif([b]^l, c_1, c_2), t_1, w} \to^{*} \config{m, \eif([v]^l, c_1, c_2), t_1, w} ~ (1)
\]
%
\[
\config{m, \eif([b]^l, c_1, c_2), t_2, w} \to^{*} \config{m, \eif([v]^l, c_1, c_2), t_2, w} ~ (2)
\]
%
, where $\config{m, b} \barrow^{*} v$ and $v$ is either $\etrue$ or $\efalse$.
\\
Considering 2 subcases where $v$ is either $\etrue$ or $\efalse$:
%
\subcaseL{$v = \etrue$}
%
By applying the evaluation rule \rname{if-t}, we have:
\[
\config{m, \eif([\etrue]^l, c_1, c_2), t_1, w} \to \config{m, c_1, t_1, w} ~ (3)
\]
%
\[
\config{m, \eif([\etrue]^l, c_1, c_2), t_2, w} \to \config{m, c_1, t_2, w} ~ (4)
\]
%
By induction hypothesis on $c_1$ with $m$, $w$ $t_1$ and $t_2$, we know there exist 2 evaluations:
\[
	\config{m, c_1, t_1, w} \to^{*} \config{m_1, \eskip, t_1', w_1} ~ (5)
\]
%
\[
	\config{m, c_1, t_2, w} \to^{*} \config{m_2, \eskip, t_2', w_2} ~ (6)
\]
%
, where $m_1 = m_2$ and $w_1 = w_2$.
\\
According to the evaluations $(1), (3), (5)$ and $(2), (4), (6)$, we have the following 2 evaluations:
% 
\[
	\config{m, \eif([b]^l, c_1, c_2), t_1, w} \to^{*} \config{m_1, \eskip, t_1', w_1}
\]
%
\[
	\config{m, \eif([b]^l, c_1, c_2), t_2, w} \to^{*} \config{m_2, \eskip, t_2', w_2}
\]
%
Since $m_1 = m_2$ and $w_1 = w_2$, this sub-case is proved.
%
\subcaseL{$v = \efalse$}
%
By applying the evaluation rule \rname{if-f}, we have:
%
\[
\config{m, \eif([\efalse]^l, c_1, c_2), t_1, w} \to \config{m, c_2, t_1, w} ~ (3)
\]
%
\[
\config{m, \eif([\efalse]^l, c_1, c_2), t_2, w} \to \config{m, c_2, t_2, w} ~ (4)
\]
%
By induction hypothesis on $c_2$ with $m$, $w$ $t_1$ and $t_2$, we know there exist evaluations:
\[
	\config{m, c_2, t_1, w} \to^{*} \config{m_1', \eskip, t_1'', w_1'} ~ (5)
\]
%
\[
	\config{m, c_2, t_2, w} \to^{*} \config{m_2', \eskip, t_2'', w_2'} ~ (6)
\]
%
, where $m_1' = m_2'$ and $w_1' = w_2'$.
\\
According to the evaluations $(1), (3), (5)$ and $(2), (4), (6)$, we have the following 2 evaluations:
% 
\[
	\config{m, \eif([b]^l, c_1, c_2), t_1, w} \to^{*} \config{m_1', \eskip, t_1'', w_1'}
\]
%
\[
	\config{m, \eif([b]^l, c_1, c_2), t_2, w} \to^{*} \config{m_2', \eskip, t_2'', w_2'}
\]
%
Since $m_1' = m_2'$ and $w_1' = w_2'$, this sub-case is proved.
%
\caseL{$c_1; c_2$}
%
According to the evaluation rule \rname{seq1}, we have the following 2 evaluations by repeatedly applying \rname{seq1} 
%
\[
\config{m, c_1;c_2, t_1, w} \to^{*} \config{m_1^1, [\eskip]^l;c_2,  t_1^1, w_1^1} ~ (1)
\]
%
%
\[
\config{m, c_1;c_2, t_2, w} \to^{*} \config{m_2^1, [\eskip]^l;c_2,  t_2^1, w_2^1} ~ (2)
\]
%
where 
%
\[
	\config{m, c_1, t_1, w} \to^{*} \config{m_1^1, [\eskip]^l, t_1^1, w_1^1]} ~ (3)
\]
%
%
\[
\config{m, c_1, t_2, w} \to^{*} \config{m_2^1, [\eskip]^l,  t_2^1, w_2^1} ~ (4)
\]
%
%
By induction hypothesis on $c_1$ and $(3), (4)$, we know $m_1^1= m_2^1$ and $w_1^1= w_2^1$.
%
\\
%
According to the evaluation rule \rname{seq2}, let $m^1 = m_1^1= m_2^1$ and $w^1 = w_1^1= w_2^1$, we have:
%
\[
\config{m^1, [\eskip]^l;c_2, t_1^1, w^1} \to \config{m^1, c_2, t_1^1, w^1}
\]
%
\[
\config{m^1, [\eskip]^l;c_2, t_2^1, w^1} \to \config{m^1, c_2, t_2^1, w^1}
\]
%
%
%
By induction hypothesis on $c_2$, $m^1$, $w^1$, $t_2^1$ and $t_1^1$ we have the following 2 evaluations 
%
\[
	\config{m^1, c_2, t_1^1, w^1} \to^{*} \config{m_1^2, \eskip, t_1^2, w_1^2]}
\]
%
\[
	\config{m^1, c_2, t_2^1, w^1} \to^{*} \config{m_2^2, \eskip, t_2^2, w_2^2]}
\]
%
and $m_1^2= m_2^2$ and $w_1^2= w_2^2$.
%
%
This case is proved.
%
\caseL{$\ewhile [b]^l \edo c'$}
%
According to the \rname{while-b} and \rname{ifw-b} rule, we have following 2 evaluations:
%
\[
	\config{m, \ewhile [b]^l \edo c', t_1, w} \to^{*}
	\config{m, \eif_w([v]^l, c';\ewhile [b]^l, 2 \edo c', \eskip), t_1, w} ~ (w1)
\]
%
\[
	\config{m, \ewhile [b]^l \edo c', t_2, w} \to^{*}
	\config{m, \eif_w([v]^l, c';\ewhile [b]^l, 2 \edo c', \eskip), t_2, w}
\]
%
where $v$ is either $\etrue$ or $\efalse$.
%
\subcaseL{$v = \etrue$ }
%
By evaluation rule \rname{ifw-true}, we have:
%
\[
	\config{m, \eif_w([\etrue]^l, c';\ewhile [b]^l \edo c', \eskip), t_1, w} 
	\to^{*} \config{m, c';\ewhile [b]^l, 2 \edo c', t_1, w+l} ~ (w2)
\]
%
\[
	\config{m, \eif_w([\etrue]^l, c';\ewhile [b]^l \edo c', \eskip), t_2, w} 
	\to^{*} \config{m, c';\ewhile [b]^l, 2 \edo c', t_2, w+l} 
\]
%
%
By rule \rname{seq1}, we have the following evaluations:
%
\[
	\config{m, c';\ewhile [b]^l \edo c', t_1, w+l} \to^{*} \config{m_1', \eskip;\ewhile [b]^l \edo c', t_1', w_1'}~ (w3)
\] 
%
%
\[
	\config{m, c';\ewhile [b]^l \edo c', t_2, w+l} \to^{*} \config{m_2', \eskip;\ewhile [b]^l \edo c', t_2', w_2'}
\] 
%
, where 
$\config{m, c', t_1, w+l} \to^{*} \config{m_1', \eskip, t_1', w_1'} ~ (w4)$
and
$\config{m, c', t_2, w+l} \to^{*} \config{m_2', \eskip, t_2', w_2'}$. 
%
\\
%
By induction hypothesis on $c'$, we know:
\[
m_1' = m_2' \land w_1' = w_2'.
\]
%
By the rule \rname{seq2}, let $m' = m_1' = m_2'$ and $w' = w_1' = w_2'$ we have:
%
\[
	\config{m', \eskip;\ewhile [b]^l \edo c', t_1', w'} \to \config{m', \ewhile [b]^l, 2 \edo c', t_1', w'} ~ (w5)
\]
%
\[
	\config{m', \eskip;\ewhile [b]^l \edo c', t_2', w'} \to \config{m', \ewhile [b]^l, 2 \edo c', t_2', w'}
\]
%
By the termination of the while loop, we know there exist 2 evaluations for $t_1'$ and $t_2'$ respectively,
by repeating the evaluation in $(w1), (w2), (w3)$ and $(w5)$:
%
\[
	\config{m', \ewhile [b]^l \edo c', t_1', w'} \to^{*} \config{m_1^n, \eif_w([\efalse]^l, c';\ewhile [b]^l, n + 1 \edo c', \eskip), t_1^n, w_1^n}
\]
%
\[
	\config{m', \ewhile [b]^l \edo c', t_2', w'} \to^{*} \config{m_2^n, \eif_w([\efalse]^l, c';\ewhile [b]^l, n + 1 \edo c', \eskip), t_2^n, w_2^n}
\]
%
By the evaluation $(w4)$ and induction hypothesis on each iteration, we know the following invariant holds for the resulted memories $m_1^i, m_2^i$ and while maps $w_1^i, w_2^i$,
in evaluation results of each iteration $i = 2, \ldots, n$, where $n \geq 1$. 
%
\[
m_1^i = m_2^i \land w_1^i = w_2^i. ~ (w6)
\]
%
%
By evaluation rule \rname{ifw-false}, we have:
%
\[
	\config{m_1^n, \eif_w([\efalse]^l, c';\ewhile [b]^l \edo c', \eskip), t_1^n, w_1^n} 
	\to^{*} \config{m_1^n, \eskip, t_1^n, w_1^n - l}
\]
%
\[
	\config{m_2^n, \eif_w([\efalse]^l, c';\ewhile [b]^l \edo c', \eskip), t_2^n, w_2^n} 
	\to^{*} \config{m_2^n, \eskip, t_2^n, w_2^n - l}
\]
%
Since $m_1^n = m_2^n$ and $w_1^n - l = w_2^n - l.$, this sub-case is proved.
%
\subcaseL{$v = \efalse$}
%
By evaluation rule \rname{ifw-false}, we have:
%
\[
	\config{m, \eif_w([\efalse]^l, c';\ewhile [b]^l \edo c', \eskip), t_1, w} 
	\to^{*} \config{m, \eskip, t_1, w - l}
\]
%
%
\[
	\config{m, \eif_w([\efalse]^l, c';\ewhile [b]^l \edo c', \eskip), t_2, w} 
	\to^{*} \config{m, \eskip, t_2, w - l}
\]
%
Because $w - l = w - l$ and $m = m$, this sub-case is proved.
%
\end{subproof}
}