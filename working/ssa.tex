\section{Labeled SSA Language}
%
%
\subsection{SSA form Language}
\[
\begin{array}{llll}
 \mbox{Arithmetic Operators} 
& \oplus_a & ::= & + ~|~ - ~|~ \times 
%
~|~ \div \\  
\mbox{Boolean Operators} 
& \oplus_b & ::= & \lor ~|~ \land
\\
  %
\mbox{Relational Operators} 
& \sim & ::= & < ~|~ \leq ~|~ == 
\\  
%
\mbox{Label} 
& l & := & \mathbb{N} 
\\ 
%
\mbox{SSA Arithmetic Expression} 
& \saexpr & ::= & 
n ~|~ \ssa{x} ~|~ \saexpr \oplus_a \saexpr  
\\
%
\mbox{SSA Boolean Expression} & \sbexpr & ::= & 
	%
	\etrue ~|~ \efalse  ~|~ \neg \sbexpr
	 ~|~ \sbexpr \oplus_b \sbexpr
	%
	~|~ \saexpr \sim \saexpr 
	\\
%
\mbox{SSA Query Expression} 
& \ssa{\qexpr} & ::= 
& { \qval ~|~ \saexpr ~|~ \qexpr \oplus_a \qexpr} 
\\
%
\mbox{Query Value} & \qval & ::= 
& {n ~|~ \chi[n] ~|~ \chi[n] \oplus_a  \chi[n] ~|~ n \oplus_a  \chi[n]
~|~ \chi[n] \oplus_a  n}
\\
%
\mbox{Value} 
& v & ::= & { n \sep \etrue \sep \efalse ~|~ [] ~|~ [v, \dots, v]}  
\\
%
\mbox{SSA Expression} & \sexpr & ::= & v ~|~ \saexpr \sep \sbexpr ~|~ [\expr, \dots, \expr]
\\	
%
\mbox{Labeled SSA Command} 
& \ssa{c} & ::= &   [\assign {\ssa{x}}{ \ssa{\expr}}]^{l} ~|~  [\assign {\ssa{x} } {\ssa{\query(\qexpr)}}]^{l}
%
~|~  {{\eifvar(\bar{\ssa{x}}, \bar{\ssa{x}}')}} 
%
\\ 
&&& 
{\ewhile ~ [ \sbexpr ]^{l} , n,
~ 
[\bar{\ssa{x}}, \bar{\ssa{x_1}}, \bar{\ssa{x_2}}] 
~ \edo ~  \ssa{c} }
\\
&&&
~|~ \ssa{c};\ssa{c}  
~|~ [\eif(\sbexpr, [\bar{\ssa{x}}, \bar{\ssa{x_1}}, \bar{\ssa{x_2}}] , \ssa{c}, \ssa{c})]^l 
~|~ [\eskip]^{l} 
\\
%
\mbox{Event} 
& \event & ::= & (\ssa{x}, v, l, n) | (\ssa{x}, \qval, v, l, n) | (\sbexpr, v, l, n)
\\
%
\mbox{Trace} & \vtrace
& ::= & \dot | \vtrace \cdot \event
\\
%
\mbox{Environment} 
& \env & ::= & \vtrace \to v \cup \{\bot\}
\\
%
\mbox{Query Environment} 
& \qenv & ::= & \vtrace \to \qval \cup \{\bot\}
\\
%
\mbox{Variable Counter} & \vcounter
& ::= & \mathcal{SVAR} \to \mathbb{N}
\end{array}
\]
We use following notations to represent the set of corresponding terms:
\[
\begin{array}{lll}
\mathcal{SVAR} & : & \mbox{Set of Variables}  
\\ 
%
\mathcal{VAL} & : & \mbox{Set of Values} 
\\ 
%
%
 \eventset  & : & \mbox{Set of Events}  
\\
%
\mathcal{SM}  & : & \mbox{{Set of SSA Memories}} 
\\
%
\dbdom  & : & \mbox{{Set of Databases}} 
\\
%
\qdom = {[-1,1]} & : & \mbox{{Domain of Query Results}}
\end{array}
\]
%
%
Environment :
\[
\env(\vtrace \cdot (x, v, l, n)) x \triangleq v
~~
\env(\vtrace \cdot (y, v, l, n)) x \triangleq \env(\vtrace) x
~~
\env(\cdot) x \triangleq \bot
\]
%
% The concatenation operation between 2 traces $t_1$ and $t_2$, i.e., $t_1 ++ t_2$ is the standard list concatenation operation as follows:
% \begin{equation}
%     t_1 ++ t_2  
%     \triangleq \left\{
%     \begin{array}{ll} 
%       t_2         & t_1 = []\\
%       \event::(t_1' ++ t_2)  & t_1 = \event::t_1'
%     \end{array}
%     \right.
%   \end{equation}
% %
% %
% The subtraction operation between 2 traces $t_1$ and $t_2$, i.e., $t_1 - t_2$ is defined as follows:
% \begin{equation}
%     t_1 - t_2  
%     \triangleq t_3 ~ s.t., t_2 ++ t_3 = t_1
%   \end{equation}
%
%
\subsection{Trace-based Operational Semantics for SSA Language}
%
%
%
{
\begin{mathpar}
\boxed{ \config{\trace,\aexpr} \aarrow v \, : \, Trace  \times AExpr \Rightarrow Value }
\\
\boxed{ \config{\trace, \bexpr} \barrow v \, : \, Trace \times BExpr \Rightarrow Value }
\end{mathpar}
}
%
Given the evaluation for the arithmetic and boolean expression, we defined the evaluation rules for query expression $\qexpr$ correspondingly as follows:
  \begin{mathpar}
  \boxed{ \config{\trace, \qexpr} \qarrow \qval \, : \, Trace  \times QExpr \qarrow QValue }
  \\
  \inferrule{ 
    \config{\trace, \qexpr_1} \qarrow \qval_1
    \and
    \config{m, \qexpr_2} \qarrow \qval_2
  }{
   \config{\trace,  \qexpr_1 \oplus_a \qexpr_2} 
   \qarrow n'
  }
  \and
  \inferrule{ 
    \config{m, \aexpr} \aarrow v
  }{
   \config{m,  \chi[\aexpr]} \qarrow \chi[v]
  }
  \end{mathpar}
%
%
\begin{figure}
\jl{
  \begin{mathpar}
  \boxed{
  Command \times Trace \times VCounter
  \xrightarrow{}
  Command \times Trace \times VCounter
  }
  \\
  \boxed{\config{\ssa{c, \vtrace, \vcounter}}
  \xrightarrow{} 
  \config{\ssa{c',  \vtrace', \vcounter'}}
  }
  \\
  %
  \inferrule
  {
  \vcounter'(\ssa{x}) = \vcounter(\ssa{x}) + 1
  \and 
  \event = (\ssa{x}, l, \vcounter'(\ssa{x}), v)
  }
  {
  \config{[\assign{\ssa{x}}{\saexpr}]^{l},  \vtrace, \vcounter} 
  \xrightarrow{} 
  \config{[\eskip]^{l}, \vtrace \cdot \event, \vcounter'}
  }
  ~\textbf{assn}
  %
  \and
  %
  \inferrule
  {
   \vtrace, b \barrow \etrue
   \and 
   \event = (b, \etrue, l, \vcounter(l))
   \and 
   \vcounter'(l) = \vcounter(l) + 1
  }
  {
  \config{\ssa{\ewhile ~ [b]^{l}, n, [\bar{{x}}', \bar{{x_1}}, \bar{{x_2}}] ~ \edo ~ c, \vtrace, \vcounter}}
  \\
  \xrightarrow{} 
  \config{\ssa{ m, 
  c[\bar{x_i}/\bar{x'}]; \ewhile ~ [b]^{l}, (n + 1), [\bar{{x}}', \bar{{x_1}}, \bar{{x_2}}]  ~ \edo ~ c,  \eskip),
  \vtrace\cdot \event, \vcounter'}}
  }
  ~\textbf{ssa-while-t}
  %
  %
    \and
  %
  \inferrule
  {
   \vtrace, b \barrow \efalse
   \and 
   \event = (b, \efalse, l, \vcounter(l))
   \and 
   \vcounter'(l) = \vcounter(l) + 1
  }
  {
  \config{\ssa{\ewhile ~ [b]^{l}, n, [\bar{{x}}', \bar{{x_1}}, \bar{{x_2}}] ~ \edo ~ c, \vtrace, \vcounter}}
  \\
  \xrightarrow{} 
  \config{\ssa{ m, 
  [\eskip]^l  \eskip),
  \vtrace \cdot \event, \vcounter'}}
  }
  ~\textbf{ssa-while-f}
  %
  \and
  %
 \inferrule
  {
   \vtrace, b \barrow \efalse
   \and 
   \event = (b, \efalse, l, \vcounter(l))
   \and 
   \vcounter'(l) = \vcounter(l) + 1  }
  {
  \config{
  \ssa{
  \eif ([b]^l, [\bar{{x}}', \bar{{x_1}}, \bar{{x_2}}], n,  
  c; \ssa{\ewhile ~ [b]^{l}, n, [\bar{{x}}', \bar{{x_1}}, \bar{{x_2}}]  ~ \edo ~ c},
  \eskip), \vtrace, \vcounter
  }}
  \\
  \xrightarrow{} 
  \config{
  \ssa{m, 
  {
  c; \ssa{\ewhile ~ [b]^{l}, n, [\bar{{x}}', \bar{{x_1}}, \bar{{x_2}}]  ~ \edo ~ c},
  }
  }
  \vtrace, \vcounter}
  }
  ~\textbf{ssa-if-true}
  %
  %
  \and
  {
  \inferrule
  {
  \query(\qval) = v
  \and
  \vcounter'[\ssa{x}] = \vcounter[\ssa{x}] + 1
  \and 
  \event = (\ssa{x}, \qval, l, \vcounter'[\ssa{x}])
  }
  {
  \config{\ssa{m, [\assign{x}{\query(\qexpr)}]^l, \vtrace, \vcounter}}
  \xrightarrow{} 
  \config{\ssa{m[v/ x], \eskip,  \vtrace ++ [\event], \vcounter'} }
  }
  ~\textbf{ssa-query}
  }
  %
  \and
  %
  %
  \inferrule
  {
  \config{\ssa{m, c_1, \vtrace, \vcounter}}
  \xrightarrow{}
  \config{\ssa{m', c_1',  \vtrace', \vcounter'}}
  }
  {
  \config{\ssa{m, c_1; c_2, \vtrace, \vcounter}} 
  \xrightarrow{} 
  \config{\ssa{m', c_1'; c_2, \vtrace', \vcounter'}}
  }
  ~\textbf{ssa-seq1}
  %
  ~~~~~~~
  %
  \inferrule
  {
  }
  {
  \config{\ssa{m, [\eskip]^{l} ; c_2, \vtrace, \vcounter}} \xrightarrow{} \config{\ssa{m, c_2, \vtrace, \vcounter}}
  }
  ~\textbf{ssa-seq2}
  %
  \and
  %
  %
  \inferrule
  {
     \vtrace, b \barrow \etrue
   \and 
   \event = (b, \etrue, l, \vcounter(l))
   \and 
   \vcounter'(l) = \vcounter(l) + 1 
  }
  {
  \config{\ssa{
  \eif([\etrue]^{l}, [\bar{{x}}, \bar{{x_1}}, \bar{{x_2}}],
  [\bar{\ssa{y}}, \bar{\ssa{y_1}}, \bar{\ssa{y_2}}],
  [\bar{\ssa{z}},\bar{\ssa{z_1}}, \bar{\ssa{z_2}}], c_1, c_2), 
  \vtrace, \vcounter}}
  \\
  \xrightarrow{} 
  \config{\ssa{ 
  c_1; 
  \eifvar(\bar{\ssa{x}},\bar{\ssa{x_2}}); 
  \eifvar(\bar{\ssa{y}},\bar{\ssa{y_1}});
  \eifvar(\bar{\ssa{z}},\bar{\ssa{z_2}}), 
  \vtrace \cdot \event, \vcounter'}}
  }
  ~\textbf{ssa-if-t}
  %
  \and
  %
  \inferrule
  {
   \vtrace, b \barrow \efalse
   \and 
   \event = (b, \efalse, l, \vcounter(l))
   \and 
   \vcounter'(l) = \vcounter(l) + 1 
  }
  {
  \config{\ssa{
  \eif([\efalse]^{l},[\bar{{x}}, \bar{{x_1}}, \bar{{x_2}}],
  [\bar{\ssa{y}}, \bar{\ssa{y_1}}, \bar{\ssa{y_2}}],
  [\bar{\ssa{z}},\bar{\ssa{z_1}}, \bar{\ssa{z_2}}], c_1, c_2), \vtrace, \vcounter}}
  \\
  \xrightarrow{} 
  \config{\ssa{m, 
  c_2;
  \eifvar(\bar{\ssa{x}},\bar{\ssa{x_2}}); 
  \eifvar(\bar{\ssa{y}},\bar{\ssa{y_1}});
  \eifvar(\bar{\ssa{z}},\bar{\ssa{z_2}}), 
  \vtrace \cdot \event, \vcounter'}}
  }
  ~\textbf{ssa-if-f}
  %
  \and
  %
  \inferrule
  {
  }
  {
   \config{\ssa{m}, \eifvar(\ssa{\bar{x}, \bar{x}'}), \ssa{\vtrace, \vcounter}} 
   \to 
   \config{\ssa{(\bar{x} \to m(\bar{x}'))::m, \eskip, \vtrace, \vcounter}}
  }~\textbf{ssa-\eifvar}
  % %
  %
  %
  \end{mathpar}
}
  % \end{subfigure}
      \caption{Trace-based Operational Semantics for SSA Language.}
      \label{fig:os_ssa}
  \end{figure}
  %
%
%
\subsection{Event and Trace}
%
%
Events preserve the order of execution. The order relation is defined in Definition~\ref{def:query_dir}.
%
\begin{defn}[Order of Annotated Variables / Events].
\label{def:query_dir}
\\
Given 2 annotated queries 
$\event_1 = (x_1, v_1, l_1, n_1), 
\event_2 = (x_2, v_2, l_2, n_2)$
:
%
\[
\event_1 \eventlt \event_2
 \triangleq 
 \left\{
 \begin{array}{ll}
    n_1 < n_2  
    & l_1 = l_2
    \\
    w_1 <_w w_2 & o.w.
\end{array}  
\right.
\]
%
$\event_1 \eventgeq \event_2$  is defined vice versa.
\end{defn}
%
%
%
Given the evaluation rules for query expression, we define its equivalence relation in Definition~\ref{def:query_equal}.
%
\begin{defn}[Equivalence of Query].
%
\label{def:query_equal}
 Given a memory $m$ and 2 query expressions $\qexpr_1$, $\qexpr_2$ s.t., $FV(\qexpr_1) \in \dom(m)$ and $FV(\qexpr_2) \in \dom(m)$:
$$
\qexpr_1 =_{q}^{m} \qexpr_2 \triangleq
\left\{
    \begin{array}{ll} 
      \etrue      
      & 
    \exists \qval_1, \qval_2.
    \begin{array}{l} 
      (\config{m,  \qexpr_1} \qarrow \qval_1 \land \config{m,  \qexpr_2 } \qarrow \qval_2) 
      \\
      \land (\forall r \in \qdom. \exists v. ~ s.t., ~ 
            \config{m, \qval_1[r/\chi]} \aarrow v \land \config{m,  \qval_2[r/\chi] } \aarrow v)  
    \end{array}\\
      \efalse         
      & \text{o.w.} 
    \end{array}
    \right.
$$
%
, where $FV(\qexpr)$ is the set of free variables in the query expression $\qexpr$.
$\qexpr_1 \neq_{q}^{m} \qexpr_2$  is defined vice versa.
%
We use $=_{q}$  and $\neq_{q}$ as the shorthands for $=_{q}^{[]}$ and $\neq^{[]}_{q}$.
\end{defn}
%
Then, we have the corresponding equivalence relation between 2 annotated queries defined in Definition~\ref{def:av_equal}:
%
\begin{defn}[Equivalence of Events]
%
\label{def:av_equal}
Given 2 annotated queries 
$ \event_1 = (x_1, v_1, l_1, n_1), 
\event_2 = (x_2, v_2, l_2, n_2)$
:
%
\[
\event_1 \eventeq \event_2
 \triangleq (l_1 = l_2 \land  w_1 =_w w_2 \land 
 \qval_1 =_q \qval_2) 
\]
%
$\event_1 \eventneq \event_2$  is defined vice versa.
%
\end{defn}
%
%
%
\todo{
Given an annotated variable $\event$ and a trace $t$,
the appending operation $\event :: t$ is 
the standard list appending operation, appends $\event$ to the head of trace $t$.
%
The concatenation operation between 2 traces $t_1$ and $t_2$, i.e., $t_1 ++ t_2$ is the standard list concatenation operation as follows:
\begin{equation}
    t_1 ++ t_2  
    \triangleq \left\{
    \begin{array}{ll} 
      t_2         & t_1 = []\\
      \event::(t_1' ++ t_2)  & t_1 = \event::t_1'
    \end{array}
    \right.
  \end{equation}
%
%
The subtraction operation between 2 traces $t_1$ and $t_2$, i.e., $t_1 - t_2$ is defined as follows:
\begin{equation}
    t_1 - t_2  
    \triangleq t_3 ~ s.t., t_2 ++ t_3 = t_1
  \end{equation}
%
Given an annotated query $\event$, $\event$ belongs to a trace $t$, i.e., $\event \eventin t$ are defined as follows:
    %
\begin{equation}
    \event \eventin t  
    \triangleq \left\{
    \begin{array}{ll} 
      \efalse       & t = []      \\
      \etrue        & t = \event'::t'  \quad \event \eventeq \event'\\ 
      \event \in t'      & t = \event'::t'  \quad \event \eventneq \event'
    \end{array}
    \right.
  \end{equation}
  %
  %
}
%
\begin{defn}[Equivalence of Program]
%
\label{def:aq_prog}
Given 2 programs $c_1$ and $c_2$:
\[
c_1 =_{c} c_2
 \triangleq 
 \left\{
    \begin{array}{ll} 
      \etrue        
      & c_1 = \eskip \land c_2 = \eskip
      \\ 
      \forall m. \exists v. ~ \config{m, \expr_1} \aarrow^{*} v \land \config{m, \expr_1} \aarrow^{*} v     
      & c_1 = \assign{x}{\expr_1} \land c_2 = \assign{x}{\expr_2} 
      \\ 
      \qexpr_1 =_{q} \qexpr_2       
      & c_1 = \assign{x}{\query(\qexpr_1)} \land c_1 = \assign{x}{\query(\qexpr_2)} 
      \\
      c_1^f =_{c} c_2^f \land c_1^t =_{c} c_2^t
      & c_1 = \eif(b, c_1^t, c_1^f) \land c_2 = \eif(b, c_2^t, c_2^f)
      \\ 
      c_1' =_{c} c_2'         
      & c_1 = \ewhile b \edo c_1' \land c_2 = \ewhile b \edo c_2'
      \\ 
      c_1^h =_{c} c_2^h \land c_1^t =_{c} c_2^t
      & c_1 = c_1^h;c_1^t \land c_2 = c_2^h;c_2^t 
    \end{array}
    \right.
\]
%
$c_1 \neq_{c} c_2$  is defined vice versa.
%
\end{defn}
%
Given 2 programs $c$ and $c'$, $c'$ is a sub-program of$c$, i.e., $c' \in_{c} c$ is defined as:
\begin{equation}
c' \in_{c} c \triangleq \exists c_1, c_2, c''. ~ s.t.,~
c =_{c} c_1; c''; c_2 \land c' =_{c} c''
\end{equation} 
%
\begin{defn}[Assigned Variables ($\avar$)]
Given a program $\ssa{c}$, its assigned variables $\avar$ is a vector containing all variables newly assigned in the program preserving the order. 
It is defined as follows:
$$
  \avar_{\ssa{c}} \triangleq
  \left\{
  \begin{array}{ll}
      [\ssa{x}]                   
      & \ssa{c} = [\ssa{\assign x e}]^{(l, w)} 
      \\
      \left[ \ssa{x} \right]                  
      & \ssa{c} = [\ssa{\assign x \query(\qexpr)}]^{(l, w)} 
      \\
      \avar_{\ssa{c_1}} ++ \avar_{\ssa{c_2}}  
      & \ssa{c} = \ssa{c_1};\ssa{c_2}
      \\
      \avar_{\ssa{c_1}} ++ \avar_{\ssa{c_2}} ++ \ssa{[\bar{x}, \bar{y}, \bar{z}]} 
      & \ssa{c} =\eif([\sbexpr]^{(l, w)} , \ssa{[\bar{x}, \bar{x_2}, \bar{x_2}], 
      [\bar{y}, \bar{y_2}, \bar{y_3}], 
      [\bar{z}, \bar{z_2}, \bar{z_3}], c_1, c_2}) 
      \\
      \avar_{\ssa{c}'} ++ [\ssa{\bar{x}}]
      & \ssa{c}   = \ewhile ([\sbexpr]^{(l, w)}, [\ssa{\bar{x}, \bar{x_2}, \bar{x_2}}], \ssa{c}')
\end{array}
\right.
$$
\end{defn}
%
\begin{defn}[Query Variables ($\qvar$)].
\\
Given a program $c$, its query variables $\qvar$ is a vector containing all variables newly assigned by a query in the programm, $\qvar \subset \mathcal{VAR}$.
It is defined as follows:
$$
  \qvar_{\ssa{c}} \triangleq
  \left\{
  \begin{array}{ll}
      []                  
      & \ssa{c} = [\ssa{\assign x e}]^{(l, w)} 
      \\
      \left[ \ssa{x} \right]                  
      & \ssa{c} = [\ssa{\assign x \query(\qexpr)}]^{(l, w)} 
      \\
      \avar_{\ssa{c_1}} ++ \avar_{\ssa{c_2}}  
      & \ssa{c} = \ssa{c_1};\ssa{c_2}
      \\
      \avar_{\ssa{c_1}} ++ \avar_{\ssa{c_2}} ++ \ssa{[\bar{x}, \bar{y}, \bar{z}]} 
      & \ssa{c} =\eif([\sbexpr]^{(l, w)} , \ssa{[\bar{x}, \bar{x_2}, \bar{x_2}], 
      [\bar{y}, \bar{y_2}, \bar{y_3}], 
      [\bar{z}, \bar{z_2}, \bar{z_3}], c_1, c_2}) 
      \\
      \avar_{\ssa{c}'} ++ [\ssa{\bar{x}}]
      & \ssa{c}   = \ewhile ([\sbexpr]^{(l, w)}, [\ssa{\bar{x}, \bar{x_2}, \bar{x_2}}], \ssa{c}')
\end{array}
\right.
$$
\end{defn}
%
We are abusing the notations and operators from list here. 
The notation $[]$ represents an empty vector
and $x::A$ represents add an element $x$ to the head of the vector $A$.
The concatenation operation between 2 vectors $A_1$ and $A_2$, i.e., $A_1 ++ A_2$ is mimic the standard list concatenation operations as follows:
%
\begin{equation}
    A_1 ++ A_2  
    \triangleq \left\{
    \begin{array}{ll} 
      A_2         & A_1 = []\\
      x::(A_1' ++ A_2)  & A_1 = x::A_1'
    \end{array}
    \right.
\end{equation}
%
We use index within parenthesis to denote the access to the element of corresponding location,
$A(i)$ denotes the element at location $i$ in the vector $A$ and 
$M(i, j)$ denotes the element at location $i$-th raw, $i$-th column in the matrix $M$. 
%
%
%
\begin{defn}[Initial Variable Counter $\vcounter^0_{c}$]
Given a program $c$ with its assigned variables $\avar_{c}$ of length $N$, its initial variable counter $\vcounter^0_{c}$ maps all the variable to $0$, i.e.:
\[
  \vcounter^0_{c}(x) = 0, x = \avar_{c}(i) \forall i = 1, \ldots, N 
\]
\end{defn}
%
%
\subsection{ Trace-based Adaptivity}
%
%
% 
%
%
%
\begin{defn}
[Events May Dependency]
\label{def:event_dep}.
\\
One event $\event_2$ may depend on another one  $\event_1$in a program $\ssa{c}$,
with a starting memory $\ssa{m}$ and hidden database $D$, denoted as 
%
$\eventdep(\event_1, \event_2, c, \ssa{m}, D)$ is defined below. 
%
\[
\begin{array}{ll}
\begin{array}{l}
\exists \ssa{m}, \ssa{m}_1, \ssa{m}_2, \ssa{m}_3, \ssa{m}_2', \ssa{m}_3', 
\vtrace_1, \vtrace_2, \vtrace_2', t_1, t_2, t_2', \ssa{c}_1, \ssa{c}_2, v_1', \sexpr_2.
\\
  \left(
  \begin{array}{l}   
\config{\ssa{m}, \ssa{c}, []} \rightarrow^{*} 
\config{\ssa{m}_1, [\assign{\ssa{x}_1}{v_1}]^{l_1} ; \ssa{c}_1, \vtrace_1, t_1} 
\\ 
 \bigwedge
 \config{\ssa{m}_1[v_1/\ssa{x}_1], c_1, \vtrace_1 ++ [\event_1], t_1[\ssa{x}_1]++} 
 \\
  \qquad \rightarrow^{*} 
  \config{\ssa{m}_2, [\assign{\ssa{x}_2}{\sexpr_2}]^{l_2} ; \ssa{c}_2, \vtrace_2, t_2} 
  \\
  \qquad \rightarrow^{*} 
  \config{\ssa{m}_3, \ssa{c}_2,  \vtrace_2 ++ [\event_2], t_2[\ssa{x}_2]++} 
  % 
 \\ 
 \bigwedge
 \config{\ssa{m}_1[v_1'/\ssa{x}_1], \ssa{c}_1, \vtrace_1, t_1} 
\rightarrow^{*} 
\config{\ssa{m}_2', \ssa{c}_2,  \vtrace_2', t_2'}
\\
\bigwedge
\qenv(\event_2) \neq \qenv(\event'_2)
\end{array}
\right)
\end{array} 
&
\begin{array}{l}
\event_1 = (\ssa{x}_1, v_1, l_1, n_1) 
\text{or} \\
\event_1 = (\ssa{x}_1, \qval_1, v_1, l_1, n_1) 
\end{array}
\\
\begin{array}{l}
\exists \ssa{m}, \ssa{m}_1, \ssa{m}_2, \ssa{m}_3, \ssa{m}_2', \ssa{m}_3', 
\vtrace_1, \vtrace_2, \vtrace_2', t_1, t_2, t_2', \ssa{c}_1, \ssa{c}_2, v_1', {\qexpr}_2.
\\
  \left(
  \begin{array}{l}   
\config{\ssa{m}, \ssa{c}, []} \rightarrow^{*} 
\config{\ssa{m}_1, [\assign{\ssa{x}_1}{\query({\qval}_1)}]^{l_1} ; \ssa{c}_1, \vtrace_1, t_1} 
\\ 
 \bigwedge
 \config{\ssa{m}_1[v_1/\ssa{x}_1], c_1, \vtrace_1 ++ [\event_1], t_1[\ssa{x}_1]++} 
 \\
\qquad \rightarrow^{*} 
\config{\ssa{m}_2, [\assign{\ssa{x}_2}{\query({\qexpr}_2)}]^{l_2} ; \ssa{c}_2, \vtrace_2, t_2} 
\\
\qquad \rightarrow^{*} 
\config{\ssa{m}_3, \ssa{c}_2,  \vtrace_2 ++ [\event_2], t_2[\ssa{x}_2]++} 
  % 
 \\ 
 \bigwedge
 \config{\ssa{m}_1[v_1'/\ssa{x}_1], \ssa{c}_1, \vtrace_1, t_1} 
\rightarrow^{*} 
\config{\ssa{m}_2', \ssa{c}_2,  \vtrace_2', t_2'}
\\
\bigwedge
\event_2 \notin \vtrace_2'
\end{array}
\right)
\end{array} 
&
\begin{array}{l}
\event_1 = (\ssa{b}_1, v_1, l_1, n_1)
\end{array}
\end{array}
 \]
%
\end{defn}
%
\begin{defn}[Variable May Dependency].
\label{def:var_dep}
\\
Given a program $\ssa{c}$ with its assigned variables $\avar_{\ssa{c}}$, 
one variable $\ssa{x}_2 \in \avar_{\ssa{c}}$ may depend on another variable 
$\ssa{x}_1 \in \avar_{\ssa{c}}$ in $\ssa{c}$ denoted as 
%
$\vardep(\ssa{x}_1, \ssa{x}_2, \ssa{c})$ is defined below.
%
\[
\exists \event_1, \event_2, \ssa{m}, D. ~
\projl{\event_1} = \ssa{x}_1
\land
\projl{\event_2} = \ssa{x}_2
\land 
\eventdep(\event_1, \event_2, c, \ssa{m}, D)
\] 
%
%
\end{defn}
%
%
\begin{defn}[Execution Based Dependency Graph].
\\
Given a program $\ssa{c}$, a database $D$, a starting memory $\ssa{m}$ with its assigned variables $\avar_c$ and initial variable counter $\vcounter^0_{c}$ with its corresponding execution:
$\config{\ssa{m}, \ssa{c}, [], \vcounter^0_{\ssa{c}}} 
\to^{*}
\config{\ssa{m'}, \eskip, \vtrace, \vcounter}$,
the dependency graph $\traceG(\ssa{c}, \ssa{m}, D) = (\vertxs, \edges, \weights, \qflag)$ is defined as:
%
\[
\begin{array}{rlcl}
  \text{Vertices} &
  \vertxs & := & \left\{ 
  x \in \mathcal{VAR}
  ~ \middle\vert ~
  x = \avar_{\ssa{c}}(i); i = 0, \ldots, |\avar_{\ssa{c}}| 
  \right\}
  \\
  \text{Directed Edges} &
  \edges & := & 
  \left\{ 
  (x, x') \in \mathcal{VAR} \times \mathcal{VAR}
  ~ \middle\vert ~
  \vardep(x, x', c); 
  x = \avar_{\ssa{c}}(i); x' = \avar_{\ssa{c}}(j); i,j = 0, \ldots, |\avar_{\ssa{c}}| 
  \right\}
  \\
  \text{Weights} &
  \weights & := & 
  \left\{ 
  (x, n) \in \mathcal{VAR} \times \mathbb{N}
  ~ \middle\vert ~
  n = \vcounter(x); x = \avar_{\ssa{c}}(i); i = 0, \ldots, |\avar_{\ssa{c}}|
  \right\}
  \\
  \text{Query Flags} &
  \qflag & := & 
  \left\{(x, n)  \in \vertxs \times \{0, 1\} 
  ~ \middle\vert ~
  \left\{
  \begin{array}{ll}
  n = 1 & x \in \qvar_{\ssa{c}} \\ 
  n = 0 & o.w.
  \end{array}
  \right\};
  x = \avar_{\ssa{c}}(i); i = 0, \ldots, |\avar_{\ssa{c}}|
  \right\}
\end{array}
\]
\end{defn}
%
%
\begin{defn}[Finite Walk ($k$)].
\label{def:finitewalk}
\\
Given a labeled weighted graph $G = (\vertxs, \edges, \weights, \qflag)$, a \emph{finite walk} $k$ in $G$ is a sequence of edges $(e_1 \ldots e_{n - 1})$ 
for which there is a sequence of vertices $(v_1, \ldots, v_{n})$ such that:
\begin{itemize}
    \item $e_i = (v_{i},v_{i + 1})$ for every $1 \leq i < n$.
    \item every vertex $v \in \vertxs$ appears in this vertices sequence $(v_1, \ldots, v_{n})$ of $k$ at most $W(v)$ times.  
\end{itemize}
$(v_1, \ldots, v_{n})$ is the vertex sequence of this walk.
\\
%
Length of this finite walk $k$ is the number of vertices in its vertex sequence, i.e., $\len(k) = n$.
\end{defn}
%
Given a labeled weighted graph $G = (\vertxs, \edges, \weights, \qflag)$, 
we use $\walks(G)$ to denote a set containing all finite walks $k$ in $G$;
and $k_{v_1 \to v_2} \in \walks(G)$where $v_1, v_2 \in \vertxs$ denotes the walk from vertex $v_1$ to $v_2$ .
%
%
\begin{defn}[Length of Finite Walk w.r.t. Query ($\qlen$)].
\label{def:qlen}
\\
Given a labeled weighted graph $G = (\vertxs, \edges, \weights, \qflag)$ and a \emph{finite walk} $k$ in $G$ with its vertex sequence $(v_1, \ldots, v_{n})$, the length of $k$ w.r.t query is defined as:
\[
  \qlen(k) = \len\big(
  v \mid v \in (v_1, \ldots, v_{n}) \land \flag(v) = 1 \big)
\]
, where $\big(v \mid v \in (v_1, \ldots, v_{n}) \land \flag(v) = 1 \big)$ is a subsequence of $k$'s vertex sequence.
\end{defn}
%
Given a program $c$ with a starting memory $m$ and database $D$, we generate its program-based graph 
$\traceG(\ssa{c, m}, D) = (\vertxs, \edges, \weights, \qflag)$.
%
Then the adaptivity bound based on program analysis for $\ssa{c}$ is the number of query vertices on a finite walk in $\progG(\ssa{c})$. This finite walk satisfies:
%
\begin{itemize}
\item the number of query vertices on this walk is maximum
\item the visiting times of each vertex $v$ on this walk is bound by its weight $\weights(v)$.
\end{itemize}
%
It is formally defined in \ref{def:trace_adapt}.
%
\begin{defn}
[Adaptivity of A Program].
\label{def:trace_adapt}
\\
Given a program $\ssa{c}$ in SSA language, 
its adaptivity is defined for all possible starting SSA memory $\ssa{m}$ and database $D$ as follows:
%
$$
A(c) = \max \big 
\{ \qlen(k) \mid D \in \dbdom , k \in \walks(\traceG(c, m, D) \big \} 
$$
\end{defn}
%
%
%
%
\todo{
The following lemma describes a property of the trace-based dependency graph.
For any program $c$ with a database $D$ and a starting memory $m$,
the directed edges in its trace-based dependency graph can only be constructed from nodes representing 
smaller annotated queries to annotated queries of greater order.
There doesn't exist backward edges with direction from greater annotated queries to smaller ones.
}
\begin{lem}
\label{lem:edgeforwarding}
[Edges are Forwarding Only].
\\
%
Given a program $c$, a database $D$, a starting memory $m$ and the corresponding trace-based dependency graph $G(c,D,m) = (\vertxs, \edges)$, 
for any directed edge $(\event', \event) \in \edges$, 
this is not the case that:
%
$$\event' \eventgeq \event$$
%
\end{lem}
%
\begin{proof}
Proof in File: {\tt ``edge\_forward.tex''}.
% \begin{proof}
This lemma is proved by showing there is a contradiction. 

\jl{
Assume there exists an edge  $(\av', \av) \in \edges$ and $\av' \avgeq \av$, where $\av' = ({\qval}',l',w')$ and $\av = ({\qval},l,w)$.
%
According to the Definition~\ref{def:trace-based_graph}, we have:
%
$$
DEP(\av', \av, c, m, D) ~ (1)
$$
%
%
Unfolding the Definition \ref{def:query_dep} in $(1)$,
we know: there exists $t_1, t_3, m_1, m_3, c_2$ s.t.,
%
\[
\config{m, c, [], []} \rightarrow^{*} 
\config{m_1, [\assign{x}{\query({\qval'})}]^{l'} ; c_2,
  t_1, w'} 
\rightarrow^{\textbf{query-v}} 
\config{m_1[v_1/x], c_2,
(t_1 ++ [\av'], w_1} \rightarrow^{*} \config{m_3, \eskip, t_3,w_3} ~ (a)  
\]
%
and
%
\[
 \bigwedge
 \begin{array}{l}   
  % 
  \left( 
  \begin{array}{l}
  \av \avin (t_3 - (t_1 ++ [\av'])) 
  % 
  \\
  \implies 
  \exists v \in \qdom, v \neq v_1, m_3', t_3', w_3'. ~  
  \config{m_1[v/x], {c_2}, t_1 ++ [\av'], w_1} 
  \\ 
  \quad \quad 
  \rightarrow^{*}
  (\config{m_3', \eskip, t_3', w_3'} 
  \land 
  \av \not \avin (t_3'-(t_1 ++ [\av'])))
\end{array} \right ) ~(b)
\\
\left( 
  \begin{array}{l}
  \av \not\avin (t_3 - (t_1 ++ [\av']))
    % 
    \\
    \implies 
  \exists v \in \qdom, v \neq v_1, m_3', t_3', w_3'. 
  \config{m_1[v/x], {c_2}, t_1 ++ [\av'], w_1}
  \\ 
  \quad \quad 
  \rightarrow^{*} 
  (\config{m_3', \eskip, t_3', w_3'} 
  \land 
  \av  \avin (t_3' - (t_1 ++ [\av'])))
\end{array} \right ) ~ (c)
\end{array}
\]
%
%
According to the Theorem \ref{thm:os_wf_trace} and $(a)$, we know both $t_1$, $t_3$ are well-formed traces.
% }
% \\
% \jl{
\\
Consider 2 cases:
 \[\av \avin (t_3 - (t_1 ++[\av'])) ~(d) ~~ \lor ~~ \av \notin_{aq} (t_3 - (t_1 ++[\av'])) ~(e)\]
%
\begin{itemize}
%
  \caseL{ (d) \[\av \avin (t_3 - (t_1 ++[\av'])) ~ (d)\]}
  By unfolding the trace subtraction operations, we have;
  \[
    \exists t_2. ~ s.t., ~ t_1 ++[\av'] ++ t_2 = t_3 \land \av \avin t_2 ~ (3)
  \]
%
%
According to the Corollary~\ref{coro:aqintrace} and $\av \avin t_2$, we have:
%
\[
  \exists t_{21}, t_{22}, \av' ~ s.t.,~ (\av \aveq \av') \land t_{21} ++ [\av'] ++ t_{22} = t_2 ~ (4)
\]
%
By rewriting $(4)$ inside $(3)$, we have:
%
\[
  \exists t_{21}, t_{22}, \av'' ~ s.t.,~ (\av \aveq \av'') \land t_1 ++[\av'] ++ t_{21} ++ [\av''] ++ t_{22} = t_3 ~ \star
\]
%
By the \emph{ordering} property in definition \ref{def:wf_trace} and $(\star)$, we know
%
\[
  \av' <_{aq} \av
\]
%
This is contradict to our assumption, where $\av' \avgeq \av$. 
%
%
\caseL{(e), \[\av \notin_{aq} (t_3 - (t_1 ++[\av'])) ~(e)\]} 
%
According to the condition $(c)$, we know: $\exists v \in \qdom, v \neq v_1, m_3', t_3', w_3'.$
    % 
\[ 
  \config{m_1[v/x], {c_2}, t_1 ++ [\av'], w_1} 
  \rightarrow^{*} (\config{m_3', \eskip, t_3', w_3'} ~ (5)
  \land \av  \in_{q} (t_3' - (t_1 ++ [\av']))) ~ (6)
\] 
%
According to the Theorem \ref{thm:os_wf_trace}, Sub-Lemma-t and $(5)$,
we know $t_3'- (t_1 ++ [\av'])$ is a well-formed trace.
\\
From $(a)$, we know that the minimal line number of $c_2$ is greater than $l'$, so we know that : 
\[
  \forall \av'' \in t_3'- (t_1 ++ [\av']), \av''>_{aq} \av'
\]
%
Unfolding the trace subtraction operations in $(6)$, we have;
\[
  \exists t_2'. ~ s.t., ~ t_1 ++[\av'] ++ t_2' = t_3' \land \av \avin t_2' ~ (7)
\]
%
%
According to the sub-lemma and $(7)$, we have:
%
\[
  \exists t_{21}', t_{22}' ~ s.t.,~ t_{21}' ++ [\av] ++ t_{22}' = t_2' ~ (8)
\]
%
By rewriting $(8)$ inside $(7)$, we have:
%
\[
  t_1 ++[\av'] ++ t_{21}' ++ [\av] ++ t_{22}' = t_3' ~ \diamond
\]
%
By the \emph{ordering} property in definition \ref{def:wf_trace} and $(\diamond)$, we know
%
\[
  \av' <_{aq} \av
\]
%
This is contradict to the assumption,  $\av' \avgeq \av$.
%
\end{itemize}
%
From both cases, we derive $\av' \avgeq \av$, which is contradict to the hypothesis, i.e., $\av' \avgeq \av$.
Then, we can conclude that 
for any directed edge $(\av', \av) \in \edges$, 
this is not the case that:
%
$$\av' \avgeq \av$$
%
}
\end{proof}
%
%
%
\end{proof}
%
%
%
\begin{lem}
\label{lem:DAG}
[Trace-based Dependency Graph is Directed Acyclic].
\\
%
{
Every trace-based dependency graph is a directed acyclic graph.
}
\end{lem}
%
{
\begin{proof}
Proof is obvious based on the Lemma \ref{lem:edgeforwarding}.
\end{proof}
}
%
\begin{lem}
[Adaptivity is Bounded].
\\
{
Given the program $c$ with a certain database $D$ and starting memory $m$, the $A(c)$ w.r.t. the $D$ and $m$ is bounded, i.e.,:
%
\[
\config{m, c, [], []} 
\rightarrow^{*} 
\config{m', \eskip, t', w'} 
\implies
A_{D, m}(c) \leq |t'|
\]
}
\end{lem}
%
\begin{proof}
{
Proof is obvious based on the Lemma \ref{lem:DAG}.
}
\end{proof}
%
%
\clearpage
%
%
\subsection{SSA Transformation and Soundness of Transformation}
in File {\tt ``ssa\_transform\_sound.tex''}
% %
\subsection{SSA Transformation}
We use a translation environment $\delta$, to map variables $x$ in the {\tt While} language to those variables $\ssa{x}$ in the SSA language.
We use a name environment denoted as $\Sigma$ as a set of ssa variables, to get a fresh variable by $fresh(\Sigma)$. 
We define $\delta_1 \bowtie \delta_2 $ in a similar way as
\cite{vekris2016refinement}.
%
\[ 
\delta_1 \bowtie \delta_2 = \{ ( x, {\ssa{x_1}, \ssa{x_2}} ) \in 
\mathcal{VAR} \times \mathcal{SVAR} \times \mathcal{SVAR} \mid x \mapsto {\ssa{x_1}} \in \delta_1 , x \mapsto {\ssa{x_2} } \in \delta_2, {\ssa{x_1} \not= {\ssa{x_2} }  }  \} 
\]
%
\[ 
\delta_1 \bowtie \delta_2 / \bar{x} = \{ ( x, {\ssa{x_1}, \ssa{x_2}} ) \in 
\mathcal{VAR} \times \mathcal{SVAR} \times \mathcal{SVAR}
 \mid x \not\in \bar{x} \land x \mapsto {\ssa{x_1}} \in \delta_1 , x \mapsto {\ssa{x_2} } \in \delta_2, {\ssa{x_1} \not= {\ssa{x_2} }   }  \} 
 \]
 %
We call a list of variables $\bar{x}$.
\[
 [\bar{x}, \bar{\ssa{x_1}}, \bar{\ssa{x_2}}] = \{ (x, x_1,x_2)  | \forall 0 \leq i < |\bar{x}|, x = \bar{x }[i] \land x_1 = \bar{x_1}[i] \land x_2 = \bar{x_2 }[i] \land |\bar{x}| = |\bar{x_1}| = |\bar{x_2}|   \}
\]
%
\begin{mathpar}
\boxed{ \delta ; e \hookrightarrow \ssa{e} }
\and
\inferrule{
}{
 \delta ; x \hookrightarrow \delta(x)
}~{\textbf{S-VAR}}
\and
\boxed{ \Sigma; \delta ; c  \hookrightarrow \ssa{c} ; \delta' ; \Sigma' }
\and
\inferrule{
  { \delta ; \bexpr \hookrightarrow \ssa{\bexpr} }
  \and
  { \Sigma; \delta ; c_1 \hookrightarrow \ssa{c_1} ; \delta_1;\Sigma_1 }
  \and
  {\Sigma_1; \delta ; c_2 \hookrightarrow \ssa{c_2} ; \delta_2 ; \Sigma_2 }
  \\
  {[\bar{x}, \ssa{\bar{{x_1}}, \bar{{x_2}}}] = \delta_1 \bowtie \delta_2  }
  \and
   {[\bar{y}, \ssa{\bar{{y_1}}, \bar{{y_2}}}] = \delta \bowtie \delta_1 / \bar{x} }
  \and
   {[\bar{z}, \ssa{\bar{{z_1}}, \bar{{z_2}}}] = \delta \bowtie \delta_2 / \bar{x} }
  \\
  { \delta' =\delta[\bar{x} \mapsto \ssa{\bar{{x}}'} ][\bar{y} \mapsto \ssa{\bar{{y}}'} ][\bar{z} \mapsto \ssa{\bar{{z}}'} ]}
  \and 
  {\ssa{\bar{{x}}', \bar{y}', \bar{z}'} \ fresh(\Sigma_2)
  }
  \and{\Sigma' = \Sigma_2 \cup \{ \ssa{ \bar{x}', \bar{y}', \bar{z}' } \} }
}{
 \Sigma; \delta ; [\eif(\bexpr, c_1, c_2)]^l  \hookrightarrow [\ssa{ \eif(\bexpr, [\bar{{x}}', \bar{{x_1}}, \bar{{x_2}}] ,[\bar{{y}}', \bar{{y_1}}, \bar{{y_2}}] ,[\bar{{z}}', \bar{{z_1}}, \bar{{z_2}}] , {c_1}, {c_2})}]^l; \delta';\Sigma'
}~{\textbf{S-IF}}
%
\and
%
\inferrule{
 {\delta ; \expr \hookrightarrow \ssa{\expr} }
 \and
 {\delta' = \delta[x \mapsto \ssa{{x}} ]}
 \and{ \ssa{x} \ fresh(\Sigma) }
 \and { \Sigma' = \Sigma \cup \{ \ssa{x} \} }
}{
 \Sigma;\delta ; [\assign x \expr]^{l} \hookrightarrow [\ssa{\assign {{x}}{ \expr}}]^{l} ; \delta'; \Sigma'
}~{\textbf{S-ASSN}}
%
\and
%
\inferrule{
 {\delta ; \query \hookrightarrow \ssa{\query}}
 \and
 {\delta ; \qexpr \hookrightarrow \ssa{\qexpr}}
 \and
 {\delta' = \delta[x \mapsto \ssa{x} ]}
 \and{ \ssa{x} \ fresh(\Sigma) }
  \and { \Sigma' = \Sigma \cup \{ \ssa{x} \} }
}{
 \Sigma;\delta ; [\assign{x}{\query(\qexpr)}]^{l} \hookrightarrow 
 [\assign {\ssa{x}}{ \ssa{\query(\qexpr)}}]^{l} ; \delta';\Sigma'
}~{\textbf{S-QUERY}}
%
%%
\and
%
%
\and
%
\inferrule{
    { \Sigma; \delta ; c \hookrightarrow \ssa{c_1} ; \delta_1; \Sigma_1 }
     \and
    { [ \bar{x}, \ssa{\bar{{x_1}}}, \ssa{\bar{{x_2}}} ] = \delta \bowtie \delta_1 }
    \\
     {
     \ssa{\bar{{x}}'} \ fresh(\Sigma_1 )}
    \and {\delta' = \delta[\bar{x} \mapsto \ssa{\bar{{x}}'}]}
    \and 
     {\delta' ; \bexpr \hookrightarrow \ssa{\bexpr} }
     \and
    {\ssa{c' = c_1[\bar{x}'/ \bar{x_1}]   } }
  }{ 
  \Sigma; \delta ;  \ewhile ~ [\bexpr]^{l} ~ \edo ~ c 
  \hookrightarrow 
  \ssa{\ewhile ~ [\bexpr]^{l}, 0, [\bar{{x}}', \bar{{x_1}}, \bar{{x_2}}] ~ \edo ~ {c} } ; \delta'; \Sigma_1 \cup \{\ssa{\bar{x}'}  \}
}~{\textbf{S-WHILE}
}
\and
%
\inferrule{
 {\Sigma;\delta ; c_1 \hookrightarrow \ssa{c_1} ; \delta_1; \Sigma_1} 
 \and
 {\Sigma_1; \delta_1 ; c_2 \hookrightarrow \ssa{c_2} ; \delta'; \Sigma'} 
}{
\Sigma;\delta ; c_1 ; c_2 \hookrightarrow \ssa{c_1} ; \ssa{c_2} \ ; \delta';\Sigma'
}~{\textbf{S-SEQ}}
\end{mathpar}

\paragraph{Concrete examples.}
\[
c_1 \triangleq
\begin{array}{l}
     \left[x \leftarrow \query(1) \right]^1; \\
     \eif \; (x ==0)^{2} \; \\
    \ethen \; \left[y \leftarrow \query(2) \right]^3\; \\
    \eelse \; \left[y \leftarrow 0 \right]^4 ; \\
    \eif \; (x == 1 )^5\; \\
    \ethen \; \left[ y \leftarrow 0 \right]^6\; \\
    \eelse \; \left[y \leftarrow \query(2) \right]^7\\
\end{array}
%
%
\hspace{20pt} \hookrightarrow  \hspace{20pt}
%
\begin{array}{l}
     \left[ \ssa{x_1} \leftarrow \query(1) \right]^1; \\
     \eif \; (\ssa{x_1 ==0})^{2}, [\ssa{ y_3, y_1,y_2  }],[],[]  \; \\
    \ethen \; \left[ \ssa{y_1} \leftarrow \query(2) \right]^3\; \\
    \eelse \; \left[\ssa{y_2 \leftarrow 0 } \right]^4 ; \\
    \eif \; (\ssa{x_1 == 1} )^{5} , [\ssa{ y_6, y_4, y_5 } ] \; \\
    \ethen \; \left[ \ssa{y_4 \leftarrow 0} \right]^6\; \\
    \eelse \; \left[\ssa{y_5} \leftarrow \query(2) \right]^7\\
\end{array}
\]
\[
c_2 \triangleq
\begin{array}{l}
   \left[ x \leftarrow \query(1) \right]^1; \\
   \left[y \leftarrow \query(2) \right]^2 ; \\
    \eif \;( x + y == 5 )^3\; \\
    \ethen \;\left[ z \leftarrow \query(3)\right]^4 \; \\
    \eelse \;\left[ \ssa{\eskip}\right]^5 ; \\
   \left[ w \leftarrow q_4 \right]^6; \\
\end{array}
\hspace{20pt} \hookrightarrow \hspace{20pt}
%
\begin{array}{l}
   \left[ \ssa{ x_1 } \leftarrow \query(1) \right]^1; \\
   \left[\ssa{ y_1} \leftarrow \query(2) \right]^2 ; \\
    \eif \;( \ssa{ x_1 + y_1 == 5} )^3, [ ],[] ,[ ]\; \\
    \ethen \;\left[ \ssa{ z_1 }
    \leftarrow \query(3)\right]^4 \; \\
    \eelse \;\left[ \eskip\right]^5 ; \\
   \left[ \ssa{ w_1} \leftarrow \query(4) \right]^6; \\
\end{array}
\]

{
\[
c_3 \triangleq
\begin{array}{l}
     \left[x \leftarrow \query(1) \right]^1 ; \\
     \left[i \leftarrow 0 \right]^2 ; \\
    \ewhile ~  [i < 100]^3 ~ \edo
    \\
    ~ \Big( 
    \left[z \leftarrow \query(3) \right]^4; \\
    \left[x \leftarrow z + x \right]^5; \\
    \left[i \leftarrow i + 1 \right]^6
    \Big) ;
\end{array}
%
\hspace{20pt} \hookrightarrow \hspace{20pt} 
%
\begin{array}{l}
     \left[\ssa{x_1} \leftarrow \query(1) \right]^1 ; \\
     \left[\ssa{i_1} \leftarrow 0 \right]^2 ; \\
    \ewhile
    ~ [\ssa{i_1} < 100]^3, 0,
    ~\ssa{[ x_3,x_1 ,x_2 ], [i_3, i_1, i_2] }~
    \edo \\
    ~ \Big( 
    \left[\ssa{z_1} \leftarrow \query(3) \right]^4; \\
    \left[ \ssa{x_2} \leftarrow \ssa{z_1 + x_3} \right]^5; \\
    \left[\ssa{i_2} \leftarrow \ssa{i_3} + 1 \right]^6
    \Big) ;
\end{array}
\]
}
%
\begin{figure}
   \[
 \begin{array}{lll}
    | \ewhile ~ [ \sbexpr ]^{l}, n, [\bar{\ssa{x}}, \bar{\ssa{x_1}}, \bar{\ssa{x_2}}] 
    ~ \edo ~  \ssa{c}|  
    &=& \ewhile ~ [|\sbexpr|]^{l},  ~ \edo ~ |\ssa{c}| 
	\\
    |\ssa{c_1 ; c_2}|  &=& |\ssa{c_1}| ; |\ssa{c_2}| 
    \\
    |[\eif(\sbexpr,
    [ \bar{\ssa{x}}, \bar{\ssa{x_1}}, \bar{\ssa{x_2}}] ,
    [ \bar{\ssa{y}}, \bar{\ssa{y_1}}, \bar{\ssa{y_2}}] , 
    [\bar{\ssa{z}}, \bar{\ssa{z_1}}, \bar{\ssa{z_2}}] , 
    \ssa{ c_1, c_2)}]^{l}|  
    &=&
    [\eif(|\sbexpr|, |\ssa{ c_1}|, |\ssa{c_2}|)]^{l}
    \\
    | [\assign {\ssa{x}}{\ssa{\expr}}]^{l}| & = & [\assign {|\ssa{x}|}{|\ssa{\expr}|} ]^{l}
    \\
    | [\assign {\ssa{x}}{\query(\ssa{\qexpr})} ]^{l} | & = & [\assign {|\ssa{x}|}{|\query(\ssa{\qexpr})|}]^{l}
    \\
    |\ssa{x}_i| & = & x 
    \\
    |n | & = & n 
    \\
    | \saexpr_1 \oplus_{a} \saexpr_2 | & = &  |\ssa{\aexpr_1}| \oplus_a |\ssa{\aexpr_2}| \\
    | \sbexpr_1 \oplus_{b} \sbexpr_2 | & = &  |\sbexpr_1| \oplus_b |\sbexpr_2|
 \end{array}
\]
    \caption{The Erasure of SSA}
    \label{fig:ssa_erasure-while}
\end{figure}
%
%
%
% 
%
\subsection{The Soundness of the Transformation}
In this subsection, we show our transformation from the {\tt While} language to its SSA form is sound with respect to the adaptivity. 
To be specific, a transformed program $\ssa{c}$ starting with appropriate configuration, generates the same trace as the program before the transformation $c$, in its corresponding configuration.
%
%
\begin{defn}[\todo{Written Variables}].
\\
We defined the assigned variables in the while language program $c$ as $\avars{c}$,the assigned variables in the ssa-form program $\ssa{c}$ as $\avarssa{\ssa{c}}$ defined as follows.
\[
\begin{array}{lll}
    \avars{\assign{x}{\expr}} & =& \{ x \} \\
    \avars{\assign{x}{\query(\qexpr)}} & =& \{ x \} \\
    \avars{c_1; c_2}  & = & \avars{c_1} \cup \avars{c_2} \\
    \avars{\ewhile ~ \bexpr ~ \edo ~ c} &= &  \avars{c} \\
    \avars{\eif(\bexpr, c_1, c_2)} & =&  \avars{c_1} \cup \avars{c_2}\\
\end{array} 
\]
%
\[
\begin{array}{lll}
    \avarssa{\ssa{\assign{x}{\expr}}} & =& \{ \ssa{x} \}
    \\
    \avarssa{\ssa{\assign{x}{\query(\ssa{\qexpr})}}} & =& \{ \ssa{x} \}
    \\
    \avarssa{\ssa{c_1; c_2 } }  & = & \avarssa{\ssa{c}_1} \cup \avarssa{\ssa{c}_2}
    \\
    \avarssa{\ewhile ~ \ssa{\bexpr, n, [\bar{x}, \bar{x_1}, \bar{x_2}] ~ \edo ~ \ssa{c}}}
    & = &  
    \{\ssa{\bar{x}}\} \cup \avarssa{\ssa{c}} 
    \\
    \avarssa{\eif(\ssa{\bexpr,[\bar{x}, \bar{x_1}, \bar{x_2}],[\bar{y}, \bar{y_1}, \bar{y_2}],[\bar{z}, \bar{z_1}, \bar{z_2}], c_1, c_2} )} 
    & =&  \{ \ssa{\bar{x}},\ssa{\bar{y}} , \ssa{\bar{z}} \} 
    \cup \avarssa{\ssa{c_1}} \cup \avarssa{\ssa{c_2}}\\
\end{array}
\]
\end{defn}
\begin{defn}[\todo{Read Variables}].
\\
{
The variables read in the while language programs $c$ as $\vars{c}$, variables used in ssa-form program $\ssa{c}$ : 
}
\[
\begin{array}{lll}
    \vars{\assign{x}{\expr}} & =& \vars{\expr}  \\
    \vars{\assign{x}{\query(\qexpr)}} & =&\{  \} \\
    \vars{ c_1; c_2  }  & = & \vars{c_1} \cup \vars{c_2} \\
    \vars{  \eloop ~ \aexpr ~ \edo ~ c  } &= &\vars{\aexpr} \cup \vars{c} \\
    \vars{\eif(\bexpr, c_1, c_2)} & =& \vars{\bexpr} \cup \vars{c_1} \cup \vars{c_2}\\
\end{array}
\]
\[
\begin{array}{lll}
    \varssa{\ssa{\assign{x}{\expr}}} & =& \varssa{\ssa{\expr}}  \\
    \varssa{\ssa{\assign{x}{\query(\qexpr)}}} & =& \{  \} \\
    \varssa{ \ssa{c_1; c_2}  }  & = & \varssa{\ssa{c}_1} \cup \varssa{\ssa{c}_2} \\
    % \varssa{  \eloop ~ \ssa{\aexpr, n, [\bar{x}, \bar{x_1}, \bar{x_2}] ~ \edo ~ c} } &= &\varssa{\ssa{\aexpr}} \cup \varssa{\ssa{c}}  \cup \{ \ssa{\bar{x_1}} \} \cup \{ \ssa{\bar{x_2}} \}\\
    {\varssa{  \ewhile ~ \ssa{\bexpr, n, [\bar{x}, \bar{x_1}, \bar{x_2}] ~ \edo ~ c} }} 
    &= &
    \varssa{\ssa{\bexpr}} \cup \varssa{\ssa{c}}  \cup \{ \ssa{\bar{x_1}} \} \cup \{ \ssa{\bar{x_2}} \}\\
    \varssa{\eif(\ssa{\bexpr,[\bar{x}, \bar{x_1}, \bar{x_2}], [\bar{y}, \bar{y_1}, \bar{y_2}],[\bar{z}, \bar{z_1}, \bar{z_2}], c_1, c_2} )} & =& \varssa{\ssa{\bexpr}} \cup \varssa{\ssa{c_1}} \cup \varssa{\ssa{c_2}} \cup \{\ssa{\bar{x_1}, \bar{x_2},\bar{y_1}, \bar{y_2},\bar{z_1}, \bar{z_2} }\}  \\
\end{array}
\]
\end{defn}
%
\begin{defn}[\todo{Necessary Variables}].
\\
{
We call the variables needed to be assigned before executing the program $c$ as necessary variables $\fv{c}$. Its ssa form is : $\fvssa{\ssa{c}}$.
}  
 \[
 \begin{array}{lll}
     \fvars{\assign{x}{\expr} }  & = & \vars{\expr}  \\
     \fvars{\assign{x}{\query(\qexpr)} }  & = & \{ \}  \\
     {\fvars{  \ewhile ~ \bexpr ~ \edo ~ c  } }&= & \vars{\bexpr} \cup \fvars{c} \\
     \fvars{\eif(\bexpr, c_1, c_2)} & =& \vars{\bexpr} \cup \fvars{c_1} \cup \fvars{c_2}  \\
      \fvars{c_1 ; c_2} & = & \fvars{c_1} \cup ( \fvars{c_2} - \avars{c_1})
 \end{array}
 \]
 \[
 \begin{array}{lll}
     \fvssa{\ssa{\assign{x}{\expr}} }  & = & \varssa{\ssa{\expr}}  \\
     \fvssa{ \ssa{ \assign{x}{\query(\qexpr)}} }  & = & \{ \}  \\
     {\fvssa{  \ewhile ~ \ssa{\bexpr, n, [\bar{x}, \bar{x_1}, \bar{x_2}] ~ \edo ~ c} } }
     &= & 
     \varssa{\ssa{\bexpr}} \cup \fvssa{\ssa{c}}[\ssa{ \bar{x_1}} / \ssa{\bar{x}}]\\
     \fvssa{\eif(\ssa{\bexpr,[\bar{x}, \bar{x_1}, \bar{x_2}],[\bar{y}, \bar{y_1}, \bar{y_2}],[\bar{z}, \bar{z_1}, \bar{z_2}], c_1, c_2} )} & =& \varssa{\ssa{\bexpr}} \cup \fvssa{\ssa{c_1}} \cup \fvssa{\ssa{c_2}}  \\
      \fvssa{\ssa{c_1 ; c_2}} & = & \fvssa{\ssa{c_1}} \cup ( \fvssa{\ssa{c_2}} - \avarssa{\ssa{c_1}})
 \end{array}
 \]
%
\end{defn}
%
The Lemma~\ref{lem:fv} and \ref{lem:same_value} proved the preserving properties for variables and values during the transformation.
%
\begin{lem}[Variable Preserving]
\label{lem:fv}
If $\Sigma;\delta ; c \hookrightarrow \ssa{c} ; \delta';\Sigma' $, $\fvssa{\ssa{c}} = \delta(\fvars{c})$. 
\end{lem}
\begin{proof}
 By induction on the transformation.
 \begin{itemize}
    \caseL{Case $\inferrule{
  { \delta ; \bexpr \hookrightarrow \ssa{\bexpr} }
  \and
  { \delta ; c_1 \hookrightarrow \ssa{c_1} ; \delta_1 }
  \and
  {\delta ; c_2 \hookrightarrow \ssa{c_2} ; \delta_2 }
  \\
  {[\bar{x}, \ssa{\bar{{x_1}}, \bar{{x_2}}}] = \delta_1 \bowtie \delta_2  }
  \and
   {[\bar{y}, \ssa{\bar{{y_1}}, \bar{{y_2}}}] = \delta \bowtie \delta_1 / \bar{x} }
  \and
   {[\bar{z}, \ssa{\bar{{z_1}}, \bar{{z_2}}}] = \delta \bowtie \delta_2 / \bar{x} }
  \\
  { \delta' =\delta[\bar{x} \mapsto \ssa{\bar{{x}}'} ]}
  \and 
  {\ssa{\bar{{x}}', \bar{y}', \bar{z}'} \ fresh }
}{
 \delta ; [\eif(\bexpr, c_1, c_2)]^l  \hookrightarrow [\ssa{ \eif(\bexpr, [\bar{{x}}', \bar{{x_1}}, \bar{{x_2}}] ,[\bar{{y}}', \bar{{y_1}}, \bar{{y_2}}] ,[\bar{{z}}', \bar{{z_1}}, \bar{{z_2}}] , {c_1}, {c_2})}]^l; \delta'
}~{\textbf{S-IF}} $}
From the definition of $\fvssa{[\eif(\sbexpr, [\bar{\ssa{x'}}, \bar{\ssa{x_1}}, \bar{\ssa{x_2}}] , \ssa{c_1}, \ssa{c_2})]^l} = \varssa{\ssa{\bexpr}} \cup \fvssa{\ssa{c_1}} \cup \fvssa{\ssa{c_2}}$. We want to show: \[\varssa{\ssa{\bexpr}}) \cup \fvssa{\ssa{c_1}} \cup \fvssa{\ssa{c_2}} = \delta( \vars{\bexpr}) \cup \delta(\fv{c_1}) \cup \delta(\fv{c_2}  )\]
By induction hypothosis on the second and third premise, we know that : $\fvssa{\ssa{c_1}} = \delta(\fv{c_1}) $ and $\fvssa{\ssa{c_2}} = \delta(\fv{c_2}) $.  We still need to show that: 
\[
  \varssa{\ssa{\bexpr}} = \delta(\vars{\bexpr})
\] 
From the first premise, we know that $\vars{b} \subseteq \dom(\delta)$. This is goal is proved by the rule $\textbf{S-VAR}$ on all the variables in $\bexpr$.\\
{\caseL{Case
$\inferrule{
    { \Sigma; \delta ; c \hookrightarrow \ssa{c_1} ; \delta_1; \Sigma_1 }
     \and
    { [ \bar{x}, \ssa{\bar{{x_1}}}, \ssa{\bar{{x_2}}} ] = \delta \bowtie \delta_1 }
    \\
     {\ssa{\bar{{x}}'} \ fresh(\Sigma_1 )}
    \and {\delta' = \delta[\bar{x} \mapsto \ssa{\bar{{x}}'}]}
    \and 
     {\delta' ; \bexpr \hookrightarrow \ssa{\bexpr} }
     \and
    {\ssa{c' = c_1[\bar{x}'/ \bar{x_1}]   } }
    % \and{ \delta' ; c \hookrightarrow \ssa{c'} ; \delta'' }
  }{ 
  \Sigma; \delta ;  \ewhile ~ [\bexpr]^{l} ~ \edo ~ c 
  \hookrightarrow 
  \ssa{\ewhile ~ [\bexpr]^{l}, 0, [\bar{{x}}', \bar{{x_1}}, \bar{{x_2}}] ~ \edo ~ {c} } ; \delta'; \Sigma_1 \cup \{\ssa{\bar{x}'}  \}
}~{\textbf{S-WHILE}}
$}
}
{
Unfolding the definition, we need to show:
\[\varssa{\ssa{\bexpr}} \cup \fvssa{\ssa{c'}}[\ssa{ \bar{x_1}} / \ssa{\bar{x}}] = \delta (\vars{\bexpr}) \cup \delta(\fv{c} ) \]
We can similarly show that $\varssa{\ssa{\bexpr}} = \delta(\vars{b})$ as in the if case. We still need to show that: 
\[
 \fvssa{\ssa{c_1[\bar{x}' / \bar{x_1}]}}[ \ssa{ \bar{x_1} } / \ssa{\bar{x}'}] =  \delta(\fv{c} )
\]
It is proved by induction hypothesis on $  { \Sigma; \delta ; c \hookrightarrow \ssa{c_1} ; \delta_1; \Sigma_1 }$.\\
}
%
\caseL{Case $\inferrule{
 {\Sigma;\delta ; c_1 \hookrightarrow \ssa{c_1} ; \delta_1; \Sigma_1} 
 \and
 {\Sigma_1; \delta_1 ; c_2 \hookrightarrow \ssa{c_2} ; \delta'; \Sigma'} 
}{
\Sigma;\delta ; c_1 ; c_2 \hookrightarrow \ssa{c_1} ; \ssa{c_2} \ ; \delta';\Sigma'
}~{\textbf{S-SEQ}}$}
To show:
  \[ \fvssa{\ssa{c_1}} \cup ( \fvssa{\ssa{c_2}} - \avarssa{\ssa{c_1}}) = \delta(\fv{c_1} )\cup \delta( \fv{c_2} - \avars{c_1}) \]
  By induction hypothesis on the first premise, we know that : $ \fvssa{\ssa{c_1}} = \delta(\fv{c_1} ) $, still to show: 
    \[ ( \fvssa{\ssa{c_2}} - \avarssa{\ssa{c_1}}) = \delta( \fv{c_2} - \avars{c_1})
    \]
    We know that $\delta_1 = \delta [\avars{c_1} \mapsto \avarssa{\ssa{c_1}} ]$, so by induction hypothesis, we know: $ \fvssa{\ssa{c_2}} = \delta[\avars{c_1} \mapsto \avarssa{\ssa{c_1}} ]( \fv{c_2})  = \delta(\fv{c_2}) \cup \avarssa{\ssa{c_1}} - \delta(\avars{c_1}) $.
    
    This case is proved.
 \end{itemize}
 
\end{proof}

{
We first define a good memory in the {\tt While} language $m$ or in the ssa language $\ssa{m}$ with respect to a translation environment $\delta$, denoted as $m \vDash \delta$ and $\ssa{m} \vDash \delta$ respectively. 
%
\begin{defn}[Well Defined Memory].
\begin{enumerate}
    % \item $m \vDash c \triangleq \forall x \in \fv{c}, \exists v, (x, v) \in m$.
    \item $ m \vDash \delta  \triangleq \forall x \in \dom(\delta), \exists v, (x,v) \in m$.
    % \item $\ssa{m} \vDash_{ssa} \ssa{c} \triangleq \forall \ssa{x} \in \fvssa{\ssa{c}}, \exists v, (\ssa{x}, v) \in \ssa{m}$.
    \item $ \ssa{m} \vDash_{ssa} \delta  \triangleq \forall \ssa{x} \in \codom(\delta), \exists v, (\ssa{x},v) \in \ssa{m}$.
\end{enumerate}
\end{defn}
%
The part declared in the translation environment $\delta$ in a ssa memory $\ssa{m}$ can be reverted to corresponding part of the memory $m$ with an inverse of $\delta$ as follows.
%
\begin{defn}[Inverse of Transformed memory]
 $m = \delta^{-1}(\ssa{m}) \triangleq \forall x \in \dom(\delta), (\delta(x), m(x)) \in \ssa{m} $.
\end{defn}
}
%
\begin{lem}[Value Preserving].
\label{lem:same_value}
{
Given $\delta; e \hookrightarrow \ssa{e}$,  $\forall m. m \vDash \delta. \forall \ssa{m}, \ssa{m} \vDash_{ssa} \delta \land m = \delta^{-1}(\ssa{m})$, then $\config{m, e} \to v $ and $\config{
\ssa{m}, \ssa{e}} \to {v}$.
}
\end{lem}

\begin{thm}[Soundness of transformation]
Given $\Sigma; \delta ; c \hookrightarrow \ssa{c} ; \delta';\Sigma' $, $\forall m. m \vDash \delta. \forall \ssa{m}, \ssa{m} \vDash_{ssa} \delta \land m = \delta^{-1}(\ssa{m})$, if there exist an execution of $c$ in the while language, starting with a trace $t$ and loop maps $w$, $\config{m, c, t, w} \to^{*} \config{m', \eskip, t', w' } $,  then there also exists a corresponding execution of $\ssa{c}$ in the ssa language so that 
  $\config{  {\ssa{m}}, \ssa{c}, t, w} \to^{*} \config{{  \ssa{m'}}, \eskip, t', w' } $ and $ m' = \delta'^{-1}(\ssa{m'}) $.
\end{thm}

\begin{proof}
 We assume that $q$ is the same when transformed to $\ssa{q}$, as the primitive in both languages. And the value remains the same during the transformation.  
 It is proved by induction on the transformation rules.
 \begin{itemize}
   \caseL{Case $\inferrule{
 {\Sigma;\delta ; c_1 \hookrightarrow \ssa{c_1} ; \delta_1;\Sigma_1} 
 \and
 {\Sigma_1; \delta_1 ; c_2 \hookrightarrow \ssa{c_2} ; \delta'; \Sigma'} 
}{
\Sigma;\delta ; c_1 ; c_2 \hookrightarrow \ssa{c_1} ; \ssa{c_2} \ ; \delta';\Sigma'
}~{\textbf{S-SEQ}}$}
We choose an arbitrary memory $m$ so that $m \vDash \delta$, we choose a trace $t$ and a loop maps $w$.
\[
\inferrule
{
{\config{m, c_1,  t,w} \xrightarrow{}^{*} \config{m_1, \eskip,  t_1,w_1}}
\and
{\config{m_1, c_2,  t_1,w_1} \xrightarrow{}^{*} \config{m', \eskip,  t',w'}}
}
{
\config{m, c_1; c_2,  t,w} \xrightarrow{}^{*} \config{m', \eskip, t',w'}
}
\]
 We choose the transformed memory ${\ssa{m}} $ so that  $ m =\delta^{-1}(\ssa{m})$.
 
 To show: $ \config{\ssa{ m, c_1;c_2 }, t, w } \xrightarrow{}^{*} \config{\ssa{m', \eskip}, t'. w' }$ and $ m' = \delta'^{-1} (\ssa{m'}) $.
 
 By induction hypothesis on the first premise, we have:
 \[ \config{\ssa{m, c_1}, t,w} \xrightarrow{}^{*} \config{\ssa{m_1, \eskip},t_1,w_1 } \land m_1 = \delta_1^{-1}(\ssa{m_1}) \]
  By induction hypothesis on the second premise, using the conclusion $ m_1 = \delta_1^{-1}(\ssa{m_1}) $.
  We have:
  \[
   \config{\ssa{m_1, c_2}, t_1,w_1} \xrightarrow{}^{*} \config{\ssa{m', \eskip},t',w' } \land m' = \delta'^{-1}(\ssa{m'})
  \]
  So we know that 
  \[
  \inferrule{
  { \config{\ssa{m, c_1}, t,w} \xrightarrow{}^{*} \config{\ssa{m_1, \eskip},t_1,w_1 }  }
  \and
  { \config{\ssa{m_1, c_2}, t_1,w_1} \xrightarrow{}^{*} \config{\ssa{m', \eskip},t',w' } }
  }{
  \config{\ssa{m, c_1;c_2 }, t,w} \xrightarrow{}^{*} \config{\ssa{m', \eskip}, t' , w' }
  }
  \]
 \caseL{Case $\inferrule{
 { \delta ; \expr \hookrightarrow \sexpr}
 \and
 {\delta' = \delta[x \mapsto \ssa{x} ]}
 \and{ \ssa{x} \ fresh(\Sigma) }
 \and {\Sigma' = \Sigma \cup \{x\} }
}{
 \Sigma;\delta ; [\assign x \expr]^{l} \hookrightarrow [\assign {\ssa{x}}{ \ssa{\expr}}]^{l} ; \delta';\Sigma'
}~{\textbf{S-ASSN}} $ }

 We choose an arbitrary memory $m$ so that $m \vDash \delta$, we choose a trace $t$ and a loop maps $w$, we know that the resulting trace is still $t$ from its evaluation rule $\textbf{assn}$ when we suppose $m, \expr \to v$.
 \[
 \inferrule
{
}
{
\config{m, [\assign x v]^{l},  t,w} \xrightarrow{} \config{m[v/x], [\eskip]^{l}, t,w}
}
~\textbf{assn}
 \]
 We choose the transformed memory ${\ssa{m}} $ so that  $ m =\delta^{-1}(\ssa{m})$.
 
 To show: $\config{\ssa{m}, [\assign {\ssa{x}}{ \ssa{\expr}}]^{l} , t, w} \to^{*} \config{\ssa{m'}, \eskip, t, w} $ and $ m' = \delta'^{-1}(\ssa{m'}) $.
 
 From the rule \textbf{ssa-assn}, we assume $\ssa{m}, \ssa{\expr} \to \ssa{v}$, we know that 
 \[
 \inferrule
{
}
{
\config{\ssa{ m, [\assign x v]^{l}},  t,w } \xrightarrow{} \config{\ssa{m[x \mapsto v], [\eskip]^{l}}, t,w }
}
~\textbf{ssa-assn}
 \]
 We also know that $\delta'= \delta[x \mapsto \ssa{x}]$ and $m = \delta^{-1}(\ssa{m})$, $m'= m[v/x]$. We can show that $ m[v/x] = \delta[x \mapsto \ssa{x}]^{-1}(\ssa{m}[\ssa{x} \mapsto v]) $.
 
\caseL{Case $\inferrule{
 {\delta ; q \hookrightarrow \ssa{q}}
 \and
 {\delta ; \expr \hookrightarrow \ssa{\expr}}
 \and
 {\delta' = \delta[x \mapsto \ssa{x} ]}
 \and{ \ssa{x} \ fresh(\Sigma) }
 \and{ \Sigma' = \Sigma \cup \{x\} }
}{
 \Sigma;\delta ; [\assign{x}{\query(\qexpr)}]^{l} \hookrightarrow [\assign {\ssa{x}}{ \ssa{\query(\qexpr)}}]^{l} ; \delta'
}~{\textbf{S-QUERY}}$} 
We choose an arbitrary memory $m$ so that $m \vDash \delta$, we choose a trace $t$ and a loop maps $w$, we know that when we suppose $\config{m, \expr} \to v$.
 \[
\inferrule
{
\query(v)(D) = \qval 
}
{
\config{m, [\assign{x}{\query(v)}]^l, t, w} \xrightarrow{} \config{m[ \qval/ x], \eskip,  t \mathrel{++} [\query(v),l,w )],w }
}
~\textbf{query}
 \]
 We choose the transformed memory ${\ssa{m}} $ so that  $ m =\delta^{-1}(\ssa{m})$.
 
 To show: $\config{\ssa{m}, [\assign {\ssa{x}}{ \ssa{\query(\qexpr)}}]^{l} , t, w} \to^{*} \config{\ssa{m'}, \eskip, t, w} $ and $ m' = \delta'^{-1}(\ssa{m'}) $.
 
 From the rule \textbf{ssa-query}, we know that 
 \[
 \inferrule
{
\ssa{\query(v)(D) = \qval} 
}
{
\config{ \ssa{ m, [\assign{\ssa{x}}{\ssa{\query(\qexpr)}}]^l}, t, w} \xrightarrow{} \config{\ssa{  m[  x \mapsto v], \eskip,}  t \mathrel{++} [(q^{(l,w )},v)],w }
}
~\textbf{ssa-query}
 \]
 We also know that $\delta'= \delta[x \mapsto \ssa{x}]$ and $m = \delta^{-1}(\ssa{m})$, $m'= m[v/x]$. We can show that $ m[v/x] = \delta[x \mapsto \ssa{x}]^{-1}(\ssa{m}[\ssa{x} \mapsto v]) $.

  \caseL{Case $\inferrule{
  { \delta ; \bexpr \hookrightarrow \ssa{\bexpr} }
  \and
  { \Sigma; \delta ; c_1 \hookrightarrow \ssa{c_1} ; \delta_1;\Sigma_1 }
  \and
  {\Sigma_1; \delta ; c_2 \hookrightarrow \ssa{c_2} ; \delta_2 ; \Sigma_2 }
  \\
  {[\bar{x}, \ssa{\bar{{x_1}}, \bar{{x_2}}}] = \delta_1 \bowtie \delta_2  }
  \and
   {[\bar{y}, \ssa{\bar{{y_1}}, \bar{{y_2}}}] = \delta \bowtie \delta_1 / \bar{x} }
  \and
   {[\bar{z}, \ssa{\bar{{z_1}}, \bar{{z_2}}}] = \delta \bowtie \delta_2 / \bar{x} }
  \\
  { \delta' =\delta[\bar{x} \mapsto \ssa{\bar{{x}}'} ][\bar{y} \mapsto \ssa{\bar{{y}}'} ][\bar{z} \mapsto \ssa{\bar{{z}}'} ]}
  \and 
  {\ssa{\bar{{x}}', \bar{y}', \bar{z}'} \ fresh(\Sigma_2)
  }
  \and{\Sigma' = \Sigma_2 \cup \{ \ssa{ \bar{x}', \bar{y}', \bar{z}' } \} }
}{
 \Sigma; \delta ; [\eif(\bexpr, c_1, c_2)]^l  \hookrightarrow [\ssa{ \eif(\bexpr, [\bar{{x}}', \bar{{x_1}}, \bar{{x_2}}] ,[\bar{{y}}', \bar{{y_1}}, \bar{{y_2}}] ,[\bar{{z}}', \bar{{z_1}}, \bar{{z_2}}] , {c_1}, {c_2})}]^l; \delta';\Sigma'
}~{\textbf{S-IF}}$}
We choose an arbitrary memory $m$ so that $m \vDash \delta$, we choose a trace $t$ and a loop maps $w$.
There are two possible evaluation rules depending on the the condition $b$, we choose the case when $b = \etrue$, we know there is an execution in ssa language so that $\ssa{\bexpr} = \etrue$, we use the rule $\textbf{if-t}$.  
 \[\inferrule
{
}
{
\config{m, [\eif(\etrue, c_1, c_2)]^{l},t,w} 
\xrightarrow{} \config{m, c_1,  t,w} \to^{*} \config{m', \eskip, t', w'}
}
\]
 We choose the transformed memory ${\ssa{m}} $ so that  $ m =\delta^{-1}(\ssa{m})$.
 
 To show: $\config{\ssa{m}, [\eif(\etrue, [\bar{\ssa{x}}', \bar{\ssa{x_1}}, \bar{\ssa{x_2}}] ,[\bar{\ssa{y}}', \bar{\ssa{y_1}}, \bar{\ssa{y_2}}] ,[\bar{\ssa{z}}', \bar{\ssa{z_1}}, \bar{\ssa{z_2}}] , c_1, c_2)]^{l}, t, w} \to^{*} \config{\ssa{m'}, \eskip, t', w'} $ and $ m' = \delta'^{-1}(\ssa{m'}) $.

We use the corresponding rule $\textbf{SSA-IF-T}$.  
\[
\inferrule
{
}
{
\config{\ssa{ { m} , [\eif(\etrue, [\bar{\ssa{x}}', \bar{\ssa{x_1}}, \bar{\ssa{x_2}}] , [\bar{\ssa{y}}', \bar{\ssa{y_1}}, \bar{\ssa{y_2}}] ,[\bar{\ssa{z}}', \bar{\ssa{z_1}}, \bar{\ssa{z_2}}] , \ssa{c_1, c_2})]^{l}},t,w} 
\xrightarrow{} \\ \config{\ssa{ m, c_1}; \eifvar(\ssa{\bar{x}', \bar{x_1}});\eifvar(\ssa{\bar{y}', \bar{y_2}});\eifvar(\ssa{\bar{z}', \bar{z_1}}),  t,w } 
}~\textbf{ssa-if-t}
\]
By induction hypothesis on $ \Sigma;\delta ; c_1 \hookrightarrow  \ssa{c_1}; \delta_1;\Sigma_1$, and we know that $\config{m, c_1,  t,w} \to^{*} \config{m', \eskip, t', w'} $, from our assumption that $ m =\delta^{-1}(\ssa{m})$, we know that 
\[\config{\ssa{ { m}, c_1},  t,w} \to^{*} \config{ \ssa{ { m_1 }, \eskip,} t', w' } \land m' = \delta_1^{-1}(\ssa{m_1}) \]
and we then have:
\[
\inferrule
{
  \config{\ssa{ { m}, c_1},  t,w} \to^{*} \config{ \ssa{ { m_1 }, \eskip,} t', w' }
}
{
 \config{\ssa{  m, c_1;} \eifvar(\ssa{\bar{x}', \bar{x_1})};\eifvar(\ssa{\bar{y}', \bar{y_1})};\eifvar(\ssa{\bar{z}', \bar{z_1})},  t,w  }  \to^{*}
 \config{\ssa{ { m_1 [ \bar{x}' \mapsto {m_1}(\bar{x_1}),\bar{y}' \mapsto {m_1}(\bar{y_2}),\bar{z}' \mapsto {m_1}(\bar{z_1}) ] }, \eskip}, t', w'  }
}
\]
Now, we want to show that $ m' = \delta[\bar{x} \mapsto \ssa{\bar{x}'},\bar{y} \mapsto \ssa{\bar{y}'},\bar{z} \mapsto \ssa{\bar{z}'} ]^{-1}(\ssa{ m_1 [ \bar{x}' \mapsto {m_1}(\bar{x_1}),\bar{y}' \mapsto {m_1}(\bar{y_2}),\bar{z}' \mapsto {m_1}(\bar{z_1}) ] }) $.

Unfold the definition, we want to show that $$\forall x  \in ( \dom(\delta) \cup \bar{x} \cup \bar{y} \cup \bar{z} ), (\delta[\bar{x} \mapsto \ssa{\bar{x}'},\bar{y} \mapsto \ssa{\bar{y}'},\bar{z} \mapsto \ssa{\bar{z}'} ](x), m'(x)) \in \ssa{m_1 [ \bar{x}' \mapsto {m_1}(\bar{x_1}),\bar{y}' \mapsto {m_1}(\bar{y_2}),\bar{z}' \mapsto {m_1}(\bar{z_1}) ] } .$$
\begin{enumerate}
    \item For variable $x$ in $\bar{x}$, we can find a corresponding ssa variable $\ssa{x} \in \ssa{\bar{x}'}$, so that $( \ssa{x}, m'(x) ) \in \ssa{ m_1 [\bar{x}' \mapsto m_1(\bar{x_1})] } $. It is because we know $[x \mapsto \ssa{x_1}]$ for certain $\ssa{x_1} \in \ssa{\bar{x_1}}$ in $\delta_1$, then by unfolding  $m' = \delta_1^{-1}(\ssa{m_1})$ and $\ssa{\bar{x_1}} \in \codom(\delta_1)$, we know $(\ssa{x_1}, m'(x)) \in \ssa{m_1}$ so that $m'(x) = \ssa{m_1}(\ssa{x_1})$.
    \item For variable $y \in \bar{y}$, we know that $y \in \dom(\delta_1)$, then $[ y \mapsto \ssa{y_2} ]$ for certain $\ssa{y_2} \in \ssa{\bar{y_2}}$ in $\delta_1$.  So we know that $(\delta_1(y), m'(y) ) \in \ssa{m_1}$, and then $m'(y) = \ssa{m_1(y_2)}$. We can show $(\ssa{y}, m'(y)) \in \ssa{m_1[\bar{y}' \mapsto m_1(\bar{y_2})]}$.
    \item For variable $z \in \bar{z}$, we know that $z \not\in \dom(\delta_1)$ by the definition (otherwise $z$ will appear in $\bar{x}$), then $[ z \mapsto \ssa{z_1} ]$ for certain $ \ssa{z_1} \in \ssa{\bar{z_1}}$ in $\delta$.  We know $(\delta(z), m(z)) \in \ssa{m}$ from our assumption, so we have $ m(z) = \ssa{m(z_1)}$. Because $z$ is not modified in $c_1$, so that $m(z) = m'(z)$. Also $\ssa{m}$ will not shrink during execution and $\ssa{z_1}$ will not be written in $\ssa{c_1}$, so $(\ssa{z_1}, m'(z)) \in \ssa{m_1}$. Then we can show that $ (\ssa{z}, m'(z) ) \in \ssa{m_1[\bar{z}' \mapsto m_1(\bar{z_1})] }$.
    \item For variable $k \in \dom(\delta)- \bar{x} - \bar{y}-\bar{z}$. From our assumption $ m = \delta^{-1}(\ssa{m})$, we can show $(\delta(k), m(k) ) \in \ssa{m}$. We know that $k$ is not written in either branch from our definition, so $(\delta(k), m'(k) ) \in \ssa{m_1} $ .
\end{enumerate}

{
\caseL{
	Case
	$
	\inferrule{
    { \Sigma; \delta ; c \hookrightarrow \ssa{c_1} ; \delta_1; \Sigma_1 }
     \and
    { [ \bar{x}, \ssa{\bar{{x_1}}}, \ssa{\bar{{x_2}}} ] = \delta \bowtie \delta_1 }
    \\
     {\ssa{\bar{{x}}'} \ fresh(\Sigma_1 )}
    \and {\delta' = \delta[\bar{x} \mapsto \ssa{\bar{{x}}'}]}
    \and 
     {\delta' ; \bexpr \hookrightarrow \ssa{\bexpr} }
     \and
    {\ssa{c' = c_1[\bar{x}'/ \bar{x_1}]   } }
    % \and{ \delta' ; c \hookrightarrow \ssa{c'} ; \delta'' }
  }{ 
  \Sigma; \delta ;  \ewhile ~ [\bexpr]^{l} ~ \edo ~ c 
  \hookrightarrow 
  \ssa{\ewhile ~ [\bexpr]^{l}, 0, [\bar{{x}}', \bar{{x_1}}, \bar{{x_2}}] ~ \edo ~ {c} } ; \delta'; \Sigma_1 \cup \{\ssa{\bar{x}'}  \}
}~{\textbf{S-WHILE}}
$
}
}
\\
{
We choose an arbitrary memory $m$ so that $m \vDash \delta$, we choose a trace $t$ and a loop maps $w$. Suppose $ \config{m ,a} \to v_N $. There are two cases, when $v_N=0$, the loop body is not executed so we can easily show that the trace is not modified.
%
When the while loop is still running ($v_N > 0$), we have the following evaluation in the while language:
\[
\inferrule
{
 \empty
}
{
\config{
m, \ewhile ~ [b]^{l} ~ \edo [c]^{l + 1},  t, w 
}
\xrightarrow{} \config{m, c ; 
\eif_w (b, c ; 
\ewhile ~ [b]^{l} ~ \edo [c]^{l + 1},  \eskip),
t, w }
}
~\textbf{while-b}
\]
which follows by the following evaluation:
\[
	\inferrule
{
 m, b \xrightarrow{} b'
}
{
\config{m, \eif_w (b, c,  \eskip) ,  t, w }
\xrightarrow{} \config{m, 
 \eif_w (b', c,  \eskip), t, w }
}
~\textbf{ifw-b}
\]
In the corresponding ssa-form language, we have the corresponding evaluation in the same way by assuming 
$m = \delta^{-1}(\ssa{m})$.
%
\[
	\inferrule
{
 {n = 0 \rightarrow i = 1 }
 \and
 {n > 0 \rightarrow i = 2 }
}
{
\config{
\ssa{m},  
\ssa{\ewhile ~ [\bexpr]^{l}, n, 
[\bar{{x}}', \bar{{x_1}}, \bar{{x_2}}] 
~ \edo ~ {c} 
},  t, w 
}
\xrightarrow{} \\ 
\config{
\ssa{m},
\eif_w 
(\ssa{b[\bar{x_i}/\bar{x'}], [\bar{{x}}', \bar{{x_1}}, \bar{{x_2}}], n,  c[\bar{x_i}/\bar{x'}] }; 
\ssa{
\ewhile ~ [b]^{l}, n+1, 
[\bar{{x}}', \bar{{x_1}}, \bar{{x_2}}]  
~ \edo ~ c} ,  \eskip),
t, w
}
}
~\textbf{ssa-while-b}
\]
This evaluation is followed by the following evaluation:
\[
	\inferrule
{
 \ssa{m, b \xrightarrow{} b'}
}
{
\config{\ssa{m, \eif_w (b, [\bar{{x}}', \bar{{x_1}}, \bar{{x_2}}] , n,  c_1,  c_2)} ,  t, w }
\xrightarrow{} \config{\ssa{ m, 
 \eif_w (b', [\bar{{x}}', \bar{{x_1}}, \bar{{x_2}}] , n , c_1 , c_2 )}, t, w }
}
~\textbf{ssa-ifw-b}
\]
%
Depending on if the counter $n$ is equal to $0$ or not, there are two possible execution paths (the variables $\ssa{\bar{x}}$ is replaced by the $\ssa{\bar{x_1}}$ or $\ssa{\bar{x_2}}$). We start from the first iteration (when $n =0$) when $v_N >0$. 
}
{
By induction hypothsis on the premise $ { \Sigma; \delta ; c \hookrightarrow \ssa{c_1} ; \delta_1; \Sigma_1 }$, we know that 
\[ \config{\ssa{{m}, c'[ \bar{x_1}/\bar{x}'  ]}, t, (w+l)  } \to^{*} \config{\ssa{{m'}, \eskip}, t'_{i}, w'  } \land m' = \delta_1^{-1}(\ssa{m'})   \]
Hence we can conclude that:
\[
  \inferrule{
   \config{\ssa{{m}, c'[ \bar{x_1}/\bar{x}'  ]}, t, (w+l) }  \to^{*} \config{\ssa{{m'}, \eskip}, t'_{1}, w'  }
  }{
  \config{\ssa{ {m}, c'[ \bar{x_1}/\bar{x}'  ];  [\eloop ~ (\valr_N-1), n+1, [\bar{\ssa{x}}', \bar{\ssa{x_1}}, \bar{\ssa{x_2}}] ~  \edo ~ c' ]^{l} },  t, (w + l)  }  \to^{*} \\ \config{ \ssa{{m'}, [\eloop ~ (\valr_N-1), n+1, [\bar{\ssa{x}}', \bar{\ssa{x_1}}, \bar{\ssa{x_2}}] ~  \edo ~ c' ]^{l}}, t'_{1}, w'  } 
  }
\]
%
Then there are two cases, 
%
\begin{enumerate}
     \item  when guard in the $\eif_w$ expression evaluates to $\efalse$, the while loop terminates and exits.
     The execution in the while language is defined in the evaluation rule $\textbf{ifw-false}$ as follows.
     \[
		\inferrule
		{
		 \empty
		}
		{
		\config{\ssa{
		m, \eif_w (
		\efalse, [\bar{{x}}', \bar{{x_1}}, \bar{{x_2}}],   n, 
		c; {\ewhile ~ [b]^{l} ~ \edo ~ c},
		\eskip)
		)} ,  t, w }
		\\
		\xrightarrow{} 
		\config{\ssa{m, 
		{\eskip}; \eifvar(\bar{x'}, \bar{x_i}) }, t, (w - l) }
		}
		~\textbf{ifw-false}
	\]
%
	The corresponding ssa-form evaluation as follows:
	\[
		\inferrule
		{
		 { n = 0 \rightarrow i = 1 }
		 \and
		 {n > 0 \rightarrow i =2}
		}
		{
		\config{\ssa{
		m, \eif_w (
		\efalse, [\bar{{x}}', \bar{{x_1}}, \bar{{x_2}}],   n, 
		{  
		c; \ssa{\ewhile ~ [b]^{l}, n, [\bar{{x}}', \bar{{x_1}}, \bar{{x_2}}]  ~ \edo ~ c},
		\eskip)
		} 
		)} ,  t, w }
		\\
		\xrightarrow{} 
		\config{\ssa{m, 
		{\eskip}; \eifvar(\bar{x'}, \bar{x_i}) }, t, (w - l) }
		}
		~\textbf{ssa-ifw-false}
	\]
	We can see that both traces are not changed during the exit of the while. We need to show that $ m' = \delta^{-1} (\ssa{m'[\bar{x} \mapsto m'(\bar{x_2})]}) $. We know that $[ \bar{x} \mapsto \bar{x_2}]$ in $\delta_1$ from the definition, so we can show that for any variable $\ssa{x_2} \in \bar{x_2}$, $( \ssa{x_2}, m'(x) ) \in \ssa{m'}$. For variables $x \in {\dom(\delta) - \bar{x} } $, the variable is not modified during the execution of $c$ so that we know $m(x) = m'(x)$, and then we can show that $(\delta(x), m'(x)) \in \ssa{m'} $ because $\delta(x)$ is not written in $\ssa{c'[\bar{x_1}/ \bar{x}']}$ .
%
  	\item 
		when guard in the $\eif_w$ expression evaluates to $\etrue$, the while terminates and exits.
     The execution in the while language is defined in the evaluation rule $\textbf{ifw-true}$.
          %
     We want to show that : assuming in the $i-th$ $(i < \ssa{n})$ iteration, starting with $t_i$ and $w_i$ and $m_i = \delta_1^{-1}(\ssa{m_i})$,
     this command is evaluated according to the while language operation semantics as
     	$
		\config{m, \eif_w (\etrue, c ; \ewhile ~ [b]^{l} ~ \edo ~ c, ,  \eskip) ,  t, w }
		\xrightarrow{}^* \config{m, c 
		t, (w + l) }
 		$.
     %
     Then the corresponding ssa form evaluation as follows : 
     %
     \[ 
     \inferrule{}{
     	\config{
		\ssa{
			m, 
			{
			\eif_w (\etrue, [\bar{{x}}', \bar{{x_1}}, \bar{{x_2}}], n,  
			c; \ssa{\ewhile ~ [b]^{l}, n, [\bar{{x}}', \bar{{x_1}}, \bar{{x_2}}]  ~ \edo ~ c},
			\eskip)
			} 
		},  t, w 
		}
		\\
		\xrightarrow{} 
		\config{
		\ssa{m, 
		{
		\eif_w (\etrue, [\bar{{x}}', \bar{{x_1}}, \bar{{x_2}}], n,  
		c; \ssa{\ewhile ~ [b]^{l}, n, [\bar{{x}}', \bar{{x_1}}, \bar{{x_2}}]  ~ \edo ~ c},
		}
		}
		t, (w + l) }
		} 
     \]  
     and $m_i = \delta^{-1}(\ssa{m_i}) $.
     We then have the evaluation in the while language:
     \[
		\inferrule
		{
		 \empty
		}
		{
		\config{m, 
		\eif_w (b, 
		c ; \ewhile ~ [b]^{l} ~ \edo ~ c, 
		\eskip),
		t, w }
		\xrightarrow{} 
		\config{m, 
		c ; \ewhile ~ [b]^{l} ~ \edo ~ c,  
		t, (w + l) }
		}
		~\textbf{ifw-true}
	\]
	We then have the following evaluation:
	\[
		\inferrule
		{
		 \empty
		}
		{
		\config{
		\ssa{
		m, 
		{
		\eif_w (\etrue, [\bar{{x}}', \bar{{x_1}}, \bar{{x_2}}], n,  
		c; \ssa{\ewhile ~ [b]^{l}, n, [\bar{{x}}', \bar{{x_1}}, \bar{{x_2}}]  ~ \edo ~ c},
		\eskip)
		} 
		},  t, w 
		}
		\\
		\xrightarrow{} 
		\config{
		\ssa{m, 
		{
		\eif_w (\etrue, [\bar{{x}}', \bar{{x_1}}, \bar{{x_2}}], n,  
		c; \ssa{\ewhile ~ [b]^{l}, n, [\bar{{x}}', \bar{{x_1}}, \bar{{x_2}}]  ~ \edo ~ c},
		}
		}
		t, (w + l) }
		}
		~\textbf{ssa-ifw-true}
	\]
%
By induction hypothsis on the premise $  { \Sigma; \delta_1 ; c \hookrightarrow \ssa{c_2} ; \delta_1; \Sigma_1 }$, we know that
%
\[
\config{\ssa{{m_i}, c'[ \bar{x_2}/\bar{x}'  ]}, t_i, (w_i+l)  } \to^{*} \config{\ssa{{m_{i+1}}, \eskip}, t_{i+1}, w_{i+1}  } \land m_{i+1} = \delta_1^{-1}(\ssa{m_{i+1}})
\]
%
Hence we can conclude that:
\[
  \inferrule{
   \config{\ssa{{m_i}, c'[ \bar{x_2}/\bar{x}'  ]}, t_i, (w_i+l) }  \to^{*} \config{\ssa{{m_{i+1}}, \eskip}, t_{i+1}, w_{i+1}  }
  }{
  \config{\ssa{ {m_i}, c'[ \bar{x_2}/\bar{x}'  ];  [\eloop ~ (\valr_N-i-1), n+1, [\bar{\ssa{x}}', \bar{\ssa{x_1}}, \bar{\ssa{x_2}}] ~  \edo ~ c' ]^{l} },  t_i, (w_i + l)  }  \to^{*} \\ \config{ \ssa{{m_{i+1}}, [\eloop ~ (\valr_N-i-1), n+1, [\bar{\ssa{x}}', \bar{\ssa{x_1}}, \bar{\ssa{x_2}}] ~  \edo ~ c' ]^{l}}, t_{i+1}, w_{i+1}  } 
  }
\]
So we can show that before the exit of the loop after ($v_N= n $) iterations, we have $t_{n} = t_{n}$ and $m_{n} = \delta_1^{-1}(\ssa{m_{n}})$.
 \end{enumerate}
%
This proof is similar when it comes to the exit as in case 1. 
}
\end{itemize}
%
\end{proof}
%
\clearpage
%
\clearpage