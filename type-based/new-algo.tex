\section{new \THESYSTEM with loop }
\subsection{Analysis Rules/Algorithms of \THESYSTEM}

There are two steps for our algorithm to get the estimation of the adaptivity of a program $\ssa{c}$ in the ssa form. 
\begin{enumerate}
    \item Estimate the variables that are new generated (via assignment) and store these variables in a global list $G$. We have the algorithm of the form : $\ag{ G;w; \ssa{c}}{G';w'} $.
    \item We start to track the dependence between variables in a matrix $M$, whose size is $|G| \times |G|$, and track whether arbitrary variable is assigned with a query result in a vector $V$ with size $|G|$. The algorithm to fill in the matrix is of the form: $\ad{\Gamma ; \ssa{c} ; i_1}{M;V;{i_2}}$. $\Gamma$ is a vector records the variables the current program $\ssa{c}$ depends on, the index $i_1$ is a pointer which refers to the position of the first new-generated variable in $\ssa{c}$ in the global list $G$, and $i_2$ points to the first new variable that is not in $\ssa{c}$ (if exists). 
\end{enumerate}

% We have the judgment of the form $\vdash^{i_1, i_2}_{M,V} ~ c  $.  Our grade is a combination of a matrix $M$, used to track the dependency of variables appeared in the statement $c$, and a vector $V$ indicating the variables associated with results from queries $q$. The size of the matrix $M$ is $L \times L$, and vector $V$ of size $L$, where $L$ is the total size of variables needed in the program $c$, which is fixed per program. We assume the program is in the style of Static Single Assignment.To be more specific, we give a quick example: $x \leftarrow e_1; x \leftarrow e_2 $ will be rewritten as $ x_1 \leftarrow e_1; x_2 \leftarrow e_2$. And the if condition $ \eif ~ e_b \ethen x \leftarrow e_1 \eelse x \leftarrow e_2  $ will look like $ \eif ~ e_b \ethen x_1 \leftarrow e_1 \eelse x_2 \leftarrow e_2  $. As we have seen, SSA requires unique variables, and these newly generated variables will be recorded in the matrix $M$.  Also, the variable at different iteration is treated as different variable in the matrix $M$ and vector $V$.

% The superscript $i_1,i_2$  specify the range of "living" or "active" variables in the matrix and vector. $i_1$ is the starting line (and column) in the matrix where the new generated variables in program $c$ starts to show up. Likewise, $i_2$ states the ending position of active range by $c$.
%  Worth to mention, $i_1,i_2$ can be used to track the exact location of newly generated variables. For example, the assignment statement $x \leftarrow e$ or $x \leftarrow q $ with $c_2 =c_1+1$, tells us the variable $x$ is at the $c_1$th line(column) of the matrix. As we can notice, the loop increases the variables needed in the matrix by $N \times a$ where $N$ is the number of rounds of the loop and $a$ is the size of the variables generated in the loop body. We will have a global map, which maps the variable name to the position in the vector. We call it $GM: VAR \to \mathbb{N}$.

We give an example of $M$ and $V$ of the program $c$.   
$$
c= \begin{array}{c}
\ssa{\assign {x_1} {q}} ;        \\
\ssa{\assign {x_2} {x_1+1}} ;\\
\ssa{\assign {x_3} {x_2+2} }
\end{array}~~~~~~~~~~~~
M =  \left[ \begin{matrix}
 & (x_1) & (x_2) & (x_3) \\
(x_1) & 0 & 0 & 0 \\
(x_2) & 1 & 0 & 0 \\
(x_3) & 1 & 1 & 0 \\
\end{matrix} \right] ~ , V = \left [ \begin{matrix}
(x_1) &  1 \\
(x_2) & 0 \\
(x_3) & 0 \\
\end{matrix} \right ]
$$
Still use the program $c$ as the example, the global list $G$ is now : $ [ x_1 , x_2 , x_3] $. 
The function $\mathsf{Left}$ and $\mathsf{Right}$ is used to generate the corresponding vector of the left side and right side of an assignment. Take $\assign {x_2} {x_1+1} $ as an example, the result is shown as follows.
\[
\sf{L}(1) = \left[ \begin{matrix}
 0  & ~~~(x_1) \\
 1 & ~~~(x_2) \\
 0 & ~~~(x_3) \\
\end{matrix}   \right ] ~~~~~~~~~~~~~~
\sf{R} (x_1+1, 1) = \left[ \begin{matrix} 
   1 & 0 & 0 \\
   (x_1) & (x_2) & (x_3) \\
\end{matrix}  \right]
\]
Now let us think about the loop.
\[\ssa{c_3} \triangleq
\begin{array}{l}
     \left[\ssa{ x_1 \leftarrow q_1}  \right]^1 ; \\
    \eloop ~ [2]^{2} , 0,\\
  \ssa{[x_3 , x_1 , x_2]} 
     ~ \edo
    \\
    ~ \Big( 
    \left[\ssa{ y_1 \leftarrow q_2} \right]^3; \\
    \left[\ssa{x_2 \leftarrow y_1  + x_3 } \right]^5
    \Big) ; \\
     \left[ \assign{z_1}{x_3 + 2}  \right]^{6}
\end{array}
~~~~~~~~~~~~
M =  \left[ \begin{matrix}
 & (x_1) & (x_3^{1}) & (y_1^{1}) & (x_3^{1})  & (x_3^{2}) & (y_1^{2}) & (x_2^{2}) & (x_3^{f}) &  (z_1) \\
(x_1) & 0 & 0 & 0 & 0 & 0 & 0 & 0 &0 &0 \\
(x_3^{1}) & 1 & 0 & 0 & 0 & 0 & 0 & 0&0&0\\
(y_1^{1}) & 0 & 0 & 0 & 0 & 0 & 0& 0& 0 &0\\
(x_2^{1}) & 0 & 1 & 1 & 0 & 0 & 0 & 0& 0&0\\
(x_3^{2}) & 0 & 0 & 0 & 1 & 0 & 0 & 0 & 0&0 \\
(y_1^{2}) & 0 & 0 & 0 & 0 & 0 & 0 & 0& 0&0\\
(x_2^{2}) & 0 & 0 & 0 & 0 & 1 & 1 & 0& 0&0\\
(x_3^{f}) & 1 & 0 & 0 & 0 & 0 & 0 & 1& 0&0\\
(z_1) & 0 & 0 & 0 & 0 & 0 & 0 & 0 & 1 &0 \\
\end{matrix} \right] ~ , V = \left [ \begin{matrix}
(x_1) &  1 \\
(x_3^{1}) & 0 \\
(y_1^{1}) & 1 \\
(x_2^{1}) &  0 \\
(x_3^{2}) & 0 \\
(y_1^{2}) & 1 \\
(x_2^{2}) &  0 \\
(x_3^{f}) &  0 \\
(z_1) &  0 \\
\end{matrix} \right ]
\]

\[\ssa{c_3'} \triangleq
\begin{array}{l}
     \left[\ssa{ x_1 \leftarrow q_1}  \right]^1 ; \\
    \eloop ~ [0]^{2} , 0,\\
  \ssa{[x_3 , x_1 , x_2]} 
     ~ \edo
    \\
    ~ \Big( 
    \left[\ssa{ y_1 \leftarrow q_2} \right]^3; \\
    \left[\ssa{x_2 \leftarrow y_1  + x_3 } \right]^5
    \Big) ; \\
    \left[ \assign{z_1}{x_3 + 2}  \right]^{6}
\end{array}
~~~~~~~~~~~~
M =  \left[ \begin{matrix}
 & (x_1) & (x_3^{f}) & (z_1)  \\
(x_1) & 0 & 0 & 0 \\
(x_3^{f}) & 1 & 0 & 0 \\
(z_1^{2}) & 0 & 1 & 0 \\
\end{matrix} \right] ~ , V = \left [ \begin{matrix}
(x_1) &  1 \\
(x_3^{f}) & 0 \\
(z_1) &  0 \\
\end{matrix} \right ]
\]
We can now look at the if statement.
\[ c_4 \triangleq
\begin{array}{l}
   \left[ x \leftarrow q_1 \right]^1; \\
   \left[y \leftarrow q_2\right]^2 ; \\
    \eif \;( x + y == 5 )^3\; \\
    \mathsf{then} \;\left[ x \leftarrow q_3 \right]^4 \; \\
    \mathsf{else} (\;\left[ x \leftarrow q_4 \right]^5 ; \\
    y \leftarrow 2 ) ;\\
   \left[ z \leftarrow x +y \right]^6; \\
\end{array}
\hspace{20pt} \hookrightarrow \hspace{20pt}
%
 \ssa{c_4} \triangleq
\begin{array}{l}
   \left[ \ssa{ x_1 \leftarrow q_1} \right]^1; \\
   \left[\ssa{ y_1 \leftarrow q_2} \right]^2 ; \\
    \eif \;( \ssa{ x_1 + y_1 == 5} )^3, [ x_4,x_2,x_3 ],[] ,[y_3,y_1,y_2 ]\; \\
    \mathsf{then} \;\left[ \ssa{ x_2 \leftarrow q_3}\right]^4 \; \\
    \mathsf{else} (\;\left[ \ssa{x_3 \leftarrow q_4} \right]^5 ; \\
     \ssa{y_2 \leftarrow 2} ) \\
   \left[ \ssa{ z_1 \leftarrow x_4 +y_3 }\right]^6; \\
\end{array}
\]
\[
M_{c4} =  \left[ \begin{matrix}
 & (x_1) & (y_1) & (x_2) & (x_3)  & (y_2) & (x_4) & (y_3) & (z_1)  \\
(x_1) & 0 & 0 & 0 & 0 & 0 & 0 & 0 &0  \\
(y_1) & 0 & 0 & 0 & 0 & 0 & 0 & 0&0\\
(x_2) & 0 & 0 & 0 & 0 & 0 & 0& 0& 0 \\
(x_3) & 0 & 0 & 0 & 0 & 0 & 0 & 0& 0\\
(y_2) & 0 & 0 & 0 & 0 & 0 & 0 & 0 & 0 \\
(x_4) & 0 & 0 & 1 & 1 & 0 & 0 & 0 &0\\
(y_3) & 0 & 1 & 0 & 0 & 1 & 0 & 0 &0\\
(z_1) & 0 & 0 & 0 & 0 & 0 & 1 & 1 & 0  \\
\end{matrix} \right] ~ , V_{c4} = \left [ \begin{matrix}
(x_1) &  1 \\
(y_1) & 1 \\
(x_2) & 1 \\
(x_3) &  1 \\
(y_2) & 0 \\
(x_4) & 0 \\
(y_3) &  0 \\
(z_1) &  0 \\
\end{matrix} \right ]
\]
% We consider to have the superscript to denote the iteration number (or map if we have nested loop), as shown in the above matrix and vector. The global map $G$ is generated by analysing the program. We can estimate the variables needed in the loop by using the loop number $N$ and the loop body. In this example, the global map for $c_3$ : $ \{ x_1 \to 1, x_2^{1} \to 2, y_1^{1} \to 3 , x_3^{1} \to 4 , x_2^{2} \to 5 , y_1^{2} \to 6 , x_3^{2} \to 7  \} $.  
% By default, $G(x_2)$ gives the location for the first appearance of the variable $x_2$. We can also allow $G(x_2 , 2)$ to get the location of the second iteration $x_2^{2}$. We also allow $G(x, i, i+n)$ to return a set of locations where $x$ appears in the vector in the certain range $[i, i+n]$, which helps to locate variables in the loop.




% Also, to be able to track the relation between variables in varied iterations in the loop. we define a dependent map $\mathsf{DM}$ based on command $c$ to provide the dependency relation(syntactically) between variables. $\mathsf{VAR}(\expr)$ gives the set of variables appears in the expression $\expr$.
% \[
% \begin{array}{lll}
% \mathsf{DM} (c_1; c_2) & \triangleq &  \mathsf{DM} (c_1) \uplus \mathsf{DM} (c_2)  \\
% \mathsf{DM} (x_1 \leftarrow \expr ) & \triangleq & \{  x_1 \to \mathsf{VARS}(\expr)  \}
% \end{array}
% \]

\subsection{The algorithm to estimate the matrix and vector}
We first generate a list of variables $G$ that will be assigned with values (via the command $\assign{x}{e}$ or $\assign{x}{q}$). 

 \begin{mathpar}
\inferrule
{
}
{ \ag{G ;w; \ssa{[\assign {x}{\expr}]^{l}}}{G ++ [\ssa{x}^{(l,w)}];w}
% G ;w; \ssa{[\assign {x}{\expr}]^{l}} \to G ++ [x^{(l,w)}];w 
}
~\textbf{ag-asgn}
\and
\inferrule
{
}
{ \ag{G ;w;  [ \assign{\ssa{x}}{q(\ssa{\expr})}]^{l}}{  G ++ [\ssa{x}^{(l,w)}] ; w} 
}~\textbf{ag-query}
%
\and 
%
\inferrule
{
\ag{G; w; \ssa{c_1}}{  G_1;w_1}
\and 
 \ag{G_1;w ; \ssa{c_2}}{  G_2; w_2}
 \\
 {G_3 = G_2 ++ \ssa{[\bar{x}^{(l,w)}]++ \ssa{[\bar{y}^{(l,w)}]}++ \ssa{[\bar{z}^{(l,w)}]} }}
}
{
\ag{G; w;
[\eif(\ssa{\bexpr},[ \bar{\ssa{x}}, \bar{\ssa{x_1}}, \bar{\ssa{x_2}}] ,[ \bar{\ssa{y}}, \bar{\ssa{y_1}}, \bar{\ssa{y_2}}],[ \bar{\ssa{z}}, \bar{\ssa{z_1}}, \bar{\ssa{z_2}}], \ssa{ c_1, c_2)}]^{l} }{ G_3 ;w}
}~\textbf{ag-if}
%
%
%
\and 
%
\inferrule
{
\ag{G; w; \ssa{c_1}}{ G_1; w_1}
\and 
\ag{G_1;w_1; \ssa{c_2}}{ G_2; w_2}
}
{
\ag{G; w;
\ssa{(c_1 ; c_2)}}{  G_2 ; w_2}
}
~\textbf{ag-seq}
\and 
\inferrule
{
{G_0 = G \quad w_0 =w }
\and
\forall 0 \leq z < N. 
{ \ag{ G_z ++ \ssa{[\bar{x}^{(l, {w_z}+l)}]} ; (w_z+l); \ssa{c}}{ G_{z+1} ; w_{z+1}}  }
\\
{G_f = G_N ++ \ssa{[\bar{x}^{(l, w_N \setminus l)}]} }
\and
{ \ssa{\aexpr} =  {N}  }
}
{\ag{G; w; [\eloop ~ \ssa{\aexpr}, n, [\bar{\ssa{x}}, \bar{\ssa{x_1}}, \bar{\ssa{x_2}}] ~ \edo ~ \ssa{c}]^{l} }{ G_f; w_N\setminus l }
}~\textbf{ag-loop}
\end{mathpar}


%
%
% \paragraph{Analysis Rules.}
% \[\begin{array}{ll}
%     \mathcal{A}( \assign x \expr )( \Gamma , i )  & =  ( \mathsf{L}(x) * ( \mathsf{R}(\expr) + \Gamma ), V, i+1 )\\
%     \mathcal{A}( \assign x q)( \Gamma ,  i )  & = ( \mathsf{L}(x) * ( \mathsf{R}(\emptyset) + \Gamma) , \mathsf{L}(x) , i+1 )\\
%     \mathcal{A}( \eif ~ e_b \ethen c_1 \eelse c_2 )( \Gamma , i ) & =   \elet \; (M_1, v_1, i_1) =  \mathcal{A}(C_1)(\Gamma +\mathsf{R}(e_b) , i)
%     \ein \; \\
%     &  \elet \;  (M_2, v_2, i_2)= \mathcal{A}(C_2) (\Gamma +\mathsf{R}(e_b) ,i_1) \ein \; \\
%     & (  M_1 \uplus M_2, V_1 \uplus V_2   , i_2 )
%     \\
%     \mathcal{A}( c_1 ; c_2 )( \Gamma ,  i )  & =  \elet \;     (M_1, v_1, i_1) = 
%     \mathcal{A}(c_1) (\Gamma  +\mathsf{R}(e_b) , i)
%     \ein \; \\
%     &  \elet \;  (M_2, v_2, i_2) =                      \mathcal{A}(c_2)(\Gamma +\mathsf{R}(e_b) ,
%       i_1) \ein \; \\ 
%       & (  M_1 \cdot M_2, V_1 \uplus V_2   , i_2 )    \\
%      \mathcal{A}( \eloop ~ \expr_N ~ (c_1) ~ \edo ~ c_2  )( \Gamma ,  i )  & =  \elet \;     (M_1, v_1, i+a) = 
%     \mathcal{A}(c_1;c_2 ) (\Gamma , i)
%     \ein \; \\
%     & ( M_{i,a}^N(c_1), V_{i, a}^N , i + N*a ) \\
%  \mathcal{A}( \eswitch(\expr, x,(v_j \rightarrow q_j )  )( \Gamma ,  i+j )  & =  \elet \;     (M_j, v_j, i+j) = 
%     \mathcal{A}(x_j \leftarrow q_j ) (\Gamma + \mathsf{R}(e), i+j-1)     
%   \ein \\
%   & ( \sum_{j=0}^{N} M_j, \sum_{j=0}^{N} V_j, i + N ) \quad j \in \{1, \dots, N\}  \\
%     \end{array}
% \]
%
%
\paragraph{Analysis Logic Rules.}
%
%
$\Gamma$ is a matrix of one row and $N$ columns, $N = |G|=|V|$.\\ 

$\mathsf{L(i)}$ generates a matrix of one column, $N$ rows, where the $i-th$ row is $1$, all the other rows are $0$.\\

$\mathsf{R(e, i)}$ generates a matrix of one row and $N$ columns, where the locations of variables in $e$ is marked as $1$. To handle loop, for instance, the variable $y$ appears many times in $G$, the argument $i$ helps to find the location of the current living variable $y$ in the expression $e$, which is the latest $y$ with the largest location $i_y< i$ in our global variable list $G$.\\ 


{$ \forall 0 \leq z < |\bar{x}|. \bar{x}(z) = x_z, \bar{x_1}(z) = x_{1z}, \bar{x_2}(z) = x_{2z} $ } \\

$ \Gamma \vdash_{M,v_{\emptyset}}^{i, i+ |\ssa{\bar{x}}|} \ssa{[ \bar{x},\bar{x_1},\bar{x_2}   ]} \triangleq { \forall 0 \leq z < |\bar{x}|.  \Gamma \vdash_{M_{x_z}, V_{\emptyset}}^{i+z, i+z+1 } x_z \leftarrow x_{1z} + x_{2z} }$ where $M = \sum_{z\in [|\bar{x}|] }M_{x_z} $\\

\framebox{$ {\Gamma} \vdash^{i_1, i_2}_{M,V} ~ c $}
%
% \begin{mathpar}
% \inferrule
% {M = \mathsf{L}(i) * ( \mathsf{R}(\expr,i) + \Gamma )
% }
% {\Gamma \vdash_{M, V_{\emptyset}}^{(i, i+1)} [\assign {\ssa{x}}{\ssa{\expr}} ]^{l}
% }
% ~\textbf{asgn}
% \and
% \inferrule
% {M = \mathsf{L}(i) * ( \Gamma)
% \\
% V= \mathsf{L}(i)
% }
% { \Gamma \vdash^{(i, i+1)}_{M, V} [ \assign{\ssa{x}}{q} ]^{l} 
% }~\textbf{query}
% %
% \and 
% %
% \inferrule
% {
% \Gamma + \mathsf{R}(\bexpr, i_1) \vdash^{(i_1, i_2)}_{M_1, V_1} \ssa{c_1} 
% % : \Phi \land \bexpr \Rightarrow \Psi
% \and 
% \Gamma + \mathsf{R}(\bexpr, i_1) \vdash^{(i_2, i_3)}_{M_2, V_2} \ssa{c_2} 
% % : \Phi \land \neg \bexpr \Rightarrow \Psi
% \\
% { \forall 0 \leq j < |\bar{x}|. \bar{x}(j) = x_j, \bar{x_1}(j) = x_{1j}, \bar{x_2}(j) = x_{2j}  }
% \\
% { \forall 0 \leq j < |\bar{x}|.  \Gamma \vdash_{M_{x_j}, V_{\emptyset}}^{i_3+j, i_3+j+1 } x_j \leftarrow x_{1j} + x_{2j} }
% \and
% { \forall 0 \leq j < |\bar{y}|.  \Gamma \vdash_{M_{y_j}, V_{\emptyset}}^{i_3+|\bar{x}|+j, i_3+|\bar{x}|+j+1 } y_j \leftarrow y_{1j} + y_{2j} }
% \\
% { \forall 0 \leq j < |\bar{z}|.  \Gamma \vdash_{M_{z_j}, V_{\emptyset}}^{i_3+|\bar{x}|+|\bar{y}|+j, i_3+|\bar{x}|+|\bar{y}|+j+1 } z_j \leftarrow z_{1j} + z_{2j} }
% \and
% {M = (M_1+M_2)+ \sum_{j\in [|\bar{x}|] }M_{x_j} + \sum_{j\in [|\bar{y}|] }M_{y_j} + \sum_{j\in [|\bar{z}|] }M_{z_j} }
% }
% {
% \Gamma \vdash^{(i_1, i_3+|\bar{x}|+|\bar{y}|+|\bar{z}|)}_{M, V_1 \uplus V_2 } 
% [\eif(\sbexpr,[ \bar{\ssa{x}}, \bar{\ssa{x_1}}, \bar{\ssa{x_2}}] ,[ \bar{\ssa{y}}, \bar{\ssa{y_1}}, \bar{\ssa{y_2}}] , [ \bar{\ssa{z}}, \bar{\ssa{z_1}}, \bar{\ssa{z_2}}] , \ssa{ c_1, c_2)}]^{l}
% }~\textbf{if}
% %
% %
% %
% \and 
% %
% \inferrule
% {
% \Gamma \vdash^{(i_1, i_2)}_{M_1, V_1} \ssa{c_1} 
% % : \Phi \Rightarrow \Phi_1
% \and 
% \Gamma \vdash^{(i_2, i_3)}_{M_2, V_2} \ssa{c_2} 
% % : \Phi_1 \Rightarrow \Psi 
% }
% {
% \Gamma \vdash^{(i_1, i_3)}_{M_1 \green{;} M_2, V_1 \uplus V_2}
% \ssa{c_1 ; c_2} 
% % : \Phi \Rightarrow \Psi
% }
% ~\textbf{seq}
% \and 
% \inferrule
% {
% B= |\ssa{\bar{x}}| 
% \and
% {\Gamma \vdash^{(i, i+B)}_{M_{10}, V_{10}} [\bar{\ssa{x}}, \bar{\ssa{x_1}}, \bar{\ssa{x_2}}] }
% \and
% {\Gamma \vdash^{(i+B,i+B+A )}_{M_{20}, V_{20}} \ssa{c} 
% }
% \\
% \forall 1 \leq j < N. 
% {\Gamma \vdash^{(i+j*(B+A), i+B+j*(B+A))}_{M_{1j}, V_{1j}}  } [\bar{\ssa{x}}, \bar{\ssa{x_1}}, \bar{\ssa{x_2}}]
% \and
% {\Gamma \vdash^{(i+B+j*(B+A),i+B+A+j*(B+A) )}_{M_{2j}, V_{2j}} \ssa{c} 
% % : \Phi \land e_n = \lceil{z+1}\rceil \Rightarrow \Psi 
% }
% \\
% {\Gamma \vdash^{(i+N*(B+A) ,i+N*(B+A)+B )}_{M, V} [\bar{\ssa{x}}, \bar{\ssa{x_1}}, \bar{\ssa{x_2}}]
% % : \Psi \Rightarrow \Phi \land e_N = \lceil{z}\rceil 
% }
% \and
% { \ssa{a} = \lceil {N} \rceil }
% \and
% {M' = M+ \sum_{0 \leq j <N} M_{1j}+M_{2j}  }
% \and
% {V' = V \uplus \sum_{0 \leq j <N} V_{1j} \uplus V_{2j}  }
% }
% {\Gamma \vdash^{(i, i+N*(B+A)+B   )}_{M', V'} 
% [\eloop ~ \ssa{\aexpr}, 0, [\bar{\ssa{x}}, \bar{\ssa{x_1}}, \bar{\ssa{x_2}}] ~ \edo ~ \ssa{c}]^{l}
% % : \Phi \land \expr_N = \lceil { N} \rceil \Rightarrow \Phi \land \expr_N = \lceil{0}\rceil
% }~\textbf{loop}
% % \and 
% % \inferrule
% % {
% % \Gamma \vdash^{(i,i+a )}_{M, V} c 
% % }
% % {\Gamma \vdash^{(i, i+ N*a)}_{M_{i,a}^N(f), V_{i, a}^N} 
% % \ewhile([\bexpr]^l,   c) : \phi \Rightarrow \psi
% % }~\textbf{while}
% %
% \and
% %
% \inferrule
% { \Gamma + \mathsf{R}(\expr,i) \vdash^{(i, i+1)}_{M, V} \assign{ x}{q_j} 
% % : \Phi \Rightarrow \Psi
% \\
% j \in \{1, \dots, N\}     }
% {\Gamma \vdash^{(i, i+1)}_{ M,V } 
% [\eswitch(\ssa{\expr}, \ssa{x},(v_j \rightarrow q_j ) ]^{l}
% % : \Phi \Rightarrow \Psi 
% }
% ~\textbf{switch}
% % %
% % \and
% % %
% % \inferrule
% % { 
% % \vDash 
% % \Phi \Rightarrow \Phi'  
% % \and
% % \Gamma \vdash^{(i_1, i_2)}_{(M',V')} c : \Phi' \Rightarrow \Psi'
% % \and
% % \vDash \Psi' \Rightarrow \Psi
% % \and 
% % \Phi \vDash M' \leq M
% % \and 
% % \Phi \vDash V' \leq V
% % }
% % {\Gamma \vdash^{(i_1, i_2)}_{(M,V)} c 
% % : \Phi \Rightarrow \Psi
% % }
% % ~\textbf{conseq}
% \end{mathpar}
\begin{mathpar}
\inferrule
{M = \mathsf{L}(i) * ( \mathsf{R}(\ssa{\expr},i) + \Gamma )
}
{
 \ad{\Gamma;[\assign {\ssa{x}}{\ssa{\expr}} ]^{l}; i }{M; V_{\emptyset}; i+1 }
% \Gamma \vdash_{M, V_{\emptyset}}^{(i, i+1)} [\assign {\ssa{x}}{\ssa{\expr}} ]^{l}
}
~\textbf{ad-asgn}
\and
\inferrule
{M = \mathsf{L}(i) * ( \mathsf{R}(\ssa{\expr},i) + \Gamma )
\\
V= \mathsf{L}(i)
}
{ 
\ad{\Gamma;[ \assign{\ssa{x}}{q(\ssa{\expr})} ]^{l} ; i }{M;V;i+1}
%  \vdash^{(i, i+1)}_{M, V} [ \assign{\ssa{x}}{q(\ssa{\expr})} ]^{l} 
}~\textbf{ad-query}
%
\and 
%
\inferrule
{
{\ad{\Gamma + \mathsf{R}(\ssa{\bexpr}, i_1); \ssa{c_1} ; i_1 }{ M_1;V_1;i_2 }}
% \Gamma + \mathsf{R}(\bexpr, i_1) \vdash^{(i_1, i_2)}_{M_1, V_1} \ssa{c_1} 
% : \Phi \land \bexpr \Rightarrow \Psi
\and 
{\ad{\Gamma + \mathsf{R}(\ssa{\bexpr}, i_1);\ssa{c_2} ; i_2 }{ M_2; V_2 ;i_3}}
% \Gamma + \mathsf{R}(\ssa{\bexpr}, i_1) \vdash^{(i_2, i_3)}_{M_2, V_2} \ssa{c_2} 
% : \Phi \land \neg \bexpr \Rightarrow \Psi
\\
% { \forall 0 \leq j < |\bar{x}|. \bar{x}(j) = x_j, \bar{x_1}(j) = x_{1j}, \bar{x_2}(j) = x_{2j}  }
{\ad{\Gamma; [ \bar{\ssa{x}}, \bar{\ssa{x_1}}, \bar{\ssa{x_2}}]; i_3 }{ M_x; V_{\emptyset}; i_3+|\bar{\ssa{x}}| }}
%
\and
%
{\ad{\Gamma; [ \bar{\ssa{y}}, \bar{\ssa{y_1}}, \bar{\ssa{y_2}}]; i_3+|\bar{\ssa{x}}| }{ M_y; V_{\emptyset}; i_3+|\bar{\ssa{x}}|+|\bar{\ssa{y}}| }}
%
\\
%
{\ad{\Gamma; [ \bar{\ssa{z}}, \bar{\ssa{z_1}}, \bar{\ssa{z_2}}]; i_3+|\bar{\ssa{x}}|+ |\bar{\ssa{y}}|}{ M_y; V_{\emptyset}; i_3+|\bar{\ssa{x}}|+|\bar{\ssa{y}}| + |\bar{\ssa{z}}| }}
\\
{M = (M_1+M_2)+ M_x+M_y +M_z }
}
{
\ad{\Gamma ; \eif([\ssa{\bexpr}]^{l},[ \bar{\ssa{x}}, \bar{\ssa{x_1}}, \bar{\ssa{x_2}}] ,[ \bar{\ssa{y}}, \bar{\ssa{y_1}}, \bar{\ssa{y_2}}] , [ \bar{\ssa{z}}, \bar{\ssa{z_1}}, \bar{\ssa{z_2}}] , \ssa{ c_1, c_2)} ; i_1}{ M ;V_1 \uplus V_2  ; i_3+|\bar{x}|+|\bar{y}|+|\bar{z}| }
}~\textbf{ad-if}
%
%
%
\and 
%
\inferrule
{
{\ad{\Gamma; \ssa{c_1} ; i_1 }{ M_1 ; V_1; i_2 }  }
% \Gamma \vdash^{(i_1, i_2)}_{M_1, V_1} \ssa{c_1} 
% : \Phi \Rightarrow \Phi_1
\and 
{\ad{\Gamma;\ssa{c_2}; i_2}{M_2; V_2 ;i_3 }}
% \Gamma \vdash^{(i_2, i_3)}_{M_2, V_2} \ssa{c_2} 
% : \Phi_1 \Rightarrow \Psi 
}
{
\ad{\Gamma ; (\ssa{c_1 ; c_2} ) ; i_1}{(M_1 {;} M_2) ; V_1 \uplus V_2 ; i_3  }
% \Gamma \vdash^{(i_1, i_3)}_{M_1 {;} M_2, V_1 \uplus V_2}
% \ssa{c_1 ; c_2} 
% : \Phi \Rightarrow \Psi
}
~\textbf{ad-seq}
\and 
\inferrule
{
B= |\ssa{\bar{x}}| \and {A = |\ssa{c}|}
% \and
% {\Gamma \vdash^{(i, i+B)}_{M_{10}, V_{10}} [\bar{\ssa{x}}, \bar{\ssa{x_1}}, \bar{\ssa{x_2}}] }
% \and
% {\Gamma \vdash^{(i+B,i+B+A )}_{M_{20}, V_{20}} \ssa{c} 
% }
\\
\forall 0 \leq j < N. 
{\ad{\Gamma;[\bar{\ssa{x}}, \bar{\ssa{x_1}}, \bar{\ssa{x_2}}]; i+ j*(B+A) }{M_{1j};V_{1j}; i+B+j*(B+A) }}
% {\Gamma \vdash^{(i+j*(B+A), i+B+j*(B+A))}_{M_{1j}, V_{1j}}  } [\bar{\ssa{x}}, \bar{\ssa{x_1}}, \bar{\ssa{x_2}}]
\\
{
\ad{\Gamma;\ssa{c} ; i+B+j*(B+A)  }{M_{2j}; V_{2j}; i+B+A+j*(B+A) }
% \Gamma \vdash^{(i+B+j*(B+A),i+B+A+j*(B+A) )}_{M_{2j}, V_{2j}} \ssa{c} 
% : \Phi \land e_n = \lceil{z+1}\rceil \Rightarrow \Psi 
}
\\
{
\ad{\Gamma ; [\bar{\ssa{x}}, \bar{\ssa{x_1}}, \bar{\ssa{x_2}}] ; i+N*(B+A) }{M; V ;i+N*(B+A)+B}
% \Gamma \vdash^{(i+N*(B+A) ,i+N*(B+A)+B )}_{M, V} [\bar{\ssa{x}}, \bar{\ssa{x_1}}, \bar{\ssa{x_2}}]
% : \Psi \Rightarrow \Phi \land e_N = \lceil{z}\rceil 
}
\\
{ \ssa{\aexpr} =  {N}  }
\and
{M' = M+ \sum_{0 \leq j <N}( M_{1j}+M_{2j})  }
\and
{V' = V \uplus \sum_{0 \leq j <N}( V_{1j} \uplus V_{2j})  }
}
{
\ad{\Gamma;\eloop ~ [\ssa{\aexpr}]^{l}, ~0, [\bar{\ssa{x}}, \bar{\ssa{x_1}}, \bar{\ssa{x_2}}] ~ \edo ~ \ssa{c}, i }{ M';V' ;i+N*(B+A)+B }
%  \vdash^{(i,   )}_{M', V'} 
% : \Phi \land \expr_N = \lceil { N} \rceil \Rightarrow \Phi \land \expr_N = \lceil{0}\rceil
}~\textbf{ad-loop}
\end{mathpar}
%
\begin{figure}
   \[
 \begin{array}{lll}
    |[\eswitch(\ssa{\expr}, \ssa{x},(v_j \rightarrow q_j )]^{l} |_{low}  &=& [\eswitch(|\ssa{\expr}|_{low}, |x|_{low},(v_j \rightarrow q_j )]^{l} \\
    | [\eloop ~ \ssa{\aexpr}, n, [\bar{\ssa{x}}, \bar{\ssa{x_1}}, \bar{\ssa{x_2}}] ~ \edo ~ \ssa{c}]^{l}|_{low}  &=& [\eloop ~ |\ssa{\aexpr}|_{low},  ~ \edo ~ |\ssa{c}|_{low}]^{l} \\
      |\ssa{c_1 ; c_2}|_{low}  &=& |\ssa{c_1}|_{low} ; |\ssa{c_2}|_{low} \\
       |[\eif(\sbexpr,[ \bar{\ssa{x}}, \bar{\ssa{x_1}}, \bar{\ssa{x_2}}] ,[ \bar{\ssa{y}}, \bar{\ssa{y_1}}, \bar{\ssa{y_2}}] , [ \bar{\ssa{z}}, \bar{\ssa{z_1}}, \bar{\ssa{z_2}}] , \ssa{ c_1, c_2)}]^{l}|_{low}  &=&
       [\eif(|\sbexpr|_{low}, |\ssa{ c_1}|_{low}, |\ssa{c_2}|_{low})]^{l}\\
       | [\assign {\ssa{x}}{\ssa{\expr}}]^{l}|_{low} & = & [\assign {|\ssa{x}|_{low}}{|\ssa{\expr}|_{low}} ]^{l}  \\
       | [\assign {\ssa{x}}{q} ]^{l} |_{low} & = & [\assign {|\ssa{x}|_{low}}{q}]^{l} \\
       |x_i|_{low} & = & x \\
       |n |_{low} & = & n \\
      | \ssa{\aexpr_1} \oplus_{a} \ssa{\aexpr_2} |_{low} & = &  |\ssa{\aexpr_1}|_{low} \oplus_a |\ssa{\aexpr_2}|_{low} \\
      | \ssa{\bexpr_1} \oplus_{b} \ssa{\bexpr_2} |_{low} & = &  |\ssa{\bexpr_1}|_{low} \oplus_b |\ssa{\bexpr_2}|_{low}
 \end{array}
\]
    \caption{The erasure of SSA}
    \label{fig:ssa_erasure}
\end{figure}


\[
\begin{array}{lll}
M_1 ; M_2 & := & M_2 \cdot M_1 + M_1 + M_2      \\
V_1 \uplus V_2 & := & \left\{
\begin{array}{ll}
1 & (V_1[i] = 1 \lor V_2[i] = 1) \land i = 1, \cdots, N \land |V_1| = |V_2|\\
0 & o.w.
\end{array}\right.\\
%
% M_1 \uplus M_2 & := & \left\{
% \begin{array}{ll}
% 1 & (M_1[i][j] = 1  \lor M_2[i][j] = 1) \land i, j = 1, \cdots, N \land |M_1| = |M_2|\\
% 0 & (M_1[i][j] = 0  \land M_2[i, j] = 0) \land i, j = 1, \cdots, N \land |M_1| = |M_2|\\
% \bot & o.w.
% \end{array}\right.\\
%
% V_{(i, a)}^N
% & := & \left\{
% \begin{array}{ll}
% V[i+ o*a, i + (o + 1) * a-1] = V[i, i + a-1] & 
%  o = 1, \cdots, N - 1 \\
% \bot & o.w.
% \end{array}\right.\\
% %
% M_{(i, a)}^N (c)
% & := & \left\{
% \begin{array}{ll}
% M[i+ o*a, i + (o + 1) * a-1][i + o*a, i + (o + 1) * a-1] & \\
% = M[i, i + a-1][i, i+ a-1] & 
%  o = 1, \cdots, N - 1 \\
% M[i+ o*a,i + (o + 1) * a-1][0, i + o * a-1] = 
% 0 & 
%  o = 1, \cdots, N - 1 \\
% M[0, i + o * a-1][i+ o*a, i + (o + 1) * a-1] & \\
% =  M[0, i + (o - 1) * a-1][i+ (o - 1)*a, i + o * a-1] & 
%  o = 1, \cdots, N - 1 \\
%  \qquad & \qquad  \qquad  \\
% M[l][k] = 
% 1& 
% \begin{array}{l}
% \forall x_l \in  \mathsf{DM}(c). \forall x_k \in  \mathsf{DM}(c)(x_l).\\
%  for \quad o = 0, \cdots, N . \\
% l \in G(x_l,i+ o*a, i+(o+1)*a-1) \land \\
% k \in G(x_k,0, i + (o ) * a-1) \land \\
% \end{array}\\
% \bot & o.w.
% \end{array}\right.\\
%
\end{array}
\]
%
% \begin{center}
% \begin{tabular}{p{15pt}|p{15pt}|p{15pt}||p{15pt}|p{15pt}
% |p{15pt}||p{15pt}|p{15pt}|
% p{15pt}|p{15pt}|p{15pt}|p{15pt}|p{15pt}| } 
%  1 & $\cdots$ & i-1 & i & $\cdots$ & \tiny{i+a-1} & {\tiny i+a } 
% & $\cdots$ & {\tiny{i+2a-1} }
% & $\cdots$ & {\tiny i+N*a-1} & {\tiny i+N*a} & $\cdots$ \\
% \hline
% $\cdots$  & \cellcolor{green} & \cellcolor{green} & \cellcolor{sandstorm} 0 & \cellcolor{sandstorm} 0 & \cellcolor{sandstorm} 0 & \cellcolor{sandstorm} 0 & \cellcolor{sandstorm} 0 & \cellcolor{sandstorm} 0 &  &  &  & \\[10pt]
% \hline
% i-1 & \cellcolor{green} & \cellcolor{green} & \cellcolor{sandstorm} 0 & \cellcolor{sandstorm} 0 & \cellcolor{sandstorm} 0 & \cellcolor{sandstorm} 0 &\cellcolor{sandstorm} 0 & \cellcolor{sandstorm} 0 &  &  &  &  \\ [10pt]
% \hline
% i & \cellcolor{periwinkle} & \cellcolor{periwinkle} & \cellcolor{pink} & \cellcolor{pink} &\cellcolor{pink} & \cellcolor{sandstorm} 0 &
% \cellcolor{sandstorm} 0 &
% \cellcolor{sandstorm} 0 &&&& \\ [10pt]
% \hline
% $\cdots$ & \cellcolor{periwinkle} & \cellcolor{periwinkle}
% &\cellcolor{pink} &\cellcolor{pink}&\cellcolor{pink} &
% \cellcolor{sandstorm} 0 & \cellcolor{sandstorm} 0 &
% \cellcolor{sandstorm} 0 &&&& \\ [10pt]
% \hline
% i+a-1 &\cellcolor{periwinkle} &\cellcolor{periwinkle} & \cellcolor{pink} & \cellcolor{pink} & \cellcolor{pink} 
% & \cellcolor{sandstorm} 0 & \cellcolor{sandstorm} 0 
% & \cellcolor{sandstorm} 0 &&&& \\ [10pt]
% \hline \hline
% {\scriptsize i+a }  & \cellcolor{periwinkle} & \cellcolor{periwinkle} & \cellcolor{trueblue} &\cellcolor{trueblue}
% & \cellcolor{trueblue}& \cellcolor{pink} 
% &\cellcolor{pink} & \cellcolor{pink} & &&& \\ [10pt]
% \hline
% $\cdots$ &\cellcolor{periwinkle} &\cellcolor{periwinkle} & \cellcolor{trueblue}  & \cellcolor{trueblue} c & \cellcolor{trueblue} 
% & \cellcolor{pink} & \cellcolor{pink} &\cellcolor{pink} 
% &&&& \\ [10pt]
% \hline
% {\small i+2a-1 } &\cellcolor{periwinkle} & \cellcolor{periwinkle} & \cellcolor{trueblue} 
% & \cellcolor{trueblue}  & \cellcolor{trueblue}
% & \cellcolor{pink} & \cellcolor{pink} & \cellcolor{pink} 
% &&&& \\ [10pt]
% \hline
% $\cdots$ & &&&&&&&&&&&  \\ [10pt]
% \hline
% {\tiny i+N*a-1 } & &&&&&&&&&&& \\ [10pt]
% \hline
% {\tiny i+N*a} & &&&&&&&&&&&\\ [10pt]
% \hline
% $\cdots$ & &&&&&&&&&&&\\ [10pt]
% \hline
% \end{tabular}
% \end{center}
%
%
%
       
% \begin{defn}
% [Validity of hoare triple]
% If $ c : \psi \Rightarrow \phi$, for any memory $m$ and database $D$ s.t., $\psi(m)$ holds, for any trace $t$, loop maps $w$ so that $ \config{m, c, t,w} \rightarrow^{*} \config{m', \eskip, t', w'}$, then $\phi(m')$ holds, written $\vDash c : \psi \Rightarrow \phi $.  
% \end{defn}
