In this section, we formally present the definition of adaptivity for a given program.
As we discussed in Section~\ref{sec:adaptivity-informal}, we first define a dependency relation between program variables,
we then define a semantics-based dependency graph, and finally look at longest walks in this graph. 
\subsection{May-dependency between Variables}
\label{sec:dep}
%%%%%%%%%%%%%%%%%%%%%%%%%%%%%%%%%%% The Detail Explanation of Variable May-Dependency and Motivation on How to define it %%%%%%%%%%%%%%%%%%%%%%%%%%%%%%%%%%%%%%%%%%%%%%%%%
We are interested in defining a notion of dependencies between program variables since assigned variables are a good proxy to study dependencies between queries---we can recover query requests from variables associated with queries. We consider dependencies that can be generated by either data or control flow.
% as follows,
% \begin{enumerate}
For example, in the program 
\[c_1 =[\assign{x}{\query(\chi[2])}]^1 ;[\assign{y}{\query(\chi[3] + x)}]^2\]
the query $\query(\chi[3] + x)$  depends on the query $\query(\chi[2]))$ through a \emph{value dependency} via  $x^1$.
% ), because $\chi[3] + x$ may depend on the data stored in x assigned by the result of $\query(\chi[2]))$. 
% From our perspective, $\query(\chi[1])$ is different from $\query(\chi[2]))$.
\\
% \\
Conversely, in the program
\[c_2 = [\assign{x}{\query(\chi[1])}]^1 ; \eif( [x > 2]^2 , [\assign{y}{\query(\chi[2])}]^3, [\eskip]^4 )\] 
the query $\query(\chi[2])$ 
depends on the query $\query(\chi[1])$ via the \emph{control dependency} of the guard of the if command involving the labeled variable $x^1$.

To define dependency between program variables we will consider two events that are generated from the same command, hence they have the same variable name or boolean expression and label, but have either different value or different query expression, captured by the following definition. 

\begin{defn}
\label{def:diff}
Two events $\event_1, \event_2 \in \eventset$ differ in their value, or query value,
denoted as $\diff(\event_1, \event_2)$, if and only if:
{\small
\begin{subequations}
\begin{align}
& \pi_1(\event_1) = \pi_1(\event_2) 
  \land  
  \pi_2(\event_1) = \pi_2(\event_2) \\
& \land  
  \big(
   (\pi_3(\event_1) \neq \pi_3(\event_2)
  \land 
  \pi_{4}(\event_1) = \pi_{4}(\event_2) = \bullet )
  \lor 
  (\pi_4(\event_1) \neq \bullet
  \land 
  \pi_4(\event_2) \neq \bullet
  \land 
  \pi_{4}(\event_1) \neq_q \pi_{4}(\event_2)) 
  \big)
\end{align}
\label{eq:diff}
\end{subequations}
}
where $\qexpr_1 =_{q} \qexpr_2$ denotes the semantics equivalence between query values\footnote{The formal definition is in the supplementary material},
and $\pi_i$ projects the $i$-th element from the quadruple of an event.
\end{defn}
\jl{
$\pi_1(\event_1) = \pi_1(\event_2) 
  \land  
  \pi_2(\event_1) = \pi_2(\event_2)$ at Eq.\ref{eq:diff}(a)
requires that $\event_1$ and $\event_2$ have the same variable name and label. 
This guarantees that $\event_1$ and $\event_2$ are generated from the same labeled command.
In Eq.\ref{eq:diff}(b),
two kinds of comparisons between the third and fourth element are for the non-query assignment and query request separately.
For events generated from the non-query assignments (via checking
$\pi_{4}(\event_1) =_q \pi_{4}(\event_2) = \bullet$), we only compare their assigned values through $\pi_3(\event_1) \neq \pi_3(\event_2)$.
But for these from query requests (via checking
$\pi_{4}(\event_1) \neq \bullet \land \pi_{4}(\event_2) \neq \bullet$),
we are comparing their query expressions by $\pi_{4}(\event_1) \neq_q \pi_{4}(\event_2)$ rather than the assigned value computed from the unknown database server.
This matches the intuitive data dependency between queries, where one query is influenced by others as long as the query request is changed.
}

{Below is the \emph{event may-dependency} between events based on formally observing their differences via $\diff$.}
\begin{defn}[Event May-Dependency]
\label{def:event_dep}
An event $\event_2$ is in the \emph{event may-dependency} relation with an assignment event $\event_1 \in \eventset^{\asn}$ in a program ${c}$  with a hidden database $D$ and a witness trace $\trace \in \mathcal{T}$,
$\eventdep(\event_1, \event_2, [\event_1 ] \tracecat \trace \tracecat [\event_2], c, D)$ if and only if
\begin{subequations}
\begin{align}
&  
\exists \trace_0, \trace_1, \trace' \in \mathcal{T},\event_1' \in \eventset^{\asn}, {c}_1, {c}_2  \in \cdom  \st \diff(\event_1, \event_1') \land \\
& 
\quad (\exists  \event_2' \in \eventset \st 
\left(
\begin{array}{ll}   
  & \config{{c}, \trace_0} \rightarrow^{*} 
  \config{{c}_1, \trace_1 \tracecat [\event_1]}  \rightarrow^{*} 
  \config{{c}_2,  \trace_1 \tracecat [\event_1] \tracecat \trace \tracecat [\event_2] } 
   \\ 
   \bigwedge &
   \config{{c}_1, \trace_1 \tracecat [\event_1']}  \rightarrow^{*}
    \config{{c}_2,  \trace_1 \tracecat[ \event_1'] \tracecat \trace' \tracecat [\event_2'] } 
  \\
  \bigwedge & 
  \diff(\event_2,\event_2' ) \land 
  \vcounter(\trace, \pi_2(\event_2))
  = 
  \vcounter(\trace', \pi_2(\event_2'))\\
  \end{array}
  \right)\\ 
  & 
  \quad
  \lor 
  \left(
  \begin{array}{l} 
  \exists \trace_3, \trace_3'  \in \mathcal{T}, \event_b \in \eventset^{\test} \st 
  \\
   \quad \config{{c}, \trace_0} \rightarrow^{*} \config{{c}_1, \trace_1 \tracecat [\event_1]}  \rightarrow^{*}
   \config{c_2,  \trace_1 \tracecat [\event_1] \tracecat
   \trace \tracecat [\event_b] \tracecat  \trace_3} 
\\ \quad \land
\config{{c}_1, \trace_1 \tracecat [\event_1']}  \rightarrow^{*} 
\config{c_2,  \trace_1 \tracecat [\event_1'] \tracecat \trace' \tracecat [(\neg \event_b)] \tracecat \trace_3'} 
\\
\quad \land \tlabel({\trace_3}) \cap \tlabel({\trace_3'})
= \emptyset
\land \vcounter(\trace', \pi_2(\event_b)) = \vcounter(\trace, \pi_2(\event_b)) 
    \land \event_2 \in \trace_3
    \land \event_2 \not\in \trace_3'
  \end{array}
  \right)
  ),
\end{align}
\label{eq:eventdep}
\end{subequations}
where $\tlabel(\trace) \subseteq \ldom$ is the set of the labels in all the events from trace $\trace$ and $\event_2 \in \trace_3$ or $\event_2 \notin \trace_3$ denotes that $\event_2$ belongs to $\trace_3$ or not.
\end{defn}
The first line in Eq.~\ref{eq:eventdep}(a) requires that $\event_1$ comes from an assignment command and then modifies its assigned value via $\diff(\event_1, \event_1')$.

\jl{Then, the following two parts in Eq~\ref{eq:eventdep}(b) and (c) capture the intuitive value dependency and control dependency respectively. 
As in the literature on non-interference, and following~\cite{Cousot19a}, we formulate these dependencies as relational properties, i.e. in terms of two different traces of execution. 
We force these two traces to differ by using the event $\event_1$ in one and $\event_1'$ in the other. 
Both parts execute the program two times w.r.t. the different values in $\event_1$ (as line:1 in Eq~\ref{eq:eventdep}(b) and line:2 in Eq~\ref{eq:eventdep}(c))
and $\event_1'$ (as line:2 in Eq~\ref{eq:eventdep}(b) and line:3 in Eq~\ref{eq:eventdep}(c)), 
but observe the difference in the newly generated traces in different ways (via $3$rd line in Eq~\ref{eq:eventdep}(b) and $4$th line in Eq~\ref{eq:eventdep}(c)). This idea is similar to the dependency definition from \cite{Cousot19a}.
}

\jl{For the value dependency we check whether the change also create a change in the value of $\event_2$ or not.
In Eq~\ref{eq:eventdep}(b) line:2, if the newly generated trace, $\trace' ++ [\event_2']$ still contains $\event_2$ as $\event_2'$, we check the difference on their value in line:3.
We additionally check that the two events we consider appear the same number of times in the two traces - this to make sure that if the events are generated by assignments in a loop, we consider the same iterations. 
If they only differ in their assigned values, i.e., $\diff(\event_2, \event_2')$ and
they are in the same loop iteration (via $\vcounter(\trace, \pi_2(\event_2)) = \vcounter(\trace', \pi_2(\event_2'))$),
then we say there is a value \emph{may-dependency} relation between $\event_1$ and $\event_2$.}

\jl{The Eq~\ref{eq:eventdep}(c) captures the control dependency through observing the disappearance $\event_2$ from newly generated traces, $\trace' \tracecat [(\neg \event_b)] \tracecat \trace_3'$ in the second execution (line:3).
$\event_2 \in \trace_3 \land \event_2 \not\in \trace_3'$ in Eq~\ref{eq:eventdep}(c) line:4 specifies this disappearance.
$\vcounter(\trace', \pi_2(\event_b)) = \vcounter(\trace, \pi_2(\event_b))$ is used to make sure the two executions are
in the same loop iteration as well.
Different from Eq~\ref{eq:eventdep}(b) line:3,
we use a testing event, $\event_b$ here because
$\vcounter(\trace, \pi_2(\event_2)) = \vcounter(\trace', \pi_2(\event_2'))$ cannot guarantee the disappearance
when there is no influence through control dependency. Checking only $\event_2$'s occurrence causes false positive.
And the presence of a test event whose value is affected by the change in $\event_1$
can guarantee that the computation goes through a control flow guard.
This is correct because the control dependency can only be passed through the guard of if or while command,
and this guard must be evaluated into two different values ($\event_b$ and $\neg \event_b$) in the two executions.
}

\jl{
  We can now extend the dependency relation to variables by considering all the assignment events generated during the program’s execution. 
}
\begin{defn}[Variable May-Dependency]
  \label{def:var_dep}
A variable ${x}_2^{l_2} \in \lvar(c)$  \emph{may-dependend} on the 
  variable ${x}_1^{l_1} \in \lvar(c)$ in a program ${c}$,
  %
  $\vardep({x}_1^{l_1}, {x}_2^{l_2}, {c})$, iff
\begin{center}
$
{\small   \begin{array}{l}
\exists \event_1, \event_2 \in \eventset^{\asn}, \trace \in \mathcal{T} \st
\pi_{1}{(\event_1)}^{\pi_{2}{(\event_1)}} = {x}_1^{l_1}
\land
\pi_{1}{(\event_2)}^{\pi_{2}{(\event_2)}} = {x}_2^{l_2}% \\ \quad 
\land 
\eventdep(\event_1, \event_2, \trace, c) 
  \end{array}
}%
$
\end{center}
  \end{defn}
\jl{From the definition, a labeled variable $x_2^{l_2}$ may depend on another labeled variable $x_1^{l_1}$ in a program $c$ under the hidden database $D$, 
as long as there exist two assignment events $\event_1$ (for $x_1^{l_1}$) and $\event_2$ for $x_2^{l_2}$
satisfy the \emph{event may-dependency} relation under a witness trace $\trace$.  
Notice that in the definition above we can also have that the two variables are the same,
this allow us to capture self-dependencies
}

\subsection{Semantics-based Dependency Graph}
\label{sec:design_choice}
We can now define the \emph{semantics-based dependency graph} of a program $c$. We want this graph to combines quantitative reachability information with dependency information. 

\jl{
For a program $c$, there are some notations used in the following definition.
The labeled variables of $c$,
$\lvar(c) \subseteq \mathcal{LV}$ contains all the variables in $c$'s assignment commands, with the command labels as superscripts. 
The set of query-associated variables (in query request assignments),
$\qvar(c) \subseteq \lvar(c)$ contains all labeled variables in $c$'s query requests. 
The set of initial traces of $c$,
$\mathcal{T}_0(c) \subseteq \mathcal{T}$
contains all possible initial trace of $c$.
Each initial trace,  $\trace_0 \in \mathcal{T}_0(c)$ contains the initial values of all input variables of $c$. 
For instance, the initial trace of $\kw{twoRounds(k)}$ example contains the initial value of the input variable $k$.
}
\begin{defn}[Semantics-based Dependency Graph]
\label{def:trace_graph}
Given a program ${c}$,
its \emph{semantics-based dependency graph} 
$\traceG({c}) = (\traceV({c}), \traceE({c}), \traceW({c}), \traceF({c}))$ is defined as follows,
{\small
\[
\begin{array}{lll}
  \text{Vertices} &
  \traceV({c}) & := \left\{ 
  x^l
  ~ \middle\vert ~ x^l \in \lvar(c)
  \right\}
  \\
  \text{Directed Edges} &
  \traceE({c}) & := 
  \left\{ 
  (x^i, y^j) 
  ~ \middle\vert ~
  x^i, y^j \in \lvar(c) \land \vardep(x^i, y^j, c) 
  \right\}
  \\
  \text{Weights} &
  \traceW({c}) & := 
  \{ 
  (x^l, w) 
  ~ \vert ~ 
  w : \mathcal{T}_0(c) \to \mathbb{N}
  \land
  x^l \in \lvar(c) 
  \\ & &
  \quad \land
  \forall \trace_0 \in \mathcal{T}_0(c), \trace' \in \mathcal{T} \st \config{{c}, \trace_0} \to^{*} 
  \config{\eskip, \trace_0\tracecat\trace'} 
  \land w(\trace_0) = \vcounter(\trace', l) \}  
  \\
  \text{Query Annotations} &
  \traceF({c}) & := 
\left\{(x^l, n)  
~ \middle\vert ~
 x^l \in \lvar(c) \land
n = 1 \Leftrightarrow x^l \in \qvar(c) \land n = 0 \Leftrightarrow  x^l \notin \qvar(c)
\right\}
\end{array},
\]
}
%%%%%%%%%%%%%%%%%%%%%%%%%%%%%%%%%%% The Detailed Version in Explaining the Trace Operators %%%%%%%%%%%%%%%%%%%%%%%%%%%%%%%%%%%%%%%%%%%%%%%%%
% There are some operators: the trace concatenation operator $\tracecat: \mathcal{T} \to \mathcal{T} \to \mathcal{T}$, combines two traces; the counting operator $\vcounter : \mathcal{T} \to \mathbb{N} \to \mathbb{N}$, 
% which counts the occurrence of of a labeled variable in the trace. The full definitions of these above operators can be found in the appendix.
% \\
A semantics-based dependency graph $\traceG({c})= (\traceV({c}), \traceE({c}), \traceW({c}), \traceF({c}))$ 
is \emph{well-formed} if and only if $ \{x^l \ |\ (x^l,w)\in \traceW({c})\} = \traceV({c}) $.
\end{defn}


%%%%%%%%%%%%%%%%%%%%%%%%%%%%%%%%%%% The Explanation of Dependency Graph %%%%%%%%%%%%%%%%%%%%%%%%%%%%%%%%%%%%%%%%%%%%%%%%%
\jl{There are four components in this graph.
\begin{enumerate}
    \item The vertices $\traceV({c})$ of a program $c$ are all its labeled variables, $\lvar(c)$ which are statically collected.
    \item $\traceF(c)$ contains the \emph{query annotation} for 
    every vertex $x^l \in \traceV(c)$. It indicates whether $x^l$ comes from a query request (1) or not (0) by checking if the labeled variable $x^l$ of the vertex is in $\qvar(c)$.
    \item Edges in $\traceE(c)$ are built from the  \emph{variable may-dependency} relation, i.e. $\vardep(x^i, y^j, c)$ in Definition~\ref{def:var_dep} between two labeled variables.
    This is the key definition in order to formalize the intuitive \emph{adaptivity}, and also ingredients the dependency information of the program.
    \item 
  The weight function in $\traceW(c)$ for each vertex, $w : \mathcal{T} \to \mathbb{N}$
maps from a starting trace $\trace_0 \in \mathcal{T}_0(c)$ to a natural number.
A weight function $w \in \traceW(c)$ is a function that for every starting trace $\trace_0 \in \mathcal{T}_0(c)$ 
gives the number of times the assignment of the corresponding vertex $x^l$ is visited. Notice that weight functions are total and with range $\mathbb{N}$. This means that if a program $c$ has some non-terminating behavior, the set $\traceW(c)$ will be empty.
To rule out this situation, we consider as well-formed only graphs which have a weight for every vertex. 
For each vertex $x^l$, it tracks its visiting times (i.e., the evaluation times of the command with the label $l$) when the program $c$ is evaluated from the initial trace $\trace_0$ into $\eskip$, $\config{{c}, \trace_0} \to^{*} \config{\eskip, \trace_0\tracecat\trace'} $.
The visiting times is computed by the counter operator $\vcounter(\trace', l)$
by counting the occurrence of the label $l$ in $\trace'$.
As an instance, in the semantics-based dependency graph of $\kw{twoRounds}$ in Figure~\ref{fig:overview-example}(b), the weight, $w_k$ of the vertex $x^3$ is a function of type $\mathcal{T}_0(\kw{twoRound(k)}) \to \mathbb{N}$.
Given input $\trace_0$, we execute the program under $\trace_0$ as $\config{\kw{twoRound(k)}, \trace_0} \to^{*} \config{\eskip, \trace_0\tracecat\trace'} $. Then $w_k(\trace_0)$ outputs the occurrence time of the label $3$ in $\trace'$.
\end{enumerate}
The main novelty of  the semantics-based dependency graph is the combination of the quantitative and dependency information. 
It can tell both the dependency between queries via the directed edge, and the times they depend on each other via the weight.
}

%%%%%%%%%%%%%%%%%%%%%%%%%%%%%%%%%%%%%%%Trace-Based Adaptivity%%%%%%%%%%%%%%%%%%%%%%%%%%%%%%%
\subsection{Trace-based Adaptivity}
We can now define the adaptivity of a program formally. This notion is formulated in terms of an initial trace, specifying the value of the input variables, as the walk on the graph $\traceG({c})$, which has the largest number of query requests.


% Given 
% the program $c$'s
% $\traceG({c})$,
% we define adaptivity 
% with respect to an initial trace $\trace_0 \in \mathcal{T}_0(c)$ via a finite walk in $\traceG({c})$, which has the most query requests.
% We show the definition of a finite walk as follows.

\begin{defn}[Walk on $\traceG({c})$]
\label{def:finitewalk}
Given the semantics-based dependency graph $\traceG({c}) = (\traceV, \traceE, \traceW, \traceF)$ of a program $c$, a \emph{walk} $k:\mathcal{T}_0(c)\to \mathbb{N}$ on $\traceG({c})$ is a function that given as input an initial trace $\trace_0$ returns a sequence of edges $(e_1 \ldots e_{n - 1})$ 
for which there is a sequence of vertices $(v_1, \ldots, v_{n})$ such that:
\begin{itemize}
\item $e_i = (v_{i},v_{i + 1}) \in \traceE$ for every $1 \leq i < n$.
\item every $v_i \in \traceV$ and $(v_i, w_i) \in \traceW$, $v_i$ appears in $(v_1, \ldots, v_{n})$ at most $w(\trace_0)$ times.  
\end{itemize}
{$(v_1, \ldots, v_{n})$ is the vertex sequence of $k(\trace_0)$ and the length of $k(\trace_0)$ is the number of vertices in its vertex sequence, i.e., $|k(\trace_0)| = n$.}
% The length of $k(\trace_0)$ is the number of vertices in its vertices sequence, i.e., $\len(k)(\trace_0) = n$.
We denote by $\walks(\traceG(c))$
the set of all the  walks $k$ in $\traceG(c)$.
\end{defn} 
Because for the adaptivity
% is intuitively 
we are interested in the dependency between queries,
we calculate a special ``length'' of a walk, the \emph{query length},  by counting only the vertices
corresponding to queries.
\begin{defn}[Query Length]
\label{def:qlen}
Given 
the semantics-based dependency graph $\traceG({c})$ of a program $c$,
 and a \emph{walk} 
 $k \in \walks(\traceG(c))$, the \emph{query length} of $k$ is a function $\qlen(k):\mathcal{T}_0(c) \to \mathbb{N}$ that given an initial trace $\trace_0$ returns
the number of vertices which correspond to query variables in the vertices sequence, $(v_1, \ldots, v_{n})$ as follows, 
\begin{center}
   $
  \qlen(k)(\trace_0) = |\big( v \mid v \in (v_1, \ldots, v_{n}) \land \qflag(v) = 1 \big)|.
$
\end{center}
% \mg{There is a problem here. The vertex sequence $(v_1, \ldots, v_{n})$ is not defined. The walk when provided with the initial trace gives you an edge sequence determining the vertex sequence but we need to do some work to extract it. In other words, this need to be rephrased, perhaps adding a definition of what is the vertex sequence associated with a walk applied to an initial state.}
\end{defn}
% Then the definition of adaptivity is presented in Def.~\ref{def:trace_adapt} below.
\begin{defn}
    [Adaptivity of a Program]
    \label{def:trace_adapt}
    Given a program ${c}$, 
    its adaptivity $A(c)$ is function 
    $A(c) : \mathcal{T}_0(c)\to \mathbb{N}$ such that for an
    initial trace $\trace_0 \in \mathcal{T}_0(c)$, 
\begin{center}
$
    A(c)(\trace_0) = \max \big 
    \{ \qlen(k)(\trace_0) \mid k \in \walks(\traceG(c)) \big \} 
$
\end{center}
\end{defn}

\subsection{The definition limitation}
\begin{example}[Over-Defined Adaptivtiy Example]
    \label{ex:multiRoundsS}
    The program's adaptivity definition in our formal model,
    (in Definition~\ref{def:trace_adapt})
     comes across an over-approximation on capturing the program's intuitive adaptivity rounds.
    It is resulted from the difference between its weight calculation and the \emph{variable may-dependency} definition.
    It occurs when the weight is computed over the traces different from the traces used in 
    witnessing the \emph{variable may-dependency} relation.
    
%%%%%%%%%%%%%%%%%%%%%%%%%%%%%%%%%%%%%% Previous Version For Reference %%%%%%%%%%%%%%%%%%%%%%%%%
    % The program $\kw{multiRoundsS(k)}$ in Figure~\ref{fig:multiRoundsS}(a) demonstrates this over-approximation.
    % It is a variant of the multiple rounds strategy 
    % % we call it a multiple rounds single iteration algorithm, 
    % % named $\kw{multiRoundsS(k)}$ 
    % with input $k$.
    % % as the input variable.
    % In each iteration
    % the query $\query(\chi[y] + p)$ in command $7$ is based on previous query results stored in $p$ and $y$
    % %  is asked by the analyst like in the multiple rounds strategy. 
    % Differ from Example~\ref{ex:multipleRounds},
    % only the query answer from the one single iteration (the $(k - 2)^{th} $ iteration) is 
    % % used to $b$. 
    % used in command $7$.
    % Because the execution trace updates 
    % $p$ using the constant $0$ for all the iterations where ($j \neq k - 2$) at line $10$ after the 
    % query request at line $7$.
    % In this way, all the query answers stored in $p$ will not be accessed in next query request at line $7$ in the iterations 
    % where  ($j \neq k - 2$).
    % Only query answer at one single iteration where ($j = k - 2 $) will be used in next query request
    % $\query(\chi[y] + p)$ at line $7$.
    % So the adaptivity for this example is $2$. 
    % % so the answers from odd iterations do not affect the queries at even iterations. 
    % % However, from the semantics-based dependency graph in Figure~\ref{fig:overappr_example}(b), 
    % However, our adaptivity model fails to realize that there is only dependency relation 
    % between $p^7$ and $p^7$ in one single iteration, 
    % but not in all the others. 
    % % there is no edge from queries at odd iterations (such as $q_1,q_3,q_5$) to queries at even iteration(such as $q_2,q_4$). The longest path is dashed with a length $3$.  
    % As shown in the semantics-based dependency graph in Figure~\ref{fig:multiRoundsS}(b), 
    % there is an edge from $p^7$ to itself representing the existence of \emph{Variable May-Dependency} from $p^7$ on itself,
    % and the visiting times of labeled variable $p^7$ is 
    % $w_{}(\trace_0)$ given the input initial trace $\trace_0$. 
    % % will always execute then branch and even iteration means else branch, so 
    % % % its dependency 
    % % it considers both branches for every iteration. 
    % % In this sense, the weight estimated for $y^6$ and $w^6$ are both 
    % % $k$.
    % As a result, the walk with the longest query length 
    % is
    % $p^7  \to \cdots \to p^7 \to y^4  \to z^1 $ with the vertex $p^7$ visited $w_{}(\trace_0)$,
    % as the dotted arrows. 
    % The adaptivity 
    % % the Program-Based Dependency graph from {\THESYSTEM} by finding 
    % based on
    % this walk
    % % walk with the longest query length 
    % is $2 + w_{}(\trace_0)$, instead of $2$. 
    % % %
    % % T% estimated from the Program-Based Dependency graph from by finding the walk with the longest query length 
    % % is $1 + 2 * k$, instead of $1 + K$.
    % Though the $\THESYSTEM$ is able to give us $2 + k$,  as an accurate bound w.r.t this definition.
%%%%%%%%%%%%%%%%%%%%%%%%%%%%%%%%%%%%%% Previous Version Above %%%%%%%%%%%%%%%%%%%%%%%%%
\jl{
    The program $\kw{multiRoundsS(k)}$ in Figure~\ref{fig:multiRoundsS}(a) demonstrates this over-approximation.
    It is a variant of the multiple rounds strategy with input $k$.
    In each iteration the query $\query(\chi[y] + p)$ in command $7$ is based on previous query results stored in $p$ and $y$
    Differ from Example~\ref{ex:multipleRounds},
    only the query answer from the one iteration, the $(k - 2)^{th}$ one is used in query request
    $\clabel{\assign{p}{\query(\chi[y]+p)} }^{7}$.
    Because the execution will reset
    $p$'s value by the constant $0$ in all the other iterations
    at line $10$ after this query request.
    In this way, all the query answers stored in $p$ is erased and isn't used
    in the query request at next iteration, except the one at the $(k - 2)$th iteration.
    In this sense, the intuitive \emph{adaptivity} rounds for this example is $2$. 
    However, our adaptivity definition fails to realize that there is only dependency relation 
    between $p^7$ and $p^7$ in one single iteration, 
    but not in all the others. 
    As shown in the semantics-based dependency graph in Figure~\ref{fig:multiRoundsS}(b), 
    there is an edge from $p^7$ to itself representing the existence of \emph{Variable May-Dependency} from $p^7$ on itself,
    and the visiting times of labeled variable $p^7$ is 
    $w_{}(\trace_0)$. $w_{}(\trace_0)$ will count the execution times of command $\clabel{\assign{p}{\query(\chi[y]+p)} }^{7}$ during execution, which is expected to be equal to the loop iteration numbers, i.e., initial value of input $k$ from the initial trace $\trace_0$.
    As a result, the walk with the longest query length 
    is
    $p^7  \to \cdots \to p^7 \to y^4  \to z^1 $ with the vertex $p^7$ visited $w_{p^7}(\trace_0)$, as the dotted arrows. 
    The adaptivity based on this walk
    is $2 + w_{p^7}(\trace_0)$, instead of $2$. 
    Though the $\THESYSTEM$ is able to give us $2 + k$,  as an accurate bound w.r.t this definition.
x}
        {\small
        \begin{figure}
            \centering
           %}
           \quad
           \begin{subfigure}{.35\textwidth}
           \begin{centering}
       {\footnotesize
           $ \begin{array}{l}
                   \kw{multiRoundsS(k)}\\
                      \clabel{ \assign{j}{0}}^{0} ; 
                       \clabel{\assign{z}{\query(0)} }^{1} ;             
                       \clabel{\assign{p}{0} }^{2} ; \\
                       \eif(\clabel{ k = 0}^{3}, 
                       \clabel{ \assign{y}{\query(z)}}^{4}, \clabel{\eskip}^5);\\
                       \ewhile ~ \clabel{j \neq k}^{6} ~ \edo ~ \\
                       \Big(
                        \clabel{\assign{p}{\query(\chi[y]+p)} }^{7}  ; 
                        \clabel{\assign{j}{j + 1}}^{8}\\
                     \eif(\clabel{ j \neq k - 2}^{9}, 
                     \clabel{ \assign{p}{0}}^{10} ,\clabel{\eskip}^{10})
                \Big);\\
                   \end{array}
           $       
       }
           \caption{}
           \end{centering}
           \end{subfigure}
           \begin{subfigure}{.6\textwidth}
               \begin{centering}
               \begin{tikzpicture}[scale=\textwidth/20cm,samples=150]
           % Variables Initialization
           \draw[] (-5, 2) circle (0pt) node{{ $z^1: {}^{w_{z^1}}_{1}$}};
           \draw[] (-5, 7) circle (0pt) node{{$p^2: {}^{w_{p^2}}_{0}$}};
           \draw[] (-5, 4) circle (0pt) node{{ $y^4: {}^{w_{y^4}}_{1}$}};
           % Variables Inside the Loop
            \draw[] (0, 6) circle (0pt) node{{ $p^7: {}^{w_{p^7}}_{1}$}};
            \draw[] (0, 2) circle (0pt) node{{ $p^{10}: {}^{w_{p^{10}}}_{0}$}};
            % Counter Variables
            \draw[] (5, 6) circle (0pt) node {{$j^0: {}^{w_{j^0}}_{0}$}};
            \draw[] (5, 2) circle (0pt) node {{ $j^8: {}^{w_{j^8}}_{0}$}};
            %
            % Value Dependency Edges:
            \draw[ ultra thick, -Straight Barb, densely dotted,] (0.8, 7) arc (220:-100:1);
            \draw[  -latex] (-1.5, 6)  to  [out=-130,in=130]  (-1.5, 2);
            % Value Dependency Edges on Initial Values:
            \draw[ ultra thick, -latex, densely dotted,] (-5, 3.5)  -- (-5, 2.5) ;
            \draw[  -latex,] (-1.5, 6)  -- (-4, 7) ;
            \draw[  ultra thick, -latex, densely dotted,] (-1.5, 6)  -- (-4, 4.7) ;
            %
            % Value Dependency For Control Variables:
            \draw[  -Straight Barb] (6.5, 2.5) arc (150:-150:1);
            % Control Dependency
            \draw[  -latex] (5, 2.5)  -- (5, 5.5) ;
            \draw[ -latex] (1.5, 6)  -- (3.5, 6) ;
            \draw[ -latex] (1.5, 6)  -- (3.5, 2) ;
            \draw[ -latex] (1.5, 1.8)  -- (3.5, 2) ; 
            % Edges Produced by Transitivity
            \draw[  -latex,] (-1.5, 6)  -- (-4, 2) ;
            \draw[ -latex] (1.5, 1.8)  -- (3.5, 6) ; 
           \end{tikzpicture}
            \caption{}
               \end{centering}
               \end{subfigure}
            \caption{(a) The multi rounds single example
            (b) The semantics-based dependency graph.}
           \label{fig:multiRoundsS}
           \end{figure}
        }
    \end{example}




\subsection{\highlight{New Dependency Relation}}
\highlight{
\begin{defn}[Value Sequence $\seq(\trace, x^l)$]
  \label{def:vseq}
  \[
\begin{array}{l}
  \seq(\trace :: (x, l, v, \bullet), x^l) \triangleq \seq(\trace)::v  \qquad
  \seq(\trace :: (x, l, v, \qval), x^l) \triangleq \seq(\trace):: \qval \qquad
  \seq([]) \triangleq []\\
  \seq(\trace :: (y, j, \_, \_), x^l) \triangleq \seq(\trace) \quad y \neq x \lor j \neq l 
\end{array}
\]
\end{defn}
%
\begin{defn}[Difference Sequence $\sdiff(\trace_1, \trace_2, x^l )$]
  \label{def:diffseq}
  Let $ s_1 = \seq(\trace_1, x^l) \land s_2 = \seq(\trace_2, x^l)$ be the value sequence of $x^l$ 
  on $\trace_1$ and $\trace_2$, and $s^l$ be the sequence with longer length and $s^t$ the 
  shorter one,
  then their difference sequence is defined as follows,
  \[
    \sdiff(\trace_1, \trace_2, x^l) \triangleq
    \begin{array}{l}
      \{ (s^t[k], s^l[k]) ~|~ 
      % \land 
      % \land 
      s^t[k] \neq s^l[k], k = 0, \ldots, len(s^t)
      \}
      \\
      \cup 
      \{ (\cdot, s^l[k]) ~|~ 
      \len(s^t) \leq \len(s^l)k = len(s^t), \ldots \len(s^l)
      \}
    \end{array}
    \]
\end{defn}
%
\begin{defn}[Variable May-Dependency]
  \label{def:var_dep}
  %
  A labeled variable $y^j \in \lvar(c)$ is in the \emph{may-dependency} relation with another
  labeled variable $x^i \in \lvar(c)$ in a program ${c}$, w.r.t. an initial trace $\trace_0 \in \mathcal{T}_0(c)$
  and two witness traces $\trace_1, \trace_2 \in \mathcal{T}$,
  denoted as 
  %
  $\dep(x^i, y^j, \trace_1, \trace_2, \trace_0, {c})$, if an only if
  \[
    \begin{array}{l}
  \exists 
  D \in \dbdom, 
  \trace_0' \in \mathcal{T} \st
  (\forall z \neq x \st   \env(\trace_0 ) z =   \env(\trace_0') z )
  \\ \quad \land 
  % \\ \quad \land 
   \config{{c}, \vtrace_0} \rightarrow^{*} 
  % \config{{c}_1, \vtrace_0' \tracecat [\event_1]}  \rightarrow^{*} 
    \config{\clabel{\eskip}^l, \vtrace_0  \tracecat \trace_1 } 
    \land 
    \config{{c}, \vtrace_0'} \rightarrow^{*} 
      \config{\clabel{\eskip}^l, \vtrace_0'  \tracecat \trace_2} 
    \land 
      \sdiff(\trace_1, \trace_2, y^j ) \neq \emptyset
    \end{array}
  \]  
  \end{defn}
}%

\subsection{\highlight{New Semantic-based Dependency Graph}}

\highlight{
  \begin{defn}[Semantics-based Dependency Graph]
  \label{def:trace_graph}
  Given a program ${c}$,
  its \emph{semantics-based dependency graph} 
  $\traceG({c}) = (\traceV({c}), \traceE({c}))$ is defined as follows,
\small
\[
\begin{array}{lcl}
  % \text{Vertices} &
  \traceV({c}) & := & 
  \{ 
  (x^l, w) 
  % \in \mathcal{LV} \times \mathbb{N}
  ~ \vert ~ 
  w : \mathcal{T} \to \mathbb{N}
  \land
  x^l \in \lvar(c) 
  \\ & &
  \land
  \forall \trace \in \mathcal{T}_0(c), 
  \trace' \in \mathcal{T} \st \config{{c}, \trace} \to^{*} \config{\eskip, \trace\tracecat\vtrace'} 
  \land w(\trace) = \vcounter(\vtrace', l) 
\}
  \\
  % \text{Edges} &
  \traceE({c}) & := & 
  \{ 
  (x^i, w, y^j) 
%   \in \mathcal{LV} \times \mathcal{LV}
  ~ \vert ~
  x^i, y^j \in \lvar(c)
  \land w \in \mathcal{P}( \mathcal{T}_0(c) \to \mathbb{N})
  \land 
  \exists \trace \in \mathcal{T}_0(c), 
  \trace_1, \trace_2 \in \mathcal{T} \st \dep(x^i, y^j,\trace_1, \trace_2, \trace_0, c)
  \\ & &
  \land \forall \trace_0 \in \mathcal{T}_0(c) \st
  w (\trace_0) = \max \left\{ | \sdiff(\trace_1, \trace_2, y)|
  ~\middle\vert~
  \forall \trace_1, \trace_2 \in \mathcal{T} \st \dep(x^i, y^j,\trace_1, \trace_2, \trace_0, c) \right\}
  \}
\end{array}
\]
\end{defn}
}
There are two components of the semantics-based dependency graph. 
\\
\highlight{
The vertices $\traceV(c)$  is a set of pairs, $(x^l, w) \in \mathcal{LV} \times (\mathcal{T} \to \mathbb{N})$,
with a labeled variable as first component and
its weight $w$ the second component.
The weight function in $\traceW(c)$ for each vertex, $w : \mathcal{T} \to \mathbb{N}$
maps from a starting trace $\trace_0 \in \mathcal{T}_0(c)$ to a natural number.
A weight function $w \in \traceW(c)$ is a function that for every starting trace $\trace_0 \in \mathcal{T}_0(c)$ 
gives the number of times the assignment of the corresponding vertex $x^l$ is visited. Notice that weight functions are total and with range $\mathbb{N}$. This means that if a program $c$ has some non-terminating behavior, the set $\traceW(c)$ will be empty.
To rule out this situation, we consider as well-formed only graphs which have a weight for every vertex. 
For each vertex $x^l$, it tracks its visiting times (i.e., the evaluation times of the command with the label $l$) when the program $c$ is evaluated from the initial trace $\trace_0$ into $\eskip$, $\config{{c}, \trace_0} \to^{*} \config{\eskip, \trace_0\tracecat\trace'} $.
The visiting times is computed by the counter operator $\vcounter(\trace', l)$
by counting the occurrence of the label $l$ in $\trace'$.
As an instance, in the semantics-based dependency graph of $\kw{twoRounds}$ in Figure~\ref{fig:tworounds}(b), the weight, $w_k$ of the vertex $x^3$ is a function of type $\mathcal{T}_0(\kw{twoRound(k)}) \to \mathbb{N}$.
Given input $\trace_0$, we execute the program under $\trace_0$ as $\config{\kw{twoRound(k)}, \trace_0} \to^{*} \config{\eskip, \trace_0\tracecat\trace'} $. Then $w_k(\trace_0)$ outputs the occurrence time of the label $3$ in $\trace'$.
\\
The directed edges $\traceE({c})$ is a set of triples $ (x^i, w, y^j) \in \mathcal{LV} \times \mathcal{P}(\mathcal{T}_0(c) \to \mathbb{N}) \times \mathcal{LV}$,
 with two labeled variables (from $x^i$ pointing to $y^j$) and a weight $w$ for this edge.
The edges are constructed directly from our variable may-dependency relation. 
For any two vertices $x^{i}$ and $y^{j}$ in $\traceV(c)$, if there exists two witness traces $\trace_1, \trace_2$ and an initial trace $\trace_0 \in \mathcal{T}_0$ such that,
they satisfy the variable may-dependency relation 
$\dep(x^i, y^j, \trace_1, \trace_2, \trace_0, c)$ , 
there is a direct edge. 
The weight of the edge is a function $w: \mathcal{T}_0(c) \to \mathbb{N}$,
where given an initial trace $\trace_0$,
it is the maximum length of the difference sequence between all pairs of the witness traces $\trace_1, \trace_2$ 
satisfying the dependency relation.
}
The main novelty of  the semantics-based dependency graph is the combination of the quantitative and dependency information. 
It can tell both the dependency between queries via the directed edge, and the times they depend on each other via the weights of edge and vertex.
% it is also reflected in $\traceW({c})$.    
\subsection{\highlight{New Trace-based Adaptivity}}
Given 
a program $c$'s semantics-based dependency graph 
$\traceG({c})$,
we define adaptivity 
with respect to an initial trace $\trace_0 \in \mathcal{T}_0(c)$ by the finite walk in the graph, which has the most query requests along the walk.
We show the definition of a finite walk as follows.
%
  \begin{defn}[Finite Walk (k)].
  \label{def:finitewalk}
  \\
  Given the semantics-based dependency graph $\traceG({c}) = (\traceV({c}), \traceE({c}), \traceW({c}), \traceF({c}))$ of a program $c$,
  a \emph{finite walk} $k$ in $\traceG({c})$ is a 
  function $k: \mathcal{T} \to $ sequence of edges.
  For a initial trace $\trace_0 \in \mathcal{T}_0(c)$, 
  $k(\trace_0)$ is a sequence of edges $(e_1 \ldots e_{n - 1})$ 
  for which there is a sequence of vertices 
  $(v_1, \ldots, v_{n})$ such that:
  \begin{itemize}
      \item \highlight{
        $e_i = (v_{i}, w_i, v_{i + 1}) \in \traceE(c)$ for every $1 \leq i < n$, 
        and $e_i$ appears in $(e_1 \ldots e_{n - 1})$  at most $w_i(\trace_0)$
        times.
      }
      \item every $(v_i, w_i) \in \traceV(c)$
      % and $(v_i, w_i) \in \traceW(c)$, 
       and $v_i$ appears in $(v_1, \ldots, v_{n})$ at most 
    $w_i(\trace_0)$
      times.  
  \end{itemize}
  %
  The length of $k(\trace_0)$ is the number of vertices in its vertices sequence, i.e., $\len(k)(\trace_0) = n$.
 \end{defn}

We use $\walks(\traceG(c))$ to denote 
% \mg{``the set'', not ``a set''}a set containing all finite walks $k$ in $G$;
the set containing all finite walks $k$ in $\traceG(c)$;
and $k_{v_1 \to v_2} \in \walks(\traceG(c))$ with $v_1, v_2 \in \traceV(c)$ denotes the walk from vertex $v_1$ to $v_2$ . 
\\
We are interested in queries, so we need to recover the 
variables corresponding to queries from the walk. We define the query length of a walk, 
instead of counting all 
the vertices in $k$'s vertices sequence, we just count the number of vertices which correspond to query variables in this sequence.
%
\begin{defn}[Query Length of the Finite Walk($\qlen$)].
\label{def:qlen}
\\
Given 
the semantics-based dependency graph 
$\traceG({c}) = (\traceV({c}), \traceE({c}))$ of a program $c$,
 and a \emph{finite walk} 
%  $k$ in $\traceG(c)(\trace)$
 $k \in \walks(\traceG(c))$. 
%  with its vertex sequence $(v_1, \ldots, v_{n})$, 
%  the length of $k$ w.r.t query is defined as:
The query length of $k$ is a function $\qlen(k): \mathcal{T} \to \mathbb{N}$, such that with an initial trace  $\trace_0 \in \mathcal{T}_0(c)$, $\qlen(k)(\trace_0)$ is
the number of vertices which correspond to query variables in the vertices sequence of the walk $k(\trace_0)$
$(v_1, \ldots, v_{n})$ as follows, 
\[
  \qlen(k)(\trace_0) = |\big( v \mid v \in (v_1, \ldots, v_{n}) \land v \in \qvar(c) \big)|.
\]
\end{defn}

\subsection{{Example From Limitation}}
\begin{example}[Accurate Adapativity for Multiple Rounds Single Example]
    \label{ex:multipleRoundSingle}
    The program's adaptivity in our formal model,
    % which we define over the program's execution-based dependency graph from the dynamic 
    % analysis 
    in Definition~\ref{def:trace_adapt} also
     comes across an over-approximation on the program's
     intuitive adaptivity rounds.
    It is resulted from difference between its weight calculation and the \emph{variable may-dependency} definition.
    It occurs when the weight is computed over the traces different from the traces used in 
    witness the \emph{variable may-dependency} relation.
    % control flow can be decided in a particular way in front of conditional branches, while the static analysis fails to witness. 
    
    % We use one example to show the over-approximated definition, 
    As the program in Figure~\ref{fig:overdefn_example}(a),
    % This example is the variant of the multiple rounds strategy, 
    % we call it a multiple rounds odd iteration algorithm.
    % This example is still 
    which is a variant of the multiple rounds strategy, 
    % we call it a multiple rounds single iteration algorithm, 
    named $\kw{multipleRoundSingle(k)}$ with input $k$.
    % as the input variable.
    In this algorithm, 
    at line 7 of every iteration, 
    a query $\query(\chi[y] + p)$ based on previous query results stored in $p$ and $y$ is asked by the analyst like in the multiple rounds strategy. 
    The difference is that only the query answers from the one single iterations ($j = k - 2 $) are 
    % used to $b$. 
    used in this query $\query(\chi[y] + p)$.
    Because the execution trace updates 
    %   $b$ using the query answers at odd iterations, so the answers from even iterations do not affect the queries at odd iterations. From the query-based dependency graph in Figure~\ref{fig:overappr_example}(b), we can see that there is no edge from queries at odd iterations (such as $q_1,q_3,q_5$) to queries at even iteration(such as $q_2,q_4$). The longest path is dashed with a length $3$.  However, {\THESYSTEM} fails to realize that odd iteration will always execute then branch and even iteration means else branch, so its dependency graph considers both branches for every iteration. In this sense, the dependency graph by {\THESYSTEM} is similar to the one in the multiple rounds strategy. We show the estimated graph in Figure~\ref{fig:overappr_example}(c). The estimated upper bound is then, $5$, instead of $3$. 
    $p$ using the constant $0$ for all the iterations where ($j \neq k - 2$) at line $10$ after the 
    query request at line $7$.
    In this way, all the query answers stored in $p$ will not be accessed in next query request at line $7$ in the iterations 
    where  ($j \neq k - 2$).
    Only query answer at one single iteration where ($j = k - 2 $) will be used in next query request
    $\query(\chi[y] + p)$ at line $7$.
    So the adaptivity for this example is $2$. 
    % so the answers from odd iterations do not affect the queries at even iterations. 
    % However, from the execution-based dependency graph in Figure~\ref{fig:overappr_example}(b), 
    However, our adaptivity model fails to realize that there is only dependency relation 
    between $p^7$ and $p^7$ in one single iteration, 
    not the others. 
    % there is no edge from queries at odd iterations (such as $q_1,q_3,q_5$) to queries at even iteration(such as $q_2,q_4$). The longest path is dashed with a length $3$.  
    As shown in the execution-based dependency graph in Figure~\ref{fig:overdefn_example}(b), 
    there is an edge from $p^7$ to itself representing the existence of \emph{Variable May-Dependency} from $p^7$ on itself,
    and the visiting times of labeled variable $p^7$ is 
    $w_k(\trace_0)$ with a initial trace $\trace_0$. 
    % will always execute then branch and even iteration means else branch, so 
    % % its dependency 
    % it considers both branches for every iteration. 
    % In this sense, the weight estimated for $y^6$ and $w^6$ are both 
    % $k$.
    As a result, the walk with the longest query length 
    is
    $p^7  \to \cdots \to p^7 \to y^4  \to z^1 $ with the vertex $p^7$ visited $w_k(\trace_0)$,
    as the dotted arrows. 
    The adaptivity 
    % the Program-Based Dependency graph from {\THESYSTEM} by finding 
    based on
    this walk
    % walk with the longest query length 
    is $2 + w(\trace_0)$, instead of $2$. 
    % %
    % T% estimated from the Program-Based Dependency graph from by finding the walk with the longest query length 
    % is $1 + 2 * k$, instead of $1 + K$.
    Though the $\THESYSTEM$ is able to give us $2 + k$,  as an accurate bound w.r.t this definition.
    %  we show the estimated graph in Figure~\ref{fig:overappr_example}(c). 
    
        {\small
    \begin{figure}
     \centering
    %}
    \quad
    \begin{subfigure}{.35\textwidth}
    \begin{centering}
    $
        \begin{array}{l}
            \kw{multipleRoundsSingle(k)}\\
               \clabel{ \assign{j}{k}}^{0} ; 
                \clabel{\assign{z}{\query(0)} }^{1} ;   \\          
                % \clabel{\assign{p}{0} }^{2} ; \\
                % \eif(\clabel{ k = 0}^{3}, 
                % \clabel{ \assign{y}{\query(z)}}^{4}, \clabel{\eskip}^5);\\
                \ewhile ~ \clabel{j > 0}^{2} ~ \edo ~ \\
                \Big(
                 \clabel{\assign{y}{\query(\chi[z]+y)} }^{3}  ; \\
              \eif(\clabel{ j \neq 2}^{4}, 
              \clabel{ \assign{y}{0}}^{5} ,\clabel{\eskip}^{6})\\
              \clabel{\assign{j}{j - 1}}^{7}
         \Big);\\
            \end{array}
    $
    \caption{}
    \end{centering}
    \end{subfigure}
    \begin{subfigure}{.6\textwidth}
        \begin{centering}
        \begin{tikzpicture}[scale=\textwidth/15cm,samples=150]
    % Variables Initialization
    % \draw[] (-5, 1) circle (0pt) node{{ $z^1: {}^{w_1}_{1}$}};
    % \draw[] (-5, 7) circle (0pt) node{{$p^2: {}^{w_1}_{0}$}};
    \draw[] (-5, 4) circle (0pt) node{{ $z^1: {}^{w_1}_{1}$}};
    % Variables Inside the Loop
     \draw[] (0, 6) circle (0pt) node{{ $y^3: {}^{w_k}_{1}$}};
     \draw[] (0, 2) circle (0pt) node{{ $y^{5}: {}^{w_k}_{0}$}};
     % Counter Variables
     \draw[] (5, 6) circle (0pt) node {{$j^0: {}^{w_1}_{0}$}};
     \draw[] (5, 2) circle (0pt) node {{ $j^8: {}^{w_k}_{0}$}};
     %
     % Value Dependency Edges:
    %  \draw[ thick, -Straight Barb] (1.4, 1.6) arc (120:-200:1);
     \draw[ ultra thick, -Straight Barb, densely dotted,
     postaction={decorate,
     decoration={
            text along path,
            % text={$\tau$},
            text={$\trace_0 \to 
        %     \begin{array}{ll}
        %        \env(\trace_0, k) & \env(\trace_0, k) \leq 1 \\
        %    2 & \env(\trace_0, k) \geq 2
        %     \end{array}
            $},
            text align={center},
            raise=0.2cm}}
     ] 
     (0.8, 7) arc (220:-100:1);
     \draw[ thick, -latex] (-1.5, 6)  to  [out=-130,in=130]  (-1.5, 2);
     % Value Dependency Edges on Initial Values:
     \draw[  ultra thick, -latex, densely dotted,] (-1.5, 6)  -- 
     node [above] {$\trace_0 \to \env(\trace_0, k) $} (-4, 4.7) ;
     %
     % Value Dependency For Control Variables:
     \draw[ thick, -Straight Barb] (6.5, 2.5) arc (150:-150:1);
    %  \draw[ ultra thick, -latex, densely dotted,] (-0.5, 1.5)  to  [out=-250,in=250]  (-0.5, 7);
     % Control Dependency
     \draw[ thick, -latex] (5, 2.5)  -- (5, 5.5) ;
     \draw[ thick,-latex] (1.5, 6)  -- (3.5, 6) ;
     \draw[ thick,-latex] (1.5, 1.8)  -- (3.5, 6) ;
     \draw[ thick,-latex] (1.5, 6)  -- (3.5, 2) ;
    %  \draw[ thick,-latex] (1.5, 4)  -- (4, 6) ;
     \draw[ thick,-latex] (1.5, 1.8)  -- (3.5, 2) ; 
    \end{tikzpicture}
     \caption{}
        \end{centering}
        \end{subfigure}
    % \end{wrapfigure}
    % \end{equation*}
    \vspace{-0.4cm}
     \caption{(a) The multi rounds single example
     (b) The execution-based dependency graph.}
    \label{fig:overdefn_example}
    \vspace{-0.5cm}
    \end{figure}
        }
    \end{example}