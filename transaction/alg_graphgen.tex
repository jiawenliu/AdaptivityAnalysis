\highlight{
With the two components $\progV(c)$, and $\progE(c)$
computed in each steps above, this step simply combine the four components into the quantitative dependency graph for program $c$ as follows,
  \[
    \progG(c) = (\progV(c), \progE(c)).
    \]
}
We prove that this graph is a sound approximation of the program's semantics-based dependency graph by soundness of each component formally in Appendix.

This estimated graph has a similar topology structure as 
the Semantics-based Dependency Graph. It has the same
vertices 
% and query annotations, 
but approximated edges and weights.  
This graph is a sound approximation of the quantitative dependency graph for a program $c$.

It is formally defined in Definition~\ref{def:prog_graph} as follows.

\begin{defn}
[Previous Estimated Dependency Graph]
\label{def:prog_graph_pre}
Given a program $c$, with its abstract weighted control flow graph $\absG(c) = (\absV, \absE)$ and 
feasible data flow relation $\flowsto(x^i, y^j, c)$ for every $x^i, y^j \in \lvar(c)$, its estimated dependency graph
is generated as follows.

\[\progG(c) = (\progV(c), \progE(c), \progW(c), \progF(c))\]

{
\[
\begin{array}{rlcl}
\text{Vertices} &
\progV & := & \left\{ 
x^l \in \mathcal{LV} 
~ \middle\vert ~
x^l \in \lvar_{c}
\right\}
\\
\text{Directed Edges} &
\progE & := & 
\left\{ 
(x^i, y^j) \in \mathcal{LV} \times \mathcal{LV}
~ \middle\vert ~
\begin{array}{l}
x^i, y^j \in \vertxs
\land
\exists n \in \mathbb{N}, z_1^{r_1}, \cdots, z_n^{r_n} \in \lvar_{{c}} \st 
n \geq 0 \land
\\
\flowsto(y^j,  z_1^{r_1}, c) 
\land \cdots \land \flowsto(z_n^{r_n}, x^i, c) 
\end{array}
\right\}
\\
\text{Weights} &
\progW & := &
\left\{ (x^l, w) \in  \mathcal{LV} \times \mathcal{A}_{in}
\mid
x^l \in \lvar_{{c}} \land (l, w) \in \absW(c)
\right\}
\\
\text{Query Annotation} &
\progF & := & 
\left\{(x^l, n)  \in  \mathcal{LV} \times \{0, 1\} 
~ \middle\vert ~
x^l \in \lvar_{c},
n = 1 \iff x^l \in \qvar_{c} \land n = 0 \iff  x^l \in \qvar_{c} .
\right\}
\end{array}
\] }
\end{defn}
The construction of the static analysis dependency graph is of great value of showing some useful properties of the target program,
such as dependency between variables, the execution upper bound of a certain command,
while the key novelty is our path searching algorithm, which connects all the information we need in the static anlaysis dependency graph and provides us a sound over-estimation of adaptivity.

\highlight{
  \begin{defn}
  [Estimated Dependency Graph]
  \label{def:prog_graph}
Given a program $c$, with its abstract weighted control flow graph $\absG(c) = (\absV, \absE, \absW)$ and 
feasible data flow relation $\flowsto(x^i, y^j, c)$ for every $x^i, y^j \in \lvar_c$, its Program-Based Weighted Data Dependency Graph
$\progG(c) = (\progV, \progE)$,
is generated as follows,
{\footnotesize
\[
\begin{array}{lcl}
\progV(c) & \triangleq &
\left\{ (x^l, w) \in  \mathcal{LV} \times \mathcal{A}_{in}
\mid
x^l \in \lvar_{{c}} \land (l, w) \in \absW(c)
\right\}
\\
\progE(c) & \triangleq &
   \Big\{ (x^i, w, y^j) \in \mathcal{LV} \times 
   \mathcal{A}_{\kw{in}} \times \mathcal{LV}
~\mid~
  \\ & & \quad 
x^i, y^j \in \lvar(c) \land \flowsto(x^i, y^j, c) \land
  \exists n \in \mathbb{N}, z_1^{r_1}, \cdots, z_n^{r_n} \in \lvar_{{c}} \st 
  n \geq 0 
  % \\ & & \quad 
  % \flowsto(x^i,  z_1^{r_1}, c) 
  \land \cdots \land \flowsto(z_n^{r_n}, y^j, c) 
  \\ & & \quad 
  \land
  w = \max \left\{ \absclr(\absevent) ~\mid~ \absevent \in \absflow(c) \land \absevent = (i, \_, j) \right\} 
\Big\}.
\end{array}
\]}
\end{defn}
The construction of the static analysis dependency graph is of great value of showing some useful properties of the target program,
such as dependency between variables, the execution upper bound of a certain command,
while the key novelty is our path searching algorithm, which connects all the information we need in the static anlaysis dependency graph and provides us a sound over-estimation of adaptivity.
}
