This section shows how to generate the abstract transition graph $\absG(c)$ of a
program $c$ through constructing its vertices and edges.

An \emph{Abstract Transition Graph}, $\absG(c)$ for a program $c$ is composed of
a vertex set $\absV(c)$ and an edge set $\absE(c)$, $\absG(c) \triangleq (\absV(c), \absE(c))$.
% For a program $c$, this analysis first generates its abstract execution control flow graph notated as follows,
% \[\absG(c) =(\absV(c), \absE(c))\]
%
\\
Every 
vertex $l \in \absV(c)$ is the label of a labeled command in $c$, which is unique.
We also call the unique label as program point.
% corresponds to a program point $l$, which is a unique
% label of a command in this program.
% $\absV(c)$ is the set of $c$'s all program points,
\\
Each edge $(l \xrightarrow{dc} l') \in \absE(c)$ is an abstract transition
between two program points $l, l'$. 
There is an edge from $l$ to $l'$ if and only if
the command with label $l'$ can execute right after the execution of the command with label $l$.
% if and only if there is a control flow between two program points.
Each edge is annotated by a constraint $dc \in \dcdom^{\top}$, which is generated from the command with label $l$.
This constraint describes the abstract execution of the command with $l$. 
%  before the introduction of the edge and weight estimation.  
% We discuss the vertices and edge of the
% abstract transition graph for a program $c$, $\absG(c)$.

\paragraph{Abstract Control Flow Graph Vertices Construction}
\label{sec:alg_abscfg-vertex}
Every 
vertex $l \in \absV(c)$ corresponds to a program point $l$, which is a unique
label of a command in this program.
Concretely,
the vertices of this graph is the set of $c$'s labels with the exit label ${\lex}$ formally as follows,
\[ 
  \absV(c) = \lvar(c)\cup\{{\lex}\}
\]
%  corresponding to a label command in the program.

\paragraph{Abstract Control Flow Graph Edge Construction}
\label{sec:alg_abscfg-edge}
Each edge $(l \xrightarrow{dc} l') \in \absE(c)$ is an abstract transition
between two program points $l, l'$. 
There is an edge from $l$ to $l'$ if and only if
the command with label $l'$ can execute right after the execution of the command with label $l$.
% if and only if there is a control flow between two program points.
Each edge is annotated by a constraint $dc$ generated from the command with label $l$.
This constraint describes the abstract execution of the command with $l$. 
It is either
the symbol $\top$, a boolean expression or a \emph{difference
constraint}~\cite{sinn2017complexity}.
This step shows how to generate the abstract transition graph $\absG(c)$ of a
program $c$ through constructing its vertices and edges.
  \\
  In the first step, \textbf{Constraint Computation} generates a constraint
  over the expression for every program's labeled command,
  which is used as the annotation of an edge.
  \\
  In the second step, \textbf{Initial and Final State Computation} generates two sets for each command. 
  The initial state is a set that contains the
  program point where this command {starts} executing, 
  and the final state is a set
  that contains the constraint of this command
  and the continuation program points after the execution of this command.
  \\ 
  In the third step, \textbf{Abstract Event Computation} generates a set of edges for the program.
  Each edge is a pair of initial and finial state.
%
\paragraph{Constraint Computation}
In this step, we first show how to compute the constraints for expressions in a program $c$,
by a program abstraction method adopted from the
algorithm in Section 6 in~\cite{sinn2017complexity}.
\\
Given a program $c$,
every expression in an assignment command or in the guard of a $\eif$ or $\ewhile$ command
is transformed into a constraint.

\highlight{Notations:}
\begin{itemize}
\item Operator: $\absexpr : \mathcal{A} \cup \mathcal{B} \to DC(\mathcal{V}  \cup \constdom)\cup \booldom \cup \{\top\}$
%
\item Constrains, $\dcdom^{\top}$ is composed of the \emph{Difference Constraints} $DC(\mathcal{V}  \cup \constdom)$, the \emph{Boolean Expressions} $\booldom$ and $\top$.
%
\begin{itemize}
\item 
\jl{
The difference constraints $DC(\mathcal{V}  \cup \constdom)$ is the set of all the inequality of
form $x' \leq y + v$ or $x' \leq v$ where $x \in \mathcal{V} $, 
$y \in \mathcal{V}$ and $v \in \constdom$.
The \emph{Symbolic Constant} set $\constdom = \mathbb{N} \cup \inpvar \cup{\infty \cup{Q_m}}$
is the set of natural numbers with $\infty$, the input variables, and a symbol $Q_m$ representing the abstract value of a query request.
An inequality $x' \leq y + v$ describes that the value of $x$ in the current state is
at most the value of $y$ in the previous state plus the symbolic constant $v$.
An inequality $x' \leq v$ describes that the value of $x$ in the current state is
at most the value $v$.
When a difference constrain shows up as an edge annotation, $l \xrightarrow{x' \leq y + v} l'$,
% Then $x'$ 
it denotes that
the value of variable $x$
after executing the command at $l$ is at most
% and the right-hand side describes 
the value of variable $y$ plus $v$ before the execution,
and $l \xrightarrow{x' \leq v} l'$ respectively denotes value of variable $x$
after executing the command at $l$ is at most
% and the right-hand side describes 
the value of the symbolic constant $v$ before the execution.
For every expression in each of the label command, it is computed in three steps via program abstraction method adopted from the Section~6 in \cite{sinn2017complexity}. 
}
%
\item The Boolean Expressions $b$ from the set $\booldom$.
$b$ on an edge $l \xrightarrow{b} l'$ describes
that after evaluating the guard with label $l$,
$b$ holds and the command with label $l$ will execute right after.
%
\item The top constraint, $\top$ denotes true. It is preserved for $\eskip$ command, or commands that don't
interfere with any counter variable.
%
\end{itemize}
\end{itemize}

\highlight{Computation Steps:}
\begin{defn}[Constraint Computation]
  \label{def:constraint_compute}
  For a program $c$, a boolean expression $\bexpr$ in the guard of a $\eif$ or $\ewhile$ command
  or an expression $\expr$ and a variable $x$
  in an assignment command $\assign{x}{\expr}$,
  % or 
  % For a boolean expression $\bexpr$ or an arithmetic expression $\aexpr$ and a variable $x$,
  the constraint $\absexpr(\bexpr, \_)$ or $\absexpr(x - v, x)$ is computed as follows,
  \[
    \begin{array}{ll} 
      \absexpr(x - v, x)  = x' \leq x - v  & x \in \grdvar \land v \in \constdom \\
      \absexpr(y + v, x)  = x' \leq y + v  & x \in \grdvar \land v \in \constdom \land y \in (\grdvar \cup \constdom) \\
      \absexpr(v, x)  = x' \leq v  & x \in \grdvar \land v \in (\grdvar \cup \constdom) \\
      \absexpr(y + v, x)  = x' \leq y + v, 
      \grdvar = \grdvar \cup \{y\} & x \in \grdvar \land v \in \constdom \land y \notin (\grdvar \cup \constdom)  \\
      \absexpr(\qexpr, x)  = x' \leq Q_m & x \in \grdvar \land \qexpr \text{ is a query expression}  \\
      \absexpr(\bexpr, \_) = \bexpr, \grdvar = \grdvar \cup FV(\bexpr) &  x \in \grdvar \land \bexpr \text{ is a boolean expression} \\
      % \absexpr(\qexpr, x)  = x' \leq 0 + Q_m & x \in \grdvar \land \qexpr \text{ is a query expression}  \\
      \absexpr(\expr, x) = \top  &  x \notin \grdvar \\
    \end{array}
    \]
    $\absexpr(\expr, x)$ is iteratively updating until stabilized over every guard and assignment command in $c$, and $\absexpr(\expr, x)$
    denotes the stabilized result in the following paper.
  \end{defn}
%
  $\grdvar$ is the set of variables used in the guard expression of every while command in the program $c$. 
  In the case 4, if a variable $x$, belonging to the set 
  $\grdvar$ is updated by a variable $y$, which isn't in this set, 
  we add $y$ into the set $\grdvar$ and repeat 
  above procedure  until $\grdvar$ and $\absexpr(\expr, x)$ is stabilized. 
  % \wq{I do not understand this sentence:-(}
  \\
Specifically 
% understanding the intuition, 
we handle a 
% simplified 
normalized expression, $x > 0$
in guards of while loop headers, and 
%  \wq{I do not understand this sentence:-(}
%  .
% \\
% The counter variables only increase, decrease or reset by expression in the form of arithmetic minus and plus (able to extend to max and min.)
the counter variable $x$ only increase, decrease or reset by 
% expression in the form of 
simple arithmetic expression (mainly multiplication, division, minus and plus (able to extend to max and min)). 
The counter variable $x$ is generalized into norm when the boolean expression $x > 0$
in $\ewhile$ doesn't have the form $x > 0$.
The way of normalizing the guards and computing the norms is adopted from the computation step 1 in Section 6.1 in paper \cite{sinn2017complexity}. 
% \\
% For more complex expression assignments, where the counter reset, or calculated from $\elog$, 
% multiplication or division, and expressions involving multiple variables, the constraint is approximated as reset of $\infty$.
% \\
% % This simplification \wq{which part we simplify here?} 
% This approximation strategy
% doesn't affect our analysis results in our examples. It is easy to extend the normalized expression 
% into more complex forms as in \cite{sinn2017complexity}, as well as the 
% counter variable manipulation with more advanced expressions.
% \\ 
% The boolean expression in the guard of $\ewhile$ command is normalized into form of $ x > 0$ where $x^l \in \lvar_c$ for some $l$.
\begin{defn}[Symbolic Expression ($\mathcal{A}_{S}$)]
  $\mathcal{A}_{S}$ is the set of all the symbolic expressions 
over $\constdom$.
% For concise, $\mathcal{A}_{\lin}$ is used as the same meaning of $\mathcal{A}_{\lin}$ in the follows, to denote the arithmetic expression 
% over the symbolic variables, (i.e., $\mathbb{N}$ with input variables).
\end{defn}
The symbolic expression set is a subset of arithmetic expressions over $\mathbb{N}$ with input variables, 
i.e., $\mathcal{A}_{S} \subseteq \mathcal{A}_{\lin}$.

\paragraph{Abstract Initial and Final State Computation}
The initial state is the
program points before executing this command, which is computed by the standard initial state generation method from control flow analysis.
The final state is a set
that contains the constraint of this command and the program points after the execution of this command.
This set is enriched 
% program's initial and final states 
from the standard control flow analysis.

%
\highlight{Notations:}
\begin{itemize}
\item The \emph{abstract initial state}: $\absinit(c) \in \ldom$.
%
\item The \emph{abstract final state}: $\absfinal(c) \in \mathcal{P}(\ldom \times \dcdom^{\top})$
\end{itemize}

\highlight{Computation Steps:}
\begin{itemize}
  \item The \emph{abstract initial state}, $\absinit(c) \in \mathcal{P}(\ldom)$
  for a command $c$ is the initial program point corresponds to the unique program label of the first executed command in $c$,
  computed as follows,
%
\[
  \begin{array}{ll}
    \absinit(\clabel{\assign{x}{\expr}}{}^l)  & = l  \\
    \absinit(\clabel{\assign{x}{\query(\qexpr)}}{}^l)  & = l \\
    \absinit(\clabel{\eskip}^{l})  & = l \\
    \absinit(\eif [b]^l \ethen c_1 \eelse c_2)  & = l\\
    \absinit(\ewhile [b]^l \edo c)  & = l \\
    \absinit(c_1 ; c_2)  & = \absinit(c_1) \\
    \todomath{\absinit(\clabel{\efun}^l: x(r, x_1, \ldots, x_n) := c) } & =\\
    \todomath{\absinit(\clabel{\assign{x}{\ecall(x, e_1, \ldots, e_n)}}^l)} & = \\
 \end{array}
 \]
%
%
\item The \emph{abstract final state} of the program $c$, 
$\absfinal(c) \in \mathcal{P}(\ldom \times \dcdom^{\top})$
is a set of pairs, $(l, dc)$ with a
program point (i.e., a label), $l$ as the first component and a constraint, 
$dc$ as the second component.
% Every pair in $\absfinal(c)$ 
The program point $l$ corresponds to the labeled command after the execution of $c$,
and the constraint $dc$ in this pair is computed by $\absexpr$ for the expression in $c$.
%  in the first step.
\\
Given a program $c$, its final state, $\absfinal(c)$ is computed as follows,
% $\absfinal(c) \in\mathcal{P}(\ldom \times \dcdom^{\top})$,
% computes the set of Abstract Final State for the command. 
 \[
  \begin{array}{ll}
    \absfinal(\clabel{\assign{x}{\expr}}{}^l)  & = \{(l, \absexpr\eapp (\expr, x))\}  \\
     \absfinal(\clabel{\assign{x}{\query(\qexpr)}}{}^l)  & = \{
      (l, x' \leq 0 + Q_m )\}  \\
     \absfinal(\clabel{\eskip}^{l})  
     & = \{(l, \top)\} \\
     \absfinal(\eif [b]^l \ethen c_1 \eelse c_2)  & = \absfinal(c_1) \cup \absfinal(c_2) \\
     \absfinal(\ewhile [b]^l \edo c)  & = \{(l, \absexpr(\bexpr, \top))\} \\
     \absfinal(c_1 ; c_2)  & =  \absfinal(c_2) \\
     \todomath{\absfinal(\clabel{\efun}^l: x(r, x_1, \ldots, x_n) := c) } & =\\
    \todomath{\absfinal(\clabel{\assign{x}{\ecall(x, e_1, \ldots, e_n)}}^l)} & = \\
\end{array}
 \]
 %
\end{itemize}
 \paragraph{Abstract Event Computation} Each abstract event is an edge between two vertices in the abstract transition graph.
 It is generated by computing the initial state and finial state interactively and recursively for a program $c$.
 
 \highlight{Notations / Formal Definitions:}
 \begin{itemize}
  \item \emph{Abstract Event}: 
  $\absevent \in $
  $\ldom \times \dcdom^{\top} \times \ldom$
  \item \emph{Abstract Event Computation}: $\absflow \in \cdom \to \mathcal{P}( \ldom \times \dcdom^{\top} \times \ldom )$
 \end{itemize}
 Its type is defined as follows,
 \begin{defn}[Abstract Event]
   \label{def:abs_event}
   Abstract Event: 
   $\absevent \in $
   $\ldom \times \dcdom^{\top} \times \ldom$
   is a 
   % pair of abstract initial state and final state.
   triple where the first and third components are labels,
   second component is a constraint from $\dcdom^{\top}$.
   % the thrid % computed from program's abstract final and initial state, $\absfinal(c)$ and $\absinit(c)$ with formal definition, and algorithm detail in Appendix.
   %  the constraint and the third corresponds to a final state.
   \end{defn}
   In an abstract event $(l, dc, l')$ of a program $c$, 
   the first label $l \in \ldom$ corresponds to an initial state of $c$, and 
   the second label $l' \in \ldom$ with the constraint $dc \in \dcdom^{\top}$ correspond to an abstract final state of $c$.
  The abstract initial state is a label from $\ldom$.
 %
We abuse the notation $\mathcal{P}(\absevent)$ for the power set of all abstract events.

 \highlight{Computation Steps:}
\\
The set of the abstract events $\absflow(c)$ for a program $c$
% .
%  Its type is formally defined 
is computed as follows in Definition~\ref{def:absevent_compute}.
 %
 \begin{defn}[Abstract Event Computation]
 \label{def:absevent_compute}
  $\absflow \in \cdom \to \mathcal{P}( \ldom \times \dcdom^{\top} \times \ldom )$
  \end{defn}
 %
%  The \emph{Abstract Execution Trace} for program $c$ is computed as follows.
%  \\
  % We now show how to compute the abstract execution trace. 
 We first append a $\eskip$ command with 
%  a symbolic label $l_e$, i.e., $\clabel{\eskip}^{l_e}$ at the end of the program $c$, and compute the $\absflow(c) = \absflow'(c')$ for $c'$, where $c' = c;\clabel{\eskip}^{l_e}$ as follows,
the label $\lex$, i.e., $\clabel{\eskip}^{\lex}$ at the end of the program $c$, and construct 
the program $c' = c;\clabel{\eskip}^{\lex}$.
Then, we compute the $\absflow(c) = \absflow'(c')$ for $c'$ as follows,
 %
 {
 \[
   \begin{array}{ll}
      \absflow'(\clabel{\assign{x}{\expr}}{}^l)  & = \emptyset  \\
      \absflow'(\clabel{\assign{x}{\query(\qexpr)}}{}^l)  & = \emptyset  \\
      \absflow'([\eskip]^{l})  & = \emptyset \\
      \absflow'(\eif [b]^l \ethen c_t \eelse c_f)  & =  \absflow'(c_t) \cup \absflow'(c_f)
        \\ & \quad 
        \cup \left\{(l, \absexpr(\bexpr, \top),  \absinit(c_t) ) \right\}
        \\ & \quad 
        \cup \left\{ (l, \absexpr(\neg\bexpr, \top), \absinit(c_f)) \right\} \\
       \absflow'(\ewhile [b]^l \edo c_w)  & =  \absflow'(c_w) \cup \{(l, \absexpr(\bexpr, \top), \absinit(c_w)) \} 
       \\ & \quad 
       \cup \{(l', dc, l)| (l', dc) \in \absfinal(c_w) \} \\
       \absflow'(c_1 ; c_2)  & = \absflow'(c_1) \cup  \absflow'(c_2) 
       \\ & \quad 
       \cup \{ (l, dc, \absinit(c_2)) | (l, dc) \in \absfinal(c_1) \} \\
       \todomath{\absflow'(\clabel{\efun}^l: x(r, x_1, \ldots, x_n) := c) } & = \\
       \todomath{\absflow'(\clabel{\assign{x}{\ecall(x, e_1, \ldots, e_n)}}^l)} & = \\
      \end{array}
   \]
   }
   Notice $\absflow'([x := \expr]^{l})$, $\absflow'([x := \query(\qexpr)]^{l})$ and $\absflow'([\eskip]^{l})$ are all empty set. 
   
 \highlight{Theorem Guarantees:}

For every event $\event$ with label $l$, if it is in an execution trace $\trace$ of the program $c$, 
   there is an abstract event in program's abstract execution trace of form $(l, \_, \_)$, formally below
   with the proof in Appendix~\ref{apdx:abscfg_sound}
   \begin{lem}[Soundness of the Abstract Events]
     \label{lem:abscfg_sound}
     For every program $c$ and
     an execution trace $\trace \in \mathcal{T}$ that is generated w.r.t.
     an initial trace  $\vtrace_0 \in \mathcal{T}_0(c)$,
     there is an abstract event $\absevent = (l, \_, \_) \in \absflow(c)$ 
     for every event $\event \in \trace$ having the label $l$, i.e., $\event = (\_, l, \_, \_)$.
      %
   \[
     \begin{array}{l}
       \forall c \in \cdom, \vtrace_0 \in \mathcal{T}_0(c), \trace \in \mathcal{T} ,  \event = (\_, l, \_, \_) \in \eventset \st
   \config{{c}, \trace_0} \to^{*} \config{\eskip, \trace_0 \tracecat \vtrace} 
   \land \event \in \trace 
   \\
   \qquad \implies \exists \absevent = (l, \_, \_) \in (\ldom\times \dcdom^{\top} \times \ldom) \st 
   \absevent \in \absflow(c)
   \end{array}
   \]
   \end{lem}

For every program point $l$, if it is the label of an assignment command in a program $c$,
there is a unique abstract event in the program's abstract events set $\absevent \in \absflow(c)$ of form $(l, \_, \_)$. 
\begin{lem}[Uniqueness of the Abstract Events Computation]
  \label{lem:abscfg_uniquex}
  For every program $c$ and
  an execution trace $\trace \in \mathcal{T}$ that is generated w.r.t.
  an initial trace $\vtrace_0 \in \mathcal{T}_0(c)$,
  there is a unique abstract event $\absevent = (l, \_, \_) \in \absflow(c)$ 
  for every assignment event $\event \in \eventset^{\asn}$ in the
  execution trace having the label $l$, i.e., $\event = (\_, l, \_, \_)$ and  $\event \in \trace$.
%
\[
  \begin{array}{l}
    \forall c \in \cdom, \vtrace_0 \in \mathcal{T}_0(c), \trace \in \mathcal{T} ,  \event = (\_, l, \_) \in \eventset^{\asn} \st
\config{{c}, \trace_0} \to^{*} \config{\eskip, \trace_0 \tracecat \vtrace} 
\land \event \in \trace 
\\
\qquad \implies \exists! \absevent = (l, \_, \_) \in (\ldom\times \dcdom^{\top} \times \ldom) \st 
\absevent \in \absflow(c)
\end{array}
\]
\end{lem}
This lemma is proved in Appendix~\ref{apdx:abscfg_uniquex}.

  \paragraph{Edge Construction}
The edge for $c$'s abstract transition graph is constructed simply by computing the program's abstract events set, $\absflow(c)$ as follows,
  \[
    \absE(c) = \{(l_1, dc, l_2) | (l_1, dc, l_2) \in \absflow(c)\}
  \]
\paragraph{Abstract Transition Graph Construction} 
With the vertices $\absV(c)$ and edges $\absE(c)$ ready, we construct the abstract transition graph, formally in
Definition~\ref{def:abs_cfg}.
%
\begin{defn}[Abstract Transition Graph]
\label{def:abs_cfg}
Given a program $c$, 
its \emph{abstract transition graph} $\absG(c) =(\absV(c), \absE(c))$ is computed as follows,
\\
$\absE(c) = \{(l_1, dc, l_2) | (l_1, dc, l_2) \in \absflow(c)\}$,
\\
$\absV(c) = \lvar(c)\cup\{\lex\}$
\end{defn}
\paragraph*{Example}
%
{\footnotesize
\begin{figure} 
    \centering
    \begin{subfigure}{.2\textwidth}
    \begin{centering}
    {\footnotesize
    $
        \begin{array}{l}
              \clabel{ \assign{a}{0}}^{0} ;   
                \clabel{\assign{j}{k} }^{1} ; \\
                \ewhile ~ \clabel{j > 0}^{2} ~ \edo ~ \\
                \Big(
                 \clabel{\assign{x}{\query(\chi[j])} }^{3}  ; \\
                 \clabel{\assign{j}{j-1}}^{4} ;\\
                \clabel{\assign{a}{x + a}}^{5}       \Big);\\
                \clabel{\assign{l}{\query(\chi[k]*a)} }^{6}\\
            \end{array}
    $
    }
    \caption{}
    \end{centering}
    \end{subfigure}
\begin{subfigure}{.35\textwidth}
        \begin{centering}
      \begin{tikzpicture}[scale=\textwidth/22cm,samples=200]
      \draw[] (-7, 10) circle (0pt) node{{ $0$}};
      \draw[] (0, 10) circle (0pt) node{{ $1$}};
      \draw[] (0, 7) circle (0pt) node{\textbf{$2$}};
      \draw[] (0, 4) circle (0pt) node{{ $3$}};
      \draw[] (0, 1) circle (0pt) node{{ $4$}};
      \draw[] (-7, 1) circle (0pt) node{{ $5$}};
      % Counter Variables
      \draw[] (6, 7) circle (0pt) node {\textbf{$6$}};
      \draw[] (6, 4) circle (0pt) node {{ $\lex$}};
      %
      % Control Flow Edges:
      \draw[  -latex] (-6, 10)  -- node [above] {$\top$}(-1.5, 10);
      \draw[ -latex] (0, 9.5)  -- node [left] {$j' \leq k$} (0, 7.5) ;
      \draw[ -latex] (0, 6.5)  -- node [right] {$j > 0 $}  (0, 4.5);
      \draw[ -latex] (0, 3.5)  -- node [right] {$\top $} (0, 1.5) ;
      \draw[ -latex] (-0.5, 1)  -- node [above] {$j' \leq j - 1$} (-6, 1) ;
      \draw[ -latex] (-6, 1.5)  -- node [left] {$\top$} (-0.5, 7)  ;
      \draw[ -latex] (0.5, 7)  -- node [above] {$ j \leq 0 $}  (5.5, 7);
      \draw[ -latex] (6, 6.5)  -- node [right] {$\top$} (6, 4.5) ;
      \end{tikzpicture}
      \caption{}
        \end{centering}
        \end{subfigure}
        \begin{subfigure}{.35\textwidth}
          \begin{centering}
        %   \todo{abstract-cfg for two round}
        \begin{tikzpicture}[scale=\textwidth/22cm,samples=200]
        \draw[] (-10, 10) circle (0pt) node{{ $0: 1$}};
        \draw[] (0, 10) circle (0pt) node{{ $1: 1$}};
        \draw[] (0, 7) circle (0pt) node{\textbf{$2: k$}};
        \draw[] (0, 4) circle (0pt) node{{ $3: k$}};
        \draw[] (0, 1) circle (0pt) node{{ $4: k$}};
        \draw[] (-10, 1) circle (0pt) node{{ $5: k$}};
        % Counter Variables
        \draw[] (6, 7) circle (0pt) node {\textbf{$6: 1$}};
        \draw[] (6, 4) circle (0pt) node {{ $\lex: 1$}};
        %
        % Control Flow Edges:
      \draw[  -latex] (-8, 10)  -- node [above] {$\top$}(-1.5, 10);
      \draw[ -latex] (0, 9.5)  -- node [left] {$j' \leq k$} (0, 7.5) ;
      \draw[ -latex] (0, 6.5)  -- node [right] {$j > 0 $}  (0, 4.5);
      \draw[ -latex] (0, 3.5)  -- node [right] {$\top $} (0, 1.5) ;
      \draw[ -latex] (-1.5, 1)  -- node [above] {$j' \leq j - 1$} (-8, 1) ;
      \draw[ -latex] (-8, 1.5)  -- node [left] {$\top$} (-1.5, 7)  ;
      \draw[ -latex] (1.5, 7)  -- node [above] {$j \leq 0 $}  (4.5, 7);
      \draw[ -latex] (6, 6.5)  -- node [right] {$\top$} (6, 4.5) ;
        \end{tikzpicture}
        \caption{}
          \end{centering}
          \end{subfigure}
      \vspace{-0.5cm}
      \caption{(a) The same $\kw{towRounds(k)}$ program as Figure~\ref{fig:overview-example}
      (b) The abstract control flow graph, $\absG(\kw{twoRounds(k)})$  (c) $\absG(\kw{twoRounds(k)})$ with the reachability bound.}
      \label{fig:abscfg_tworound}
      \vspace{-0.5cm}
    \end{figure}
    }