\begin{proof}
This lemma is proved by showing there is a contradiction. 

\jl{
Assume there exists an edge  $(\av', \av) \in \edges$ and $\av' \avgeq \av$, where $\av' = ({\qval}',l',w')$ and $\av = ({\qval},l,w)$.
%
According to the Definition~\ref{def:trace-based_graph}, we have:
%
$$
DEP(\av', \av, c, m, D) ~ (1)
$$
%
%
Unfolding the Definition \ref{def:query_dep} in $(1)$,
we know: there exists $t_1, t_3, m_1, m_3, c_2$ s.t.,
%
\[
\config{m, c, [], []} \rightarrow^{*} 
\config{m_1, [\assign{x}{\query({\qval'})}]^{l'} ; c_2,
  t_1, w'} 
\rightarrow^{\textbf{query-v}} 
\config{m_1[v_1/x], c_2,
(t_1 ++ [\av'], w_1} \rightarrow^{*} \config{m_3, \eskip, t_3,w_3} ~ (a)  
\]
%
and
%
\[
 \bigwedge
 \begin{array}{l}   
  % 
  \left( 
  \begin{array}{l}
  \av \avin (t_3 - (t_1 ++ [\av'])) 
  % 
  \\
  \implies 
  \exists v \in \qdom, v \neq v_1, m_3', t_3', w_3'. ~  
  \config{m_1[v/x], {c_2}, t_1 ++ [\av'], w_1} 
  \\ 
  \quad \quad 
  \rightarrow^{*}
  (\config{m_3', \eskip, t_3', w_3'} 
  \land 
  \av \not \avin (t_3'-(t_1 ++ [\av'])))
\end{array} \right ) ~(b)
\\
\left( 
  \begin{array}{l}
  \av \not\avin (t_3 - (t_1 ++ [\av']))
    % 
    \\
    \implies 
  \exists v \in \qdom, v \neq v_1, m_3', t_3', w_3'. 
  \config{m_1[v/x], {c_2}, t_1 ++ [\av'], w_1}
  \\ 
  \quad \quad 
  \rightarrow^{*} 
  (\config{m_3', \eskip, t_3', w_3'} 
  \land 
  \av  \avin (t_3' - (t_1 ++ [\av'])))
\end{array} \right ) ~ (c)
\end{array}
\]
%
%
According to the Theorem \ref{thm:os_wf_trace} and $(a)$, we know both $t_1$, $t_3$ are well-formed traces.
% }
% \\
% \jl{
\\
Consider 2 cases:
 \[\av \avin (t_3 - (t_1 ++[\av'])) ~(d) ~~ \lor ~~ \av \notin_{aq} (t_3 - (t_1 ++[\av'])) ~(e)\]
%
\begin{itemize}
%
  \caseL{ (d) \[\av \avin (t_3 - (t_1 ++[\av'])) ~ (d)\]}
  By unfolding the trace subtraction operations, we have;
  \[
    \exists t_2. ~ s.t., ~ t_1 ++[\av'] ++ t_2 = t_3 \land \av \avin t_2 ~ (3)
  \]
%
%
According to the Corollary~\ref{coro:aqintrace} and $\av \avin t_2$, we have:
%
\[
  \exists t_{21}, t_{22}, \av' ~ s.t.,~ (\av \aveq \av') \land t_{21} ++ [\av'] ++ t_{22} = t_2 ~ (4)
\]
%
By rewriting $(4)$ inside $(3)$, we have:
%
\[
  \exists t_{21}, t_{22}, \av'' ~ s.t.,~ (\av \aveq \av'') \land t_1 ++[\av'] ++ t_{21} ++ [\av''] ++ t_{22} = t_3 ~ \star
\]
%
By the \emph{ordering} property in definition \ref{def:wf_trace} and $(\star)$, we know
%
\[
  \av' <_{aq} \av
\]
%
This is contradict to our assumption, where $\av' \avgeq \av$. 
%
%
\caseL{(e), \[\av \notin_{aq} (t_3 - (t_1 ++[\av'])) ~(e)\]} 
%
According to the condition $(c)$, we know: $\exists v \in \qdom, v \neq v_1, m_3', t_3', w_3'.$
    % 
\[ 
  \config{m_1[v/x], {c_2}, t_1 ++ [\av'], w_1} 
  \rightarrow^{*} (\config{m_3', \eskip, t_3', w_3'} ~ (5)
  \land \av  \in_{q} (t_3' - (t_1 ++ [\av']))) ~ (6)
\] 
%
According to the Theorem \ref{thm:os_wf_trace}, Sub-Lemma-t and $(5)$,
we know $t_3'- (t_1 ++ [\av'])$ is a well-formed trace.
\\
From $(a)$, we know that the minimal line number of $c_2$ is greater than $l'$, so we know that : 
\[
  \forall \av'' \in t_3'- (t_1 ++ [\av']), \av''>_{aq} \av'
\]
%
Unfolding the trace subtraction operations in $(6)$, we have;
\[
  \exists t_2'. ~ s.t., ~ t_1 ++[\av'] ++ t_2' = t_3' \land \av \avin t_2' ~ (7)
\]
%
%
According to the sub-lemma and $(7)$, we have:
%
\[
  \exists t_{21}', t_{22}' ~ s.t.,~ t_{21}' ++ [\av] ++ t_{22}' = t_2' ~ (8)
\]
%
By rewriting $(8)$ inside $(7)$, we have:
%
\[
  t_1 ++[\av'] ++ t_{21}' ++ [\av] ++ t_{22}' = t_3' ~ \diamond
\]
%
By the \emph{ordering} property in definition \ref{def:wf_trace} and $(\diamond)$, we know
%
\[
  \av' <_{aq} \av
\]
%
This is contradict to the assumption,  $\av' \avgeq \av$.
%
\end{itemize}
%
From both cases, we derive $\av' \avgeq \av$, which is contradict to the hypothesis, i.e., $\av' \avgeq \av$.
Then, we can conclude that 
for any directed edge $(\av', \av) \in \edges$, 
this is not the case that:
%
$$\av' \avgeq \av$$
%
}
\end{proof}
%
%
%