\begin{example}[Multiple Rounds Algorithm]
\label{ex:multipleRounds}
%
Our first example, Algorithm $\kw{multipleRounds}$ in Fig.~\ref{fig:multipleRounds}(a), is an advanced adaptive data analysis algorithm .
This is a simplified version of the \emph{Monitor Augment} from \cite{RogersRSSTW20} with complete program in Appendix.
It takes the user input $k$ which decides the 
number of iterations.
It starts from an initialized empty tracking list $I$,
goes $k$ rounds and at every round, tracking list $I$ is updated by a query result of $\query(\chi[I])$.
After $r$ rounds, the algorithm returns the columns of the hidden database $D$ not specified in the tracking list $I$.
The functions $\kw{updnscore}(p,a)$,
$\kw{updcscore}(p,a)$, $\kw{update}(I,ns,cs)$ simplify the computations of updating $ns$, $cs$ and $I$.%

% \jl{
    Different from $\kw{twoRounds(k)}$ in Figure~\ref{fig:overview-example},
the query request, $\clabel{\assign{a}{\query(I)}}^6$ in each loop iteration is not independent. 
$\query(I)$ in each iteration depends on the tracking list $I$ from all the previous iterations, 
and $I$ is updated by all the query results in the previous iterations as well. 
In this sense, all these $k$ queries are adaptively chosen according to our discussion in overview.
% }
% \jl{
The program-based dependency graph is presented in Figure~\ref{fig:multipleRounds}(b). 
We omitted its execution-based dependency graph $\traceG(\kw{multipleRounds(k)})$ because they have the same graph topology and only differ in weights.
For each vertex $v$ in $\progG(\kw{multipleRounds(k)})$ in Figure~\ref{fig:multipleRounds}(b),
we use $w_{v}$ to denote its weight function in $\traceG(\kw{multipleRounds(k)})$.
% }

% \jl{
As the adaptivity definition in our formal adaptivity model in Definition~\ref{def:trace_adapt},
there is a finite walk along the dashed arrows,
$a^{6} \to I^9 \to ns^{7} \to  \cdots \to ns^7$ , 
where the vertices $a^{6}$, $I^9$ and $ns^{7}$ are visited $w_{a^{6}}(\trace_0)$,
$w_{I^9}(\trace_0)$ and $w_{ns^{7}}(\trace_0)$
times respectively with input $\trace_0$.
The vertex $a^{6}$ has query annotation $1$, and it is visited $w_{a^{6}}(\trace_0)$ times.
In this sense, the adaptivity of this program is
$w_{a^{6}}(\trace_0)$ given input $\trace_0$, i.e., $A(\kw{multipleRounds(k)})(\trace_0) = w_{a^{6}}(\trace_0)$.
Since $w_{a^{6}}(\trace_0)$
counts the execution times of command $\clabel{\assign{a}{\query(I)}}^6;$,
this count is at most the loop iteration numbers, i.e., $k$'s initial value, $\env(\trace_0) k$ from the initial trace $\trace_0$.
% }
{
Next, we show that {$\THESYSTEM$} provides a tight upper bound for this example by $\pathsearch(\kw{multipleRounds(k)})$.
It first finds a path on $\progG(\kw{multipleRounds(k)})$
$a^{6}: {}^k_1 \to I^9:{}^k_0 \to ns^7:{}^k_0$ with three weighted vertices. 
Then $\pathsearch$ approximates this path to a walk $a^{6}: {}^k_1 \to I^9:{}^k_0 \to ns^7:{}^k_0 \to a^{6}: {}^k_1 \cdots$.
In this walk, $a^6, I^9, ns^{7}$ are all visited $k$ times respectively. 
So $\progA(\kw{multipleRounds(k)}) = k$.
We know for any initial trace $\trace_0$, $\config{\trace_0, k} \earrow \env(\trace_0)$
So we guarantee $A(\kw{multipleRounds(k)})(\trace_0) \leq \env(\trace_0)$ for any $\trace_0$
and $k$ is a sound bound.
}
\end{example}
%
\begin{figure}
\centering
\begin{subfigure}{0.25\textwidth}
    \small{
    $
\begin{array}{l}
\kw{multipleRounds(k)} \triangleq\\
    \clabel{\assign{j}{k}}^0;
    \clabel{\assign{I}{[]}}^1; \\
    \clabel{\assign{ns}{0}}^2; 
    \clabel{\assign{cs}{0}}^3; \\
    \ewhile ~ \clabel{j > 0}^{4} ~ \edo ~ \\
    \Big(
    \clabel{\assign{j}{j-1}}^{5} ;
    \clabel{\assign{a}{\query(I)}}^6; \\
    \clabel{\assign{ns}{\kw{updnscore}(ns, a)}}^7; \\
    \clabel{\assign{cs}{\kw{updcscore}(cs, a)}}^8; \\
    \clabel{\assign{I}{\kw{updI}(I, ns, cs)}}^9
    \Big) 
\end{array}
    $
    }
    \caption{}
\end{subfigure}
        \begin{subfigure}{.6\textwidth}
        \begin{centering}
        \begin{tikzpicture}[scale=\textwidth/25cm,samples=200]
    % Variables Initialization
     \draw[] (-7, 1) circle (0pt) node{{ $I^1: {}^1_{0}$}};
     \draw[] (-7, 7) circle (0pt) node{{$ns^2: {}^{1}_{0}$}};
     \draw[] (-7, 4) circle (0pt) node{{ $cs^3: {}^{1}_{0}$}};
     % Variables Inside the Loop
     \draw[] (0, 10) circle (0pt) node{{ $a^6: {}^{k}_{1}$}};
     \draw[] (0, 7) circle (0pt) node{{ $ns^7: {}^{k}_{0}$}};
     \draw[] (0, 4) circle (0pt) node{{ $cs^8: {}^{k}_{0}$}};
     \draw[] (0, 1) circle (0pt) node{{ $I^9: {}^{k}_{0}$}};
     % Counter Variables
     \draw[] (7, 9) circle (0pt) node {{$j^0: {}^{1}_{0}$}};
     \draw[] (7, 6) circle (0pt) node {{ $j^5: {}^{k}_{0}$}};
     %
     % Value Dependency Edges:
     \draw[  -latex,] (0, 1.5)  -- (0, 3.5) ;
     \draw[ ultra thick, -latex, densely dotted,] (0, 7.5)  -- (0, 9.5) ;
     \draw[  -Straight Barb] (1.4, 4) arc (120:-200:1);
     \draw[  -Straight Barb] (8.5, 6.5) arc (150:-150:1);
     \draw[  -Straight Barb] (1, 7.5) arc (220:-100:1);
     \draw[  -latex] (7, 6.5)  -- (7, 8.5) ;
     % Value Dependency Edges on Initial Values:
     \draw[  -latex,] (-1.5, 1)  -- (-5.5, 1) ;
     \draw[  -latex,] (-1.5, 4)  -- (-5.5, 4) ;
     \draw[  -latex,] (-1.5, 7)  -- (-5.5, 7) ;
     %
     \draw[ ultra thick, -latex, densely dotted,] (-1, 9.5)  to  [out=-130,in=130]  (-1, 1.5);
     \draw[ ultra thick, -latex, densely dotted,] (-0.8, 1.7)  to  [out=-230,in=230]  (-0.5, 6.5);
     % Value Dependency from cs8 -> a6
     \draw[  -latex, ] (-0.8, 4.0)  to  [out=-230,in=230]  (-0.5, 9.5);
     % Value Dependency from a6 -> I1
     \draw[  -latex,] (-1.2, 9.7)  -- (-5.5, 1);
     \draw[  -Straight Barb] (1.7, 1.5) arc (120:-200:1);
     % Control Dependency
     \draw[  -latex] (1.5, 7)  -- (5.8, 6) ;
     \draw[  -latex] (1.5, 4)  -- (5.8, 6) ;
     \draw[  -latex] (1.5, 1)  -- (5.8, 6) ;
     \draw[  -latex] (1.5, 10)  -- (5.8, 6) ;
     % Edges Produced by Transitivity by Control Dependency
     \draw[  -latex] (1.5, 7)  -- (5.8, 9) ;
     \draw[  -latex] (1.5, 4)  -- (5.8, 9) ;
     \draw[  -latex] (1.5, 1)  -- (5.8, 9) ;
     \draw[  -latex] (1.5, 10)  -- (5.8, 9) ;
     % Edges Produced by Transitivity from vertext a6 Dependency
     \draw[  -latex,] (-1.2, 9.7)  -- (-5.5, 4);
     \draw[  -latex,] (-1.2, 9.7)  -- (-5.5, 7);
     \draw[ -latex] (-1, 9.5)  to  [out=-130,in=130]  (-1, 7.5);
     \draw[ -latex] (0.5, 9.5)  to  [out=-50,in=50]  (0.5, 4);
     \draw[  -Straight Barb] (0.5, 10.5) arc (150:-150:1);
     % Edges Produced by Transitivity from vertext cs8 Dependency
     \draw[  -latex,] (-1.2, 4)  -- (-5.5, 1);
     \draw[  -latex,] (-1.2, 4)  -- (-5.5, 7);
     \draw[  -latex,] (0, 4.5)  -- (0, 6.5) ;
     % Edges Produced by Transitivity from vertext I9 Dependency
     \draw[  -latex,] (-1.2, 1)  -- (-5.5, 4);
     \draw[  -latex,] (-1.2, 1)  -- (-5.5, 7);
     \draw[  -latex,] (0.5, 1.0)  to  [out=50,in=-50]  (0.5, 9.5);
     % Edges Produced by Transitivity from vertext ns7 Dependency
     \draw[  -latex,] (-1.2, 7)  -- (-5.5, 1);
     \draw[  -latex,] (-1.2, 7)  -- (-5.5, 4);
     \draw[ -latex] (0.5, 6.5)  to  [out=-50,in=50]  (0.5, 1.5);
     \draw[ -latex] (0.5, 6.5)  to  [out=-50,in=50]  (0.5, 4.5);
    \end{tikzpicture}
     \caption{}
        \end{centering}
        \end{subfigure}
    \vspace{-0.4cm}
    \caption{(a) The simplified multiple rounds example (b) The estimated dependency graph by {\THESYSTEM}}
    \label{fig:multipleRounds}
\end{figure}
%