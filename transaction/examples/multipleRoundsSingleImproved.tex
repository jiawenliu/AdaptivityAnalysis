\begin{example}[Accurate Adapativity for Multiple Rounds Single Example]
    \label{ex:multipleRoundSingle}
\highlight{As the program in Figure~\ref{fig:multipleRoundsSingle}(a),
    which we have shown before as a limitation of our previous \emph{adaptivity} definition.
    It is a variant of the multiple rounds strategy, 
    % we call it a multiple rounds single iteration algorithm, 
    named $\kw{multipleRoundSingle(k)}$ with input $k$.
    % as the input variable.
    In this algorithm, 
    at line 3 of every iteration, 
    a query $\query(\chi[z] + y)$ based on previous query results stored in $z$ and $y$ is asked by the analyst 
    as the $\kw{multipleRounds}$ strategy. 
    The difference is that only the query answers from the one single iterations ($j = 2 $) are 
    % used to $b$. 
    used in this query $\query(\chi[z] + y)$.
    Because the execution trace updates 
    $y$ by the constant $0$ for all the iterations where ($j \neq 2$) at line $5$ after the 
    query request at line $3$.
    In this way, all the query answers stored in $y$ will not be accessed in next query request at line $3$ in the iterations 
    where  ($j \neq 2$).
    Only query answer at one single iteration where ($j = 2 $) will be used in next query request
    $\query(\chi[z] + y)$ at line $3$.
    So the adaptivity for this example is $2$. 
    % However, from the semantics-based dependency graph in Figure~\ref{fig:overappr_example}(b), 
    However, the previous adaptivity model fails to realize that there is only dependency relation 
    between $y^3$ and $y^3$ in one single iteration, 
    not the others. 
    % there is no edge from queries at odd iterations (such as $q_1,q_3,q_5$) to queries at even iteration(such as $q_2,q_4$). The longest path is dashed with a length $3$.  
    As shown in the semantics-based dependency graph in Figure~\ref{fig:multipleRoundsSingle}(b), 
    there is an edge from $y^3$ to itself representing the existence of \emph{Variable May-Dependency} from $y^3$ on itself,
    and the visiting times of labeled variable $y^3$ is 
    $w_k(\trace_0)$ with an initial trace $\trace_0$. 
    If we only have the restriction on the visiting times of vertices, the walk with the longest query length 
    is
    $y^3  \to \cdots \to y^3 \to z^1 $ with the vertex $y^3$ visited $w_k(\trace_0)$,
    as the dotted arrows, which is $k$.
    However, with the restriction on the visiting times of edges, we have the length of this longest walk only $2$, which captures the intuitive \emph{adaptivity} we want!
    % %
    % The $\THESYSTEM$ is able to give us $2$,  as an accurate bound w.r.t this definition.
    }
        \begin{figure}
     \centering
    \quad
    \begin{subfigure}{.35\textwidth}
        \begin{centering}
    {\small
        $ \begin{array}{l}
                \kw{multiRoundsS(k)} \triangleq \\
                   \clabel{ \assign{j}{0}}^{0} ; 
                    \clabel{\assign{z}{\query(0)} }^{1} ;             
                    \clabel{\assign{p}{0} }^{2} ; \\
                    \eif(\clabel{ k = 0}^{3}, \\
                     \quad \clabel{ \assign{y}{\query(z)}}^{4}, \\
                     \quad \clabel{\eskip}^5);\\
                    \ewhile ~ \clabel{j \neq k}^{6} ~ \edo ~ \Big(
                     \\
                     \quad \clabel{\assign{p}{\query(\chi[y]+p)} }^{7}  ; \\
                     \quad \clabel{\assign{j}{j + 1}}^{8}\\
                     \eif(\clabel{ j \neq k - 2}^{9}, \\
                     \quad \quad \clabel{ \assign{p}{0}}^{10} ,\\ 
                     \quad \quad \clabel{\eskip}^{10})\\
                     \quad \Big);\\
                \end{array}
        $       
    }
        \caption{}
        \end{centering}
        \end{subfigure}
    \begin{subfigure}{.6\textwidth}
        \begin{centering}
        \begin{tikzpicture}[scale=\textwidth/15cm,samples=150]
    % Variables Initialization
    \draw[] (-5, 4) circle (0pt) node{{ $z^1: {}^{w_1}_{1}$}};
    % Variables Inside the Loop
     \draw[] (0, 6) circle (0pt) node{{ $y^3: {}^{w_k}_{1}$}};
     \draw[] (0, 2) circle (0pt) node{{ $y^{5}: {}^{w_k}_{0}$}};
     % Counter Variables
     \draw[] (5, 6) circle (0pt) node {{$j^0: {}^{w_1}_{0}$}};
     \draw[] (5, 2) circle (0pt) node {{ $j^8: {}^{w_k}_{0}$}};
     %
     % Value Dependency Edges:
     \draw[ ultra thick, -Straight Barb, densely dotted,] (0.8, 7) arc (220:-100:1);
     % The Weight for this edge
     \draw[](1.2, 9.5) node 
     {\highlight{\footnotesize
            $\trace_0 \to 
            \left\{\begin{array}{ll}
               \env(\trace_0) k & \env(\trace_0) k  \leq 1 \\
           2 & \env(\trace_0) k \geq 2
            \end{array}\right\}
            $}};
     \draw[ thick, -latex] (-1, 6)  to  [out=-130,in=130]  
    % The Weight for this edge
    node [] {\highlight{$\trace_0 \to 1 $}} (-1, 2);
     % Value Dependency Edges on Initial Values:
     \draw[ ultra thick, -latex, densely dotted,] (-1.5, 6)  -- 
    % The Weight for this edge
    node [left] {\highlight{$\trace_0 \to \env(\trace_0) k $}} (-4, 4.7) ;
     %
     % Value Dependency For Control Variables:
     \draw[ thick, -Straight Barb] (6.5, 2.5) arc  (150:-150:1);
    % The Weight for this edge
    \draw[](8, 2) node [] {\highlight{$\trace_0 \to \env(\trace_0) k  $}};
     % Control Dependency
     \draw[ thick, -latex] (5, 2.5)  -- 
    % The Weight for this edge
    node [right] {\highlight{$\trace_0 \to \env(\trace_0) k $}} (5, 5.5);
     \draw[ thick,-latex] (1.5, 6)  -- (3.5, 6) ;
     \draw[ thick,-latex] (1.5, 1.8)  -- 
    % The Weight for this edge
    node [] {\highlight{$\trace_0 \to \env(\trace_0) k $}} (3.5, 6) ;
     \draw[ thick,-latex] (1.5, 6)  -- (3.5, 2) ;
     \draw[ thick,-latex] (1.5, 1.8)  -- (3.5, 2) ; 
    % Edges Produced by Transitivity
    \draw[  -latex,] (-1.5, 6)  -- (-4, 2) ;
    \draw[ -latex] (1.5, 1.8)  -- (3.5, 6) ; 
    \end{tikzpicture}
     \caption{}
        \end{centering}
        \end{subfigure}
     \caption{(a) The multi rounds single example
     (b) The semantics-based dependency graph.}
    \label{fig:multipleRoundsSingle}
    \end{figure}
    \end{example}