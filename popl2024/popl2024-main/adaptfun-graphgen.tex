This section constructs the estimated dependency graph.
\begin{defn}
[Estimated Dependency Graph]
\label{def:prog_graph}
Given a program $c$ 
with the feasible data flow relation $\flowsto(x^i, y^j, c)$ for every $x^i, y^j \in \lvar(c)$, 
and the $\absW(c)$ as the set of estimated weight for every program label $l$,
its estimated dependency graph
is generated as follows.
\[\progG(c) = (\progV(c), \progE(c), \progW(c), \progF(c))\]
Each of the four components is defined as follows.
{\small
\[
\begin{array}{lll}
\progV & := & \left\{ 
x^l
~ \middle\vert ~
x^l \in \lvar(c)
\right\}
\\
\progE & := & 
\left\{ 
(x^i, y^j) 
~ \middle\vert ~
\begin{array}{l}
x^i, y^j \in \vertxs
\land
\exists n \in \mathbb{N}, z_1^{r_1}, \ldots, z_n^{r_n} \in \lvar(c) \st
n \geq 0 \land
\\
\flowsto(y^j, z_1^{r_1}, c) 
\land \ldots \land \flowsto(z_n^{r_n}, x^i, c) 
\end{array}
\right\}
\\
\progW & := &
\left\{ (x^l, \hat{w}) 
% \in \lvar(c) \times \constdom
\mid
x^l \in \progV(c) \land 
\hat{w} = 
\sum\left\{ \absclr(\absevent, c) \middle\vert \absevent \in \absflow(c) \land \absevent = (l, \_, \_) \right\}
\right\}
\\
\progF & := & 
\left\{(x^l, n) 
% \in \lvar(c) \times \{0, 1\} 
~ \middle\vert ~
\begin{array}{l}
 x^l \in \lvar(c) \land
 n = 1 \iff x^l \in \qvar(c) \\
\land n = 0 \iff x^l \in \qvar(c) .
\end{array}
\right\}
\end{array}
\] }
\end{defn}
The construction of the static analysis dependency graph is of great value in showing some useful properties of the target program,
such as dependency between variables, the execution upper bound of a certain command,
while the key novelty is our path-searching algorithm, which connects all the information we need in the static analysis dependency graph and provides us with a sound estimation of adaptivity.
As in Fig.~\ref{fig:overview-example}(c), the estimated dependency graph for
running example is tight in terms of both edges and weights.
