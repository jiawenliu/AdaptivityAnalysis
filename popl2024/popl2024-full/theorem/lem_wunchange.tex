
%
\todo{
\begin{lem}[While Map Remains Unchanged (Invariant)]
\label{lem:wunchange}
Given a program $c$ with a starting memory $m$, trace $t$ and while map $w$, s.t.,
$\config{m, c, t, w} \to^{*} \config{m', \eskip, t', w'}$ and $Labels(c) \cap Keys(w) = \emptyset$, then 
\[
	w = w'
\]
\end{lem}
\begin{subproof}[Proof of Lemma~\ref{lem:wunchange}]
%
By induction on program $c$, we have following cases:
%
\caseL{$[\assign x \expr]^{l}$}
%
From the evaluation rule \rname{assn1} and rule \rname{assn2}, we know there exists an evaluation
$$
\config{m, [\assign x \expr]^{l}, t, w} \to^{*} \config{m[v/x], \eskip, t', w}
$$
%
, where $\config{m,e} \to^{*} v$.
%
The resulting while map is $w$, this case is proved because $w = w$.
%
\caseL{$[\assign{x}{\query(\qexpr)}]^{l}$}
%
By repeatedly applying the operational semantics rule \rname{query-e}, we know there exist $\qval$ s.t.,:
%
\[
\config{m, [\assign{x}{\query(\qexpr)}]^{l}, t, w} \to^{*} \config{m', [\assign{x}{\query(\qval)}]^{l}, t', w}
\]
%
According to the evaluation rule \rname{query-v}, we have the following transition:
\[
	\config{m', [\assign{x}{\query(\qval)}]^{l}, t', []} \to \config{m', \eskip, t' ++ [(\qval, l, w)], w}
\]
%
The resulting while map is $w$, this case is proved because $w = w$.
%%
\caseL{$\eif([b]^l, c_1, c_2)$}
%
According to the evaluation rule \rname{if}, we know there exists an evaluation by repeatedly applying the rule \rname{if}: 
%
\[
\config{m, \eif([b]^l, c_1, c_2), t, w} \to^{*} \config{m_b, \eif([v]^l, c_1, c_2), t', w} ~ (1)
\]
%
where $v$ is either $\etrue$ or $\efalse$.
\\
Considering 2 subcases where $v$ is either $\etrue$ or $\efalse$:
%
\subcaseL{$v = \etrue$}
%
By applying the evaluation rule \rname{if-t}, we have:
\[
\config{m_b, \eif([\etrue]^l, c_1, c_2), t', w} \to \config{m_b, c_1, t', w} ~ (t1)
\]
%
By induction hypothesis on $c_1$ with starting memory $m_b$, trace $t'$ and while map $w$, let $\config{m_1, \eskip, t_1, w_1}$ be the configuration s.t.
\[
	\config{m_b, c_1, t', w} \to^{*} \config{m_1, \eskip, t_1, w_1} ~(t2)
\]
we know $w_1 = w$.
\\
According to the evaluation $(a)$, $(t1)$ and $(t2)$, we have the following evaluation:
% 
\[
	\config{m, \eif([b]^l, c_1, c_2), t, w} \to^{*} \config{m_1, \eskip, t_1, w}
\]
%
, where the resulting while map is still $w$, this sub-case is proved.
%
\subcaseL{$v = \efalse$}
%
By applying the evaluation rule \rname{if-f}, we have:
\[
\config{m_b, \eif([\efalse]^l, c_1, c_2), t', w} \to \config{m', c_2, t', w} ~ (f1)
\]
%
By induction hypothesis on $c_1$ with starting memory $m_b$, trace $t'$ and while map $w$, 
let $\config{m_2, \eskip, t_2, w_2}$ be the configuration s.t.
\[
	\config{m_b, c_2, t', w} \to^{*} \config{m_2, \eskip, t_2, w_2} ~(f2)
\]
we know $w_2 = w$.
\\
According to the evaluation $(a)$, $(f1)$ and $(f2)$, we the following evaluation:
\[
	\config{m, \eif([b]^l, c_1, c_2), t, w} \to^{*} \config{m_2, \eskip, t_2, w}
\]
%
, where the resulting while map is still $w$, this sub-case is proved.
%
\caseL{$c_1; c_2$}
%
According to the evaluation rule \rname{seq1}, we have the following evaluation by repeatedly applying \rname{seq1} 
%
\[
\config{m, c_1;c_2, t, w} \to^{*} \config{m_1, [\eskip]^l;c_2,  t_1, w_1} ~ (1)
\]
%
where 
%
\[
	\config{m, c_1, t, w} \to^{*} \config{m_1, [\eskip]^l, t_1, w_1]} ~ (2)
\]
%
By induction hypothesis on $c_1$ and $(2)$, we know $w_1 = w$.
%
According to the evaluation rule \rname{seq2}, we have:
%
\[
\config{m_1, [\eskip]^l;c_2,  t_1, w} \to \config{m_1, c_2, t_1, w} ~ (3)
\]
%
let $\config{m_2, [\eskip]^l, t_2, w_2]}$ be the configuration s.t.,:
%
\[
	\config{m_1, c_2, t_1, w} \to^{*} \config{m_2, [\eskip]^l, t_2, w_2]}~ (4).
\]
%
%
By induction hypothesis on $c_2$, $m_1$, $t_1$, $w$ and $(4)$, we know $w_2 = w$.
%
\\
According to the evaluations $(1)$, $(3)$ and $(4)$, we have:
\[
	\config{m, c_1;c_2, t, w} \to^{*} \config{m', [\eskip]^l, t', w}
\]
%
This case is proved.
%
\caseL{$\ewhile [b]^l \edo c'$}
%
According to the \rname{while-b} and \rname{ifw-b} rule, we have following evaluation:
\[
	\config{m, \ewhile [b]^l \edo c', t, w} \to^{*}
	\config{m, \eif_w([v]^l, c', \eskip), t, w} ~ (w1)
\]
%
where $v$ is either $\etrue$ or $\efalse$.
%
\subcaseL{$v = \etrue$ }
%
By evaluation rule \rname{ifw-true}, we have:
%
\[
	\config{m, \eif_w([\etrue]^l, c';\ewhile [b]^l \edo c', \eskip), t, w} 
	\to^{*} \config{m, c';\ewhile [b]^l \edo c', t, w+l} ~ (w2)
\]
%
By rule \rname{seq1}, we have the following evaluation:
%
\[
	\config{m, c';\ewhile [b]^l \edo c', t, w+l} \to^{*} \config{m', \eskip;\ewhile [b]^l \edo c', t', w'} ~ (w3)
\] 
%
where $\config{m, c', t, w+l} \to^{*} \config{m', \eskip, t', w'} ~ (w4)$. 
%
By induction hypothesis on $c'$ and the evaluation $(w4)$, we know $w' = w + l$. Then we have $\config{m, c', t, w+l} \to^{*} \config{m', \eskip, t', w+l} ~ (w5)$.
%
By the rule \rname{seq2}, we have:
%
\[
	\config{m', \eskip;\ewhile [b]^l \edo c', t', w + l} \to \config{m', \ewhile [b]^l \edo c', t', w + l} ~ (w6)
\]
%
By the termination of the while loop, we know there exist an evaluation by repeating the evaluation in $(w1), (w2), (w3)$ and $(w6)$:
%
\[
	\config{m', \ewhile [b]^l \edo c', t', w + l} \to^{*} \config{m', \eif_w([\efalse]^l, c';\ewhile [b]^l \edo c', \eskip), t'', w''}
\]
%
By the evaluation $(w5)$, we know in the evaluation of each iterations, the resulting while map is always $w + l$. So we have $w'' = w + l$.
%
By evaluation rule \rname{ifw-false}, we have:
%
\[
	\config{m, \eif_w([\efalse]^l, c';\ewhile [b]^l \edo c', \eskip), t'', w + l} 
	\to^{*} \config{m, \eskip, t, (w + l)/l}
\]
%
By the hypothesis that $l \notin Keys(w)$, unfolding the $(w + l)/l$ operation, we have:
%
\[
	w = (w + l)/l.
\]
%
This sub-case is proved.
%
\subcaseL{$v = \efalse$}
%
By evaluation rule \rname{ifw-false}, we have:
%
\[
	\config{m, \eif_w([\efalse]^l, c';\ewhile [b]^l \edo c', \eskip), t, w} 
	\to^{*} \config{m, \eskip, t, w/l}
\]
%
By the hypothesis that $l \notin Keys(w)$, unfolding the $w/l$ operation, we have:
%
\[
	w = w/l.
\]
%
This sub-case is proved.
%
\end{subproof}
}


