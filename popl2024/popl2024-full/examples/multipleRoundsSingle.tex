\begin{example}[Over-Defined Adaptivtiy Example]
    \label{ex:multiRoundsS}
    The program's adaptivity definition in our formal model,
    (in Definition~\ref{def:trace_adapt})
     comes across an over-approximation on capturing the program's intuitive adaptivity rounds.
    It is resulted from the difference between its weight calculation and the \emph{variable may-dependency} definition.
    It occurs when the weight is computed over the traces different from the traces used in 
    witnessing the \emph{variable may-dependency} relation.
    
%%%%%%%%%%%%%%%%%%%%%%%%%%%%%%%%%%%%%% Previous Version For Reference %%%%%%%%%%%%%%%%%%%%%%%%%
    % The program $\kw{multiRoundsS(k)}$ in Figure~\ref{fig:multiRoundsS}(a) demonstrates this over-approximation.
    % It is a variant of the multiple rounds strategy 
    % % we call it a multiple rounds single iteration algorithm, 
    % % named $\kw{multiRoundsS(k)}$ 
    % with input $k$.
    % % as the input variable.
    % In each iteration
    % the query $\query(\chi[y] + p)$ in command $7$ is based on previous query results stored in $p$ and $y$
    % %  is asked by the analyst like in the multiple rounds strategy. 
    % Differ from Example~\ref{ex:multipleRounds},
    % only the query answer from the one single iteration (the $(k - 2)^{th} $ iteration) is 
    % % used to $b$. 
    % used in command $7$.
    % Because the execution trace updates 
    % $p$ using the constant $0$ for all the iterations where ($j \neq k - 2$) at line $10$ after the 
    % query request at line $7$.
    % In this way, all the query answers stored in $p$ will not be accessed in next query request at line $7$ in the iterations 
    % where  ($j \neq k - 2$).
    % Only query answer at one single iteration where ($j = k - 2 $) will be used in next query request
    % $\query(\chi[y] + p)$ at line $7$.
    % So the adaptivity for this example is $2$. 
    % % so the answers from odd iterations do not affect the queries at even iterations. 
    % % However, from the semantics-based dependency graph in Figure~\ref{fig:overappr_example}(b), 
    % However, our adaptivity model fails to realize that there is only dependency relation 
    % between $p^7$ and $p^7$ in one single iteration, 
    % but not in all the others. 
    % % there is no edge from queries at odd iterations (such as $q_1,q_3,q_5$) to queries at even iteration(such as $q_2,q_4$). The longest path is dashed with a length $3$.  
    % As shown in the semantics-based dependency graph in Figure~\ref{fig:multiRoundsS}(b), 
    % there is an edge from $p^7$ to itself representing the existence of \emph{Variable May-Dependency} from $p^7$ on itself,
    % and the visiting times of labeled variable $p^7$ is 
    % $w_{}(\trace_0)$ given the input initial trace $\trace_0$. 
    % % will always execute then branch and even iteration means else branch, so 
    % % % its dependency 
    % % it considers both branches for every iteration. 
    % % In this sense, the weight estimated for $y^6$ and $w^6$ are both 
    % % $k$.
    % As a result, the walk with the longest query length 
    % is
    % $p^7  \to \cdots \to p^7 \to y^4  \to z^1 $ with the vertex $p^7$ visited $w_{}(\trace_0)$,
    % as the dotted arrows. 
    % The adaptivity 
    % % the Program-Based Dependency graph from {\THESYSTEM} by finding 
    % based on
    % this walk
    % % walk with the longest query length 
    % is $2 + w_{}(\trace_0)$, instead of $2$. 
    % % %
    % % T% estimated from the Program-Based Dependency graph from by finding the walk with the longest query length 
    % % is $1 + 2 * k$, instead of $1 + K$.
    % Though the $\THESYSTEM$ is able to give us $2 + k$,  as an accurate bound w.r.t this definition.
%%%%%%%%%%%%%%%%%%%%%%%%%%%%%%%%%%%%%% Previous Version Above %%%%%%%%%%%%%%%%%%%%%%%%%
\jl{
    The program $\kw{multiRoundsS(k)}$ in Figure~\ref{fig:multiRoundsS}(a) demonstrates this over-approximation.
    It is a variant of the multiple rounds strategy with input $k$.
    In each iteration the query $\query(\chi[y] + p)$ in command $7$ is based on previous query results stored in $p$ and $y$
    Differ from Example~\ref{ex:multipleRounds},
    only the query answer from the one iteration, the $(k - 2)^{th}$ one is used in query request
    $\clabel{\assign{p}{\query(\chi[y]+p)} }^{7}$.
    Because the execution will reset
    $p$'s value by the constant $0$ in all the other iterations
    at line $10$ after this query request.
    In this way, all the query answers stored in $p$ is erased and isn't used
    in the query request at next iteration, except the one at the $(k - 2)$th iteration.
    In this sense, the intuitive \emph{adaptivity} rounds for this example is $2$. 
    However, our adaptivity definition fails to realize that there is only dependency relation 
    between $p^7$ and $p^7$ in one single iteration, 
    but not in all the others. 
    As shown in the semantics-based dependency graph in Figure~\ref{fig:multiRoundsS}(b), 
    there is an edge from $p^7$ to itself representing the existence of \emph{Variable May-Dependency} from $p^7$ on itself,
    and the visiting times of labeled variable $p^7$ is 
    $w_{}(\trace_0)$. $w_{}(\trace_0)$ will count the execution times of command $\clabel{\assign{p}{\query(\chi[y]+p)} }^{7}$ during execution, which is expected to be equal to the loop iteration numbers, i.e., initial value of input $k$ from the initial trace $\trace_0$.
    As a result, the walk with the longest query length 
    is
    $p^7  \to \cdots \to p^7 \to y^4  \to z^1 $ with the vertex $p^7$ visited $w_{p^7}(\trace_0)$, as the dotted arrows. 
    The adaptivity based on this walk
    is $2 + w_{p^7}(\trace_0)$, instead of $2$. 
    Though the $\THESYSTEM$ is able to give us $2 + k$,  as an accurate bound w.r.t this definition.
x}
        {\small
        \begin{figure}
            \centering
           %}
           \quad
           \begin{subfigure}{.35\textwidth}
           \begin{centering}
       {\footnotesize
           $ \begin{array}{l}
                   \kw{multiRoundsS(k)} \triangleq \\
                      \clabel{ \assign{j}{0}}^{0} ; 
                       \clabel{\assign{z}{\query(0)} }^{1} ;             
                       \clabel{\assign{p}{0} }^{2} ; \\
                       \eif(\clabel{ k = 0}^{3}, \\
                        \quad \clabel{ \assign{y}{\query(z)}}^{4}, \\
                        \quad \clabel{\eskip}^5);\\
                       \ewhile ~ \clabel{j \neq k}^{6} ~ \edo ~ \Big(
                        \\
                        \quad \clabel{\assign{p}{\query(\chi[y]+p)} }^{7}  ; \\
                        \quad \clabel{\assign{j}{j + 1}}^{8}\\
                        \eif(\clabel{ j \neq k - 2}^{9}, \\
                        \quad \quad \clabel{ \assign{p}{0}}^{10} ,\\ 
                        \quad \quad \clabel{\eskip}^{10})\\
                        \quad \Big);\\
                   \end{array}
           $       
       }
           \caption{}
           \end{centering}
           \end{subfigure}
           \begin{subfigure}{.6\textwidth}
               \begin{centering}
               \begin{tikzpicture}[scale=\textwidth/15cm,samples=200]
           % Variables Initialization
           \draw[] (-5, 2) circle (0pt) node{{ $z^1: {}^{w_{z^1}}_{1}$}};
           \draw[] (-5, 7) circle (0pt) node{{$p^2: {}^{w_{p^2}}_{0}$}};
           \draw[] (-5, 4) circle (0pt) node{{ $y^4: {}^{w_{y^4}}_{1}$}};
           % Variables Inside the Loop
            \draw[] (0, 6) circle (0pt) node{{ $p^7: {}^{w_{p^7}}_{1}$}};
            \draw[] (0, 2) circle (0pt) node{{ $p^{10}: {}^{w_{p^{10}}}_{0}$}};
            % Counter Variables
            \draw[] (5, 6) circle (0pt) node {{$j^0: {}^{w_{j^0}}_{0}$}};
            \draw[] (5, 2) circle (0pt) node {{ $j^8: {}^{w_{j^8}}_{0}$}};
            %
            % Value Dependency Edges:
            \draw[ ultra thick, -Straight Barb, densely dotted,] (0.8, 7) arc (220:-100:1);
            \draw[  -latex] (-1.5, 6)  to  [out=-130,in=130]  (-1.5, 2);
            % Value Dependency Edges on Initial Values:
            \draw[ ultra thick, -latex, densely dotted,] (-5, 3.5)  -- (-5, 2.5) ;
            \draw[  -latex,] (-1.5, 6)  -- (-4, 7) ;
            \draw[  ultra thick, -latex, densely dotted,] (-1.5, 6)  -- (-4, 4.7) ;
            %
            % Value Dependency For Control Variables:
            \draw[  -Straight Barb] (6.5, 2.5) arc (150:-150:1);
            % Control Dependency
            \draw[  -latex] (5, 2.5)  -- (5, 5.5) ;
            \draw[ -latex] (1.5, 6)  -- (3.5, 6) ;
            \draw[ -latex] (1.5, 6)  -- (3.5, 2) ;
            \draw[ -latex] (1.5, 1.8)  -- (3.5, 2) ; 
            % Edges Produced by Transitivity
            \draw[  -latex,] (-1.5, 6)  -- (-4, 2) ;
            \draw[ -latex] (1.5, 1.8)  -- (3.5, 6) ; 
           \end{tikzpicture}
            \caption{}
               \end{centering}
               \end{subfigure}
            \caption{(a) The multi rounds single example
            (b) The semantics-based dependency graph.}
           \label{fig:multiRoundsS}
           \end{figure}
        }
    \end{example}