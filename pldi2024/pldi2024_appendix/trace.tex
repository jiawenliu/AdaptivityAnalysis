%
\subsection{Trace}
%
% An event $\event \in \eventset$ belongs to a trace $\trace$, i.e., $\event \eventin \trace$ are defined as follows:
% %
% \begin{equation}
%   \event \eventin \trace  
%   \triangleq \left\{
%   \begin{array}{ll} 
%     \etrue                  & \trace =  (\trace' \tracecate \event') \land (\event \eventeq \event') \\
%     \event \eventin \trace' & \trace =  (\trace' \tracecate \event') \land (\event \eventneq \event') \\ 
%     \efalse                 & o.w.
%   \end{array}
%   \right.
% \end{equation}
% %
% A well-formed event $\event \in \eventset$ belongs to a trace $\trace$ in signature, 
% i.e., $\event \sigin \trace$ are defined as follows:
%   %
% \begin{equation}
%   \event \sigin \trace  
%   \triangleq \left\{
%   \begin{array}{ll} 
%     \etrue                  & \trace =  (\trace' \tracecate \event') \land (\event \sigeq \event') \\
%     \event \sigin \trace'   & \trace =  (\trace' \tracecate \event') \land (\event \signeq \event') \\ 
%     \efalse                 & o.w.
%   \end{array}
%   \right.
% \end{equation}
%
%
%
%
% \todo{
% \[
% \mbox{Post-Processed Trace} \qquad \trace \qquad ::= \qquad \tracecate | \trace \tracecate \event
% % %
% \]
% Trace appending: $\tracecate: \mathcal{T} \to \eventset \to \mathcal{T}$
% \[
%   \trace \tracecate \event \triangleq
%   \left\{
%   \begin{array}{ll} 
%     {} \tracecate \event           & \trace =  \tracecate \\
%     \trace \tracecate \event    & \trace =  \trace' \tracecate \event\\ 
%   \end{array}
%   \right.
% \]
%
% \mg{I suggest to use more definition environments throughout this section.}
% \mg{Is ++ a constructor or a defined operation? It is in the grammar
%   but it seems that you are actually defining it here.}
%   \jl{It is a defined operator, I mixed them wrongly.}
  \begin{defn}[Trace Concatenation, $\tracecat: \mathcal{T} \to \mathcal{T} \to \mathcal{T}$]
Given two traces $\trace_1, \trace_2 \in \mathcal{T}$, the trace concatenation operator 
$\tracecat$ is defined as:
% \[
  % \trace_1 \tracecat \trace_2 \triangleq
  % \left\{
  % \begin{array}{ll} 
  %   \trace_1 \tracecat [] \triangleq \trace_1 & 
  %   \trace_1 \tracecat (\trace_2' :: \event) \triangleq  (\trace_1  \tracecat \trace_2')  :: \event 
  % \end{array}
  % \right.
% \]
% \[
  % \trace_1 \tracecat \trace_2 \triangleq
  % \left\{
  % \begin{array}{ll} 
  %   \trace_1 \tracecat [] \triangleq \trace_1 & 
  %   \trace_1 \tracecat (\trace_2' :: \event) \triangleq  (\trace_1  \tracecat \trace_2')  :: \event 
  % \end{array}
  % \right.
% \]
\[
  \trace_1 \tracecat \trace_2 \triangleq
  \left\{
  \begin{array}{ll} 
     \trace_1 & \trace_2 = [] \\
     (\trace_1  \tracecat \trace_2')  :: \event & \trace_2 = \trace_2' :: \event
  \end{array}
  \right.
\]
\end{defn}
%
% \todo{ need to consider the occurrence times }
% \\
% \mg{This definition is not well given. You use a different operator to define it t[:e] which you say it is a shorthand. It cannot be a shorthand because it is used in the definition. I think you need to define two operations, either in sequence or mutually recursive.}
% \jl{I moved this definition from the main paper into the appendix \ref{apdx:flowsto_event_soundness}. Because this operator is only being used in the soundness proof. And I also feel it doesn't worth to spend many lines in the main paper for defining this complex notation.}
% Subtrace: $[ : ] : \mathcal{{T} \to \eventset \to \eventset \to \mathcal{T}}$ 
% \wqside{Confusing, I can not understand the subtraction, it takes a trace, and two events, and this operator is used to subtract these two events?}
% \[
%   \trace[\event_1 : \event_2] \triangleq
%   \left\{
%   \begin{array}{ll} 
%   \trace'[\event_1: \event_2]             & \trace = \event :: \trace' \land \event \eventneq \event_1 \\
%   \event_1 :: \trace'[:\event_2]  & \trace = \event :: \trace' \land \event \eventeq \event_1 \\
%   {[]} & \trace = [] \\
%   \end{array}
%   \right.
% \]
% For any trace $\trace$ and two events $\event_1, \event_2 \in \eventset$,
% $\trace[\event_1 : \event_2]$ takes the subtrace of $\trace$ starting with $\event_1$ and ending with $\event_2$ including $\event_1$ and $\event_2$.
% \\
% We use $\trace[:\event_2] $ as the shorthand of subtrace starting from head and ending with $\event_2$, and similary for $\trace[\event_1:]$.
% \[
%   \trace[:\event] \triangleq
%   \left\{
%   \begin{array}{ll} 
%  \event' :: \trace'[: \event]             & \trace = \event' :: \trace' \land \event' \eventneq \event \\
%   \event'  & \trace = \event' :: \trace' \land \event' \eventeq \event \\
%   {[]}  & \trace = [] 
%   \end{array}
%   \right.
% % \]
% % \[
%   \quad
%   \trace[\event: ] \triangleq
%   \left\{
%   \begin{array}{ll} 
%   \trace'[\event: ]     & \trace =  \event' :: \trace' \land \event \eventneq \event' \\
%   \event' :: \trace'  & \trace = \event' :: \trace' \land \event \eventeq \event' \\
%   {[ ] } & \trace = []
%   \end{array}
%   \right.
% \]
% %
% \mg{why in the next definition you use ( ) while in the previous ones you didn't? They seem like the same cases. And why you use o.w. instead of []?}
% An event $\event \in \eventset$ belongs to a trace $\trace$, i.e., $\event \eventin \trace$ are defined as follows:
% %
% \begin{equation}
%   \event \eventin \trace  
%   \triangleq \left\{
%   \begin{array}{ll} 
%     \etrue                  & \trace =  (\event' :: \trace') \land (\event \eventeq \event')
%                               \\
%     \event \eventin \trace' & \trace =  (\event' :: \trace') \land (\event \eventneq \event') \\ 
%     \efalse                 & o.w.
%   \end{array}
%   \right.
% \end{equation}
\begin{defn}(An Event Belongs to A Trace)
  An event $\event \in \eventset$ belongs to a trace $\trace$, i.e., $\event \in \trace$ are defined as follows:
%
\begin{equation}
  \event \in \trace  
  \triangleq \left\{
  \begin{array}{ll} 
    \etrue                  & \trace =  \trace' :: \event'
     \land \event = \event'
                              \\
    \event \in \trace' & \trace =  \trace' :: \event'
    \land \event \neq \event' \\ 
    \efalse                 & \trace = []
  \end{array}
  \right.
\end{equation}
As usual, we denote by $\event \notin \trace$ that the event $\event$ doesn't belong to the trace $\trace$.
\end{defn}
%
% An event $\event \in \eventset$ belongs to a trace $\trace$ up to value, 
% i.e., $\event \sigin \trace$ are defined as follows:
%   %
% \begin{equation}
%   \event \sigin \trace  
%   \triangleq \left\{
%   \begin{array}{ll} 
%     \etrue                  & \trace =  (\trace' \tracecate \event')                          \land \pi_1(\event_1) = \pi_1(\event_2) 
%                               \land  \pi_2(\event_1) = \pi_2(\event_2)  
%                               % \land \vcounter(\trace \event) = \vcounter()
%                               \\
%     \event \sigin \trace'   & \trace =  (\trace' \tracecate \event') 
%                               \land 
%                               (\pi_1(\event_1) \neq \pi_1(\event_2) 
%                               \lor  \pi_2(\event_1) \neq \pi_2(\event_2)) 
%                               \\ 
%     \efalse                 & o.w.
%   \end{array}
%   \right.
% \end{equation}
%
% % \mg{Why the previous definition used :: and now you switch to ++? Cannot this just be defined using ::? I am trying to anticipate places where a reader might be confused. Also, this definition would be much simpler if we defined event in a more uniform way. Ideally, we want to distinguish three cases, we don't need to distinguish 7 cases.}\\
% % \jl{My bad, I was really too sticky to the convention.
% I though the list appending  $::$ can only append element on the left side.}
% \mg{We introduce a counting operator $\vcounter : \mathcal{T} \to \mathbb{N} \to \mathbb{N}$ whose behavior is defined as follows, \sout{Counter $\vcounter : \mathcal{T} \to \mathbb{N} \to \mathbb{N}$ }}.
% \wq{The operator counter actually provides the number of times a specific label appears in a trace. Only a number, the position of label is ignored.}
% \[
% \begin{array}{lll}
% \vcounter((x, l, v) :: \trace ) l \triangleq \vcounter(\trace) l + 1
% &
% \vcounter((b, l, v):: \trace ) l \triangleq \vcounter(\trace) l + 1
% &
% \vcounter((x, l, \qval, v):: \trace ) l \triangleq \vcounter(\trace) l + 1
% \\
% \vcounter((x, l', v):: \trace ) l \triangleq \vcounter(\trace ) l
% &
% \vcounter((b, l', v):: \trace ) l \triangleq \vcounter(\trace ) l
% &
% \vcounter((x, l', \qval, v):: \trace ) l \triangleq \vcounter(\trace ) l
% \\
% \vcounter({[]}) l \triangleq 0
% &&
% \end{array}
% \]
% \[
% \begin{array}{lll}
% \vcounter(\trace  \tracecat [(x, l, v)] ) l \triangleq \vcounter(\trace) l + 1
% &
% \vcounter(\trace  \tracecat [(b, l, v)] ) l \triangleq \vcounter(\trace) l + 1
% &
% \vcounter(\trace  \tracecat [(x, l, \qval, v)] ) l \triangleq \vcounter(\trace) l + 1
% \\
% \vcounter(\trace  \tracecat [(x, l', v)] ) l \triangleq \vcounter(\trace ) l
% &
% \vcounter(\trace  \tracecat [(b, l', v)] ) l \triangleq \vcounter(\trace ) l
% &
% \vcounter(\trace  \tracecat [(x, l', \qval, v)]) l \triangleq \vcounter(\trace ) l
% \\
% \vcounter({[]}) l \triangleq 0
% &&
% \end{array}
% \]
We introduce a counting operator $\vcounter : \mathcal{T} \to \mathbb{N} \to \mathbb{N}$ whose behavior is defined as follows,
% \[
% \begin{array}{lll}
% \vcounter(\trace :: (x, l, v, \bullet) ) l \triangleq \vcounter(\trace) l + 1
% &
% \vcounter(\trace  ::(b, l, v, \bullet) ) l \triangleq \vcounter(\trace) l + 1
% &
% \vcounter(\trace  :: (x, l, v, \qval) ) l \triangleq \vcounter(\trace) l + 1
% \\
% \vcounter(\trace  :: (x, l', v, \bullet) ) l \triangleq \vcounter(\trace ) l, l' \neq l
% &
% \vcounter(\trace  :: (b, l', v, \bullet) ) l \triangleq \vcounter(\trace ) l, l' \neq l
% &
% \vcounter(\trace  :: (x, l', v, \qval)) l \triangleq \vcounter(\trace ) l, l' \neq l
% \\
% \vcounter({[]}) l \triangleq 0
% &&
% \end{array}
% \]
\[
\begin{array}{ll}
\vcounter(\trace :: (x, l, v, \bullet), l ) \triangleq \vcounter(\trace, l) + 1
&
\vcounter(\trace  ::(b, l, v, \bullet), l) \triangleq \vcounter(\trace, l) + 1
\\
\vcounter(\trace  :: (x, l, v, \qval), l) \triangleq \vcounter(\trace, l) + 1
&
\vcounter(\trace  :: (x, l', v, \bullet), l) \triangleq \vcounter(\trace, l), l' \neq l
\\
\vcounter(\trace  :: (b, l', v, \bullet), l) \triangleq \vcounter(\trace, l), l' \neq l
&
\vcounter(\trace  :: (x, l', v, \qval), l) \triangleq \vcounter(\trace, l), l' \neq l
\\
\vcounter({[]}, l) \triangleq 0
&
\end{array}
\]
%
% The Latest Label $\llabel : \mathcal{T} \to \mathcal{VAR} \to \mathbb{N}$ 
% The label of the latest assignment event which assigns value to variable $x$.
% \[
%   \begin{array}{lll}
% \llabel((x, l, v):: \trace) x \triangleq l
% &
% \llabel((b, l, v)):: \trace x \triangleq \llabel(\trace) x
% &
% \llabel((x, l, \qval, v):: \trace) x \triangleq l
% \\
% \llabel((y, l, v):: \trace) x \triangleq \llabel(\trace ) x
% &
% \llabel((y, l, \qval, v):: \trace) x \triangleq \llabel(\trace ) x
% \\
% \llabel({[]}) x \triangleq \bot
% &&
% \end{array}
% \]
%
% \todo{wording}
% \mg{This wording needs to be fixed. Also notice that the type is wrong, a label is not always returned.}
%  The Latest Label $\llabel : \mathcal{T} \to \mathcal{VAR} \to \mathbb{N}$ 
% The label of the latest assignment event which assigns value to variable $x$.
% \[
%   \begin{array}{lll}
% \llabel(\trace  \tracecat [(x, l, v)]) x \triangleq l
% &
% \llabel(\trace  \tracecat [(b, l, v)]) x \triangleq \llabel(\trace) x
% &
% \llabel(\trace  \tracecat [(x, l, \qval, v)]) x \triangleq l
% \\
% \llabel(\trace  \tracecat [(y, l, v)]) x \triangleq \llabel(\trace ) x
% &
% \llabel(\trace  \tracecat [(y, l, \qval, v)]) x \triangleq \llabel(\trace ) x
% \\
% \llabel({[]}) x \triangleq \bot
% &&
% \end{array}
% \]
We introduce an operator $\llabel : \mathcal{T} \to \mathcal{VAR} \to \ldom \cup \{\bot\}$, which 
takes a trace and a variable and returns the label of the latest assignment event which assigns value to that variable.
Its behavior is defined as follows,
% \begin{defn}[Latest Label]
  \[
    % \begin{array}{lll}
  \llabel(\trace  :: (x, l, \_, \_)) x \triangleq l
  ~~~
  \llabel(\trace  :: (y, l, \_, \_)) x \triangleq \llabel(\trace ) x, y \neq x
  % &
  ~~~
  \llabel(\trace :: (b, l, v, \bullet)) x \triangleq \llabel(\trace) x
  % &
  % \\
  % \llabel(\trace  :: (y, l, v, \bullet)) x \triangleq \llabel(\trace ) x
  % &
  % \llabel(\trace :: (y, l, v, \qval)) x \triangleq \llabel(\trace ) x
  % &
  ~~~
  \llabel({[]}) x \triangleq \bot
  % \end{array}
  \]
% \end{defn}
%
% \mg{This wording needs to be fixed but also the description does not make sense. This operator seems to just collect all the labels in a trace. Again, this definition would be shorter with a more uniform definition of events.}
% The Trace Label Set $\tlabel : \mathcal{T} \to \mathcal{P}{(\mathbb{N})}$ 
% The label of the latest assignment event which assigns value to variable $x$.
% \[
%   \begin{array}{llll}
% \tlabel_{(\trace  \tracecat [(x, l, v)])} \triangleq \{l\} \cup \tlabel_{(\trace )}
% &
% \tlabel_{(\trace  \tracecat [(b, l, v)])} \triangleq \{l\} \cup \tlabel_{(\trace)}
% &
% \tlabel_{(\trace  \tracecat [(x, l, \qval, v)])} \triangleq \{l\} \cup \tlabel_{(\trace)}
% &
% \tlabel_{[]} \triangleq \{\}
% \end{array}
% \]
% \begin{defn}
  The operator $\tlabel : \mathcal{T} \to \mathcal{P}{(\ldom)}$ gives the set of labels in every event belonging to 
  a trace, whoes behavior is defined as follows,
\[
  % \begin{array}{llll}
\tlabel{(\trace  :: (\_, l, \_, \_))} \triangleq \{l\} \cup \tlabel{(\trace )}
~~~
\tlabel({[ ]}) \triangleq \{\}
% \end{array}
\]

If we observe the operational semantics rules, we can find that no rule will shrink the trace. 
So we have the Lemma~\ref{lem:tracenondec} with proof in Appendix~\ref{apdx:tracenondec}, 
specifically the trace has the property that its length never decreases during the program execution.
\begin{lem}
[Trace Non-Decreasing]
\label{lem:tracenondec}
For every program $c \in \cdom$ and traces $\trace, \trace' \in \mathcal{T}$, if 
$\config{c, \trace} \rightarrow^{*} \config{\eskip, \trace'}$,
then there exists a trace $\trace'' \in \mathcal{T}$ with $\trace \tracecat \trace'' = \trace'$
%
$$
\forall \trace, \trace' \in \mathcal{T}, c \st
\config{c, \trace} \rightarrow^{*} \config{\eskip, \trace'} 
\implies \exists \trace'' \in \mathcal{T} \st \trace \tracecat \trace'' = \trace'
$$
\end{lem}

Since the equivalence over two events is defined over the query value equivalence, 
when there is an event belonging to a trace, 
if this event is a query assignment event, 
it is possible that 
the event showing up in this trace has a different form of query value, 
but they are equivalent by Definition~\ref{def:query_equal}.
So we have the following Corollary~\ref{coro:aqintrace} with proof in Appendix~\ref{apdx:aqintrace}.
\begin{coro}
\label{coro:aqintrace}
For every event and a trace $\trace \in \mathcal{T}$,
if $\event \in \trace$, 
then there exist another event $\event' \in \eventset$ and traces $\trace_1, \trace_2 \in \mathcal{T}$
such that $\trace_1 \tracecat [\event'] \tracecat \trace_2 = \trace $
with 
$\event$ and $\event'$ equivalent but may differ in their query value.
\[
  \forall \event \in \eventset, \trace \in \mathcal{T} \st
\event \in \trace \implies \exists \trace_1, \trace_2 \in \mathcal{T}, 
\event' \in \eventset \st (\event \in \event') \land \trace_1 \tracecat [\event'] \tracecat \trace_2 = \trace  
\]
\end{coro}