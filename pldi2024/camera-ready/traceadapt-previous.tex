 In this section, we formally present the definition of adaptivity of a program, which is the length of the 'longest' walk with most queries involved in the execution-based dependency graph of this program. We first present the construction of the execution-based dependency graph before the introduction of the formal definition of adaptivity. 

\subsection{Execution-based Dependency Graph}
\label{sec:design_choice}
%%%%%%%%%%%%%%%%%%%%%%%%%%%%%%%%%%% The Detailed Version in Introducing The Dependency Graph %%%%%%%%%%%%%%%%%%%%%%%%%%%%%%%%%%%%%%%%%%%%%%%%%
% we formally define the execution-based dependency graph as follows. There are some notations used in the definition. The labeled variables of a program $c$  
% is a subset of the labeled  variables $\mathcal{LV}$, denoted by $\lvar(c) \in \mathcal{P}(\mathcal{VAR} \times \mathcal{L}) \subseteq \mathcal{LV}$.
% \wq{I think LV(c) means all the labeled assigned variables, because in Fig.3.b, j2 is not in the graph so j2 is not in LV(tworounds), please verify. If so, maybe LV(c) is not a good name, people may think it means all the label variables, instead of just assigned ones.}
% The set of query-associated variables (in query request assignments) for a program $c$ is denoted as $\qvar(c)$, where $\qvar: \cdom \to 
% \mathcal{P}(\mathcal{LV})$. The set of initial traces of a program $c$ is a subset of the  trace universe $\trace$, in which every initial trace contains the value for all the input variables of $c$. For instance, the initial trace $\trace_0$ contains the value of input variable $k$ in the $\kw{twoRounds(k)}$ example.
\jl{
The \emph{execution-based dependency graph} is formally defined in Definition~\ref{def:trace_graph}. 
The labeled variables of a program $c$,
$\lvar(c) \subseteq \mathcal{LV}$ contains all the variables in $c$'s assignment commands, with the command labels as superscripts. 
The set of query-associated variables (in query request assignments),
$\qvar(c) \subseteq \lvar(c)$ contains all labeled variables in $c$'s query requests. 
The set of initial traces of a program $c$,
$\mathcal{T}_0(c) \subseteq \mathcal{T}$
contains all possible initial trace of $c$.
Each initial trace,  $\trace_0 \in \mathcal{T}_0(c)$ contains the initial value of every input variable of $c$. 
For instance, the initial trace $\trace_0$ contains the value of input variable $k$ in the $\kw{twoRounds(k)}$ example.
}
\begin{defn}[Execution-based Dependency Graph]
\label{def:trace_graph}
Given a program ${c}$,
its \emph{execution-based dependency graph} 
$\traceG({c}) = (\traceV({c}), \traceE({c}), \traceW({c}), \traceF({c}))$ is defined as follows,
{\small
\[
\begin{array}{lll}
  \text{Vertices} &
  \traceV({c}) & := \left\{ 
  x^l
  ~ \middle\vert ~ x^l \in \lvar(c)
  \right\}
  \\
  \text{Directed Edges} &
  \traceE({c}) & := 
  \left\{ 
  (x^i, y^j) 
  ~ \middle\vert ~
  x^i, y^j \in \lvar(c) \land \vardep(x^i, y^j, c) 
  \right\}
  \\
  \text{Weights} &
  \traceW({c}) & := 
  \{ 
  (x^l, w) 
  ~ \vert ~ 
  w : \mathcal{T}_0(c) \to \mathbb{N}
  \land
  x^l \in \lvar(c) 
  \\ & &
  \quad \land
  \forall \vtrace_0 \in \mathcal{T}_0(c), \trace' \in \mathcal{T} \st \config{{c}, \trace_0} \to^{*} \config{\eskip, \trace_0\tracecat\vtrace'} 
  \implies w(\trace_0) = \vcounter(\vtrace', l) \}
  \\
  \text{Query Annotations} &
  \traceF({c}) & := 
\left\{(x^l, n)  
~ \middle\vert ~
 x^l \in \lvar(c) \land
n = 1 \Leftrightarrow x^l \in \qvar(c) \land n = 0 \Leftrightarrow  x^l \notin \qvar(c)
\right\}
\end{array},
\]
}
%%%%%%%%%%%%%%%%%%%%%%%%%%%%%%%%%%% The Detailed Version in Explaining the Trace Operators %%%%%%%%%%%%%%%%%%%%%%%%%%%%%%%%%%%%%%%%%%%%%%%%%
% There are some operators: the trace concatenation operator $\tracecat: \mathcal{T} \to \mathcal{T} \to \mathcal{T}$, combines two traces; the counting operator $\vcounter : \mathcal{T} \to \mathbb{N} \to \mathbb{N}$, 
% which counts the occurrence of of a labeled variable in the trace. The full definitions of these above operators can be found in the appendix.
% \\
\jl{where $\tracecat: \mathcal{T} \to \mathcal{T} \to \mathcal{T}$ is the trace concatenation operator, which combines two traces, 
and $\vcounter : \mathcal{T} \to \mathbb{N} \to \mathbb{N}$ is the counting operator, 
which counts the occurrence of of a labeled variable in the trace. All the definition details are in the appendix.}
\end{defn}


%%%%%%%%%%%%%%%%%%%%%%%%%%%%%%%%%%% The Explanation of Dependency Graph %%%%%%%%%%%%%%%%%%%%%%%%%%%%%%%%%%%%%%%%%%%%%%%%%
\wq{ The occurrence time is computed by the counter operator $w(\trace) = \vcounter(\vtrace', l)$. As an instance, in the execution-based dependency graph of $\kw{twoRounds}$ in Figure~\ref{fig:overview-example}(b), the weight of vertices in the while loop $w_k$ is a function which returns the value $n$ of variable $k$ in the starting trace $\tau$, and the commands in the loop are indeed executed $n$ times when starting with $\trace$.}
Every vertex also has a query annotation, mapping each $x^l \in \traceV(c)$ to $0$ or $1$, 
indicating whether the vertex comes from a query request assignment (1) or not (0) by checking if the labeled variable $x^l$ of the vertex is in $\qvar(c)$.
One interesting point is that instead of building a graph whose vertices are just variables coming from query request assignments, we choose the combination of all the labeled assigned variables of the program and the query annotations. The reason is that the results of previous queries can be stored or used in variables
which aren't associated to the query request,
it is necessary to track the dependency between all the assigned variables of the program. 
The main novelty is the combination of the quantitative and dependency information on the execution-based dependency graph, which means this graph can tell both the dependency between queries and the times they depend on each other.
The vertices $\traceV({c})$ of a program $c$ are the labeled assigned variables, which are statically collected. The weight function on every vertex $w : \mathcal{T} \to \mathbb{N}$
mapping from a starting trace $\vtrace \in \mathcal{T}_0(c)$ to a natural number, tracks the occurrence times of this vertex in the newly generated trace $\trace'$ when the program $c$ is evaluated from the starting trace $\config{{c}, \trace} \to^{*} \config{\eskip, \trace\tracecat\vtrace'} $.



\wq{we think the query $\query(\chi[2])$ (assigned variable $y^3$) may depend on the query $\query(\chi[1]$ (assigned variable $x^1$). 
    but vulnerable to queries request protected by differential privacy mechanisms. In our loop language, a query $q(e)$ represents a query request to the database through a mechanism, which add random noise to protect the return results. In this setting, the results of one query will be randomized due to the noise attached by the mechanism which fails the first candidate because witnessing the results of one query can no longer tells whether the change of the results comes from another query or the change of noise of the differential privacy mechanism. For example, suppose we have a program $p$ which requests two simple queries $q_1()$ and $q_2()$ with no arguments as follows.
  \[
  c_1 =\assign{x}{\query(\chi[2])} ;\assign{y}{\query(\chi[3] + x)}.
  \]
 $ c = \assign{x}{\query(\chi[1])} ; \assign{y}{\query(\chi[2])}$,
 and 
Specifically, in the {\tt Query While} language, the query request is composed by two components: a symbol $\query$ representing a linear query type and 
% an argument
the query expression $\qexpr$ as an argument, 
which represents the function specifying what the query asks. 
From our perspective, $\query(\chi[1])$ is different from $\query(\chi[2]))$. Informally,  
in this example: $c_1 = \assign{x}{\query(0)}; \assign{z}{\query(\chi[x])}$.
This candidate definition works well 
Nevertheless, the first definition fails to catch control dependency because it just monitors the changes to a query, but misses the appearance of the query when the answers of its previous queries change. For instance, it fails to handle 
}$
      c_2 = \assign{x}{\query(\chi[1])} ; \eif( x > 2 , \assign{y}{\query(\chi[2])}, \eskip )
  $, but the second definition can. However, it only considers the control dependency and misses the data dependency. This reminds us to define a \emph{may-dependency} relation between labeled variables by combining the two definitions to capture the two situations.
 $ p = \assign{x}{\query(\chi[1])} ; \assign{y}{\query(\chi[2])}$. 
This candidate definition works well with respect to data dependency. However, if fails to handle control dependency since it just monitors the changes to the answer of a query when the answer of previous queries returned change. The key point is that this query may also not be asked because of an analyst decision which depend on the answers of previous queries. An example of this situation is shown in program $p_1$ as follows.
There are two possible situations that a query will be "affected" by previous queries' results,  
either when the query expression directly uses the results of previous queries (data dependency), or when the control flow of the program with respect to a query (whether to ask this query or not) depends on the results of previous queries (control flow dependency). To this end, our assigned variable dependency definition has the following two cases.   
{
\begin{enumerate}
    \item One variable may depend on a previous variable if and only if a change of the value assigned to the previous variable may also change the value assigned to the variable.
    \item One variable may depend on a previous variable if and only if a change of the value assigned to the previous variable may also change the appearance of the assignment command to this variable 
    % in\wq{during?} 
    during execution.
\end{enumerate}
}
 The first case captures the data dependency. 
For instance, in a simple program $c_1 =[\assign{x}{\query(\chi[2])}]^1 ;[\assign{y}{\query(\chi[3] + x)}]^2$, we think $\query(\chi[3] + x)$ (variable $y^2$) may depend on the query $\query(\chi[2]))$ (variable $x^1$), because the equipped function of the former $\chi[3] + x$ may depend on the data stored in x assigned with the result of $\query(\chi[2]))$. From our perspective, $\query(\chi[1])$ is different from $\query(\chi[2]))$. The second case captures the control dependency, for instance, in the program $
      c_2 = [\assign{x}{\query(\chi[1])}]^1 ; \eif( [x > 2]^2 , [\assign{y}{\query(\chi[2])}]^3, [\eskip]^4 )
$

  
\jl{The may-dependency between the labeled assigned variables is defined below.}

\begin{defn}[Variable May-Dependency]
  \label{def:var_dep}
  A variable ${x}_2^{l_2} \in \lvar(c)$ is in the \emph{variable may-dependency} relation with another
  variable ${x}_1^{l_1} \in \lvar(c)$ in a program ${c}$, denoted as 
  %
  $\vardep({x}_1^{l_1}, {x}_2^{l_2}, {c})$, if and only if.
\[
  \begin{array}{l}
\exists \event_1, \event_2 \in \eventset^{\asn}, \trace \in \mathcal{T} , D \in \dbdom \st
\pi_{1}{(\event_1)}^{\pi_{2}{(\event_1)}} = {x}_1^{l_1}
\land
\pi_{1}{(\event_2)}^{\pi_{2}{(\event_2)}} = {x}_2^{l_2}% \\ \quad 
\land 
\eventdep(\event_1, \event_2, \trace, c, D) 
  \end{array}
\]  %
  \end{defn}
 \jl{second Pass}

  \wq{I think we miss something here. $\tau = \event_1 ++ ... ++\event_2$}?
\jl{nope, it is $[\event] \tracecat [\event_2]$} \wq{I do think, it is better to have $\event_1 ++ \tau ++ \event_2$, which matches Def 3.}
\jl{Def 3 is also $[\event] \tracecat \trace \tracecat [\event_2]$ rather than $\event_1 ++ \tau ++ \event_2$}

We use $\eventset^{\asn}$ for the assignment event set, $\eventset^{\test}$ for the test event set, $\dbdom $ for the domain of the hidden database. The projection operators $\pi_1,\pi_2,\pi_3$ and $\pi_4$ project the corresponding element of the quadruple of an event. From the definition, a labeled assigned variables $x_2^{l_2}$ may depend on another labeled assigned variable $x_1^{l_1}$ in a program $c$ when the given database is $D$, and there exist two assignment events $\event_1$ and $\event_2$ with these two labeled variables which satisfy the event may-dependency relation under a corresponding trace $\trace$.  We use the event may dependency relation between two assignment events to show the may dependency between the corresponding assigned variables.
The aforementioned two dependency cases are reflected in our event may dependency relation as follows.

\begin{defn}[Assignment Event May-Dependency].
\label{def:event_dep}
\\ 
  An \wq{assignment} event $\event_2$ is in the \emph{event may-dependency} relation with an assignment
  event $\event_1 \in \eventset^{\asn}$ in a program ${c}$
  with a hidden database $D$ and a trace $\trace \in \mathcal{T}$ denoted as 
  %
  $\eventdep(\event_1, \event_2, [\event_1 ] \tracecat \trace \tracecat [\event_2], c, D)$, iff
  %
  \[
    \begin{array}{l}
  \exists \vtrace_0,
  \vtrace_1, \vtrace' \in \mathcal{T},\event_1' \in \eventset^{\asn}, {c}_1, {c}_2  \in \cdom  \st
  \diff(\event_1, \event_1') \land 
      \\ \quad
      (
        \exists  \event_2' \in \eventset \st 
    \left(
    \begin{array}{ll}   
   & \config{{c}, \vtrace_0} \rightarrow^{*} 
  \config{{c}_1, \vtrace_1 \tracecat [\event_1]}  \rightarrow^{*} 
    \config{{c}_2,  \vtrace_1 \tracecat [\event_1] \tracecat \vtrace \tracecat [\event_2] } 
    % 
   \\ 
   \bigwedge &
    \config{{c}_1, \vtrace_1 \tracecat [\event_1']}  \rightarrow^{*} 
    \config{{c}_2,  \vtrace_1 \tracecat[ \event_1'] \tracecat \vtrace' \tracecat [\event_2'] } 
  \\
  \bigwedge & 
  \diff(\event_2,\event_2' ) \land 
  \vcounter(\vtrace, \pi_2(\event_2))
  = 
  \vcounter(\vtrace', \pi_2(\event_2'))\\
  \end{array}
  \right)
  \\ \quad
  \lor 
  \exists \vtrace_3, \vtrace_3'  \in \mathcal{T}, \event_b \in \eventset^{\test} \st 
  \\ \quad
  \left(
  \begin{array}{ll}   
    & \config{{c}, \vtrace_0} \rightarrow^{*} 
      \config{{c}_1, \vtrace_1 \tracecat [\event_1]}  \rightarrow^{*} 
      \config{c_2,  \vtrace_1 \tracecat [\event_1] \tracecat \trace \tracecat [\event_b] \tracecat  \trace_3} 
    \\ 
    \bigwedge &
    \config{{c}_1, \vtrace_1 \tracecat [\event_1']}  \rightarrow^{*} 
    \config{c_2,  \vtrace_1 \tracecat [\event_1'] \tracecat \trace' \tracecat [(\neg \event_b)] \tracecat \trace_3'} 
    \\
    \bigwedge &  \tlabel_{\trace_3} \cap \tlabel_{\trace_3'} = \emptyset
     \land \vcounter(\trace', \pi_2(\event_b)) = \vcounter(\trace, \pi_2(\event_b)) 
    %   \land \event_2 \eventin \trace_3
    % \land \event_2 \not\eventin \trace_3'
    \land \event_2 \in \trace_3
    \land \event_2 \not\in \trace_3'
  \end{array}
  \right)
  )
\end{array}
   \]
% , where ${\tt label}(\event_2) = \pi_2(\event_2)$.
  %  
%
\end{defn}

The event universe is denoted as $\eventset$.
The belongs to operator $\in : \eventset \to \mathcal{T} \to \{\etrue, \efalse \} $ and its opposite $\not\in$
express whether or not an event belongs to a trace.
We have the operator $\tlabel : \mathcal{T} \to \ldom$, which gives the set of labels in every event belonging to the trace. We compare two assignment events with the same labeled variables by defining $\diff(\event_1, \event_2)$ below. 
 
 \begin{defn}[Events Different up to Value ($\diff$)]
 \label{def:diff}
  Two events $\event_1, \event_2 \in \eventset$ are  \emph{Different up to Value}, 
  denoted as $\diff(\event_1, \event_2)$ if and only if:
  \[
    \begin{array}{l}
  \pi_1(\event_1) = \pi_1(\event_2) 
  \land  
  \pi_2(\event_1) = \pi_2(\event_2) \\
  \land  
  \big(
    (\pi_3(\event_1) \neq \pi_3(\event_2)
  \land 
  \pi_{4}(\event_1) =_q \pi_{4}(\event_2) = \bullet )
  \lor 
  (\pi_4(\event_1) \neq \bullet
  \land 
  \pi_4(\event_2) \neq \bullet
  \land 
  \pi_{4}(\event_1) \neq_q \pi_{4}(\event_2)) 
  \big)
  \end{array}
  \]
  \end{defn}
We use $\qexpr_1 =_{q} \qexpr_2$ and $\qexpr_1 \neq_{q} \qexpr_2$ to notate query expression equivalence and in-equivalence, distinct from standard equality. First of all, $\diff(\event_1, \event_2)$ only compares the 'value' assigned to the same labeled variable so the two events share the same labeled variable (the first two elements of the event quadruple). If the labeled variable are not the same, it is trivially false. Next, the 'value' has different meaning according to the type of the assignment. If the two assignment events both are generated from a standard assignment command (by checking whether the fourth element of the event is $\bullet$), we think they are different up to their value if the value assigned to the variable (the third element of the event) varies. On the other hand, if they both come from query request assignment commands, we directly compare the query function (the fourth element of the event). We do not consider the case when one event is generated from the standard assignment and the other comes from the query request assignment, which will not hold. The motivation of defining  $\diff(\event_1, \event_2)$ is that it allows us to check whether the value assigned to a specific variable is changed or not in two runs, which is very important in our dependency case 1.   


\wq{Need to pass the following, not yet}
Our event \emph{may-dependency} relation of 
two events $\event_1 \in \eventset^{\asn}$ and $\event_2 \in \eventset$, 
for a program $c$ and hidden database $D$ is w.r.t to
a trace $[\event_1 ] \tracecat \trace \tracecat [\event_2]$.
The $\event_1 \in \eventset^{\asn}$ is an assignment event because only a change on an assignment event will affect the execution trace, according to our operational semantics.
In order to observe the changes of $\event_2$ under the modification of $\event_1$, this trace 
$[\event_1 ] \tracecat \trace \tracecat [\event_2]$
starts with $\event_1$ and ends with $\event_2$.
% }
{The \emph{may-dependency} relation considers both the value dependency and value control dependency as discussed in Section~\ref{sec:design_choice}. The relation can be divided into two parts naturally in Definition~\ref{def:event_dep} (line $2-4$, $5-8$ respectively, starting from line $1$). The idea of the event $\event_1$ may depend on $\event_2$ can be briefly described:
we have one execution of the program as reference (See line $2$ and $6$, for the two kinds of dependency). 
When the value assigned to the 
% first variable 
first variable in $\event_1$ is modified, the reference trace $\trace_1 \tracecat [\event_1]$ is modified correspondingly to $\trace_1 \tracecat [\event_1']$.
We use $\diff(\event_1, \event_1')$ at line $1$ to express this modification, which guarantees that $\event_1$ and $\event_1'$ only differ in their assigned values and are equal on variable name and label. We perform a second run of the program by continuing the execution of the same program from the same execution point, 
but with the modified trace $\trace_1 \tracecat [\event_1']$ (See line $3$, $7$). 
The expected may dependency will be caught by observing two different possible changes (See line $4, 8$ respectively) when comparing the second execution with the reference one (similar definitions as in \cite{Cousot19a}). 

% \wq{
% In the first situation, we are witnessing 
In the first part (line $2-4$ of Definition~\ref{def:event_dep}), we witness
% that the value assigned to the second variable in $\event_2$
the appearance of $\event_2'$ in the second execution, and
% a variation in $\event_2$, which changes into $\event_2'$.
a variation between $\event_2$ and $\event_2'$ on their values.
% changes in $\event_2'$.
% \jl{
We have special requirement $\diff(\event_2, \event_2')$, which guarantees that they
have the same variable name and label but only differ 
% % in their assigned value. 
in their evaluated values.
% assigned to the same variable. 
In particularly for queries, if $\event_2$ and $\event_2'$ are 
% query assignment events, then 
generated from query requesting, then $\diff(\event_2, \event_2')$ guarantees that
they differ in their query values rather than the 
% query requesting value. 
query requesting results. 
Additionally, in order to handle multiple occurrences of the same event through iterations of the while loop,
 where  $\event_2$ and $\event_2'$ could be 
in different while loops,
we restrict the same occurrence of $\event_2$'s label in $\trace$ from the first execution with  the occurrence of $\event_2'$'s label in $\trace'$ from the second execution,
through $\vcounter(\vtrace, \pi_2(\event_2))
= 
\vcounter(\vtrace', \pi_2(\event_2'))$ at line $4$.
% }
% }

% \wq{
In the second part (line $5-8$ of Definition~\ref{def:event_dep}), we 
% are witnessing 
witness
the disappearance of $\event_2$ through observing the change of a testing event $\event_b$.
% In order to change the appearance of 
% % and event, the command that generating $\event_2$ must not be executed in 
% 5yhan event, 
To witness
the disappearance, the command that generates $\event_2$ must not be executed in 
the second execution. 
The only way to control whether a command will be executed, is through the change of a guard's 
evaluation result in an if or while command, which generates a testing event $\event_b$ in the first place.
So we observe when
$\event_b$ changes into $\neg \event_b$ in the second execution firstly, 
whether it follows with the disappearance of $\event_2$ in the second trace. We restrict the occurrence of $\event_b$'s label in the two traces being the same
}
% s to the occurrence times of $\event_2'$'s label in the second trace,
through $\vcounter(\trace', \pi_2(\event_b)) = \vcounter(\trace, \pi_2(\event_b))$ to handle the while loop.
% changes in $\event_2'$, have the same variable and label and only differ in their assigned value. 
Again, for queries, we observe the disappearance based on the query value equivalence.
% {%
%
%
\subsection{Events Dependency}
%
%
\begin{defn}
[Value Dependency of Events]
\label{def:event_valdep}.
\\
An event $\event_2$ is in the \emph{value may-dependency} relation with an assignment
event $\event_1 \in \eventset^{\asn}$ in a program $\ssa{c}$
with hidden database $D$, denoted as 
%
$\eventdep^{val}(\event_1, \event_2, c, D)$, if and only if
%
\[
\exists \vtrace_0,
\vtrace_1, \vtrace_2, \vtrace_2' \in \mathcal{T}, \event_2' \in \eventset, \event_1' \in \eventset^{\asn}, \ssa{c}_1, \ssa{c}_2,  \ssa{c}_2'.
  \left(
  \begin{array}{ll}   
 & \config{\ssa{c}, \vtrace_0} \rightarrow^{*} 
\config{\ssa{c}_1, \vtrace_1 \cdot \event_1}  \rightarrow^{*} 
  \config{\ssa{c}_2,  \vtrace_1 \cdot \event_1 \cdot \vtrace_2 \cdot \event_2 } 
  % 
 \\ 
 \bigwedge &
  \config{\ssa{c}_1, \vtrace_1 \cdot \event_1'}  \rightarrow^{*} 
  \config{\ssa{c}_2',  \vtrace_1 \cdot \event_1  \cdot \vtrace_2' \cdot \event_2' } 
\\
\bigwedge & \ismdiff(\event_2,\event_2')
\end{array}
\right)
 \]
%
\end{defn}
%
% \begin{defn}
% [Testing Dependency of Events]
% \label{def:event_testdep}.
% \\
% One event $\event_2$ may have a testing dependency on a testing event $\event_1 = (\ssa{b}_1, l_1, n_1, v_1)$
% in a program $\ssa{c}$, with a hidden database $D$, 
% denoted as 
% %
% $\eventdep^{\ctl}(\event_1, \event_2, c, D)$, is defined as follows: 
% %
% \[
% \exists \event_1' \in \eventset^{\test}, \vtrace_0,
% \vtrace_1, \vtrace_2, \vtrace_2', \ssa{c}_1.
%   \left(
%   \begin{array}{ll}   
%   & \config{\ssa{c}, \vtrace_0} \rightarrow^{*} 
%     \config{\ssa{c}_1, \vtrace_1 \cdot \event_1}  \rightarrow^{*} 
%     \config{\eskip,  \vtrace_1 \cdot \event_1 \vtrace_2} 
%   \\ 
%   \bigwedge &
%   \config{\ssa{c}_1, \vtrace_1 \cdot \event_1'}  \rightarrow^{*} 
%   \config{\eskip,  \vtrace_1 \cdot \event_1 \vtrace_2' } 
%   \\
%   \bigwedge &
%   \event_2 \sigin \vtrace_2 \land \event_2 \notsigin \vtrace_2'
% \end{array}
% \right)
%  \]
% %
% \end{defn}
%
\begin{defn}
[Control Dependency of Events]
\label{def:event_ctldep}.
\\
An event $\event_2$ is in the \emph{control may-dependency} relation with an assignment
event $\event_1 \in \eventset^{\asn}$ in a program $\ssa{c}$
with hidden database $D$, denoted as 
%
$\eventdep^{\ctl}(\event_1, \event_2, c, D)$, if and only if: 
%
\[
\exists \vtrace_1, \vtrace_2, \vtrace_2', \vtrace_0 \in \mathcal{T}, 
\event_1' \in \eventset^{\asn}, \ssa{c}_1.
\left(
\begin{array}{ll}   
  & \config{\ssa{c}, \vtrace_0} \rightarrow^{*} 
    \config{\ssa{c}_1, \vtrace_1 \cdot \event_1}  \rightarrow^{*} 
    \config{\eskip,  \vtrace_1 \cdot \event_1 \cdot \vtrace_2} 
  \\ 
  \bigwedge &
  \config{\ssa{c}_1, \vtrace_1 \cdot \event_1'}  \rightarrow^{*} 
  \config{\eskip,  \vtrace_1 \cdot \event_1 \cdot \vtrace_2' } 
  \\
  \bigwedge &
  \event_2 \ismin \vtrace_2 \land \event_2 \notismin \vtrace_2'
\end{array}
\right)
 \]
%
\end{defn}
%
%
\begin{defn}[Event May Dependency].
\label{def:event_dep}
\\ 
An event $\event_2 \in \eventset^{\asn}$ is in the \emph{may-dependency} relation with another
event $\event_1 \in \eventset^{\asn}$ in a program $\ssa{c}$ with a hidden database $D$, denoted as $\eventdep(\event_1, \event_2, c, D)$,
if and only if
\[
\eventdep^{val}(\event_1, \event_2, c, D) 
\lor
\eventdep^{\ctl}(\event_1, \event_2, c, D) 
\] %
%
\end{defn}
%
\begin{defn}[Variable May Dependency].
\label{def:var_dep}
\\
An variable $\ssa{x}_2^{l_2} \in \lvar_{\ssa{c}}$ is in the \emph{variable may-dependency} relation with another
variable $\ssa{x}_1^{l_1} \in \lvar_{\ssa{c}}$ in a program $\ssa{c}$, denoted as 
%
$\vardep(\ssa{x}_1^{l_1}, \ssa{x}_2^{l_2}, \ssa{c})$, if an only if.
%
\[
\exists \event_1, \event_2 \in \eventset^{\asn}, D \in \dbdom. ~
\projl{\event_1} = (\ssa{x}_1, l_1)
\land
\projl{\event_2} = (\ssa{x}_2, l_2)
\land 
\eventdep(\event_1, \event_2, c, D)
\] 
%
%
\end{defn}
%
\begin{defn}[Labeled Variables ($\lvar_{c} \subseteq \mathcal{VAR} \times \mathbb{N}$ or 
$\lvar : c \to \mathcal{P}(\mathcal{VAR} \times \mathbb{N})$]
$$
  \lvar_{c} \triangleq
  \left\{
  \begin{array}{ll}
      \{\ssa{x}^l\}                   
      & \ssa{c} = [\ssa{\assign x e}]^{l} 
      \\
      \{\ssa{x}^l\}                   
      & \ssa{c} = [\ssa{\assign x \query(\qexpr)}]^{l} 
      \\
      \lvar_{\ssa{c_1}} \cup \lvar{(\ssa{c_2})}  
      & \ssa{c} = \ssa{c_1};\ssa{c_2}
      \\
      \lvar_{\ssa{c}} \cup \lvar_{\ssa{c_2}} \cup \ssa{\{\bar{x}, \bar{y}, \bar{z}\}} 
      & \ssa{c} =\eif([\sbexpr]^{l} , \ssa{[\bar{x}, \bar{x_2}, \bar{x_2}], 
      [\bar{y}, \bar{y_2}, \bar{y_3}], 
      [\bar{z}, \bar{z_2}, \bar{z_3}], c_1, c_2}) 
      \\
      \lvar_{\ssa{c}'} \cup \{\ssa{\bar{x}}\}
      & \ssa{c}   = \ewhile ([\sbexpr]^{l}, [\ssa{\bar{x}, \bar{x_2}, \bar{x_2}}], \ssa{c}')
\end{array}
\right.
$$
\end{defn}
%
\begin{defn}[Query Variables ($\qvar_{c} \subseteq \mathcal{VAR} \times \mathbb{N}$)].
\\
Given a program $c$, its query variables $\qvar$ is a vector containing all variables newly assigned by a query in the programm, $\qvar \subset \mathcal{VAR}$.
It is defined as follows:
$$
  \qvar_{\ssa{c}} \triangleq
  \left\{
  \begin{array}{ll}
      \{\}                  
      & \ssa{c} = [\ssa{\assign x e}]^{(l, w)} 
      \\
      \{\ssa{x}^l\}                  
      & \ssa{c} = [\ssa{\assign x \query(\qexpr)}]^{(l, w)} 
      \\
      \qvar_{\ssa{c_1}} \cup \qvar_{\ssa{c_2}}  
      & \ssa{c} = \ssa{c_1};\ssa{c_2}
      \\
      \qvar_{\ssa{c_1}} \cup \qvar_{\ssa{c_2}} 
      & \ssa{c} =\eif([\sbexpr]^{l} , \ssa{[\bar{x}, \bar{x_2}, \bar{x_2}], 
      [\bar{y}, \bar{y_2}, \bar{y_3}], 
      [\bar{z}, \bar{z_2}, \bar{z_3}], c_1, c_2}) 
      \\
      \qvar_{\ssa{c}'}
      & \ssa{c}   = \ewhile ([\sbexpr]^{l}, [\ssa{\bar{x}, \bar{x_2}, \bar{x_2}}], \ssa{c}')
\end{array}
\right.
$$
\end{defn}
%
%
\begin{defn}[Execution Based Dependency Graph].
\\
Given a program $\ssa{c}$ with its assigned variables $\lvar_c$ 
the dependency graph $\traceG(\ssa{c}, D) = (\vertxs, \edges, \weights, \qflag)$ is defined as:
%
\[
\begin{array}{rlcl}
  \text{Vertices} &
  \vertxs & := & \left\{ 
  x^l \in \mathcal{VAR} \times \mathbb{N}
  ~ \middle\vert ~
  x^l \in \lvar({\ssa{c}})
  \right\}
  \\
  \text{Directed Edges} &
  \edges & := & 
  \left\{ 
  (\ssa{x}_1^{l_1}, \ssa{x}_2^{l_2}) \in \mathcal{VAR} \times \mathbb{N} \times (\mathcal{VAR} \times \mathbb{N})
  ~ \middle\vert ~
  \vardep(\ssa{x}_1^{l_1}, \ssa{x}_2^{l_2}, c) \land
  \ssa{x}_1^{l_1}, \ssa{x}_2^{l_2} \in \lvar_{\ssa{c}}
  \right\}
  \\
  \text{Weights} &
  \weights & := & 
  \left\{ 
  (x^l, n) \in \mathcal{VAR} \times \mathbb{N} \times \mathbb{N}
  ~ \middle\vert ~ 
  n = \max \left\{ \vcounter(\vtrace')l | \vtrace \in \mathcal{T} \st 
   \config{\ssa{c}, \trace} \to^{*} \config{\eskip, \vtrace'}  
   \land x^l \in \lvar({\ssa{c}}) 
   \right\}
  \right\}
  \\
  \text{Query Flags} &
  \qflag & := & 
  \left\{(x^l, n)  \in \mathcal{VAR} \times \mathbb{N}  \times \{0, 1\} 
  ~ \middle\vert ~
  \left\{
  \begin{array}{ll}
  n = 1 & x^l \in \qvar({\ssa{c}}) \\ 
  n = 0 & o.w.
  \end{array}
  \right\};
  x^l \in \lvar({\ssa{c}})
  \right\}
\end{array}
\]
\end{defn}
%
%
\begin{defn}[Finite Walk ($k$)].
\label{def:finitewalk}
\\
Given a labeled weighted graph $G = (\vertxs, \edges, \weights, \qflag)$, a \emph{finite walk} $k$ in $G$ is a sequence of edges $(e_1 \ldots e_{n - 1})$ 
for which there is a sequence of vertices $(v_1, \ldots, v_{n})$ such that:
\begin{itemize}
    \item $e_i = (v_{i},v_{i + 1})$ for every $1 \leq i < n$.
    \item every vertex $v \in \vertxs$ appears in this vertices sequence $(v_1, \ldots, v_{n})$ of $k$ at most $W(v)$ times.  
\end{itemize}
$(v_1, \ldots, v_{n})$ is the vertex sequence of this walk.
\\
%
Length of this finite walk $k$ is the number of vertices in its vertex sequence, i.e., $\len(k) = n$.
\end{defn}
%
Given a labeled weighted graph $G = (\vertxs, \edges, \weights, \qflag)$, 
we use $\walks(G)$ to denote a set containing all finite walks $k$ in $G$;
and $k_{v_1 \to v_2} \in \walks(G)$where $v_1, v_2 \in \vertxs$ denotes the walk from vertex $v_1$ to $v_2$ .
%
%
\begin{defn}[Length of Finite Walk w.r.t. Query ($\qlen$)].
\label{def:qlen}
\\
Given a labeled weighted graph $G = (\vertxs, \edges, \weights, \qflag)$ and a \emph{finite walk} $k$ in $G$ with its vertex sequence $(v_1, \ldots, v_{n})$, the length of $k$ w.r.t query is defined as:
\[
  \qlen(k) = \len\big(
  v \mid v \in (v_1, \ldots, v_{n}) \land \flag(v) = 1 \big)
\]
, where $\big(v \mid v \in (v_1, \ldots, v_{n}) \land \flag(v) = 1 \big)$ is a subsequence of $k$'s vertex sequence.
\end{defn}
%
%
% \subsection{SSA Transformation and Soundness of Transformation}
% in File {\tt ``ssa\_transform\_sound.tex''}
% %
\subsection{SSA Transformation}
We use a translation environment $\delta$, to map variables $x$ in the {\tt While} language to those variables $\ssa{x}$ in the SSA language.
We use a name environment denoted as $\Sigma$ as a set of ssa variables, to get a fresh variable by $fresh(\Sigma)$. 
We define $\delta_1 \bowtie \delta_2 $ in a similar way as
\cite{vekris2016refinement}.
%
\[ 
\delta_1 \bowtie \delta_2 = \{ ( x, {\ssa{x_1}, \ssa{x_2}} ) \in 
\mathcal{VAR} \times \mathcal{SVAR} \times \mathcal{SVAR} \mid x \mapsto {\ssa{x_1}} \in \delta_1 , x \mapsto {\ssa{x_2} } \in \delta_2, {\ssa{x_1} \not= {\ssa{x_2} }  }  \} 
\]
%
\[ 
\delta_1 \bowtie \delta_2 / \bar{x} = \{ ( x, {\ssa{x_1}, \ssa{x_2}} ) \in 
\mathcal{VAR} \times \mathcal{SVAR} \times \mathcal{SVAR}
 \mid x \not\in \bar{x} \land x \mapsto {\ssa{x_1}} \in \delta_1 , x \mapsto {\ssa{x_2} } \in \delta_2, {\ssa{x_1} \not= {\ssa{x_2} }   }  \} 
 \]
 %
We call a list of variables $\bar{x}$.
\[
 [\bar{x}, \bar{\ssa{x_1}}, \bar{\ssa{x_2}}] = \{ (x, x_1,x_2)  | \forall 0 \leq i < |\bar{x}|, x = \bar{x }[i] \land x_1 = \bar{x_1}[i] \land x_2 = \bar{x_2 }[i] \land |\bar{x}| = |\bar{x_1}| = |\bar{x_2}|   \}
\]
%
\begin{mathpar}
\boxed{ \delta ; e \hookrightarrow \ssa{e} }
\and
\inferrule{
}{
 \delta ; x \hookrightarrow \delta(x)
}~{\textbf{S-VAR}}
\and
\boxed{ \Sigma; \delta ; c  \hookrightarrow \ssa{c} ; \delta' ; \Sigma' }
\and
\inferrule{
  { \delta ; \bexpr \hookrightarrow \ssa{\bexpr} }
  \and
  { \Sigma; \delta ; c_1 \hookrightarrow \ssa{c_1} ; \delta_1;\Sigma_1 }
  \and
  {\Sigma_1; \delta ; c_2 \hookrightarrow \ssa{c_2} ; \delta_2 ; \Sigma_2 }
  \\
  {[\bar{x}, \ssa{\bar{{x_1}}, \bar{{x_2}}}] = \delta_1 \bowtie \delta_2  }
  \and
   {[\bar{y}, \ssa{\bar{{y_1}}, \bar{{y_2}}}] = \delta \bowtie \delta_1 / \bar{x} }
  \and
   {[\bar{z}, \ssa{\bar{{z_1}}, \bar{{z_2}}}] = \delta \bowtie \delta_2 / \bar{x} }
  \\
  { \delta' =\delta[\bar{x} \mapsto \ssa{\bar{{x}}'} ][\bar{y} \mapsto \ssa{\bar{{y}}'} ][\bar{z} \mapsto \ssa{\bar{{z}}'} ]}
  \and 
  {\ssa{\bar{{x}}', \bar{y}', \bar{z}'} \ fresh(\Sigma_2)
  }
  \and{\Sigma' = \Sigma_2 \cup \{ \ssa{ \bar{x}', \bar{y}', \bar{z}' } \} }
}{
 \Sigma; \delta ; [\eif(\bexpr, c_1, c_2)]^l  \hookrightarrow [\ssa{ \eif(\bexpr, [\bar{{x}}', \bar{{x_1}}, \bar{{x_2}}] ,[\bar{{y}}', \bar{{y_1}}, \bar{{y_2}}] ,[\bar{{z}}', \bar{{z_1}}, \bar{{z_2}}] , {c_1}, {c_2})}]^l; \delta';\Sigma'
}~{\textbf{S-IF}}
%
\and
%
\inferrule{
 {\delta ; \expr \hookrightarrow \ssa{\expr} }
 \and
 {\delta' = \delta[x \mapsto \ssa{{x}} ]}
 \and{ \ssa{x} \ fresh(\Sigma) }
 \and { \Sigma' = \Sigma \cup \{ \ssa{x} \} }
}{
 \Sigma;\delta ; [\assign x \expr]^{l} \hookrightarrow [\ssa{\assign {{x}}{ \expr}}]^{l} ; \delta'; \Sigma'
}~{\textbf{S-ASSN}}
%
\and
%
\inferrule{
 {\delta ; \query \hookrightarrow \ssa{\query}}
 \and
 {\delta ; \qexpr \hookrightarrow \ssa{\qexpr}}
 \and
 {\delta' = \delta[x \mapsto \ssa{x} ]}
 \and{ \ssa{x} \ fresh(\Sigma) }
  \and { \Sigma' = \Sigma \cup \{ \ssa{x} \} }
}{
 \Sigma;\delta ; [\assign{x}{\query(\qexpr)}]^{l} \hookrightarrow 
 [\assign {\ssa{x}}{ \ssa{\query(\qexpr)}}]^{l} ; \delta';\Sigma'
}~{\textbf{S-QUERY}}
%
%%
\and
%
%
\and
%
\inferrule{
    { \Sigma; \delta ; c \hookrightarrow \ssa{c_1} ; \delta_1; \Sigma_1 }
     \and
    { [ \bar{x}, \ssa{\bar{{x_1}}}, \ssa{\bar{{x_2}}} ] = \delta \bowtie \delta_1 }
    \\
     {
     \ssa{\bar{{x}}'} \ fresh(\Sigma_1 )}
    \and {\delta' = \delta[\bar{x} \mapsto \ssa{\bar{{x}}'}]}
    \and 
     {\delta' ; \bexpr \hookrightarrow \ssa{\bexpr} }
     \and
    {\ssa{c' = c_1[\bar{x}'/ \bar{x_1}]   } }
  }{ 
  \Sigma; \delta ;  \ewhile ~ [\bexpr]^{l} ~ \edo ~ c 
  \hookrightarrow 
  \ssa{\ewhile ~ [\bexpr]^{l}, 0, [\bar{{x}}', \bar{{x_1}}, \bar{{x_2}}] ~ \edo ~ {c} } ; \delta'; \Sigma_1 \cup \{\ssa{\bar{x}'}  \}
}~{\textbf{S-WHILE}
}
\and
%
\inferrule{
 {\Sigma;\delta ; c_1 \hookrightarrow \ssa{c_1} ; \delta_1; \Sigma_1} 
 \and
 {\Sigma_1; \delta_1 ; c_2 \hookrightarrow \ssa{c_2} ; \delta'; \Sigma'} 
}{
\Sigma;\delta ; c_1 ; c_2 \hookrightarrow \ssa{c_1} ; \ssa{c_2} \ ; \delta';\Sigma'
}~{\textbf{S-SEQ}}
\end{mathpar}

\paragraph{Concrete examples.}
\[
c_1 \triangleq
\begin{array}{l}
     \left[x \leftarrow \query(1) \right]^1; \\
     \eif \; (x ==0)^{2} \; \\
    \ethen \; \left[y \leftarrow \query(2) \right]^3\; \\
    \eelse \; \left[y \leftarrow 0 \right]^4 ; \\
    \eif \; (x == 1 )^5\; \\
    \ethen \; \left[ y \leftarrow 0 \right]^6\; \\
    \eelse \; \left[y \leftarrow \query(2) \right]^7\\
\end{array}
%
%
\hspace{20pt} \hookrightarrow  \hspace{20pt}
%
\begin{array}{l}
     \left[ \ssa{x_1} \leftarrow \query(1) \right]^1; \\
     \eif \; (\ssa{x_1 ==0})^{2}, [\ssa{ y_3, y_1,y_2  }],[],[]  \; \\
    \ethen \; \left[ \ssa{y_1} \leftarrow \query(2) \right]^3\; \\
    \eelse \; \left[\ssa{y_2 \leftarrow 0 } \right]^4 ; \\
    \eif \; (\ssa{x_1 == 1} )^{5} , [\ssa{ y_6, y_4, y_5 } ] \; \\
    \ethen \; \left[ \ssa{y_4 \leftarrow 0} \right]^6\; \\
    \eelse \; \left[\ssa{y_5} \leftarrow \query(2) \right]^7\\
\end{array}
\]
\[
c_2 \triangleq
\begin{array}{l}
   \left[ x \leftarrow \query(1) \right]^1; \\
   \left[y \leftarrow \query(2) \right]^2 ; \\
    \eif \;( x + y == 5 )^3\; \\
    \ethen \;\left[ z \leftarrow \query(3)\right]^4 \; \\
    \eelse \;\left[ \ssa{\eskip}\right]^5 ; \\
   \left[ w \leftarrow q_4 \right]^6; \\
\end{array}
\hspace{20pt} \hookrightarrow \hspace{20pt}
%
\begin{array}{l}
   \left[ \ssa{ x_1 } \leftarrow \query(1) \right]^1; \\
   \left[\ssa{ y_1} \leftarrow \query(2) \right]^2 ; \\
    \eif \;( \ssa{ x_1 + y_1 == 5} )^3, [ ],[] ,[ ]\; \\
    \ethen \;\left[ \ssa{ z_1 }
    \leftarrow \query(3)\right]^4 \; \\
    \eelse \;\left[ \eskip\right]^5 ; \\
   \left[ \ssa{ w_1} \leftarrow \query(4) \right]^6; \\
\end{array}
\]

{
\[
c_3 \triangleq
\begin{array}{l}
     \left[x \leftarrow \query(1) \right]^1 ; \\
     \left[i \leftarrow 0 \right]^2 ; \\
    \ewhile ~  [i < 100]^3 ~ \edo
    \\
    ~ \Big( 
    \left[z \leftarrow \query(3) \right]^4; \\
    \left[x \leftarrow z + x \right]^5; \\
    \left[i \leftarrow i + 1 \right]^6
    \Big) ;
\end{array}
%
\hspace{20pt} \hookrightarrow \hspace{20pt} 
%
\begin{array}{l}
     \left[\ssa{x_1} \leftarrow \query(1) \right]^1 ; \\
     \left[\ssa{i_1} \leftarrow 0 \right]^2 ; \\
    \ewhile
    ~ [\ssa{i_1} < 100]^3, 0,
    ~\ssa{[ x_3,x_1 ,x_2 ], [i_3, i_1, i_2] }~
    \edo \\
    ~ \Big( 
    \left[\ssa{z_1} \leftarrow \query(3) \right]^4; \\
    \left[ \ssa{x_2} \leftarrow \ssa{z_1 + x_3} \right]^5; \\
    \left[\ssa{i_2} \leftarrow \ssa{i_3} + 1 \right]^6
    \Big) ;
\end{array}
\]
}
%
\begin{figure}
   \[
 \begin{array}{lll}
    | \ewhile ~ [ \sbexpr ]^{l}, n, [\bar{\ssa{x}}, \bar{\ssa{x_1}}, \bar{\ssa{x_2}}] 
    ~ \edo ~  \ssa{c}|  
    &=& \ewhile ~ [|\sbexpr|]^{l},  ~ \edo ~ |\ssa{c}| 
	\\
    |\ssa{c_1 ; c_2}|  &=& |\ssa{c_1}| ; |\ssa{c_2}| 
    \\
    |[\eif(\sbexpr,
    [ \bar{\ssa{x}}, \bar{\ssa{x_1}}, \bar{\ssa{x_2}}] ,
    [ \bar{\ssa{y}}, \bar{\ssa{y_1}}, \bar{\ssa{y_2}}] , 
    [\bar{\ssa{z}}, \bar{\ssa{z_1}}, \bar{\ssa{z_2}}] , 
    \ssa{ c_1, c_2)}]^{l}|  
    &=&
    [\eif(|\sbexpr|, |\ssa{ c_1}|, |\ssa{c_2}|)]^{l}
    \\
    | [\assign {\ssa{x}}{\ssa{\expr}}]^{l}| & = & [\assign {|\ssa{x}|}{|\ssa{\expr}|} ]^{l}
    \\
    | [\assign {\ssa{x}}{\query(\ssa{\qexpr})} ]^{l} | & = & [\assign {|\ssa{x}|}{|\query(\ssa{\qexpr})|}]^{l}
    \\
    |\ssa{x}_i| & = & x 
    \\
    |n | & = & n 
    \\
    | \saexpr_1 \oplus_{a} \saexpr_2 | & = &  |\ssa{\aexpr_1}| \oplus_a |\ssa{\aexpr_2}| \\
    | \sbexpr_1 \oplus_{b} \sbexpr_2 | & = &  |\sbexpr_1| \oplus_b |\sbexpr_2|
 \end{array}
\]
    \caption{The Erasure of SSA}
    \label{fig:ssa_erasure-while}
\end{figure}
%
%
%
% 
%
\subsection{The Soundness of the Transformation}
In this subsection, we show our transformation from the {\tt While} language to its SSA form is sound with respect to the adaptivity. 
To be specific, a transformed program $\ssa{c}$ starting with appropriate configuration, generates the same trace as the program before the transformation $c$, in its corresponding configuration.
%
%
\begin{defn}[\todo{Written Variables}].
\\
We defined the assigned variables in the while language program $c$ as $\avars{c}$,the assigned variables in the ssa-form program $\ssa{c}$ as $\avarssa{\ssa{c}}$ defined as follows.
\[
\begin{array}{lll}
    \avars{\assign{x}{\expr}} & =& \{ x \} \\
    \avars{\assign{x}{\query(\qexpr)}} & =& \{ x \} \\
    \avars{c_1; c_2}  & = & \avars{c_1} \cup \avars{c_2} \\
    \avars{\ewhile ~ \bexpr ~ \edo ~ c} &= &  \avars{c} \\
    \avars{\eif(\bexpr, c_1, c_2)} & =&  \avars{c_1} \cup \avars{c_2}\\
\end{array} 
\]
%
\[
\begin{array}{lll}
    \avarssa{\ssa{\assign{x}{\expr}}} & =& \{ \ssa{x} \}
    \\
    \avarssa{\ssa{\assign{x}{\query(\ssa{\qexpr})}}} & =& \{ \ssa{x} \}
    \\
    \avarssa{\ssa{c_1; c_2 } }  & = & \avarssa{\ssa{c}_1} \cup \avarssa{\ssa{c}_2}
    \\
    \avarssa{\ewhile ~ \ssa{\bexpr, n, [\bar{x}, \bar{x_1}, \bar{x_2}] ~ \edo ~ \ssa{c}}}
    & = &  
    \{\ssa{\bar{x}}\} \cup \avarssa{\ssa{c}} 
    \\
    \avarssa{\eif(\ssa{\bexpr,[\bar{x}, \bar{x_1}, \bar{x_2}],[\bar{y}, \bar{y_1}, \bar{y_2}],[\bar{z}, \bar{z_1}, \bar{z_2}], c_1, c_2} )} 
    & =&  \{ \ssa{\bar{x}},\ssa{\bar{y}} , \ssa{\bar{z}} \} 
    \cup \avarssa{\ssa{c_1}} \cup \avarssa{\ssa{c_2}}\\
\end{array}
\]
\end{defn}
\begin{defn}[\todo{Read Variables}].
\\
{
The variables read in the while language programs $c$ as $\vars{c}$, variables used in ssa-form program $\ssa{c}$ : 
}
\[
\begin{array}{lll}
    \vars{\assign{x}{\expr}} & =& \vars{\expr}  \\
    \vars{\assign{x}{\query(\qexpr)}} & =&\{  \} \\
    \vars{ c_1; c_2  }  & = & \vars{c_1} \cup \vars{c_2} \\
    \vars{  \eloop ~ \aexpr ~ \edo ~ c  } &= &\vars{\aexpr} \cup \vars{c} \\
    \vars{\eif(\bexpr, c_1, c_2)} & =& \vars{\bexpr} \cup \vars{c_1} \cup \vars{c_2}\\
\end{array}
\]
\[
\begin{array}{lll}
    \varssa{\ssa{\assign{x}{\expr}}} & =& \varssa{\ssa{\expr}}  \\
    \varssa{\ssa{\assign{x}{\query(\qexpr)}}} & =& \{  \} \\
    \varssa{ \ssa{c_1; c_2}  }  & = & \varssa{\ssa{c}_1} \cup \varssa{\ssa{c}_2} \\
    % \varssa{  \eloop ~ \ssa{\aexpr, n, [\bar{x}, \bar{x_1}, \bar{x_2}] ~ \edo ~ c} } &= &\varssa{\ssa{\aexpr}} \cup \varssa{\ssa{c}}  \cup \{ \ssa{\bar{x_1}} \} \cup \{ \ssa{\bar{x_2}} \}\\
    {\varssa{  \ewhile ~ \ssa{\bexpr, n, [\bar{x}, \bar{x_1}, \bar{x_2}] ~ \edo ~ c} }} 
    &= &
    \varssa{\ssa{\bexpr}} \cup \varssa{\ssa{c}}  \cup \{ \ssa{\bar{x_1}} \} \cup \{ \ssa{\bar{x_2}} \}\\
    \varssa{\eif(\ssa{\bexpr,[\bar{x}, \bar{x_1}, \bar{x_2}], [\bar{y}, \bar{y_1}, \bar{y_2}],[\bar{z}, \bar{z_1}, \bar{z_2}], c_1, c_2} )} & =& \varssa{\ssa{\bexpr}} \cup \varssa{\ssa{c_1}} \cup \varssa{\ssa{c_2}} \cup \{\ssa{\bar{x_1}, \bar{x_2},\bar{y_1}, \bar{y_2},\bar{z_1}, \bar{z_2} }\}  \\
\end{array}
\]
\end{defn}
%
\begin{defn}[\todo{Necessary Variables}].
\\
{
We call the variables needed to be assigned before executing the program $c$ as necessary variables $\fv{c}$. Its ssa form is : $\fvssa{\ssa{c}}$.
}  
 \[
 \begin{array}{lll}
     \fvars{\assign{x}{\expr} }  & = & \vars{\expr}  \\
     \fvars{\assign{x}{\query(\qexpr)} }  & = & \{ \}  \\
     {\fvars{  \ewhile ~ \bexpr ~ \edo ~ c  } }&= & \vars{\bexpr} \cup \fvars{c} \\
     \fvars{\eif(\bexpr, c_1, c_2)} & =& \vars{\bexpr} \cup \fvars{c_1} \cup \fvars{c_2}  \\
      \fvars{c_1 ; c_2} & = & \fvars{c_1} \cup ( \fvars{c_2} - \avars{c_1})
 \end{array}
 \]
 \[
 \begin{array}{lll}
     \fvssa{\ssa{\assign{x}{\expr}} }  & = & \varssa{\ssa{\expr}}  \\
     \fvssa{ \ssa{ \assign{x}{\query(\qexpr)}} }  & = & \{ \}  \\
     {\fvssa{  \ewhile ~ \ssa{\bexpr, n, [\bar{x}, \bar{x_1}, \bar{x_2}] ~ \edo ~ c} } }
     &= & 
     \varssa{\ssa{\bexpr}} \cup \fvssa{\ssa{c}}[\ssa{ \bar{x_1}} / \ssa{\bar{x}}]\\
     \fvssa{\eif(\ssa{\bexpr,[\bar{x}, \bar{x_1}, \bar{x_2}],[\bar{y}, \bar{y_1}, \bar{y_2}],[\bar{z}, \bar{z_1}, \bar{z_2}], c_1, c_2} )} & =& \varssa{\ssa{\bexpr}} \cup \fvssa{\ssa{c_1}} \cup \fvssa{\ssa{c_2}}  \\
      \fvssa{\ssa{c_1 ; c_2}} & = & \fvssa{\ssa{c_1}} \cup ( \fvssa{\ssa{c_2}} - \avarssa{\ssa{c_1}})
 \end{array}
 \]
%
\end{defn}
%
The Lemma~\ref{lem:fv} and \ref{lem:same_value} proved the preserving properties for variables and values during the transformation.
%
\begin{lem}[Variable Preserving]
\label{lem:fv}
If $\Sigma;\delta ; c \hookrightarrow \ssa{c} ; \delta';\Sigma' $, $\fvssa{\ssa{c}} = \delta(\fvars{c})$. 
\end{lem}
\begin{proof}
 By induction on the transformation.
 \begin{itemize}
    \caseL{Case $\inferrule{
  { \delta ; \bexpr \hookrightarrow \ssa{\bexpr} }
  \and
  { \delta ; c_1 \hookrightarrow \ssa{c_1} ; \delta_1 }
  \and
  {\delta ; c_2 \hookrightarrow \ssa{c_2} ; \delta_2 }
  \\
  {[\bar{x}, \ssa{\bar{{x_1}}, \bar{{x_2}}}] = \delta_1 \bowtie \delta_2  }
  \and
   {[\bar{y}, \ssa{\bar{{y_1}}, \bar{{y_2}}}] = \delta \bowtie \delta_1 / \bar{x} }
  \and
   {[\bar{z}, \ssa{\bar{{z_1}}, \bar{{z_2}}}] = \delta \bowtie \delta_2 / \bar{x} }
  \\
  { \delta' =\delta[\bar{x} \mapsto \ssa{\bar{{x}}'} ]}
  \and 
  {\ssa{\bar{{x}}', \bar{y}', \bar{z}'} \ fresh }
}{
 \delta ; [\eif(\bexpr, c_1, c_2)]^l  \hookrightarrow [\ssa{ \eif(\bexpr, [\bar{{x}}', \bar{{x_1}}, \bar{{x_2}}] ,[\bar{{y}}', \bar{{y_1}}, \bar{{y_2}}] ,[\bar{{z}}', \bar{{z_1}}, \bar{{z_2}}] , {c_1}, {c_2})}]^l; \delta'
}~{\textbf{S-IF}} $}
From the definition of $\fvssa{[\eif(\sbexpr, [\bar{\ssa{x'}}, \bar{\ssa{x_1}}, \bar{\ssa{x_2}}] , \ssa{c_1}, \ssa{c_2})]^l} = \varssa{\ssa{\bexpr}} \cup \fvssa{\ssa{c_1}} \cup \fvssa{\ssa{c_2}}$. We want to show: \[\varssa{\ssa{\bexpr}}) \cup \fvssa{\ssa{c_1}} \cup \fvssa{\ssa{c_2}} = \delta( \vars{\bexpr}) \cup \delta(\fv{c_1}) \cup \delta(\fv{c_2}  )\]
By induction hypothosis on the second and third premise, we know that : $\fvssa{\ssa{c_1}} = \delta(\fv{c_1}) $ and $\fvssa{\ssa{c_2}} = \delta(\fv{c_2}) $.  We still need to show that: 
\[
  \varssa{\ssa{\bexpr}} = \delta(\vars{\bexpr})
\] 
From the first premise, we know that $\vars{b} \subseteq \dom(\delta)$. This is goal is proved by the rule $\textbf{S-VAR}$ on all the variables in $\bexpr$.\\
{\caseL{Case
$\inferrule{
    { \Sigma; \delta ; c \hookrightarrow \ssa{c_1} ; \delta_1; \Sigma_1 }
     \and
    { [ \bar{x}, \ssa{\bar{{x_1}}}, \ssa{\bar{{x_2}}} ] = \delta \bowtie \delta_1 }
    \\
     {\ssa{\bar{{x}}'} \ fresh(\Sigma_1 )}
    \and {\delta' = \delta[\bar{x} \mapsto \ssa{\bar{{x}}'}]}
    \and 
     {\delta' ; \bexpr \hookrightarrow \ssa{\bexpr} }
     \and
    {\ssa{c' = c_1[\bar{x}'/ \bar{x_1}]   } }
    % \and{ \delta' ; c \hookrightarrow \ssa{c'} ; \delta'' }
  }{ 
  \Sigma; \delta ;  \ewhile ~ [\bexpr]^{l} ~ \edo ~ c 
  \hookrightarrow 
  \ssa{\ewhile ~ [\bexpr]^{l}, 0, [\bar{{x}}', \bar{{x_1}}, \bar{{x_2}}] ~ \edo ~ {c} } ; \delta'; \Sigma_1 \cup \{\ssa{\bar{x}'}  \}
}~{\textbf{S-WHILE}}
$}
}
{
Unfolding the definition, we need to show:
\[\varssa{\ssa{\bexpr}} \cup \fvssa{\ssa{c'}}[\ssa{ \bar{x_1}} / \ssa{\bar{x}}] = \delta (\vars{\bexpr}) \cup \delta(\fv{c} ) \]
We can similarly show that $\varssa{\ssa{\bexpr}} = \delta(\vars{b})$ as in the if case. We still need to show that: 
\[
 \fvssa{\ssa{c_1[\bar{x}' / \bar{x_1}]}}[ \ssa{ \bar{x_1} } / \ssa{\bar{x}'}] =  \delta(\fv{c} )
\]
It is proved by induction hypothesis on $  { \Sigma; \delta ; c \hookrightarrow \ssa{c_1} ; \delta_1; \Sigma_1 }$.\\
}
%
\caseL{Case $\inferrule{
 {\Sigma;\delta ; c_1 \hookrightarrow \ssa{c_1} ; \delta_1; \Sigma_1} 
 \and
 {\Sigma_1; \delta_1 ; c_2 \hookrightarrow \ssa{c_2} ; \delta'; \Sigma'} 
}{
\Sigma;\delta ; c_1 ; c_2 \hookrightarrow \ssa{c_1} ; \ssa{c_2} \ ; \delta';\Sigma'
}~{\textbf{S-SEQ}}$}
To show:
  \[ \fvssa{\ssa{c_1}} \cup ( \fvssa{\ssa{c_2}} - \avarssa{\ssa{c_1}}) = \delta(\fv{c_1} )\cup \delta( \fv{c_2} - \avars{c_1}) \]
  By induction hypothesis on the first premise, we know that : $ \fvssa{\ssa{c_1}} = \delta(\fv{c_1} ) $, still to show: 
    \[ ( \fvssa{\ssa{c_2}} - \avarssa{\ssa{c_1}}) = \delta( \fv{c_2} - \avars{c_1})
    \]
    We know that $\delta_1 = \delta [\avars{c_1} \mapsto \avarssa{\ssa{c_1}} ]$, so by induction hypothesis, we know: $ \fvssa{\ssa{c_2}} = \delta[\avars{c_1} \mapsto \avarssa{\ssa{c_1}} ]( \fv{c_2})  = \delta(\fv{c_2}) \cup \avarssa{\ssa{c_1}} - \delta(\avars{c_1}) $.
    
    This case is proved.
 \end{itemize}
 
\end{proof}

{
We first define a good memory in the {\tt While} language $m$ or in the ssa language $\ssa{m}$ with respect to a translation environment $\delta$, denoted as $m \vDash \delta$ and $\ssa{m} \vDash \delta$ respectively. 
%
\begin{defn}[Well Defined Memory].
\begin{enumerate}
    % \item $m \vDash c \triangleq \forall x \in \fv{c}, \exists v, (x, v) \in m$.
    \item $ m \vDash \delta  \triangleq \forall x \in \dom(\delta), \exists v, (x,v) \in m$.
    % \item $\ssa{m} \vDash_{ssa} \ssa{c} \triangleq \forall \ssa{x} \in \fvssa{\ssa{c}}, \exists v, (\ssa{x}, v) \in \ssa{m}$.
    \item $ \ssa{m} \vDash_{ssa} \delta  \triangleq \forall \ssa{x} \in \codom(\delta), \exists v, (\ssa{x},v) \in \ssa{m}$.
\end{enumerate}
\end{defn}
%
The part declared in the translation environment $\delta$ in a ssa memory $\ssa{m}$ can be reverted to corresponding part of the memory $m$ with an inverse of $\delta$ as follows.
%
\begin{defn}[Inverse of Transformed memory]
 $m = \delta^{-1}(\ssa{m}) \triangleq \forall x \in \dom(\delta), (\delta(x), m(x)) \in \ssa{m} $.
\end{defn}
}
%
\begin{lem}[Value Preserving].
\label{lem:same_value}
{
Given $\delta; e \hookrightarrow \ssa{e}$,  $\forall m. m \vDash \delta. \forall \ssa{m}, \ssa{m} \vDash_{ssa} \delta \land m = \delta^{-1}(\ssa{m})$, then $\config{m, e} \to v $ and $\config{
\ssa{m}, \ssa{e}} \to {v}$.
}
\end{lem}

\begin{thm}[Soundness of transformation]
Given $\Sigma; \delta ; c \hookrightarrow \ssa{c} ; \delta';\Sigma' $, $\forall m. m \vDash \delta. \forall \ssa{m}, \ssa{m} \vDash_{ssa} \delta \land m = \delta^{-1}(\ssa{m})$, if there exist an execution of $c$ in the while language, starting with a trace $t$ and loop maps $w$, $\config{m, c, t, w} \to^{*} \config{m', \eskip, t', w' } $,  then there also exists a corresponding execution of $\ssa{c}$ in the ssa language so that 
  $\config{  {\ssa{m}}, \ssa{c}, t, w} \to^{*} \config{{  \ssa{m'}}, \eskip, t', w' } $ and $ m' = \delta'^{-1}(\ssa{m'}) $.
\end{thm}

\begin{proof}
 We assume that $q$ is the same when transformed to $\ssa{q}$, as the primitive in both languages. And the value remains the same during the transformation.  
 It is proved by induction on the transformation rules.
 \begin{itemize}
   \caseL{Case $\inferrule{
 {\Sigma;\delta ; c_1 \hookrightarrow \ssa{c_1} ; \delta_1;\Sigma_1} 
 \and
 {\Sigma_1; \delta_1 ; c_2 \hookrightarrow \ssa{c_2} ; \delta'; \Sigma'} 
}{
\Sigma;\delta ; c_1 ; c_2 \hookrightarrow \ssa{c_1} ; \ssa{c_2} \ ; \delta';\Sigma'
}~{\textbf{S-SEQ}}$}
We choose an arbitrary memory $m$ so that $m \vDash \delta$, we choose a trace $t$ and a loop maps $w$.
\[
\inferrule
{
{\config{m, c_1,  t,w} \xrightarrow{}^{*} \config{m_1, \eskip,  t_1,w_1}}
\and
{\config{m_1, c_2,  t_1,w_1} \xrightarrow{}^{*} \config{m', \eskip,  t',w'}}
}
{
\config{m, c_1; c_2,  t,w} \xrightarrow{}^{*} \config{m', \eskip, t',w'}
}
\]
 We choose the transformed memory ${\ssa{m}} $ so that  $ m =\delta^{-1}(\ssa{m})$.
 
 To show: $ \config{\ssa{ m, c_1;c_2 }, t, w } \xrightarrow{}^{*} \config{\ssa{m', \eskip}, t'. w' }$ and $ m' = \delta'^{-1} (\ssa{m'}) $.
 
 By induction hypothesis on the first premise, we have:
 \[ \config{\ssa{m, c_1}, t,w} \xrightarrow{}^{*} \config{\ssa{m_1, \eskip},t_1,w_1 } \land m_1 = \delta_1^{-1}(\ssa{m_1}) \]
  By induction hypothesis on the second premise, using the conclusion $ m_1 = \delta_1^{-1}(\ssa{m_1}) $.
  We have:
  \[
   \config{\ssa{m_1, c_2}, t_1,w_1} \xrightarrow{}^{*} \config{\ssa{m', \eskip},t',w' } \land m' = \delta'^{-1}(\ssa{m'})
  \]
  So we know that 
  \[
  \inferrule{
  { \config{\ssa{m, c_1}, t,w} \xrightarrow{}^{*} \config{\ssa{m_1, \eskip},t_1,w_1 }  }
  \and
  { \config{\ssa{m_1, c_2}, t_1,w_1} \xrightarrow{}^{*} \config{\ssa{m', \eskip},t',w' } }
  }{
  \config{\ssa{m, c_1;c_2 }, t,w} \xrightarrow{}^{*} \config{\ssa{m', \eskip}, t' , w' }
  }
  \]
 \caseL{Case $\inferrule{
 { \delta ; \expr \hookrightarrow \sexpr}
 \and
 {\delta' = \delta[x \mapsto \ssa{x} ]}
 \and{ \ssa{x} \ fresh(\Sigma) }
 \and {\Sigma' = \Sigma \cup \{x\} }
}{
 \Sigma;\delta ; [\assign x \expr]^{l} \hookrightarrow [\assign {\ssa{x}}{ \ssa{\expr}}]^{l} ; \delta';\Sigma'
}~{\textbf{S-ASSN}} $ }

 We choose an arbitrary memory $m$ so that $m \vDash \delta$, we choose a trace $t$ and a loop maps $w$, we know that the resulting trace is still $t$ from its evaluation rule $\textbf{assn}$ when we suppose $m, \expr \to v$.
 \[
 \inferrule
{
}
{
\config{m, [\assign x v]^{l},  t,w} \xrightarrow{} \config{m[v/x], [\eskip]^{l}, t,w}
}
~\textbf{assn}
 \]
 We choose the transformed memory ${\ssa{m}} $ so that  $ m =\delta^{-1}(\ssa{m})$.
 
 To show: $\config{\ssa{m}, [\assign {\ssa{x}}{ \ssa{\expr}}]^{l} , t, w} \to^{*} \config{\ssa{m'}, \eskip, t, w} $ and $ m' = \delta'^{-1}(\ssa{m'}) $.
 
 From the rule \textbf{ssa-assn}, we assume $\ssa{m}, \ssa{\expr} \to \ssa{v}$, we know that 
 \[
 \inferrule
{
}
{
\config{\ssa{ m, [\assign x v]^{l}},  t,w } \xrightarrow{} \config{\ssa{m[x \mapsto v], [\eskip]^{l}}, t,w }
}
~\textbf{ssa-assn}
 \]
 We also know that $\delta'= \delta[x \mapsto \ssa{x}]$ and $m = \delta^{-1}(\ssa{m})$, $m'= m[v/x]$. We can show that $ m[v/x] = \delta[x \mapsto \ssa{x}]^{-1}(\ssa{m}[\ssa{x} \mapsto v]) $.
 
\caseL{Case $\inferrule{
 {\delta ; q \hookrightarrow \ssa{q}}
 \and
 {\delta ; \expr \hookrightarrow \ssa{\expr}}
 \and
 {\delta' = \delta[x \mapsto \ssa{x} ]}
 \and{ \ssa{x} \ fresh(\Sigma) }
 \and{ \Sigma' = \Sigma \cup \{x\} }
}{
 \Sigma;\delta ; [\assign{x}{\query(\qexpr)}]^{l} \hookrightarrow [\assign {\ssa{x}}{ \ssa{\query(\qexpr)}}]^{l} ; \delta'
}~{\textbf{S-QUERY}}$} 
We choose an arbitrary memory $m$ so that $m \vDash \delta$, we choose a trace $t$ and a loop maps $w$, we know that when we suppose $\config{m, \expr} \to v$.
 \[
\inferrule
{
\query(v)(D) = \qval 
}
{
\config{m, [\assign{x}{\query(v)}]^l, t, w} \xrightarrow{} \config{m[ \qval/ x], \eskip,  t \mathrel{++} [\query(v),l,w )],w }
}
~\textbf{query}
 \]
 We choose the transformed memory ${\ssa{m}} $ so that  $ m =\delta^{-1}(\ssa{m})$.
 
 To show: $\config{\ssa{m}, [\assign {\ssa{x}}{ \ssa{\query(\qexpr)}}]^{l} , t, w} \to^{*} \config{\ssa{m'}, \eskip, t, w} $ and $ m' = \delta'^{-1}(\ssa{m'}) $.
 
 From the rule \textbf{ssa-query}, we know that 
 \[
 \inferrule
{
\ssa{\query(v)(D) = \qval} 
}
{
\config{ \ssa{ m, [\assign{\ssa{x}}{\ssa{\query(\qexpr)}}]^l}, t, w} \xrightarrow{} \config{\ssa{  m[  x \mapsto v], \eskip,}  t \mathrel{++} [(q^{(l,w )},v)],w }
}
~\textbf{ssa-query}
 \]
 We also know that $\delta'= \delta[x \mapsto \ssa{x}]$ and $m = \delta^{-1}(\ssa{m})$, $m'= m[v/x]$. We can show that $ m[v/x] = \delta[x \mapsto \ssa{x}]^{-1}(\ssa{m}[\ssa{x} \mapsto v]) $.

  \caseL{Case $\inferrule{
  { \delta ; \bexpr \hookrightarrow \ssa{\bexpr} }
  \and
  { \Sigma; \delta ; c_1 \hookrightarrow \ssa{c_1} ; \delta_1;\Sigma_1 }
  \and
  {\Sigma_1; \delta ; c_2 \hookrightarrow \ssa{c_2} ; \delta_2 ; \Sigma_2 }
  \\
  {[\bar{x}, \ssa{\bar{{x_1}}, \bar{{x_2}}}] = \delta_1 \bowtie \delta_2  }
  \and
   {[\bar{y}, \ssa{\bar{{y_1}}, \bar{{y_2}}}] = \delta \bowtie \delta_1 / \bar{x} }
  \and
   {[\bar{z}, \ssa{\bar{{z_1}}, \bar{{z_2}}}] = \delta \bowtie \delta_2 / \bar{x} }
  \\
  { \delta' =\delta[\bar{x} \mapsto \ssa{\bar{{x}}'} ][\bar{y} \mapsto \ssa{\bar{{y}}'} ][\bar{z} \mapsto \ssa{\bar{{z}}'} ]}
  \and 
  {\ssa{\bar{{x}}', \bar{y}', \bar{z}'} \ fresh(\Sigma_2)
  }
  \and{\Sigma' = \Sigma_2 \cup \{ \ssa{ \bar{x}', \bar{y}', \bar{z}' } \} }
}{
 \Sigma; \delta ; [\eif(\bexpr, c_1, c_2)]^l  \hookrightarrow [\ssa{ \eif(\bexpr, [\bar{{x}}', \bar{{x_1}}, \bar{{x_2}}] ,[\bar{{y}}', \bar{{y_1}}, \bar{{y_2}}] ,[\bar{{z}}', \bar{{z_1}}, \bar{{z_2}}] , {c_1}, {c_2})}]^l; \delta';\Sigma'
}~{\textbf{S-IF}}$}
We choose an arbitrary memory $m$ so that $m \vDash \delta$, we choose a trace $t$ and a loop maps $w$.
There are two possible evaluation rules depending on the the condition $b$, we choose the case when $b = \etrue$, we know there is an execution in ssa language so that $\ssa{\bexpr} = \etrue$, we use the rule $\textbf{if-t}$.  
 \[\inferrule
{
}
{
\config{m, [\eif(\etrue, c_1, c_2)]^{l},t,w} 
\xrightarrow{} \config{m, c_1,  t,w} \to^{*} \config{m', \eskip, t', w'}
}
\]
 We choose the transformed memory ${\ssa{m}} $ so that  $ m =\delta^{-1}(\ssa{m})$.
 
 To show: $\config{\ssa{m}, [\eif(\etrue, [\bar{\ssa{x}}', \bar{\ssa{x_1}}, \bar{\ssa{x_2}}] ,[\bar{\ssa{y}}', \bar{\ssa{y_1}}, \bar{\ssa{y_2}}] ,[\bar{\ssa{z}}', \bar{\ssa{z_1}}, \bar{\ssa{z_2}}] , c_1, c_2)]^{l}, t, w} \to^{*} \config{\ssa{m'}, \eskip, t', w'} $ and $ m' = \delta'^{-1}(\ssa{m'}) $.

We use the corresponding rule $\textbf{SSA-IF-T}$.  
\[
\inferrule
{
}
{
\config{\ssa{ { m} , [\eif(\etrue, [\bar{\ssa{x}}', \bar{\ssa{x_1}}, \bar{\ssa{x_2}}] , [\bar{\ssa{y}}', \bar{\ssa{y_1}}, \bar{\ssa{y_2}}] ,[\bar{\ssa{z}}', \bar{\ssa{z_1}}, \bar{\ssa{z_2}}] , \ssa{c_1, c_2})]^{l}},t,w} 
\xrightarrow{} \\ \config{\ssa{ m, c_1}; \eifvar(\ssa{\bar{x}', \bar{x_1}});\eifvar(\ssa{\bar{y}', \bar{y_2}});\eifvar(\ssa{\bar{z}', \bar{z_1}}),  t,w } 
}~\textbf{ssa-if-t}
\]
By induction hypothesis on $ \Sigma;\delta ; c_1 \hookrightarrow  \ssa{c_1}; \delta_1;\Sigma_1$, and we know that $\config{m, c_1,  t,w} \to^{*} \config{m', \eskip, t', w'} $, from our assumption that $ m =\delta^{-1}(\ssa{m})$, we know that 
\[\config{\ssa{ { m}, c_1},  t,w} \to^{*} \config{ \ssa{ { m_1 }, \eskip,} t', w' } \land m' = \delta_1^{-1}(\ssa{m_1}) \]
and we then have:
\[
\inferrule
{
  \config{\ssa{ { m}, c_1},  t,w} \to^{*} \config{ \ssa{ { m_1 }, \eskip,} t', w' }
}
{
 \config{\ssa{  m, c_1;} \eifvar(\ssa{\bar{x}', \bar{x_1})};\eifvar(\ssa{\bar{y}', \bar{y_1})};\eifvar(\ssa{\bar{z}', \bar{z_1})},  t,w  }  \to^{*}
 \config{\ssa{ { m_1 [ \bar{x}' \mapsto {m_1}(\bar{x_1}),\bar{y}' \mapsto {m_1}(\bar{y_2}),\bar{z}' \mapsto {m_1}(\bar{z_1}) ] }, \eskip}, t', w'  }
}
\]
Now, we want to show that $ m' = \delta[\bar{x} \mapsto \ssa{\bar{x}'},\bar{y} \mapsto \ssa{\bar{y}'},\bar{z} \mapsto \ssa{\bar{z}'} ]^{-1}(\ssa{ m_1 [ \bar{x}' \mapsto {m_1}(\bar{x_1}),\bar{y}' \mapsto {m_1}(\bar{y_2}),\bar{z}' \mapsto {m_1}(\bar{z_1}) ] }) $.

Unfold the definition, we want to show that $$\forall x  \in ( \dom(\delta) \cup \bar{x} \cup \bar{y} \cup \bar{z} ), (\delta[\bar{x} \mapsto \ssa{\bar{x}'},\bar{y} \mapsto \ssa{\bar{y}'},\bar{z} \mapsto \ssa{\bar{z}'} ](x), m'(x)) \in \ssa{m_1 [ \bar{x}' \mapsto {m_1}(\bar{x_1}),\bar{y}' \mapsto {m_1}(\bar{y_2}),\bar{z}' \mapsto {m_1}(\bar{z_1}) ] } .$$
\begin{enumerate}
    \item For variable $x$ in $\bar{x}$, we can find a corresponding ssa variable $\ssa{x} \in \ssa{\bar{x}'}$, so that $( \ssa{x}, m'(x) ) \in \ssa{ m_1 [\bar{x}' \mapsto m_1(\bar{x_1})] } $. It is because we know $[x \mapsto \ssa{x_1}]$ for certain $\ssa{x_1} \in \ssa{\bar{x_1}}$ in $\delta_1$, then by unfolding  $m' = \delta_1^{-1}(\ssa{m_1})$ and $\ssa{\bar{x_1}} \in \codom(\delta_1)$, we know $(\ssa{x_1}, m'(x)) \in \ssa{m_1}$ so that $m'(x) = \ssa{m_1}(\ssa{x_1})$.
    \item For variable $y \in \bar{y}$, we know that $y \in \dom(\delta_1)$, then $[ y \mapsto \ssa{y_2} ]$ for certain $\ssa{y_2} \in \ssa{\bar{y_2}}$ in $\delta_1$.  So we know that $(\delta_1(y), m'(y) ) \in \ssa{m_1}$, and then $m'(y) = \ssa{m_1(y_2)}$. We can show $(\ssa{y}, m'(y)) \in \ssa{m_1[\bar{y}' \mapsto m_1(\bar{y_2})]}$.
    \item For variable $z \in \bar{z}$, we know that $z \not\in \dom(\delta_1)$ by the definition (otherwise $z$ will appear in $\bar{x}$), then $[ z \mapsto \ssa{z_1} ]$ for certain $ \ssa{z_1} \in \ssa{\bar{z_1}}$ in $\delta$.  We know $(\delta(z), m(z)) \in \ssa{m}$ from our assumption, so we have $ m(z) = \ssa{m(z_1)}$. Because $z$ is not modified in $c_1$, so that $m(z) = m'(z)$. Also $\ssa{m}$ will not shrink during execution and $\ssa{z_1}$ will not be written in $\ssa{c_1}$, so $(\ssa{z_1}, m'(z)) \in \ssa{m_1}$. Then we can show that $ (\ssa{z}, m'(z) ) \in \ssa{m_1[\bar{z}' \mapsto m_1(\bar{z_1})] }$.
    \item For variable $k \in \dom(\delta)- \bar{x} - \bar{y}-\bar{z}$. From our assumption $ m = \delta^{-1}(\ssa{m})$, we can show $(\delta(k), m(k) ) \in \ssa{m}$. We know that $k$ is not written in either branch from our definition, so $(\delta(k), m'(k) ) \in \ssa{m_1} $ .
\end{enumerate}

{
\caseL{
	Case
	$
	\inferrule{
    { \Sigma; \delta ; c \hookrightarrow \ssa{c_1} ; \delta_1; \Sigma_1 }
     \and
    { [ \bar{x}, \ssa{\bar{{x_1}}}, \ssa{\bar{{x_2}}} ] = \delta \bowtie \delta_1 }
    \\
     {\ssa{\bar{{x}}'} \ fresh(\Sigma_1 )}
    \and {\delta' = \delta[\bar{x} \mapsto \ssa{\bar{{x}}'}]}
    \and 
     {\delta' ; \bexpr \hookrightarrow \ssa{\bexpr} }
     \and
    {\ssa{c' = c_1[\bar{x}'/ \bar{x_1}]   } }
    % \and{ \delta' ; c \hookrightarrow \ssa{c'} ; \delta'' }
  }{ 
  \Sigma; \delta ;  \ewhile ~ [\bexpr]^{l} ~ \edo ~ c 
  \hookrightarrow 
  \ssa{\ewhile ~ [\bexpr]^{l}, 0, [\bar{{x}}', \bar{{x_1}}, \bar{{x_2}}] ~ \edo ~ {c} } ; \delta'; \Sigma_1 \cup \{\ssa{\bar{x}'}  \}
}~{\textbf{S-WHILE}}
$
}
}
\\
{
We choose an arbitrary memory $m$ so that $m \vDash \delta$, we choose a trace $t$ and a loop maps $w$. Suppose $ \config{m ,a} \to v_N $. There are two cases, when $v_N=0$, the loop body is not executed so we can easily show that the trace is not modified.
%
When the while loop is still running ($v_N > 0$), we have the following evaluation in the while language:
\[
\inferrule
{
 \empty
}
{
\config{
m, \ewhile ~ [b]^{l} ~ \edo [c]^{l + 1},  t, w 
}
\xrightarrow{} \config{m, c ; 
\eif_w (b, c ; 
\ewhile ~ [b]^{l} ~ \edo [c]^{l + 1},  \eskip),
t, w }
}
~\textbf{while-b}
\]
which follows by the following evaluation:
\[
	\inferrule
{
 m, b \xrightarrow{} b'
}
{
\config{m, \eif_w (b, c,  \eskip) ,  t, w }
\xrightarrow{} \config{m, 
 \eif_w (b', c,  \eskip), t, w }
}
~\textbf{ifw-b}
\]
In the corresponding ssa-form language, we have the corresponding evaluation in the same way by assuming 
$m = \delta^{-1}(\ssa{m})$.
%
\[
	\inferrule
{
 {n = 0 \rightarrow i = 1 }
 \and
 {n > 0 \rightarrow i = 2 }
}
{
\config{
\ssa{m},  
\ssa{\ewhile ~ [\bexpr]^{l}, n, 
[\bar{{x}}', \bar{{x_1}}, \bar{{x_2}}] 
~ \edo ~ {c} 
},  t, w 
}
\xrightarrow{} \\ 
\config{
\ssa{m},
\eif_w 
(\ssa{b[\bar{x_i}/\bar{x'}], [\bar{{x}}', \bar{{x_1}}, \bar{{x_2}}], n,  c[\bar{x_i}/\bar{x'}] }; 
\ssa{
\ewhile ~ [b]^{l}, n+1, 
[\bar{{x}}', \bar{{x_1}}, \bar{{x_2}}]  
~ \edo ~ c} ,  \eskip),
t, w
}
}
~\textbf{ssa-while-b}
\]
This evaluation is followed by the following evaluation:
\[
	\inferrule
{
 \ssa{m, b \xrightarrow{} b'}
}
{
\config{\ssa{m, \eif_w (b, [\bar{{x}}', \bar{{x_1}}, \bar{{x_2}}] , n,  c_1,  c_2)} ,  t, w }
\xrightarrow{} \config{\ssa{ m, 
 \eif_w (b', [\bar{{x}}', \bar{{x_1}}, \bar{{x_2}}] , n , c_1 , c_2 )}, t, w }
}
~\textbf{ssa-ifw-b}
\]
%
Depending on if the counter $n$ is equal to $0$ or not, there are two possible execution paths (the variables $\ssa{\bar{x}}$ is replaced by the $\ssa{\bar{x_1}}$ or $\ssa{\bar{x_2}}$). We start from the first iteration (when $n =0$) when $v_N >0$. 
}
{
By induction hypothsis on the premise $ { \Sigma; \delta ; c \hookrightarrow \ssa{c_1} ; \delta_1; \Sigma_1 }$, we know that 
\[ \config{\ssa{{m}, c'[ \bar{x_1}/\bar{x}'  ]}, t, (w+l)  } \to^{*} \config{\ssa{{m'}, \eskip}, t'_{i}, w'  } \land m' = \delta_1^{-1}(\ssa{m'})   \]
Hence we can conclude that:
\[
  \inferrule{
   \config{\ssa{{m}, c'[ \bar{x_1}/\bar{x}'  ]}, t, (w+l) }  \to^{*} \config{\ssa{{m'}, \eskip}, t'_{1}, w'  }
  }{
  \config{\ssa{ {m}, c'[ \bar{x_1}/\bar{x}'  ];  [\eloop ~ (\valr_N-1), n+1, [\bar{\ssa{x}}', \bar{\ssa{x_1}}, \bar{\ssa{x_2}}] ~  \edo ~ c' ]^{l} },  t, (w + l)  }  \to^{*} \\ \config{ \ssa{{m'}, [\eloop ~ (\valr_N-1), n+1, [\bar{\ssa{x}}', \bar{\ssa{x_1}}, \bar{\ssa{x_2}}] ~  \edo ~ c' ]^{l}}, t'_{1}, w'  } 
  }
\]
%
Then there are two cases, 
%
\begin{enumerate}
     \item  when guard in the $\eif_w$ expression evaluates to $\efalse$, the while loop terminates and exits.
     The execution in the while language is defined in the evaluation rule $\textbf{ifw-false}$ as follows.
     \[
		\inferrule
		{
		 \empty
		}
		{
		\config{\ssa{
		m, \eif_w (
		\efalse, [\bar{{x}}', \bar{{x_1}}, \bar{{x_2}}],   n, 
		c; {\ewhile ~ [b]^{l} ~ \edo ~ c},
		\eskip)
		)} ,  t, w }
		\\
		\xrightarrow{} 
		\config{\ssa{m, 
		{\eskip}; \eifvar(\bar{x'}, \bar{x_i}) }, t, (w - l) }
		}
		~\textbf{ifw-false}
	\]
%
	The corresponding ssa-form evaluation as follows:
	\[
		\inferrule
		{
		 { n = 0 \rightarrow i = 1 }
		 \and
		 {n > 0 \rightarrow i =2}
		}
		{
		\config{\ssa{
		m, \eif_w (
		\efalse, [\bar{{x}}', \bar{{x_1}}, \bar{{x_2}}],   n, 
		{  
		c; \ssa{\ewhile ~ [b]^{l}, n, [\bar{{x}}', \bar{{x_1}}, \bar{{x_2}}]  ~ \edo ~ c},
		\eskip)
		} 
		)} ,  t, w }
		\\
		\xrightarrow{} 
		\config{\ssa{m, 
		{\eskip}; \eifvar(\bar{x'}, \bar{x_i}) }, t, (w - l) }
		}
		~\textbf{ssa-ifw-false}
	\]
	We can see that both traces are not changed during the exit of the while. We need to show that $ m' = \delta^{-1} (\ssa{m'[\bar{x} \mapsto m'(\bar{x_2})]}) $. We know that $[ \bar{x} \mapsto \bar{x_2}]$ in $\delta_1$ from the definition, so we can show that for any variable $\ssa{x_2} \in \bar{x_2}$, $( \ssa{x_2}, m'(x) ) \in \ssa{m'}$. For variables $x \in {\dom(\delta) - \bar{x} } $, the variable is not modified during the execution of $c$ so that we know $m(x) = m'(x)$, and then we can show that $(\delta(x), m'(x)) \in \ssa{m'} $ because $\delta(x)$ is not written in $\ssa{c'[\bar{x_1}/ \bar{x}']}$ .
%
  	\item 
		when guard in the $\eif_w$ expression evaluates to $\etrue$, the while terminates and exits.
     The execution in the while language is defined in the evaluation rule $\textbf{ifw-true}$.
          %
     We want to show that : assuming in the $i-th$ $(i < \ssa{n})$ iteration, starting with $t_i$ and $w_i$ and $m_i = \delta_1^{-1}(\ssa{m_i})$,
     this command is evaluated according to the while language operation semantics as
     	$
		\config{m, \eif_w (\etrue, c ; \ewhile ~ [b]^{l} ~ \edo ~ c, ,  \eskip) ,  t, w }
		\xrightarrow{}^* \config{m, c 
		t, (w + l) }
 		$.
     %
     Then the corresponding ssa form evaluation as follows : 
     %
     \[ 
     \inferrule{}{
     	\config{
		\ssa{
			m, 
			{
			\eif_w (\etrue, [\bar{{x}}', \bar{{x_1}}, \bar{{x_2}}], n,  
			c; \ssa{\ewhile ~ [b]^{l}, n, [\bar{{x}}', \bar{{x_1}}, \bar{{x_2}}]  ~ \edo ~ c},
			\eskip)
			} 
		},  t, w 
		}
		\\
		\xrightarrow{} 
		\config{
		\ssa{m, 
		{
		\eif_w (\etrue, [\bar{{x}}', \bar{{x_1}}, \bar{{x_2}}], n,  
		c; \ssa{\ewhile ~ [b]^{l}, n, [\bar{{x}}', \bar{{x_1}}, \bar{{x_2}}]  ~ \edo ~ c},
		}
		}
		t, (w + l) }
		} 
     \]  
     and $m_i = \delta^{-1}(\ssa{m_i}) $.
     We then have the evaluation in the while language:
     \[
		\inferrule
		{
		 \empty
		}
		{
		\config{m, 
		\eif_w (b, 
		c ; \ewhile ~ [b]^{l} ~ \edo ~ c, 
		\eskip),
		t, w }
		\xrightarrow{} 
		\config{m, 
		c ; \ewhile ~ [b]^{l} ~ \edo ~ c,  
		t, (w + l) }
		}
		~\textbf{ifw-true}
	\]
	We then have the following evaluation:
	\[
		\inferrule
		{
		 \empty
		}
		{
		\config{
		\ssa{
		m, 
		{
		\eif_w (\etrue, [\bar{{x}}', \bar{{x_1}}, \bar{{x_2}}], n,  
		c; \ssa{\ewhile ~ [b]^{l}, n, [\bar{{x}}', \bar{{x_1}}, \bar{{x_2}}]  ~ \edo ~ c},
		\eskip)
		} 
		},  t, w 
		}
		\\
		\xrightarrow{} 
		\config{
		\ssa{m, 
		{
		\eif_w (\etrue, [\bar{{x}}', \bar{{x_1}}, \bar{{x_2}}], n,  
		c; \ssa{\ewhile ~ [b]^{l}, n, [\bar{{x}}', \bar{{x_1}}, \bar{{x_2}}]  ~ \edo ~ c},
		}
		}
		t, (w + l) }
		}
		~\textbf{ssa-ifw-true}
	\]
%
By induction hypothsis on the premise $  { \Sigma; \delta_1 ; c \hookrightarrow \ssa{c_2} ; \delta_1; \Sigma_1 }$, we know that
%
\[
\config{\ssa{{m_i}, c'[ \bar{x_2}/\bar{x}'  ]}, t_i, (w_i+l)  } \to^{*} \config{\ssa{{m_{i+1}}, \eskip}, t_{i+1}, w_{i+1}  } \land m_{i+1} = \delta_1^{-1}(\ssa{m_{i+1}})
\]
%
Hence we can conclude that:
\[
  \inferrule{
   \config{\ssa{{m_i}, c'[ \bar{x_2}/\bar{x}'  ]}, t_i, (w_i+l) }  \to^{*} \config{\ssa{{m_{i+1}}, \eskip}, t_{i+1}, w_{i+1}  }
  }{
  \config{\ssa{ {m_i}, c'[ \bar{x_2}/\bar{x}'  ];  [\eloop ~ (\valr_N-i-1), n+1, [\bar{\ssa{x}}', \bar{\ssa{x_1}}, \bar{\ssa{x_2}}] ~  \edo ~ c' ]^{l} },  t_i, (w_i + l)  }  \to^{*} \\ \config{ \ssa{{m_{i+1}}, [\eloop ~ (\valr_N-i-1), n+1, [\bar{\ssa{x}}', \bar{\ssa{x_1}}, \bar{\ssa{x_2}}] ~  \edo ~ c' ]^{l}}, t_{i+1}, w_{i+1}  } 
  }
\]
So we can show that before the exit of the loop after ($v_N= n $) iterations, we have $t_{n} = t_{n}$ and $m_{n} = \delta_1^{-1}(\ssa{m_{n}})$.
 \end{enumerate}
%
This proof is similar when it comes to the exit as in case 1. 
}
\end{itemize}
%
\end{proof}
%
\clearpage
%
\todo{The following lemma describes a property of the trace-based dependency graph.
For any program $c$ with a database $D$ and a initial trace $\trace$,
the directed edges in its trace-based dependency graph can only be constructed from variable with  
smaller labels variables of greater ones.
There doesn't exist backward edges with direction from greater labeled variables to smaller ones.
\begin{lem}
\label{lem:edgeforwarding}
[Edges are Forwarding Only].
\\
%
%
$$
\forall \trace \in \mathcal{T}, D \in \dbdom \st G(c, D) =  (\vertxs, \edges, \weights, \qflag) 
\implies
\forall (\event', \event) \in \edges \st \event' \eventleq \event
$$
%
\end{lem}
%
\begin{proof}
Proof in File: {\tt ``edge\_forward.tex''}.
% \begin{proof}
This lemma is proved by showing there is a contradiction. 

\jl{
Assume there exists an edge  $(\av', \av) \in \edges$ and $\av' \avgeq \av$, where $\av' = ({\qval}',l',w')$ and $\av = ({\qval},l,w)$.
%
According to the Definition~\ref{def:trace-based_graph}, we have:
%
$$
DEP(\av', \av, c, m, D) ~ (1)
$$
%
%
Unfolding the Definition \ref{def:query_dep} in $(1)$,
we know: there exists $t_1, t_3, m_1, m_3, c_2$ s.t.,
%
\[
\config{m, c, [], []} \rightarrow^{*} 
\config{m_1, [\assign{x}{\query({\qval'})}]^{l'} ; c_2,
  t_1, w'} 
\rightarrow^{\textbf{query-v}} 
\config{m_1[v_1/x], c_2,
(t_1 ++ [\av'], w_1} \rightarrow^{*} \config{m_3, \eskip, t_3,w_3} ~ (a)  
\]
%
and
%
\[
 \bigwedge
 \begin{array}{l}   
  % 
  \left( 
  \begin{array}{l}
  \av \avin (t_3 - (t_1 ++ [\av'])) 
  % 
  \\
  \implies 
  \exists v \in \qdom, v \neq v_1, m_3', t_3', w_3'. ~  
  \config{m_1[v/x], {c_2}, t_1 ++ [\av'], w_1} 
  \\ 
  \quad \quad 
  \rightarrow^{*}
  (\config{m_3', \eskip, t_3', w_3'} 
  \land 
  \av \not \avin (t_3'-(t_1 ++ [\av'])))
\end{array} \right ) ~(b)
\\
\left( 
  \begin{array}{l}
  \av \not\avin (t_3 - (t_1 ++ [\av']))
    % 
    \\
    \implies 
  \exists v \in \qdom, v \neq v_1, m_3', t_3', w_3'. 
  \config{m_1[v/x], {c_2}, t_1 ++ [\av'], w_1}
  \\ 
  \quad \quad 
  \rightarrow^{*} 
  (\config{m_3', \eskip, t_3', w_3'} 
  \land 
  \av  \avin (t_3' - (t_1 ++ [\av'])))
\end{array} \right ) ~ (c)
\end{array}
\]
%
%
According to the Theorem \ref{thm:os_wf_trace} and $(a)$, we know both $t_1$, $t_3$ are well-formed traces.
% }
% \\
% \jl{
\\
Consider 2 cases:
 \[\av \avin (t_3 - (t_1 ++[\av'])) ~(d) ~~ \lor ~~ \av \notin_{aq} (t_3 - (t_1 ++[\av'])) ~(e)\]
%
\begin{itemize}
%
  \caseL{ (d) \[\av \avin (t_3 - (t_1 ++[\av'])) ~ (d)\]}
  By unfolding the trace subtraction operations, we have;
  \[
    \exists t_2. ~ s.t., ~ t_1 ++[\av'] ++ t_2 = t_3 \land \av \avin t_2 ~ (3)
  \]
%
%
According to the Corollary~\ref{coro:aqintrace} and $\av \avin t_2$, we have:
%
\[
  \exists t_{21}, t_{22}, \av' ~ s.t.,~ (\av \aveq \av') \land t_{21} ++ [\av'] ++ t_{22} = t_2 ~ (4)
\]
%
By rewriting $(4)$ inside $(3)$, we have:
%
\[
  \exists t_{21}, t_{22}, \av'' ~ s.t.,~ (\av \aveq \av'') \land t_1 ++[\av'] ++ t_{21} ++ [\av''] ++ t_{22} = t_3 ~ \star
\]
%
By the \emph{ordering} property in definition \ref{def:wf_trace} and $(\star)$, we know
%
\[
  \av' <_{aq} \av
\]
%
This is contradict to our assumption, where $\av' \avgeq \av$. 
%
%
\caseL{(e), \[\av \notin_{aq} (t_3 - (t_1 ++[\av'])) ~(e)\]} 
%
According to the condition $(c)$, we know: $\exists v \in \qdom, v \neq v_1, m_3', t_3', w_3'.$
    % 
\[ 
  \config{m_1[v/x], {c_2}, t_1 ++ [\av'], w_1} 
  \rightarrow^{*} (\config{m_3', \eskip, t_3', w_3'} ~ (5)
  \land \av  \in_{q} (t_3' - (t_1 ++ [\av']))) ~ (6)
\] 
%
According to the Theorem \ref{thm:os_wf_trace}, Sub-Lemma-t and $(5)$,
we know $t_3'- (t_1 ++ [\av'])$ is a well-formed trace.
\\
From $(a)$, we know that the minimal line number of $c_2$ is greater than $l'$, so we know that : 
\[
  \forall \av'' \in t_3'- (t_1 ++ [\av']), \av''>_{aq} \av'
\]
%
Unfolding the trace subtraction operations in $(6)$, we have;
\[
  \exists t_2'. ~ s.t., ~ t_1 ++[\av'] ++ t_2' = t_3' \land \av \avin t_2' ~ (7)
\]
%
%
According to the sub-lemma and $(7)$, we have:
%
\[
  \exists t_{21}', t_{22}' ~ s.t.,~ t_{21}' ++ [\av] ++ t_{22}' = t_2' ~ (8)
\]
%
By rewriting $(8)$ inside $(7)$, we have:
%
\[
  t_1 ++[\av'] ++ t_{21}' ++ [\av] ++ t_{22}' = t_3' ~ \diamond
\]
%
By the \emph{ordering} property in definition \ref{def:wf_trace} and $(\diamond)$, we know
%
\[
  \av' <_{aq} \av
\]
%
This is contradict to the assumption,  $\av' \avgeq \av$.
%
\end{itemize}
%
From both cases, we derive $\av' \avgeq \av$, which is contradict to the hypothesis, i.e., $\av' \avgeq \av$.
Then, we can conclude that 
for any directed edge $(\av', \av) \in \edges$, 
this is not the case that:
%
$$\av' \avgeq \av$$
%
}
\end{proof}
%
%
%
\end{proof}
%
%
%
\begin{lem}
\label{lem:DAG}
[Trace-based Dependency Graph is Directed Acyclic].
\\
%
$\forall \trace \in \mathcal{T}, D \in \dbdom $, $G(c, D)$ is a directed acyclic graph.
\end{lem}
}
%}
% {In this section, we formally present the definition of adaptivity of a program, which is the length of the 'longest' walk with
the most queries involved in the semantics-based dependency graph of this program. We first present the construction of the semantics-based dependency graph before the introduction of the formal definition of adaptivity. 

\subsection{Semantics-based Dependency Graph}
\label{sec:design_choice}
%%%%%%%%%%%%%%%%%%%%%%%%%%%%%%%%%%% The Detailed Version in Introducing The Dependency Graph %%%%%%%%%%%%%%%%%%%%%%%%%%%%%%%%%%%%%%%%%%%%%%%%%
% we formally define the semantics-based dependency graph as follows. There are some notations used in the definition. The labeled variables of a program $c$  
% is a subset of the labeled  variables $\mathcal{LV}$, denoted by $\lvar(c) \in \mathcal{P}(\mathcal{VAR} \times \mathcal{L}) \subseteq \mathcal{LV}$.
% \wq{I think LV(c) means all the labeled assigned variables, because in Fig.3.b, j2 is not in the graph so j2 is not in LV(tworounds), please verify. If so, maybe LV(c) is not a good name, people may think it means all the label variables, instead of just assigned ones.}
% The set of query-associated variables (in query request assignments) for a program $c$ is denoted as $\qvar(c)$, where $\qvar: \cdom \to 
% \mathcal{P}(\mathcal{LV})$. The set of initial traces of a program $c$ is a subset of the  trace universe $\trace$, in which every initial trace contains the value for all the input variables of $c$. For instance, the initial trace $\trace_0$ contains the value of input variable $k$ in the $\kw{twoRounds(k)}$ example.
\jl{
The \emph{semantics-based dependency graph} is formally defined in Definition~\ref{def:trace_graph}. 
For a program $c$, there are some notations used in the definition.
The labeled variables of $c$,
$\lvar(c) \subseteq \mathcal{LV}$ contains all the variables in $c$'s assignment commands, with the command labels as superscripts. 
The set of query-associated variables (in query request assignments),
$\qvar(c) \subseteq \lvar(c)$ contains all labeled variables in $c$'s query requests. 
The set of initial traces of $c$,
$\mathcal{T}_0(c) \subseteq \mathcal{T}$
contains all possible initial trace of $c$.
Each initial trace,  $\trace_0 \in \mathcal{T}_0(c)$ contains the initial values of all input variables of $c$. 
For instance, the initial trace of $\kw{twoRounds(k)}$ example contains the initial value of the input variable $k$.
}
\begin{defn}[Semantics-based Dependency Graph]
\label{def:trace_graph}
Given a program ${c}$,
its \emph{semantics-based dependency graph} 
$\traceG({c}) = (\traceV({c}), \traceE({c}), \traceW({c}), \traceF({c}))$ is defined as follows,
{\small
\[
\begin{array}{lll}
  \text{Vertices} &
  \traceV({c}) & := \left\{ 
  x^l
  ~ \middle\vert ~ x^l \in \lvar(c)
  \right\}
  \\
  \text{Directed Edges} &
  \traceE({c}) & := 
  \left\{ 
  (x^i, y^j) 
  ~ \middle\vert ~
  x^i, y^j \in \lvar(c) \land \vardep(x^i, y^j, c) 
  \right\}
  \\
  \text{Weights} &
  \traceW({c}) & := 
  \{ 
  (x^l, w) 
  ~ \vert ~ 
  w : \mathcal{T}_0(c) \to \mathbb{N}
  \land
  x^l \in \lvar(c) 
  \\ & &
  \quad \land
  \forall \trace_0 \in \mathcal{T}_0(c), \trace' \in \mathcal{T} \st \config{{c}, \trace_0} \to^{*} 
  \config{\eskip, \trace_0\tracecat\trace'} 
  \land w(\trace_0) = \vcounter(\trace', l) \}  
  \\
  \text{Query Annotations} &
  \traceF({c}) & := 
\left\{(x^l, n)  
~ \middle\vert ~
 x^l \in \lvar(c) \land
n = 1 \Leftrightarrow x^l \in \qvar(c) \land n = 0 \Leftrightarrow  x^l \notin \qvar(c)
\right\}
\end{array},
\]
}
%%%%%%%%%%%%%%%%%%%%%%%%%%%%%%%%%%% The Detailed Version in Explaining the Trace Operators %%%%%%%%%%%%%%%%%%%%%%%%%%%%%%%%%%%%%%%%%%%%%%%%%
% There are some operators: the trace concatenation operator $\tracecat: \mathcal{T} \to \mathcal{T} \to \mathcal{T}$, combines two traces; the counting operator $\vcounter : \mathcal{T} \to \mathbb{N} \to \mathbb{N}$, 
% which counts the occurrence of of a labeled variable in the trace. The full definitions of these above operators can be found in the appendix.
% \\
\jl{where $\tracecat: \mathcal{T} \to \mathcal{T} \to \mathcal{T}$ is the trace concatenation operator, which combines two traces, 
and $\vcounter : \mathcal{T} \to \mathbb{N} \to \mathbb{N}$ is the counting operator, 
which counts the occurrence of of a labeled variable in the trace. All the definition details are in the appendix.}
A semantics-based dependency graph $\traceG({c})= (\traceV({c}), \traceE({c}), \traceW({c}), \traceF({c}))$ 
is \emph{well-formed} if and only if $ \{x^l \ |\ (x^l,w)\in \traceW({c})\} = \traceV({c}) $.
\end{defn}


%%%%%%%%%%%%%%%%%%%%%%%%%%%%%%%%%%% The Explanation of Dependency Graph %%%%%%%%%%%%%%%%%%%%%%%%%%%%%%%%%%%%%%%%%%%%%%%%%
% \wq{ The occurrence time is computed by the counter operator $w(\trace) = \vcounter(\trace', l)$. As an instance, in the semantics-based dependency graph of $\kw{twoRounds}$ in Figure~\ref{fig:overview-example}(b), the weight of vertices in the while loop $w_k$ is a function which returns the value $n$ of variable $k$ in the starting trace $\tau$, and the commands in the loop are indeed executed $n$ times when starting with $\trace$.}
% Every vertex also has a query annotation, mapping each $x^l \in \traceV(c)$ to $0$ or $1$, 
% indicating whether the vertex comes from a query request assignment (1) or not (0) by checking if the labeled variable $x^l$ of the vertex is in $\qvar(c)$.
% One interesting point is that instead of building a graph whose vertices are just variables coming from query request assignments, we choose the combination of all the labeled assigned variables of the program and the query annotations. The reason is that the results of previous queries can be stored or used in variables
% which aren't associated to the query request,
% it is necessary to track the dependency between all the assigned variables of the program. 
% The main novelty is the combination of the quantitative and dependency information on the semantics-based dependency graph, which means this graph can tell both the dependency between queries and the times they depend on each other.
% The vertices $\traceV({c})$ of a program $c$ are the labeled assigned variables, which are statically collected. The weight function on every vertex $w : \mathcal{T} \to \mathbb{N}$
% mapping from a starting trace $\trace \in \mathcal{T}_0(c)$ to a natural number, tracks the occurrence times of this vertex in the newly generated trace $\trace'$ when the program $c$ is evaluated from the starting trace $\config{{c}, \trace} \to^{*} \config{\eskip, \trace\tracecat\trace'} $.
% One interesting point is that instead of building a graph whose vertices are just variables coming from query request assignments, we choose the combination of all the labeled assigned variables of the program and the query annotations. The reason is that the results of previous queries can be stored or used in variables
% % which aren't associated to the query request,
% it is necessary to track the dependency between all the assigned variables of the program. 
\jl{There are four components in this graph.
\begin{enumerate}
    \item The vertices $\traceV({c})$ of a program $c$ are all its labeled variables, $\lvar(c)$ which are statically collected.
    \item $\traceF(c)$ contains the \emph{query annotation} for 
    every vertex $x^l \in \traceV(c)$. It indicates whether $x^l$ comes from a query request (1) or not (0) by checking if the labeled variable $x^l$ of the vertex is in $\qvar(c)$.
    \item Edges in $\traceE(c)$ are built from the $\vardep(x^i, y^j, c)$ relation between two labeled variables.
    This is the key definition in order to formalize the intuitive \emph{may-dependency} relation between queries and the \emph{adaptivity}. We present this formalization detail in Section~\ref{sec:dep} below.
    \item 
    The weight function in $\traceW(c)$ for each vertex, $w : \mathcal{T} \to \mathbb{N}$
maps from a starting trace $\trace_0 \in \mathcal{T}_0(c)$ to a natural number.
For each vertex $x^l$, it tracks its visiting times (i.e., the evaluation times of the command with the label $l$) when the program $c$ is evaluated from the initial trace $\trace_0$ into $\eskip$, $\config{{c}, \trace_0} \to^{*} \config{\eskip, \trace_0\tracecat\trace'} $.
The visiting times is computed by the counter operator $\vcounter(\trace', l)$
by counting the occurrence of the label $l$ in $\trace'$.
As an instance, in the semantics-based dependency graph of $\kw{twoRounds}$ in Figure~\ref{fig:overview-example}(b), the weight, $w_k$ of the vertex $x^3$ is a function of type $\mathcal{T}_0(\kw{twoRound(k)}) \to \mathbb{N}$.
Given input $\trace_0$, we execute the program under $\trace_0$ as $\config{\kw{twoRound(k)}, \trace_0} \to^{*} \config{\eskip, \trace_0\tracecat\trace'} $. Then $w_k(\trace_0)$ outputs the occurrence time of the label $3$ in $\trace'$.
\end{enumerate}
The main novelty of  the semantics-based dependency graph is the combination of the quantitative and dependency information. 
It can tell both the dependency between queries via the directed edge, and the times they depend on each other via the weight.
}

\subsection{May-Dependency}
\label{sec:dep}
%%%%%%%%%%%%%%%%%%%%%%%%%%%%%%%%%%% The Detail Explanation of Variable May-Dependency and Motivation on How to define it %%%%%%%%%%%%%%%%%%%%%%%%%%%%%%%%%%%%%%%%%%%%%%%%%
% \wq{we think the query $\query(\chi[2])$ (assigned variable $y^3$) may depend on the query $\query(\chi[1]$ (assigned variable $x^1$). 
  %  but vulnerable to queries request protected by differential privacy mechanisms. In our loop language, a query $q(e)$ represents a query request to the database through a mechanism, which add random noise to protect the return results. In this setting, the results of one query will be randomized due to the noise attached by the mechanism which fails the first candidate because witnessing the results of one query can no longer tells whether the change of the results comes from another query or the change of noise of the differential privacy mechanism. For example, suppose we have a program $p$ which requests two simple queries $q_1()$ and $q_2()$ with no arguments as follows.
%   \[
%   c_1 =\assign{x}{\query(\chi[2])} ;\assign{y}{\query(\chi[3] + x)}.
%   \]
%  $ c = \assign{x}{\query(\chi[1])} ; \assign{y}{\query(\chi[2])}$,
%  and 
% Specifically, in the {\tt Query While} language, the query request is composed by two components: a symbol $\query$ representing a linear query type and 
% % an argument
% the query expression $\qexpr$ as an argument, 
% which represents the function specifying what the query asks. 
% From our perspective, $\query(\chi[1])$ is different from $\query(\chi[2]))$. Informally,  
%
% in this example: $c_1 = \assign{x}{\query(0)}; \assign{z}{\query(\chi[x])}$.
% This candidate definition works well 
% Nevertheless, the first definition fails to catch control dependency because it just monitors the changes to a query, but misses the appearance of the query when the answers of its previous queries change. For instance, it fails to handle $}
%       c_2 = \assign{x}{\query(\chi[1])} ; \eif( x > 2 , \assign{y}{\query(\chi[2])}, \eskip )
%   $, but the second definition can. However, it only considers the control dependency and misses the data dependency. This reminds us to define a \emph{may-dependency} relation between labeled variables by combining the two definitions to capture the two situations.
%  $ p = \assign{x}{\query(\chi[1])} ; \assign{y}{\query(\chi[2])}$. 
% This candidate definition works well with respect to data dependency. However, if fails to handle control dependency since it just monitors the changes to the answer of a query when the answer of previous queries returned change. The key point is that this query may also not be asked because of an analyst decision which depend on the answers of previous queries. An example of this situation is shown in program $p_1$ as follows.
% There are two possible situations that a query will be "affected" by previous queries' results,  
% either when the query expression directly uses the results of previous queries (data dependency), or when the control flow of the program with respect to a query (whether to ask this query or not) depends on the results of previous queries (control flow dependency). To this end, our assigned variable dependency definition has the following two cases.   
% {
% \begin{enumerate}
%     \item One variable may depend on a previous variable if and only if a change of the value assigned to the previous variable may also change the value assigned to the variable.
%     \item One variable may depend on a previous variable if and only if a change of the value assigned to the previous variable may also change the appearance of the assignment command to this variable 
%     % in\wq{during?} 
%     during execution.
% \end{enumerate}
% }
%
%  The first case captures the data dependency. 
% For instance, in a simple program $c_1 =[\assign{x}{\query(\chi[2])}]^1 ;[\assign{y}{\query(\chi[3] + x)}]^2$, we think $\query(\chi[3] + x)$ (variable $y^2$) may depend on the query $\query(\chi[2]))$ (variable $x^1$), because the equipped function of the former $\chi[3] + x$ may depend on the data stored in x assigned with the result of $\query(\chi[2]))$. From our perspective, $\query(\chi[1])$ is different from $\query(\chi[2]))$. The second case captures the control dependency, for instance, in the program $
%       c_2 = [\assign{x}{\query(\chi[1])}]^1 ; \eif( [x > 2]^2 , [\assign{y}{\query(\chi[2])}]^3, [\eskip]^4 )
%
This section formalizes the \emph{may-dependency} relation between queries and introduces the
\emph{variable may-dependency} definition.

\jl{There are two possible situations that a query will be ``influenced'' by previous queries' results,
where either the query request is changed when the results of previous queries are changed (data dependency),
or the  query request is disappeared when the results of previous queries are changed (control dependency). In this sense, our formal dependency definition considers both the two cases as follows,
\begin{enumerate}
    \item One query may depend on a previous query if and only if a change of the value returned
    to the previous query request may also change this query request.
    \item One query may depend on a previous query if and only if a change of the value returned
    to the previous query request may also change the appearance of this query quest.
\end{enumerate}
The first case captures the data dependency. 
For instance, in a simple program $c_1 =[\assign{x}{\query(\chi[2])}]^1 ;[\assign{y}{\query(\chi[3] + x)}]^2$, we think $\query(\chi[3] + x)$ (variable $y^2$) may depend on the query $\query(\chi[2]))$ (variable $x^1$), because the equipped function of the former $\chi[3] + x$ may depend on the data stored in x assigned with the result of $\query(\chi[2]))$. From our perspective, $\query(\chi[1])$ is different from $\query(\chi[2]))$.
\\
The second case captures the control dependency.
For instance, in the program
$c_2 = [\assign{x}{\query(\chi[1])}]^1 ; \eif( [x > 2]^2 , [\assign{y}{\query(\chi[2])}]^3, [\eskip]^4 )$, 
we think the query $\query(\chi[2])$ ( or the labeled variable $y^3$) may depend on the query $\query(\chi[1])$ (via the labeled variable $x^1$). }

\jl{ 
Since both of the two ``influences'' are passing through labeled variables, we choose to formally define the \emph{may-dependency}
relation over all labeled variables, and then recover the query requests from query-associated variables, $\qvar(c)$.
It relies on the formal observation of the ``influence'' via events in Definition~\ref{def:diff} and the \emph{may-dependency} between events in Definition~\ref{def:event_dep}.
}
  \begin{defn}[Events Differ in Value ($\diff$)]
    \label{def:diff}
    Two events $\event_1, \event_2 \in \eventset$ differ in their value, or query value,
    denoted as $\diff(\event_1, \event_2)$, if and only if:
    {\small
    \begin{subequations}
    \begin{align}
    & \pi_1(\event_1) = \pi_1(\event_2) 
      \land  
      \pi_2(\event_1) = \pi_2(\event_2) \\
    & \land  
      \big(
       (\pi_3(\event_1) \neq \pi_3(\event_2)
      \land 
      \pi_{4}(\event_1) = \pi_{4}(\event_2) = \bullet )
      \lor 
      (\pi_4(\event_1) \neq \bullet
      \land 
      \pi_4(\event_2) \neq \bullet
      \land 
      \pi_{4}(\event_1) \neq_q \pi_{4}(\event_2)) 
      \big)
    \end{align}
    \label{eq:diff}
    \end{subequations}
    }
    where $\qexpr_1 =_{q} \qexpr_2$ denotes the semantics equivalence between query values,
    and $\pi_i$ projects the $i$-th element from the quadruple of an event.
    \end{defn}
%%%%%%%%%%%%%%%%%%%%%%%%%%%%%%%%%%% The "Detail" (I think redundant) Explanation of Variable May-Dependency and Motivation on How to define it %%%%%%%%%%%%%%%%%%%%%%%%%%%%%%%%%%%%%%%%%%%%%%%%%
% We use $\qexpr_1 =_{q} \qexpr_2$ and $\qexpr_1 \neq_{q} \qexpr_2$ to notate query expression equivalence and in-equivalence, distinct from standard equality. 
% First of all, $\diff(\event_1, \event_2)$ only compares the 'value' assigned to the same labeled variable so the two events share the same labeled variable (the first two elements of the event quadruple). If the labeled variable are not the same, it is trivially false. Next, the 'value' has different meaning according to the type of the assignment. If the two assignment events both are generated from a standard assignment command (by checking whether the fourth element of the event is $\bullet$), we think they are different up to their value if the value assigned to the variable (the third element of the event) varies. On the other hand, if they both come from query request assignment commands, we directly compare the query function (the fourth element of the event). We do not consider the case when one event is generated from the standard assignment and the other comes from the query request assignment, which will not hold. The motivation of defining  $\diff(\event_1, \event_2)$ is that it allows us to check whether the value assigned to a specific variable is changed or not in two runs, which is very important in our dependency case 1. 
\jl{
$\pi_1(\event_1) = \pi_1(\event_2) 
  \land  
  \pi_2(\event_1) = \pi_2(\event_2)$ at Eq.\ref{eq:diff}(a)
requires that $\event_1$ and $\event_2$ have the same variable name and label. 
This guarantees that $\event_1$ and $\event_2$ are generated from the same labeled command.
In Eq.\ref{eq:diff}(b),
two kinds of comparisons between the third and fourth element are for the non-query assignment and query request separately.
For events generated from the non-query assignments (via checking
$\pi_{4}(\event_1) =_q \pi_{4}(\event_2) = \bullet$), we only compare their assigned values through $\pi_3(\event_1) \neq \pi_3(\event_2)$.
But for these from query requests (via checking
$\pi_{4}(\event_1) \neq \bullet \land \pi_{4}(\event_2) \neq \bullet$),
we are comparing their query expressions by $\pi_{4}(\event_1) \neq_q \pi_{4}(\event_2)$ rather than the assigned value computed from the unknown database server.
This matches the intuitive data dependency between queries, where one query is influenced by others as long as the query request is changed.
}

\jl{Below is the \emph{event may-dependency} between events based on formally observing their differences via $\diff$.}
\begin{defn}[Event May-Dependency]
\label{def:event_dep}
An event $\event_2$ is in the \emph{event may-dependency} relation with an assignment event $\event_1 \in \eventset^{\asn}$ in a program ${c}$  with a hidden database $D$ and a witness trace $\trace \in \mathcal{T}$,
$\eventdep(\event_1, \event_2, [\event_1 ] \tracecat \trace \tracecat [\event_2], c, D)$ if and only if
\begin{subequations}
\begin{align}
&  
\exists \trace_0, \trace_1, \trace' \in \mathcal{T},\event_1' \in \eventset^{\asn}, {c}_1, {c}_2  \in \cdom  \st \diff(\event_1, \event_1') \land \\
& 
\quad (\exists  \event_2' \in \eventset \st 
\left(
\begin{array}{ll}   
  & \config{{c}, \trace_0} \rightarrow^{*} 
  \config{{c}_1, \trace_1 \tracecat [\event_1]}  \rightarrow^{*} 
  \config{{c}_2,  \trace_1 \tracecat [\event_1] \tracecat \trace \tracecat [\event_2] } 
   \\ 
   \bigwedge &
   \config{{c}_1, \trace_1 \tracecat [\event_1']}  \rightarrow^{*}
    \config{{c}_2,  \trace_1 \tracecat[ \event_1'] \tracecat \trace' \tracecat [\event_2'] } 
  \\
  \bigwedge & 
  \diff(\event_2,\event_2' ) \land 
  \vcounter(\trace, \pi_2(\event_2))
  = 
  \vcounter(\trace', \pi_2(\event_2'))\\
  \end{array}
  \right)\\ 
  & 
  \quad
  \lor 
  \left(
  \begin{array}{l} 
  \exists \trace_3, \trace_3'  \in \mathcal{T}, \event_b \in \eventset^{\test} \st 
  \\
   \quad \config{{c}, \trace_0} \rightarrow^{*} \config{{c}_1, \trace_1 \tracecat [\event_1]}  \rightarrow^{*}
   \config{c_2,  \trace_1 \tracecat [\event_1] \tracecat
   \trace \tracecat [\event_b] \tracecat  \trace_3} 
\\ \quad \land
\config{{c}_1, \trace_1 \tracecat [\event_1']}  \rightarrow^{*} 
\config{c_2,  \trace_1 \tracecat [\event_1'] \tracecat \trace' \tracecat [(\neg \event_b)] \tracecat \trace_3'} 
\\
\quad \land \tlabel({\trace_3}) \cap \tlabel({\trace_3'})
= \emptyset
\land \vcounter(\trace', \pi_2(\event_b)) = \vcounter(\trace, \pi_2(\event_b)) 
    \land \event_2 \in \trace_3
    \land \event_2 \not\in \trace_3'
  \end{array}
  \right)
  ),
\end{align}
\label{eq:eventdep}
\end{subequations}
where $\tlabel(\trace) \subseteq \ldom$ is the set of the labels in all the events from trace $\trace$ and $\event_2 \in \trace_3$ or $\event_2 \notin \trace_3$ denotes that $\event_2$ belongs to $\trace_3$ or not.
\end{defn}
The first line in Eq.~\ref{eq:eventdep}(a) requires that $\event_1$ comes from an assignment command and then modifies its assigned value via $\diff(\event_1, \event_1')$.

\jl{Then, the following two parts in Eq~\ref{eq:eventdep}(b) and (c) capture the intuitive value dependency and control dependency respectively. 
Both parts execute the program two times w.r.t. the different values in $\event_1$ (as line:1 in Eq~\ref{eq:eventdep}(b) and line:2 in Eq~\ref{eq:eventdep}(c))
and $\event_1'$ (as line:2 in Eq~\ref{eq:eventdep}(b) and line:3 in Eq~\ref{eq:eventdep}(c)), 
but observe the difference in the newly generated traces in different ways (via $3$rd line in Eq~\ref{eq:eventdep}(b) and $4$th line in Eq~\ref{eq:eventdep}(c)). This idea is similar to the dependency definition from \cite{Cousot19a}.
}

\jl{In Eq~\ref{eq:eventdep}(b) line:2, if the newly generated trace, $\trace' ++ [\event_2']$ still contains $\event_2$ as $\event_2'$, we check the difference on their value in line:3.
If they only differ in their assigned values, i.e., $\diff(\event_2, \event_2')$ and
they are in the same loop iteration (via $\vcounter(\trace, \pi_2(\event_2)) = \vcounter(\trace', \pi_2(\event_2'))$),
then we say there is a value \emph{may-dependency} relation between $\event_1$ and $\event_2$.}

\jl{The Eq~\ref{eq:eventdep}(c) captures the control dependency through observing the disappearance $\event_2$ from newly generated traces, $\trace' \tracecat [(\neg \event_b)] \tracecat \trace_3'$ in the second execution (line:3).
$\event_2 \in \trace_3 \land \event_2 \not\in \trace_3'$ in Eq~\ref{eq:eventdep}(c) line:4 specifies this disappearance.
$\vcounter(\trace', \pi_2(\event_b)) = \vcounter(\trace, \pi_2(\event_b))$ is used to make sure the two executions are
in the same loop iteration as well.
Different from Eq~\ref{eq:eventdep}(b) line:3,
we use a testing event, $\event_b$ here because
$\vcounter(\trace, \pi_2(\event_2)) = \vcounter(\trace', \pi_2(\event_2'))$ cannot guarantee the disappearance if there are nested loops.
This is correct because the control dependency can only be passed through the guard of if or while command,
and this guard must be evaluated into two different values ($\event_b$ and $\neg \event_b$) in the two executions.
}

\jl{
Then Considering all the assignment events newly generated during a program’s executions, 
as long as there is one pair of events satisfying the \emph{event may-dependency}, 
we say that the two labeled variables in the two assignment events satisfy the \emph{variable may-dependency} relation below.}

\begin{defn}[Variable May-Dependency]
  \label{def:var_dep}
  A variable ${x}_2^{l_2} \in \lvar(c)$ is in the \emph{variable may-dependency} relation with another
  variable ${x}_1^{l_1} \in \lvar(c)$ in a program ${c}$, denoted as 
  %
  $\vardep({x}_1^{l_1}, {x}_2^{l_2}, {c})$, if and only if.
\[
  \begin{array}{l}
\exists \event_1, \event_2 \in \eventset^{\asn}, \trace \in \mathcal{T} , D \in \dbdom \st
\pi_{1}{(\event_1)}^{\pi_{2}{(\event_1)}} = {x}_1^{l_1}
\land
\pi_{1}{(\event_2)}^{\pi_{2}{(\event_2)}} = {x}_2^{l_2}% \\ \quad 
\land 
\eventdep(\event_1, \event_2, \trace, c, D) 
  \end{array}
\]  %
  \end{defn}
\jl{From the definition, a labeled assigned variables $x_2^{l_2}$ may depend on another labeled assigned variable $x_1^{l_1}$ in a program $c$ under the hidden database $D$, 
as long as there exist two assignment events $\event_1$ (for $x_1^{l_1}$) and $\event_2$ for $x_2^{l_2}$
satisfy the \emph{event may-dependency} relation under a witness trace $\trace$.  
}
%%%%%%%%%%%%%%%%%%%%%%%%%%%%%%%%%%%%%%%Trace-Based Adaptivity%%%%%%%%%%%%%%%%%%%%%%%%%%%%%%%
\subsection{Trace-based Adaptivity}
Given 
a program $c$'s semantics-based dependency graph 
$\traceG({c})$,
we define adaptivity 
with respect to an initial trace $\trace_0 \in \mathcal{T}_0(c)$ by the finite walk in the graph, which has the most query requests along the walk.
We show the definition of a finite walk as follows.

\begin{defn}[Finite Walk (k)]
\label{def:finitewalk}
Given the semantics-based dependency graph $\traceG({c}) = (\traceV, \traceE, \traceW, \traceF)$ of a program $c$, a \emph{finite walk} $k$ in $\traceG({c})$ is a function $k$.
Given an input initial trace $\trace_0 \in \mathcal{T}_0(c)$, $k(\trace_0)$ is a sequence of edges $(e_1 \ldots e_{n - 1})$ 
for which there is a sequence of vertices $(v_1, \ldots, v_{n})$ such that:
\begin{itemize}
\item $e_i = (v_{i},v_{i + 1}) \in \traceE$ for every $1 \leq i < n$.
\item every $v_i \in \traceV$ and $(v_i, w_i) \in \traceW$, $v_i$ appears in $(v_1, \ldots, v_{n})$ at most $w(\trace_0)$ times.  
\end{itemize}
The length of $k(\trace_0)$ is the number of vertices in its vertices sequence, i.e., $\len(k)(\trace_0) = n$.
\end{defn} 
\jl{$\walks(\traceG(c))$ is 
the set of all the finite walks $k$ in $\traceG(c)$,
and $k_{v_1 \to v_2} \in \walks(\traceG(c))$ denotes the walk from vertex $v_1$ to $v_2$.}

\jl{Because the adaptivity are intuitively describing the dependency between queries,
so we calculate a special ``length'', the \emph{query length} of a walk by counting only the vertices
corresponding to queries. This is formally defined below.}
\begin{defn}[Query Length of the Finite Walk($\qlen$)]
\label{def:qlen}
Given 
the semantics-based dependency graph $\traceG({c})$ of a program $c$,
 and a \emph{finite walk} 
 $k \in \walks(\traceG(c))$. 
The query length of $k$, $\qlen(k)$ is a function $\mathcal{T}_0(c) \to \mathbb{N}$, such that given an input initial trace $\trace_0$, $\qlen(k)(\trace_0)$ is
the number of vertices which correspond to query variables in the vertex sequence, $(v_1, \ldots, v_{n})$ as follows, 
\[
  \qlen(k)(\trace_0) = |\big( v \mid v \in (v_1, \ldots, v_{n}) \land \qflag(v) = 1 \big)|.
\]
\end{defn}
Then the definition of adaptivity is presented in Definition~\ref{def:trace_adapt} below.
\begin{defn}
    [Adaptivity of a Program]
    \label{def:trace_adapt}
    Given a program ${c}$, 
    its adaptivity $A(c)$ is function 
    $A(c) : \mathcal{T} \to \mathbb{N}$ such that for an
    initial trace $\trace_0 \in \mathcal{T}_0(c)$, 
   $$
    A(c)(\trace_0) = \max \big 
    \{ \qlen(k)(\trace_0) \mid k \in \walks(\traceG(c)) \big \} $$
    \end{defn}

\subsection{The Walk Through Example}
{\small
\begin{figure}
\centering
\begin{subfigure}{.2\textwidth}
\begin{centering}
$
    \begin{array}{l}
    \kw{towRounds(k)} \triangleq \\
           \clabel{ \assign{a}{0}}^{0} ;
            \clabel{\assign{j}{k} }^{1} ; \\
            \ewhile ~ \clabel{j > 0}^{2} ~ \edo ~ \\
            \Big(
             \clabel{\assign{x}{\query(\chi[j] \cdot \chi[k])} }^{3}  ; \\
             \clabel{\assign{j}{j-1}}^{4} ;\\
            \clabel{\assign{a}{x + a}}^{5}       \Big);\\
            \clabel{\assign{l}{\query(\chi[k]*a)} }^{6}\\
        \end{array}
$
\caption{}
\end{centering}
\end{subfigure}
\begin{subfigure}{.4\textwidth}
%}
\qquad
\begin{centering}
\begin{tikzpicture}[scale=\textwidth/16cm,samples=250]
\draw[] (0, 10) circle (0pt) node
{{ $a^0: {}^{\lambda \trace_0. 1}_{0}$}};
\draw[] (0, 7) circle (0pt) node
{\textbf{$x^3: {}^{\lambda \trace_0. \env(\trace_0) k}_{1}$}};
\draw[] (0, 4) circle (0pt) node {{ $a^5: {}^{\lambda \trace_0. \env(\trace_0) k}_{0}$}};
\draw[] (0, 1) circle (0pt) node
{{ $l^6: {}^{\lambda \trace_0. 1}_{1}$}};
% Counter Variables
\draw[] (8, 9) circle (0pt) node {\textbf{$j^1: {}^{\lambda \trace_0. 1}_{0}$}};
\draw[] (8, 6) circle (0pt) node {{ $j^4: {}^{\lambda \trace_0. \env(\trace_0) k}_{0}$}};
%
% Value Dependency Edges:
\draw[ ultra thick, -latex, densely dotted,] (0, 1.5)  -- (0, 3.5) ;
\draw[ ultra thick, -latex, densely dotted,] (0, 4.5)  -- (0, 6.5) ;
\draw[ thick, -latex] (0, 4.5)  to  [out=-230,in=230]  (0, 9.5) ;
\draw[ thick, -Straight Barb] (1.5, 3.8) arc (120:-200:1);
\draw[ thick, -Straight Barb] (9, 6.5) arc (150:-150:1);
\draw[ thick, -latex] (8, 6.5)  -- (8, 8.5) ;
\draw[ thick, -latex] (0, 1.5)  to  [out=-230,in=230]  (0, 9.5) ;
% Control Dependency
\draw[ thick,-latex] (2, 7)  -- (6, 9) ;
\draw[ thick,-latex] (2, 4.5)  -- (6, 9) ;
\draw[ thick,-latex] (2, 7)  -- (6, 6) ;
\draw[ thick,-latex] (2, 4.5)  -- (6, 6) ;
\end{tikzpicture}
\caption{}
\end{centering}
\end{subfigure}
   \begin{subfigure}{.36\textwidth}
   \begin{centering}
   \begin{tikzpicture}[scale=\textwidth/18cm,samples=200]
\draw[] (0, 10) circle (0pt) node
{{ $a^0: {}^1_{0}$}};
\draw[] (0, 7) circle (0pt) node
{\textbf{$x^3: {}^{k}_{1}$}};
\draw[] (0, 4) circle (0pt) node
{{ $a^5: {}^{k}_{0}$}};
\draw[] (0, 1) circle (0pt) node
{{ $l^6: {}^{1}_{1}$}};
% Counter Variables
\draw[] (5, 9) circle (0pt) node {\textbf{$j^1: {}^{1}_{0}$}};
\draw[] (5, 6) circle (0pt) node {{ $j^4: {}^{k}_{0}$}};
%
% Value Dependency Edges:
\draw[ ultra thick, -latex, densely dotted,] (0, 1.5)  -- (0, 3.5) ;
\draw[ ultra thick, -latex, densely dotted,] (0, 4.5)  -- 
% node [left] {\highlight{$\trace_0 \to \env(\trace_0) k $}}
(0, 6.5) ;
\draw[ thick, -latex] (0, 4.5)  to  [out=-230,in=230]  
% node [left] {\highlight{$\trace_0 \to \env(\trace_0) k $}}
(0, 9.5) ;
\draw[ thick, -Straight Barb] (1.5, 3.5) arc (120:-200:1);
\draw[ thick, -Straight Barb] (6.5, 6.5) arc (150:-150:1);
    % The Weight for this edge
    % \draw[](9, 6) node [] {\highlight{$\trace_0 \to \env(\trace_0) k  $}};
\draw[ thick, -latex] (5, 6.5)  -- (5, 8.5) ;
% Control Dependency
\draw[ thick,-latex] (1.5, 7)  -- (4, 9) ;
\draw[ thick,-latex] (1.5, 4)  -- (4, 9) ;
\draw[ thick,-latex] (1.5, 7)  -- (4, 6) ;
\draw[ thick,-latex] (1.5, 4)  -- (4, 6) ;
\draw[ thick, -latex] (0, 1.5)  to  [out=-230,in=230]  (0, 9.5) ;
\end{tikzpicture}
\caption{}
   \end{centering}
   \end{subfigure}
\vspace{-0.4cm}
 \caption{(a) The program $\kw{towRounds(k)}$, an example 
%  of a program 
with two rounds of adaptivity (b) The corresponding execution-based dependency graph (c) The program-based dependency graph from $\THESYSTEM$.
}
\label{fig:overview-example}
% \vspace{-0.8cm}
\end{figure}
}}