%%
%% This is file `sample-acmsmall-conf.tex',
%% generated with the docstrip utility.
%%
%% The original source files were:
%%
%% samples.dtx  (with options: `acmsmall-conf')
%% 
%% IMPORTANT NOTICE:
%% 
%% For the copyright see the source file.
%% 
%% Any modified versions of this file must be renamed
%% with new filenames distinct from sample-acmsmall-conf.tex.
%% 
%% For distribution of the original source see the terms
%% for copying and modification in the file samples.dtx.
%% 
%% This generated file may be distributed as long as the
%% original source files, as listed above, are part of the
%% same distribution. (The sources need not necessarily be
%% in the same archive or directory.)
%%
%%
%% Commands for TeXCount
%TC:macro \cite [option:text,text]
%TC:macro \citep [option:text,text]
%TC:macro \citet [option:text,text]
%TC:envir table 0 1
%TC:envir table* 0 1
%TC:envir tabular [ignore] word
%TC:envir displaymath 0 word
%TC:envir math 0 word
%TC:envir comment 0 0
%%
%%
%% The first command in your LaTeX source must be the \documentclass
%% command.
%%
%% For submission and review of your manuscript please change the
%% command to \documentclass[manuscript, screen, review]{acmart}.
%%
%% When submitting camera ready or to TAPS, please change the command
%% to \documentclass[sigconf]{acmart} or whichever template is required
%% for your publication.
%%
%%
\documentclass[manuscript,acmsmall,anonymous,review,screen,nonacm=true, authorversion=true,drafttrue]{acmart}

%%
%% \BibTeX command to typeset BibTeX logo in the docs
\AtBeginDocument{%
  \providecommand\BibTeX{{%
    Bib\TeX}}}

%% Rights management information.  This information is sent to you
%% when you complete the rights form.  These commands have SAMPLE
%% values in them; it is your responsibility as an author to replace
%% the commands and values with those provided to you when you
%% complete the rights form.
\setcopyright{acmcopyright}
\copyrightyear{2018}
\acmYear{2018}
\acmDOI{XXXXXXX.XXXXXXX}

%% These commands are for a PROCEEDINGS abstract or paper.
\acmConference[Conference acronym 'XX]{Make sure to enter the correct
  conference title from your rights confirmation emai}{June 03--05,
  2018}{Woodstock, NY}
%%
%%  Uncomment \acmBooktitle if the title of the proceedings is different
%%  from ``Proceedings of ...''!
%%
%%\acmBooktitle{Woodstock '18: ACM Symposium on Neural Gaze Detection,
%%  June 03--05, 2018, Woodstock, NY}
\acmPrice{15.00}
\acmISBN{978-1-4503-XXXX-X/18/06}


%%
%% Submission ID.
%% Use this when submitting an article to a sponsored event. You'll
%% receive a unique submission ID from the organizers
%% of the event, and this ID should be used as the parameter to this command.
%%\acmSubmissionID{123-A56-BU3}

%%
%% For managing citations, it is recommended to use bibliography
%% files in BibTeX format.
%%
%% You can then either use BibTeX with the ACM-Reference-Format style,
%% or BibLaTeX with the acmnumeric or acmauthoryear sytles, that include
%% support for advanced citation of software artefact from the
%% biblatex-software package, also separately available on CTAN.
%%
%% Look at the sample-*-biblatex.tex files for templates showcasing
%% the biblatex styles.
%%

%%
%% The majority of ACM publications use numbered citations and
%% references.  The command \citestyle{authoryear} switches to the
%% "author year" style.
%%
%% If you are preparing content for an event
%% sponsored by ACM SIGGRAPH, you must use the "author year" style of
%% citations and references.
%% Uncommenting
%% the next command will enable that style.
%%\citestyle{acmauthoryear}

\usepackage{booktabs}   %% For formal tables:
                        %% http://ctan.org/pkg/booktabs
\usepackage{subcaption} %% For complex figures with subfigures/subcaptions
                        %% http://ctan.org/pkg/subcaption

\usepackage{hhline}
\newcommand{\defeq}{\mathrel{\doteq}}

\newcommand{\lzero}{0}

\newcommand{\kw}[1]{\mathtt{#1}}

\newcommand{\expr}{e}
\newcommand{\vall}{w}
\newcommand{\valr}{v}
\newcommand{\eif}{\kw{if}}
\newcommand{\eapp}{\;}
\newcommand{\eprojl}{\kw{fst}}
\newcommand{\eprojr}{\kw{snd}}
%\newcommand{\eprov}[1]{\eta_{#1}}
\newcommand{\etrue}{\kw{true}}
\newcommand{\efalse}{\kw{false}}
\newcommand{\econst}{c}
\newcommand{\eop}{\delta}
\newcommand{\efix}{\mathop{\kw{fix}}}
%\newcommand{\labelA}{\ell}

\newcommand{\tr}{T}
\newcommand{\trift}{\eif^{\kw{t}}}
\newcommand{\triff}{\eif^{\kw{f}}}
\newcommand{\trprojl}{\eprojl}
\newcommand{\trprojr}{\eprojr}
\newcommand{\trtrue}{\etrue}
\newcommand{\trfalse}{\efalse}
\newcommand{\trconst}{\econst}
\newcommand{\trop}{\eop}
\newcommand{\trfix}{\efix}
\newcommand{\trapp}[5]{#1 \; #2 \mathrel{\triangleright} {\efix #3(#4).#5}}

\newcommand{\adap}{\kw{adap}}
\newcommand{\ddep}[1]{\kw{depth}_{#1}}
\newcommand{\nat}{\mathbb{N}}
\newcommand{\natb}{\nat_{\bot}}
\newcommand{\natbi}{\natb^\infty}
\newcommand{\nnatA}{n}
\newcommand{\nnatB}{m}
\newcommand{\nnatbA}{s}
\newcommand{\nnatbB}{t}
\newcommand{\nnatbiA}{q}
\newcommand{\nnatbiB}{r}

\newcommand{\type}{\tau}
\newcommand{\tbase}{\kw{b}}
\newcommand{\tbool}{\kw{bool}}
\newcommand{\tarr}[5]{#1; #3 \xrightarrow{#4; \, #5} #2}
\newcommand{\env}{\theta}

\newcommand{\bigstep}{\mathrel{\Downarrow}}

\newcommand{\dmap}{\rho}
\newcommand{\dmapb}{\bot_\dmap}
\newcommand{\supp}{\kw{supp}}
\newcommand{\dom}{\kw{dom}}

\newcommand{\tvdash}[1]{\vdash_{#1}}

\newcommand{\THESYSTEM}{\textsf{AdaptFun}}

%%
%% end of the preamble, start of the body of the document source.
\begin{document}

%%
%% The "title" command has an optional parameter,
%% allowing the author to define a "short title" to be used in page headers.
\title[short]{Program Analysis for Adaptive Data Analysis}  

%%
%% The "author" command and its associated commands are used to define
%% the authors and their affiliations.
%% Of note is the shared affiliation of the first two authors, and the
%% "authornote" and "authornotemark" commands
%% used to denote shared contribution to the research.
% \author{Ben Trovato}
% \authornote{Both authors contributed equally to this research.}
% \email{trovato@corporation.com}
% \orcid{1234-5678-9012}
% \author{G.K.M. Tobin}
% \authornotemark[1]
% \email{webmaster@marysville-ohio.com}
% \affiliation{%
%   \institution{Institute for Clarity in Documentation}
%   \streetaddress{P.O. Box 1212}
%   \city{Dublin}
%   \state{Ohio}
%   \country{USA}
%   \postcode{43017-6221}
% }

% \author{Lars Th{\o}rv{\"a}ld}
% \affiliation{%
%   \institution{The Th{\o}rv{\"a}ld Group}
%   \streetaddress{1 Th{\o}rv{\"a}ld Circle}
%   \city{Hekla}
%   \country{Iceland}}
% \email{larst@affiliation.org}

% \author{Valerie B\'eranger}
% \affiliation{%
%   \institution{Inria Paris-Rocquencourt}
%   \city{Rocquencourt}
%   \country{France}
% }

% \author{Aparna Patel}
% \affiliation{%
%  \institution{Rajiv Gandhi University}
%  \streetaddress{Rono-Hills}
%  \city{Doimukh}
%  \state{Arunachal Pradesh}
%  \country{India}}

% \author{Huifen Chan}
% \affiliation{%
%   \institution{Tsinghua University}
%   \streetaddress{30 Shuangqing Rd}
%   \city{Haidian Qu}
%   \state{Beijing Shi}
%   \country{China}}

% \author{Charles Palmer}
% \affiliation{%
%   \institution{Palmer Research Laboratories}
%   \streetaddress{8600 Datapoint Drive}
%   \city{San Antonio}
%   \state{Texas}
%   \country{USA}
%   \postcode{78229}}
% \email{cpalmer@prl.com}

% \author{John Smith}
% \affiliation{%
%   \institution{The Th{\o}rv{\"a}ld Group}
%   \streetaddress{1 Th{\o}rv{\"a}ld Circle}
%   \city{Hekla}
%   \country{Iceland}}
% \email{jsmith@affiliation.org}

% \author{Julius P. Kumquat}
% \affiliation{%
%   \institution{The Kumquat Consortium}
%   \city{New York}
%   \country{USA}}
% \email{jpkumquat@consortium.net}

%%
%% By default, the full list of authors will be used in the page
%% headers. Often, this list is too long, and will overlap
%% other information printed in the page headers. This command allows
%% the author to define a more concise list
%% of authors' names for this purpose.
% \renewcommand{\shortauthors}{Trovato et al.}

%%
%% The abstract is a short summary of the work to be presented in the
%% article.
\begin{abstract}
  Data analyses are usually designed to identify some property of the population from which the data are drawn, generalizing beyond the specific data sample. For this reason, data analyses are often designed in a way that guarantees that they produce a low generalization error.
  That is, they are designed so that the result of a data analysis run on a sample data does not differ too much from the result one would achieve by running the analysis over the entire population. 
  
  An adaptive data analysis can be seen as a process composed by multiple queries interrogating some data, where the choice of which query to run next may rely on the results of previous queries. 
  The generalization error of each individual query/analysis can be controlled by using an array of well-established statistical techniques.
  However, when queries are arbitrarily composed, the different errors can propagate through the chain of different queries and bring to a high generalization error. 
  To address this issue, data analysts are designing several techniques that not only guarantee bounds on the generalization errors of single queries, but that also guarantee bounds on the generalization error of the composed analyses. 
  The choice of which of these techniques to use, often depends on the chain of queries that an adaptive data analysis can generate.
  
  In this work, we consider adaptive data analyses implemented as while-like programs and we design a program analysis which can help with identifying which technique to use to control their generalization errors. 
  More specifically, we formalize the intuitive notion of \emph{adaptivity} as a quantitative property of programs. 
  We do this because the adaptivity level of a data analysis is a key measure to choose the right technique. 
  Based on this definition, we design a program analysis for soundly approximating this quantity.
  The program analysis generates a representation of the data analysis as a weighted dependency graph, where the weight is an upper bound on the number of times each variable can be reached, and uses a path search strategy to guarantee an upper bound on the adaptivity. 
  We implement our program analysis  and show that it can help to analyze the adaptivity of several concrete data analyses with different adaptivity structures.
  \end{abstract}

%%
%% The code below is generated by the tool at http://dl.acm.org/ccs.cfm.
%% Please copy and paste the code instead of the example below.
%%
% \begin{CCSXML}
% <ccs2012>
%  <concept>
%   <concept_id>00000000.0000000.0000000</concept_id>
%   <concept_desc>Do Not Use This Code, Generate the Correct Terms for Your Paper</concept_desc>
%   <concept_significance>500</concept_significance>
%  </concept>
%  <concept>
%   <concept_id>00000000.00000000.00000000</concept_id>
%   <concept_desc>Do Not Use This Code, Generate the Correct Terms for Your Paper</concept_desc>
%   <concept_significance>300</concept_significance>
%  </concept>
%  <concept>
%   <concept_id>00000000.00000000.00000000</concept_id>
%   <concept_desc>Do Not Use This Code, Generate the Correct Terms for Your Paper</concept_desc>
%   <concept_significance>100</concept_significance>
%  </concept>
%  <concept>
%   <concept_id>00000000.00000000.00000000</concept_id>
%   <concept_desc>Do Not Use This Code, Generate the Correct Terms for Your Paper</concept_desc>
%   <concept_significance>100</concept_significance>
%  </concept>
% </ccs2012>
% \end{CCSXML}

% \ccsdesc[500]{Do Not Use This Code~Generate the Correct Terms for Your Paper}
% \ccsdesc[300]{Do Not Use This Code~Generate the Correct Terms for Your Paper}
% \ccsdesc{Do Not Use This Code~Generate the Correct Terms for Your Paper}
% \ccsdesc[100]{Do Not Use This Code~Generate the Correct Terms for Your Paper}

%%
%% Keywords. The author(s) should pick words that accurately describe
%% the work being presented. Separate the keywords with commas.
\keywords{Adaptive data analysis, program analysis, dependency graph}
%% A "teaser" image appears between the author and affiliation
%% information and the body of the document, and typically spans the
%% page.
% \begin{teaserfigure}
%   \includegraphics[width=\textwidth]{sampleteaser}
%   \caption{Seattle Mariners at Spring Training, 2010.}
%   \Description{Enjoying the baseball game from the third-base
%   seats. Ichiro Suzuki preparing to bat.}
%   \label{fig:teaser}
% \end{teaserfigure}

\received{20 February 2007}
\received[revised]{12 March 2009}
\received[accepted]{5 June 2009}

%%
%% This command processes the author and affiliation and title
%% information and builds the first part of the formatted document.
\maketitle

%%%%%%%%%%%%%%%%%%%%%%%%%%%%%%%%%%%%%%%%%%% Introduction and Overview %%%%%%%%%%%%%%%%%%%%%%%%%%%%%%%%%%%%%%%%%%% 
\section{Introduction}
\label{sec:intro}
% The topic, motivation, the importance of adaptivity 
Consider a dataset $X$ consisting of $n$ independent samples from some unknown population $\dist$.  How can we ensure that the conclusions drawn from $X$ \emph{generalize} to the population $\dist$?  Despite decades of research in statistics and machine learning on methods for ensuring generalization, there is an increased recognition that many scientific findings generalize poorly (e.g. 
\cite{Ioannidis05,GelmanL13}
).  While there are many reasons a conclusion might fail to generalize, one that is receiving increasing attention is \emph{adaptivity}, which occurs when the choice of method for analyzing the dataset depends on previous interactions with the same dataset~\cite{GelmanL13}.
%
 Adaptivity can arise from many common practices, such as exploratory data analysis, using the same data set for feature selection and regression, and the re-use of datasets across research projects.  Unfortunately, adaptivity invalidates traditional methods for ensuring generalization and statistical validity, which assume that the method is selected independently of the data. The misinterpretation of adaptively selected results has even been blamed for a ``statistical crisis'' in empirical science~\cite{GelmanL13}.
%  ~\cite{GelmanL13}.

\begin{figure}
    \centering
    \includegraphics[width=0.7\columnwidth]{overview.png}
    \caption{Overview of our Adaptive Data Analysis model.
    We have a population that we are interested in studying, and a dataset containing individual samples from this population. 
    The adaptive data analysis we are interested in running has access to the dataset through queries of some pre-determined family (e.g., statistical or linear queries) mediated by a mechanism. 
    This mechanism uses randomization to reduce the generalization error of the queries issued to the data.}
    \label{fig:adaptivity-model-overview}
\vspace{-0.5cm}
\end{figure}

A line of work initiated by \cite{DworkFHPRR15}, \cite{HardtU14} posed the question: Can we design \emph{general-purpose} methods that ensure generalization in the presence of adaptivity, together with guarantees on their accuracy?  
The idea that has emerged in these works is to use randomization to help ensure generalization. 
Specifically, these works have proposed to mediate the access of an adaptive data analysis to the data by means of queries from some pre-determined family (we will consider here a specific family of queries often called "statistical" or "linear" queries) that are sent to a  \emph{mechanism} which uses some randomized process to guarantee that the result of the query does not depend too much on the specific
sampled dataset. 
This guarantees that the result of the queries generalizes well. This approach is described in Fig.~\ref{fig:adaptivity-model-overview}.  
This line of work has identified many new algorithmic techniques for ensuring generalization in adaptive data analysis, leading to algorithms with greater statistical power than all previous approaches. Common methods proposed by these works include, the addition of noise to the result of a query, data splitting, etc. Moreover, these works have also identified problematic strategies for adaptive analysis, showing limitations on the statistical power one can hope to achieve. Subsequent works have then further extended the methods and techniques in this approach and further extended the theoretical underpinning of this approach, e.g.~\cite{dwork2015reusable,dwork2015generalization,BassilyNSSSU16,UllmanSNSS18,FeldmanS17,jung2019new,SteinkeZ20,RogersRSSTW20}.

A key development in this line of work is that the best method for ensuring generalization in an adaptive data analysis depends to a large extent on the number of \emph{rounds of adaptivity}, the depth of the chain of queries. 
As an informal example, the program $x \leftarrow q_1(D);y \leftarrow q_2(D,x);z \leftarrow q_3(D,y)$ has three rounds of adaptivity, since $q_2$  depends on $D$ not only directly because it is one of its input but also via the result of $q_1$, which is also run on $D$, and similarly,  $q_3$ depends on $D$ directly but also via the result of $q_2$, which in turn depends on the result of $q_1$.
The works we discussed above showed that, not only does the analysis of the generalization error depend on the number of rounds, but knowing the number of rounds actually allows one to choose methods that lead to the smallest possible generalization error - we will discuss this further in Section~\ref{sec:overview}. 

For example, these works showed that when an adaptive data analysis uses a large number of rounds of adaptivity then a low generalization error can be achieved by a mechanism  
adding to the result of each query Gaussian noise scaled to the number of rounds. When instead  an adaptive data analysis uses a small number of rounds of adaptivity then a low generalization error can be achieved by using more specialized methods, such as data splitting mechanism or the reusable holdout technique from~\cite{DworkFHPRR15}.
To better understand this idea, we show in Fig.~\ref{fig:generalization_errors} three experiments showcasing these situations.
More precisely, in Fig.~\ref{fig:generalization_errors}(a) we show the results of a specific analysis\footnote{We will use formally a program implementing this analysis (Fig.~\ref{fig:overview-example}) as a running example in the rest of the paper.} with two rounds of adaptivity.
This analysis can be seen as a classifier which first runs 400 non-adaptive queries on the first 400 attributes of the data, looking for correlations between the attributes and a label, and then runs one last query which depends on all these correlations.
Without any mechanism the generalization error of the last query is pretty large, and the lower generalization error is achieved when the data-splitting method is used.
Fig.~\ref{fig:generalization_errors}(c) shows how this situation also change with the number of queries. Specifically, it shows the root mean square error of the last \emph{adaptive} query when the numbers queries varies. This also highlight the fact that different mechanisms, for the same analysis, produce results with very different generalization error.
In Fig.~\ref{fig:generalization_errors}(b), we show the results of a specific analysis\footnote{We will present this analysis formally in Section~\ref{sec:examples}.} with four hundreds rounds of adaptivity.
At each step, this analysis runs an adaptive query based on the results of the previous ones. Without any mechanism, the generalization error of most of the queries is pretty large, and this error can be lowered by using Gaussian noise. 
{\small
\begin{figure}
\centering
\begin{subfigure}{.32\textwidth}
\begin{centering}
\includegraphics[width=1.0\textwidth]{tworound.png}
\caption{}
\end{centering}
\end{subfigure}
\quad
\begin{subfigure}{.32\textwidth}
\begin{centering}
\includegraphics[width=1.0\textwidth]{multipleround.png}
\caption{}
\end{centering}
\end{subfigure}
\begin{subfigure}{.32\textwidth}
\begin{centering}
\includegraphics[width=1.0\textwidth]{twoRounds-rmse-fourmechs.png}
\caption{}
\end{centering}
\end{subfigure}
\vspace{-0.5cm}
 \caption{
 The generalization errors of two adaptive data analysis examples, under different choices of mechanisms.
 (a) Data analysis with 2 rounds adaptivity, 
 (b) Data analysis with 400 rounds adaptivity.
 (c) Same Data analysis as (a) with different query numbers.
}
\label{fig:generalization_errors}
\vspace{-0.6cm}
\end{figure}
}
%gap

This scenario motivates us to explore the design of program analysis techniques that can be used to estimate the number of \emph{rounds of adaptivity} that a program implementing a data analysis can perform. These techniques could be used to help a data analyst in the choice of the mechanism to use,
and they
could ultimately be integrated into a tool for adaptive data analysis such as the \emph{Guess and Check} framework by~\cite{RogersRSSTW20}. 

The first problem we face is \emph{how to formally define} a model for adaptive data analysis which is general enough to support the methods we discussed above and which would permit to formulate the notion of adaptivity these methods use. We take the approach of designing a programming framework for submitting queries to some \emph{mechanism} giving access to the data mediated by one of the techniques we mentioned before, e.g., adding Gaussian noise, randomly selecting a subset of the data, using the reusable holdout technique, etc. In this approach, a program models an \emph{analyst} asking a sequence of queries to the mechanism. The mechanism runs the queries on the data applying one of the methods above and returns the result to the program. The program can then use this result to decide which query to run next. Overall, we are interested in controlling the generalization of the query results returned by the mechanism, by means of the adaptivity. 

The second problem we face is \emph{how to define the adaptivity of a given program}.
Intuitively, a query $Q$ may depend on another query $P$, if there are two values that $P$ can return which affect in different ways the execution of $Q$. 
For example, as shown in \cite{dwork2015reusable}, and as we did in our example in Fig.~\ref{fig:generalization_errors}(a), one can design a machine learning algorithm for constructing a classifier which first computes each feature's correlation with the label via a sequence of queries, and then constructs the classifier based on the correlation values. If one feature's correlation changes, the classifier depending on features is also affected.  
This notion of dependency builds on the execution trace as a \emph{causal history}. In particular, we are interested in the history or provenance of a query up until this is executed, we are not then concerned about how the result is used --- except for tracking whether the result of the query may further cause some other query. This is because we focus on the generalization error of queries and not their post-processing. % 
To formalize this intuition as a quantitative program property,
we use a trace semantics recording the execution history of programs on some given input --- and we create a dependency graph, where the dependency between different variables (queries are also assigned to variables) is explicit and track which variable is associated with a query request. We then enrich this graph with weights describing the number of times each variable is evaluated in a program evaluation starting with an initial state. The adaptivity is then defined as the length of the walk visiting most query-related variables on this graph\footnote{Formally, graphs will be well-defined only for terminating programs, this will guarantee that the longest walk is finite}. In other words, we define adaptivity as a \emph{quantitative form of program dependency}.

% \jl{ 
% To define adaptivity in our programming framework, we consider a weighted dependency graph over variables assigned in the program, where each edge is built by a semantics dependency relation between these variables. The dependency relation relies on the trace semantics of our programming framework which records the execution history of programs implementing adaptive data analysis. 
% %The novelty comes from the definition of relation of dependency between nodes, which consists of the edge in the graph. For now, we can think of each node is associated with a variable, storing the value assigned to its variable 
% }
% \jl{The general idea beneath this dependency relation is that modifying the value of some variable in an execution trace will later affect the following execution trace.
% By tracking if a variable is assigned by a query or not, we are able to distinguish whether one query may depend on the other.}

The third problem we face is \emph{how to estimate the adaptivity of a given program}. 
The adaptive data analysis model we consider and our definition of adaptivity suggest that for this task we can use a  program analysis that is based on some form of dependency analysis. This analysis needs to take into consideration:
1) the fact that, in general, a query $Q$ is not a monolithic block but rather it may depend, through the use of variables and values, on other parts of the program. Hence, it needs to consider some form of data flow analysis. 
2) the fact that, in general, the decision on whether to run a query or not may depend on some other value. Hence, 
 it needs to consider some form of control flow analysis.
 3) the fact that, in general, we are not only interested in whether there is a dependency or not, but in the length of the chain of dependencies. Hence, it needs to consider some quantitative information about the program dependencies. 
 
To address these considerations and be able to estimate a sound upper bound on the adaptivity of a program, 
we develop a static program analysis algorithm, named {\THESYSTEM}, which combines data flow and control flow analysis with reachability bound analysis~\cite{GulwaniZ10}. This combination gives tighter bounds on the adaptivity of a program than the ones one would achieve by directly using the data and control flow analyses or the ones that one would achieve by directly using reachability bound analysis techniques alone. We evaluate {\THESYSTEM} on a number of examples showing that it is able to efficiently estimate precise upper bounds on the adaptivity of different programs. 
All the proofs and extended definitions can be found in the supplementary material.

To summarize, our work aims at the design of a static analysis for programs implementing adaptive analysis that can estimate their rounds of adaptivity. Specifically, our contributions are:
\begin{enumerate}
    \item A programming framework for adaptive data analyses where programs represent analysts that can query generalization-preserving mechanisms mediating the access to some data. 
    \item 
    A formal definition of the notion of adaptivity under the analyst-mechanism model. 
    This definition is built on a variable-based dependency graph that is constructed using sets of program execution traces.
    \item 
    A static program analysis algorithm {\THESYSTEM} combining data flow, control flow and  reachability bound analysis in order to provide tight bounds on the adaptivity of a program.
    \item A soundness proof of the program analysis showing that the adaptivity estimated by {\THESYSTEM} bounds the true adaptivity of the program. 
    \item An implementation of {\THESYSTEM} and an experimental evaluation of the bounds this implementation provides on several examples.
\end{enumerate}
\section{Overview}
\label{sec:overview}
\subsection{Some results in Adaptive Data Analysis}
%\wq{I think we can move this subsection into appendix. Maybe just leave theorm 1.2 and 1.3}
%\jl{I don't agree}
In Adaptive Data Analysis, an \emph{analyst} is interested in studying some distribution $\dist$ over some domain $\univ$.  Following previous works~\cite{DworkFHPRR15,HardtU14,BassilyNSSSU16}, we focus on the setting where the analyst is interested in answers to \emph{statistical queries} (also known as \emph{linear queries}) over the distribution.  A statistical query is usually defined by some function $\qquery \from \univ \to [-1,1]$ (often other codomains such as $[0,1]$ or $[-R,+R]$, for some $R$, are considered).  The analyst wants to learn the \emph{population mean}, which is defined as 
$\qquery(\dist) = \ex{\sample \sim \dist}{\qquery(\sample)}$. 
%
We assume that the distribution $\dist$ can only be accessed via a set of \emph{samples} $\sample_1,\dots,\sample_n$ drawn independently and identically distributed (i.i.d.) from $\dist$.  These samples are held by a mechanism $\mech(\sample_1,\dots,\sample_n)$ who receives the query $\qquery$ and computes an answer 
$\answer \approx \qquery(\dist)$.
%
The na\"ive way to approximate the population mean is to use the \emph{empirical mean}, which (abusing notation) is defined as 
$\qquery(\sample_1,\dots,\sample_n) = \frac{1}{n} \sum_{i=1}^{n} \qquery(X_i)$.
However, the mechanism $M$ can adopt some methods for improving the generalization error $| a- \qquery(\dist)|$.

In this work we consider analysts that ask a sequence of $k$ queries $\qquery_1,\dots,\qquery_k$.  If the queries are all chosen in advance, independently of the answers $a_1,\dots,a_k$ of each other, then we say they are \emph{non-adaptive}.  If the choice of each query $\qquery_j$ depends on the prefix $\qquery_1,\answer_1,\dots,\qquery_{j-1},\answer_{j-1}$ then they are \emph{fully adaptive}.  An important intermediate notion is \emph{$\qrounds$-round adaptive}, where the sequence can be partitioned into $\qrounds$ batches of non-adaptive queries.  Note that non-adaptive queries are $1$-round and fully adaptive queries are $k$-round adaptive.

We now review what is known about the problem of answering $r$-round adaptive queries.  
\begin{thm}[\cite{BassilyNSSSU16}] 
\label{thm:nonadapt-adapt}
\begin{enumerate}

\item For any distribution $\dist$, and any $k$ \emph{non-adaptive} statistical queries, with high probability,
% $$
$
\max_{j=1,\dots,k} | \answer_j - \qquery_j(\dist) | = O\left( \sqrt{\frac{\log k}{n}}  \right)
% $$
$.
%
\item For any distribution $\dist$, and  any $k$  \emph{$\qrounds$-round adaptive} statistical queries, with $\qrounds \geq 2$, with high probability, the empirical mean (rounded to an appropriate number of bits of precision)\footnote{With infinite precision even two queries may give unbounded error, when the first query's result encodes the whole data.} satisfies:\\
% $$
$
\max_{j=1,\dots,k} | \answer_j - \qquery_j(\dist) | = O\left( \sqrt{  \frac{k}{n}}  \right)
% $$
$
\end{enumerate}
\end{thm}
In fact, these bounds are tight (up to constant factors) which means that even allowing one extra round of adaptivity leads to an exponential increase in the generalization error, from $\log k$ to $k$.

\citet{DworkFHPRR15} and \citet{BassilyNSSSU16} showed that by using carefully calibrated Gaussian noise in order to limit the dependency of a single query on the specific data instance, one 
can actually achieve much stronger generalization error as a function of the number of queries, specifically.
\begin{thm}[\cite{DworkFHPRR15, BassilyNSSSU16}] \label{thm:gaussiannoise} For any distribution $\dist$, any $k$, any $\qrounds \geq 2$ and any \emph{$\qrounds$-round adaptive} statistical queries, if we answer queries with carefully calibrated Gaussian noise, with high probability,  we have:
\begin{center}
  $
\max_{j=1,\dots,k} | \answer_j - \qquery_j(\dist) | = O\left( \frac{\sqrt[4]{k}}{\sqrt{n}}  \right)
$  
\end{center}
\end{thm}
% Notice that in order to Theorem~\ref{thm:gaussiannoise} has different quantification in that the optimal choice of mechanism depends on the number of queries.  Thus, we need to know the number of queries \emph{a priori} to choose the best mechanism.
More interestingly, \citet{DworkFHPRR15}
also gave a refined bounds that can be achieved with different mechanisms depending on the number of rounds of adaptivity.   \begin{thm}[\cite{DworkFHPRR15}] \label{thm:gaussiannoise2} For any $r$ and $k$, there exists a mechanism such that for any distribution $\dist$, and any $\qrounds \geq 2$ any \emph{$\qrounds$-round adaptive} statistical queries, with high probability, it satisfies
\begin{center}
  $
\max_{j=1,\dots,k} | \answer_j - \qquery_j(\dist) | = O\left( \frac{r \sqrt{\log k}}{\sqrt{n}}  \right)
$  
\end{center}
\end{thm}
Notice that Theorem~\ref{thm:gaussiannoise2} has different quantification in that the optimal choice of mechanism depends on the number of queries {and number of rounds of adaptivity}.  This suggests that if one knows a good \emph{a priori upper bound on the number of rounds of adaptivity}, one can choose the appropriate mechanism and get a much better guarantee in terms of the generalization error.
As an example, as we can see in Fig.~\ref{fig:generalization_errors}, if we know that an algorithm is 2-rounds adaptive, we can choose data splitting as {the} mechanism, while if we know that an algorithm has many rounds of adaptivity we can choose Gaussian noise. It is worth to stressing that by knowing the number of rounds of adaptivity one can also compute a concrete upper bound on the generalization error of a data analysis. This information allows one to have a quantitative, a priori, estimation of the effectiveness of a data analysis. 
This motivates us to design a static program analysis aimed at giving good \emph{a priori} upper bounds on the number of rounds of adaptivity of a program. 

{\small
\begin{figure}
\centering
\begin{subfigure}{.2\textwidth}
\begin{centering}
$
    \begin{array}{l}
    \kw{towRounds(k)} \triangleq \\
           \clabel{ \assign{a}{0}}^{0} ;
            \clabel{\assign{j}{k} }^{1} ; \\
            \ewhile ~ \clabel{j > 0}^{2} ~ \edo ~ \\
            \Big(
             \clabel{\assign{x}{\query(\chi[j] \cdot \chi[k])} }^{3}  ; \\
             \clabel{\assign{j}{j-1}}^{4} ;\\
            \clabel{\assign{a}{x + a}}^{5}       \Big);\\
            \clabel{\assign{l}{\query(\chi[k]*a)} }^{6}\\
        \end{array}
$
\caption{}
\end{centering}
\end{subfigure}
\begin{subfigure}{.4\textwidth}
%}
\qquad
\begin{centering}
\begin{tikzpicture}[scale=\textwidth/16cm,samples=250]
\draw[] (0, 10) circle (0pt) node
{{ $a^0: {}^{\lambda \trace_0. 1}_{0}$}};
\draw[] (0, 7) circle (0pt) node
{\textbf{$x^3: {}^{\lambda \trace_0. \env(\trace_0) k}_{1}$}};
\draw[] (0, 4) circle (0pt) node {{ $a^5: {}^{\lambda \trace_0. \env(\trace_0) k}_{0}$}};
\draw[] (0, 1) circle (0pt) node
{{ $l^6: {}^{\lambda \trace_0. 1}_{1}$}};
% Counter Variables
\draw[] (8, 9) circle (0pt) node {\textbf{$j^1: {}^{\lambda \trace_0. 1}_{0}$}};
\draw[] (8, 6) circle (0pt) node {{ $j^4: {}^{\lambda \trace_0. \env(\trace_0) k}_{0}$}};
%
% Value Dependency Edges:
\draw[ ultra thick, -latex, densely dotted,] (0, 1.5)  -- (0, 3.5) ;
\draw[ ultra thick, -latex, densely dotted,] (0, 4.5)  -- (0, 6.5) ;
\draw[ thick, -latex] (0, 4.5)  to  [out=-230,in=230]  (0, 9.5) ;
\draw[ thick, -Straight Barb] (1.5, 3.8) arc (120:-200:1);
\draw[ thick, -Straight Barb] (9, 6.5) arc (150:-150:1);
\draw[ thick, -latex] (8, 6.5)  -- (8, 8.5) ;
\draw[ thick, -latex] (0, 1.5)  to  [out=-230,in=230]  (0, 9.5) ;
% Control Dependency
\draw[ thick,-latex] (2, 7)  -- (6, 9) ;
\draw[ thick,-latex] (2, 4.5)  -- (6, 9) ;
\draw[ thick,-latex] (2, 7)  -- (6, 6) ;
\draw[ thick,-latex] (2, 4.5)  -- (6, 6) ;
\end{tikzpicture}
\caption{}
\end{centering}
\end{subfigure}
   \begin{subfigure}{.36\textwidth}
   \begin{centering}
   \begin{tikzpicture}[scale=\textwidth/18cm,samples=200]
\draw[] (0, 10) circle (0pt) node
{{ $a^0: {}^1_{0}$}};
\draw[] (0, 7) circle (0pt) node
{\textbf{$x^3: {}^{k}_{1}$}};
\draw[] (0, 4) circle (0pt) node
{{ $a^5: {}^{k}_{0}$}};
\draw[] (0, 1) circle (0pt) node
{{ $l^6: {}^{1}_{1}$}};
% Counter Variables
\draw[] (5, 9) circle (0pt) node {\textbf{$j^1: {}^{1}_{0}$}};
\draw[] (5, 6) circle (0pt) node {{ $j^4: {}^{k}_{0}$}};
%
% Value Dependency Edges:
\draw[ ultra thick, -latex, densely dotted,] (0, 1.5)  -- (0, 3.5) ;
\draw[ ultra thick, -latex, densely dotted,] (0, 4.5)  -- 
% node [left] {\highlight{$\trace_0 \to \env(\trace_0) k $}}
(0, 6.5) ;
\draw[ thick, -latex] (0, 4.5)  to  [out=-230,in=230]  
% node [left] {\highlight{$\trace_0 \to \env(\trace_0) k $}}
(0, 9.5) ;
\draw[ thick, -Straight Barb] (1.5, 3.5) arc (120:-200:1);
\draw[ thick, -Straight Barb] (6.5, 6.5) arc (150:-150:1);
    % The Weight for this edge
    % \draw[](9, 6) node [] {\highlight{$\trace_0 \to \env(\trace_0) k  $}};
\draw[ thick, -latex] (5, 6.5)  -- (5, 8.5) ;
% Control Dependency
\draw[ thick,-latex] (1.5, 7)  -- (4, 9) ;
\draw[ thick,-latex] (1.5, 4)  -- (4, 9) ;
\draw[ thick,-latex] (1.5, 7)  -- (4, 6) ;
\draw[ thick,-latex] (1.5, 4)  -- (4, 6) ;
\draw[ thick, -latex] (0, 1.5)  to  [out=-230,in=230]  (0, 9.5) ;
\end{tikzpicture}
\caption{}
   \end{centering}
   \end{subfigure}
\vspace{-0.4cm}
 \caption{(a) The program $\kw{towRounds(k)}$, an example 
%  of a program 
with two rounds of adaptivity (b) The corresponding execution-based dependency graph (c) The program-based dependency graph from $\THESYSTEM$.
}
\label{fig:overview-example}
% \vspace{-0.8cm}
\end{figure}
}


\subsection{ {\THESYSTEM} formally through an example.}
We illustrate the key technical components of our framework through a simple adaptive data analysis with two rounds of adaptivity.
% They are 1. the query while language for expressing a data analysis formally, 2. the definition of \emph{adaptivity} (\emph{adaptivity} is the short for \emph{rounds of adaptivity} used in the rest of the paper) based on the language semantics, and 3. the static analysis algorithm providing a sound upper bound on a data analysis' adaptivity.
% }
% \detailed{
% In "two rounds strategy" analysis, the analyst asks in total $k+1$ queries to the mechanism in two phases, the symbol $k$ is an input from the data analyst of this strategy and has no limit on the kind, which can be a constant, or a symbol or even an expression such as $(k+3)*2$.
% } 
%
In this analysis, an analyst asks $k+1$ queries to a mechanism in two phases.
In the first phase, the analyst asks $k$ queries and stores the answers that are provided by the mechanism. In the second phase, the analyst constructs a new query based on the results of the previous $k$ queries and sends this query to the mechanism. 
The mechanism is abstract here and our goal is to use static analysis to provide an upper bound on adaptivity to help choose the mechanism.
This data analysis assumes that the data domain $\univ$ 
contains at least $k$ numeric attributes 
(every query in the first phase focuses on one), which we index just by natural numbers.
The implementation of this data analysis in the language of {\THESYSTEM} is presented in Fig.~\ref{fig:overview-example}(a).

The {\THESYSTEM} language extends a standard while language\footnote{Programs components are labeled, so that we can uniquely identify every component.} with a query request constructor denoted $\query$.
 Queries have the form $\query(\qexpr)$, where $\qexpr$ is a special expression (see syntax in Section~\ref{sec:loop_language}) 
representing a function $\from \univ \to U$ on rows of an hidden database that is only accessible through the mechanisms. The domain $\univ$ of this function is the (arbitrary) domain of rows of the database. The codomain $U$ of this function is the query output space which, depending on the specific program, could be $[-1,1]$, $[0,1]$ or $[-R,+R]$, for some $R$. We use this formalization because we are interested in linear queries which, as we discussed in the previous section, compute the empirical mean of functions on rows.
 As an example, $x \leftarrow \query(\chi[j] \cdot \chi[k])$ computes an approximation, according to the used mechanism, of the empirical mean of the product of the $j^{th}$ attribute and $k^{th}$ attribute, identified by $\chi[j] \cdot \chi[k]$. Notice that we don't materialize the mechanism but we assume that it is implicitly run when we execute the query. 
 In Fig.~\ref{fig:overview-example}(a), the queries inside the while loop correspond to the first phase of the data analysis and compute the sum of the empirical mean of
the product of the $j$th attribute with the $k$th attribute. 
The query outside the loop corresponds to the second phase and computes an approximation of the empirical mean of the last attribute weighted by the sum of the empirical mean of the first $k$ attributes.


This example is intuitively 2-rounds adaptive since we have two clearly distinguished phases, and the queries that we ask in the first phase do not depend on each other (the query $\chi[j] \cdot \chi[k]$ at line $3$ only relies on the counter $j$ and input $k$), while the last query 
(at line 6) depends on the results of all the previous queries. 
However, capturing this concept formally is surprisingly challenging. The difficulty comes from the quantitative nature of this concept and how this quantitative nature interacts with data and control dependency. We describe how we capture it next. 
% \mg{this is weaker than it was in the previous submission.}

%%%%%%%%%%%%%%%%%%%%%%%%%%%%%%%%%%%Some details that might be useful when make passes %%%%%%%%%%%%%%%%%
% \jl{ The $\bullet$ stands for no query, for instance, the second event in the trace $(j, 1, \env(\trace)k , \bullet) $ tells us the assignment at line $1$ does not request a query.} \jl{The third event is a testing event corresponding to the guard of the while loop at line $2$. The evaluation of the query request in the second phase is tracked in }
% % \jl{ 
% The $\bullet$ is a default value for non-query event, 
% for instance, the second event in the trace $(j, 1, K , \bullet) $ tells us the assignment at line $1$ does not request a query.
% The third event is a testing event corresponding to the guard of the while loop at line $2$. The evaluation of the query request in the second phase is tracked in 
% % }
\subsubsection{Adaptivity definition}
\label{sec:adaptivity-informal}
%%%%%%%%%%%%%%%%%%%%%%%%%%%%%%%%%%% Details Below that might be useful when make passes %%%%%%%%%%%%%%%%%
% \detailed{To formally define the adaptivity, we build a directed graph representing the possible dependencies between queries of a program and we call this graph: execution-based dependency graph. The vertices represent the assigned program variables and the edges satisfy the dependency relations between vertices.   Fig.~\ref{fig:overview-example}(b) is the execution dependency graph we build based on the "two rounds strategy program" in Fig.~\ref{fig:overview-example}(a). In brief, the graph is built by collecting the assigned variables with labels of the target program as vertices, which are $a^0$, $j^1$,...$a^5$,$l^6$. We check if there is an edge between two vertices by our dependency relation over two labeled variables (defined in Section~\ref{sec:dep_adaptivity} ). This dependency relation relies on the execution of the program recorded by a trace generated by our trace semantics, which is the reason we call this graph "execution-based". 
% Intuitively from Fig.~\ref{fig:overview-example}(a), the query in the second phase (at line 6) depends on the query results in the first phase stored in $a$ at line 5, and the variable $a$ also relies on the queries at line 3. Correspondingly, we have two edges $(l^6, a^5)$ and $(a^5, x^3)$ in our execution-based dependency graph in Fig.~\ref{fig:overview-example}(b). Besides, we also have special edge which is a circle, to track any variable being updated with its previous value recursively. For instance, the counter $j$ and the variable $a$ are updated based on previous values $k$ times in the first phase and we see two circle edges on $a^5$ and $j^4$.}

The central property we are after in this work is the \emph{adaptivity of a program}. We define formally this notion in three steps (details in Section~\ref{sec:adaptivity}). First, we define a notion of dependency, or better \emph{may-dependency}, between variables. To do this we take inspiration from previous works on dependency analysis and information flow control and we say that a variable \emph{may depend} on another one if changing the execution of the latter can affect the execution of the former. 
We can see in Fig.~\ref{fig:overview-example}(a) that the value of the variable $l$, which corresponds to the result of the execution of the query in the second phase (in the command with label 6), is affected by the value of the variable $x$, which corresponds to the result of the execution of the query at line 3 in the first phase, via the variable $a$.
To formally define this notion of dependency, as in information flow control, we use the execution history of programs recorded by a trace semantics (see Definition~\ref{def:var_dep}).
% \mg{Please, double check that I refer to the right definition. }  

Second, we build an annotated weighted directed graph representing the possible dependencies between labeled variables. We call this graph the \emph{semantics-based dependency graph} to stress that this graph summarizes the dependencies we could see if we knew the overall behavior of the program. 
The vertices of the graph are the assigned program variables with the label of their assignments, edges are pairs of labeled variables which satisfy the dependency relations, weights are functions associated with vertices and describe the number of times the assignment corresponding to the vertex is executed when the program is run in a given starting state\footnote{In our trace semantics the state is recorded in the trace, so an initial state is actually represented by an initial trace. We will use this terminology in later sections.}, and the annotations, which we call \emph{query annotations}, are bits associated with vertices and describe if the corresponding assignment comes from a query (1) or not (0).
The \emph{semantics-based dependency graph} of the $\kw{twoRounds(k)}$ program
we gave in Fig.~\ref{fig:overview-example}(a) is described in Fig.~\ref{fig:overview-example}(b) (we use dashed arrows for two edges that will be highlighted in the next step, for the moment these can be considered similar to the other edges---i.e. solid arrows).
\review{More explanation of the semantics-based dependency graph in Figure 3, including the lambda tau. rho (tau) k notation. Also, consider replacing it with notation that is easier to search using PDF readers such as lastVal(tau, k).}
We have all the variables that are assigned in the program with their labels, and edges representing dependency relations between them. 
For example, we have two edges $(l^6, a^5)$ and $(a^5, x^3)$ describing the dependency between the variables assigned by queries. The vertices $l^6$ and $x^3$ are the only ones with query annotation $1$ (the subscript), since they are the only two variables that are in assignments involving  queries. Notice that the graph contains cycles---in this example it contains two self-loops. These cycles capture the fact that the variables $a^5$ and $j^4$ are updated at every iteration of the loop using their previous values. Cycles are essential to capture mutual dependencies like the ones that are generated in loops. Adaptivity is a quantitative notion, so capturing this form of dependencies is not enough. 
This is why we also use weights. \highlight{The weight of a vertex is a function that given an initial state returns a natural number representing 
the number of times this vertex is visited during the program execution starting in this initial state.  }
% \highlight{To do: more explanation about weight, {$\lambda \trace.1$}, and replacing it}
For example, the vertex $l^{6}$ has weight {$\lambda \trace.1$} since for every initial state {$\trace$} the corresponding statement will be executed one time. The vertex $a^5$ on the other hand has weight {$\lambda \trace. \env(\trace) k$ since the corresponding assignment will be executed a number of times that correspond to the value of $k$ in the initial state $\trace$, and $\env$ is the operator reading value of $k$ from $\trace$.
}

% It is a function which takes an initial state, $\trace_0$ as input,
% then executes the program, and counts the evaluation times of the query request $\clabel{\assign{l}{\query(\chi[k]*a)} }^{6}$ during the execution.
% % returns $1$ for every starting state, since 
% Since this query at line $6$ is outside of any loop, we are expecting this function always return the count $1$ given any initial state.
% The query annotation of this vertex is $1$, which  indicates that 
% $\clabel{\assign{l}{\query(\chi[k] * a)}}^6$ is a query request.
% For another vertex, $a^{5}:{}^{w_{a^{5}}}_0$ in the while loop, we expect its weight function
% returns different counts if the input initial traces have different initial value for $k$.
% Because $\clabel{\assign{a}{x + a}}^{5}$ will be executed different times if the input $k$  is different.
% Its subscript $0$ representing this is a non-query assignment.



% Besides, we also have special edge which is a circle, to track any variable being updated with its previous value recursively. 
% For instance, the loop counter $j$ and the variable $a$ are updated based on previous values $k$ times in the first phase and we see two circle edges on $a^5$ and $j^4$.

%%%%%%%%%%%%%%%%%%%%%%%%%%%%%%%%%%% Details Below that might be useful when make passes %%%%%%%%%%%%%%%%%
% \detailed{The existence of circle edge \jl{(there isn't a name 'circle edge', the terminology is cycle)}
%  allows our graph to express situation when a variable relies on its previous value recursively inside a while loop, but not show how many times of this reliance, which is necessary to define adaptivity. For instance, if we modify our two round example a little bit to make the query $query(\chi[j]\dot \chi[k])$ at line $3$ relies on its previous result to $query(\chi[j]\dot\chi[k] + x)$, then intuitively its adaptivity becomes $k+1$. To this end, we add quantitative information to our graph: weight on every vertex.
% The weight of a vertex is a function that given a starting state returns a natural number representing 
% the number of times the vertex is visited when the program is executed starting from this state.}
% \jl{The existence of cycle
%  allows our graph to handle the while loop.
% When the variable in a while loop relies on its value in the previous iterations, the cycle expresses this reliance.
% But it cannot express the times of this reliance.
% For instance, if we modify the command $3$
% of the $\kw{twoRounds(k)}$ example
% into $\clabel{\assign{x}{\query(\chi[j] \cdot \chi[k] + x)}}^3$. 
% Then $x$ in every iteration relies on the result in the previous iteration
% and the intuitive adaptivity becomes $k+1$. But we don't know the number $k$ by only constructing the edge $x^3 \to x^3$.
% To this end, we add quantitative information to our graph: weight on every vertex.
% The weight of a vertex is a function that given a starting state returns a natural number representing 
% the number of times the vertex is visited during the program execution.
% }
% Each vertex in this graph has a superscript representing its weight, and a subscript $1$ or $0$ telling if the vertex corresponds to a query or not. We will call this subscript a query annotation. 
% For example, in Fig.~\ref{fig:overview-example}(b), the vertex $l^{6}:{}^{w_1}_1$, 
% has weight $w_1$, a constant function which returns $1$ for every starting state, since 
% this query at line $6$ is at most executed once regardless of the initial trace.
% The query annotation of this vertex is $1$, which  indicates that 
% $\clabel{\assign{l}{\query(\chi[k] * a)}}^6$ is a query request.
% Another vertex, $x^{3}:{}^{w_k}_1$, appears in the while loop. 
% It has as weight a function $w_k$ that for every initial state returns the value that $k$ has in this state, since this is also the number the while loop will be iterated. 
% The node $j^{4}:{}^{w_k}_0$ has as a subscript $0$ representing a non-query assignment.
% \jl{
% For example, in Fig.~\ref{fig:overview-example}(b), the vertex $l^{6}:{}^{w_{l^{6}}}_1$, 
% has weight ${w_{l^{6}}}$. It is a function which takes an initial state, $\trace_0$ as input,
% then executes the program, and counts the evaluation times of the query request $\clabel{\assign{l}{\query(\chi[k]*a)} }^{6}$ during the execution.
% % returns $1$ for every starting state, since 
% Since this query at line $6$ is outside of any loop, we are expecting this function always return the count $1$ given any initial state.
% The query annotation of this vertex is $1$, which  indicates that 
% $\clabel{\assign{l}{\query(\chi[k] * a)}}^6$ is a query request.
% For another vertex, $a^{5}:{}^{w_{a^{5}}}_0$ in the while loop, we expect its weight function
% returns different counts if the input initial traces have different initial value for $k$.
% Because $\clabel{\assign{a}{x + a}}^{5}$ will be executed different times if the input $k$  is different.
% Its subscript $0$ representing this is a non-query assignment.
%
%It has as weight a function $w_k$ that for every initial state returns the value that $k$ has in this state, since this is also the number the while loop will be iterated. 
% The node $j^{4}:{}^{w_k}_0$ has as a subscript $0$ representing a non-query assignment.
% }
%%%%%%%%%%%%%%%%%%%%%%%%%%%%%%%%%%% Details Below that might be useful when make passes %%%%%%%%%%%%%%%%%
% \detailed{Since the edges between two vertices represent the fact that one program variable may depend on the other,
% we can define the program adaptivity with respect to a initial trace by means of a walk traversing the graph, visiting each vertex no more than its weight with respect to the initial trace, and visiting as many query nodes as possible.
% Still, look again at our example, we can see that
% in the walk along the dotted arrows,  $l^{6} \to a^5 \to x^3 $, there are $2$ vertices with query annotation $1$ and that this number is maximal, i.e. we cannot find another walk having more than $2$ vertices with query annotation $1$, under the assumption that $k \geq 1$. So the adaptivity of the program in Fig.~\ref{fig:overview-example}(a)  is $2$,
% as expected.
% }
Third, we can finally define adaptivity using the semantics-based dependency graph. We actually define this notion with respect to an initial state $\tau$, since different states can give very different adaptivities.  
\highlight{rephrase the next sentence.}We consider 
% the longest walk  that visits each vertex $v$ of the semantics-based dependency graph no more than the value that the weight $w_v$ assign to $\tau$, and visits as many query nodes as possible. 
any  walk  that visits a vertex $v$ of the semantics-based dependency graph no more than the value that the vertex's weight $w_v$ associates to the initial state $\tau$, and that visits a maximal number of query vertices.
The number of query vertices visited is the adaptivity of the program with respect to $\tau$.
In Fig.~\ref{fig:overview-example}(b), assuming that $\tau(k) \geq 1$, we can see that the 
walk along the dashed arrows,  $l^{6} \to a^5 \to x^3 $ has two vertices with query annotation $1$, and we cannot find another walk having more than $2$ query vertices, although there is another walk, $l^{6} \to x^3 $, which has $2$ query vertices. So the adaptivity of the program in Fig.~\ref{fig:overview-example}(a) with respect to $\tau$ is $2$. If we consider an initial state $\tau$ such that $\tau(k)=0$ we have that the adaptivity with respect to $\tau$ is instead $1$. 
%%%%%%Gap: %%%%%%%%%%%%%%%%%%%%%%%%%%%%%%%%%%%%%%%%%%%%%%%%%%%%%%%%%%%%%%%%%%%%%%%%%%%%%%%%%%%%%%%%%%%%%%%%%%%%%%%%%%%%%%%%%%%%%%%%%%%%%%%%%%%%%%%
% \begin{figure}
%     \centering   
%     \includegraphics[width=1.0\textwidth]{architecture.png}
%     \vspace{-0.8cm}
%   \caption{High level architecture}
%     \label{fig:structure}
%     \vspace{-0.6cm}
% \end{figure}

\subsubsection{Static analysis}
%%%%%%%%%%%%%%%%%%%%%%%%%%%%%%% Previous Version Below that might be useful when make passes %%%%%%%%%%%%%%%%%
%  \detailed{The definition of adaptivity comes from the aforementioned execution-based dependency graph, 
%  our static analysis statically provides a sound upper bound on this adaptivity, via constructing another weighted graph, we call it estimated dependency graph. The upper bound is then found by searching a sound path with respect to our adaptivity in the generated graph. Different from the execution-based dependency which needs the trace from the execution, estimated one is built by our static analysis algorithm which takes the program itself as input. In brief, our algorithm is consist of a graph-generation algorithm, a weight computation algorithm and finally a path searching algorithm in the generated weighted graph.
%  }
%%%%%%%%%%%%%%%%%%%%%%%%%%%%%%% Previous Version Below that might be useful when make passes %%%%%%%%%%%%%%%%%
% \todo{In order to have a sound and accurate upper bound on the  adaptivity of a program $c$,
% we design a program analysis framework named {\THESYSTEM}.
% This framework composes two algorithms as shown in the double-stroke box and the dashed box in Fig.~\ref{fig:adaptfun}.
% The first algorithm in the double-stroke box combines the quantitative and dependency analysis techniques.
% It produces an estimated \emph{dependency graph} for a program.
% The second algorithm in the dashed box is a walk length estimation algorithm.
% It computes the upper bound on the program's \emph{adaptivity} over the estimated graph.}
% \jl{Since the definition of adaptivity comes from the aforementioned execution-based dependency graph, 
%  our static analysis statically provides a sound upper bound on this adaptivity via approximating this graph. The estimated graph is called \emph{estimated dependency graph}. 
%  The upper bound is then computed by searching the walk in this graph such that it can give a sound bound on the adaptivity.
%  Different from the execution-based dependency graph, the estimated one is produced by our static anlaysis algorithm, which only takes the program as input and does not rely on the execution history.
%  In brief, our algorithm is consist of a weighted graph-generation algorithm and a adaptivity computation algorithm over the graph.
%  }
 
 %%%%%%%%%%%%%%%%%%%%%%%%%%%%%%% Previous Version Above for Reference  %%%%%%%%%%%%%%%%%
To compute statically a sound and accurate upper bound on the \emph{adaptivity} of a program $c$,
we design a program analysis framework named {\THESYSTEM} (formally in Section \ref{sec:algorithm}). 
The structure of {\THESYSTEM} (Fig.~\ref{fig:adaptfun}) reflects in part the definition of adaptivity we discussed. Specifically, {\THESYSTEM} is composed by two algorithms (the ones in dashed boxes in the figure), one for building a dependency graph, called \emph{estimated dependency graph}, and the other to estimate the adaptivity from this graph.  
The first algorithm generates the \emph{estimated dependency graph} using several program analysis techniques. Specifically,
 {\THESYSTEM} extracts the vertices and the query annotations by looking at the assigned variables of the program, it estimates the edges by using control flow and data flow analysis, and it estimates the weights by using symbolic reachability-bound analysis---weights in this graph are symbolic expressions over input variables. 
% This combined analysis allow us to obtain more accurate upper bounds than what we would obtain by using any of these single analysis technique in isolation.
The second algorithm estimates the
% longest 
walk which respects the weights and which visits the maximal number of query vertices.
%  as possible. 
The two algorithms together gives us an  upper bound on the program's \emph{adaptivity}.

 \begin{figure}
  \centering    
\includegraphics[width=1.0\columnwidth]{adapfun.png}
  \vspace{-0.8cm}
  \caption{The overview of {\THESYSTEM}}
  \label{fig:adaptfun}
  \vspace{-0.5cm}
\end{figure}

 
%%%%%%%%%%%%%%%%%%%%%%%%%%%%%%%%%%% Details Below that might be useful when others are making passes %%%%%%%%%%%%%%%%%
%   \detailed{Fig.~\ref{fig:overview-example}(c) is the resulting estimated graph of our static analysis algorithm which consumes the program in Fig.~\ref{fig:overview-example}(a).The edges are generated by our graph generation algorithm which combines control flow analysis and data flow analysis, presented in Section~\ref{sec:alg_edgegen}). We can easily see the generated graph in Fig.~\ref{fig:overview-example}(c) is a safe approximation of its execution-based counterpart in Fig.~\ref{fig:overview-example}(b), in the way that we can find a corresponding edge in Fig.~\ref{fig:overview-example}(c) for all the edges in Fig.~\ref{fig:overview-example}(b). We call the weight of every vertex computed by our algorithm as estimated weight,  }
%   estimated by using a reachability-bound estimation algorithm (presented in Section~\ref{sec:alg_weightgen}). \detailed{Different from the execution-based weight $w_1$ or $w_k$ in Fig.~\ref{fig:overview-example}(b) which is a function whose output relies on the initial trace, our estimated weight} can be symbolic and provide a sound upper bound on its execution-based weight of the corresponding vertex in the execution-based dependency graph. For instance, 
%   the estimated weight $k$ of the vertex $x^{3}$ in Fig.~\ref{fig:overview-example}(c) is a sound upper bound on the execution-based weight $w_k$ of vertex $x^{3}$ in Fig.~\ref{fig:overview-example}(b), with the same starting trace $\trace$, $w_k(\trace) \leq\trace(k)$. $\trace(k)$ means getting the value of variable $k$ in the trace $\trace$. The soundness of this step is proved in Theorem~\ref{thm:addweight_soundness}.   
%
We show in Fig.~\ref{fig:overview-example}(c) the estimated dependency graph that our static analysis algorithm returns for the program $\kw{twoRounds(k)}$ in Fig.~\ref{fig:overview-example}(a).
Vertices and query annotations are the same as the ones in Fig.~\ref{fig:overview-example}(b), simply inferred by scanning the program.
 Every edge in Fig.~\ref{fig:overview-example}(b) is precisely inferred by our combined data flow and control flow analysis, this is why Fig.~\ref{fig:overview-example}(c) contains exactly the same edges.
The weight of every vertex is computed using a reachability-bound estimation algorithm which outputs a symbolic expression over the input variables, representing an upper bound on the number of times each assignment is executed.
% \wq{symbolic and provide a sound upper bound on its execution-based weight of the corresponding vertex in the execution-based dependency graph.
% $w_k(\trace) \leq \trace(k)$. $\trace(k)$ means getting the value of variable $k$ in the trace $\trace$. The soundness of this step is proved in Theorem~\ref{thm:addweight_soundness}.}
For example, consider the vertex $x^{3}$, its weight is $k$ and this provides an upper bound on the value returned by the weight function $\lambda \trace. \rho(\trace)k$ associated with vertex $x^{3}$ in Fig.~\ref{fig:overview-example}(b) for any initial state. 
% Indeed, 
% for any initial trace $\trace_0$, when $w_{x^{3}}(\trace_0)$ executes the program and counts the
% execution times of command $3$,
% we expect that this counts is at most the the loop iterations, i.e. $k$'s initial value from $\trace_0$.

The algorithm searching for the walk first finds a path $l^6:{}^1_1 \to a^5: {}^k_0 \to x^3: {}^k_1$, and then constructs a walk based on this path. Every vertex on this walk is visited once, and the number of vertices with query annotation $1$ in this walk is $2$, which is the upper bound we expect.
{It is worth noting here that $x^3$ and $a^5$ can only be visited once because there isn't an edge to go back to them, even though they both have the weight $k$}.  So the algorithm $\pathsearch$ computes the upper bound $2$ instead of $2k+1$. Note that $2$ is not always tight, for example when $k = 0$.
% In this sense, instead of simply computing the weighted length of this path ($2k+1$) as adaptivity, the algorithm $\pathsearch$ computes the upper bound $2$. Note that $2$ is not always tight, for example when $k = 0$.
% \todo{Can you double check if this is clear?}
% \mg{I think we should add a sentence to say that this bound is actually not always tight.}

%%%%%%%%%%%%%%%%%%%%%%%%%%%%%%%%%%%%%%%%%%% Language %%%%%%%%%%%%%%%%%%%%%%%%%%%%%%%%%%%%%%%%%%%
\section{Labeled Query  While  Language}
\label{sec:loop_language}
{The language of {\THESYSTEM} is a standard while language with labels to identify different components and with primitives for queries, and equipped with a  trace-based operational semantics which is the main technical tool we will use to define the program's adaptivity.

%\subsection{Syntax}
%\label{sec:syntax}
\vspace{-0.1cm}
{\small
\[
\begin{array}{llll}
\mbox{Arithmetic Expression} 
& \aexpr & ::= & 
n ~|~ {x} ~|~ \aexpr \oplus_a \aexpr 
~|~ \elog \aexpr  ~|~ \esign \aexpr ~|~ \max(a, a) ~|~ \min(a, a)
\\
\mbox{Boolean Expression} & \bexpr & ::= & 
%
\etrue ~|~ \efalse  ~|~ \neg \bexpr
 ~|~ \bexpr \oplus_b \bexpr
%
~|~ \aexpr \sim \aexpr 
\\
%
\mbox{Expression} & \expr & ::= & v ~|~ \aexpr \sep \bexpr ~|~ [\expr, \dots, \expr]
\\  
%
\mbox{Value} 
& v & ::= & { n \sep \etrue \sep \efalse ~|~ [] ~|~ [v, \dots, v]}  
\\
%
\mbox{Query Expression} 
& {\qexpr} & ::= 
& { \qval ~|~ \aexpr ~|~ \qexpr \oplus_a \qexpr ~|~ \chi[\aexpr]} 
\\
%
\mbox{Query Value} & \qval & ::= 
& {n ~|~ \chi[n] ~|~ \qval \oplus_a  \qval ~|~ n \oplus_a  \chi[n]
    ~|~ \chi[n] \oplus_a  n}\\
\mbox{Label} 
& l & \in & \mathbb{N} \cup \{\lin, \lex\} \\
\mbox{Labeled Command} 
& {c} & ::= &   [\assign {{x}}{ {\expr}}]^{l} ~|~  [\assign {{x} } {{\query(\qexpr)}}]^{l}
~|~ {\ewhile [ \bexpr ]^{l} \edo {c} }
 \\
 &&&
~|~ {c};{c}  
~|~ \eif([\bexpr]{}^l , {c}, {c}) 
~|~ [\eskip]^l 
\end{array}
\]
\vspace{-0.1cm}
}

%%%%%%%%%%%%%%%%%%%%%%%%%%%%%%%%%%% Details on Explaining the Syntax and Operational Semantics,
%%%%%%%%%%%%%%%%%%%%%%%%%%%%%%%%%%% that might be useful when others are making passes %%%%%%%%%%%%%%%%%
% \wq{As we have seen in the $\kw{twoRounds(k)}$ example in Fig.~\ref{fig:overview-example}(a), a program is expressed by labeled commands $c$. Our language supports the composition of labeled commands $c;c$, $[skip]^l$, the if condition $\eif([\bexpr]^l, c, c)$, the while command $\ewhile [\bexpr]^l \edo {c} $. The label $l$ records the location of this command, as a nature number $\mathbb{N}$ indicating the line number. Besides, it can also be $in$ or $ex$, used for annotating input variables and results and will not show up in labeled commands.} \wqside{Is it true that in and ex will not appear in the l in command?} 
%
% The boolean condition $b$ of the if and while commands is a standard boolean expression, which covers {\tt true} or {\tt false}, basic boolean connectives such as logical and logical or denoted by $\oplus_b$, logical negation $\neg$, and also comparison between arithmetic expressions $a \sim a$,  where $\sim$ stands for the basic operations such as $\leq,=,<,$ etc.  The arithmetic expression $a$ can be a constant $n$ denoting integer, a variable $x$ from some countable variable set $\mathcal{VAR}$, binary operation $\oplus_a$ such as addition, product, subtraction, etc, over arithmetic expressions, log and sign operation, minimal and maximum of two arithmetic expressions. An expression $e$ is then either a standard arithmetic expression or a boolean expression, or a list of expressions.
%
% \wq{As a reminder, the vertices of either the execution-based or program-based graphs in Fig.~\ref{fig:overview-example}(b) or Fig.~\ref{fig:overview-example}(c) are assigned variables and these assigned variables come from our two assignment commands: the standard assignment $[\assign{x}{\expr}]^l$, and our main novelty, the query request assignment $[\assign{x}{q(\qexpr)}]^l$.  The query is specified by a query expression $\qexpr$, which contains the necessary information for a query request. The query expression can be either the normal form $\alpha$, or just an arithmetic expression $a$ to express constant queries, or $\chi[\aexpr]$ representing the values at a certain index $\aexpr$ in a row $\chi$ of the database. Besides, we also allow combined access to the database in query expressions by means of $\qexpr \oplus \qexpr$.} For example, $\chi[3] + 5$ represents a query which asks the value from the column 3 of each database raw $\chi$, adds 5 to each of these values, and then computes the average of these values.
% In reality, if a data analyst wants to ask a simple linear query which returns the first element of the row, they can simply use the command $ \assign{x}{q(\chi[1])}$ in their data analysis program.
Expressions include
standard arithmetic (with value $n \in \mathbb{N}\cup \{ \infty \}$) and boolean expression, ($\aexpr$ and $\bexpr$) and extended query expressions $\qexpr$.
A query expression $\qexpr$ can be either a simple arithmetic expression $a$, an expression of the form $\chi[\aexpr]$ where $\chi$ represents a row of the database  and  $\aexpr$ represents an index used to identify a specific attribute of the row $\chi$, a combination of two query expressions, $\qexpr \oplus_{a} \qexpr$, or a normal form $\qval$.
For example, the query expression $\chi[3] + 5$  denotes the computation that
obtains
the value in the $3$rd column of $\chi$ in one row and then adds $5$ to it.

Commands are the typical ones from while languages with an additional command $\assign{x}{\query(\qexpr)}$ for query requests which can be used to interrogate
 the database and compute the linear query corresponding to $\qexpr$.
Each command is annotated with a label $l$, and we will use natural numbers as labels to record
the location of each command, so that we can uniquely identify them.
We also have a set of labels $\ldom$, a set $\mathcal{LV}$  of labeled variables (simply variables with a label), and a set $\cdom$ of all the programs.
We denote by $\mathbb{LV}(c)$ the set of labeled variables assigned in an assignment command in the program $c$.  
We denote by  $\qvar(c)$ the set of labeled variables that are assigned the result of a query in the program $c$.
 \highlight{We provide the table of notations in Table~\ref{tb:notation} for quick reference.}

% \mg{Please double check this notation.}


% In this the command, the query expression $\qexpr$ is sent to the database server as a request.
% Then the server will compute the average value of $\qexpr$ over each row of the hidden database $\chi$ and return us the result.
% For instance, when we execute the command $\assign{x}{\query(\chi[3] + 5)}$,
% the server will receive query request in form of $\chi[3] + 5$,
% then compute the average value of $\chi[3] + 5$ over each raw of $\chi$, and return the result to us. 
% The server is used as external API for computing the results over the hidden database $\chi$.

\subsection{ Trace-based Operational Semantics}

%  \wq{Our operational semantics uses the trace $\trace \in \mathcal{T}$ to track the history of the program execution, we use $\mathcal{T} $ for the set of traces. To be precise, the trace is a list of events and an event tracks the useful information about one step of the evaluation. When a program $c$ is evaluated in our operational semantics$\config{c, \trace} \to \config{c', \trace'} $, it starts with an initial trace $\trace$, evaluates to $c'$ , and along with the program evaluation, events are collected in the evaluation order and appended to $\trace$, and then we get the result trace $\trace'$. We can easily get the history of the evaluation of the program $c$ by looking at these newly added events in $\trace$ with respect to the initial trace $\trace$.  }
% \highlight{Comment: A lot of notation is provided in section 3.1, some standard and some less-so. A table to summarize would be helpful for the reader to orient themselves during the rest of the paper. On a minor note: the combine operator (::) is used only once (in figure 5) which the concat operator (++) is used everywhere else (e.g., definitions 2 and 4) with singleton traces.}
We use a trace-based operational semantics tracking the history of program execution. The operational semantics is parameterized by a database that can be accessed only through queries. Since this database is fixed, we omit it from the semantics but it is important to keep in mind that this database exists and it is what allows us to evaluate queries.
A \emph{trace}
$\trace$ is a list of \emph{events} generated when executing specific commands. We denote by $\mathcal{T}$ the set of traces and we will use list notation for traces,
 where $[]$ is the empty trace, the operator $\traceadd$ combines an event and a trace in a new trace, 
and the operator $\tracecat$ concatenates two traces. 
We have two kinds of events: \emph{assignment events} and \emph{testing events}. 
Each event consists of a quadruple,
and we use $\eventset^{\asn}$ and $\eventset^{\test}$ to denote the set of all assignment events and testing events, respectively.
% \begin{center}
% $ \begin{array}{lllll}
% \mbox{Event} 
% & \event & ::= & 
% {({x}, l, v, \bullet)} ~|~ { ({x}, l, v, \qval)}  ~|~{(\bexpr, l, v, \bullet)}  
% \\
% \end{array}$
% \end{center}
\begin{center}
  $ \begin{array}{lllll}
  \mbox{Event} 
  & \event & ::= & 
  ({x}, l, v, \bullet) ~|~ ({x}, l, v, \qval)  & \mbox{Assignment Event} \\
  &&& ~|~(\bexpr, l, v, \bullet)  & \mbox{Testing Event}
  \\
  \end{array}$
  \end{center}
An assignment event tracks the execution of an assignment  or a query request and consists of the assigned variable, the label of the command that generates it, the value assigned to the variable, and the normal form  $\qval$ of the query expression that has been requested, if this command is a query request, otherwise a default value $\bullet$.
A testing event tracks the execution of an if or while command and consists of the guard of the command, the label, the result of evaluating the guard, the last element $\bullet$. 
 We use the operator $\env (\trace) x$ to fetch the latest value assigned to  $x$ in the trace $\trace$, the operator
$\vcounter$ to count the occurrence of a labeled variable in the trace. \highlight{ The function $\kw{lastVal}(\tau, x)$ mentioned in Section~\ref{sec:adaptivity-informal} can be 
expressed as $\lambda \trace. \env (\trace) x$. For any initial trace $\trace$, $\kw{lastVal}(\tau, x)$ returns the latest value of $x$ in $\trace$.
}
We denote by $\tlabel(\trace) \subseteq \ldom$ the set of the labels occurring in $\trace$.
Finally, we use $\mathcal{T}_0(c) \subseteq \mathcal{T}$ to denote the set of \emph{initial traces}, the ones
which assign a value to the input variables. We use $\eventset$ to denote the set of all events.



\highlight{WQ: I propose to remove most rules in Fig5, only have query rule.}
The trace-based operational semantics is described in terms of a small step evaluation relation
$\config{c, \trace} \to \config{c', \trace}'$  describing how a configuration program-trace evaluates to another
configuration program-state. The rules for the operational semantics are described in Fig.~\ref{fig:os}.
The rules for assignment and query generate assignment events, while the rules for while and if generate testing events. 
The rules for the standard while language constructs correspond to the usual rules extended to deal with traces. 
We have relations $\config{\trace, \expr} \earrow v $  and $\config{\trace, \bexpr} \barrow v $  to evaluate expressions and boolean expressions, respectively. Their definitions are in the supplementary material.
%   \mg{Can you please confirm that this is true?} Yes
The only rule that is non-standard is the $\textbf{query}$ rule. When evaluating a query, the query expression $\qexpr$ is first simplified to its normal form $\alpha$ using an evaluation relation $\config{\trace, \qexpr} \qarrow \qval$. 
Then normal form $\qval$ characterizes the linear query that is run against the database. The query result $v$ is the expected value of the function $\lambda \chi.\qval$ applied to each row of the dataset. We summarize this process with the notation $\query(\qval) = v$ in the rule $\textbf{query}$. 
Once the answer of the query is computed, the rules record all the needed information in the trace.  We will use $\to^*$ for the reflexive and transitive closure of $\to$. 
%%%%%%%%%%%%%%%%%%%%%%%%%%%%%%%%%%% The Detailed Version in Explaining the Event %%%%%%%%%%%%%%%%%%%%%%%%%%%%%%%%%%%%%%%%%%%%%%%%%
% First of all, as the key component of the program evaluation history, an event $\event$ stores necessary information on the evaluation results of commands. Depending on the types of commands, there are two kinds of events: the assignment event which is generated in the standard assignment and query request assignment commands, and the testing event which is generated in an if or while command. Both assignment and testing events are quadruples, but store different contents. 
%\jl{
% The key component of the program evaluation history is the event $\event$,
% which stores necessary information on the evaluation results of commands.
%%%%%%%%%%%%%%%%%%%%%%%%%%%%%%%%%%% The Detailed Version in Explaining the Assignment evaluation %%%%%%%%%%%%%%%%%%%%%%%%%%%%%%%%%%%%%%%%%%%%%%%%%
% The assignment event 
% targets the assignment so it needs to maintain the mapping between labeled assigning variable and the result assigned, to this end, the first three elements of the quadruple of an assignment event are the variable, the label, and the result $v$ of the assigned expression. Look at the rule $\textbf{assn}$ and rule $\textbf{query}$ in Fig.~\ref{fig:os}, the result $v$ is different: in the standard assignment, $v$ is the evaluation result of the assigned expression $e$ by the standard expression evaluation $\config{\trace, \expr} \earrow v $; in the query request assignment, the query expression $\qexpr$ is evaluated to its normal form $\alpha$ by the query expression evaluation $\config{\trace, \qexpr} 
% \qarrow \qval$ and is sent to a hidden mechanism, $query(\alpha) = v$ means that the return result of this query represented by $\alpha$ from the mechanism is $v$. Another difference of the generated assignment events in these two rules lands in the last element of the quadruple, which stores the query information. In the rule $\textbf{query}$, the fourth element is the query normal form $\alpha$ which is sent to the mechanism, while in the standard assignment rule $\textbf{assn}$, we use $\bullet$ to show this event is not directly related to a query request.
% \jl{The assignment event is generated when evaluating an assignment command or query request. It stores the value assigned to each variable and tracks the query request.
% The first three elements are the variable, the label of this command, and the value assigned to this variable.
% The forth element is the normal form of a query expression, $\qval$ if this command is a query request, otherwise a default value $\bullet$.
% As in rule $\textbf{assn}$ and rule $\textbf{query}$ in Fig.~\ref{fig:os}.
% When evaluating a query request, the query expression $\qexpr$ is first simplified to its normal form $\alpha$ by the evaluation rule $\config{\trace, \qexpr} \qarrow \qval$. 
% Then $\qval$ is sent to the hidden database on unknown server, which computes the query result and send back to us.
% This computation process is simplified into $\query(\qval) = v$ in the rule $\textbf{query}$.
% }
%%%%%%%%%%%%%%%%%%%%%%%%%%%%%%%%%%% The Detailed Version in Explaining the Expression Evaluation %%%%%%%%%%%%%%%%%%%%%%%%%%%%%%%%%%%%%%%%%%%%%%%%%
% \detailed{
% The expression evaluation $\config{\trace, \expr} \earrow v $ relies on the evaluation of arithmetic expressions $\config{\trace,\aexpr} \aarrow v $ and boolean expressions $\config{\trace, \bexpr} \barrow v $, they are standard and we leave The full rules in the appendix. The evaluation rules of query expressions are presented below.}
The query expression evaluation relation  $\config{\trace, \qexpr} \qarrow \qval$ is defined by the following rules which reduce a query expression to its normal form.
{\small
\begin{mathpar}
\inferrule{ 
  \config{\trace, \aexpr} \aarrow n
}{
 \config{\trace,  \aexpr} 
 \qarrow n
}
\and
\inferrule{ 
  \config{\trace, \qexpr_1} \qarrow \qval_1
  \and
  \config{\trace, \qexpr_2} \qarrow \qval_2
}{
 \config{\trace,  \qexpr_1 \oplus_a \qexpr_2} 
 \qarrow \qval_1 \oplus_a \qval_2
}
\and
\inferrule{ 
  \config{\trace, \aexpr} \aarrow n
}{
 \config{\trace, \chi[\aexpr]} \qarrow \chi[n]
}
\and
\inferrule{ 
  \empty
}{
 \config{\trace,  \qval} 
 \qarrow \qval
}
 \end{mathpar}
 }
%%%%%%%%%%%%%%%%%%%%%%%%%%%%%%%%%%% The Detailed Version in Explaining the Testing Event %%%%%%%%%%%%%%%%%%%%%%%%%%%%%%%%%%%%%%%%%%%%%%%%%
% The testing event is generated when evaluating if and while commands. To record the control flow information, the first element of the event is the guard $b$ in both if and while rules $\textbf{if-t,if-f}$ and rule $\textbf{while-t, while-f}$. The third element then stores the evaluation results of this guard, either true or false. Since the guard can not be a query request, the last element is $\bullet$. 
%%%%%%%%%%%%%%%%%%%%%%%%%%%%%%%%%%% The Details for The If and While Evaluation %%%%%%%%%%%%%%%%%%%%%%%%%%%%%%%%%%%%%%%%%%%%%%%%%
% \detailed{The rules for if hand while both have two versions, when the guard evaluates to true and false, respectively. In these rules, the evaluation of the guard also generates testing event and our trace is updated as well. }
% The rules for if and while both have two versions, 
% when the boolean expressions in the guards are evaluated to true and false, respectively. 
% In these rules, the evaluation of the guard generates a testing event and the trace is updated as well by appending this event.
% The rule $\textbf{query}$ evaluates the argument of a query request to a normal form and obtain the answer $v_q$ of the query $\query(v)$ from the mechanism. 
% Then the trace expanded by appending the query expression $\query(v)$ with the current annotation $(l,w)$. 
% The rule for assignment is standard and the trace remains unchanged. The sequence rule keeps tracking the modification of the trace, and the evaluation rule for if conditional 

%By the operational semantics rules, we prove no rule will shrink the trace in Appendix.
{\footnotesize %
\begin{figure}
\begin{mathpar}
\boxed{
\mbox{Command $\times$ Trace}
\xrightarrow{}
\mbox{Command $\times$ Trace}
}
\and
\boxed{\config{{c, \trace}}
\xrightarrow{} 
\config{{c',  \trace'}}
}
\\
\inferrule
{
\config{\trace, \expr} \earrow v 
\and
\event = ({x}, l, v, \bullet)
}
{
\config{[\assign{{x}}{\expr}]^{l},  \trace } 
\xrightarrow{} 
\config{\clabel{\eskip}^l, \trace \traceadd \event}
}
~\textbf{assn}
%
\and
%
{
\inferrule
{
 \trace, \qexpr \qarrow \qval
 \and 
\query(\qval) = v
\and 
\event = ({x}, l, v, \qval)
}
{
\config{{[\assign{x}{\query(\qexpr)}]^l, \trace}}
\xrightarrow{} 
\config{{\clabel{\eskip}^l,  \trace \traceadd \event} }
}
~\textbf{query}
}
%
\and
%
\inferrule
{
 \trace, b \barrow \etrue
 \and 
 \event = (b, l, \etrue, \bullet)
}
{
\config{{\ewhile [b]^{l} \edo c, \trace}}
\xrightarrow{} 
\config{{
c; \ewhile [b]^{l} \edo c,
\trace \traceadd \event}}
}
~\textbf{while-t}
%
%
\quad
%
\inferrule
{
 \trace, b \barrow \efalse
 \and 
 \event = (b, l, \efalse, \bullet)
}
{
\config{{\ewhile [b]^{l}, \edo c, \trace}}
\xrightarrow{} 
\config{{
  \clabel{\eskip}^l,
\trace \traceadd \event}}
}
~\textbf{while-f}
%
%
\and
\inferrule
{
\config{{c_1, \trace}}
\xrightarrow{}
\config{{c_1',  \trace'}}
}
{
\config{{c_1; c_2, \trace}} 
\xrightarrow{} 
\config{{c_1'; c_2, \trace'}}
}
~\textbf{seq1}
\and
\inferrule
{
  \config{{c_2, \trace}}
  \xrightarrow{}
  \config{{c_2',  \trace'}}
}
{
\config{{\clabel{\eskip}^l; c_2, \trace}} \xrightarrow{} \config{{ c_2', \trace'}}
}
~\textbf{seq2}
\quad
\inferrule
{
 \trace, b \barrow \etrue \and \event = (b, l, \etrue, \bullet)
}
{
 \config{{
\eif([b]^{l}, c_1, c_2), 
\trace}}
\xrightarrow{} 
\config{{c_1, \trace \traceadd \event}}
}
~\textbf{if-t}
% \and
% %
% \inferrule
% {
%  \trace, b \barrow \efalse
%  \and 
%  \event = (b, l, \efalse, \bullet)
% }
% {
% \config{{\eif([b]^{l}, c_1, c_2), \trace}}
% \xrightarrow{} 
% \config{{c_2, \trace \traceadd \event}}
% }
~\textbf{if-f}
\end{mathpar}
  \vspace{-0.5cm}
    \caption{Trace-based Operational Semantics for Language.}
    \label{fig:os}
  \vspace{-0.1cm}
\end{figure}
}

    \begin{table}
      \caption{\highlight{Table of Notations}}
      \label{tb:notation}
      \begin{center}
        \begin{tabular}{| c |c |c| c| }
          \hline 
          $\mathcal{LV}$   & universe of labeled variables  & $\qvar(c)$ & labeled query variables in $c$\\ 
          $\cdom$  & set of all programs &  $\trace$ &   trace, a list of \emph{events}\\  
          $\mathcal{T}$  &  set of traces &  $\trace \traceadd \event$  & combine a trace and an event  \\
          $\ldom$ & set of labels  & $\trace \tracecat \trace'$ &  trace concatenation \\
          $\tlabel(\trace) \subseteq \ldom$  &set of the labels occurring in $\trace$  &  $ \env (\trace) x$  & fetch the latest value of  $x$ in a given $\trace$ \\
          $\mathcal{T}_0(c) \subseteq \mathcal{T}$ &  set of \emph{initial traces} & $\kw{lastVal} (\trace, x)$  & fetch the latest value of  $x$ in any $\trace$\\
          $\mathbb{LV}(c)$  & labeled variables in $c$ & $\vcounter(\trace, x^i)$ & occurrence of $x^i$ in the trace $\trace$\\
          \hline 
        \end{tabular}
        \end{center}
      \end{table}
}
%%%%%%%%%%%%%%%%%%%%%%%%%%%%%%%%%%%%%%%%%%% Adaptivity %%%%%%%%%%%%%%%%%%%%%%%%%%%%%%%%%%%%%%%%%%% 
\section{Definition of Adaptivity}
\label{sec:adaptivity}
In this section, we formally present the definition of adaptivity of a program, which is the length of the 'longest' walk with
the most queries involved in the semantics-based dependency graph of this program. We first present the construction of the semantics-based dependency graph before the introduction of the formal definition of adaptivity. 

\subsection{Semantics-based Dependency Graph}
\label{sec:design_choice}
%%%%%%%%%%%%%%%%%%%%%%%%%%%%%%%%%%% The Detailed Version in Introducing The Dependency Graph %%%%%%%%%%%%%%%%%%%%%%%%%%%%%%%%%%%%%%%%%%%%%%%%%
% we formally define the semantics-based dependency graph as follows. There are some notations used in the definition. The labeled variables of a program $c$  
% is a subset of the labeled  variables $\mathcal{LV}$, denoted by $\lvar(c) \in \mathcal{P}(\mathcal{VAR} \times \mathcal{L}) \subseteq \mathcal{LV}$.
% \wq{I think LV(c) means all the labeled assigned variables, because in Fig.3.b, j2 is not in the graph so j2 is not in LV(tworounds), please verify. If so, maybe LV(c) is not a good name, people may think it means all the label variables, instead of just assigned ones.}
% The set of query-associated variables (in query request assignments) for a program $c$ is denoted as $\qvar(c)$, where $\qvar: \cdom \to 
% \mathcal{P}(\mathcal{LV})$. The set of initial traces of a program $c$ is a subset of the  trace universe $\trace$, in which every initial trace contains the value for all the input variables of $c$. For instance, the initial trace $\trace_0$ contains the value of input variable $k$ in the $\kw{twoRounds(k)}$ example.
\jl{
The \emph{semantics-based dependency graph} is formally defined in Definition~\ref{def:trace_graph}. 
For a program $c$, there are some notations used in the definition.
The labeled variables of $c$,
$\lvar(c) \subseteq \mathcal{LV}$ contains all the variables in $c$'s assignment commands, with the command labels as superscripts. 
The set of query-associated variables (in query request assignments),
$\qvar(c) \subseteq \lvar(c)$ contains all labeled variables in $c$'s query requests. 
The set of initial traces of $c$,
$\mathcal{T}_0(c) \subseteq \mathcal{T}$
contains all possible initial trace of $c$.
Each initial trace,  $\trace_0 \in \mathcal{T}_0(c)$ contains the initial values of all input variables of $c$. 
For instance, the initial trace of $\kw{twoRounds(k)}$ example contains the initial value of the input variable $k$.
}
\begin{defn}[Semantics-based Dependency Graph]
\label{def:trace_graph}
Given a program ${c}$,
its \emph{semantics-based dependency graph} 
$\traceG({c}) = (\traceV({c}), \traceE({c}), \traceW({c}), \traceF({c}))$ is defined as follows,
{\small
\[
\begin{array}{lll}
  \text{Vertices} &
  \traceV({c}) & := \left\{ 
  x^l
  ~ \middle\vert ~ x^l \in \lvar(c)
  \right\}
  \\
  \text{Directed Edges} &
  \traceE({c}) & := 
  \left\{ 
  (x^i, y^j) 
  ~ \middle\vert ~
  x^i, y^j \in \lvar(c) \land \vardep(x^i, y^j, c) 
  \right\}
  \\
  \text{Weights} &
  \traceW({c}) & := 
  \{ 
  (x^l, w) 
  ~ \vert ~ 
  w : \mathcal{T}_0(c) \to \mathbb{N}
  \land
  x^l \in \lvar(c) 
  \\ & &
  \quad \land
  \forall \trace_0 \in \mathcal{T}_0(c), \trace' \in \mathcal{T} \st \config{{c}, \trace_0} \to^{*} 
  \config{\eskip, \trace_0\tracecat\trace'} 
  \land w(\trace_0) = \vcounter(\trace', l) \}  
  \\
  \text{Query Annotations} &
  \traceF({c}) & := 
\left\{(x^l, n)  
~ \middle\vert ~
 x^l \in \lvar(c) \land
n = 1 \Leftrightarrow x^l \in \qvar(c) \land n = 0 \Leftrightarrow  x^l \notin \qvar(c)
\right\}
\end{array},
\]
}
%%%%%%%%%%%%%%%%%%%%%%%%%%%%%%%%%%% The Detailed Version in Explaining the Trace Operators %%%%%%%%%%%%%%%%%%%%%%%%%%%%%%%%%%%%%%%%%%%%%%%%%
% There are some operators: the trace concatenation operator $\tracecat: \mathcal{T} \to \mathcal{T} \to \mathcal{T}$, combines two traces; the counting operator $\vcounter : \mathcal{T} \to \mathbb{N} \to \mathbb{N}$, 
% which counts the occurrence of of a labeled variable in the trace. The full definitions of these above operators can be found in the appendix.
% \\
\jl{where $\tracecat: \mathcal{T} \to \mathcal{T} \to \mathcal{T}$ is the trace concatenation operator, which combines two traces, 
and $\vcounter : \mathcal{T} \to \mathbb{N} \to \mathbb{N}$ is the counting operator, 
which counts the occurrence of of a labeled variable in the trace. All the definition details are in the appendix.}
A semantics-based dependency graph $\traceG({c})= (\traceV({c}), \traceE({c}), \traceW({c}), \traceF({c}))$ 
is \emph{well-formed} if and only if $ \{x^l \ |\ (x^l,w)\in \traceW({c})\} = \traceV({c}) $.
\end{defn}


%%%%%%%%%%%%%%%%%%%%%%%%%%%%%%%%%%% The Explanation of Dependency Graph %%%%%%%%%%%%%%%%%%%%%%%%%%%%%%%%%%%%%%%%%%%%%%%%%
% \wq{ The occurrence time is computed by the counter operator $w(\trace) = \vcounter(\trace', l)$. As an instance, in the semantics-based dependency graph of $\kw{twoRounds}$ in Figure~\ref{fig:overview-example}(b), the weight of vertices in the while loop $w_k$ is a function which returns the value $n$ of variable $k$ in the starting trace $\tau$, and the commands in the loop are indeed executed $n$ times when starting with $\trace$.}
% Every vertex also has a query annotation, mapping each $x^l \in \traceV(c)$ to $0$ or $1$, 
% indicating whether the vertex comes from a query request assignment (1) or not (0) by checking if the labeled variable $x^l$ of the vertex is in $\qvar(c)$.
% One interesting point is that instead of building a graph whose vertices are just variables coming from query request assignments, we choose the combination of all the labeled assigned variables of the program and the query annotations. The reason is that the results of previous queries can be stored or used in variables
% which aren't associated to the query request,
% it is necessary to track the dependency between all the assigned variables of the program. 
% The main novelty is the combination of the quantitative and dependency information on the semantics-based dependency graph, which means this graph can tell both the dependency between queries and the times they depend on each other.
% The vertices $\traceV({c})$ of a program $c$ are the labeled assigned variables, which are statically collected. The weight function on every vertex $w : \mathcal{T} \to \mathbb{N}$
% mapping from a starting trace $\trace \in \mathcal{T}_0(c)$ to a natural number, tracks the occurrence times of this vertex in the newly generated trace $\trace'$ when the program $c$ is evaluated from the starting trace $\config{{c}, \trace} \to^{*} \config{\eskip, \trace\tracecat\trace'} $.
% One interesting point is that instead of building a graph whose vertices are just variables coming from query request assignments, we choose the combination of all the labeled assigned variables of the program and the query annotations. The reason is that the results of previous queries can be stored or used in variables
% % which aren't associated to the query request,
% it is necessary to track the dependency between all the assigned variables of the program. 
\jl{There are four components in this graph.
\begin{enumerate}
    \item The vertices $\traceV({c})$ of a program $c$ are all its labeled variables, $\lvar(c)$ which are statically collected.
    \item $\traceF(c)$ contains the \emph{query annotation} for 
    every vertex $x^l \in \traceV(c)$. It indicates whether $x^l$ comes from a query request (1) or not (0) by checking if the labeled variable $x^l$ of the vertex is in $\qvar(c)$.
    \item Edges in $\traceE(c)$ are built from the $\vardep(x^i, y^j, c)$ relation between two labeled variables.
    This is the key definition in order to formalize the intuitive \emph{may-dependency} relation between queries and the \emph{adaptivity}. We present this formalization detail in Section~\ref{sec:dep} below.
    \item 
    The weight function in $\traceW(c)$ for each vertex, $w : \mathcal{T} \to \mathbb{N}$
maps from a starting trace $\trace_0 \in \mathcal{T}_0(c)$ to a natural number.
For each vertex $x^l$, it tracks its visiting times (i.e., the evaluation times of the command with the label $l$) when the program $c$ is evaluated from the initial trace $\trace_0$ into $\eskip$, $\config{{c}, \trace_0} \to^{*} \config{\eskip, \trace_0\tracecat\trace'} $.
The visiting times is computed by the counter operator $\vcounter(\trace', l)$
by counting the occurrence of the label $l$ in $\trace'$.
As an instance, in the semantics-based dependency graph of $\kw{twoRounds}$ in Figure~\ref{fig:overview-example}(b), the weight, $w_k$ of the vertex $x^3$ is a function of type $\mathcal{T}_0(\kw{twoRound(k)}) \to \mathbb{N}$.
Given input $\trace_0$, we execute the program under $\trace_0$ as $\config{\kw{twoRound(k)}, \trace_0} \to^{*} \config{\eskip, \trace_0\tracecat\trace'} $. Then $w_k(\trace_0)$ outputs the occurrence time of the label $3$ in $\trace'$.
\end{enumerate}
The main novelty of  the semantics-based dependency graph is the combination of the quantitative and dependency information. 
It can tell both the dependency between queries via the directed edge, and the times they depend on each other via the weight.
}

\subsection{May-Dependency}
\label{sec:dep}
%%%%%%%%%%%%%%%%%%%%%%%%%%%%%%%%%%% The Detail Explanation of Variable May-Dependency and Motivation on How to define it %%%%%%%%%%%%%%%%%%%%%%%%%%%%%%%%%%%%%%%%%%%%%%%%%
% \wq{we think the query $\query(\chi[2])$ (assigned variable $y^3$) may depend on the query $\query(\chi[1]$ (assigned variable $x^1$). 
  %  but vulnerable to queries request protected by differential privacy mechanisms. In our loop language, a query $q(e)$ represents a query request to the database through a mechanism, which add random noise to protect the return results. In this setting, the results of one query will be randomized due to the noise attached by the mechanism which fails the first candidate because witnessing the results of one query can no longer tells whether the change of the results comes from another query or the change of noise of the differential privacy mechanism. For example, suppose we have a program $p$ which requests two simple queries $q_1()$ and $q_2()$ with no arguments as follows.
%   \[
%   c_1 =\assign{x}{\query(\chi[2])} ;\assign{y}{\query(\chi[3] + x)}.
%   \]
%  $ c = \assign{x}{\query(\chi[1])} ; \assign{y}{\query(\chi[2])}$,
%  and 
% Specifically, in the {\tt Query While} language, the query request is composed by two components: a symbol $\query$ representing a linear query type and 
% % an argument
% the query expression $\qexpr$ as an argument, 
% which represents the function specifying what the query asks. 
% From our perspective, $\query(\chi[1])$ is different from $\query(\chi[2]))$. Informally,  
%
% in this example: $c_1 = \assign{x}{\query(0)}; \assign{z}{\query(\chi[x])}$.
% This candidate definition works well 
% Nevertheless, the first definition fails to catch control dependency because it just monitors the changes to a query, but misses the appearance of the query when the answers of its previous queries change. For instance, it fails to handle $}
%       c_2 = \assign{x}{\query(\chi[1])} ; \eif( x > 2 , \assign{y}{\query(\chi[2])}, \eskip )
%   $, but the second definition can. However, it only considers the control dependency and misses the data dependency. This reminds us to define a \emph{may-dependency} relation between labeled variables by combining the two definitions to capture the two situations.
%  $ p = \assign{x}{\query(\chi[1])} ; \assign{y}{\query(\chi[2])}$. 
% This candidate definition works well with respect to data dependency. However, if fails to handle control dependency since it just monitors the changes to the answer of a query when the answer of previous queries returned change. The key point is that this query may also not be asked because of an analyst decision which depend on the answers of previous queries. An example of this situation is shown in program $p_1$ as follows.
% There are two possible situations that a query will be "affected" by previous queries' results,  
% either when the query expression directly uses the results of previous queries (data dependency), or when the control flow of the program with respect to a query (whether to ask this query or not) depends on the results of previous queries (control flow dependency). To this end, our assigned variable dependency definition has the following two cases.   
% {
% \begin{enumerate}
%     \item One variable may depend on a previous variable if and only if a change of the value assigned to the previous variable may also change the value assigned to the variable.
%     \item One variable may depend on a previous variable if and only if a change of the value assigned to the previous variable may also change the appearance of the assignment command to this variable 
%     % in\wq{during?} 
%     during execution.
% \end{enumerate}
% }
%
%  The first case captures the data dependency. 
% For instance, in a simple program $c_1 =[\assign{x}{\query(\chi[2])}]^1 ;[\assign{y}{\query(\chi[3] + x)}]^2$, we think $\query(\chi[3] + x)$ (variable $y^2$) may depend on the query $\query(\chi[2]))$ (variable $x^1$), because the equipped function of the former $\chi[3] + x$ may depend on the data stored in x assigned with the result of $\query(\chi[2]))$. From our perspective, $\query(\chi[1])$ is different from $\query(\chi[2]))$. The second case captures the control dependency, for instance, in the program $
%       c_2 = [\assign{x}{\query(\chi[1])}]^1 ; \eif( [x > 2]^2 , [\assign{y}{\query(\chi[2])}]^3, [\eskip]^4 )
%
This section formalizes the \emph{may-dependency} relation between queries and introduces the
\emph{variable may-dependency} definition.

\jl{There are two possible situations that a query will be ``influenced'' by previous queries' results,
where either the query request is changed when the results of previous queries are changed (data dependency),
or the  query request is disappeared when the results of previous queries are changed (control dependency). In this sense, our formal dependency definition considers both the two cases as follows,
\begin{enumerate}
    \item One query may depend on a previous query if and only if a change of the value returned
    to the previous query request may also change this query request.
    \item One query may depend on a previous query if and only if a change of the value returned
    to the previous query request may also change the appearance of this query quest.
\end{enumerate}
The first case captures the data dependency. 
For instance, in a simple program $c_1 =[\assign{x}{\query(\chi[2])}]^1 ;[\assign{y}{\query(\chi[3] + x)}]^2$, we think $\query(\chi[3] + x)$ (variable $y^2$) may depend on the query $\query(\chi[2]))$ (variable $x^1$), because the equipped function of the former $\chi[3] + x$ may depend on the data stored in x assigned with the result of $\query(\chi[2]))$. From our perspective, $\query(\chi[1])$ is different from $\query(\chi[2]))$.
\\
The second case captures the control dependency.
For instance, in the program
$c_2 = [\assign{x}{\query(\chi[1])}]^1 ; \eif( [x > 2]^2 , [\assign{y}{\query(\chi[2])}]^3, [\eskip]^4 )$, 
we think the query $\query(\chi[2])$ ( or the labeled variable $y^3$) may depend on the query $\query(\chi[1])$ (via the labeled variable $x^1$). }

\jl{ 
Since both of the two ``influences'' are passing through labeled variables, we choose to formally define the \emph{may-dependency}
relation over all labeled variables, and then recover the query requests from query-associated variables, $\qvar(c)$.
It relies on the formal observation of the ``influence'' via events in Definition~\ref{def:diff} and the \emph{may-dependency} between events in Definition~\ref{def:event_dep}.
}
  \begin{defn}[Events Differ in Value ($\diff$)]
    \label{def:diff}
    Two events $\event_1, \event_2 \in \eventset$ differ in their value, or query value,
    denoted as $\diff(\event_1, \event_2)$, if and only if:
    {\small
    \begin{subequations}
    \begin{align}
    & \pi_1(\event_1) = \pi_1(\event_2) 
      \land  
      \pi_2(\event_1) = \pi_2(\event_2) \\
    & \land  
      \big(
       (\pi_3(\event_1) \neq \pi_3(\event_2)
      \land 
      \pi_{4}(\event_1) = \pi_{4}(\event_2) = \bullet )
      \lor 
      (\pi_4(\event_1) \neq \bullet
      \land 
      \pi_4(\event_2) \neq \bullet
      \land 
      \pi_{4}(\event_1) \neq_q \pi_{4}(\event_2)) 
      \big)
    \end{align}
    \label{eq:diff}
    \end{subequations}
    }
    where $\qexpr_1 =_{q} \qexpr_2$ denotes the semantics equivalence between query values,
    and $\pi_i$ projects the $i$-th element from the quadruple of an event.
    \end{defn}
%%%%%%%%%%%%%%%%%%%%%%%%%%%%%%%%%%% The "Detail" (I think redundant) Explanation of Variable May-Dependency and Motivation on How to define it %%%%%%%%%%%%%%%%%%%%%%%%%%%%%%%%%%%%%%%%%%%%%%%%%
% We use $\qexpr_1 =_{q} \qexpr_2$ and $\qexpr_1 \neq_{q} \qexpr_2$ to notate query expression equivalence and in-equivalence, distinct from standard equality. 
% First of all, $\diff(\event_1, \event_2)$ only compares the 'value' assigned to the same labeled variable so the two events share the same labeled variable (the first two elements of the event quadruple). If the labeled variable are not the same, it is trivially false. Next, the 'value' has different meaning according to the type of the assignment. If the two assignment events both are generated from a standard assignment command (by checking whether the fourth element of the event is $\bullet$), we think they are different up to their value if the value assigned to the variable (the third element of the event) varies. On the other hand, if they both come from query request assignment commands, we directly compare the query function (the fourth element of the event). We do not consider the case when one event is generated from the standard assignment and the other comes from the query request assignment, which will not hold. The motivation of defining  $\diff(\event_1, \event_2)$ is that it allows us to check whether the value assigned to a specific variable is changed or not in two runs, which is very important in our dependency case 1. 
\jl{
$\pi_1(\event_1) = \pi_1(\event_2) 
  \land  
  \pi_2(\event_1) = \pi_2(\event_2)$ at Eq.\ref{eq:diff}(a)
requires that $\event_1$ and $\event_2$ have the same variable name and label. 
This guarantees that $\event_1$ and $\event_2$ are generated from the same labeled command.
In Eq.\ref{eq:diff}(b),
two kinds of comparisons between the third and fourth element are for the non-query assignment and query request separately.
For events generated from the non-query assignments (via checking
$\pi_{4}(\event_1) =_q \pi_{4}(\event_2) = \bullet$), we only compare their assigned values through $\pi_3(\event_1) \neq \pi_3(\event_2)$.
But for these from query requests (via checking
$\pi_{4}(\event_1) \neq \bullet \land \pi_{4}(\event_2) \neq \bullet$),
we are comparing their query expressions by $\pi_{4}(\event_1) \neq_q \pi_{4}(\event_2)$ rather than the assigned value computed from the unknown database server.
This matches the intuitive data dependency between queries, where one query is influenced by others as long as the query request is changed.
}

\jl{Below is the \emph{event may-dependency} between events based on formally observing their differences via $\diff$.}
\begin{defn}[Event May-Dependency]
\label{def:event_dep}
An event $\event_2$ is in the \emph{event may-dependency} relation with an assignment event $\event_1 \in \eventset^{\asn}$ in a program ${c}$  with a hidden database $D$ and a witness trace $\trace \in \mathcal{T}$,
$\eventdep(\event_1, \event_2, [\event_1 ] \tracecat \trace \tracecat [\event_2], c, D)$ if and only if
\begin{subequations}
\begin{align}
&  
\exists \trace_0, \trace_1, \trace' \in \mathcal{T},\event_1' \in \eventset^{\asn}, {c}_1, {c}_2  \in \cdom  \st \diff(\event_1, \event_1') \land \\
& 
\quad (\exists  \event_2' \in \eventset \st 
\left(
\begin{array}{ll}   
  & \config{{c}, \trace_0} \rightarrow^{*} 
  \config{{c}_1, \trace_1 \tracecat [\event_1]}  \rightarrow^{*} 
  \config{{c}_2,  \trace_1 \tracecat [\event_1] \tracecat \trace \tracecat [\event_2] } 
   \\ 
   \bigwedge &
   \config{{c}_1, \trace_1 \tracecat [\event_1']}  \rightarrow^{*}
    \config{{c}_2,  \trace_1 \tracecat[ \event_1'] \tracecat \trace' \tracecat [\event_2'] } 
  \\
  \bigwedge & 
  \diff(\event_2,\event_2' ) \land 
  \vcounter(\trace, \pi_2(\event_2))
  = 
  \vcounter(\trace', \pi_2(\event_2'))\\
  \end{array}
  \right)\\ 
  & 
  \quad
  \lor 
  \left(
  \begin{array}{l} 
  \exists \trace_3, \trace_3'  \in \mathcal{T}, \event_b \in \eventset^{\test} \st 
  \\
   \quad \config{{c}, \trace_0} \rightarrow^{*} \config{{c}_1, \trace_1 \tracecat [\event_1]}  \rightarrow^{*}
   \config{c_2,  \trace_1 \tracecat [\event_1] \tracecat
   \trace \tracecat [\event_b] \tracecat  \trace_3} 
\\ \quad \land
\config{{c}_1, \trace_1 \tracecat [\event_1']}  \rightarrow^{*} 
\config{c_2,  \trace_1 \tracecat [\event_1'] \tracecat \trace' \tracecat [(\neg \event_b)] \tracecat \trace_3'} 
\\
\quad \land \tlabel({\trace_3}) \cap \tlabel({\trace_3'})
= \emptyset
\land \vcounter(\trace', \pi_2(\event_b)) = \vcounter(\trace, \pi_2(\event_b)) 
    \land \event_2 \in \trace_3
    \land \event_2 \not\in \trace_3'
  \end{array}
  \right)
  ),
\end{align}
\label{eq:eventdep}
\end{subequations}
where $\tlabel(\trace) \subseteq \ldom$ is the set of the labels in all the events from trace $\trace$ and $\event_2 \in \trace_3$ or $\event_2 \notin \trace_3$ denotes that $\event_2$ belongs to $\trace_3$ or not.
\end{defn}
The first line in Eq.~\ref{eq:eventdep}(a) requires that $\event_1$ comes from an assignment command and then modifies its assigned value via $\diff(\event_1, \event_1')$.

\jl{Then, the following two parts in Eq~\ref{eq:eventdep}(b) and (c) capture the intuitive value dependency and control dependency respectively. 
Both parts execute the program two times w.r.t. the different values in $\event_1$ (as line:1 in Eq~\ref{eq:eventdep}(b) and line:2 in Eq~\ref{eq:eventdep}(c))
and $\event_1'$ (as line:2 in Eq~\ref{eq:eventdep}(b) and line:3 in Eq~\ref{eq:eventdep}(c)), 
but observe the difference in the newly generated traces in different ways (via $3$rd line in Eq~\ref{eq:eventdep}(b) and $4$th line in Eq~\ref{eq:eventdep}(c)). This idea is similar to the dependency definition from \cite{Cousot19a}.
}

\jl{In Eq~\ref{eq:eventdep}(b) line:2, if the newly generated trace, $\trace' ++ [\event_2']$ still contains $\event_2$ as $\event_2'$, we check the difference on their value in line:3.
If they only differ in their assigned values, i.e., $\diff(\event_2, \event_2')$ and
they are in the same loop iteration (via $\vcounter(\trace, \pi_2(\event_2)) = \vcounter(\trace', \pi_2(\event_2'))$),
then we say there is a value \emph{may-dependency} relation between $\event_1$ and $\event_2$.}

\jl{The Eq~\ref{eq:eventdep}(c) captures the control dependency through observing the disappearance $\event_2$ from newly generated traces, $\trace' \tracecat [(\neg \event_b)] \tracecat \trace_3'$ in the second execution (line:3).
$\event_2 \in \trace_3 \land \event_2 \not\in \trace_3'$ in Eq~\ref{eq:eventdep}(c) line:4 specifies this disappearance.
$\vcounter(\trace', \pi_2(\event_b)) = \vcounter(\trace, \pi_2(\event_b))$ is used to make sure the two executions are
in the same loop iteration as well.
Different from Eq~\ref{eq:eventdep}(b) line:3,
we use a testing event, $\event_b$ here because
$\vcounter(\trace, \pi_2(\event_2)) = \vcounter(\trace', \pi_2(\event_2'))$ cannot guarantee the disappearance if there are nested loops.
This is correct because the control dependency can only be passed through the guard of if or while command,
and this guard must be evaluated into two different values ($\event_b$ and $\neg \event_b$) in the two executions.
}

\jl{
Then Considering all the assignment events newly generated during a program’s executions, 
as long as there is one pair of events satisfying the \emph{event may-dependency}, 
we say that the two labeled variables in the two assignment events satisfy the \emph{variable may-dependency} relation below.}

\begin{defn}[Variable May-Dependency]
  \label{def:var_dep}
  A variable ${x}_2^{l_2} \in \lvar(c)$ is in the \emph{variable may-dependency} relation with another
  variable ${x}_1^{l_1} \in \lvar(c)$ in a program ${c}$, denoted as 
  %
  $\vardep({x}_1^{l_1}, {x}_2^{l_2}, {c})$, if and only if.
\[
  \begin{array}{l}
\exists \event_1, \event_2 \in \eventset^{\asn}, \trace \in \mathcal{T} , D \in \dbdom \st
\pi_{1}{(\event_1)}^{\pi_{2}{(\event_1)}} = {x}_1^{l_1}
\land
\pi_{1}{(\event_2)}^{\pi_{2}{(\event_2)}} = {x}_2^{l_2}% \\ \quad 
\land 
\eventdep(\event_1, \event_2, \trace, c, D) 
  \end{array}
\]  %
  \end{defn}
\jl{From the definition, a labeled assigned variables $x_2^{l_2}$ may depend on another labeled assigned variable $x_1^{l_1}$ in a program $c$ under the hidden database $D$, 
as long as there exist two assignment events $\event_1$ (for $x_1^{l_1}$) and $\event_2$ for $x_2^{l_2}$
satisfy the \emph{event may-dependency} relation under a witness trace $\trace$.  
}
%%%%%%%%%%%%%%%%%%%%%%%%%%%%%%%%%%%%%%%Trace-Based Adaptivity%%%%%%%%%%%%%%%%%%%%%%%%%%%%%%%
\subsection{Trace-based Adaptivity}
Given 
a program $c$'s semantics-based dependency graph 
$\traceG({c})$,
we define adaptivity 
with respect to an initial trace $\trace_0 \in \mathcal{T}_0(c)$ by the finite walk in the graph, which has the most query requests along the walk.
We show the definition of a finite walk as follows.

\begin{defn}[Finite Walk (k)]
\label{def:finitewalk}
Given the semantics-based dependency graph $\traceG({c}) = (\traceV, \traceE, \traceW, \traceF)$ of a program $c$, a \emph{finite walk} $k$ in $\traceG({c})$ is a function $k$.
Given an input initial trace $\trace_0 \in \mathcal{T}_0(c)$, $k(\trace_0)$ is a sequence of edges $(e_1 \ldots e_{n - 1})$ 
for which there is a sequence of vertices $(v_1, \ldots, v_{n})$ such that:
\begin{itemize}
\item $e_i = (v_{i},v_{i + 1}) \in \traceE$ for every $1 \leq i < n$.
\item every $v_i \in \traceV$ and $(v_i, w_i) \in \traceW$, $v_i$ appears in $(v_1, \ldots, v_{n})$ at most $w(\trace_0)$ times.  
\end{itemize}
The length of $k(\trace_0)$ is the number of vertices in its vertices sequence, i.e., $\len(k)(\trace_0) = n$.
\end{defn} 
\jl{$\walks(\traceG(c))$ is 
the set of all the finite walks $k$ in $\traceG(c)$,
and $k_{v_1 \to v_2} \in \walks(\traceG(c))$ denotes the walk from vertex $v_1$ to $v_2$.}

\jl{Because the adaptivity are intuitively describing the dependency between queries,
so we calculate a special ``length'', the \emph{query length} of a walk by counting only the vertices
corresponding to queries. This is formally defined below.}
\begin{defn}[Query Length of the Finite Walk($\qlen$)]
\label{def:qlen}
Given 
the semantics-based dependency graph $\traceG({c})$ of a program $c$,
 and a \emph{finite walk} 
 $k \in \walks(\traceG(c))$. 
The query length of $k$, $\qlen(k)$ is a function $\mathcal{T}_0(c) \to \mathbb{N}$, such that given an input initial trace $\trace_0$, $\qlen(k)(\trace_0)$ is
the number of vertices which correspond to query variables in the vertex sequence, $(v_1, \ldots, v_{n})$ as follows, 
\[
  \qlen(k)(\trace_0) = |\big( v \mid v \in (v_1, \ldots, v_{n}) \land \qflag(v) = 1 \big)|.
\]
\end{defn}
Then the definition of adaptivity is presented in Definition~\ref{def:trace_adapt} below.
\begin{defn}
    [Adaptivity of a Program]
    \label{def:trace_adapt}
    Given a program ${c}$, 
    its adaptivity $A(c)$ is function 
    $A(c) : \mathcal{T} \to \mathbb{N}$ such that for an
    initial trace $\trace_0 \in \mathcal{T}_0(c)$, 
   $$
    A(c)(\trace_0) = \max \big 
    \{ \qlen(k)(\trace_0) \mid k \in \walks(\traceG(c)) \big \} $$
    \end{defn}

\subsection{The Walk Through Example}
{\small
\begin{figure}
\centering
\begin{subfigure}{.2\textwidth}
\begin{centering}
$
    \begin{array}{l}
    \kw{towRounds(k)} \triangleq \\
           \clabel{ \assign{a}{0}}^{0} ;
            \clabel{\assign{j}{k} }^{1} ; \\
            \ewhile ~ \clabel{j > 0}^{2} ~ \edo ~ \\
            \Big(
             \clabel{\assign{x}{\query(\chi[j] \cdot \chi[k])} }^{3}  ; \\
             \clabel{\assign{j}{j-1}}^{4} ;\\
            \clabel{\assign{a}{x + a}}^{5}       \Big);\\
            \clabel{\assign{l}{\query(\chi[k]*a)} }^{6}\\
        \end{array}
$
\caption{}
\end{centering}
\end{subfigure}
\begin{subfigure}{.4\textwidth}
%}
\qquad
\begin{centering}
\begin{tikzpicture}[scale=\textwidth/16cm,samples=250]
\draw[] (0, 10) circle (0pt) node
{{ $a^0: {}^{\lambda \trace_0. 1}_{0}$}};
\draw[] (0, 7) circle (0pt) node
{\textbf{$x^3: {}^{\lambda \trace_0. \env(\trace_0) k}_{1}$}};
\draw[] (0, 4) circle (0pt) node {{ $a^5: {}^{\lambda \trace_0. \env(\trace_0) k}_{0}$}};
\draw[] (0, 1) circle (0pt) node
{{ $l^6: {}^{\lambda \trace_0. 1}_{1}$}};
% Counter Variables
\draw[] (8, 9) circle (0pt) node {\textbf{$j^1: {}^{\lambda \trace_0. 1}_{0}$}};
\draw[] (8, 6) circle (0pt) node {{ $j^4: {}^{\lambda \trace_0. \env(\trace_0) k}_{0}$}};
%
% Value Dependency Edges:
\draw[ ultra thick, -latex, densely dotted,] (0, 1.5)  -- (0, 3.5) ;
\draw[ ultra thick, -latex, densely dotted,] (0, 4.5)  -- (0, 6.5) ;
\draw[ thick, -latex] (0, 4.5)  to  [out=-230,in=230]  (0, 9.5) ;
\draw[ thick, -Straight Barb] (1.5, 3.8) arc (120:-200:1);
\draw[ thick, -Straight Barb] (9, 6.5) arc (150:-150:1);
\draw[ thick, -latex] (8, 6.5)  -- (8, 8.5) ;
\draw[ thick, -latex] (0, 1.5)  to  [out=-230,in=230]  (0, 9.5) ;
% Control Dependency
\draw[ thick,-latex] (2, 7)  -- (6, 9) ;
\draw[ thick,-latex] (2, 4.5)  -- (6, 9) ;
\draw[ thick,-latex] (2, 7)  -- (6, 6) ;
\draw[ thick,-latex] (2, 4.5)  -- (6, 6) ;
\end{tikzpicture}
\caption{}
\end{centering}
\end{subfigure}
   \begin{subfigure}{.36\textwidth}
   \begin{centering}
   \begin{tikzpicture}[scale=\textwidth/18cm,samples=200]
\draw[] (0, 10) circle (0pt) node
{{ $a^0: {}^1_{0}$}};
\draw[] (0, 7) circle (0pt) node
{\textbf{$x^3: {}^{k}_{1}$}};
\draw[] (0, 4) circle (0pt) node
{{ $a^5: {}^{k}_{0}$}};
\draw[] (0, 1) circle (0pt) node
{{ $l^6: {}^{1}_{1}$}};
% Counter Variables
\draw[] (5, 9) circle (0pt) node {\textbf{$j^1: {}^{1}_{0}$}};
\draw[] (5, 6) circle (0pt) node {{ $j^4: {}^{k}_{0}$}};
%
% Value Dependency Edges:
\draw[ ultra thick, -latex, densely dotted,] (0, 1.5)  -- (0, 3.5) ;
\draw[ ultra thick, -latex, densely dotted,] (0, 4.5)  -- 
% node [left] {\highlight{$\trace_0 \to \env(\trace_0) k $}}
(0, 6.5) ;
\draw[ thick, -latex] (0, 4.5)  to  [out=-230,in=230]  
% node [left] {\highlight{$\trace_0 \to \env(\trace_0) k $}}
(0, 9.5) ;
\draw[ thick, -Straight Barb] (1.5, 3.5) arc (120:-200:1);
\draw[ thick, -Straight Barb] (6.5, 6.5) arc (150:-150:1);
    % The Weight for this edge
    % \draw[](9, 6) node [] {\highlight{$\trace_0 \to \env(\trace_0) k  $}};
\draw[ thick, -latex] (5, 6.5)  -- (5, 8.5) ;
% Control Dependency
\draw[ thick,-latex] (1.5, 7)  -- (4, 9) ;
\draw[ thick,-latex] (1.5, 4)  -- (4, 9) ;
\draw[ thick,-latex] (1.5, 7)  -- (4, 6) ;
\draw[ thick,-latex] (1.5, 4)  -- (4, 6) ;
\draw[ thick, -latex] (0, 1.5)  to  [out=-230,in=230]  (0, 9.5) ;
\end{tikzpicture}
\caption{}
   \end{centering}
   \end{subfigure}
\vspace{-0.4cm}
 \caption{(a) The program $\kw{towRounds(k)}$, an example 
%  of a program 
with two rounds of adaptivity (b) The corresponding execution-based dependency graph (c) The program-based dependency graph from $\THESYSTEM$.
}
\label{fig:overview-example}
% \vspace{-0.8cm}
\end{figure}
}
%%%%%%%%%%%%%%%%%%%%%%%%%%%%%%%%%%%%%%%%%%% Adaptfun %%%%%%%%%%%%%%%%%%%%%%%%%%%%%%%%%%%%%%%%%%% 
\section{The Adaptivity Analysis Algorithm - {\THESYSTEM}}
\label{sec:algorithm}
{The adaptivity from the static program analysis result is defined firstly. 
Then the algorithms for {\THESYSTEM}.
{\THESYSTEM} consists of three phases: 
\begin{enumerate}
    \item An algorithm to generate a precise data control flow graph
    \item An algorithm to perform a Reachability number analysis to calculate the weight of each node in the graph generated in phase 1.
    \item An algorithm to find the appropriate path in the weighted data control flow graph
\end{enumerate}
%
\subsection{Adaptivity Based on Program Analysis in \THESYSTEM}
%
\subsubsection{$\flowsto$}
Given a program  ${c}$ with its labelled variables $\lvar_c$,
and two variables ${x^i}, y^j  \in \lvar_c $ 
% showing up as $i$-th, $j$-th elements in $\lvar$ 
% (i.e., ${x} = \lvar(i)$ and ${y} = \lvar(j)$),
we say $y^j$ flows to ${x^i}$ in ${c}$ if and only if 
the value of $y^j$ directly or indirectly influence the evaluation of the value of ${x}$ as follows:
%
\begin{itemize}
\item (Explicit Influence) The program ${c}$ contains either 
a command $[\assign{{x}}{\aexpr}]^i$ or $[\assign{{x}}{\query({\qexpr})}]^i$,
such that ${y}$ shows up as a free variable in $\sexpr$ or ${\qexpr}$.
We use $\flowsto({x^i, y^j, c})$ to denote $y^j$ flows to $x^i$ in ${c}$.
%
\item (Implicit Influence) The program ${c}$ contains either a while loop
command
or if condition command, 
such that $x$ shows up in the guard
and $y$ shows up in the left hand of an assignment command, and this assignment command shows up
 in the body associated to that condition command.
\end{itemize}
%
This is formally defined in \ref{def:flowsto}.
We use $FV(\expr)$, $FV(\sbexpr)$ and $FV(\qexpr)$ denote the set of free variables in 
expression $\expr$, boolean expression $\sbexpr$ and query expression $\qexpr$ respectively.
%
\\
$\live^l(c) \subseteq \lvar_c$ 
is a subset of program $c$'s labelled variables.
For every labelled variable $x^l$ in this set, 
the value assigned to that variable
in the assignment command associated to that label is reachable at the entry point of  executing the command of label $l$.
This is formally defined and computed in the first step of the analysis algorithm.
%
\begin{defn}[Data Flows between Assigned Variables ($\flowsto$)].
\label{def:flowsto}
\\
In a program  ${c}$,
an variable ${x^i}  \in \lvar_c $ is in the \emph{flows to} relation with another variable ${y^j} \in \lvar_c$
in ${c}$, denoted as $\flowsto({x^i, y^j, c})$, is defined as follows:
%
\[
\begin{array}{l}
\flowsto({x^i, y^j, c}) \triangleq 
\\
\left( \bigvee
\begin{array}{l}
(\exists \sexpr . ~ [\assign{y}{\sexpr}]^j \in_{c} {c} 
\land {x} \in VAR(\sexpr) \land (x^i \in \live^j(c)))
\\
(\exists {\qexpr}. ~ [\assign{y}{\query({\qexpr})}]^j \in_{c} {c} 
\land x \in VAR({\qexpr}) \land (x^i \in \live^j(c))))
\\
\big(\exists  ~ {c_w}, {(\expr \lor \qexpr)}, \sbexpr, l \in \mathbb{N}. ~
	\ewhile [\sbexpr]^l \edo {c_w} \in_{c} {c}
	\land 
	[{\assign{y}{\expr \lor \query(\qexpr)}}]^{j} \in_{c}  {c_w}
\big) \land {x} \in VAR(\sbexpr) \land (x^i \in \live^l(c)))
\\
\big(
\exists ~ \sbexpr, l \in \mathbb{N}, {c_1}, {c_2}, {\expr}_1, {\expr}_2. ~
	\eif([\sbexpr]^l, {c_1}, {c_2}) \in_{c} {c} \land
	([{\assign{y}{\expr_1}}]^j \in_{c} {c_1} \lor 
	[{\assign{y}{\expr_2}}]^j \in_{c} {c_2})
\land {x} \in VAR(\sbexpr) \land (x^i \in \live^l(c)))
\big)
% \\
% (\exists z^r \in \lvar_c \st \flowsto(x^i, z^r, c) \land \flowsto(z^r, y^j, c))
\end{array}
\right).
\end{array}
\]
%
\end{defn}
%
Program Entry Point: $\entry_c : \mbox{Command} \to \mathbb{N}$ 
\[
  \entry_c \triangleq 
\left\{
  \begin{array}{ll} 
     l       
    & c = [\eskip]{}^l
    \\ 
    l    & c = [\assign{x}{\expr_1}]{}^l
    \\ 
    l      
    & c = [\assign{x}{\query(\qexpr_1)}]{}^l
    \\
   l
    & c_1 = \eif([b]{}^l, c_t, c_f)
    \\ 
    l         
    & c = \ewhile [b]^l \edo c'
    \\ 
    \entry_{c1}
    & c = c1;c2
  \end{array}
  \right.
\]
%
\begin{defn}[Equivalence of Program]
%
\label{def:aq_prog}
Given 2 programs $c_1$ and $c_2$:
\[
c_1 =_{c} c_2
\triangleq 
\left\{
  \begin{array}{ll} 
    \etrue        
    & c_1 = \eskip \land c_2 = \eskip
    \\ 
    \forall \trace \in \mathcal{T} \st \exists v \in \mathcal{VAL}
    \st \config{ \trace, \expr_1} \aarrow v \land \config{ \trace, \expr_1} \aarrow v     
    & c_1 = \assign{x}{\expr_1} \land c_2 = \assign{x}{\expr_2} 
    \\ 
    \qexpr_1 =_{q} \qexpr_2       
    & c_1 = \assign{x}{\query(\qexpr_1)} \land c_1 = \assign{x}{\query(\qexpr_2)} 
    \\
    c_1^f =_{c} c_2^f \land c_1^t =_{c} c_2^t
    & c_1 = \eif(b, c_1^t, c_1^f) \land c_2 = \eif(b, c_2^t, c_2^f)
    \\ 
    c_1' =_{c} c_2'         
    & c_1 = \ewhile b \edo c_1' \land c_2 = \ewhile b \edo c_2'
    \\ 
    c_1^h =_{c} c_2^h \land c_1^t =_{c} c_2^t
    & c_1 = c_1^h;c_1^t \land c_2 = c_2^h;c_2^t 
  \end{array}
  \right.
\]
%
$c_1 \neq_{c} c_2$  is defined vice versa.
%
\end{defn}
%
Given 2 programs $c$ and $c'$, $c'$ is a sub-program of$c$, i.e., $c' \in_{c} c$ is defined as:
\begin{equation}
c' \in_{c} c \triangleq \exists c_1, c_2, c''. ~ s.t.,~
c =_{c} c_1; c''; c_2 \land c' =_{c} c''
\end{equation} 
%

\subsubsection{Program Analysis Based Dependency Graph}
\begin{defn}
    [Program-Based Dependency Graph].
    \label{def:prog_graph}
    \\
Given a program ${c}$
its program-based graph 
$\progG({c}) = (\vertxs, \edges, \weights, \flag)$ is. defined as:
\\
\[
\begin{array}{rlcl}
\text{Vertices} &
\vertxs & := & \left\{ 
x^l \in \mathcal{VAR} \times \mathbb{N}
~ \middle\vert ~
x^l \in \lvar({{c}})
\right\}
\\
\text{Directed Edges} &
\edges & := & 
\left\{ 
  ({x}_1^{i}, {x}_2^{j}) \in \mathcal{VAR} \times \mathbb{N} \times (\mathcal{VAR} \times \mathbb{N})
  ~ \middle\vert ~
  \begin{array}{l}
    {x}_1^{i}, {x}_2^{j} \in \vertxs
	\land
    \\
    \exists n \in \mathbb{N}, z_1^{r_1}, \cdots, z_n^{r_n} \in \lvar_{{c}} \st 
    n \geq 0 \land
    \\
    \flowsto(x^i,  z_1^{r_1}, c) 
    \land \cdots \land \flowsto(z_n^{r_n}, y^j, c) 
  \end{array}
\right\}
\\
\text{Weights} &
\weights & := &
% \bigcup
% \begin{array}{l}
	\big\{ (x^l, w) \in \mathcal{VAR} \times \mathbb{N} \times (\mathbb{N} \cup \aexpr)
	\mid
	x^l \in \lvar_{{c}} \land w = \rb(x^l, c)
	\big\} 
	% \\
	% \big\{(x^l, 1)  \in \mathcal{VAR} \times \mathbb{N} \times \{1\} 
	% \mid
	% x^l \in \lvar_{{c}} \land \flag(x^l) = 0
	\big\}
% \end{array} 
\\
\text{Query Flags} &
\qflag & := & 
\left\{(x^l, n)  \in \mathcal{VAR} \times \mathbb{N}  \times \{0, 1\} 
~ \middle\vert ~
 x^l \in \lvar_{c},
 \left\{
\begin{array}{ll}
n = 1 & x^l \in \qvar_{c} \\ 
n = 0 & o.w.
\end{array}
\right\}
\right\}
\end{array}
\]
%
\[
\begin{array}{rlcl}
\text{Vertices} &
\vertxs & := & \left\{ 
x^l 
~ \middle\vert ~
x^l \in \lvar_{{c}}
\right\}
\\
\text{Directed Edges} &
\edges & := & 
\left\{ 
  ({x}_1^{i}, {x}_2^{j}) 
  ~ \middle\vert ~
  \begin{array}{l}
    {x}_1^{i}, {x}_2^{j} \in \lvar_{{c}},
    \\
    \exists n \in \mathbb{N}, z_1^{r_1}, \cdots, z_n^{r_n} \in \lvar_{{c}} \st 
    n \geq 0 \land
    \\
    \flowsto(x^i,  z_1^{r_1}, c) 
    \land \cdots \land \flowsto(z_n^{r_n}, y^j, c) 
  \end{array}
\right\}
\\
\text{Weights} &
\weights & := &
% \bigcup
% \begin{array}{l}
	\big\{ (x^l, w)
	\mid
	x^l \in \lvar_{{c}}, w = \rb(x^l, c)
	\big\} 
	% \\
	% \big\{(x^l, 1)  \in \mathcal{VAR} \times \mathbb{N} \times \{1\} 
	% \mid
	% x^l \in \lvar_{{c}} \land \flag(x^l) = 0
	\big\}
% \end{array} 
\\
\text{Query Flags} &
\qflag & := & 
\left\{(x^l, n)   
~ \middle\vert ~
 x^l \in \lvar_{c},
 x^l \in \qvar_{c} \implies n = 1,
 x^l \notin \qvar_{c} \implies n = 0
%  \left\{
% \begin{array}{ll}
% n = 1 & x^l \in \qvar_{c} \\ 
% n = 0 & o.w.
% \end{array}
% \right\}
\right\}
\end{array}
\]
\end{defn} 
%
Given a program ${c}$, we generate its program-based graph 
$\progG({c}) = (\vertxs, \edges, \weights, \qflag)$.
%
Then the adaptivity bound based on program analysis for ${c}$ is the number of query vertices on a finite walk in $\progG({c})$. This finite walk satisfies:
\begin{itemize}
\item the number of query vertices on this walk is maximum
\item the visiting times of each vertex $v$ on this walk is bound by its reachability bound $\weights(v)$.
\end{itemize}
It is formally defined in \ref{def:prog_adapt}.
%
%
\begin{defn}
[{Program-Based Adaptivity}].
\label{def:prog_adapt}
\\
{
Given a program ${c}$ and its program-based graph 
$\progG({c}) = (\vertxs, \edges, \weights, \qflag)$,
%
the program-based adaptivity for $c$ is defined as%
\[
\progA({c}) 
:= \max
\left\{ \qlen(k)\ \mid \  k\in \walks(\progG({c}))\right \}.
\]
}
\end{defn}  
%
%
% {
% \begin{defn}[Variable Flags ($\flag$)].
% \\
% Given a program  ${c}$ with its labelled variables $\lvar$, the $\flag$ is a vector of the same length as $\lvar$, s.t. for each variable ${x}$ showing up as the $i$-th element in $\lvar$ (i.e., ${x} = \lvar(i)$), 
% $\flag(i) \in \{0, 1, 2\}$ is defined as follows:
% %
% %
% \[
% \flag(i) := 
% \left\{
% \begin{array}{ll}
% 2 & 
% {x^l} \in \lvar_{c} \land 
% (\exists {\qexpr}. ~ s.t., ~
% [\assign{{x}}{\query({\qexpr})}]^l \in_{c} {c})
% \\
% 1 &  
% \begin{array}{l}
% {x^l} \in \lvar_{c} \bigwedge \\
% \left(
% \begin{array}{l}
% \big(\exists  ~ {c'}, {\expr}, \sbexpr, l, l'. ~
% 	\ewhile [\sbexpr]^l \edo {c'} \in_{c} {c}
% 	\land 
% 	[{\assign{x}{\expr}}]^{l'} \in_{c}  {c'}
% \big) \bigvee
% \\
% \big(\exists ~ \sbexpr, l, l_1, l_2, {c_1}, {c_2}, {\expr}_1, {\expr}_2. ~
% 	\eif([\sbexpr]^l, {c_1}, {c_2}) \in_{c} {c} \land
% 	([{\assign{x}{\expr_1}}]^{l1} \in_{c} {c_1} \lor 
% 	[{\assign{x}{\expr_2}}]^{l2} \in_{c} {c_2})
% \big)
% \end{array}
% \right)
% \end{array}
% \\
% 0 & \text{o.w.}
% \end{array}
% \right\}. 
% \] 
% %
% \end{defn}
%
% Operations on $\flag$ are defined as follows:
% \begin{equation}
% \begin{array}{llll}
% {\flag_1 \uplus \flag_2}(i) & := &
% \left\{
% \begin{array}{ll}
% k & k = \max{\big\{\flag_1(i), \flag_2(i)\big\}} 
% \land |\flag_1| = |\flag_2|\\
% 0 & o.w.
% \end{array}\right.
% & i = 1, \cdots, |\flag_1|  
% \\
% {\flag \uplus n}(i) & := & 
% \max\big\{ \flag(i), n \big\} 
% & i = 1, \ldots, |\flag|    
% \\
% \left[ n \right]^k (i) & := &  n
% & i = 1, \ldots, k ~ \land ~ |\left[ n \right]^k| = k
% \end{array}
% \end{equation}
%
%
%
% \begin{defn}[Data Flow Matrix ($\Mtrix$)]
% The data flow matrix $\Mtrix$ of a program $c$ is a matrix of size $|\lvar_c| \times |\lvar_c|$ 
% s.t.,
% %
% \[
% \Mtrix(i, j) \triangleq
% \left\{
% \begin{array}{ll}
% 1	&	\flowsto({x^i, y^j, c}) \\
% 0	& o.w.
% \end{array}
% \right., {x^i}, y^j  \in \lvar_c.
% \]
% %
% \end{defn}
% %
% Operations on the data flow matrices are defined as follows:
% %
% \begin{equation}
% \Mtrix_1 ; \Mtrix_2 
% := \Mtrix_2 \cdot \Mtrix_1 + \Mtrix_1 + \Mtrix_2
% \end{equation}
% %
% Consider the same program $c$ as above, its data flow matrix $\Mtrix$ and $\flag$ for the program $c$ is as follows:
% $$
% {c} = 
% \begin{array}{l}
% \left[{\assign {x_1} {\query(0)}}	\right]^1;
% \\
% \left[{\assign {x_2} {x_1 + 1}}		\right]^2;
% \\
% \left[{\assign {x_3} {x_2 + 2}}		\right]^3
% \end{array}
% ~~~~~~~~~~~~
% \Mtrix
% =  \left[ 
% \begin{matrix}
% 0 & 0 & 0 \\
% 1 & 0 & 0 \\
% 1 & 1 & 0 \\
% \end{matrix} \right] ~ , 
% \flag = \left [ \begin{matrix}
% 1 \\
% 0 \\
% 0 \\
% \end{matrix} \right ]
% $$
% %
% % There are two special matrices used for generating the data flow matrix $\Mtrix$ in the analysis algorithm. They are the left matrix $\lMtrix_i$ and right matrix $\mathsf{R_{(e, i)}}$.

% % Given a program  ${c}$ with its labelled variables $\lvar$ of length $N$,
% % the left matrix $\lMtrix_i$ generates a matrix of $1$ column, $N$ rows, 
% % where the $i$-th row is $1$ and all the other rows are $0$.
% % %
% % \begin{defn}[Left Matrix ($\lMtrix_i$)].
% % \\
% % Given a program  ${c}$ with its labelled variables $\lvar$ of size $N$, 
% % the left matrix $\lMtrix_i$ is defined as follows:
% % \[
% % \lMtrix_i(j) : = 
% % \left
% % \{
% % \begin{array}{ll}
% % 1 & j = i \\
% % 0 & o.w.
% % \end{array}
% % \right.,
% % j = 1, \ldots, N.
% % \]
% % \end{defn}
% % %
% % Given a program  ${c}$ with its labelled variables $\lvar$ of length $N$,
% % the right matrix $\rMtrix_{\expr, i}$ generates a matrix of one row and $N$ columns, 
% % where the locations of free variables in $\expr$ is marked as $1$. 
% % %
% % %
% % \begin{defn}[Right Matrix ($\rMtrix_{\expr}$)].
% % \\
% % Given a program  ${c}$ with its labelled variables $\lvar$ of length $N$, 
% % the right matrix $\rMtrix_{\expr}$ is defined as follows:
% % \[
% % \rMtrix_{\expr}(j) : = 
% % \left\{
% % \begin{array}{ll}
% % 1 & {x} \in FV(\expr) 
% % \\
% % 0 & o.w.
% % \end{array}
% % \right.,
% % {x} = \lvar(j) ~ , ~ j = 1, \ldots, N.
% % \]
% % %
% % %
% % \end{defn}
% % %
% % Using the same example program ${c}$ as above with labelled variables $\lvar = [ {x_1 , x_2 , x_3} ] $,
% % the left and right matrices w.r.t. its $2$-nd command 
% % $\left[{\assign {x_2} {x_1 + 1}}\right]^2$  are as follows:
% % \[
% % \lMtrix_1 = \left[ \begin{matrix}
% % 0   \\
% % 1 	 \\
% % 0   \\
% % \end{matrix}   \right ] 
% % ~~~~~~~~~~~~~~
% % \rMtrix_{{x}_1 + 1}
% % = \left[ \begin{matrix} 
% % 1 & 0 & 0 \\
% % \end{matrix}  \right]
% % \]
% %
% %
% %
\subsection{ $\THESYSTEM$ Analysis Algorithm}
\wq{To do: Add $\THESYSTEM$, a data flow analysis algorithm to scan the program and give a graph.}
{\THESYSTEM} consists of three phases: 
\begin{enumerate}
    \item An algorithm to generate a precise data control flow graph
    \item An algorithm to perform a Reachability number analysis to calculate the weight of each node in the graph generated in phase 1.
    \item An algorithm to find the appropriate path in the weighted data control flow graph
\end{enumerate}

To be precise, we show the details. 
\subsubsection{Phase 1, precise data control flow graph}
There are 3 steps to generate the graph in this phase.
\begin{enumerate}
\item Generation of control flow graph
    \item Reaching definition analysis
   \item  data-control flow analysis
\end{enumerate}

\paragraph{Generate CFG}
 Define $\mathsf{init}$: Command -> label, which returns the initial label of the statement. 
 \[
 \begin{array}{ll}
    init([x := e]^{l})  & = l  \\
     init([x := q(e)]^{l})  & = l \\
     init([skip]^{l})  & = l \\
     init([if [b]^l then C_1 else C_2]^{l})  & = l \\
     init([while [b]^l do C]^{l})  & = l \\
     init(C_1 ; C_2)  & = init(C_1) \\
 \end{array}
 \]
  Define $\mathsf{final}$: Command -> Powerset(label), which returns the final labels of the statement. 
 \[
 \begin{array}{ll}
    final([x := e]^{l})  & = \{l\}  \\
     final([x := q(e)]^{l})  & = \{l\}  \\
     final([skip]^{l})  & = \{l\} \\
     final([if [b]^l then C_1 else C_2]^{l})  & = final(C_1) \cup final(C_2) \\
     final([while [b]^l do C]^{l})  & = \{l\} \footnote{while terminates after b evaluates to false} \\
     final(C_1 ; C_2)  & =  final(C_2) \\
 \end{array}
 \]
 Define block B to be either the command of the form of assignment, skip, or test of the form of $[b]^{l}$.\\
 Define $\mathsf{blocks}$ : command -> Powerset(Block)
 \[
 \begin{array}{ll}
    blocks([x := e]^{l})  & = \{[x := e]^{l}\}  \\
     block([x := q(e)]^{l})  & = \{[x := q(e)]^{l}\}  \\
     blocks([skip]^{l})  & = \{[skip]^{l}\} \\
     blocks([if [b]^l then C_1 else C_2]^{l})  & = {[b]^{l}} \cup blocks(C_1) \cup blocks(C_2) \\
     blocks([while [b]^l do C]^{l})  & = \{[b]^{l}\} \cup blocks(C) \\
     blocks(C_1 ; C_2)  & = blocks(C_1) \cup  blocks(C_2) \\
 \end{array}
 \]
 Define $\mathsf{labels}$ to get the labels of blocks.
 \[
   labels(C) = \{l | [B]^{l} \in blocks(C) \}
 \]  

The control flow graph is generated by edges between labels. Define $\mathsf{flow}$: command -> P (label $\times$ label ).

\[
 \begin{array}{ll}
    flow([x := e]^{l})  & = \emptyset  \\
     flow([x := q(e)]^{l})  & = \emptyset  \\
     flow([skip]^{l})  & = \emptyset \\
     flow([if [b]^l then C_1 else C_2)  & =  flow(C_1) \cup flow(C_2)\cup \{(l, init(C_1)) , (l, init(C_2)) \} \\
     flow([while [b]^l do C)  & =  flow(C) \cup \{(l, init(C)) \} \cup \{(l', l)| l' \in final(C) \} \\
     flow(C_1 ; C_2)  & = flow(C_1) \cup  flow(C_2) \cup \{ (l,init(C_2)) | l \in final(C_1) \} \\
 \end{array}
 \]
 
 \paragraph{Reaching definition analysis}
 Set $?$ to be undefined, $label^{?}$ is label $\cup \{?\}$.\\
 Define $\mathsf{kill}$: blocks -> Powerset(Var $\times$ $label^l$, which produces the set of labelled variables of assignment destroyed by the block.\\
 Define $\mathsf{gen}$: blocks->Powerset(Var $\times$ $label^l$, which generates the set of labelled variables generated by the block.\\
 Define $defs(x)(C)$: Var -> Set of labels, gives all the labels where assigns value to variable x in the target program C.
  \[
 \begin{array}{ll}
    kill([x := e]^{l})  & = \{ (x, ?) \} \cup \{ (x, l') | l' \in defs(x) \} \\
     kill([x := q(e)]^{l})  & = \{ (x, ?) \} \cup \{ (x, l') | l' \in defs(x) \}  \\
     kill([skip]^{l})  & = \emptyset \\
     kill([ [b]^l ]^{l})  & =  \emptyset \\
      gen([x := e]^{l})  & = \{ (x, l) \}  \\
     gen([x := q(e)]^{l})  & = \{ (x, l) \}  \\
     gen([skip]^{l})  & = \emptyset \\
     gen([ [b]^l ]^{l})  & =  \emptyset 
 \end{array}
 \]
 Define $in(l)$, $out(l)$: label -> (var $\times$ $label^l$) for the entry point and exit point of the node $l$ in the control flow graph.
 \[
 \begin{array}{lll}
    in(l)  & = \{ (x, ?) | x is assigned in C  \} & l = init(C)\\
    & \cup \{ out(l')|  | (l',l) \in flow(C) \} & ow \\
     out(l)  & =  gen(B^{l}) \cup \{ in(l) \setminus kill(B^l)  \} & B^l \in blocks(C)   
 \end{array}
 \]
 
The Reaching definition is calculated by the Worklist algorithm as below.
\begin{enumerate}
    \item initial in[l]=out[l]=$\emptyset$
    \item initial in[entry label] = $\emptyset$
    \item initialize a work queue, contains all the blocks in C
    \item while |W| != 0 \\
         pop l in W\\
          old = out[l]\\
          in(l) =  out(l') where (l',l) in flow(C)\\
           out(l) = gen($b^l$) $\cup$ (in(l) - kill($b^l$) ) where $b^l$ in block(C)   \\
          if (old != out(l)) W= W $\cup$ \{l'| (l,l') in flow(C)\}\\
          end while
\end{enumerate}

 \paragraph{control-data flow graph}
 We build this graph by using the results of Reaching definition analysis, to be specific, in(l) for every label. 
 
 Define dcgd: Command -> Labelled VAR $\times$ 
 Labelled VAR. dcgd is short for data control
 dependency graph. $RD_{in}$ is the result of reaching definition in last step. 
 
  \[
 \begin{array}{ll}
    dcdg([x := e]^{l})  & = \{ (y^i, x^l) | y \in VAR(e) \land (y,i) \in RD_{in}(l) \}  \\
     dcdg([x := q(e)]^{l})  & = \{ (y^i, x^l) | y \in VAR(e) \land (y,i) \in RD_{in}(l) \}  \\
     dcdg([skip]^{l})  & = \emptyset \\
     dcdg([if [b]^l then C_1 else C_2)  & =  dcdg(C_1) \cup dcdg(C_2)\\ & \cup \{(x^i,y^j) | x \in VAR(b) \land (x,i) \in RD_{in}(l) \land ([y = \_]^j) \in blocks(C_1) \} \\
     &\cup \{(x^i,y^j) | x \in VAR(b) \land (x,i) \in RD_{in}(l) \land ([y = \_]^j) \in blocks(C_2) \} \\
     dcdg([while [b]^l do C)  & =  dcdg(C) \cup \{(x^i,y^j) | x \in VAR(b) \land (x,i) \in RD_{in}(l) \land ([y = \_]^j) \in blocks(C) \} \\
     dcdg(C_1 ; C_2)  & = dcdg(C_1) \cup  dcdg(C_2) \\
 \end{array}
 \]
 
 

\subsubsection{Phase 2, reachability number analysis}

In last phase, we get a dependency graph whose node is computation blocks uniquely decided by its label. In this phase, we want to add more information to every node in the graph, which is the approximated visiting times (how many times this block is exeucted). The algorithm is defined in Algorithm~\ref{alg:add_weights}, with 3 main functions, the PREPROCESSING, $\rb$ and ADDWEIGHT. 
\begin{algorithm}
\caption{
{Add weights to dependency graph (the main algorithm of phase 2)}
\label{alg:add_weights}
}
\begin{algorithmic}[1]
\REQUIRE the program $c$, the dependency graph $G = (\vertxs, \edges, \weights, \qflag)$ from phase 1
\STATE  prel = PREPROCESSING(c) 
\STATE {\bf for} $x^l \in \vertxs$: 
\STATE \qquad {\bf if} $l \in prel$:
\STATE \qquad \qquad $\weights(x^l) = \rb(c, l)$
\STATE \qquad {\bf else}:
\STATE \qquad \qquad $\weights(x^l) = 1$
\RETURN $G$
\end{algorithmic}
\end{algorithm}

\paragraph{Preprocessing} This step is straightforward. Since variable's visiting time outside of any while loop is at most 1, we do not need to analyze the visiting times of every node in the graph from phase 1.
We have a pre-processing algorithm to go through the programs and returns the list of labels associating with a loop and whose visiting times need to be analyzed.

\paragraph{Getting Reachability Bounds}
This is defined in Algorithm~\ref{alg:rb}.
To be precise, we use static analysis method from \cite{Sumit2010rechability}, which is able to "provide the symbolic worst case bound on the number of times a block is reached", let us call it reachability bound analysis. 

This analysis only works to find the symbolic bound of one block (in our graph, it corresponds to one node). This algorithm is summarized as follows.
\begin{enumerate}
    \item Build a transition system which describes the relation between variables in this target block and these variables in the successive visit to this block. There are defined translate functions which translate the statement to transition system and the corresponding operations such as the composition of transition systems, merging two transition systems from two control branches and so on. It is worth to mention that it calculates the transitive closure of the transition system obtained from a loop body, which can be analogy to computing the invariant of a loop.  
    \item Use Ranking function which takes the transition system and outputs the bound
\end{enumerate}

\begin{algorithm}
\caption{
{Reachability Bound Analysis ($\rb$)}
\label{alg:rb}
}
\begin{algorithmic}[1]
\REQUIRE the program $C$, the target while loop with label $l$.
\STATE  T  = GenerateTransitionSystem(C,l) 
\STATE B = 1 + ComputeBound(T)
\RETURN B
\end{algorithmic}
\end{algorithm}

The algorithm $GenerateTransitionSystem(C,l)$ can be described as follows. It uses the control flow graph generated from the program $C$, and splits the node marked by $l$ into two nodes $l_1$ and $l_2$ to generate a new control flow graph from the $l_1$ to $l_2$. It translates the node in the graph into a transition systems by the its translate function and replaces node with transition systems. For loops in the graph, the loop itself is replaced by the transitive closure of the transition systems of its body. Finally, the new generated control flow graph can be transformed to a transition system. The transition system is a disjunction of transitions, and every transition is expressed as a conjuction of formulas over program variables $x,y,z$ in the target block (l) and its successive visits $x',y',z'$ in the same block.
\\
The transition formula for each command are as follows:
\\
\todo{adding the naive transition formula for assignment and if}
\\
\jl{$translate(c)$  is defined as follows:}
\\
$translate(\assign{x}{e}) = \{x := e\} if x \in VAR(b)$
\\
% $translate(\assign{x}{e}) = \{\} if x \notin VAR(b)$
% \\
$translate(\assign{x}{\qexpr}) = \{x \in VAR(b) \implies x := \qexpr\} $
\\
% $translate(\assign{x}{\qexpr}) = \{\} if x \notin VAR(b)$
% \\
$translate(\eif(\bexpr, c_1, c_2)) = \{\bexpr \implies translate(c_1), \neg\bexpr \implies translate(c_2)\}$
\\
$translate(\ewhile(\bexpr, c_w)) = \{ ComputeBound(GenerateTransitionSystem(C,l)) \times translate(c_w) \}$
\\
$translate(c_1;c_2) = translate(c_1) + translate(c_2)$
\\
\todo{adding the compose method for composing 2 transition formulas}
$compose\{x := e_1, \cdots, x := e_2 \} = \{x := e_2\}$
\\
$\{y := e_1, \cdots, x := e_2 \} \land \flowsto(y, x, c) = \{x := [y \to e_1] e_2\}$
\\
\todo{add the $ComputeBound$ function, specifically the ranking function}
\\
The function $computeBound$ takes into a transition system (a disjuction of transitions), and computes the bound. There are ranking functions which take a transition and return the bound, that is used by $computeBound$. There are some heuristics in compute the bound based on the transitions systems, if interested, please look at the paper for more details.

\paragraph{Add Weight} We also need to take care about the situation when a bound can not be predicted by {$\rb$}, we need to use another loose analysis to get a loose bound.



\clearpage
\subsubsection{Phase 3, path finding algorithm in weighted graph (graph in phase 1 with weights predicted in phase 2) }
A combination of DFS and BFS Algorithm:
\begin{algorithm}
    \caption{
    {Longest Adaptivity Search Algorithm ($\pathsearch$)}
    \label{alg:adpt_alg}
    }
    \begin{algorithmic}[1]
    \REQUIRE $G = (\vertxs, \edges, \weights, \qflag)$ \#\{A Symbolic Weighted Directed Graph\}
    % with a start vertex $s$ and destination vertex $t$ .
    \STATE  {\bf {$\kw{\pathsearch(G)}$}:}  
    \STATE {\bf init} 
    % \\
    % current node: $c$, 
    \\
    $q$: empty queue.
    % \\
    % $\kw{visited}$: List of length $|\vertxs|$, initialize with $\efalse$.
    % \\
    % $\kw{SSCvisited}$: List of length $|\vertxs|$, initialize with $\efalse$.
    % \\ 
    % $\kw{adapt_{scc}(SCC_i) = \pathsearch_{scc}(SCC_i)}$.
    \\
    $\kw{adapt}$ : the adaptivity of this graph initialize with $0$.
    \\
    \STATE Find all Strong Connected Components (SCC) in $G$: $\kw{SCC_1}, \cdots, \kw{SCC_n}, 0 \leq n \leq |\vertxs|$, where $\kw{SCC_i} = (\vertxs_i, \edges_i, \weights_i, \qflag_i)$.
    % and assign each vertex $x^i$ with an SCC number $\kw{SCC}(x^i)$
    \STATE {\bf for} every SCC: $\kw{SCC_i}$, compute its Adaptivity $\kw{SCC_i}$:
    \STATE \quad $\kw{adapt_{scc}[SCC_i] = \pathsearch_{scc}(SCC_i)}$;
    \STATE {\bf for} every vertex $\kw{SCC_i}$:
    \STATE \qquad $q.append(\kw{SCC_i})$;
    \STATE \qquad $\kw{adapt_{tmp}} = 0$;
    \STATE \qquad {\bf while} $q$ isn't empty:
    \STATE \qquad \qquad $\kw{SCC}= q.pop()$;  \#\{take the top SCC from head of queue\}
    \STATE \qquad \qquad  $\kw{adapt_{tmp}}_0= \kw{adapt_{tmp}}$; \#\{record the adaptivity of last level\}
    \STATE \qquad \qquad  $\kw{SCC_{max}}$;  \#\{record the SCC with longest walk in this level\}
    % initialize cycle-adapt = 0.
    \STATE \qquad \qquad {\bf for} all children SCC of $\kw{SCC}$: $\kw{SCC'}$:
    % \STATE \qquad \qquad   cycle-adapt$ = \max($cycle-adapt, $\kw{dfs_{refine}(G, v, v)})$;
    % \STATE \qquad \qquad \qquad \#\{compute the adaptivity of vertex $v$  on $\kw{SCC}(v)$, and update r[v] with the SCC-adapt\}
    % \STATE \qquad \qquad \qquad $ r[v] = r[s] + \kw{dfs_{refine}(G, v, visited)})$; 
    \STATE \qquad \qquad \qquad {\bf if} $(\kw{adapt_{tmp}} < \kw{adapt_{tmp}}_0 + \kw{adapt_{scc}(SCC')})$:
    \STATE \qquad \qquad \qquad \qquad $\kw{adapt_{tmp}} = \kw{adapt_{tmp}}_0 + \kw{adapt_{scc}[SCC']}$; 
    \STATE \qquad \qquad \qquad \qquad $\kw{SCC_{max} = SCC'} $; \#\{update the SCC with longest walk in this level\} 
    % \STATE \qquad   $r[c] = r[c] + $cycle-adapt;
    % \STATE \qquad for all unvisited vertex $v$ having directed edge from c and $! \kw{cycle}(c)$:
    % \STATE \qquad \qquad $r[v] = r[c] + \flag(v)$; 
    % \STATE \qquad \qquad \qquad  \#\{mark all the nodes with the same $\kw{SCC}$ number as visited\} 
    % \STATE \qquad \qquad \qquad  \#\{append the unvisited vertex to the rear of the queue\}
    % \STATE \qquad \qquad \qquad  \#\{mark all the nodes with the same $\kw{SCC}$ number as visited\} 
    % \STATE \qquad \qquad for $v \in V$,   $\kw{visited}[s] = 1$;
    \STATE \qquad \qquad \qquad $q.append(\kw{SCC_{max}})$;
    \STATE \qquad $\kw{adapt} = \max(\kw{adapt}, \kw{adapt_{tmp}})$;    
    \RETURN $\kw{adapt}$.
    \end{algorithmic}
    \end{algorithm}
    %
    %
    % \begin{algorithm}
    % \caption{
    % {Longest Adaptivity Search Algorithm ($\pathsearch$)}
    % \label{alg:adpt_alg}
    % }
    % \begin{algorithmic}
    % \REQUIRE Weighted Directed Graph $G = (\vertxs, \edges, \weights, \flag)$ with a start vertex $s$ and destination vertex $t$ .
    % \STATE  {\bf {bfs $(G)$}:}  
    % \STATE {\bf init} 
    % \\
    % current node: $c$, 
    % \\
    % queue: $q$ : List, add into $a$ an arbitrary v from $\vertxs$. 
    % \\
    % visited: List of length $|\vertxs|$, initialize with $\efalse$.
    % \\
    % results: $r$ : List of length $|\vertxs|$, initialize with -1.
    % \\
    % curr$\kw{flowcapacity}$: INT, initialize MAXINT.
    % \\
    % querynum: INT, initialize 0. \#\{To count the query numbers when we are walking inside a cycle\}
    % \\
    % \STATE \qquad {\bf while} $q$ isn't empty:
    % \STATE \qquad \qquad take the vertex from head $c= q.pop()$
    % \STATE \qquad \qquad mark $c$ as visited, visited $[c] = 1$.
    % \STATE \qquad \qquad {\bf if} $\kw{cycle}(c)$  \#\{we are inside a cycle\}
    % \STATE \qquad \qquad \qquad curr$\kw{flowcapacity}$ = min($\weights$(c), curr$\kw{flowcapacity}$).
    % \STATE \qquad \qquad \qquad querynum += $\flag(c)$.
    % \STATE \qquad \qquad  \qquad for all unvisited vertex $v$ having directed edge from c:
    % \STATE \qquad \qquad \qquad \qquad r[v] = r[c]; q.add(v)
    % \STATE \qquad \qquad \qquad  {\bf if}  $v$ is visited, then the circle finished
    % \STATE \qquad \qquad \qquad \qquad update the result $r[v] =  \max(r[v], r[c] + $curr$\kw{flowcapacity}$*querynum)
    % \STATE \qquad \qquad \qquad \qquad curr$\kw{flowcapacity}$ = MAXINT
    % \STATE \qquad \qquad \qquad \qquad querynum = 0.  
    % \STATE \qquad \qquad {\bf else} 
    % \STATE \qquad \qquad \qquad for all unvisited vertex $v$ having directed edge from c:
    % \STATE \qquad \qquad \qquad  \qquad $r[v] = \max(r[v], r[c] + \flag(c))$; q.add(v)
    % \RETURN max($r$)
    % \end{algorithmic}
    % \end{algorithm}
    %
    \begin{algorithm}
      \caption{
      {Over-Approximated Adaptivity on SCC}
      \label{alg:overadp_alg}
      }
      \begin{algorithmic}[1]
      \REQUIRE $G = (\vertxs, \edges, \weights, \qflag)$ \#\{An Strong Connected Symbolic Weighted Directed Graph\}
      % with a start vertex $s$ and destination vertex $t$ .
      \STATE {\bf {$\kw{\pathsearch_{scc-naive}(G)}$}:}  
      \STATE {\bf init} 
      \\
      $\kw{r_{scc}}$: the Adaptivity of this SCC
      % \STATE  {\bf def} {$\kw{dfs_{naive}(G, c,visited)}$}: 
      % % \STATE {\bf init} 
      % % \\
      % % current node: $c$, 
      % % \\
      % % visited: List of length $|\vertxs|$, initialize with $\efalse$.
      % % \\
      % % \STATE {\bf if} $c = s$:
      % % \RETURN \qquad  $\weights(s)*\flag(s) $.
      % \STATE \qquad $r[c] = \weights(c)*\qflag(c) $
      % \STATE \qquad {\bf for}  all vertex $v$ having directed edge from $c$:
      % \STATE \qquad \qquad {\bf if}  $v$ is unvisited:
      % \STATE \qquad \qquad \qquad  \#\{mark $v$ as visited\} $\kw{visited}[v] = 1$;
      % \STATE \qquad \qquad \qquad $r[c] += \kw{dfs_{naive}(G, v, visited)}$;
      % \STATE \qquad {\bf else}: \#\{There is a cycle finished\}
      % \RETURN \qquad \qquad $\weights(v)*\flag(v) $.
      \STATE  {\bf for} every vertex $v$ in $\vertxs$:
      % \STATE  \qquad initialize \kw{visited} with $\efalse$.
      \STATE  \qquad $r_{scc} += \weights(v)*\qflag(v)$  
      \RETURN $r[c]$
      \end{algorithmic}
      \end{algorithm}%
%
    % \begin{algorithm}
    %     \caption{
    %     {Over-Approximated Adaptivity on SCC}
    %     \label{alg:overadp_alg}
    %     }
    %     \begin{algorithmic}
    %     \REQUIRE Weighted Directed Graph $G = (\vertxs, \edges, \weights, \qflag)$ with a start vertex $s$ and destination vertex $t$ .
    %     \STATE  {\bf {$\kw{dfs_{naive}(G, c,visited)}$}:}  
    %     % \STATE {\bf init} 
    %     % \\
    %     % current node: $c$, 
    %     % \\
    %     % visited: List of length $|\vertxs|$, initialize with $\efalse$.
    %     % \\
    %     % \STATE {\bf if} $c = s$:
    %     % \RETURN \qquad  $\weights(s)*\flag(s) $.
    %     \STATE $r[c] = \weights(c)*\qflag(c) $
    %     \STATE {\bf for}  all vertex $v$ having directed edge from $c$:
    %     \STATE \qquad {\bf if}  $v$ is unvisited:
    %     \STATE \qquad \qquad  \#\{mark $v$ as visited\} $\kw{visited}[v] = 1$;
    %     \STATE \qquad \qquad $r[c] += \kw{dfs_{naive}(G, v, visited)}$;
    %     % \STATE \qquad {\bf else}: \#\{There is a cycle finished\}
    %     % \RETURN \qquad \qquad $\weights(v)*\flag(v) $.
    %     \RETURN $r[c]$
    %     \end{algorithmic}
    %     \end{algorithm}%
        %
    \begin{algorithm}
            \caption{
            {Adaptivity on $\kw{SCC}$}
            \label{alg:adaptscc}
            }
            \begin{algorithmic}[1]
              \REQUIRE $G = (\vertxs, \edges, \weights, \qflag)$ \#\{An Strong Connected Symbolic Weighted Directed Graph\}
            \STATE  {\bf {$\kw{\pathsearch_{scc}(G)}$}:}  
            \STATE {\bf init} 
            \\
            $\kw{r_{scc}}$: the Adaptivity of this SCC
            \STATE  {\bf def} {$\kw{dfs(G, c,visited)}$}:
            \STATE \qquad {\bf init} 
            % \STATE \qquad current node: $c$, 
            % \\
            % visited: List of length $|\vertxs|$, initialize with $\efalse$.
            \\ \qquad  $\kw{r_{scc}}$ : initialize $0$, the adaptivity of this graph
            \\ \qquad  $\kw{r}$ : INT List of length $|\vertxs|$, initialize with $\qflag(v)$ for every vertex. The adaptivity reaching each vertex.
            \\ \qquad  $\kw{flowcapacity}$: INT List of length $|\vertxs|$, initialize MAXINT. 
            \#\{For every vertex, recording the minimum weight when the walk reaching 
            that vertex, inside a cycle\}
            \\ \qquad  $\kw{querynum[v]}$: INT List of length $|\vertxs|$, initialize with $\qflag(v)$ for every vertex. 
            \#\{For every vertex, recording the query numbers when the path reaching 
            that vertex, inside a cycle\}
            \\
            % \STATE {\bf if} $c = s$:
            % \STATE \qquad update the length of the longest path reaching this vertex
            % $r[s] =  r[s] + $$\kw{flowcapacity}$[s] * querynum[s].
            % \RETURN  \qquad $r[s]$.      
            \STATE \qquad {\bf for}  all vertex $v$ having directed edge from $c$:
            \STATE \qquad \qquad {\bf if} $\kw{visited}[v] = \efalse$:
            \STATE \qquad \qquad \qquad $\kw{flowcapacity[v] = \min(\weights(v), {flowcapacity}[c])}$;
            \STATE \qquad \qquad \qquad $\kw{querynum[v] = querynum[c] + \qflag(v)}$;
            % \STATE \qquad \qquad \qquad \#\{do not update the length of the longest walk reaching $v$ until the cycle is finished\}
            % \STATE \qquad \qquad \qquad $\kw{r[v] =  r[c] + flowcapacity[v] \times querynum[v]} $; \#\{do not update the length of the longest walk reaching $v$ until the cycle is finished\}
            \STATE \qquad \qquad \qquad $\kw{r[v] =  \max(r[v], flowcapacity[v] \times querynum[v]}) $; 
            % \#\{do not update the length of the longest walk reaching $v$ until the cycle is finished\}
            \STATE \qquad \qquad \qquad  $\kw{visited}[v] = \etrue$; \#\{mark $v$ as visited\}
            \STATE \qquad \qquad \qquad $\kw{dfs_{refine}(G, v, visited)}$;
            \STATE \qquad \qquad {\bf else}: \#\{There is a cycle finished\}
            % \STATE \qquad \qquad \qquad \#\{update the length of the longest path reaching this vertex\}
            \STATE \qquad \qquad \qquad 
            $\kw{r[v] =  \max(r[v], r[c] +  \min(\weights(v), {flowcapacity}[c]) * (querynum[c] + \qflag(v)))}$; \#\{update the length of the longest walk reaching this vertex on this cycle\}
            %  $\kw{r[v] =  \max(r[v], r[c] + flowcapacity[v] * querynum[v])}$; \#\{update the length of the longest walk reaching this vertex on this cycle\}
            %  \STATE \qquad \qquad \qquad \#\{Recover the $\kw{flowcapacity}$ and querynumber to previous state, for different loops\}
            % \STATE \qquad \qquad \qquad $\kw{flowcapacity[v] = flowcapacity[c]}$; \#\{Recover the $\kw{flowcapacity}$\}
            % \STATE \qquad \qquad \qquad $\kw{querynum[v] = querynum[c]}$;\#\{Recover the $\kw{querynum}$\}
            \STATE \qquad {\bf return}  $\kw{r[c]}$
            \STATE  {\bf for} every vertex $v$ in $\vertxs$:
            \STATE  \qquad initialize the $\kw{visited}$ list with $\efalse$.
            \STATE  \qquad $\kw{r_{scc} = \max(r_{scc}, dfs(G, v, \kw{visited} ))}$  
            \RETURN  $\kw{r_{scc}}$
            \end{algorithmic}
            \end{algorithm}

            % \begin{algorithm}
        % \caption{
        % {Refined Adaptivity on $\kw{SCC}$}
        % \label{alg:dfscycle_alg}
        % }
        % \begin{algorithmic}
        % \REQUIRE Weighted Directed Graph $G = (\vertxs, \edges, \weights, \qflag)$ with a start vertex $s$ and destination vertex $t$ .
        % \STATE  {\bf {$\kw{dfs_{refine}(G, c, visited)}$}:}  
        % \STATE {\bf init} 
        % \\
        % current node: $c$, 
        % % \\
        % % visited: List of length $|\vertxs|$, initialize with $\efalse$.
        % \\
        % results: $r$ : INT List of length $|\vertxs|$, initialize with $\qflag(v)$ for every vertex.
        % \\
        % $\kw{flowcapacity}$: INT List of length $|\vertxs|$, initialize MAXINT. 
        % \#\{For every vertex, recording the minimum weight when the walk reaching 
        % that vertex, inside a cycle\}
        % \\
        % querynum: INT List of length $|\vertxs|$, initialize with $\qflag(v)$ for every vertex. 
        % \#\{For every vertex, recording the query numbers when the walk reaching 
        % that vertex, inside a cycle\}
        % \\
        % % \STATE {\bf if} $c = s$:
        % % \STATE \qquad update the length of the longest path reaching this vertex
        % % $r[s] =  r[s] + $$\kw{flowcapacity}$[s] * querynum[s].
        % % \RETURN  \qquad $r[s]$.      
        % \STATE {\bf for}  all vertex $v$ having directed edge from $c$:
        % \STATE \qquad \qquad $\kw{flowcapacity}$[v] = min($\weights(v)$, $\kw{flowcapacity}$[c]);
        % \STATE \qquad \qquad querynum[v] = querynum[c] + $\qflag(v)$;
        % \STATE \qquad \qquad \#\{do not update the length of the longest walk reaching $v$ until the cycle is finished\}
        % \STATE \qquad \qquad $r[v] =  r[c] $;
        % \STATE \qquad {\bf if}  $v$ is unvisited:
        % \STATE \qquad \qquad \#\{mark $v$ as visited\} $\kw{visited}[v] = 1$;
        % \STATE \qquad \qquad $\kw{dfs_{refine}(G, v, visited)}$;
        % \STATE \qquad {\bf else}: \#\{There is a cycle finished\}
        % \STATE \qquad \qquad \#\{update the length of the longest path reaching this vertex\}
        % \STATE \qquad \qquad 
        %  $r[v] =  \max(r[v], r[c] + $$\kw{flowcapacity}$[v] * querynum[v]);
        %  \STATE \qquad \qquad \#\{Recover the $\kw{flowcapacity}$ and querynumber to previous state, for different loops\}
        %  \STATE \qquad \qquad $\kw{flowcapacity}$[v] = $\kw{flowcapacity}$[c];
        %  \STATE \qquad \qquad querynum[v] = querynum[c];
        % \RETURN  $r[c]$
        % \end{algorithmic}
        % \end{algorithm}
        % %

\begin{thm}[Soundness of $\pathsearch$]
    \label{thm:sound_adaptalg}
    For every program $c$, given its \emph{Program-Based Dependency Graph} $\progG$,
     $$\pathsearch(\progG) \geq \progA(\progG).$$
\end{thm}
% \begin{thm}[Soundness of $\pathsearch$]
  \label{thm:sound_adaptalg}
  For every program $c$, given its \emph{Program-Based Dependency Graph} $\progG$,
   $$\pathsearch(\progG) \geq \progA(\progG).$$
\end{thm}
proof Summary:
\\
1. 
\\
2. for every two nodes with a walk $k_{x,y}$ from $x$ to $y$ on $\progG$, we have a path
 $p_{x,y} = (x, v_1, \cdots, y)$ by the $\kw{bfs}$ algorithm,
and $adapt[\sccgraph(x)] + adapt[\sccgraph(v_1)] + \cdots + adapt[\sccgraph(y)] \geq \{\qlen(k_{x,y})\}$.
\\
Then we have
$\max\{adapt[\sccgraph(x)] + adapt[\sccgraph(v_1)] + \cdots + adapt[\sccgraph(y)] | x, y \in \progG, p_{x,y} \in \paths(k_{x,y}) \}
\geq \max\{\qlen(k_{x,y}) | x, y \in \progG, k_{x,y} \in \walks(k_{x,y})\}$
\\
i.e., 
$\pathsearch(\progG(c)) \geq \progA(c)$.
\begin{proof}
  Taking arbitrary program $c \in \cdom$, let $\progG(c) = (\progV, \progE, \progW, \progF)$ be its 
  program based dependency graph.
  Taking arbitrary walk $k_{x,y} \in \walks{\progG}$, with vertices sequence
  $(x, s_1, \cdots, y)$, it is sufficient to show:
  \[
    \qlen(k_{x,y}) = \len(s | s \in (x, s_1, \cdots, y) \land \qflag(s) = 1) \leq \pathsearch(\progG(c))
  \]
  By $\pathsearch(\progG)$ algorithm, let $\kw{\sccgraph_1}, \cdots, \kw{\sccgraph_n}$ be all the strong connected components on $\progG$ with $0 \leq n \leq |\vertxs|$,
  where each $\kw{\sccgraph_i} = (\vertxs_i, \edges_i, \weights_i, \qflag_i)$,
  \\
  and $\kw{adapt_{scc}(\sccgraph_i)}$ be the results of $\pathsearch_{scc}(\sccgraph_i)$ for each $\sccgraph_i$.
    % i.e.,
  % \[
  %   \]
  \\
  There are 2 cases:
  \caseL{$x, y$ on the same SCC}.
  Let  $\sccgraph$ be this SCC $x$ and $y$ on, by Lemma~\ref{lem:sound_adaptalg_scc}, we know
  \[
    \qlen(k_{x,y}) \leq \max\{\qlen(k) | k \in \walks(\sccgraph)\} \leq \pathsearch_{scc}(\sccgraph)
  \]
%
By $\pathsearch(\progG)$ algorithm,let $\kw{adapt}$ be the output variable,
we know $\kw{adapt} \geq \kw{adapt_{tmp}} \geq  \kw{adapt_{scc}(SSC)} $.
\\
i.e., 
\[
  \qlen(k_{x,y}) \leq \pathsearch(\progG(c)) 
  \]
This case is proved.
%
%
\caseL{$x, y$ on different SSC}.
Let $\sccgraph_x, \sccgraph_1, \cdots, \sccgraph_m, \sccgraph_y, 0 \leq m$ be all the SCC this walk pass by, where each vertex in 
$(x, s_1, \cdots, s_n, y) $ belongs to a single SCC number. 
\\
By the property of SCC, we know every 2 SCCs are single direct connected. Then we can divide this walk into $m+2$ sub-walks:
\\
$k_x = (x, s_1, \cdots, s_{scc_x})$;
\\
$k_1 = (s_{scc_x}, \cdots, s_{scc_1})$;
\\
$\cdots$
\\
$k_y = (s_{scc_m}, \cdots, s_y)$;
\\
where $k_x \in \walks(\sccgraph_x), \cdots, k_y \in \walks(\sccgraph_y)$.
\\
By Lemma~\ref{lem:sound_adaptalg_scc}, we know for each walk $k_i$:
\[ \qlen(k_i) \leq \max\{\qlen(k_i) | k_i \in \walks(\sccgraph_i)\} \leq \pathsearch_{scc}(\sccgraph_i) = \kw{adapt_{scc}(SSC_i)} \]
%
Then we have:
\[ 
  \qlen(k_{x,y}) = \qlen(k_x) + \qlen(k_1) + \cdots + \qlen(k_y) \leq 
  \kw{adapt_{scc}(SSC_x)} + \kw{adapt_{scc}(SSC_1)}  + \cdots + \kw{adapt_{scc}(SSC_y)}
  \leq \kw{adapt}
  \]
, where $\kw{adapt}$ is the output of $\pathsearch(\progG)$.
This case is proved.
\end{proof}

\begin{lem}[Soundness of $\pathsearch_{scc}$]
  \label{lem:sound_adaptalg_scc}
  For every program $c$, given its \emph{Program-Based Dependency Graph} $\progG$, if $\sccgraph$ is a strong connected sub-graph of $\progG$, then
  $\max\{\qlen(k) | k \in \walks(\sccgraph)\} \leq \pathsearch_{scc}(\sccgraph) $.
  %
  \[
    \forall c \in \cdom, \sccgraph \in \mathcal{Graph} \st \sccgraph \subseteq_{\kw{graph}} \progG(c)
    \implies 
    \max\{\qlen(k) | k \in \walks(\sccgraph)\} \leq \pathsearch_{scc}(\sccgraph) 
    \]
\end{lem}

ProofSummary:
\\
(1) for each node $x$ on SCC, by property of SCC, 
for every walk on SCC $k_{x, x} = (x, s_1, \cdots, x)$,
with set of unique vertex $\{v_1, \cdots, x\}$
there are $\paths(p_{x,x})$ on $\sccgraph$.
\\
(2) For every path $p_{x,x}^{i} = (x, v_1, \cdots, x) \in \paths(p_{x,x})$,  
$\kw{flowcapacity} (p_{x,x}^{i})$ is the maximum visiting times for every $v \in (x, v_1, \cdots, x)$, 
$\visit(s) (s_1, \cdots, x)) \leq \kw{flowcapacity}(p_{x,x}^{i})$;
\\
(3) $\kw{querynum}(p_{x,x}^{i})  * \kw{flowcapacity}(p_{x,x}^{i}$)  $\geq\len(s | s \in ( s_1, \cdots, x) \land \qflag(s) = 1) =  \qlen(k)$,
\\
(4) Then, the $\max\limits_{p_{x,x}^{i} \in \paths(p_{x,x})} \geq \max\{\qlen(k_{x, x}) | k_{x, x} \in \walks(k_{x, x})\}$
\\
(5) Then,  $\max\{\kw{querynum}(p_{x,x}^{i})  * \kw{flowcapacity}(p_{x,x}^{i}) | x \in \sccgraph \land {p_{x,x}^{i} \in \paths(p_{x,x})} \} 
\geq \max\{\qlen(k_{x, x}^i) |x \in \sccgraph \land  k_{x, x}^i \in \walks(k_{x, x})\}$
\\
(6) We also know by the property of SCC, $\forall x, y \in \sccgraph, $ let $k_{x, y}$ be arbitrary walk on $\sccgraph$,
 $\qlen(k_{x, y}) \leq \max\{\qlen(k_{x, x}^i) | k_{x, x}^i \in \walks(k_{x, x})\}$.
\\
(7) Then,$ \max\{\qlen(k_{x, x}^i) |x \in \sccgraph \land  k_{x, x}^i \in \walks(k_{x, x})\} \geq  \max\{\qlen(k_{x, y}^i) |x, y \in \sccgraph \land  k_{x, y}^i \in \walks(k_{x, y})\}$
\\
i.e., 
$ \max\{\qlen(k_{x, x}^i) |x \in \sccgraph \land  k_{x, x}^i \in \walks(k_{x, x})\} \geq  \max\{\qlen(k) | k\in \walks(\sccgraph)\} = \progA(\sccgraph)$.
\\
(8) We also know 
$\pathsearch_{scc}(\sccgraph) = \max\{\kw{querynum}(p_{x,x}^{i})  * \kw{flowcapacity}(p_{x,x}^{i}) | x \in \sccgraph \land {p_{x,x}^{i} \in \paths(p_{x,x})} \} $ by the $\pathsearch_{scc}$ algorithm.
\\
Then we have
$\pathsearch_{scc}(\sccgraph) \geq \progA(\sccgraph)$
\\
\begin{proof}
  Taking arbitrary program $c \in \cdom$, let $\progG(c) = (\vertxs, \edges, \weights, \qflag)$ be its 
  program based dependency graph and $\sccgraph = (\sccV, \sccE, \sccW, \sccF)$ be an arbitrary sub SCC graph of $\progG$.
  \\
There are 2 cases:
\caseL{$\sccgraph$ contains no edge and only 1 vertex $v$, i.e., $|\edges| = 0 \land |\vertxs| = 1$}.
%
In this case there is no walk in this graph, i.e., $\walks(\sccgraph) = \emptyset$.
\\
The adaptivity is $\qflag(v)$.
\\
This case is proved.
  %
  \caseL{$\sccgraph$ contains at least 1 edge and at least 1 vertex $v$, i.e., $1 \leq |\edges| \land 1 \leq |\vertxs|$}
%
  Taking arbitrary walk $k_{x,y} \in \walks{(\sccgraph})$, with vertices sequence
  $(x, s_1, \cdots, y)$, it is sufficient to show:
  \[
    \qlen(k_{x,y}) = \len(s | s \in (x, s_1, \cdots, y) \land \qflag(s) = 1) \leq \pathsearch_{scc}(\sccgraph)
  \]
  By $\pathsearch_{scc}(\sccgraph)$ algorithm line 19, in the iteration where $x$ is the starting vertex,
  we know $\pathsearch_{scc}(\sccgraph) = \kw{r_{scc}} = \max(\kw{r_{scc}, \kw{dfs(\sccgraph, x, visited)}})$,
  then it is sufficient to show:
  $$
  \len(s | s \in (x, s_1, \cdots, y) \land \qflag(s) = 1) \leq \kw{dfs(\sccgraph, x, visited)}.
  $$
  %
  Let  $\{v_1, \cdots, x\}$ be the set of all the distinct vertices of $k_{x,y}$'s vertices sequence $(x, s_1, \cdots, y)$, and 
  $(v_1, \cdots, x)$ a subsequence containing all the vertices in $\{x, v_1, \cdots, y\}$.
  \\
  By the property of strong connected graph, as well as the definition of walk,
  we have a path $p_{x,y} $ from $x$ to $y$ with this vertices sequence: $(x, v_1, \cdots, y)$.
  \\
  By the property of strong connected graph, we also have a path start from $y$ and go back to $x$.
  \\
  Let $p_{y, x}$ be this path with vertices sequence $(y, v_1', \cdots, x)$.
  \\
  Then we have a walk $k_{x,x}$ with vertices sequence $(x, s_1, \cdots, y) + (v_1', \cdots, x)$.
  \\
  Since $\qlen(k_{x,y}) \leq \qlen(k_{x,x})$,
  it is enough to show 
  $$
  \qlen(k_{x,x}) = \len(s | s \in (x, s_1, \cdots, y, v_1', \cdots, x) \land \qflag(s) = 1) \leq \pathsearch_{scc}(\sccgraph)
  $$
  %
  By $\kw{dfs(\sccgraph, x, visited)}$ algorithm defined inside $\pathsearch_{scc}(\sccgraph)$, 
  in the line 13, in the condition where the path go back to $x$,
  \\
  we know $\pathsearch_{scc}(\sccgraph) = r[x]$ and
  $r[x] = \max\{\kw{flowcapacity}(p) \times \kw{querynum}(p) | p \in \paths(\sccgraph)\}$.
  \\
  Then it is sufficient to show: 
  % $\len(s | s \in (x, s_1, \cdots, y, v_1', \cdots, x) \land \qflag(s) = 1) \leq r[x]$.
  % \\
  % \\
  % Then we know 
  % \\
  $$ 
  \len(s | s \in (x, s_1, \cdots, y, v_1', \cdots, x) \land \qflag(s) = 1) \leq \kw{flowcapacity}(p_{x, y} + p_{y, x}) \times \kw{querynum}(p_{x, y} + p_{y, x}) 
  $$
  %
  , where $(p_{x, y} + p_{y, x})$ is the path $p_{x, y}$ concatenated by path $p_{y, x}$ and we know $(p_{x, y} + p_{y, x}) \in \paths(\sccgraph)$.
  \\
Since $\kw{flowcapacity}(p_{x, y} + p_{y, x})$ is the maximum visiting times for every $v \in (x, v_1, \cdots, y, v_1', \cdots, x)$, 
\\
we know in the vertices sequence of walk $k_{x,x}$, 
$\visit(s) (x, s_1, \cdots, y, v_1', \cdots, x)  \leq \kw{flowcapacity}(p_{x, y} + p_{y, x})$
  \\
  Also by the algorithm, $\kw{querynum}(p_{x, y} + p_{y, x})$ is the number of vertices with $\qflag$ equal to $1$,
  \\
  Then we know 
  \\
  $\len(s | s \in (x, s_1, \cdots, y, v_1', \cdots, x) \land \qflag(s) = 1) \leq \kw{flowcapacity}(p_{x, y} + p_{y, x}) \times \kw{querynum}(p_{x, y} + p_{y, x}) $
  \\
  This case is proved.
  %
%
%
\end{proof}
% % \paragraph{Variable Collection Algorithm, $\varCol$}
% % % The $\varCol$ algorithm shows how the labelled variables $\lvar$ are collected 
% % % (via the command ${\assign{x}{\expr}}$ or ${\assign{x}{\query(\qexpr)}}$) from the program ${c}$ in the first step.
% % % The algorithmic rules for $\varCol$ algorithm is defined in Figure~\ref{fig:var_col}. 
% % % It has the form: $\ag{\lvar; w; {c}}{ \lvar'; w'} $. 
% % % The input of $\varCol$ is the labelled variables $\lvar$ collected before the program ${c}$, a while map $w$ consistent with previous estimation, a program ${c}$. 
% % % The output of the algorithm is the updated labelled variables $\lvar'$, along with the updated while map $w$ for next steps' collecting.   
% % The $\varCol$ algorithm shows how the labelled variables $\lvar$ are collected 
% % (via the command ${\assign{x}{\expr}}$ or ${\assign{x}{\query(\qexpr)}}$) from the program ${c}$ in the first step, 
% % along with constructing the flag for each variable, i.e., $\flag$.
% % The algorithmic rules for $\varCol$ algorithm is defined in Figure~\ref{fig:var_col}. 
% % It has the form: 
% % {$\ag{\lvar; \flag; {c}}{ \lvar'; \flag'} $}. 
% % The input of $\varCol$ is a program ${c}$, 
% % the labelled variables $\lvar$ collected before the program ${c}$ 
% % as well as the flags $\flag$ for every corresponding variable .
% % The output of the algorithm is the updated labelled variables $\lvar'$ and flags $\flag'$ thorough the program ${c}$
% % %
% % % We have the algorithmic rules for $\varCol$ algorithm of the form: $\ag{\lvar; w; {c}}{\lvar';w'} $ as in Figure \ref{fig:var_col}. 
% % %
% % \begin{figure}
% % {
% % \begin{mathpar}
% % \inferrule
% % {
% % \empty
% % }
% % { \ag{\lvar ; \flag; {[\assign {x}{\expr}]^{l}}}
% % {\lvar ++ [{x}]; \flag++[0]}
% % }
% % ~\textbf{\varCol-asgn}
% % \and
% % \inferrule
% % {
% % }
% % { \ag{\lvar; \flag; [ \assign{{x}}{\query({\qexpr})}]^{l}}
% % {\lvar ++ [{x}]; \flag ++ [2]} 
% % }~\textbf{\varCol-query}
% % %
% % \and 
% % %
% % \inferrule
% % {
% % \ag{\lvar; [];  {c_1}}{\lvar_1; \flag_1}
% % \and 
% % \ag{\lvar_1; []; {c_2}}{ \lvar_2; \flag_2}
% % \and
% % \lvar_3 = \lvar_2 ++ \lvar'
% % \and
% % \flag_3 = \flag ++ ((\flag_1 ++ \flag_2) \uplus 1)
% % }
% % {
% % \ag{\lvar; \flag;
% % [\eif({\bexpr}, { c_1, c_2)}]^{l} }
% % {\lvar_3; \flag_3}
% % }~\textbf{\varCol-if}
% % %
% % %
% % %
% % \and 
% % %
% % \inferrule
% % {
% % \ag{\lvar; \flag {c_1}}{\lvar_1; \flag_1}
% % \and 
% % \ag{\lvar_1; \flag_1 ; {c_2}}{\lvar_2; \flag_2}
% % }
% % {
% % \ag{\lvar; \flag;
% % {(c_1 ; c_2)}}{\lvar_2 ; \flag_2}
% % }
% % ~\textbf{\varCol-seq}
% % \and 
% % %
% % %
% % {
% % \inferrule
% % {
% % { \ag{\lvar; [] ; {c}}
% % {\lvar'; \flag' }  }
% % \\
% % \lvar'' = \lvar'
% % \and 
% % \flag'' = \flag ++ (\flag' \uplus 1)
% % }
% % {
% % \ag{\lvar; \flag;  
% % \ewhile [{b}]^{l}
% % \edo  {c} }{\lvar''; \flag''}
% % }
% % ~\textbf{\varCol-while}
% % }
% % \end{mathpar}
% % }
% % \caption{The Algorithmic Rules of $\varCol$ }
% % \label{fig:var_col}
% % \end{figure}
% % %
% % %
% % The assignment commands are the source of variables $\varCol$ collecting, 
% % in the case $\textbf{\varCol-asgn}$ and $\textbf{\varCol-query}$, 
% % the output labelled variables are extended by ${x}$. 
% % \\
% % \todo{
% % When it comes to the $\eif \ldots \ethen \ldots \eelse$ command in the rule $\textbf{\varCol-if}$, variables assigned in the then branch ${c_1}$, as well as the variables assigned in the else branch ${c_2}$, and the new generated variables $\bar{{x}},\bar{{y}},\bar{{z}}$ in $ [ \bar{{x}}, \bar{{x_1}}, \bar{{x_2}}] ,[ \bar{{y}}, \bar{{y_1}}, \bar{{y_2}}],[ \bar{{z}}, \bar{{z_1}}, \bar{{z_2}}]$.
% % \\ 
% % The sequence command ${c_1;c_2}$ is standard by accumulating the predicted variables in the two commands ${c_1}$ and ${c_2}$ preserving their order. 
% % \\
% % The while command $\ewhile {\bexpr}, [{\bar{x}}] \ldots \edo {c}$ considers the newly generated variables by SSA transformation ${\bar{x}}$
% % as well and the newly labelled variables in its body ${c}$.
% % \\
% % %
% % Below we present the definition for a valid index, to have a clear understanding on the variable collecting algorithm:
% % }
% % %
% % %
% % \todo{
% % \begin{defn}[Valid Index (Remove?)]
% % Given an assigned variable list $\lvar$, $\lvar; \vDash ({c},i_1,i_2)$ iff 
% % $\lvar' = \lvar[0,\ldots, i_1-1], \lvar';{c} \to \lvar'' \land \lvar'' = \lvar[0, \ldots, i_2-1] $.  
% % \end{defn}}
% % %
% % %
% \todo{Data Dependency Analysis Algorithm Needed: (Possibly modify based on existing one, or a different one) get the more precise dependency information. 
% i.e., instead of dependency on all the over-approximated variables, 
% but dependency on only the variables assumed to be live.
% }
% \paragraph{Data Dependency Analysis Algorithm}
% %
% In this data flow matrix generating algorithm, we analyze the data flow information among all labelled variables $\lvar$ collected via the the $\varCol$ algorithm of length $N$.
% %
% We track the data flow relations between all these labelled variables. These informations are stored in a matrix $\Mtrix$, whose size is $N \times N$. 
% % We also track whether arbitrary variable is assigned with a query result in a vector $\flag$ with size $|\lvar|$. 
% %
% The algorithm to fill in the matrix is of the form: 
% {$\ad{\Gamma ; {c} ; \lvar}{\Mtrix}$}
% $\ad{\Gamma ; {c} ; i_1, i_2}{\Mtrix; \flag}$. 
% $\Gamma$ is a vector records the variables the current program ${c}$ depends on, the index $i_1$ is a pointer which refers to the position of the first new-generated variable in ${c}$ in the labelled variables $\lvar$, and $i_2$ points to the first new variable that is not in ${c}$ (if exists). 
% % %
% % %
% % {
% % \begin{defn}[Valid Gamma (Remove?)]
% % $\Gamma \vDash i_1$ iff $\forall i \geq i_1, \Gamma(i_1)=0 $.  
% % \end{defn}
% % }
% %%
% %
% % \framebox{$ {\Gamma} \vdash^{i_1, i_2}_{\Mtrix, \flag} ~ c $}
% % \begin{mathpar}
% % \inferrule
% % {\Mtrix = \lMtrix_i * ( \rMtrix_{{\expr},i} + \Gamma )
% % }
% % {
% %  \ad{\Gamma;[\assign {{x}}{{\expr}} ]^{l}; i }{\Mtrix; \flag_{0}; i+1 }
% % }
% % ~\textbf{\graphGen-asgn}
% % \and
% % {
% % \inferrule
% % {\Mtrix = \lMtrix_i * ( \rMtrix_{{\expr},i} + \Gamma )
% % \\
% % \flag = \lMtrix_i \and \flag(i) = 1
% % }
% % { 
% % \ad{\Gamma;[ \assign{{x}}{\query({\expr})} ]^{l} ; i }
% % {\Mtrix;\flag;i+1}
% % }~\textbf{\graphGen-query}}
% % %
% % \and 
% % %
% % {
% % \inferrule
% % {
% % {\ad{\Gamma + \rMtrix_{{\bexpr}, i_1}; {c_1} ; i_1 }{ \Mtrix_1;\flag_1;i_2 }}
% % \and 
% % {\ad{\Gamma + \rMtrix_{{\bexpr}, i_1};{c_2} ; i_2 }{ \Mtrix_2; \flag_2 ;i_3}}
% % \\
% % {\ad{\Gamma; [ \bar{{x}}, \bar{{x_1}}, \bar{{x_2}}]; i_3 }{ M_x; \flag_{\emptyset}; i_3+|\bar{{x}}| }}
% % %
% % \\
% % %
% % {\ad{\Gamma; [ \bar{{y}}, \bar{{y_1}}, \bar{{y_2}}]; i_3+|\bar{{x}}| }{ \Mtrix_y; \flag_{\emptyset}; i_3+|\bar{{x}}|+|\bar{{y}}| }}
% % %
% % \\
% % %
% % {\ad{\Gamma; [ \bar{{z}}, \bar{{z_1}}, \bar{{z_2}}]; i_3+|\bar{{x}}|+ |\bar{{y}}|}{ \Mtrix_y; \flag_{\emptyset}; i_3+|\bar{{x}}|+|\bar{{y}}| + |\bar{{z}}| }}
% % \\
% % {\Mtrix = (\Mtrix_1 + \Mtrix_2)+ \Mtrix_x+ \Mtrix_y + \Mtrix_z }
% % }
% % {
% % \ad{\Gamma ; \eif([{\bexpr}]^{l},[ \bar{{x}}, \bar{{x_1}},
% % \bar{{x_2}}] ,[ \bar{{y}}, \bar{{y_1}}, \bar{{y_2}}], 
% % [ \bar{{z}}, \bar{{z_1}}, \bar{{z_2}}],
% % { c_1, c_2)} ; i_1}{ \Mtrix ; \flag_1 \uplus \flag_2 \uplus 2  ; i_3+|\bar{x}|+|\bar{y}|+|\bar{z}| }
% % }
% % ~\textbf{\graphGen-if}
% % }
% % %
% % %
% % %
% % \and 
% % %
% % \inferrule
% % {
% % {\ad{\Gamma; {c_1} ; i_1 }{ \Mtrix_1 ; \flag_1; i_2 }  }
% % \and 
% % {
% % \ad{\Gamma;{c_2}; i_2}{ \Mtrix_2; \flag_2 ;i_3 }}
% % }
% % {
% % \ad{\Gamma ; ({c_1 ; c_2} ) ; i_1}{( \Mtrix_1 {;} \Mtrix_2) ; \flag_1 \uplus V_2 ; i_3  }
% % }
% % ~\textbf{\graphGen-seq}
% % %
% % \and 
% % %
% % \and 
% % %
% % { 
% % \inferrule
% % {
% % B= |{\bar{x}}| \and {A = |{c}|}
% % \\
% % {\ad{\Gamma;[\bar{{x}}, \bar{{x_1}}, \bar{{x_2}}]; i+ (B+A) }{ \Mtrix_{1};V_{1}; i+B+(B+A) }}
% % \\
% % {
% % \ad{\Gamma;{c} ; i+B+(B+A)  }{ \Mtrix_{2}; \flag_{2}; i+B+A+(B+A) }
% % }
% % \\
% % {
% % \ad{\Gamma ; [\bar{{x}}, \bar{{x_1}}, \bar{{x_2}}] ; i+(B+A) }{ \Mtrix; \flag ;i+(B+A)+B}
% % }
% % \\
% % { \Mtrix' = \Mtrix + ( \Mtrix_{1} + \Mtrix_{2}) }
% % \and
% % {
% % \flag' = \flag \uplus (( \flag_{1} \uplus \flag_{2}) \uplus 2)  }
% % }
% % {
% % \ad{\Gamma;
% % \ewhile ~ [ b ]^{l} ~ {n} ~
% % [\bar{{x}}, \bar{{x_1}}, \bar{{x_2}}] 
% % ~ \edo ~  c;
% % i }{ \Mtrix'; \flag' ;i+(B+A)+B }
% % }~\textbf{\graphGen-while}
% % }
% % \end{mathpar}
% {
% \framebox{$ \ad{\Gamma; c; \lvar_c}{\Mtrix}$}
% \begin{mathpar}
% \inferrule
% {
% {x}^l \in \lvar_c
% \and 
% \Mtrix = \lMtrix_i * ( \rMtrix_{{\expr}} + \Gamma )
% }
% {
% \ad{\Gamma; [\assign {{x}}{{\expr}} ]^{l}; \lvar_c}
% {\Mtrix}
% }
% ~\textbf{\graphGen-asgn}
% \and
% {
% \inferrule
% {
% {x}^l \in \lvar_c
% \and 
% \Mtrix = \lMtrix_i * ( \rMtrix_{{\expr}} + \Gamma )
% }
% { 
% \ad{\Gamma;[ \assign{{x}}{\query({\qexpr})} ]^{l} ; \lvar_c }
% {\Mtrix}
% }~\textbf{\graphGen-query}}
% %
% \and 
% %
% {
% \inferrule
% {
% {\ad{\Gamma + \rMtrix_{{\bexpr}}; {c_1} ; \lvar_c }{ \Mtrix_1}}
% \and 
% {\ad{\Gamma + \rMtrix_{{\bexpr}}; {c_2}; \lvar_c }{ \Mtrix_2}}
% \and
% {\Mtrix = (\Mtrix_1 + \Mtrix_2)}
% }
% {
% \ad{\Gamma ; \eif([{\bexpr}]^{l},{ c_1, c_2)}}
% { \Mtrix }
% }
% ~\textbf{\graphGen-if}
% }
% %
% %
% %
% \and 
% %
% \inferrule
% {
% {\ad{\Gamma; {c_1}; \lvar_c }{ \Mtrix_1}  }
% \and 
% {
% \ad{\Gamma;{c_2}; \lvar_c }{ \Mtrix_2}}
% }
% {
% \ad{\Gamma ; ({c_1 ; c_2} ); \lvar_c}
% {( \Mtrix_1 {;} \Mtrix_2) }
% }
% ~\textbf{\graphGen-seq}
% %
% \and 
% %
% \and 
% %
% { 
% \inferrule
% {
% {
% \ad{\Gamma + \rMtrix_{{\bexpr}};{c}; \lvar_c  }{ \Mtrix'}
% }
% }
% {
% \ad{\Gamma;
% \ewhile [ \sbexpr ]^{l} \edo  {c}; \lvar_c }{\Mtrix'}
% }~\textbf{\graphGen-while}
% }
% \end{mathpar}
% }
% %
% Below we define the valid data flow matrix, to have a clear understanding on the data flow generating algorithm:
% \begin{defn}[Valid Matrix]
% For a labelled variables $\lvar$, $\lvar \vDash (\Mtrix,\flag)$ iff the cardinality of $\lvar$ equals to the one of $\flag$, $|\lvar| = |\flag|$ 
% and the matrix $\Mtrix$ is of size $|\flag| \times |\flag|$.
% \end{defn}
% %
% \todo{Improvement if possible: Combining reachability bounds analysis into the static dependency analysis algorithm above, rather than adopting an external tool entirely.}
% %
% \paragraph{Reachability Bounds}
% Given a program $c$ with its labelled variables $\lvar$,
% we use the $\rb({x}, {c})$ algorithm, from paper \cite{10.1145/1806596.1806630}, to estimate the reachability bound for each variable ${x} \in \lvar$. 
% The input of $\rb$ is a program ${c}$ in SSA language and a variable ${x} $ from ${c}$.
% The output of $\rb({x}, {c})$ is an integer representing the reachability bound of ${x}$ in ${c}$.
% %

% %
% The following example programs ${c}2$ and ${c}3$ with while loop illustrate how the algorithm works.
% The collected labelled variables, $\lvar_{{c}2}$ and $\lvar_{{c}3}$,
% data flow matrix $\Mtrix_{{c}2}$ and  $\Mtrix_{{c}3}$
% and variable flags $\flag_{{c}2}$ and $\flag_{{c}3}$
% for program ${c}2$ and ${c}3$
% are presented in the right hand side.
%
% \[
% {{c}2 \triangleq
% \begin{array}{l}
% \left[{ x_1} \leftarrow \query(1)  \right]^1 ; 
% \\
% \left[{i_1} \leftarrow 0 \right]^2 ; 
% \\
% \ewhile
% ~ [{i_1} < 2]^3
% 	\\
% ~{[ x_3,x_1 ,x_2 ], [i_3, i_1, i_2] }
% ~ \edo 
% \\
% ~ \Big( 
% \left[{y}_1 \leftarrow \query(2) \right]^4;
% \\
% \left[{x_2 \leftarrow y_1  + x_3 } \right]^5;
% \\
% \left[{i_2 \leftarrow 1  + i_3 } \right]^6
% \Big) ; 
% \\
% \left[ {\assign{z_1}{x_3}} + 2  \right]^{7}
% \end{array}
% ,
% ~~~~
% \lvar_{{c}2} = \left [ \begin{matrix}
% {x}_1 \\
% {x}_3 \\
% {y}_1 \\
% {x}_2 \\
% {z}_1 \\
% {i}_1 \\
% {i}_2 \\
% {i}_3 
% \end{matrix} \right ]
% % \Mtrix =  \left[ \begin{matrix}
% %  & (x_1)  & (y_1) & (x_2) & (x_3) &  (z_1) & i_1 & i_2 & i_3\\
% % (x_1) & 0 & 0 & 0 & 0 & 0 & 0 & 0 & 0 \\
% % (y_1) & 0 & 0 & 0 & 0 & 0 & 1 & 1 & 1 \\
% % (x_2) & 0 & 1 & 0 & 1 & 0 & 1 & 1 & 1 \\
% % (x_3) & 1 & 0  & 1& 0 & 0 & 1 & 1 & 1 \\
% % (z_1) & 0 & 0 & 0 & 1 & 0 & 0 & 0 & 0 \\
% % (i_1) & 0 & 0 & 0 & 0 & 0 & 0 & 0 & 0 \\
% % (i_2) & 0 & 1 & 0 & 1 & 0 & 1 & 0 & 1 \\
% % (i_3) & 1 & 0  & 1& 0 & 0 & 1 & 1 & 1 \\
% % \end{matrix} \right]
% ,
% ~~~~~~
% \Mtrix_{{c}2} =  \left[ \begin{matrix}
% 0 & 0 & 0 & 0 & 0 & 0 & 0 & 0 \\
% 0 & 0 & 0 & 0 & 0 & 1 & 1 & 1 \\
% 0 & 1 & 0 & 1 & 0 & 1 & 1 & 1 \\
% 1 & 0  & 1& 0 & 0 & 1 & 1 & 1 \\
% 0 & 0 & 0 & 1 & 0 & 0 & 0 & 0 \\
% 0 & 0 & 0 & 0 & 0 & 0 & 0 & 0 \\
% 0 & 1 & 0 & 1 & 0 & 1 & 0 & 1 \\
% 1 & 0  & 1& 0 & 0 & 1 & 1 & 1 \\
% \end{matrix} \right]
% ,
% ~~~~
% \flag_{{c}2} = \left [ \begin{matrix}
% 1 \\
% 2 \\
% 1 \\
% 2 \\
% 0 \\
% 0 \\
% 2 \\
% 1 
% \end{matrix} \right ]
% }
% \]
% %
% %
% \[
% {{{c}3}  \triangleq
% \begin{array}{l}
% \left[{ x}_1 \leftarrow \query(1)  \right]^1 ;
% \\
% \left[{i_1} \leftarrow 1 \right]^2 ; 
% \\
% \ewhile ~ [i < 0]^{3} ,
% \\
% ~{[ x_3,x_1 ,x_2 ], [i_3, i_1, i_2] }
% ~ \edo
% \\
% ~ \Big( 
% \left[{ y_1} \leftarrow \query(2) \right]^3; \\
% \left[{x_2 \leftarrow y_1  + x_3 } \right]^5
% \Big) ; \\
% \left[ {\assign{z_1}{x_3}} + 2  \right]^{6}
% \end{array},
% ~~~~~~
% \lvar_{{c}3} = \left [ \begin{matrix}
% {x}_1 \\
% {i}_1 \\
% {x}_3 \\
% {i}_3 \\
% {z}_1 \\
% \end{matrix} \right ]
% ,~~~~~~
% \Mtrix_{{c}3}  =  \left[ \begin{matrix}
% 0 & 0 & 0 & 0 & 0 \\
% 0 & 0 & 0 & 0 & 0 \\
% 1 & 0 & 0 & 0 & 0 \\
% 0 & 1 & 0 & 0 & 0 \\
% 0 & 0 & 1 & 0 & 0 \\
% \end{matrix} \right]
% ,~~~~~~
% \flag_{{c}3} = \left [ \begin{matrix}
% 1 \\
% 0 \\
% 2 \\
% 2 \\
% 0 \\
% \end{matrix} \right ]
% }
% \]
% %
% We can now look at the if statement.
% \[ 
% %
% {c}4 \triangleq
% \begin{array}{l}
% 	\left[ {x}_1 \leftarrow \query(1) \right]^1; 
% 	\\
% 	\left[{y}_1 \leftarrow \query(2) \right]^2 ; 
% 	\\
% \eif \;( { x_1 + y_1 == 5} )^3,  \\
% {[ x_4,x_2,x_3 ],[] ,[y_3,y_1,y_2 ]} 
% \\
% \mathsf{then} ~ \left[ 
% {x}_2 \leftarrow \query(3) \right]^4 
% \\
% \mathsf{else} ~ \left[ 
% {x}_3 \leftarrow \query(4) \right]^5 ; 
% \\
% {y}_2 \leftarrow 2 ) \\
% \left[ { z_1 \leftarrow x_4 +y_3 }\right]^6
% \end{array},
% % \]
% % \[
% ~~~~~~
% \lvar_{{c}4} =  \left[ \begin{matrix}
% {x}_1 \\
% {y}_1 \\
% {x}_2 \\
% {x}_3 \\
% {y}_2 \\
% {x}_4 \\
% {y}_3 \\
% {z}_1 \\
% \end{matrix} \right], 
% ~~~~~ 
% \Mtrix_{{c}4} =  \left[ \begin{matrix}
% 0 & 0 & 0 & 0 & 0 & 0 & 0 & 0 \\
% 0 & 0 & 0 & 0 & 0 & 0 & 0 & 0 \\
% 0 & 0 & 0 & 0 & 0 & 0 & 0 & 0 \\
% 0 & 0 & 0 & 0 & 0 & 0 & 0 & 0 \\
% 0 & 0 & 0 & 0 & 0 & 0 & 0 & 0 \\
% 0 & 0 & 1 & 1 & 0 & 0 & 0 & 0 \\
% 0 & 1 & 0 & 0 & 1 & 0 & 0 & 0 \\
% 0 & 0 & 0 & 0 & 0 & 1 & 1 & 0 \\
% \end{matrix} \right], 
% ~~~~~ 
% \flag_{{c}4} = \left [ \begin{matrix}
% 1 \\
% 1 \\
% 1 \\
% 1 \\
% 0 \\
% 0 \\
% 0 \\
% 0 \\
% \end{matrix} \right ]
% \]
%
%
%
%


% By specifying the departure and destination vertices $s$ and $t$, the $\pathssearch(\progG, s, t)$ algorithm will 
% give the number of query vertices on a finite walk from $s$ to $t$, which contains the maximum number of query vertices.
% The pseudo-code of $\pathssearch(\progG, s, t)$ algorithm is defined in the Algorithm \ref{alg:adpt_alg}.
% %
% \begin{algorithm}
% \caption{
% {Walk Search Algorithm ($\pathssearch$)}
% \label{alg:adpt_alg}
% }
% \begin{algorithmic}
% \REQUIRE Weighted Directed Graph $G = (\vertxs, \edges, \weights, \flag)$ with a start vertex $s$ and destination vertex $t$ .
% \STATE  {\bf {bfs $(G, s, t)$}:}  
% \STATE \qquad {\bf init} 
% current node: $c = s$, 
% queue: $q = [c]$, 
% vector recoding if the vertex is visited: 
% visited$ = [0]*|\vertxs|$,
% result: $r$
% \STATE \qquad {\bf while} $q$ isn't empty:
% \STATE \qquad \qquad take the vertex from beginning $c= q.pop()$
% \STATE \qquad \qquad mark $c$ as visited, visited $[c] = 1$
% \STATE \qquad \qquad curr$\kw{flowcapacity}$ = min($\weights$(c), curr$\kw{flowcapacity}$).
% \STATE \qquad \qquad put all unvisited vertex $v$ having directed edge from c into $q$. 
% \STATE \qquad \qquad if $v$ is visited, then there is a circle in the graph, we update the result $r = r + $curr$\kw{flowcapacity}$
% \RETURN $r$
% \end{algorithmic}
% \end{algorithm}
%
%
% \subsection{\todo{Soundness of the \THESYSTEM}}

% {
% 	\begin{thm}[Soundness of the \THESYSTEM].
% 	Given a program ${c}$, we have:
% 	%
% 	\[
% 	\progA({c}) \geq A({c}).
% 	\]
% 	\end{thm}
% }
% {
% \begin{proof}
% Given a program ${c}$, 
% we construct its program-based graph $\progG({c}) = (\vertxs, \edges, \weights, \qflag)$
% by Definition~\ref{def:prog-based_graph}
% According to the Definition \ref{def:prog_adapt}, we have:
% %
% \[
% 	\progA({c}) 
% 	:= \max\left\{ \qlen(k)\ \mid \  k\in \walks(\progG({c}))\right \}.
% \]
% %
% According to the Definition \ref{def:trace-based_adapt}, we have the trace-based adaptivity as follows:
% $$
% A({c}) = \max \big 
% \{ \len(p) \mid {m} \in \mathcal{SM},D \in \dbdom ,p \in \paths(\traceG({c}, \text{D}, {m}) \big \} 
% $$
% %
% Then, we need to show:
% \[
% \max \big 
% \{ \len(p) \mid {m} \in \mathcal{SM},D \in \dbdom ,p \in \paths(\traceG({c}, \text{D}, {m}) \big \} 
% \leq
% \max\left\{ \qlen(k) \ \mid \  k\in \walks(\progG({c}))\right \}
% \]
% %
% It is sufficient to show that:
% \[
% 	\forall p, {m}, D, ~ s.t., ~ p \in \paths(\traceG({c}, \text{D}, {m}),
% 	\exists k \in \walks(\progG({c})) \land 
% 	\len(p) \leq \qlen(k)
% \]
% %
% Taking an arbitrary starting memory $m$ and an arbitrary underlying database $D$,
% we construct a trace-based graph $\traceG({c}, \text{D}, {m}) = (\vertxs, \edges)$ by the definition \ref{def:trace-based_graph}.
% %
% \\
% %
% Let $\midG({c},{m},\text{D}) = \{\midV, \midE, \midF\}$ be the intermediate graph by Definition~\ref{def:midgraph}.
% \\
% By Lemma~\ref{lem:bie_trace_to_mid}, we know:
% \[
% 	\forall p, {m}, D, ~ s.t., ~ p \in \paths(\traceG({c}, \text{D}, {m}),
% 	\exists p' \in \paths(\midG({c},{m},\text{D})) \land 
% 	\len(p) = \len_q(p')
% \]
% %
% Then it is sufficient to show that:
% %
% \[
% 	\forall p, {m}, D, ~ s.t., ~ p \in \paths(\midG({c}, \text{D}, {m}),
% 	\exists k \in \walks(\progG({c})) \land 
% 	\qlen(p) \leq \qlen(k)
% \]
% %
% We prove a stronger statement instead:
% \[
% 	\forall p, {m}, D, ~ s.t., ~ p \in \paths(\midG({c}, \text{D}, {m}),
% 	\exists k \in \walks(\progG({c})) \land 
% 	\qlen(p) = \qlen(k)	
% \]
% %
% %
% By Lemma~\ref{lem:sujv_mid_to_prog}, let $g$ be the surjective function $g: \progV \to \midV$ s.t.:
% %
% $$
% \forall \av \in \midV. ~ \progF(f(\av)) = \midF(\av) 
% \land |\kw{image}(f(\av))| \leq W(f(\av)).
% $$
% %
% %
% % \item(1) $\len(p_{\av_1 \to \av_2}) = \len(k_{f(\av_1) \to f(\av_2)})$
% % %
% % \item(2) $\forall \av \in p_{\av_1 \to \av_2}. ~ f(\av) \in k_{f(\av_1) \to f(\av_2)}$
% % %
% % \item(3) $\forall \av \in p_{\av_1 \to \av_2}. ~ 
% % \kw{image}(f(\av)) \cap {p_{\av_1 \to \av_2}}| = \# \{f(\av) \mid f(\av) \in k_{f(\av_1) \to f(\av_2)}\}$
% %
% Let ${m}$ and $D$ be an arbitrary memory and database $D$,
% taking an arbitrary path $p_{\av_1 \to \av_n} \in \paths(\midG({c}, \text{D}, {m})$ with:
% %
% \item Edge sequence: $(e, \ldots, e_{n-1})$
% %
% \item Vertices sequence: $(\av_1, \ldots, \av_n)$.
% \\
% By Lemma~\ref{lem:sujpathwalk_mid_to_prog}, let $h: \paths(\midG({c}, \text{D}, {m})) \to \walks(\progG({c}))$ be the surjective function satisfies:
% %
% \[
% 	\forall p_{\av_1 \to \av_n} \in \paths(\midG({c}, \text{D}, {m}))
% 	\text{ with }
% 	\left\{
% 	\begin{array}{ll}
% 	\mbox{edge sequence:} & (e, \ldots, e_{n-1})
% 	\\ 
% 	\mbox{vertices sequence:} & (\av_1, \ldots, \av_n)
% 	\end{array}
% 	\right.
% \]
% %
% \[
% 	\exists k_{f(\av_1) \to f(\av_n)} \in \walks(\progG({c}))
% 	\text{ with }
% 	\left\{
% 	\begin{array}{ll}
% 	\mbox{edge sequence:} & (g(e), \ldots, g(e_{n-1}) 
% 	\\ 
% 	\mbox{vertices sequence:} & (f(\av_1), \ldots, f(\av_{n}))
% 	\end{array}
% 	\right.
% \]
% %
% We have the walk:
% $k_{f(\av_1) \to f(\av_n)} \in \walks(\progG({c}))$ with:
% %
% \item Edges sequence: $(g(e), \ldots, g(e_{n-1}) $
% %
% \item Vertices sequence: $(f(\av_1), \ldots, f(\av_{n}))$.
% \\
% It is sufficient to show 
% %
% \[
% 	\qlen(p_{\av_1 \to \av_n}) = \qlen(k_{f(\av_1) \to f(\av_n)})
% \]
% %
% Unfold the definition of $\qlen$, it is suffice to show:
% \[
% \len \big( \av \mid \av \in (\av_1, \ldots, \av_n) \land \midF(\av) = 2 \big) 
% = \len \big(f(\av) \mid f(\av) \in (f(\av_1), \ldots, f(\av_{n})) \land \progF(f(\av)\big) = 2)	
% ~ (a)
% \]
% %
% By Lemma~\ref{lem:sujv_mid_to_prog}, we know:
% %
% \[
% 	\forall \av \in \midV. ~ \midF(\av) = \progF(f(\av)) ~(b)
% \]
% By rewriting $(b)$ in $(a)$, we have this case proved.
% %
% \\
% \todo{
% \begin{defn}[Intermediate Graph $\midG$].
% 	\label{def:midgraph}
% 	\\
% 	$\mathcal{AV}$ : Annotated Variables based on program execution
% 	\\
% 	Given a program ${c}$ with its labelled variables $\lvar$ of length $N$,
% 	a database $D$, a starting memory ${m}$,
% 	s.t., $\Gamma \vdash_{\Mtrix_c, \flag_c} {c}$,
% 	the intermediate graph 
% 	$\midG({c},{m},\text{D}) = (\vertxs, \edges, \flag)$ is defined as:%
% \[
% \begin{array}{rlcl}
% 	\text{Vertices} &
% 	\vertxs & := & \left\{ 
% 	\av \in \mathcal{AV} \middle\vert
% 	\exists {m'},  w', \qtrace, \vtrace.  ~ s.t., ~  
% 	\config{{m} ,{c}, [], [], []}  \to^{*}  \config{{m'} , \eskip, \qtrace, \vtrace, w' }
% 	\land \av \in \vtrace
% 	\right\}
% 	\\
% 	\text{Directed Edges} &
% 	\edges & := & 
% 	\left\{ 
% 	(\av, \av') \in \mathcal{AV} \times \mathcal{AV} 
% 	~ \middle\vert ~
% 	\flowsto(\av, \av', {c},{m},D) 
% 	\right\}
% 	\\
% 	\text{Flags} &
% 	\flag & := & 
% 	\big\{ (\av, n)  \in \vertxs \times \{0, 1, 2\} 
% 	\mid 
% 	(\pi_1(\av) = \lvar(i) \land n = \flag_c(i)); ~
% 	i = 1, \ldots, N
% 	\big\}
% \end{array}
% \]
% \end{defn}
% }
% %
% \\
% \todo{
% 	\begin{lem}[$\vardep$ is Transitive].
% 	\label{lem:vardep_trans}
% 	\\
% 	Given a program ${c}$, with a starting memory ${m}$ and a hidden database $D$, s.t., 
% 	$\config{{m}, {c}, [], [], []} \rightarrow^{*} \config{{m}', \eskip, \qtrace, \vtrace, w} $.
% 	Then, $\forall \av_1, \av_2, \av_3 \in \vtrace$:
% \[
% 	\Big(\vardep(\av_1, \av_2, {c}, {m}, D) \land 
% 	\vardep(\av_2, \av_3, {c}, {m}, D) \Big)
% 	\implies
% 	\vardep(\av_1, \av_3, {c}, {m}, D)
% \]
% 	\end{lem}
% 	\begin{subproof}[of Lemma~\ref{lem:vardep_trans}]
% 	Proof by unfolding and rewriting the Definition~\ref{def:var_dep}.
% 	\end{subproof}
% }
% \\
% %
% \todo{
% 	\begin{lem}[$\flowsto$ is Transitive ??].
% 	\label{lem:flowsto_trans}
% 	\\
% 	Given a program ${c}$ with its labelled variables $\lvar$ of length $N$. 
% 	Then $\forall x_1, x_2, x_3 \in \lvar$
% \[
% 	\Big(\flowsto(x_1, x_2) \land \flowsto(x_2, x_3) \Big)
% 	\implies
% 	\flowsto(x_1, x_3)
% \]
% 	\end{lem}
% 	\begin{subproof}[of Lemma~\ref{lem:flowsto_trans}]
% 	Proof by unfolding the Definition~\ref{def:flowsto}.
% 	\end{subproof}
% }
% \\
% %
% \todo{
% 	\begin{lem}[$\qdep$ Implies $\vardep$].
% 	\label{lem:querydep_vardep}
% 	\\
% 	Given a program ${c}$, with a starting memory ${m}$ and a hidden database $D$, s.t., 
% 	$\config{{m}, {c}, [], [], []} \rightarrow^{*} \config{{m}', \eskip, \qtrace, \vtrace, w} $.
% 	Then, $\forall \av_1, \av_2 \in \qtrace$
% \[
% 	\qdep(\av_1, \av_2, {c}, {m}, D) \implies 
% 	\vardep(\pi_2(\av_1), \pi_2(\av_2), {c}, {m}, D)
% \]
% 	\end{lem}
% 	\begin{subproof}[of Lemma~\ref{lem:querydep_vardep}]
% 	Proof by unfolding the Definition~\ref{def:var_dep} and Definition~\ref{def:query_dep}.
% 	\end{subproof}
% }
% \\
% %
% \todo{
% 	\begin{lem}[$\vardep$ Implies \flowsto].
% 	\label{lem:vardep_flows}
% 	\\
% 	Given a program ${c}$, with a starting memory ${m}$ and a hidden database $D$, s.t., 
% 	$\config{{m}, {c}, [], [], []} \rightarrow^{*} \config{{m}', \eskip, \qtrace, \vtrace, w} $.
% 	Then, $\forall \av_1, \av_2 \in \vtrace$
% \[
% 	\vardep(\av_1, \av_2, {c}, {m}, D) \implies 
% 	\flowsto(\pi_1(\av_1), \pi_1(\av_2))
% \]
% 	\end{lem}
% 	\begin{subproof}[of Lemma~\ref{lem:querydep_vardep}]
% 	Proof by showing contradiction based on the Definition~\ref{def:var_dep} and Definition~\ref{def:flowsto}.
% 	Let $\av_1, \av_2 \in \vtrace$ be 2 arbitrary annotated variables in the variable trace $\vtrace$,
% 	s.t., $\vardep(\av_1, \av_2, {c}, {m}, D)$.
% 	\\
% 	Unfolding the $\vardep$ definition, we have:	
% 	\end{subproof}
% }
% \\
% %
% \todo{
% 	\begin{lem}[Injective Mapping of vertices from $\traceG$ to $\midG$].
% 	\label{lem:injv_trace_to_mid}
% 	\\
% 	$\traceG({c}) = \{\traceV, \traceE\}$
% 	\\
% 	$\midG({c},{m},\text{D}) = \{\midV, \midE, \midF\}$
% \[
% 	\exists ~ \kw{injective} ~ f: \mathcal{AQ} \to \mathcal{AV}. 
% 	~ \forall \av \in \traceV. ~ 
% 	f(\av) \in \midV \land \midF(f(\av)) = 2
% \]
% 	\end{lem}
% \begin{subproof}
% Proving by Definition~\ref{def:midgraph} and Definition~\ref{def:prog_adapt}.
% \end{subproof}
% }
% \\
% \todo{
% 	\begin{lem}[One-on-One Mapping from $\edges$ of $\traceG$ to $\paths(\midG)$].
% 	\label{lem:bie_trace_to_mid}
% 	\\
% 	$\traceG({c}) = \{\traceV, \traceE\}$
% 	\\
% 	$\midG({c},{m},\text{D}) = \{\midV, \midE, \midF\}$
% 	\\
% 	An injective function $ f: \traceV \to \midV$ s.t.,
% 	$\forall \av \in \traceV. ~ \midF(f(\av)) = 2$ 
% \[
% 	\forall e = (\av_1, \av_2) \in \traceE. ~ 
% 	\exists p_{f(\av_1) \to f(\av_2)} \in \paths(\midG({c}, \text{D}, {m}))
% \]
% 	\end{lem}
% \begin{subproof}
% Proving by Lemma~\ref{lem:injv_trace_to_mid} and Definition~\ref{def:midgraph} and acyclic property of $\traceG$ and $\midG$.
% \end{subproof}
% }
% \\
% \todo{
% 	\begin{lem}[Surjective Mapping of Vertices from $\midG$ to $\progG$].
% 	\label{lem:sujv_mid_to_prog}
% 	\\
% 	$\midG({c},{m},\text{D}) = \{\midV, \midE, \midF\}$
% 	\\
% 	$\progG({c}) = \{\progV, \progE, \progF, \progW\}$
% 	\\
% 	$\exists ~ \kw{surjective} ~ f: \mathcal{AV} \to \mathcal{SVAR}.$
% 	%
% \[
% 	\forall \av \in \midV. ~ 
% 	f(\av) \in \progV \land \progF(f(\av)) = \midF(\av) \land
% 	|\kw{image}(f(\av))| \leq W(f(\av))
% \]
% \end{lem}
% \begin{subproof}
% Proving by Definition~\ref{def:midgraph}.
% \end{subproof}
% }
% \\
% \todo{
% 	\begin{lem}[Surjective Mapping from $\edges$ of $\midG)$ to $\edges$ of $\progG$].
% 	\label{lem:suje_mid_to_prog}
% 	\\
% 	$\midG({c},{m},\text{D}) = \{\midV, \midE, \midF\}$
% 	\\
% 	$\progG({c}) = \{\progV, \progE, \progF, \progW\}$
% 	\\
% 	A surjective function $f: \progV \to \midV$ s.t.,
% 	$\forall \av \in \midV. ~ \progF(f(\av)) = \midF(\av) \land |\kw{image}(f(\av))| \leq W(f(\av))$
% 	%
% \[
% 	\exists ~ \kw{surjective} ~ g: \midE \to \progE. ~
% 	\forall e_{mid} = (\av_1, \av_2) \in \midE. 
% 	\exists e_{prog} = ({f(\av_1), f(\av_2)}) \in \progE
% \]
% \end{lem}
% \begin{subproof}
% Proving by Lemma~\ref{lem:sujv_mid_to_prog}.
% \end{subproof}
% }
% \\
% \todo{
% 	\begin{lem}[Surjective Mapping from $\paths(\midG)$ to $\walks(\progG)$].
% 	\label{lem:sujpathwalk_mid_to_prog}
% 	\\
% 	$\midG({c},{m},\text{D}) = \{\midV, \midE, \midF\}$
% 	\\
% 	$\progG({c}) = \{\progV, \progE, \progF, \progW\}$
% 	\\
% 	A surjective function $f: \progV \to \midV$ s.t.,
% 	$\forall \av \in \midV. ~ \progF(f(\av)) = \midF(\av) \land |\kw{image}(f(\av))| \leq W(f(\av))$
% 	\\
% 	A surjective function $g: \midE \to \progE$ s.t.,
% 	$\forall e_{mid} = (\av_1, \av_2) \in \midE. 
% 	\exists e_{prog} = ({f(\av_1) \to f(\av_2)}) \in \progE$
% 	\\
% 	$\exists ~ \kw{surjective} ~ h: \paths(\midG({c},{m},\text{D})) \to \walks(\progG({c}))$ s.t.:
% 	%
% \[
% 	\forall p_{\av_1 \to \av_2} \in \paths(\midG({c},{m},\text{D}))
% 	\text{ with }
% 	\left\{
% 	\begin{array}{ll}
% 	\mbox{edge sequence:} & (e, \ldots, e_{n-1})
% 	\\ 
% 	\mbox{vertices sequence:} & (\av_1, \ldots, \av_n)
% 	\end{array}
% 	\right.
% \]
% \[
% 	\exists k_{f(\av_1) \to f(\av_2)} \in \walks(\progG({c}))
% 	\text{ with }
% 	\left\{
% 	\begin{array}{ll}
% 	\mbox{edge sequence:} & (g(e), \ldots, g(e_{n-1}) 
% 	\\ 
% 	\mbox{vertices sequence:} & (f(\av_1), \ldots, f(\av_{n}))
% 	\end{array}
% 	\right.
% \]
% % \item $(e, \ldots, e_{n-1})$, $(\av_1, \ldots, \av_n)$ are the edges sequence and vertices sequence of $p_{\av_1 \to \av_2}$.
% % then, 
% %  $\len(p_{\av_1 \to \av_2}) = \len(k_{f(\av_1) \to f(\av_2)})$
% % %
% % \item $\forall \av \in p_{\av_1 \to \av_2}. ~ f(\av) \in k_{f(\av_1) \to f(\av_2)}$
% % %
% % \item $\forall \av \in p_{\av_1 \to \av_2}. ~ 
% % \kw{image}(f(\av)) \cap {p_{\av_1 \to \av_2}}| = \# \{f(\av) \mid f(\av) \in k_{f(\av_1) \to f(\av_2)}\}
% % $
% \end{lem}
% %
% \begin{subproof}
% Proving by induction on the length of $l = p_{\av_1 \to \av_2} \in \paths(\midG({c},{m},\text{D}))$, and Lemma~\ref{lem:suje_mid_to_prog} and Lemma~\ref{lem:sujv_mid_to_prog}.
% \caseL{ $l = 1$: }
% \caseL{ $l = l' + 1$, $l' \geq 1$: }
% \end{subproof}
% }
% \end{proof}
% %

% %
% }}
%%%%%%%%%%%%%%%%%%%%%%%%%%%%%%%%%%%%%%%%%%% Evaluation %%%%%%%%%%%%%%%%%%%%%%%%%%%%%%%%%%%%%%%%%%% 
\section{Path-sensitivity}
\label{sec:examples}
%
\label{ex:multipleRounds}
\label{ex:multiRoundsS}
We illustrate here how our analysis work on two different examples.
%\begin{example}[Multiple Rounds Algorithm]
%\label{ex:multipleRounds}
%
%%%%%%%%%%%%%%%%%%%%%%%%%%%%%%%%%% Previous Version For Reference %%%%%%%%%%%%%%%%%%%%%%%%%%%%%%%%%%
% We look at an advanced adaptive data analysis algorithm - $\kw{multipleRounds}$ algorithm in Fig.~\ref{fig:multipleRounds}(a).
% This is a simplified version of the \emph{Monitor Augment} from \cite{RogersRSSTW20} with complete program in Apdix.
% It takes the user input $k$ which decides the 
% number of iterations.
% It starts from an initialized empty tracking list $I$,
% goes $k$ rounds and at every round, tracking list $I$ is updated by a query result of $\query(\chi[I])$.
% After $r$ rounds, the algorithm returns the columns of the hidden database $D$ not specified in the tracking list $I$.
% The functions $\kw{updnscore}(p,a)$,
% $\kw{updcscore}(p,a)$, $\kw{update}(I,ns,cs)$ simplify the computations of updating $ns$, $cs$ and $I$.%
Our first example, Algorithm $\kw{multipleRounds}$ in Fig.~\ref{fig:multipleRounds}(a), is a simplified form of the \emph{monitor argument} by \citet{RogersRSSTW20}.
The input $k$ is the number of iterations.
It uses a list $I$ to track queries. Specifically at each iteration it updates $I$ by using the result of a query which relies on $I$:  $\query(\chi[I])$.
After $k$ iterations, the algorithm returns the columns of the hidden database $D$ which are not contained in the  tracking list $I$.
The functions $\kw{updnscore}(p,a)$,
$\kw{updcscore}(p,a)$, $\kw{update}(I,ns,cs)$ simplify the computations of updating $ns$, $cs$ and $I$. They depends on the result of the query but they do not perform queries themselves%
%
%
%%%%%%%%%%%%%%%%%%%%%%%%%%%%%%%%%% Previous Version For Reference %%%%%%%%%%%%%%%%%%%%%%%%%%%%%%%%%%
% {Different from $\kw{twoRounds(k)}$ in Fig.~\ref{fig:overview-example},
% the query request, in each loop iteration is not independent. 
% The query in the $j^{th}$ iteration now depends on the tracking list $I$ from the previous  $(j - 1)^{th}$ iteration, 
% $I$ is updated by all the query results in the previous $j-1$ iterations. 
% In this sense, all these $k$ queries are adaptively chosen according to our discussion in overview.
% }
Different from the code in the example $\kw{twoRounds(k)}$,
the query request, $\clabel{\assign{a}{\query(I)}}^6$, in each loop iteration
depends on the tracking list $I$, which in turn depends on  all the querieas from previous iterations. 
%In particular, $I$ depends updated by all the query results in the previous iterations as well. 
In this sense, all these $k$ queries are fully adaptively chosen, and so the adaptivity is $k$.
%%%%%%%%%%%%%%%%%%%%%%%%%%%%%%%%%% Previous Version For Reference %%%%%%%%%%%%%%%%%%%%%%%%%%%%%%%%%%
% The program-based dependency graph is presented  in Fig.~\ref{fig:multi_graphs}(b). 
% We omitted its execution-based dependency graph $\traceG(\kw{multipleRounds(k)})$ because they have the same graph topology and only differ in weights.
% For the vertices in Fig.~\ref{fig:multi_graphs}(b) that have the weight $k$,
% their weights in $\traceG(\kw{multipleRounds(k)})$ are 
% the function $f_k$. $f_k$ takes an initial trace as input and returns the value of $k$ from the initial trace. 
% And the vertices with weight $1$ have the constant function $f_1 : \tdom_0(\kw{twoRounds(k)}) \to \{1\}$ 
% in  $\traceG(\kw{multipleRounds(k)})$. 
% For simplicity, we abuse the same symbols, $f_k$ and $f_1$ for all the following examples in their execution-based dependency graph
% to denote the weight function of a vertex. $f_k$ and $f_1$ compute the same value as defined above with the input initial trace w.r.t. different examples.
The estimated dependency graph $\progG(\kw{multipleRounds(k)})$ is presented in Fig.~\ref{fig:multipleRounds}(b) and we omitted the semantics-based dependency graph $\traceG(\kw{multipleRounds(k)})$ because it has the same topology and only differ in weights.
%
%%%%%%%%%%%%%%%%%%%%%%%%%%%%%%%%%% Previous Version For Reference %%%%%%%%%%%%%%%%%%%%%%%%%%%%%%%%%%
% As the adaptivity definition in our formal adaptivity model in Def.~\ref{def:trace_adapt},
% there is a finite walk along the dashed arrows,
% $a^{6} \to I^9 \to ns^{7} \to  \cdots \to ns^7$ , 
% where every vertex is visited $f_k(\trace_0)$ times given input $\trace_0$.
% The vertex $a^{6}$ has query annotation $1$, and it is visited $f_k(\trace_0)$ times.
% In this sense, the adaptivity of this program is
% $f_k(\trace_0)$ given input $\trace_0$.
% Since $f_k(\trace_0)$ computes the value of input variable $k$ from $\trace_0$, we have
% \begin{equation}
%     \label{eq:adapt_multipleRounds}
%     \forall \trace_0 \in \tdom_0(\kw{multipleRounds(k)}) \st  A(\kw{multipleRounds(k)})(\trace_0) = \env(\trace_0) k
% \end{equation} 
% where $\env$ is the environment operator.
% \jl{
% By the adaptivity definition in Def.~\ref{def:trace_adapt},
% there is a finite walk along the dashed arrows,
% $a^{6} \to I^9 \to ns^{7} \to  \cdots \to ns^7$ , 
% where $a^{6}$, $I^9$ and $ns^{7}$ are visited $w_{a^{6}}(\trace_0)$,
% $w_{I^9}(\trace_0)$ and $w_{ns^{7}}(\trace_0)$
% times respectively with input $\trace_0$.
% The vertex $a^{6}$ has query annotation $1$, and it is visited $w_{a^{6}}(\trace_0)$ times.
% In this sense, the adaptivity of this program is
% $w_{a^{6}}(\trace_0)$ given input $\trace_0$, i.e., $A(\kw{multipleRounds(k)})(\trace_0) = w_{a^{6}}(\trace_0)$.
% Since $w_{a^{6}}(\trace_0)$
% counts the execution times of command $\clabel{\assign{a}{\query(I)}}^6;$,
% this count is at most the loop iteration numbers, i.e., $k$'s initial value, $\env(\trace_0) k$ from the initial trace $\trace_0$.
% }
{
Our program analysis {$\THESYSTEM$} provides a tight upper bound for this example using $\pathsearch(\kw{multipleRounds(k)})$.
It first finds a path on the graph $\progG(\kw{multipleRounds(k)})$
$a^{6}: {}^k_1 \to I^9:{}^k_0 \to ns^7:{}^k_0$ with three weighted vertices. 
Then $\pathsearch$ algorithm transforms this path into a walk $a^{6}: {}^k_1 \to I^9:{}^k_0 \to ns^7:{}^k_0 \to a^{6}: {}^k_1 \cdots$, where $a^6, I^9, ns^{7}$ are all visited $k$ times respectively. 
So $\progA(\kw{multipleRounds(k)}) = k$.
We know for any initial trace $\trace_0$, $\config{\trace_0, k} \earrow \env(\trace_0)k$, i.e., $A(\kw{multipleRounds(k)})(\trace_0) \leq \env(\trace_0)k$ for any $\trace_0$, 
and so what we have produced is a tight and sound bound.
}
% \end{example}
%
\begin{figure}
\centering
\begin{subfigure}[t]{0.3\textwidth}
    \small{
    $
\begin{array}{l}
\kw{multipleRounds(k)} \triangleq\\
    \clabel{\assign{j}{k}}^0;
    \clabel{\assign{I}{[]}}^1; \\
    \clabel{\assign{ns}{0}}^2; 
    \clabel{\assign{cs}{0}}^3; \\
    \ewhile ~ \clabel{j > 0}^{4} ~ \edo ~ \\
    \Big(
    \clabel{\assign{j}{j-1}}^{5} ;\\
    \clabel{\assign{a}{\query(I)}}^6; \\
    \clabel{\assign{ns}{\kw{updnscore}(ns, a)}}^7; \\
    \clabel{\assign{cs}{\kw{updcscore}(cs, a)}}^8; \\
    \clabel{\assign{I}{\kw{updI}(I, ns, cs)}}^9
    \Big) 
\end{array}
    $
    }
    \caption{}
\end{subfigure}
        \begin{subfigure}{.45\textwidth}
        \begin{centering}
        \begin{tikzpicture}[scale=\textwidth/15cm,samples=200]
    % Variables Initialization
     \draw[] (-7, 1) circle (0pt) node{{ $I^1: {}^1_{0}$}};
     \draw[] (-7, 7) circle (0pt) node{{$ns^2: {}^{1}_{0}$}};
     \draw[] (-7, 4) circle (0pt) node{{ $cs^3: {}^{1}_{0}$}};
     % Variables Inside the Loop
     \draw[] (0, 10) circle (0pt) node{{ $a^6: {}^{k}_{1}$}};
     \draw[] (0, 7) circle (0pt) node{{ $ns^7: {}^{k}_{0}$}};
     \draw[] (0, 4) circle (0pt) node{{ $cs^8: {}^{k}_{0}$}};
     \draw[] (0, 1) circle (0pt) node{{ $I^9: {}^{k}_{0}$}};
     % Counter Variables
     \draw[] (7, 9) circle (0pt) node {{$j^0: {}^{1}_{0}$}};
     \draw[] (7, 6) circle (0pt) node {{ $j^5: {}^{k}_{0}$}};
     %
     % Value Dependency Edges:
     \draw[  -latex,] (0, 1.5)  -- (0, 3.5) ;
     \draw[ ultra thick, -latex, densely dotted,] (0, 7.5)  -- (0, 9.5) ;
     \draw[  -Straight Barb] (1.4, 4) arc (120:-200:1);
     \draw[  -Straight Barb] (8.5, 6.5) arc (150:-150:1);
     \draw[  -Straight Barb] (1, 7.5) arc (220:-100:1);
     \draw[  -latex] (7, 6.5)  -- (7, 8.5) ;
     % Value Dependency Edges on Initial Values:
     \draw[  -latex,] (-1.5, 1)  -- (-5.5, 1) ;
     \draw[  -latex,] (-1.5, 4)  -- (-5.5, 4) ;
     \draw[  -latex,] (-1.5, 7)  -- (-5.5, 7) ;
     %
     \draw[ ultra thick, -latex, densely dotted,] (-1, 9.5)  to  [out=-130,in=130]  (-1, 1.5);
     \draw[ ultra thick, -latex, densely dotted,] (-0.8, 1.7)  to  [out=-230,in=230]  (-0.5, 6.5);
     % Value Dependency from cs8 -> a6
     \draw[  -latex, ] (-0.8, 4.0)  to  [out=-230,in=230]  (-0.5, 9.5);
     % Value Dependency from a6 -> I1
     \draw[  -latex,] (-1.2, 9.7)  -- (-5.5, 1);
     \draw[  -Straight Barb] (1.7, 1.5) arc (120:-200:1);
     % Control Dependency
     \draw[  -latex] (1.5, 7)  -- (5.8, 6) ;
     \draw[  -latex] (1.5, 4)  -- (5.8, 6) ;
     \draw[  -latex] (1.5, 1)  -- (5.8, 6) ;
     \draw[  -latex] (1.5, 10)  -- (5.8, 6) ;
     % Edges Produced by Transitivity by Control Dependency
     \draw[  -latex] (1.5, 7)  -- (5.8, 9) ;
     \draw[  -latex] (1.5, 4)  -- (5.8, 9) ;
     \draw[  -latex] (1.5, 1)  -- (5.8, 9) ;
     \draw[  -latex] (1.5, 10)  -- (5.8, 9) ;
     % Edges Produced by Transitivity from vertext a6 Dependency
     \draw[  -latex,] (-1.2, 9.7)  -- (-5.5, 4);
     \draw[  -latex,] (-1.2, 9.7)  -- (-5.5, 7);
     \draw[ -latex] (-1, 9.5)  to  [out=-130,in=130]  (-1, 7.5);
     \draw[ -latex] (0.5, 9.5)  to  [out=-50,in=50]  (0.5, 4);
     \draw[  -Straight Barb] (0.5, 10.5) arc (150:-150:1);
     % Edges Produced by Transitivity from vertext cs8 Dependency
     \draw[  -latex,] (-1.2, 4)  -- (-5.5, 1);
     \draw[  -latex,] (-1.2, 4)  -- (-5.5, 7);
     \draw[  -latex,] (0, 4.5)  -- (0, 6.5) ;
     % Edges Produced by Transitivity from vertext I9 Dependency
     \draw[  -latex,] (-1.2, 1)  -- (-5.5, 4);
     \draw[  -latex,] (-1.2, 1)  -- (-5.5, 7);
     \draw[  -latex,] (0.5, 1.0)  to  [out=50,in=-50]  (0.5, 9.5);
     % Edges Produced by Transitivity from vertext ns7 Dependency
     \draw[  -latex,] (-1.2, 7)  -- (-5.5, 1);
     \draw[  -latex,] (-1.2, 7)  -- (-5.5, 4);
     \draw[ -latex] (0.5, 6.5)  to  [out=-50,in=50]  (0.5, 1.5);
     \draw[ -latex] (0.5, 6.5)  to  [out=-50,in=50]  (0.5, 4.5);
     \end{tikzpicture}
     \caption{}
        \end{centering}
        \end{subfigure}
    \vspace{-0.4cm}
    \caption{(a) The simplified multiple rounds example (b) The estimated dependency graph from $\THESYSTEM$}
    \vspace{-0.5cm}
    \label{fig:multipleRounds}
\end{figure}
%
% %
\begin{example}[Linear Regression Algorithm with Gradient Decent Optimization]
\label{ex:linearregression}
    The linear regression algorithm with gradient decent Optimization works well 
    in our $\THESYSTEM$ as well.
            %   \[
            %   %
            %   \begin{array}{l}
            %   \kw{linearRegression(step, rate)} \triangleq \\
            %          \clabel{ a \leftarrow 0}^{0} ; \\
            %          \clabel{ c \leftarrow 0}^{1} ; \\
            %           \clabel{\assign{j}{\kw{step}} }^{2} ; \\
            %         %   \clabel{\assign{d}{10000000} }^{2} ; \\
            %           \ewhile ~ \clabel{j > 0}^{3} ~ \edo ~ \\
            %           \Big(
            %               \clabel{\assign{da}{\query(-2 * (\chi[1] - (\chi[0]\times a + c)) \times (\chi[0]))} }^{4}  ; \\
            %               \clabel{\assign{dc}{\query(-2 * (\chi[1] - (\chi[0]\times a + c)))} }^{5}  ; \\
            %               \clabel{\assign{a}{a - \kw{rate} * da} }^{6}  ; \\
            %               \clabel{\assign{c}{c - \kw{rate} * dc} }^{7}  ; \\
            %            \clabel{\assign{j}{j-1}}^{8} 
            %         %   \clabel{a \leftarrow x :: a}^{6} 
            %           \Big);
            %       \end{array}
            %   \]
              %
              %
                   %
\begin{figure}
\centering
\begin{subfigure}{0.45\textwidth}
    \centering
    {\small
        \[
        \begin{array}{l}
            \kw{linearRegressionGD(k, rate)} \triangleq \\
                   \clabel{ a \leftarrow 0}^{0} ; 
                   \clabel{ c \leftarrow 0}^{1} ; 
                    \clabel{\assign{j}{\kw{k}} }^{2} ; \\
                  %   \clabel{\assign{d}{10000000} }^{2} ; \\
                    \ewhile ~ \clabel{j > 0}^{3} ~ \edo ~ \\
                    \Big(
                        \clabel{\assign{da}{\query(-2 * (\chi[1] - (\chi[0]\times a + c)) \times (\chi[0]))} }^{4}  ; \\
                        \clabel{\assign{dc}{\query(-2 * (\chi[1] - (\chi[0]\times a + c)))} }^{5}  ; \\
                        \clabel{\assign{a}{a - \kw{rate} * da} }^{6}  ; 
                        \clabel{\assign{c}{c - \kw{rate} * dc} }^{7}  ; \\
                     \clabel{\assign{j}{j-1}}^{8} 
                  %   \clabel{a \leftarrow x :: a}^{6} 
                    \Big);
                \end{array}
        \]
        }
     \caption{}
        \end{subfigure}
      \begin{subfigure}{.45\textwidth}
          \begin{centering}
          \begin{tikzpicture}[scale=\textwidth/20cm,samples=200]
    % Variables Initialization
    \draw[] (-6, 1) circle (0pt) node{{ $a^0: {}^1_{0}$}};
    \draw[] (-6, 4) circle (0pt) node{{ $c^1: {}^{1}_{0}$}};
    % Variables Inside the Loop
       \draw[] (0, 10) circle (0pt) node{{ $da^4: {}^{k}_{1}$}};
       \draw[] (0, 7) circle (0pt) node{{ $dc^5: {}^{k}_{0}$}};
       \draw[] (0, 4) circle (0pt) node{{ $a^6: {}^{k}_{0}$}};
       \draw[] (0, 1) circle (0pt) node{{ $c^7: {}^{k}_{0}$}};
       % Counter Variables
       \draw[] (7, 9) circle (0pt) node {{$j^0: {}^{1}_{0}$}};
       \draw[] (7, 6) circle (0pt) node {{ $j^8: {}^{k}_{0}$}};
       %
       % Value Dependency Edges:
       \draw[ thick, -latex,] (0, 1.5)  -- (0, 3.5) ;
       \draw[ thick, -Straight Barb] (1.8, 4.2) arc (220:-100:1);
       \draw[ thick, -Straight Barb] (7.5, 6.5) arc (150:-150:1);
       \draw[](10, 6) node[] {\highlight{$k$}} ;
       \draw[ thick, -Straight Barb] (1.7, 1.) arc (120:-200:1);
       \draw[](4, 0) node[] {\highlight{$k$}} ;
       \draw[ thick, -latex] (6, 6.5)  -- 
       node [right] {\highlight{$k$}}(6, 8.5) ;
       % Value Dependency Edges on Initial Values:
       \draw[ thick, -latex,] (-2, 1)  -- 
       node [above] {\highlight{$k$}}(-4.5, 1) ;
       \draw[ thick, -latex,] (-2, 4)  -- 
       node [above] {\highlight{$k$}}(-4.5, 4) ;
       %
       \draw[ ultra thick, -latex, densely dotted,] (-1, 1.5)  to  [out=-220,in=220]  
       node [below] {\highlight{$k$}}(-1, 6.5);
       \draw[ ultra thick, -latex, densely dotted,] (-1, 4.5)  to  [out=-220,in=220]  
       node [above] {\highlight{$k$}}(-1, 9.5);
       \draw[ ultra thick, -latex, densely dotted,]  (1, 6.2) to  [out=-60,in=60] 
       node [below] {\highlight{$k$}}(0.5, 1.5) ;
       \draw[ ultra thick, -latex, densely dotted,]  (1.2, 9.2)  to  [out=-50,in=50] 
       node [above] {\highlight{$k$}}(0.5, 4.5);
       % Control Dependency
      %  \draw[ thick,-latex] (1.5, 7)  -- (4, 9) ;
      %  \draw[ thick,-latex] (1.5, 4)  -- (4, 9) ;
       \draw[ thick,-latex] (1.8, 10)  -- 
       node [above] {\highlight{$k$}}(5.5, 6) ;
       \draw[ thick,-latex] (1.8, 7)  -- (5.5, 6) ;
       \draw[ thick,-latex] (1.8, 4)  -- 
       node [above] {\highlight{$k$}}(5.5, 6) ;
       \draw[ thick,-latex] (1.8, 1)  -- 
       node [below] {\highlight{$k$}}(5.5, 6) ;
       \end{tikzpicture}
       \caption{}
          \end{centering}
          \end{subfigure}
          %
        \begin{subfigure}{.8\textwidth}
            \begin{centering}
            \begin{tikzpicture}[scale=\textwidth/20cm,samples=200]
      % Variables Initialization
      \draw[] (-6, 1) circle (0pt) node{{ $a^0: {}^1_{0}$}};
      \draw[] (-6, 4) circle (0pt) node{{ $c^1: {}^{1}_{0}$}};
      % Variables Inside the Loop
         \draw[] (0, 10) circle (0pt) node{{ $da^4: {}^{k}_{1}$}};
         \draw[] (0, 7) circle (0pt) node{{ $dc^5: {}^{k}_{0}$}};
         \draw[] (0, 4) circle (0pt) node{{ $a^6: {}^{k}_{0}$}};
         \draw[] (0, 1) circle (0pt) node{{ $c^7: {}^{k}_{0}$}};
         % Counter Variables
         \draw[] (7, 9) circle (0pt) node {{$j^0: {}^{1}_{0}$}};
         \draw[] (7, 6) circle (0pt) node {{ $j^8: {}^{k}_{0}$}};
         %
         % Value Dependency Edges:
         \draw[ thick, -latex,] (0, 1.5)  -- (0, 3.5) ;
         \draw[ thick, -Straight Barb] (1.8, 4.2) arc (220:-100:1);
         \draw[ thick, -Straight Barb] (7.5, 6.5) arc (150:-150:1);
         \draw[](10, 6) node[] {\highlight{$\trace_0 \to \env(\trace_0) k $}} ;
         \draw[ thick, -Straight Barb] (1.7, 1.) arc (120:-200:1);
         \draw[](4, 0) node[] {\highlight{$\trace_0 \to \env(\trace_0) k $}} ;
         \draw[ thick, -latex] (6, 6.5)  -- 
         node [right] {\highlight{$\trace_0 \to \env(\trace_0) k $}}(6, 8.5) ;
         % Value Dependency Edges on Initial Values:
         \draw[ thick, -latex,] (-2, 1)  -- 
         node [above] {\highlight{$\trace_0 \to \env(\trace_0) k $}}(-4.5, 1) ;
         \draw[ thick, -latex,] (-2, 4)  -- 
         node [above] {\highlight{$\trace_0 \to \env(\trace_0) k $}}(-4.5, 4) ;
         %
         \draw[ ultra thick, -latex, densely dotted,] (-1, 1.5)  to  [out=-220,in=220]  
         node [below] {\highlight{$\trace_0 \to \env(\trace_0) k $}}(-1, 6.5);
         \draw[ ultra thick, -latex, densely dotted,] (-1, 4.5)  to  [out=-220,in=220]  
         node [above] {\highlight{$\trace_0 \to \env(\trace_0) k $}}(-1, 9.5);
         \draw[ ultra thick, -latex, densely dotted,]  (1, 6.2) to  [out=-60,in=60] 
         node [below] {\highlight{$\trace_0 \to \env(\trace_0) k $}}(0.5, 1.5) ;
         \draw[ ultra thick, -latex, densely dotted,]  (1.2, 9.2)  to  [out=-50,in=50] 
         node [above] {\highlight{$\trace_0 \to \env(\trace_0) k $}}(0.5, 4.5);
         % Control Dependency
        %  \draw[ thick,-latex] (1.5, 7)  -- (4, 9) ;
        %  \draw[ thick,-latex] (1.5, 4)  -- (4, 9) ;
         \draw[ thick,-latex] (1.8, 10)  -- 
         node [above] {\highlight{$\trace_0 \to \env(\trace_0) k $}}(5.5, 6) ;
         \draw[ thick,-latex] (1.8, 7)  -- (5.5, 6) ;
         \draw[ thick,-latex] (1.8, 4)  -- 
         node [above] {\highlight{$\trace_0 \to \env(\trace_0) k $}}(5.5, 6) ;
         \draw[ thick,-latex] (1.8, 1)  -- 
         node [below] {\highlight{$\trace_0 \to \env(\trace_0) k $}}(5.5, 6) ;
         \end{tikzpicture}
         \caption{}
            \end{centering}
            \end{subfigure}
    \vspace{-0.5cm}
    \caption{(a) The linear regression algorithm 
    (b) The program-based dependency graph from $\THESYSTEM$
    (c) The execution-based dependency graph.}
    \vspace{-0.5cm}
    \label{fig:linear_regression}
\end{figure}
%
Analysis Result: $ \progA(\kw{linearRegressionGD(k, rate)}) = k$
\end{example} 
%
 
This linear regression algorithm 
% in order to
aims to
model a linear relationship between a dependent variable $y$,
% corresponding to the observed value in the column $\chi[1]$ in database, 
and an independent variable $x$, $y = a \times x + c$, specifically approximating the 
model parameter $a$ and $c$.
In order to have a good approximation on the model parameter 
$a$ and $c$, 
% corresponding to the observed value in the column $\chi[0]$ in database, 
it sends query to a training data set adaptively in every iteration.
This training data set contains two columns (can extend to higher dimensional data sets), first column is used as the observed value for the independent variable $x$,
second column is used as the observed label value for the dependent variable $y$.
This algorithm is written in our {\tt Query While} language in Figure~\ref{fig:linear_regression}(a) as $\kw{linearRegressionGD(k, rate)}$.
% taking the iteration number $\kw{step}$ 

This linear regression algorithm starts from initializing the linear model parameters and the counter variable,
and then goes into the training iterations.
In each iteration, it computes the differential value w.r.t. parameter
$a$ and $c$ respectively,
through requesting two queries, $\query(-2 * (\chi[1] - (\chi[0]\times a + c)) \times (\chi[0]))$ and 
$\query(-2 * (\chi[1] - (\chi[0]\times a + c)))$
at line 4 and 5.
Then, it uses these two differential values stored in variable $da$ and $dc$ to update the linear model parameters $a$ and $c$.
%
Its the program-based dependency graph is shown in Figure~\ref{fig:linear_regression}(b). Its execution-based dependency graph share the same graph, only needs to change the weight, $k$ into $w_k$ and $1$ for $w_1$ as we do in the previous example.
% We omit the detail of how to 
% generate this graph, which is similar to the generation procedure in 
% Example~\ref{alg:multiRound}.
In the execution-based dependency graph, there are multiple walks having the same longest query length.
For example, the walk $c^7 \to dc^6 : \to c^7 \to \cdots \to dc^6$ along the 
dotted arrows, where each vertex is visited $w_k(\trace_0)$ times for an initial trace $\trace_0$.
% By counting the total occurrence time of vertices with annotation $1$ in this walk, we have this program's adaptivity $k$.
There is actually other walks having the same query length $k$, the 
walk $a^7 \to da^6  \to a^7 \to \cdots \to da^6 $ along the 
dotted arrows, where each vertex is visited $w_k(\trace_0)$ times.
% the dotted path corresponds to a finite walk with the longest query length and its adaptivity on this walk is $k$.
But it doesn't affect the adaptivity for this program, which is still the maximal query length $w_k(\trace_0)$ with respect to initial trace $\trace_0$.
Also, $\THESYSTEM$, estimates the adaptivity $k$ for this example. Similarly as the multiple round example, we can show it is a tight bound.
%
% \begin{example}
[Multiple Rounds Odds Algorithm]
\label{ex:overapproximate}
The $\THESYSTEM$ comes across an over-approximation due to its path-insensitive nature. 
It occurs when the control flow can be decided in a particular way in front of conditional branches,
while the static analysis fails to witness. 

As in Figure~\ref{fig:overappr_example}(a), $\kw{multipleRoundsOdd}(k)$
is an example program with $1 + k$ adaptivity rounds and two paths while loop.
% we call it a multiple rounds odd iteration algorithm. 
In each iteration, 
the query $\query(\chi[x])$ in command $5$ is based on previous query results stored in $x$, which is similar to Example~\ref{ex:multipleRounds}.
The difference is that, only the query answers from the even iterations ($i = 0, 2, \cdots $) are 
% used to $b$. 
used in the query 
in command $7$, $\query(\chi[\ln(y)])$.
  Because the execution trace only updates 
%   $b$ using the query answers at odd iterations, so the answers from even iterations do not affect the queries at odd iterations. From the query-based dependency graph in Figure~\ref{fig:overappr_example}(b), we can see that there is no edge from queries at odd iterations (such as $q_1,q_3,q_5$) to queries at even iteration(such as $q_2,q_4$). The longest path is dashed with a length $3$.  However, {\THESYSTEM} fails to realize that odd iteration will always execute then branch and even iteration means else branch, so its dependency graph considers both branches for every iteration. In this sense, the dependency graph by {\THESYSTEM} is similar to the one in the multiple rounds strategy. We show the estimated graph in Figure~\ref{fig:overappr_example}(c). The estimated upper bound is then, $5$, instead of $3$. 
$x$ using the query answers in even iterations, so the answers from odd iterations do not affect the queries in even iterations. 
From the execution-based dependency graph in Figure~\ref{fig:overappr_example}(b), 
we can see that the weight for the vertex $y^5$ is 
$f_{k/2}$.
$f_{k/2} : \trace_0 \to (\env(\trace_0) k) / 2$, computes return the value of $k/2$ from input initial $\trace_0$.  
However, {\THESYSTEM} fails to realize that all the odd iterations only execute the first branch
and only even iterations execute the second branch. 
% its dependency 
So it considers both branches for every iteration when estimating the adaptivity. 
In this sense, the weight estimated for $y^5$ and $p^6$ are both 
$k$ as in Figure~\ref{fig:overappr_example}(c).
As a result, {\THESYSTEM}  estimates the longest walk from Figure~\ref{fig:overappr_example}(c) as
$y^5  \to x^7  \to y^5  \to \cdots \to x^7  $ with each vertex being visited $k$ times.
And the computed adaptivity  
% estimated from the program-based dependency graph graph from by finding the walk with the longest query length 
is $1 + 2 * k$, instead of $1 + k$. 
% We omitted the estimated graph, which is identical to the graph in Figure~\ref{fig:overappr_example}(b). 
%
{ \small
\begin{figure}
\centering
    \begin{subfigure}{0.33\textwidth}
\centering
\small{
    \[
    %
    \begin{array}{l}
        \kw{multipleRoundsOdd}(k) \triangleq \\
        \clabel{ \assign{j}{k}}^{0} ; 
        \clabel{ \assign{x}{\query(\chi[0])} }^{1} ; \\
            \ewhile ~ \clabel{j > 0}^{2} ~ \edo ~ 
            \Big(
             \clabel{\assign{j}{j-1}}^{3} ;\\
             \eif(\clabel{j \% 2 == 0}^{4}, \\
             \clabel{\assign{y}{\chi[x]}}^{5}, 
             \clabel{\assign{p}{\chi[x]}}^{6});\\                            
             \clabel{\assign{x}{\query(\chi(\ln(y)))} }^{7} \Big)
        \end{array}
    \]
}
 \caption{}
    \end{subfigure}
%
\begin{subfigure}{.31\textwidth}
    \begin{centering}
    \begin{tikzpicture}[scale=\textwidth/11cm,samples=200]
% Variables Initialization
\draw[] (5, 1) circle (0pt) node{{ $x^1: {}^{f_1}_{1}$}};
% Variables Inside the Loop
 \draw[] (0, 7) circle (0pt) node{{ $y^5: {}^{f_k/2}_{1}$}};
 \draw[] (0, 4) circle (0pt) node{{ $p^6: {}^{f_k/2}_{1}$}};
 \draw[] (0, 1) circle (0pt) node{{ $x^7: {}^{f_k}_{1}$}};
 % Counter Variables
 \draw[] (5, 7) circle (0pt) node {{$j^0: {}^{f_1}_{0}$}};
 \draw[] (5, 4) circle (0pt) node {{ $j^3: {}^{f_k}_{0}$}};
 %
 % Value Dependency Edges:
 \draw[ thick, -latex,]  (0, 3.5) -- (0, 1.5) ;
%  \draw[ thick, -Straight Barb] (1, 4.2) arc (220:-100:1);
 \draw[ thick, -Straight Barb] (6.5, 4.5) arc (150:-150:1);
 \draw[ thick, -latex] (5, 4.5)  -- (5, 6.5) ;
%  \draw[ thick, -Straight Barb] (1., 1.5) arc (120:-200:1);
 % Value Dependency Edges on Initial Values:
 \draw[ thick, -latex,] (1.5, 1)  -- (4, 1) ;
 %
 \draw[ ultra thick, -latex, densely dotted,] (-0.6, 1.5)  to  [out=-220,in=220]  (-0.5, 6.5);
\draw[ ultra thick, -latex, densely dotted,]  (0.5, 6.5) to  [out=-30,in=30] (0.6, 1.6) ;
%  \draw[ ultra thick, -latex, densely dotted,]  (0.5, 10)  to  [out=-50,in=50] (0.5, 4);
 % Control Dependency
 \draw[ thick,-latex] (1.5, 7)  -- (4, 6) ;
 \draw[ thick,-latex] (1.5, 4)  -- (4, 6) ;
 \draw[ thick,-latex] (1.5, 1)  -- (4, 6) ;
%  \draw[ thick,-latex] (1.5, 10)  -- (4, 6) ;
 \end{tikzpicture}
 \caption{}
    \end{centering}
    \end{subfigure}
    \begin{subfigure}{.31\textwidth}
        \begin{centering}
        \begin{tikzpicture}[scale=\textwidth/11cm,samples=200]
    % Variables Initialization
    \draw[] (5, 1) circle (0pt) node{{ $x^1: {}^1_{1}$}};
    % Variables Inside the Loop
     \draw[] (0, 7) circle (0pt) node{{ $y^5: {}^{k}_{1}$}};
     \draw[] (0, 4) circle (0pt) node{{ ${p^6: {}^{k}_{1}}$}};
     \draw[] (0, 1) circle (0pt) node{{ ${x^7: {}^{k}_{1}}$}};
     % Counter Variables
     \draw[] (5, 7) circle (0pt) node {{$j^0: {}^{1}_{0}$}};
     \draw[] (5, 4) circle (0pt) node {{ $j^3: {}^{k}_{0}$}};
     %
% Value Dependency Edges:
 \draw[ thick, -latex,]  (0, 3.5) -- (0, 1.5) ;
%  \draw[ thick, -Straight Barb] (1, 4.2) arc (220:-100:1);
 \draw[ thick, -Straight Barb] (6.5, 4.5) arc (150:-150:1);
 \draw[ thick, -latex] (5, 4.5)  -- (5, 6.5) ;
%  \draw[ thick, -Straight Barb] (1., 1.5) arc (120:-200:1);
 % Value Dependency Edges on Initial Values:
 \draw[ thick, -latex,] (1.5, 1)  -- (4, 1) ;
 %
 \draw[ ultra thick, -latex, densely dotted,] (-0.6, 1.5)  to  [out=-220,in=220]  (-0.5, 6.5);
\draw[ ultra thick, -latex, densely dotted,]  (0.5, 6.5) to  [out=-30,in=30] (0.6, 1.6) ;
%  \draw[ ultra thick, -latex, densely dotted,]  (0.5, 10)  to  [out=-50,in=50] (0.5, 4);
 % Control Dependency
 \draw[ thick,-latex] (1.5, 7)  -- (4, 6) ;
 \draw[ thick,-latex] (1.5, 4)  -- (4, 6) ;
 \draw[ thick,-latex] (1.5, 1)  -- (4, 6) ;
%  \draw[ thick,-latex] (1.5, 10)  -- (4, 6) ;
     \end{tikzpicture}
     \caption{}
        \end{centering}
        \end{subfigure}
        \vspace{-0.4cm}
\caption{(a) The multiple rounds odd example 
(b) The execution-based dependency graph
(c) The program-based dependency graph graph from $\THESYSTEM$.}
    \label{fig:overappr_example}
    \vspace{-0.5cm}
\end{figure}
}
%
\end{example}
\begin{example}[Single Adaptivity Round Example]
    \label{ex:multiRoundsS}
    The program's adaptivity definition in our formal model,
    (in Definition~\ref{def:trace_adapt})
    comes across an over-approximation when capturing the program's intuitive adaptivity rounds.
    It results from the difference between its weight calculation and the \emph{variable may-dependency} definition.
    It occurs when the weight is computed over the traces different from the traces used in 
    witnessing the \emph{variable may-dependency} relation.
    
    The program $\kw{multiRoundsS(k)}$ in Figure~\ref{fig:multiRoundsS}(a) demonstrates this over-approximation.
    It is a variant of the multiple rounds strategy with input $k$.
    In each iteration, the query request $\clabel{\assign{p}{\query(\chi[y]+p)} }^{7}$ is based on value stored in $p$ and $y$ from previous iteration.
    Differ from Example~\ref{ex:multipleRounds},
    only the query answer from the $(k - 2)^{th}$ iteration is used in the query request, $\clabel{\assign{p}{\query(\chi[y]+p)} }^{7}$ of the next $(k - 1)^{th}$ iteration.
    % $\clabel{\assign{p}{\query(\chi[y]+p)} }^{7}$.
    In all the other iterations, $j \neq (k - 2)$, the if-control goes to the first branch
    % Because the execution will reset
    $p$'s value is reset by the constant $0$ in command $ \clabel{\assign{p}{0}}^{9}$.
    % in all the other iterations
    % at line $10$ after this query request.
    In this way, all the query answers stored in $p$ are erased and are not used
    in the query request at the next iteration, except the one at the $(k - 2)^{th}$ iteration.
    Intuitively, when $k \geq 2$, only the $\clabel{\assign{p}{\query(\chi[y]+p)} }^{7}$ in the $(k - 1)^{th}$ iteration
    depends on the query in $(k - 2)^{th}$ iteration and the \emph{adaptivity} round is $2$.
    When $k = 0$, the program does not go into the loop and there is no dependency between any query request.
    When $k = 1$, the program goes to the first branch in the first if-control with guard $\clabel{ k = 1}^{3}$.
    In this case, the second query request $ \clabel{ \assign{y}{\query(z)}}^{4}$ depends on the first one,
    $\clabel{\assign{z}{\query(0)} }^{1}$ and then the next query in the loop, $\clabel{\assign{p}{\query(\chi[y]+p)} }^{7}$ depends on the second one. Intuitively, the adaptivity is $3$.
    % In this sense, the intuitive \emph{adaptivity} rounds for this example is $2$. 
    However, our adaptivity definition fails to realize that there is only a dependency relation 
    between $p^7$ to itself at the $(k - 2)^{th}$ iteration.
    % but not in all the others. 
    As shown in the semantics-based dependency graph in Figure~\ref{fig:multiRoundsS}(b), 
    there is a cycle on $p^7$ representing the existence of the \emph{Variable May-Dependency} from $p^7$ on itself.
    % and the visiting times of labeled variable $p^7$ is 
    % $\lambda \trace_0 \st k$. 
    Weight of this vertex is $\lambda \trace_0 \st \env(\trace_0) k$,
    % The function $\lambda \trace_0 \st \env(\trace_0) k$ 
    which returns the evaluation times of the command $\clabel{\assign{p}{\query(\chi[y]+p)} }^{7}$ during the program execution under the initial trace $\trace_0$.
    % , which is expected to be equal to the loop iteration numbers, i.e., an initial value of input $k$ from the initial trace $\trace_0$.
    Since the command $\clabel{\assign{p}{\query(\chi[y]+p)} }^{7}$  will always be evaluated the same time as the loop iteration numbers, i.e. $k$,
    the weight function returns $\env(\trace_0) k$.
    However, $\env(\trace_0) k$ is the total number that this command is evaluated, rather than the number of the evaluations in which this command depends on other query requests.
    As a result, the walk with the longest query length 
    is
    $p^7 \to \cdots \to p^7 \to y^4 \to z^1 $ with the vertex $p^7$ visited $\env(\trace_0) k$ times, as the dotted arrows. 
    The adaptivity based on this walk
    is $\lambda \trace_0 \st \env(\trace_0) k + 2$,
    which is expected to be $0$ when $k = 0$ and $2$ when $k \geq 2$.
    %  instead of $\max\{0, 2, 3\}$. 
    % Though the $\THESYSTEM$ is able to give us $2 + k$, as an accurate bound w.r.t this definition.
    {
    \begin{figure}
    \centering
    %}
    \quad
    \begin{subfigure}{.8\textwidth}
    \begin{centering}
    {
    $ \begin{array}{l}
    \kw{multiRoundsS(k)} \triangleq \\
    \clabel{ \assign{j}{0}}^{0} ; 
    \clabel{\assign{z}{\query(0)} }^{1} ; 
    \clabel{\assign{p}{0} }^{2} ; \\
    \eif(\clabel{ k = 1}^{3},
    \clabel{ \assign{y}{\query(z)}}^{4},
    \clabel{\eskip}^5);\\
    \ewhile ~ \clabel{j \neq k}^{6} ~ \edo ~ \Big(
    \\
    \qquad \clabel{\assign{p}{\query(\chi[y]+p)} }^{7} ; \\
    \qquad 
    \eif(\clabel{ j \neq k - 2}^{8}, 
    \clabel{ \assign{p}{0}}^{9} ,
    \clabel{\eskip}^{10})  \\ 
    \qquad \clabel{\assign{j}{j + 1}}^{11} ; 
    \Big)
    \end{array}
    $ 
    }
    \caption{}
    \end{centering}
    \end{subfigure}
    \begin{subfigure}{.8\textwidth}
    \begin{centering}
    \begin{tikzpicture}[scale=\textwidth/15cm,samples=200]
    % Variables Initialization
    \draw[] (-5, 2) circle (0pt) node{{ $z^1: {}^{\lambda \trace_0 \st 1}$}};
    \draw[] (-5, 7) circle (0pt) node{{$p^2: {}^{\lambda \trace_0 \st 1}$}};
    \draw[] (-5, 4) circle (0pt) node{{ $y^4: {}^{\lambda \trace_0 \st 1}$}};
    % Variables Inside the Loop
    \draw[] (0, 6) circle (0pt) node{ $p^7: {}^{\lambda \trace_0 \st \env(\trace_0) k}$};
    \draw[] (0, 2) circle (0pt) node{ $p^{10}: {}^{\lambda \trace_0 \st \env(\trace_0) k}$};
    % Counter Variables
    \draw[] (5, 6) circle (0pt) node {$j^0: {}^{\lambda \trace_0 \st 1}$};
    \draw[] (5, 2) circle (0pt) node { $j^8: {}^{\lambda \trace_0 \st \env(\trace_0) k}$};
    %
    % Value Dependency Edges:
    \draw[ thick, -Straight Barb, densely dotted,] (0.8, 7) arc (220:-100:1);
    \draw[ -latex] (-1.5, 5.5) to [out=-130,in=130] (-1.5, 2);
    % Value Dependency Edges on Initial Values:
    \draw[ thick, -latex, densely dotted,] (-5, 3.5) -- (-5, 2.5) ;
    \draw[ -latex,] (-1.5, 5.5) -- (-4, 7) ;
    \draw[ thick, -latex, densely dotted,] (-1.5, 5.5) -- (-4, 4.7) ;
    \draw[ -latex,] (-1.5, 5.5) -- (-4, 2) ;
    %
    % Value Dependency For Control Variables:
    \draw[ -Straight Barb] (6.5, 2.5) arc (150:-150:1);
    % Control Dependency
    \draw[ -latex] (5, 2.5) -- (5, 5.5) ;
    \draw[ -latex] (1.2, 6) -- (3.5, 6) ;
    \draw[ -latex] (1.2, 6) -- (3.5, 2) ;
    \draw[ -latex] (1.5, 1.8) -- (3.5, 2) ; 
    % Edges Produced by Transitivity
    \draw[ -latex] (1.5, 1.8) -- (3.5, 6) ; 
    \end{tikzpicture}
    \caption{}
    \end{centering}
    \end{subfigure}
    \caption{(a) The loop example with single adaptivity rounds.
    (b) The corresponding semantics-based dependency graph.}
    \label{fig:multiRoundsS}
    \end{figure}
    }
    \end{example}

\section{Implementation}
\label{sec:implementation}
\subsection{Implementation Results}
We implemented $\THESYSTEM$ as a tool which takes a labeled command as input  
% labeled command 
and outputs an upper bound on the program adaptivity and on the number of query requests.
This implementation consists of an 
abstract control flow graph generation, weight estimation (as presented in Section~\ref{sec:alg_weightgen}),
edge estimation (as presented in Section~\ref{sec:alg_edgegen}) in Ocaml, 
and the adaptivity computation algorithm shown in Section~\ref{sec:alg_adaptcompute} in Python.
The OCaml program takes the labeled command as input and outputs the program-based dependency graph,
feeds into the python program and the python program provides the adaptivity upper bound and the query number as the final output.

We evaluated this implementation on $17$ example programs with the evaluation results shown  in Table~\ref{tb:adapt-imp}.
In this table,
the first column is the name of each program.
For each program $c$, the second column is its intuitive adaptivity rounds,
the third column is the adaptivity $A(c)(\trace_0)$ w.r.t the input initial trace $\trace_0 \in \mathcal{T}_0(c)$ as definition~\ref{def:trace_adapt}.
($\env(\trace_0) k$ computes $k$'s value in $\trace_0$)
% we defined through our formal semantic model above.
% in Section~\ref{sec:dep_adaptivity}.
% % In the third column, we use $k$ represent the weight function $w_k$ 
% $A(c)$ is a function takes an initial trace $\trace_0 \in \mathcal{T}_0(c)$, 
% the adaptivity w.r.t. $\trace_0$ is represented in this column.
% (in program's execution-based dependency graph) which return value of variable $k$ 
% from an initial trace $\trace_0$, same for natural numbers.
The last column is the output of the $\THESYSTEM$ implementation, which consists of two expressions.
The first one is the upper bound for adaptivity and the second one is the 
upper bound for the total number of query requests in the program.

The first 3 programs we evaluated are $\kw{twoRounds(k)}$, $ \kw{multiRounds(k)}$, 
and the $\kw{linearRegressGD(k, rate)}$ which we discussed in overview and above section.
% Section~\ref{sec:examples}.
% The same for the third programprograms in the table row 
For these examples, $A(c)$ 
% based dynamic analysis 
give the accurate adaptivity definition, 
simultaneously the $\THESYSTEM$ outputs the tight bounds for both of the adaptivity and query requesting number as expected.
% Look at 
% the $ \kw{linearRegressionGD(k)}$ which we discussed in Section~\ref{sec:examples}, the $\THESYSTEM$ outputs the accurate bound $k$ as expected.
But for the forth program $\kw{multiRoundOdd(k)}$, $\THESYSTEM$ outputs an over-approximated upper bound $1 + 2*k$ for the $A(c)$, 
which is consistent with our expectation as discussed in Example~\ref{ex:multipleRoundSingle}. 
The fifth program is the evaluation results for the example in Example~\ref{ex:multipleRoundSingle}, where $\THESYSTEM$ outputs
the tight bound for $A(c)$ but $A(c)$ is a loose definition of the program's actual adaptivity rounds.
%
The programs in the table from  $\kw{seq()}$ to $ \kw{nestedWhileMPRV(k)}$ are 
designed for testing the programs under different possible situations.
These programs contain control dependency, data value dependency,
the nested while, dependency through multiple variables, dependency across nested loops, and so on. 
Overall for these examples, our system gives both the accurate adaptivity definition and 
adaptivity upper bound simultaneously through the dynamic analysis and 
static analysis.
The full programs are defined below from Example~\ref{ex:twoRoundsComplete} to Example~\ref{ex:nestedWhileMPRV}.
%
For the other programs, the complete program can be found in Appendix, and the implementation in GitHub.
\begin {table}[H]
    \caption{Experimental results of {\THESYSTEM} implementation}
        \label{tb:adapt-imp}
        \begin{center}
        \centering
{\footnotesize
        \begin{tabular}{ r | p{12mm} | c | p{25mm} | l}
         Program $c$ & adaptive \newline rounds 
         & $A(c) (\trace_0)$ 
         & $\THESYSTEM$ \newline $\progA(c)$, $\# \query$ 
         & performance \\ 
         \hline
         \hline
         $  \kw{twoRounds(k)}$ & $2$ & $2$ & $2$, $k$ &  \\
         $  \kw{multiRounds(k)}$ & $k$ & $ \env(\trace_0) k $ & $k$, $k$  &    \\
         $  \kw{linearRegressGD(k, r)}$ & $k$ & $\env(\trace_0) k$ & $k$, $2 * \env(\trace_0) k$  &    \\
         $  \kw{multiRoundsOdd(k)}$ & $1 + k$ & $1 + (\env(\trace_0) k) $  & $1 +2 * k$, $1 + 2*k$  &    \\
         $  \kw{multiRoundsSin(k)}$    & $2$ & $2 + (\env(\trace_0) k) $ & $2 + k$, $2 + k$  &    \\
         $\kw{seq()}$ & $4$ & $4$ & $4$, $4$  \\ 
         $\kw{seqRV()}$ & $4$ & $4$ & $4$, $4$ \\  
         $ \kw{ifVD}$ & $3$ & $3$ & $3$, $3$ \\
         $\kw{ifCD()}$ & $3$ & $3$ & $3$, $3$  &    \\
         $ \kw{whileNested(k)}$ & $1+k$ & $1+ (\env(\trace_0) k)$ & $1+k$  &    \\
         $ \kw{whileM(k)}$ & $1 + k$ & $1 +2 * \lfloor \frac{\env(\trace_0) k}{2} \rfloor$ & $1 +2 * \lfloor \frac{k}{2} \rfloor$, $1 + 2 * k$  &    \\
         $ \kw{whileRV(k)}$ & $1 + 2*k$ & $1 + 2*(\env(\trace_0) k)$ & $1 + 2*k$, $2 + 3 * k$  &    \\
         $ \kw{whileVCD(k)}$ & $1 + 2*k$ & $1 + 2*(\env(\trace_0) k)$ & $1 + 2 * k$, $2 + 2 * k$  &    \\
         $ \kw{whileMPVCD(k)}$ & $2 + k$ & $2 + (\env(\trace_0) k)$  & $2 + k$, $1 + 2 * k$   &    \\
         $ \kw{nestWhileVD(k)}$ & $2 + k^2$ & $2 + (\env(\trace_0) k)^2$  & $2 + k^2$, $1 + k + k^2$   &    \\
         $ \kw{nestedWhileRV(k)}$ & $1 + 2*k$ & $1 + 2*(\env(\trace_0) k)$ & $1 + 2*k$,  $1 + k + k^2$   &    \\
         $ \kw{nestedWhileMR(k)}$ & $1 + k + k^2$ & $1 + (\env(\trace_0) k) + (\env(\trace_0) k)^2$  & $1 + k + k^2$,  $2 + k + k^2$  &    \\
         $ \kw{nestedWhileMPRV(k)}$ & $1 + k + k^2$ & $1 + (\env(\trace_0) k) + (\env(\trace_0) k)^2$  & $1 + k + k^2$,  $2 + k + k^2$  &    \\
        \end{tabular}
}        
\end{center}
\end{table}
%
 \subsection{The Evaluated Examples}  
 \paragraph{The complete algorithm for the data analysis example with two adaptivity rounds} 
 
\begin{example}[Complete Two Rounds Algorithm]
    \label{ex:twoRoundsComplete}
    \[
    %
        \kw{twoRounds(k)} \triangleq
    \begin{array}{l}
           \clabel{ a \leftarrow []}^{1} ; \\
            \clabel{\assign{j}{k} }^{2} ; \\
            \ewhile ~ \clabel{j > 0}^{3} ~ \edo ~ \\
            \Big(
             \clabel{\assign{x}{\query(\chi[k - j]\cdot \chi[k])} }^{4}  ; \\
             \clabel{\assign{j}{j-1}}^{5} ;\\
            \clabel{a \leftarrow x :: a}^{6}       \Big);\\
            \clabel{l \leftarrow (\kw{sign}\big (\sum_{i\in [k]} \chi[i]\times\ln\frac{1+a[i]}{1-a[i]} \big ))}^{7}\\
        \end{array}
    \]
    %
    \begin{algorithm}
    \footnotesize
    \caption{A two-round analyst strategy for random data (The example in  \cite{dwork2015generalization})}
    \label{alg:twoRound}
    \begin{algorithmic}
    \REQUIRE Mechanism $\mathcal{M}$ with a hidden data set $D \in \{-1,+1\}^{n\times (k+1)} \subset \dbdom$.
    \STATE  {\bf for}\ $j\in [k]$\ {\bf do}.  
    \STATE \qquad {\bf define} $q_j(d)=d(j)\cdot d(k)$ where $d \in \{D(i) ~|~ i = 0, \cdots, n\} \subseteq \{-1,+1\}^{k+1}$.
    \STATE \qquad {\bf let} $a_j=\mathcal{M}(q_j)$ 
    \STATE \qquad \COMMENT{In the line above, $\mathcal{M}$ computes approx. the exp. value  of $q_j$ over $D$. So, $a_j\in [-1,+1]$.}
    \STATE {\bf define} $q_{k}(d)= d(k) \cdot \kw{sign}\big (\sum_{i\in [k]} x(i) \cdot \ln\frac{1+a_i}{1-a_i} \big )$ where $x\in \{-1,+1\}^{k+1}$.
    \STATE\COMMENT{In the line above,  $\kw{sign}(y)=\left \{ \begin{array}{lr} +1 & \kw{if}\ y\geq 0\\ -1 &\kw{otherwise} \end{array} \right . $.}
    \STATE {\bf let} $a_{k+1}=\mathcal{M}(q_{k+1})$
    \STATE\COMMENT{In the line above,  $\mathcal{M}$ computes approx. the exp. value  of $q_{k+1}$ over $X$. So, $a_{k+1}\in [-1,+1]$.}
    \RETURN $a_{k+1}$.
    \ENSURE $a_{k+1}\in [-1,+1]$
        % \ENSURE 
    \end{algorithmic}
    \end{algorithm}
    %
%
    \end{example}
 %
 \paragraph{The complete algorithm for the data analysis example with multiple adaptivity rounds} 
 
    \begin{example}[Complete Multiple Round Algorithm]
    %
    \begin{algorithm}
    \footnotesize
    \caption{A multi-round analyst strategy for random data base \cite{dwork2015generalization}}
    \label{alg:multiRound}
    \begin{algorithmic}
    \REQUIRE Mechanism $\mathcal{M}$ with a hidden state $X\in [N]^{n}$ sampled u.a.r., control set size $c$
    \STATE Define control dataset $C = \{0,1, \cdots, c - 1\}$
    \STATE Initialize $Nscore(i) = 0$ for $i \in [N]$, $I = \emptyset$ and $Cscore(C(i)) = 0$ for $i \in [c]$
    \STATE  {\bf for}\ $j\in [k]$\ {\bf do} 
    \STATE \qquad {\bf let} $p=\uniform(0,1)$ 
    \STATE \qquad {\bf define} $q (x) = \bernoulli ( p )$ .
    \STATE \qquad {\bf define} $qc (x) = \bernoulli ( p )$ .
    \STATE \qquad {\bf let} $a = \mathcal{M}(q)$ 
    \STATE \qquad {\bf for}\ $i \in [N]$\ {\bf do}
    \STATE \qquad \qquad $Nscore(i) = Nscore(i) + (a - p)*(q (i) - p)$ if $i \notin I$
    \STATE \qquad {\bf for}\ $i \in [c]$\ {\bf do}
    \STATE \qquad \qquad $Cscore(C(i)) = Cscore(C(i)) + (a - p)*(qc (i) - p)$
    \STATE \qquad {\bf let} $I = \{i | i\in [N] \land Nscore(i) > \max(Cscore)\}$
    \STATE \qquad {\bf let} $D = D \setminus I$ 
    \RETURN $D$.
    \end{algorithmic}
    \end{algorithm}
    %
    {\small
    \begin{figure}
        \begin{subfigure}{0.8\textwidth}
        \begin{centering}
        $
    \kw{multiRounds(k, c, N)} \triangleq
    \begin{array}{l}
        \clabel{\assign{j}{N}}^0 ; 
         \clabel{\assign{cs}{0}}^1; 
         \clabel{\assign{ns}{0}}^2;
         \clabel{\assign{I}{0}}^3; 
         \clabel{\assign{w}{k}}^{4} ;\\
         \ewhile ~ \clabel{j > 0}^{5} ~ \edo ~ \\
         \Big(
         \clabel{\assign{j}{j-1}}^{6} ;
         \clabel{\assign{cs}{0 + cs}}^7; 
         \clabel{\assign{ns}{0 + ns}}^8
         \Big); \\
    
         \ewhile ~ \clabel{w > 0}^{9} ~ \edo ~ \\
        \Big(
        \clabel{\assign{w}{w-1}}^{10} ;
        \left[p \leftarrow c \right]^{11}; 
        \left[q \leftarrow c \right]^{12}; 
        \left[ a \leftarrow \query (\chi[I]) \right]^{13};\\
        \clabel{\assign{i}{N}}^{14} ; 
        \ewhile ~ \clabel{i > 0}^{15} ~ \edo ~ \\
        \Big(
        \clabel{\assign{i}{i-1}}^{16} ;
        \clabel{\assign{cs(i)}{cs(i) + (a - p) * (q - p)}}^{17}; \\
        \eif (\clabel{ I < i}^{18}, \clabel{\assign{ns(i)}{{ns(i) + (a - p) * (q - p)}}}^{19},
        \clabel{\assign{ns}{ns(i)}}^{20}    )
        \Big); \\
        \clabel{\assign{i2}{N}}^{21} ; \\
        \ewhile ~ \clabel{i2 > 0}^{22} ~ \edo ~ \\
        \Big(
        \clabel{\assign{i2}{i2-1}}^{23} ;
        \eif (\clabel{ns(i2) > \kw{max}(cs)}^{24}, 
        \clabel{\assign{I}{i + I}}^{25},
        \clabel{\assign{I}{I}}^{26})
        \Big)
        \Big) 
    \end{array}
       $
       \caption{}
        \end{centering}
        \end{subfigure}
        \vspace{-0.3cm}
        \caption{(a) The labeled program implementing the multiple round algorithm (b)The same program in the SSA version}
        \vspace{-0.5cm}
        \label{fig:multiround_complete}
        \end{figure}
    }
    %
    \end{example}
 %
\paragraph{The complete Linear Regression Algorithm with Gradient Decent Optimization}
%
\begin{example}[Linear Regression Algorithm with Gradient Decent Optimization]
\label{ex:linearregression}
    The linear regression algorithm with gradient decent Optimization works well 
    in our $\THESYSTEM$ as well.
            %   \[
            %   %
            %   \begin{array}{l}
            %   \kw{linearRegression(step, rate)} \triangleq \\
            %          \clabel{ a \leftarrow 0}^{0} ; \\
            %          \clabel{ c \leftarrow 0}^{1} ; \\
            %           \clabel{\assign{j}{\kw{step}} }^{2} ; \\
            %         %   \clabel{\assign{d}{10000000} }^{2} ; \\
            %           \ewhile ~ \clabel{j > 0}^{3} ~ \edo ~ \\
            %           \Big(
            %               \clabel{\assign{da}{\query(-2 * (\chi[1] - (\chi[0]\times a + c)) \times (\chi[0]))} }^{4}  ; \\
            %               \clabel{\assign{dc}{\query(-2 * (\chi[1] - (\chi[0]\times a + c)))} }^{5}  ; \\
            %               \clabel{\assign{a}{a - \kw{rate} * da} }^{6}  ; \\
            %               \clabel{\assign{c}{c - \kw{rate} * dc} }^{7}  ; \\
            %            \clabel{\assign{j}{j-1}}^{8} 
            %         %   \clabel{a \leftarrow x :: a}^{6} 
            %           \Big);
            %       \end{array}
            %   \]
              %
              %
                   %
\begin{figure}
\centering
\begin{subfigure}{0.45\textwidth}
    \centering
    {\small
        \[
        \begin{array}{l}
            \kw{linearRegressionGD(k, rate)} \triangleq \\
                   \clabel{ a \leftarrow 0}^{0} ; 
                   \clabel{ c \leftarrow 0}^{1} ; 
                    \clabel{\assign{j}{\kw{k}} }^{2} ; \\
                  %   \clabel{\assign{d}{10000000} }^{2} ; \\
                    \ewhile ~ \clabel{j > 0}^{3} ~ \edo ~ \\
                    \Big(
                        \clabel{\assign{da}{\query(-2 * (\chi[1] - (\chi[0]\times a + c)) \times (\chi[0]))} }^{4}  ; \\
                        \clabel{\assign{dc}{\query(-2 * (\chi[1] - (\chi[0]\times a + c)))} }^{5}  ; \\
                        \clabel{\assign{a}{a - \kw{rate} * da} }^{6}  ; 
                        \clabel{\assign{c}{c - \kw{rate} * dc} }^{7}  ; \\
                     \clabel{\assign{j}{j-1}}^{8} 
                  %   \clabel{a \leftarrow x :: a}^{6} 
                    \Big);
                \end{array}
        \]
        }
     \caption{}
        \end{subfigure}
      \begin{subfigure}{.45\textwidth}
          \begin{centering}
          \begin{tikzpicture}[scale=\textwidth/20cm,samples=200]
    % Variables Initialization
    \draw[] (-6, 1) circle (0pt) node{{ $a^0: {}^1_{0}$}};
    \draw[] (-6, 4) circle (0pt) node{{ $c^1: {}^{1}_{0}$}};
    % Variables Inside the Loop
       \draw[] (0, 10) circle (0pt) node{{ $da^4: {}^{k}_{1}$}};
       \draw[] (0, 7) circle (0pt) node{{ $dc^5: {}^{k}_{0}$}};
       \draw[] (0, 4) circle (0pt) node{{ $a^6: {}^{k}_{0}$}};
       \draw[] (0, 1) circle (0pt) node{{ $c^7: {}^{k}_{0}$}};
       % Counter Variables
       \draw[] (7, 9) circle (0pt) node {{$j^0: {}^{1}_{0}$}};
       \draw[] (7, 6) circle (0pt) node {{ $j^8: {}^{k}_{0}$}};
       %
       % Value Dependency Edges:
       \draw[ thick, -latex,] (0, 1.5)  -- (0, 3.5) ;
       \draw[ thick, -Straight Barb] (1.8, 4.2) arc (220:-100:1);
       \draw[ thick, -Straight Barb] (7.5, 6.5) arc (150:-150:1);
       \draw[](10, 6) node[] {\highlight{$k$}} ;
       \draw[ thick, -Straight Barb] (1.7, 1.) arc (120:-200:1);
       \draw[](4, 0) node[] {\highlight{$k$}} ;
       \draw[ thick, -latex] (6, 6.5)  -- 
       node [right] {\highlight{$k$}}(6, 8.5) ;
       % Value Dependency Edges on Initial Values:
       \draw[ thick, -latex,] (-2, 1)  -- 
       node [above] {\highlight{$k$}}(-4.5, 1) ;
       \draw[ thick, -latex,] (-2, 4)  -- 
       node [above] {\highlight{$k$}}(-4.5, 4) ;
       %
       \draw[ ultra thick, -latex, densely dotted,] (-1, 1.5)  to  [out=-220,in=220]  
       node [below] {\highlight{$k$}}(-1, 6.5);
       \draw[ ultra thick, -latex, densely dotted,] (-1, 4.5)  to  [out=-220,in=220]  
       node [above] {\highlight{$k$}}(-1, 9.5);
       \draw[ ultra thick, -latex, densely dotted,]  (1, 6.2) to  [out=-60,in=60] 
       node [below] {\highlight{$k$}}(0.5, 1.5) ;
       \draw[ ultra thick, -latex, densely dotted,]  (1.2, 9.2)  to  [out=-50,in=50] 
       node [above] {\highlight{$k$}}(0.5, 4.5);
       % Control Dependency
      %  \draw[ thick,-latex] (1.5, 7)  -- (4, 9) ;
      %  \draw[ thick,-latex] (1.5, 4)  -- (4, 9) ;
       \draw[ thick,-latex] (1.8, 10)  -- 
       node [above] {\highlight{$k$}}(5.5, 6) ;
       \draw[ thick,-latex] (1.8, 7)  -- (5.5, 6) ;
       \draw[ thick,-latex] (1.8, 4)  -- 
       node [above] {\highlight{$k$}}(5.5, 6) ;
       \draw[ thick,-latex] (1.8, 1)  -- 
       node [below] {\highlight{$k$}}(5.5, 6) ;
       \end{tikzpicture}
       \caption{}
          \end{centering}
          \end{subfigure}
          %
        \begin{subfigure}{.8\textwidth}
            \begin{centering}
            \begin{tikzpicture}[scale=\textwidth/20cm,samples=200]
      % Variables Initialization
      \draw[] (-6, 1) circle (0pt) node{{ $a^0: {}^1_{0}$}};
      \draw[] (-6, 4) circle (0pt) node{{ $c^1: {}^{1}_{0}$}};
      % Variables Inside the Loop
         \draw[] (0, 10) circle (0pt) node{{ $da^4: {}^{k}_{1}$}};
         \draw[] (0, 7) circle (0pt) node{{ $dc^5: {}^{k}_{0}$}};
         \draw[] (0, 4) circle (0pt) node{{ $a^6: {}^{k}_{0}$}};
         \draw[] (0, 1) circle (0pt) node{{ $c^7: {}^{k}_{0}$}};
         % Counter Variables
         \draw[] (7, 9) circle (0pt) node {{$j^0: {}^{1}_{0}$}};
         \draw[] (7, 6) circle (0pt) node {{ $j^8: {}^{k}_{0}$}};
         %
         % Value Dependency Edges:
         \draw[ thick, -latex,] (0, 1.5)  -- (0, 3.5) ;
         \draw[ thick, -Straight Barb] (1.8, 4.2) arc (220:-100:1);
         \draw[ thick, -Straight Barb] (7.5, 6.5) arc (150:-150:1);
         \draw[](10, 6) node[] {\highlight{$\trace_0 \to \env(\trace_0) k $}} ;
         \draw[ thick, -Straight Barb] (1.7, 1.) arc (120:-200:1);
         \draw[](4, 0) node[] {\highlight{$\trace_0 \to \env(\trace_0) k $}} ;
         \draw[ thick, -latex] (6, 6.5)  -- 
         node [right] {\highlight{$\trace_0 \to \env(\trace_0) k $}}(6, 8.5) ;
         % Value Dependency Edges on Initial Values:
         \draw[ thick, -latex,] (-2, 1)  -- 
         node [above] {\highlight{$\trace_0 \to \env(\trace_0) k $}}(-4.5, 1) ;
         \draw[ thick, -latex,] (-2, 4)  -- 
         node [above] {\highlight{$\trace_0 \to \env(\trace_0) k $}}(-4.5, 4) ;
         %
         \draw[ ultra thick, -latex, densely dotted,] (-1, 1.5)  to  [out=-220,in=220]  
         node [below] {\highlight{$\trace_0 \to \env(\trace_0) k $}}(-1, 6.5);
         \draw[ ultra thick, -latex, densely dotted,] (-1, 4.5)  to  [out=-220,in=220]  
         node [above] {\highlight{$\trace_0 \to \env(\trace_0) k $}}(-1, 9.5);
         \draw[ ultra thick, -latex, densely dotted,]  (1, 6.2) to  [out=-60,in=60] 
         node [below] {\highlight{$\trace_0 \to \env(\trace_0) k $}}(0.5, 1.5) ;
         \draw[ ultra thick, -latex, densely dotted,]  (1.2, 9.2)  to  [out=-50,in=50] 
         node [above] {\highlight{$\trace_0 \to \env(\trace_0) k $}}(0.5, 4.5);
         % Control Dependency
        %  \draw[ thick,-latex] (1.5, 7)  -- (4, 9) ;
        %  \draw[ thick,-latex] (1.5, 4)  -- (4, 9) ;
         \draw[ thick,-latex] (1.8, 10)  -- 
         node [above] {\highlight{$\trace_0 \to \env(\trace_0) k $}}(5.5, 6) ;
         \draw[ thick,-latex] (1.8, 7)  -- (5.5, 6) ;
         \draw[ thick,-latex] (1.8, 4)  -- 
         node [above] {\highlight{$\trace_0 \to \env(\trace_0) k $}}(5.5, 6) ;
         \draw[ thick,-latex] (1.8, 1)  -- 
         node [below] {\highlight{$\trace_0 \to \env(\trace_0) k $}}(5.5, 6) ;
         \end{tikzpicture}
         \caption{}
            \end{centering}
            \end{subfigure}
    \vspace{-0.5cm}
    \caption{(a) The linear regression algorithm 
    (b) The program-based dependency graph from $\THESYSTEM$
    (c) The execution-based dependency graph.}
    \vspace{-0.5cm}
    \label{fig:linear_regression}
\end{figure}
%
Analysis Result: $ \progA(\kw{linearRegressionGD(k, rate)}) = k$
\end{example} 
%
 
This linear regression algorithm 
% in order to
aims to
model a linear relationship between a dependent variable $y$,
% corresponding to the observed value in the column $\chi[1]$ in database, 
and an independent variable $x$, $y = a \times x + c$, specifically approximating the 
model parameter $a$ and $c$.
In order to have a good approximation on the model parameter 
$a$ and $c$, 
% corresponding to the observed value in the column $\chi[0]$ in database, 
it sends query to a training data set adaptively in every iteration.
This training data set contains two columns (can extend to higher dimensional data sets), first column is used as the observed value for the independent variable $x$,
second column is used as the observed label value for the dependent variable $y$.
This algorithm is written in our {\tt Query While} language in Figure~\ref{fig:linear_regression}(a) as $\kw{linearRegressionGD(k, rate)}$.
% taking the iteration number $\kw{step}$ 

This linear regression algorithm starts from initializing the linear model parameters and the counter variable,
and then goes into the training iterations.
In each iteration, it computes the differential value w.r.t. parameter
$a$ and $c$ respectively,
through requesting two queries, $\query(-2 * (\chi[1] - (\chi[0]\times a + c)) \times (\chi[0]))$ and 
$\query(-2 * (\chi[1] - (\chi[0]\times a + c)))$
at line 4 and 5.
Then, it uses these two differential values stored in variable $da$ and $dc$ to update the linear model parameters $a$ and $c$.
%
Its the program-based dependency graph is shown in Figure~\ref{fig:linear_regression}(b). Its execution-based dependency graph share the same graph, only needs to change the weight, $k$ into $w_k$ and $1$ for $w_1$ as we do in the previous example.
% We omit the detail of how to 
% generate this graph, which is similar to the generation procedure in 
% Example~\ref{alg:multiRound}.
In the execution-based dependency graph, there are multiple walks having the same longest query length.
For example, the walk $c^7 \to dc^6 : \to c^7 \to \cdots \to dc^6$ along the 
dotted arrows, where each vertex is visited $w_k(\trace_0)$ times for an initial trace $\trace_0$.
% By counting the total occurrence time of vertices with annotation $1$ in this walk, we have this program's adaptivity $k$.
There is actually other walks having the same query length $k$, the 
walk $a^7 \to da^6  \to a^7 \to \cdots \to da^6 $ along the 
dotted arrows, where each vertex is visited $w_k(\trace_0)$ times.
% the dotted path corresponds to a finite walk with the longest query length and its adaptivity on this walk is $k$.
But it doesn't affect the adaptivity for this program, which is still the maximal query length $w_k(\trace_0)$ with respect to initial trace $\trace_0$.
Also, $\THESYSTEM$, estimates the adaptivity $k$ for this example. Similarly as the multiple round example, we can show it is a tight bound.
%
 %          
\paragraph*{The complete Programs for Examples from line:6 - 17 in Table.\ref{tb:adapt-imp}}
%
    \begin{example}[The Complete Gradient Decent Optimization Algorithm]
        This example is the gradient decent algorithm example is a generalization of the linear regression on a higher degree data relation.
        It uses gradient decent algorithm to minimize 
        the mean square loss function
        for a two-degree relation
         $y = a_1 \times x_1^2 + a_2 \times x_2 + c$
        on the dataset of two feature columns and one indicator column.
     \[
     %
     \begin{array}{l}
     \kw{gradientDecent(step, rate, t, n)} \triangleq \\
        \clabel{ a_1 \leftarrow 0}^{0} ; \\
        \clabel{ a_2 \leftarrow 0}^{1} ; \\
        \clabel{ c \leftarrow 0}^{2} ; \\
        \clabel{\assign{j}{\kw{step}} }^{3} ; \\
        \ewhile ~ \clabel{j > 0}^{4} ~ \edo ~ \\
      \Big(
          \clabel{\assign{da1}{\query(-2 * (\chi[2] - (\chi[0]^2 \times a_1 + \chi[1] \times a_2 + c)) \times (\chi[0]))} }^{5}  ; \\
          \clabel{\assign{da2}{\query(-2 * (\chi[2] - (\chi[0]^2 \times a_1 + \chi[1] \times a_2 + c)) \times (\chi[1]))} }^{6}  ; \\  \clabel{\assign{dc}{\query(-2 * (\chi[2] - (\chi[0]^2 \times a_1 + \chi[1] \times a_2 + c)))} }^{5}  ; \\
          \clabel{\assign{a_1}{a_1 - \kw{rate} * da1} }^{7}  ; \\
          \clabel{\assign{a_2}{a_2 - \kw{rate} * da2} }^{8}  ; \\
          \clabel{\assign{c}{c - \kw{rate} * dc} }^{9}  ; \\
       \clabel{\assign{j}{j-1}}^{10} 
      \Big);
  \end{array}
     \]
     %
     %
        This approach can be generalized to the regression of a variety of 
        relations in machine learning area.
   %
     \end{example}
%

    \begin{example}[Sequence with Linear Query Dependency]
        \label{ex:seq}
        This example algorithm contains only sequence of four query commands.
        Each of them depends on a previous query.
        The longest dependency depth, i.e., the adaptivity is expectation to be $4$.
        %
        %
        \[
        %
            \kw{seq()} \triangleq 
        \begin{array}{l} 
               \clabel{ \assign{x}{\chi[0]}}^{0} ; 
   \clabel{\assign{y}{\chi[x + 1]} }^{1} ; \\
   \clabel{\assign{z}{\chi[y + 1]}}^{2}; 
    \clabel{\assign{w}{\chi[z + 1]} }^{3}
            \end{array}
        \]
        Evaluation Result: $ \progA( \kw{seq()}) = 4$
        \end{example}
    %
    \begin{example}[Sequence with Query Dependency between Related Variables]
        \label{ex:seqRV}
        %
        This example algorithm contains a sequence of four query commands.
        Each of them depends on one or more of the previous queries.
        The longest dependency depth, i.e., the adaptivity is expectation to be $4$.
        %
        \[
        %
            \kw{seqRV()} \triangleq 
        \begin{array}{l} 
               \clabel{ \assign{x}{\chi[0]}}^{0} ;
   \clabel{\assign{y}{\chi[x + 1]} }^{1} ; \\
   \clabel{\assign{z}{\chi[y + x]}}^{2}; 
    \clabel{\assign{w}{\chi[z + 1] \cdot \chi[y]} }^{3}
            \end{array}
        \]
        Evaluation Result: $ \progA(\kw{seqMultiVar()}) = 4$
    \end{example}
    %
        \begin{example}[If with Data-Value Dependency Separated]
            \label{ex:ifVD}
            This example algorithm contains a $\eif$ command and a query requests
            in each branch.
            Only the query in the first branch depend on the query in the command $0$,
            and the variable in the guard is not assigned by a query request.
            % Each of them depends on one or more of the previous queries.   %
            The longest dependency depth, i.e., the adaptivity is expectation to be $3$.
            \[
            %
            \kw{ifVD}(k) \triangleq 
            \begin{array}{l}
               \quad \clabel{ \assign{z}{\query(\chi[0])}}^{0} ; 
               \quad \clabel{\assign{x}{k / 2} }^{1} ; \\
               \quad \eif(\clabel{x < 0}^2,
               \quad \clabel{\assign{y}{\query(\chi[z])}}^{3},
               \quad \clabel{\assign{y}{\query(\chi[0])}}^{4})
   \end{array}
            \]
            Evaluation Result: $ \progA( \kw{ifVD()}) = 3$
        \end{example}
    
            \begin{example}[If with Data-Control Dependency Overlapped]
   \label{ex:ifCD}
   %
   This example algorithm contains a $\eif$ command and a query requests
   in each branch.
   The variable in the guard is assigned by a query request in command $1$.
   The two queries in the branches depend on the second query in command $1$
   but not depend on the query in the command $0$.
   Even though the variable $x$ isn't used in the query expression in the query $3$ and $4$,
   there are still dependency relation because $x$ is in the guard.
%
The longest dependency depth, i.e., the adaptivity is expectation to be $3$.
   \[
   %
   \kw{ifCD()} \triangleq 
   \begin{array}{l}
\clabel{ \assign{z}{\query(\chi[0])}}^{0} ;
\clabel{\assign{x}{\query(\chi[z])} }^{1} ; \\
\eif(\clabel{x < 0}^{2}, 
\clabel{\assign{y}{\query(\chi[0] + \chi[1])}}^{3}, 
\clabel{\assign{y}\query{(\chi[0])}}^{4})
   \end{array}
   \]
   %
   Evaluation Result: $ \progA( \kw{ifCD()}) = 3$
   \end{example}
    
    
\begin{example}[While with Nested Query Dependency]
\label{ex:whileNested}
This example algorithm contains a simple while loop.
There is one query requests in the loop body at command $3$.
In each iteration, the query request depend on the query result from previous iteration.
The longest dependency depth, i.e., the adaptivity is expectation to be $k$.
%
\[
%
\kw{whileNested}(k) \triangleq
\begin{array}{l}
    \clabel{ \assign{j}{k} }^{0} ; 
    \clabel{ \assign{a}{\query(\chi[0])} }^{1} ; \\
        \ewhile ~ \clabel{j > 0}^{2} ~ \edo ~ \\
        \Big(
         \clabel{\assign{x}{\query(\chi[a]) }}^{3}  ; 
         \clabel{\assign{a}{x + a}}^{4} ;
        \clabel{\assign{j}{j-1}}^{5}       \Big)
    \end{array}
\]
The Evaluation Result: $ \progA(\kw{whileRec}(k)) = 1 + k$
   \end{example}
    %
            \begin{example}[While with Multi-Path Query Dependency]
   \label{ex:whileM}
   %
   This example algorithm contains a simple while loop and a $\eif$ command in the loop body.
% There is one query requests in the loop body at command $3$.
Each branch  has a query request (in the commands $5$ and $6$)
depend on the query at command $1$ and the query at command $7$.
Among the $\frac{k}{2}$ iterations,
% the query at command $7$ depend on the query at line $5$, otherwise not.
 result from previous iteration.
The longest dependency depth, i.e., the adaptivity is expectation to be $1 +2 * \lfloor \frac{k}{2} \rfloor$.
            %
            \[
            %
            \kw{whileM}(k) \triangleq 
            \begin{array}{l}
   \clabel{ \assign{j}{k}}^{0} ; 
   \clabel{ \assign{x}{\query(\chi[0])} }^{1} ; \\
\ewhile ~ \clabel{j > 0}^{2} ~ \edo ~ \\
\Big(
 \clabel{\assign{j}{j-1}}^{3} ;\\
 \eif(\clabel{j \% 2 == 0}^{4}, 
 \clabel{\assign{y}{\chi[x]}}^{5}, 
 \clabel{\assign{w}{\chi[x]}}^{6});\\        
 \clabel{\assign{x}{\query(\chi(\ln(y)))} }^{7} \Big)
   \end{array}
            \]
            The Evaluation Result: $ \progA(\kw{whileM}(k)) = 1 +2 * \lfloor \frac{k}{2} \rfloor $
        \end{example}
    %
            \begin{example}[While with Query Dependency through Related Variables]
   \label{ex:whileRV}
   This example algorithm contains a simple while loop
    and a sequence of three query requests in the loop body.
% There is one query requests in the loop body at command $3$.
In each iteration, every query request depend on one or more
query results from previous iteration.
% the query at command $7$ depend on the query at line $5$, otherwise not.
The longest dependency depth, i.e., the adaptivity is expectation to be $1 +2 * k$.
   \[
   %
   \kw{whileRV}(k) \triangleq 
   \begin{array}{l}
   \clabel{\assign{j}{k} }^{0} ; 
   \clabel{ \assign{x}{\query(\chi[0])}}^{1} ; 
\clabel{ \assign{y}{\query(\chi[1])}}^{2} ; \\
    \ewhile ~ \clabel{j > 0}^{3} ~ \edo ~ \\
    \Big(
     \clabel{\assign{j}{j-1}}^{4} ;
     \clabel{\assign{z}{\query(\chi(x + \ln(y)))} }^{5}  ; 
     \clabel{ \assign{x}{\query(\chi[z])}}^{6} ; 
     \clabel{ \assign{y}{\query(\chi[z])}}^{7} 
    \Big)
\end{array}
   \]
   The Evaluation Result: $ \progA(\kw{whileRV}(k)) = 1 + 2 * k $
            \end{example}
   %
   %
   \begin{example}[While with Query Dependency trhough Control Flow and Data Flow]
\label{ex:whileVCD}
%
This example algorithm contains a simple while loop
and a sequence of three query requests in the loop body.
The variable in the guard is assigned by a query request in command $0$.
In each iteration, the query at $3$ depends on either the query at line $1$, and the query result at line $4$ from the previous iteration.
%  in the branches depend on the second query in command $1$
In each iteration, the query at $4$ depends on either the query at line $0$ and the query at line $3$ in the same iteration.
% Even though the variable $x$ isn't used in the query expression in the query $3$ and $4$,
The longest dependency depth, i.e., the adaptivity is expectation to be $1 +2 * k$.
\[
\kw{whileVCD}() \triangleq
\begin{array}{l}
    \clabel{ \assign{x}{\query(\chi[0])} }^{0} ; 
    \clabel{ \assign{z}{\query(\chi[0])} }^{1} ; \\
        \ewhile ~ \clabel{x > 0}^{2} ~ \edo ~ \\
        \Big(
        \clabel{\assign{x}{\query(\chi(z))} }^{3}  ; 
        \clabel{\assign{z}{\query(\chi(x))}}^{4}
      \Big)
    \end{array}
\]
The Evaluation Result: $ \progA(\kw{whileVCD}(k)) = 1 + 2 * k $
   \end{example}
    %
   \begin{example}[While with Multiple Path Query Dependency Dependency]
\label{ex:whileMPVCD}
%
This example algorithm contains a simple while loop and a $\eif$ command in the loop body.
% There is one query requests in the loop body at command $3$.
Each branch  has a query request (in the commands $5$ and $6$)
depend on either the query at command $1$ or the query at command $7$.
% Among the $\frac{k}{2}$ iterations,
% % the query at command $7$ depend on the query at line $5$, otherwise not.
%  result from previous iteration.
The longest dependency depth, i.e., the adaptivity is expectation to be $2 + k$.
\[
    %
    \kw{whileMPVCD}(k) \triangleq
    \begin{array}{l}
        \clabel{ \assign{x}{\query(k)}}^{0} ; 
        \clabel{\assign{y}{0} }^{1} ; 
            \ewhile ~ \clabel{x > 0}^{2} ~ \edo ~ \\
            \Big(
             \eif(\clabel{y > 0}^{3}, 
             \clabel{\assign{y}{\query(\chi[12])}}^{4}, 
             \clabel{\assign{w}{\query(\chi[9])}}^{5});        
             \\
             \clabel{\assign{x}{x-1}}^{6}\Big);\\
             \clabel{\assign{y}{\query(\chi(\ln(y)))} }^{7} 
        \end{array}
    \]
    The Evaluation Result: $ \progA(\kw{whileMPVCD}(k)) = 2 + k $
\end{example}
   %
\begin{example}[Nested While with Nested Query Dependency]
    \label{ex:nestWhileVD}
    %
    This example algorithm contains two nested while loops.
    The query in the outer loop at line $5$ depends on either the query at line $1$ or
    the query results at line $8$ from the previous iteration of the inner loop.
    The longest dependency depth, i.e., the adaptivity is expectation to be $2 + k^2$.
        %
    \[
    %
    \kw{nestWhileVD}(k) \triangleq 
    \begin{array}{l}
        \clabel{ \assign{i}{k} }^{0} ; 
        \clabel{\assign{x}{\query(\chi[0])}}^{1} ; \\
            \ewhile ~ \clabel{i > 0}^{2} ~ \edo ~ 
            \Big(
             \clabel{\assign{i}{i-1}}^{3} ;
             \clabel{\assign{j}{k}}^{4} ;
             \clabel{\assign{y}{\query(\chi(\ln(x)))} }^{5}  ; \\
             \ewhile ~ \clabel{j > 0}^{6} ~ \edo ~ 
             \Big(
              \clabel{\assign{j}{j-1}}^{7};
              \clabel{\assign{x}{\query(\chi(\ln(x)))} }^{8}
              \Big) \Big)
        \end{array}
    \]
    The Evaluation Result: $ \progA(\kw{nestWhileVD}(k)) = 2 + k^2 $
\end{example}
    
    \begin{example}[Nested While with Query Dependency through Related Variables]
        \label{ex:nestedWhileRV}
        %
        This example algorithm contains two nested while loops, one query in the outer loop, and one query in the inner loop.
        The query in the outer loop at line $8$ depends on only the query result at line $7$
        from the last iteration of the inner loop.
        %  either the query at line $1$ or
        However, the query at line $7$ depends on  either the query at line $1$ 
        the query results at line $8$ from the previous iteration.
        The longest dependency depth, i.e., the adaptivity is expectation to be $1 + 2 * k $.
            %
        \[
        %
            \kw{nestWhileRV}(k) \triangleq 
        \begin{array}{l}
            \clabel{ \assign{i}{k} }^{0} ; 
            \clabel{\assign{x}{\query(\chi[0])}}^{1} ; \\
   \ewhile ~ \clabel{i > 0}^{2} ~ \edo ~ 
   \Big(
    \clabel{\assign{i}{i-1}}^{3} ;
    \clabel{\assign{j}{k}}^{4} ;\\
    \ewhile ~ \clabel{j > 0}^{5} ~ \edo ~ 
    \Big(
     \clabel{\assign{j}{j-1}}^{6};
     \clabel{\assign{y}{\query(\chi(x) + \chi(1))} }^{7}
     \Big); \\
    \clabel{\assign{x}{\query(\chi(\ln(y)))} }^{8}
     \Big)
            \end{array}
        \]
        The Evaluation Result: $ \progA(\kw{nestWhileRV}(k)) = 1 + 2 * k $
    \end{example}
%
   
        \begin{example}[Nested While with Nest Query Dependency and Related Variable Accross Outer and Inner Loop]
            \label{ex:nestedWhileMR}
            %
            This example algorithm contains two nested while loops, one query in the outer loop, and one query in the inner loop as well.
            The two queries depend on both the query results assigned to themselves in previous iteration.
            The longest dependency depth, i.e., the adaptivity is expectation to be $1 + k + k^2 $.
            \[
            %
            \kw{nestWhileMR}(k) \triangleq 
            \begin{array}{l}
                \clabel{\assign{i}{k} }^{0} ; 
                \clabel{ \assign{x}{\query(\chi[0])}}^{1} ; 
                \clabel{ \assign{y}{\query(\chi[1])}}^{2} ; 
                \ewhile ~ \clabel{i > 0}^{3} ~ \edo ~ \\
                \Big(
                \clabel{\assign{i}{i-1}}^{4} ;
                \clabel{\assign{j}{k}}^{5} ;
                \clabel{\assign{y}{\query(\chi(\ln(x) + y))} }^{6}  ; \\
                \ewhile ~ \clabel{j > 0}^{7} ~ \edo ~ 
                \Big(
                \clabel{\assign{j}{j-1}}^{8};
                \clabel{\assign{x}{\query(\chi(\ln(y))+\chi[x])} }^{9}
                \Big) \Big)
            \end{array}
            \]
            The Evaluation Result: 
            $ \progA(\kw{nestWhileMR}(k)) = 1 + k + k^2$
            \\
            Reachability Bound The Evaluation Result: \\
            weight for Variable: j of label 6 is: 0 + 0 + 1 * k * k\\
            weight for Variable: y of label 7 is: 0 + 0 + 1 * k * k\\
            weight for Variable: j of label 4 is: 0 + 1 * k\\
            weight for Variable: i of label 3 is: 0 + 1 * k\\
            weight for Variable: x of label 8 is: 0 + 1 * k\\
            weight for Variable: x of label 1 is: 1\\
            weight for Variable: i of label 0 is: 1\\
            \end{example}
            \begin{example}[Nested While with MultiplePath and Nested Recursive Multiple Variable 
   Data-Value Dependency Across Outer and Inner Loop]
   \label{ex:nestedWhileMPRV}
   %
   We then show a more complex example with nested while command and nested data-flow across the outer and inner while loop through multiple variables.
   This example also contains the if command with data dependency occurred through the if guard.
   The longest dependency depth, i.e., the adaptivity is expectation to be $1 + k + k^2 $.
   %
   \[
   %
   \kw{nestWhileMPRV}(k) \triangleq 
   \begin{array}{l}
\clabel{\assign{i}{k} }^{0} ; 
\clabel{ \assign{x}{\query(\chi[0])}}^{1} ; 
\clabel{ \assign{y}{\query(\chi[1])}}^{2} ; \\
    \ewhile ~ \clabel{i > 0}^{3} ~ \edo ~ 
    \Big(
     \clabel{\assign{i}{i-1}}^{4} ;
     \clabel{\assign{j}{k}}^{5} ;\\
     \eif(\clabel{x > 0}^6, \clabel{\assign{y}{\query(\chi(\ln(x) + y))} }^{7},
     \clabel{\assign{y}{\query(\chi(x))} }^{8} )
      ; \\
     \ewhile ~ \clabel{j > 0}^{9} ~ \edo ~ 
     \Big(
      \clabel{\assign{j}{j-1}}^{10};
      \clabel{\assign{x}{\query(\chi(\ln(y))+\chi[x])} }^{11}
      \Big) \Big)
\end{array}
   \]
   \end{example}
   The Evaluation Result: $ \progA(\kw{nestWhileMPRV}(k)) = 1 + k + k^2$
   \\
   Reachability Bound The Evaluation Result: \\
            weight for Variable: j of label 10 is: 0 + 0 + 1 * k * k \\
   weight for Variable: x of label 11 is: 0 + 0 + 1 * k * k \\
   weight for Variable: y of label 7 is: 0 + 1 * k \\
   weight for Variable: y of label 8 is: 0 + 1 * k \\
   weight for Variable: j of label 5 is: 0 + 1 * k \\
   weight for Variable: i of label 4 is: 0 + 1 * k \\
   weight for Variable: y of label 2 is: 1 \\
   weight for Variable: x of label 1 is: 1 \\
   weight for Variable: i of label 0 is: 1 \\

%%%%%%%%%%%%%%%%%%%%%%%%%%%%%%%%%%%%%%%%%%% Related Work and Conclusion %%%%%%%%%%%%%%%%%%%%%%%%%%%%%%%%%%%%%%%%%%% 
\section{Related Work}

% \dg{Please cite adaptive Fuzz, and explain how its adaptivity analysis differs from ours.}

%In terms of techniques, our work relies on ideas from both static analysis and dynamic analysis. We discuss closely related work in both areas.


\paragraph{Dependency Definitions and Analysis} 
There is a vast literature on dependency definitions and dependency analysis. 
We consider a semantics definition of dependencies which consider (intraprocedural) data and control dependency~\cite{bilardi1996framework,cytron1991efficiently,pollock1989incremental}.    
Our definition is inspired by classical works on traditional dependency analysis~\cite{DenningD77} and noninterference~\cite{GoguenM82a}.
Formally, our definition is similar to the one by \citet{Cousot19a}, which also identifies dependencies by considering differences in two execution traces. 
However, Cousot excludes some forms of implicit dependencies, e.g. the ones generated by empty observations,  which instead we consider. 
%
Common tools to study dependencies are dependency graphs~\cite{ferrante1987program}. We use here a semantics-based approach to dependency graph similar, for example, to works by \citet{austin1992dynamic}, \citet{hammer2006dynamic} and \cite{mastroeni2008data}.
%propose ways of constructing different kinds of program slices, by choose different program 
%DDGs have been used in many other domains. \citet{nagar2018automated} use DDGs to find serializability violations. dependency. 
% For example, in either syntactic or semantics sense.
% This abstract dependency is based on properties rather than exact data.
% Aims to give finer and smaller program slice. 
%They actually use a combination of  
%static and dynamic dependency graphs but in a manner that is different from how we use the two. Their slicing uses both static and dynamic dependency graphs, while we use the dynamic dependency graph as the basis of a definition, which is then soundly approximated by an analysis based on the static dependency graph.}
%\paragraph{Static program analysis} 
%Our algorithm in Section~\ref{sec:algorithm} is influenced by previous works in static analysis related to effect systems, control-flow analysis, and data-flow analysis. 
%The idea of statically estimating a sound upper bound for the adaptivity from the semantics is indirectly inspired from prior work on cost analysis via effect systems~\cite{cciccek2017relational,radivcek2017monadic,qu2019relational}. The idea of defining adaptivity using data flow is inspired by the work of graded Hoare logic~\cite{gaboardi2021graded}, which reasons about data flows as a resource. 
%
Our approach shares some similarities with the use of dependency graphs in works analyzing dependencies between events, e.g. in event programming. \citet{memon2007event} uses an event-flow graph, representing all the possible event interactions, where vertices are GUI event edges represent pairs of events that can be performed immediately one after the other. In a similar way, we use edges to track the may-dependence between variables looking at all the possible interactions. 
% of one variable with respect to another variable. The main difference is in the way the graph is constructed. {\THESYSTEM} relies on the structure of the target program, while the event-flow model only considers the event type.
\citet{arlt2012lightweight} use a weighted edges indicating a dependency between two events, e.g. one event possibly reads data written by the other event, with the weight showing the intensity of the dependency (the quantity of data involved). We also use weights but on vertices and with different meaning, they are functions describing the number of times the vertices can be visited given an initial state.
% WCET on systems: \cite{} 
% [GustafssonEL05]Towards a Flow Analysis for Embedded System C Programs
% --> abstract interpretation.
% --> on embedded system of c program
% [AlbertAGP08] Automatic Inference of Upper Bounds for Recurrence Relations in Cost Analysis
% --> invariant generation through ranking functions
%
% General While langue:
% [BrockschmidtEFFG16]
% Analyzing Runtime and Size Complexity of Integer Programs
% --> invariant generation through ranking functions
% [AliasDFG10] Multi-dimensional Rankings, Program Termination, and Complexity Bounds of Flowchart Programs
% --> invariant generation through ranking functions
% [Flores-MontoyaH14]Resource Analysis of Complex Programs with Cost Equations
% --> invariant generation through cost equations or ranking functions
%
% [GulwaniJK09]Control-flow Refinement and Progress Invariants for Bound Analysis
% --> program abstraction and invariant inference
% []Bound Analysis using Backward Symbolic Execution
% --> program abstraction and invariant inference
%
% [CicekBG0H17]relational Cost Analysis 0
% Monadic refinements for relational cost analysis
% [RajaniG0021]A unifying type-theory for higher-order (amortized) cost analysis
% --> type-system
Differently from all these previous works, we use a dependency graph with quantitative information needed to identify the length of chain of dependencies. Our weight estimation is inspired by  works in complexity analysis and WCET. 
Specifically, it is inspired by works on  reachability-bound analysis using program abstraction and invariant inference~\cite{GulwaniZ10, SinnZV17,GulwaniJK09} and work on invariant inference through cost equations and ranking functions~\cite{BrockschmidtEFFG16,AlbertAGP08,AliasDFG10,Flores-MontoyaH14}.
% The techniques are based on
% type system~\cite{CicekBG0H17, RajaniG0021}, Hoare logic~\cite{CarbonneauxHS15}, abstract interpretation~\cite{GustafssonEL05, HumenbergerJK18},
% i
% or a combination of
% In general, these techniques give the approximated upper bound of the program's total running time or resource cost.
% However, they failed to consider the case where the cost -- the adaptivity-- could decrease when there isn't a dependency relation between variables.


\paragraph{Generalization in Adaptive Data Analysis}
Starting from the works by \citet{DworkFHPRR15} and \citet{HardtU14}, several works have designed methods that ensure generalization for adaptive data analyses~\cite{dwork2015reusable,dwork2015generalization,BassilyNSSSU16,UllmanSNSS18,FeldmanS17,jung2019new,SteinkeZ20,RogersRSSTW20}.
Several of these works drew inspiration from differential privacy, a notion of formal data privacy. By limiting the influence that an individual can have on the result of a data analysis, even in adaptive settings, differential privacy can also be used to limit the influence that a specific data sample can have on the statistical validity of a data analysis. This connection is actually in two directions, as discussed for example by \citet{YeomGFJ18}.
%
Considering this connection between generalization and privacy, it is not surprising that some of the works on programming language techniques for privacy-preserving data analysis are related to our work. 
Adaptive Fuzz~\cite{Winograd-CortHR17} is a programming framework for differential privacy that is designed around the concept of adaptivity. 
This framework is based on a typed functional language that distinguish between several forms of adaptive and non-adaptive composition theorem with the goal of achieving better upper bounds on the privacy cost. Adaptive Fuzz uses a type system and some partial evaluation to guarantee that the programs respect differential privacy. However, it does not include any technique to bound the number of rounds of adaptivity. 
\citet{lobo2021programming} propose a language for differential privacy where one can reason about the accuracy of programs in terms of confidence intervals on the error that the use of differential privacy can generate. These are akin to bounds on the generalization error. This language is based on a static analysis which however cannot handle adaptivity. 
%
The way we formalize the access to the data mediated by a mechanism is a reminiscence of how the interaction with an oracle is modeled in the verification of security properties. As an example, the recent works by \citet{BarbosaBGKS21} and \citet{AguirreBGGKS21} use different techniques to track the number of accesses to an oracle. However, reasoning about the number of accesses is easier than estimating the adaptivity of these calls, as we do instead here.

%\cite{SatoABGGH19}

% together with guarantees on their accuracy? 
% {The first important application of the differential privacy concept started from the work by \cite{DworkFHPRR15}}, which applied this concept into guaranteeing the generalization error of adaptive data analysis. 

% Previous works on reducing the risk of spurious scientific discoveries are under the assumption that a fixed collection of learning algorithms to be applied are selected non-adaptively before seeing the data. In contrast with them, they developed this work under the adaptive data analysis settings. They formalized the generalization error for adaptive data analysis and then presented their validation guarantees.
% Concretely, they proposed the famous transfer theorem.
% And based on this theorem, they proved high probabilistic bounds on the generalization error for $\epsilon$- and $(\epsilon,\delta)$-differentially private adaptive data analysis as well as adaptive analysis with statistic and non-statistic queries.
% These works connected the theory with the practice of data analysis, which in my perspective, is the most significant and interesting contribution of this paper. 
% %
% % At the end, they presented the application of applying concrete differentially private techniques into adaptive data analysis.

% {This extension from differential privacy to adaptive data analysis made significant progress in reducing the overfitting risks in practical works (i.e. the data analysis in adaptive setting). Further works on improving these probabilistic bounds, guaranteeing the generalization error such as \cite{dwork2015generalization}, \cite{BassilyNSSSU16}, \cite{dwork2015reusable}, \cite{jung2019new} etc. are all influenced by this work.}

% {Following all previous works on preserving the statistical validity in adaptive data analysis, \cite{smith2017information} made a survey.}
% This survey started from giving formal and clear introduction to the concept of adaptive data analysis.
%  % by giving formal definitions of the concepts and clear representations and notations. 
% Then, it summarized the probability bounds on the generalization error w.r.t. the true population when applying different mechanisms in numeric adaptive data analysis.
% The mechanisms includes split data with adaptivity, adding Gaussian noise with specific standard derivation, adopting differentially private algorithms etc.
% Next, he extended the scope onto the non-numeric adaptive data analysis and presented the corresponding probabilistic bounds based on the information measures. 

% {This survey was developed in an easy-to-understand way and included a thorough knowledge on state-of-the-art works on adaptive data analysis, which helped me to sort out the results from existing works and relations between them.}

% {Following the same line of work, \cite{jung2019new} gave a new analysis on the role of differential privacy in adaptive data analysis.}
% They gave a substantially better probability bounds on differential privacy's generalization guarantee based on a new proof technique of the transfer theorem (initially from \cite{dwork2015generalization}). 

% The key point in their proof technique is looking into the posterior distributions, which is an insightful new perspective on proving the generalization error bound. This new perspective also brought a better understanding in the specific reason of analysis overfitting and the role of differential privacy in adaptive data analysis. This new technique I think will be fruitful in future work.
% %  based on my personal interests on the posterior distribution analysis
% % Another very meaningful structural insight inspired by this paper is the role of differential privacy and sample accuracy. This is pointed to the end of the paper that the sample accuracy serves to guarantee that the reported answers are close to their posterior means and differential privacy serves to guarantee that the posterior means are close to their true answers.

% % There are also some interesting works on further improving the accuracy bound unresolved in this paper, such as replace the Markov-like dependence with a Chernoff-like dependence. I'm deeply interested in making contributions on it.

% % {Based on all the excellent theory works on guaranteeing the statistical validity of adaptive data analysis, I started to think from the perspective of the programming language.}
% {Existing works on adaptive data analysis are trying to improve the probabilistic bounds for generalization error in terms of the adaptive and non-adaptive queries numbers and size of the data sample on pen-and-paper proofs.
% However, we still cannot guarantee implementations of the corresponding algorithms adhere to this generalization error bounds.
% Given an arbitrary data analysis program, we are unable to tell its generalization error. So in my consideration, verifying the programs' generalization error would be a possible interesting research direction. Furthermore, since the size of the data sample can be determined by the input or the users, then the most interesting and challenging point would be figuring out the program's adaptive query numbers. 
% This motivated us to look into the verification of algorithms' adaptivity, in order to formally verifiy their generalization error.}

% \paragraph{Data analysis} There is a significant amount of work on programming for data analysis. Many popular platforms are based on the R language\cite{ihaka1996r, marcon2021orchestrating}. Jaql is a declarative scripting language for large-scale data analysis\cite{beyer2011jaql}. 


% Program Analysis in terms of dependency graph:


% \subsection{Dynamic Program Graph Analysis}

% \cite{sinha2001interprocedural}: 
% Support Interprocedural Control dependence analyzing, semantically.
% \\
% They identified the dependence information between the interactions of among procedures, specifically the control dependence between procedures.
% Their analysis support the relationship of control and data dependence to semantics dependence.

% % \cite{austin1992dynamic}: Dynamic Program Dependency Graph.
% % \\
% % They gave the dynamic analysis for the program's dependency, by producing 3 different kinds of graph, 
% % including the data flow graph, storage dependence and control dependence graph from program's execution traces. 
% % \\
% % Then, they constructing dynamic execution graphs by adopting the 3 graph, aims to expose the parallization of the programs

% \cite{hammer2006dynamic}: dynamic path conditions in dependence graphs.
% They adopting the dynamic information from program trace to the path condition in dependency graph. Then based on these information, 
% they present new approach combining dynamic slicing, which could reveal both dependences holding during program execution as well as why these dependences are holding. 
% Aims to have a finer and preciser analysis of the program.

% % \subsection{Utilization of the Dynamic Program Dependency}

% % \cite{nagar2018automated}: Utilize dependency graph for finding serializability violation. 
% % \\
% % Combine with the dependency graph of serialization and abstract execution, to statically finding bounded serializability violation. 
% % Then reduce the problem of serializability to satisfiability of a formula in FOL.
% % Also reason about unbounded executions.

% \subsection{Static Program Dependency}

% \cite{mastroeni2008data}: They propose ways of constructing different kinds of program slices, by choose different program dependency. For example, in either syntactic or semantics sense.
% This abstract dependency is based on properties rather than exact data.
% Aims to give finer and smaller program slice. 

% \subsection{Utilization Static Program Flow Graph}

% \cite{arlt2012lightweight}: Lightweight Static Analysis for GUI Testing. They give the relevant event graph based on black and white Box.
% To construct finer Event Sequence Graph, 
% they propose new approach to select relevant event sequences among the event sequences generated by black box.
% This new approach based on static analysis on bytecode of the applications, 
% giving a precisely defined dependency between a fixed number of events in event sequence.
% Then, they inferred a finer Event Dependency graph, aims to give a better lightweight static analysis on applications.



\section{Conclusion and future works}
We presented {\THESYSTEM}, a program analysis useful to provide an upper bound on the adaptivity of a data analysis under a specific data analysis model. This estimation can help data analysts to control the generalization errors of their analyses by choosing different algorithmic techniques based on the adaptivity. Besides, a key contribution of our works is the formalization of the notion of adaptivity for adaptive data analysis. 

In future work, we plan to address some of the limitations of {\THESYSTEM}. Our algorithm may over-estimate the adaptivity of a program, as shown in Section~\ref{sec:examples}, due to its path-insensitive nature. We plan in future work to explore the possibility of making {\THESYSTEM} path-sensitive. While we believe that in many concrete situations in data analysis requiring a concrete bound for loops is not a strong limitation, we also plan to explore how to add support for dynamic or unbounded loops. To extend our work in this direction we plan to use classical abstraction techniques, at the cost of a more imprecise estimation.  

%%
%% The acknowledgments section is defined using the "acks" environment
%% (and NOT an unnumbered section). This ensures the proper
%% identification of the section in the article metadata, and the
%% consistent spelling of the heading.
% \begin{acks}
% To Robert, for the bagels and explaining CMYK and color spaces.
% \end{acks}

%%
%% The next two lines define the bibliography style to be used, and
%% the bibliography file.
\bibliographystyle{ACM-Reference-Format}
\bibliography{main.bib}


%%
%% If your work has an appendix, this is the place to put it.
% \appendix

% \section{Research Methods}

% \subsection{Part One}

% Lorem ipsum dolor sit amet, consectetur adipiscing elit. Morbi
% malesuada, quam in pulvinar varius, metus nunc fermentum urna, id
% sollicitudin purus odio sit amet enim. Aliquam ullamcorper eu ipsum
% vel mollis. Curabitur quis dictum nisl. Phasellus vel semper risus, et
% lacinia dolor. Integer ultricies commodo sem nec semper.

% \subsection{Part Two}

% Etiam commodo feugiat nisl pulvinar pellentesque. Etiam auctor sodales
% ligula, non varius nibh pulvinar semper. Suspendisse nec lectus non
% ipsum convallis congue hendrerit vitae sapien. Donec at laoreet
% eros. Vivamus non purus placerat, scelerisque diam eu, cursus
% ante. Etiam aliquam tortor auctor efficitur mattis.

% \section{Online Resources}

% Nam id fermentum dui. Suspendisse sagittis tortor a nulla mollis, in
% pulvinar ex pretium. Sed interdum orci quis metus euismod, et sagittis
% enim maximus. Vestibulum gravida massa ut felis suscipit
% congue. Quisque mattis elit a risus ultrices commodo venenatis eget
% dui. Etiam sagittis eleifend elementum.

% Nam interdum magna at lectus dignissim, ac dignissim lorem
% rhoncus. Maecenas eu arcu ac neque placerat aliquam. Nunc pulvinar
% massa et mattis lacinia.

\end{document}
\endinput
%%
%% End of file `sample-acmsmall-conf.tex'.
