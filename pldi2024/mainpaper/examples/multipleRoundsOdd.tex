As our second example, we want to show a program where $\THESYSTEM$ is not precise because of its path-insensitive nature. 
    One such program is $\kw{multiRoundsOdd(k)}$ presented in Fig.~\ref{fig:multiRoundsOdd}(a). For this program, $\THESYSTEM$ over-approximate the adaptivity.
% The $\THESYSTEM$ comes across an over-approximation due to its path-insensitive nature. 
The problem witnessed by this example occurs when the control flow is more sophisticated than what the static analysis can handle. 
$\kw{multiRoundsOdd}(k)$
has adaptivity $1 + k$ and a  while loop with two paths.
In each iteration, the queries $\clabel{\assign{y}{\query(\chi[x])}}^{5}$
and $\clabel{\assign{p}{\query(\chi[x])}}^{6}$ are based on the results of previous queries stored in $x$,
similarly to Example~\ref{ex:multipleRounds}.
The difference is that only the query answer from $\clabel{\assign{y}{\query(\chi[x])}}^{5}$ in the first branch
is used in the query request command at line $7$, $\clabel{\assign{x}{\query(\chi(\ln(y)))} }^{7}$.
However, this branch is only executed in even iterations ($j = 0, 2, \cdots $).
From the semantics-based dependency graph in Figure~\ref{fig:multiRoundsOdd}(b),
% \footnote{We omit some edges which are constructed by the transition of $\flowsto$ relation
% for concise, but these edges exist
% in $\traceG(\kw{multiRoundsO(k)})$ because they can be constructed directly by $\vardep$ relation.
% These omissions don't affect the illustration and adaptivity computation.} 
the weight function $\lambda \trace_0 \st \env(\trace_0) \frac{k}{2}$ for the vertex $y^5$ counts the
number of times $\clabel{\assign{y}{\query(\chi[x])}}^{5}$ is evaluated during the program execution under an initial trace $\trace_0$, i.e., half of the initial value of $k$ from $\trace_0$.
However, {\THESYSTEM} fails to realize that all the odd iterations only execute the first branch
and that only even iterations execute the second branch. 
So it considers both branches for every iteration when estimating the adaptivity. 
In this sense, the weight estimated for $y^5$ and $p^6$ are both 
$k$ as in Figure~\ref{fig:multiRoundsOdd}(c).
As a result, {\THESYSTEM} computes $y^5  \to x^7  \to y^5  \to \cdots \to x^7 $
as the walk of maximal query length in Figure~\ref{fig:multiRoundsOdd}(c)
where each vertex is visited $k$ times, and so the estimated adaptivity is $1 + 2 * k$, instead of $1 + k$. 
%
{ \small
\begin{figure}
    % \vspace{-0.5cm}
\centering
    \begin{subfigure}{1.0\textwidth}
\centering
\small{
    \[
    %
    \begin{array}{l}
        \kw{multiRoundsOdd}(k) \triangleq \\
        \clabel{ \assign{j}{k}}^{0} ; 
        \clabel{ \assign{x}{\query(\chi[0])} }^{1} ; \\
            \ewhile ~ \clabel{j > 0}^{2} ~ \edo ~ \\
            \Big(
             \clabel{\assign{j}{j-1}}^{3} ;
             \eif(\clabel{j \% 2 == 0}^{4}, 
             \clabel{\assign{y}{\query(\chi[x])}}^{5}, 
             \clabel{\assign{p}{\query(\chi[x])}}^{6});                          
             \clabel{\assign{x}{\query(\chi(\ln(y)))} }^{7} \Big)
        \end{array}
    \]
}
\vspace{-0.4cm}
\caption{}
    \end{subfigure}
%
\begin{subfigure}{.48\textwidth}
    \begin{centering}
    \begin{tikzpicture}[scale=\textwidth/15cm,samples=200]
% Variables Initialization
\draw[] (9, 1) circle (0pt) node{{ $x^1: {}^{\lambda \trace_0 . 1}_{1}$}};
% Variables Inside the Loop
 \draw[] (0, 7) circle (0pt) node{{ $y^5: {}^{\lambda \trace_0 . \env(\trace_0) \lceil \frac{k}{2}\rceil}_{1} $}};
 \draw[] (0, 4) circle (0pt) node{{ $p^6: {}^{\lambda \trace_0 . \env(\trace_0) \lfloor \frac{k}{2}\rfloor }_{1} $}};
 \draw[] (0, 1) circle (0pt) node{{ $x^7: {}^{\lambda \trace_0 . \env(\trace_0) k }_{1}$}};
 % Counter Variables
 \draw[] (9, 7) circle (0pt) node {{$j^0: {}^{\lambda \trace_0 . 1}_{0}$}};
 \draw[] (9, 4) circle (0pt) node {{ $j^3: {}^{ \lambda \trace_0 . \env(\trace_0) k}_{0}$}};
 %
 % Value Dependency Edges:
 \draw[  -latex,]  (0, 3.5) -- (0, 1.5) ;
 \draw[  -Straight Barb] (6.5, 4.5) arc (150:-150:1);
 \draw[  -latex] (8, 4.5)  -- (8, 6.5) ;
%  \draw[  -Straight Barb] (1., 1.5) arc (120:-200:1);
 % Value Dependency Edges on Initial Values:
 %
\draw[ thick,  -latex, densely dotted] (-0.6, 1.5)  to  [out=-220,in=220]  (-0.5, 6.5);
\draw[ thick, -latex, densely dotted]  (0.5, 6.5) to  [out=-30,in=30] (0.6, 1.6) ;
 % Control Dependency
 \draw[ -latex] (2.0, 7)  -- (7, 6.5) ;
 \draw[ -latex] (2.0, 4)  -- (7, 6.5) ;
 \draw[ -latex] (2.0, 1)  -- (7, 6.5) ;
% Control Dependency
\draw[ -latex] (2.0, 7)  -- (7, 4) ;
\draw[ -latex] (2.0, 4)  -- (7, 4) ;
\draw[ -latex] (2.0, 1)  -- (7, 4) ;
% Edges produced by the transivity
\draw[ -latex] (2.0, 7)  -- (7, 1.5) ;
\draw[ -latex] (2.0, 4)  -- (7, 1.5) ;
\draw[  -latex,] (2.0, 1)  -- (7, 1.5) ;
\draw[ -latex ]  (0, 4.5) -- (0, 6.5) ;
\draw[ -latex ] (0.8, 7.5) arc (220:-100:1);
\draw[ -latex ] (1.2, 1.0) arc (120:-200:1);
\end{tikzpicture}
\vspace{-0.55cm}
\caption{}
    \end{centering}
    \end{subfigure}
    \begin{subfigure}{.48\textwidth}
        \begin{centering}
        \begin{tikzpicture}[scale=\textwidth/15cm,samples=200]
    % Variables Initialization
    \draw[] (5, 1) circle (0pt) node{{ $x^1: {}^1_{1}$}};
    % Variables Inside the Loop
     \draw[] (0, 7) circle (0pt) node{{ $y^5: {}^{k}_{1}$}};
     \draw[] (0, 4) circle (0pt) node{{ ${p^6: {}^{k}_{1}}$}};
     \draw[] (0, 1) circle (0pt) node{{ ${x^7: {}^{k}_{1}}$}};
     % Counter Variables
     \draw[] (5, 7) circle (0pt) node {{$j^0: {}^{1}_{0}$}};
     \draw[] (5, 4) circle (0pt) node {{ $j^3: {}^{k}_{0}$}};
     %
% Value Dependency Edges:
 \draw[  -latex,]  (0, 3.5) -- (0, 1.5) ;
 \draw[  -Straight Barb] (6.5, 4.5) arc (150:-150:1);
 \draw[  -latex] (5, 4.5)  -- (5, 6.5) ;
 % Value Dependency Edges on Initial Values:
 \draw[  -latex,] (1.5, 1)  -- (4, 1) ;
 %
    \draw[thick, -latex, densely dotted] (-0.6, 1.5)  to  [out=-220,in=220]  (-0.5, 6.5);
    \draw[thick, -latex, densely dotted]  (0.5, 6.5) to  [out=-30,in=30] (0.6, 1.6) ;
 % Control Dependency
 \draw[ -latex] (1.5, 7)  -- (4, 6) ;
 \draw[ -latex] (1.5, 4)  -- (4, 6) ;
 \draw[ -latex] (1.5, 1)  -- (4, 6) ;
% Control Dependency
\draw[ -latex] (1.5, 7)  -- (4, 4) ;
\draw[ -latex] (1.5, 4)  -- (4, 4) ;
\draw[ -latex] (1.5, 1)  -- (4, 4) ;
% Edges produced by the transivity
\draw[ -latex] (1.5, 7)  -- (4, 1) ;
\draw[ -latex] (1.5, 4)  -- (4, 1) ;
\draw[ -latex ]  (0, 4.5) -- (0, 6.5) ;
\draw[ -latex ] (0.8, 7.5) arc (220:-100:1);
\draw[ -latex ] (1.2, 1.0) arc (120:-200:1);
     \end{tikzpicture}
     \vspace{-0.55cm}
     \caption{}
        \end{centering}
        \end{subfigure}
        \vspace{-0.4cm}
\caption{(a) The multiple rounds odd example 
(b) The semantics-based dependency graph
(c) The estimated dependency graph from $\THESYSTEM$.}
    \label{fig:multiRoundsOdd}
    \vspace{-0.55cm}
\end{figure}
}
%
