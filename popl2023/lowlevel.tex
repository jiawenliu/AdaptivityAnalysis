In last section, we show the challenges to directly analyze programs in (labelled) loop language and introduce an approach to eliminate - towards single-static-assignment form.

\subsection{The limit of loop language for analysis}
we see the power of the labelled loop language to achieve the adaptivity semantically, from its being capable to express many adaptive data analysis algorithm,  allowing the construction of the query-based dependency graph using traces from the execution, and so on.
However, it is not powerful enough to reach the adaptivity syntactically. The main difficulty is its implicit control flow which raises extra complexity to figure out where some variables used come from. We use three simple but relevant examples to show why the loop language suffers. We use $q_1,q_2,q_3$ to represent three linear queries.

\[
 s_1 = \begin{array}{l}
      \clabel{ \assign{x}{q_1}}^{1} ; \\
      \eif  [(x < 0 )]^{2} \\
      \ethen \clabel{\assign{x}{q_2}}^{3}\\
      \eelse \clabel{\eskip}^{4} ; \\
      \clabel{\assign{y}{q(x+\chi(3))}}^{5}
 \end{array}
 ~~~~~
  s_2 = \begin{array}{l}
      \clabel{ \assign{x}{q_1}}^{1} ; \\
      \eif  [(x < 0 )]^{2} \\
      \ethen \clabel{\assign{x}{q_2}}^{3}\\
      \eelse \clabel{\assign{x}{q_3}}^{4} ; \\
      \clabel{\assign{y}{q(x+\chi(3))}}^{5}
 \end{array}
 ~~~~~~~~
  s_3 = \begin{array}{l}
      \clabel{ \assign{x}{q_1}}^{1} ; \\
      \eif  [(x < 0 )]^{2} \\
      \ethen \clabel{\assign{z}{q_2}}^{3}\\
      \eelse \clabel{\eskip}^{4} ; \\
      \clabel{\assign{y}{q(x+\chi(3)}}^{5}
 \end{array}
\]
In these three examples, the variable $x$ at line $5$ is implicit. In program $s_1$, it refers to the either $x$ at line $1$, or $x$ at line $3$, which means the result of query request $q(x+\chi(3))$ assigned to the variable $y$ may depend on $q_1$(bound to $x$ at line $1$) or $q_2$($x$ at line $3$). When we have a look at the other two programs $s_2$ and $s_3$, it is another talk. We think $q(x+\chi(3))$ may depend on either $q_2$($x$ at line $3$) or $q_3$($x$ at line $4$) in $s_2$,     while $q(x+\chi(3))$ only depends on $q_1$ at line $1$ in program $s_3$. These three examples are structural similar in loop language, however, the dependency between variables are quite dissimilar. We consider variables here because query request is also bound to variables. To solve this dilemma, we move to single static assignment as follows.   
\[
 s_1^{s} = \begin{array}{l}
      \clabel{ \assign{{\ssa{x_1}}}{q_1}}^{1} ; \\
      \eif  [({\ssa{x_1} }< 0 )]^{2}\\
      ([], [{ \ssa{x_3, x_1,x_2} }], []) \\
      \ethen \clabel{\assign{{\ssa{x_2}}}{q_2}}^{3}\\
      \eelse \clabel{\eskip}^{4} ; \\
      \clabel{\assign{{\ssa{y_1}}}{q({\ssa{x_3} + \chi(3)})}}^{5}
 \end{array}
 ~~~~~
  s_2^{s} = \begin{array}{l}
      \clabel{ \assign{{\ssa{x_1}}}{q_1}}^{1} ; \\
      \eif  [({\ssa{x_1}} < 0 )]^{2}, \\
      ( [{\ssa{x_4, x_2,x_3}}], [], [] ) \\
      \ethen \clabel{\assign{{\ssa{x_2}}}{q_2}}^{3}\\
      \eelse \clabel{\assign{{\ssa{x_3}}}{q_3}}^{4} ; \\
      \clabel{\assign{{\ssa{y_1}}}{q({\ssa{x_4}}+\chi(3))}}^{5}
 \end{array}
 ~~~~~~~~
  s_3^{s} = \begin{array}{l}
      \clabel{ \assign{{\ssa{x_1}}}{q_1}}^{1} ; \\
      \eif  [({\ssa{x_1}} < 0 )]^{2} \\
       ( [], [], [] ) \\
      \ethen \clabel{\assign{{\ssa{z_1}}}{q_2}}^{3}\\
      \eelse \clabel{\eskip}^{4} ; \\
      \clabel{\assign{{\ssa{y_1}}}{q({\ssa{x_1}}+\chi(3))}}^{5}
 \end{array}
\]
To distinguish between the loop language and in ssa form, we denote the ssa variable ${\ssa{x_1}}$ in bold. As we can see, the data flow becomes explicit in ssa form and the analysis on the dependency between variables in the program becomes much clear now. Considering this advantage, we aim to estimate the adaptivity through an analysis on program in ssa form. 


% \begin{enumerate}
%     \item Variable may be overwritten. Suppose $p = [\assign{x}{q_1()}]^{1}; [\assign{x}{q_2()}]^{2}; [\assign{y}{q_3(x)}]^{3} $. It increases the difficulty of static analysis to figure out the variable $x$ used at line $3$ inside $q_3(x)$ refers to the one at line $1$ or $2$.
%     \item Dependency of queries inside the loop body is hard to estimate.  
% \end{enumerate}

\subsection{ The ssa loop language }
We present the syntax of the ssa loop language, a language based on the loop language, representing programs in single static assignment form. We omit the standard parts inherited from the loop language and only focus on the syntax relevant to ssa.  
\[
\begin{array}{llll}
%  \mbox{Arithmatic Operators} & \oplus_a & ::= & + ~|~ - ~|~ \times 
% %
% ~|~ \div \\  
%   \mbox{Boolean Operators} & \oplus_b & ::= & \lor ~|~ \land ~|~ \neg\\
%   %
%   \mbox{Relational Operators} & \sim & ::= & < ~|~ \leq ~|~ == \\  
%  \mbox{Label} & l & := & \mathbb{N} \\ 
%  \mbox{While Map} & w & \in & \mbox{Label} \times \mathbb{N} \\
\mbox{SSA Arithmetic Expressions} & \ssa{\aexpr} & ::= & 
	%
	{n} ~|~ {\ssa{x}} ~|~ \chi ~|~ \ssa{\aexpr} \oplus_a \ssa{\aexpr} ~|~ [] ~|~ [\ssa{\aexpr}, \dots, \ssa{\aexpr}] \\
% 	\sep \pi (l , \aexpr, \aexpr) \\
    %
\mbox{SSA Boolean Expressions} & \ssa{\bexpr} & ::= & 
	%
	\textrm{true} ~|~ \textrm{false}  ~|~ \neg \ssa{\bexpr}
	 ~|~ \ssa{\bexpr} \oplus_b \ssa{\bexpr}
	%
	~|~ \ssa{\aexpr} \sim \ssa{\aexpr} \\
\mbox{SSA Expressions} & \ssa{\expr} & ::= & \ssa{\aexpr} \sep \ssa{\bexpr} \\	
\mbox{SSA Labelled commands} & \ssa{c} & ::= &   [\assign {\ssa{x}}{ \ssa{\expr}}]^{l} ~|~  [\assign {{\ssa{x}} } {q({\ssa{e}})}]^{l}
%
% ~|~ [\eswitch( \ssa{\expr}, \ssa{x}, \ssa{v_i} \to \ssa{ q_i})]^{l} 
~|~  {{ifvar(\bar{\ssa{x}}, \bar{\ssa{x}}')}}  ~|~ [\eskip]^{l}  \\ 
&&& [\eloop ~ {\ssa{\aexpr}}, {n},  [\bar{\ssa{x}}, \bar{\ssa{x_1}}, \bar{\ssa{x_2}}] ~ \edo ~ {\ssa{c}} ]^{l} ~|~ \ssa{c};\ssa{c}  ~|~ \\ &&& [\eif(\ssa{\bexpr}, ([\bar{\ssa{x}}, \bar{\ssa{x_1}}, \bar{\ssa{x_2}}] , [\bar{\ssa{y}}, \bar{\ssa{y_1}}, \bar{\ssa{y_2}}],[\bar{\ssa{z}}, \bar{\ssa{z_1}}, \bar{\ssa{z_2}}] ) , \ssa{c}, \ssa{c})]^l 	
	\\
\mbox{SSA Memory} & \ssa{m} & ::= & \emptyset ~|~ { (\ssa{x} \to v) :: \ssa{m} } \\
%
% \mbox{Trace} & t & ::= & [] ~|~ \ssa{({q(v)^{(l, w) }) :: t}} \\
% \mbox{Annotated Query} & \mathcal{AQ}  & ::= & \{ q(v)^{(l,w)}  \} \\
\mbox{SSA Variables} & \mathcal{SV}  & ::= & \{ \ssa{x} \} \\
\mbox{Annotated SSA Variables} & \mathcal{LV}  & ::= & \{ \ssa{x}^{(l,w)}  \}
\end{array}
\]
We use $\ssa{\aexpr}$ to express arithmetic expressions which now contains ssa variable $\ssa{x} \in \mathcal{SV}$, and the boolean expression as $\ssa{\bexpr}$. The ssa expression can be either $\ssa{a}$ and $\ssa{b}$. We also have the ssa variables annotated in a similar way as the annotated queries in the loop language, denoted as $\mathcal{LV}$. The labelled commands $\ssa{c}$ are now in the ssa form. In the assignment $[\assign {\ssa{x}}{ \ssa{\expr}}]^{l}$,  query request $[\assign {{\ssa{x}} } {q({\ssa{e}})}]^{l}$ , the expression ${\ssa{\expr}}$ is now in the ssa form, similar for the $\ssa{\aexpr}$ in the loop and conditional $\ssa{\bexpr}$ in the if command. The if command now contains the extra part $([\bar{\ssa{x}}, \bar{\ssa{x_1}}, \bar{\ssa{x_2}}] , [\bar{\ssa{y}}, \bar{\ssa{y_1}}, \bar{\ssa{y_2}}],[\bar{\ssa{z}}, \bar{\ssa{z_1}}, \bar{\ssa{z_2}}] )$, which helps to track the dependency of new assigned variables in both branches($[\bar{\ssa{x}}, \bar{\ssa{x_1}}, \bar{\ssa{x_2}}]$), then branch $[\bar{\ssa{y}}, \bar{\ssa{y_1}}, \bar{\ssa{y_2}}]$, and else branch $[\bar{\ssa{z}}, \bar{\ssa{z_1}}, \bar{\ssa{z_2}}] $. The $\bar{\ssa{x}}$ is a list of ssa variables, in which every element $\ssa{x}$ may depends on the corresponding element $\ssa{x_1}$ from $\bar{\ssa{x_1}}$ collected in the then branch or the corresponding element $\ssa{x_2}$ from $\bar{\ssa{x_2}}$ collected in the else branch. Every tuple $(\ssa{x,x_1,x_2 })$ from $[\bar{\ssa{x}}, \bar{\ssa{x_1}}, \bar{\ssa{x_2}}]$ can be understood as $\ssa{x} = \phi(\ssa{x_1,x_2})$ in the normal ssa form. The previous example $s_2^{s}$ can be used for reference. The second part $[\bar{\ssa{y}}, \bar{\ssa{y_1}}, \bar{\ssa{y_2}}]$ focuses on the then branch. The list of ssa variables $y_1$ stores the assigned ssa variables before the if command, whose non-ssa version (variables in the loo language) will be modified only in the then branch. We can look at program $s_1$ as a reference,in which $x$ at line $1$ may be modified only in the then branch at line $3$. The list $\bar{\ssa{y_2}}$ tracks the ssa variables assigned only in the then branch. If the variables are assigned in both branches such as in the program $s_2$, they goes into $[\bar{\ssa{x}}, \bar{\ssa{x_1}}, \bar{\ssa{x_2}}]$. Then we think every ssa variable in $\bar{\ssa{y}}$ may come from the corresponding variable $\ssa{y_1}$ in $\bar{\ssa{y_1}}$ before the if command or $\ssa{y_2}$ in $\bar{\ssa{y_2}}$ in the then branch. In this sense, we can also regard every tuple $(\ssa{y,y_1,y_2 })$ from $[\bar{\ssa{y}}, \bar{\ssa{y_1}}, \bar{\ssa{y_2}}]$ as $\ssa{y} = \phi(\ssa{y_1,y_2})$.  The rest part $[\bar{\ssa{z}}, \bar{\ssa{z_1}}, \bar{\ssa{z_2}}]$ focus on the else branch and can be understood similarly. Also, the loop command also has similar part $ [\bar{\ssa{x}}, \bar{\ssa{x_1}}, \bar{\ssa{x_2}}]$, focusing on the loop body. The new command ${{ifvar(\bar{\ssa{x}}, \bar{\ssa{x}}')}}$ does not have explicit label because it is only used for evaluation internally, we will discuss more about it when used in the operational semantics for the ssa loop language. The ssa memory $\ssa{m}$ is a map from ssa variable $\ssa{x}$ to values.

We show the example of the complete version of two round algorithm $TRC^{ssa}$ in the ssa language in Figure~\ref{fig:tworound_complete}.

\subsection{Trace-based Operational Semantics of ssa loop language}
When switching to the ssa loop language, we show that we are still be able to achieve what we can get in Section~\ref{sec:loop_language}. The operational semantics of the ssa loop language mimics its counterpart, of the form $\config{\ssa{m}, \ssa{c}, t, w} \to \config{\ssa{m'}, \eskip, t', w'}$. It still uses a trace to track the query requests during the execution, starting from the ssa configuration with ssa memory $\ssa{m}$ and programs in its ssa form $\ssa{c}$, which allows a similar construction of the query-based dependency graph in the ssa language as its counterpart. We choose selected evaluation rules in Figure~\ref{fig:ssa_evaluation}.

\begin{figure}
    \begin{mathpar}
% \boxed{\config{\ssa{m, a}} \xrightarrow{} \config{\ssa{a}} \;} 
% \inferrule{
%     \ssa{m}(\ssa{x}) = \ssa{v}
% }{ \config{\ssa{m, x }} \to \ssa{v} 
% }~{\textbf{SSA-VAR}}
% \and
\boxed{ \config{\ssa{ m, c}, t,w} \xrightarrow{} \config{\ssa{ m', c'},  t', w'} \; }
\and
% { Memory \times Com  \times Trace \times WhileMap \Rightarrow^{}  Memory \times Com  \times Trace \times WhileMap}
% \and
\inferrule
{
 \config{\ssa{m, \expr} } \to \ssa{\expr'}
}
{
\config{ \ssa{m}, [\assign{{\ssa{x}}}{{q(\ssa{\expr})}}]^l, t, w} \xrightarrow{} \config{ \ssa{m}, [\assign{{\ssa{x}}}{{q(\ssa{\expr'})}}]^l, t, w}
}
~\textbf{ssa-query-arg}
%
\and
%
\inferrule
{
{q(v) = v_q} 
}
{
\config{ \ssa{m}, [\assign{{\ssa{x}}}{{q(v)}}]^l, t, w} \xrightarrow{} \config{\ssa{  m}[ v_q / \ssa{ x} ], \eskip,  t \mathrel{++} [q(v)^{(l,w )}],w  }
}
~\textbf{ssa-query}
% %
% \and
% %
% \inferrule
% {
% }
% {
% \config{\ssa{ m, [\assign x v]^{l},  t,w} } \xrightarrow{} \config{\ssa{m[x \mapsto v], [\eskip]^{l}, t,w} }
% }
% ~\textbf{ssa-assn}
% %
% \and
% %
% \inferrule
% {
% \config{m, c_1,  t,w} \xrightarrow{} \config{m', c_1',  t',w}
% }
% {
% \config{m, c_1; c_2,  t,w} \xrightarrow{} \config{m', c_1'; c_2, t',w}
% }
% ~\textbf{seq1}
% %
% \and
% %
% \inferrule
% {
% }
% {
% \config{m, [\eskip]^{l} ; c_2,  t,w} \xrightarrow{} \config{m, c_2,  t,w}
% }
% ~\textbf{seq2}
% %
% \and
% %
% \inferrule
% {
% \config{ \ssa{m, \bexpr}} \barrow \ssa{\bexpr'}
% }
% {
% \config {\ssa{m, [\eif(\bexpr, [\bar{{x}}, \bar{{x_1}}, \bar{{x_2}}] ,[\bar{\ssa{y}}, \bar{\ssa{y_1}}, \bar{\ssa{y_2}}] ,[\bar{\ssa{z}},\bar{\ssa{z_1}}, \bar{\ssa{z_2}}] , c_1, c_2)]^{l},  t,w} } 
% \xrightarrow{} \config{ \ssa{ m,  [\eif(\ssa{bexpr'},[ \bar{{x}}, \bar{{x_1}}, \bar{{x_2}}] ,[\bar{\ssa{y}}, \bar{\ssa{y_1}}, \bar{\ssa{y_2}}] ,[\bar{\ssa{z}},\bar{\ssa{z_1}}, \bar{\ssa{z_2}}] , c_1, c_2)]^{l},  t, w} }
% }
% ~\textbf{ssa-if}
%
\and
%
\inferrule
{
}
{
\config{\ssa{m, [\eif(\etrue, [\bar{\ssa{x}}, \bar{\ssa{x_1}}, \bar{\ssa{x_2}}] ,[\bar{\ssa{y}}, \bar{\ssa{y_1}}, \bar{\ssa{y_2}}] ,[\bar{\ssa{z}}, \bar{\ssa{z_1}}, \bar{\ssa{z_2}}] , c_1, c_2)]^{l}},t,w} \\
\xrightarrow{} \config{\ssa{m, c_1}; { \ ifvar(\bar{\ssa{x}},\bar{\ssa{x_1}}); ifvar(\bar{\ssa{y}},\bar{\ssa{y_2}});ifvar(\bar{\ssa{z}},\bar{\ssa{z_1}}) }  ,  t,w}
}
~\textbf{ssa-if-t}
%
\and
%
\inferrule
{
}
{
\config{\ssa{m, [\eif(\efalse, [\bar{\ssa{x}}, \bar{\ssa{x_1}}, \bar{\ssa{x_2}}] ,[\bar{\ssa{y}}, \bar{\ssa{y_1}}, \bar{\ssa{y_2}}] ,[\bar{\ssa{z}},\bar{\ssa{z_1}}, \bar{\ssa{z_2}}] , c_1, c_2)]^{l}},t,w} 
\\
\xrightarrow{} \config{\ssa{m, c_2} ; { ifvar(\bar{\ssa{x}},\bar{\ssa{x_2}}); ifvar(\bar{\ssa{y}},\bar{\ssa{y_1}});ifvar(\bar{\ssa{z}},\bar{\ssa{z_2}}) },  t,w}
}
~\textbf{ssa-if-f}
% %
% \and
% %
% {
% \inferrule
% {
% \empty
% }
% {
% \config{\ssa{m}, [ \eswitch(\ssa{v_k},\ssa{x}, (\ssa{v_i} \to q_i))]^{l},  t,w} 
% \xrightarrow{} \config{\ssa{m},  [\assign {\ssa{x}}{ q_k}]^{l},  t, w }
% }
% ~\textbf{ssa-switch-v}
% }
%
\and
%
\inferrule{
}{
 \config{ \ssa{m}, ifvar(\ssa{\bar{x}, \bar{x}'}),{ t,w }} \to \config{ \ssa{m [  m(\bar{x}')/ \bar{x}], \eskip , t,w }  }
}~\textbf{ssa-ifvar}
% %
% \and
% %
% {\inferrule
% {
% }
% {
% \config{m, \ewhile([\bexpr]^l, c),  t,w} 
% \xrightarrow{} \config{m,  \eunfold{[\bexpr^{l}] }{\ewhile([\bexpr]^l,   c)} ,  t,w}
% }
% ~\textbf{while} }
% %
% \and
% %
% {\inferrule
% {
% \config{m, \bexpr} \rightarrow \bexpr'
% }
% {
% \config{m, \eunfold{[\bexpr]^l}{ c}, D, t,w} 
% \xrightarrow{} \config{m, \eunfold{[\bexpr']^l}{ c}, D, t,  w  }
% }
% ~\textbf{unfold}}
% %
% \and
% %
% {\inferrule
% {
% }
% {
% \config{m, \eunfold{[\efalse]^l}{c}, D, t,w} 
% \xrightarrow{} \config{m, [\eskip]^{l}, D, t,  (w \setminus l) }
% }
% ~\textbf{unfold-f}}

% \and
% %
% {\inferrule
% {
% }
% {
% \config{m, \eunfold{[\etrue]^l}{ c}, D, t,w} 
% \xrightarrow{} \config{m, c, D, t, (w+l) }
% }
% ~\textbf{unfold-t} }
% %
% \and
% %
% {
% \inferrule
% {
%   \config{m, \expr } \xrightarrow{} \expr'
% }
% {
% \config{m, [\eswitch(\expr, x, (v_i \to q_i))]^{l},  t,w} 
% \xrightarrow{} \config{m, [ \eswitch(\expr',x, (v_i \to q_i))]^{l},  t, w }
% }
% ~\textbf{switch}
% }
% %
% \and
% %
% {
% \inferrule
% {
% \empty
% }
% {
% \config{m, [ \eswitch(v_k,x, (v_i \to q_i))]^{l},  t,w} 
% \xrightarrow{} \config{m,  [\assign x q_k]^{l},  t, w }
% }
% ~\textbf{switch-v}
% }
%
\and
%
{\inferrule
{
 {{ \valr_N > 0} }\and 
 { {n} = 0 \implies i =1 } \and
 { {n} > 0 \implies i =2 }
}
{
\config{ \ssa{m},  [\eloop ~ \valr_N, n, [\bar{\ssa{x}}, \bar{\ssa{x_1}}, \bar{\ssa{x_2}}] ~ \edo ~ \ssa{c} ]^{l}  ,  t, w } \\
\xrightarrow{} \config{\ssa{ m, c[\bar{x_i} /  \bar{x}   ]};  [\eloop ~ (\valr_N-1), n+1, [\bar{\ssa{x}}, \bar{\ssa{x_1}}, \bar{\ssa{x_2}}] ~ \edo ~ \ssa{c}]^{l} ,  t, (w + l)  }
}
~\textbf{ssa-loop}
}
%
\and
%
{
\inferrule
{
 \valr_N = 0 \and
 { {n} = 0 \implies i =1 } \and
 { {n} > 0 \implies i =2 }
}
{
\config{\ssa{m},  [\eloop ~ \valr_N, n, [\bar{\ssa{x}}, \bar{\ssa{x_1}}, \bar{\ssa{x_2}}] ~ \edo ~ \ssa{c} ]^{l}  ,  t, w }
\xrightarrow{} \config{\ssa{ m[m(\bar{x_i})/\bar{x} ]}, [\eskip]^{l} ,  t, (w \setminus l)  }
}
~\textbf{ssa-loop-exit}
}
%
\end{mathpar}
    \caption{Operational semantics for the ssa loop language}
    \label{fig:ssa_evaluation}
\end{figure}

The key idea underneath the operational semantics is to have the trace and the execution path being constructed in a similar way as in the loop language. Take the query request as an example, the argument $\ssa{e}$ which contains ssa variables will be evaluated to a value $v$ first before the request is sent to the database in rule $\textbf{ssa-query-arg}$. The trace expands in the rule $\textbf{ssa-query}$ likewise in the loop language. The query $q$, a primitive symbol representing the abstract query in both the ssa language and  the loop language, makes no difference in the two languages. Since we add the extra part $[\bar{\ssa{x}}, \bar{\ssa{x_1}}, \bar{\ssa{x_2}}] ,[\bar{\ssa{y}}, \bar{\ssa{y_1}}, \bar{\ssa{y_2}}] ,[\bar{\ssa{z}},\bar{\ssa{z_1}}, \bar{\ssa{z_2}}]  $ in the if command compared to its counterpart in the loop language, the rules relevant to the if condition ($\textbf{ssa-if-t}$ and $\textbf{ssa-if-f}$) use the command $ifvar(\ssa{\bar{x}, \bar{x}'})$ to update the ssa memory $\ssa{m}$ with the mapping from the new generated variable $\ssa{x}$ in $\bar{\ssa{x}}$ to the appropriate value $\ssa{m(x')}$ where $\ssa{x'}$ is the corresponding variable w.r.t $\ssa{x}$ in $\bar{\ssa{x'}}$. The rule $\textbf{ssa-ifvar}$ reflects the usage of $ifvar(\ssa{\bar{x}, \bar{x}'})$. It is easier to understand the usage of $ifvar(\ssa{\bar{x}, \bar{x}'})$ in the rule $\textbf{ssa-if-t}$ when we think about how ssa works: in the ssa form, when a variable to be used may come from two sources (e.g. $\ssa{x_1}$ and $\ssa{x_2}$ in the rule), it generates a new variable $\ssa{x}$, assigning it with $\phi(\ssa{x_1}, \ssa{x_2})$,  and replaces the variable to be used with newly assigned $\ssa{x}$. We know that in the future program after the if command, only the variables $\bar{\ssa{x}}$ will be available instead of   $\bar{\ssa{x_1}}, \bar{\ssa{x_2}}$ from two branches. For the evaluation of the program after the if command, we need to tell the memory the exact value of the newly generated variable $\ssa{x}$, which is the value stored in $\ssa{x_1}$ when the conditional $\ssa{b}$ is true, or the value in $\ssa{x_2}$ when $\ssa{b}$ is false. To this end, the internal command $ifvar(\ssa{\bar{x}, \bar{x}'})$ plays its role. For the if rule, we need to instantiate the variables from $\bar{\ssa{x}}$ whose values come from two branches, $\bar{\ssa{y}}$ whose values from then branch or assignment before the if command, and $\bar{\ssa{z}}$ whose values from else branch or before the if command. Correspondingly, we need to have three ifvar commands.   

The evaluation of loop depends on the loop counter $\ssa{\aexpr}$, which will be evaluated to a value $v_N$. When $v_N$ is greater than 0, the loop is still executing, and all the variables $\ssa{x}$ in $\bar{\ssa{x}}$ of the loop body $\ssa{c}$ are replaced as the corresponding variables in $\bar{\ssa{x_1}}$ in the first iteration($n=0$), or $\bar{\ssa{x_2}}$ in other iterations($n > 0$). The loop turns to an exit when $v_N > 0$, and the memory $\ssa{m}$ updates the mapping of variables in $\bar{\ssa{x}}$ with $\bar{\ssa{x_1}}$ if the iteration counter $n=0$, which means the loop body is not executed once. When the loop enters the exit after executing the body a few times($n$), the variables in $\bar{\ssa{x}}$ is instantiated with the value from the body $\ssa{m}(\bar{\ssa{x_2}})$. 

With the trace-based operational semantics of the ssa language at hand, we are able to provide our query-based dependency graph in the ssa language.

\begin{defn}[Query may dependency in ssa ]
One query ${q(v_1)}$ may depend on another query ${q(v_2)}$ in a program $\ssa{c}$, with a starting while map $w$, denoted as
$\mathsf{DEP_{ssa}}({q(v_1)}^{(l_1, w_1)}, {q(v_2)}^{(l_2, w_2)}, \ssa{c},w, \ssa{m},D)$ is defined as below. 
\[
  \begin{array}{l}
     \forall t. \exists \ssa{m_1,m_3},t_1,t_3. 
\config{\ssa{m, c},  t,w} \rightarrow^{*} \config{\ssa{m_1}, [\assign{\ssa{x}}{q(v_1)}]^{l_1} ; \ssa{c_2},
  t_1,w_1} \rightarrow \\ \config{\ssa{m_1}[q(v_1)(D)/\ssa{x}], \ssa{c_2},
  t_1++[q(v_1)^{(l_1, w_1)}], w_1} \rightarrow^{*} \config{\ssa{m_3}, \eskip,
  t_3,w_3} \\  
  \land 
\Big( q(v_1)^{(l_1,w_1)} \in (t_3-t) \land q(v_2)^{(l_2,w_2)} \in (t_3-t) \implies  \exists v \in \codom(q(v_1)),\ssa{m_3'}, t_3', w_3'. \\
 \config{\ssa{m_1}[v/\ssa{x}], \ssa{c_2}, t_1++[(q(v_1)^{(l_1,w_1)})], w_1} \rightarrow^{*} \config{\ssa{m_3'}, \eskip, t_3', w_3'} \land (q(v_2)^{(l_2,w_2)}) \not \in (t_3'-t)
\Big)\\
\land 
\Big(q(v_1)^{(l_1,w_1)} \in (t_3-t) \land q(v_2)^{(l_2,w_2)} \not\in (t_3-t) \implies  \exists v \in \codom(q(v_1)),\ssa{m_3'}, t_3', w_3'. \\
 \config{\ssa{m_1}[v/\ssa{x}], {c_2}, t_1++[(q(v_1)^{(l_1,w_1)})], w_1} \rightarrow^{*} \config{\ssa{m_3'}, \eskip, t_3', w_3'} \land (q(v_2)^{(l_2,w_2)})  \in (t_3'-t)
\Big)\\
\end{array}
\]
\end{defn}



\begin{defn}
Dependency Graph in ssa.
\\
Given a program $\ssa{c}$, a database $D$, a starting memory $\ssa{m}$, an initial whilemap $w$, the dependency graph $G_{s}(\ssa{c},D,\ssa{m},w) = (V, E)$ is defined as: \\
$V =\{q(v)^{l,w} \in \mathcal{AQ} \mid \forall t. \exists \ssa{m'},  w', t'.  \config{\ssa{m} ,\ssa{c}, t, w}  \to^{*}  \config{\ssa{m'} , \eskip, t', w' }  \land q(v)^{l,w} \in {(t'-t)}  \}$.
\\
$E = \left\{(q(v)^{(l,w)},q(v')^{(l',w')}) \in \mathcal{AQ} \times \mathcal{AQ} 
~ \left \vert ~ \begin{array}{l}
  \mathsf{DEP_{ssa}}(q(v)^{(l,w)},q(v')^{(l',w')}, \ssa{c},w,\ssa{m},D)     \\
  \land \mathsf{To}(q'(v')^{(l',w')}, q(v)^{(l,w)}    
\end{array} \right. 
\right\}$.
\end{defn}

\subsection{Transformation }
We build a bridge between the two languages through a transformation in spirit of the work \cite{vekris2016refinement}. The command transformation of the form $ \Sigma; \delta ; c  \hookrightarrow \ssa{c} ; \delta' ; \Sigma'$ translates the labelled command $c$ in the loop language to its counterpart in ssa language. The key idea beneath single static assignment form is to give every assignment an unique variable. To this end, the SSA name environment $\Sigma$, a set of ssa variables already used before the transformation process, is used to generate a fresh ssa variable via a function $fresh(\Sigma)$. Additionally, translating variables read in the program in the loop language to its unique ssa variable requires a translation environment $\delta$, a map from variable $x \in \mathcal{VAR}$ to its ssa form $\ssa{x} \in \mathcal{SV}$. Also, the translation environment $\delta$ and the ssa name environment $\Sigma$ will be updated to $\delta'$ and $\Sigma'$ respectively, along with the transformation of the target command. The expression is much simpler, of the form $ \delta; \expr \hookrightarrow \ssa{\expr}$, which transforms the variables in $\expr$ to ssa variables stored in the translation environment $\delta$, shown in the rule $\textbf{S-VAR}$.

We present selected transformation rules in Figure~\ref{fig:trans_rules}. The rules $\textbf{S-ASSN}$ and $\textbf{S-QUERY}$ both use $fresh(\Sigma)$ to generate a new fresh variable $\ssa{x}$ to guarantee unique assignment in its ssa form. The translation environment is updated with the mapping from $x$ to the new generated ssa variable $\ssa{x}$ for reference to $x$ used in the future. The ssa name environment is also modified by recording the new variable $\ssa{x}$. The transformation of sequence is standard, with both environments $\delta$ and $\Sigma$ updated during the transformation procedure.   

We look at the rule $\textbf{S-IF}$ by first introducing the binary operation $\bowtie$ on two translation environments $\delta_1$ and $\delta_2$. 
\[ \delta_1 \bowtie \delta_2 = \{ ( x, {\ssa{x_1}, \ssa{x_2}} ) \in \mathcal{VAR} \times \mathcal{SV} \times \mathcal{SV} \mid x \mapsto {\ssa{x_1}} \in \delta_1 , x \mapsto {\ssa{x_2} } \in \delta_2, {\ssa{x_1} \not= {\ssa{x_2} }  }  \} \]
\[ \delta_1 \bowtie \delta_2 / \bar{x} = \{ ( x, {\ssa{x_1}, \ssa{x_2}} ) \in \mathcal{VAR} \times \mathcal{SV} \times \mathcal{SV} \mid x \not\in \bar{x} \land x \mapsto {\ssa{x_1}} \in \delta_1 , x \mapsto {\ssa{x_2} } \in \delta_2, {\ssa{x_1} \not= {\ssa{x_2} }   }  \} \]
This definition of $\delta_1 \bowtie \delta_2$ combines two translation environments by only keeping mapping who share the same key in both environments. It returns a set of tuples with three elements $(x,\ssa{x_1}, \ssa{x_2})$ to show that a variable $x$ in the loop language may be translated to either $\ssa{x_1}$ or $\ssa{x_2}$, depending on the control flow. We use $\bar{x}$ to represent a list of variables $x$, in this sense, the results of  $\delta_1 \bowtie \delta_2$ is denoted as $ [\bar{x}, \bar{\ssa{x_1}}, \bar{\ssa{x_2}}]$.
\[
 [\bar{x}, \bar{\ssa{x_1}}, \bar{\ssa{x_2}}] = \{ (x, x_1,x_2)  | \forall 0 \leq i < |\bar{x}|, x = \bar{x }[i] \land x_1 = \bar{x_1}[i] \land x_2 = \bar{x_2 }[i] \land |\bar{x}| = |\bar{x_1}| = |\bar{x_2}|   \}
\]

When it comes to the if command, a variable in loop language may be translated to two possible ssa variables in three cases: (1) the variable $x$ is assigned in both two branches, whose mapping of $x$ is stored in $\delta_1$(then branch) and $\delta_2$(else branch). (2) the variable $x$ is assigned before the if command (this mapping in $\delta$) and only assigned in the then branch $\delta_1$ (3) the variable $x$ is assigned before the if command (this mapping in $\delta$) and only assigned in the else branch $\delta_2$. This corresponds to the aforementioned discussion of the if command in ssa language. We leave these mapping explicitly in the if command of the ssa language. We use the variant of $\delta \bowtie \delta_1$, $\delta \bowtie \delta_1 / \bar{x}$ to guarantee that the variables stored in $ [\bar{y}, \bar{\ssa{y_1}}, \bar{\ssa{y_2}}]$ only appears in the then branch, not in the else branch. Similarly for $ [\bar{z}, \bar{\ssa{z_1}}, \bar{\ssa{z_2}}]$. In the translated if command, the variable in $\bar{x}, \bar{y}, \bar{z}$ is replaced with the fresh ssa variables stored in $\bar{\ssa{x}},\bar{\ssa{y}},\bar{\ssa{z}}$.    

The loop transformation rule also deserves a deep discussion. Except the normal transformation of the loop counter $\aexpr$ to $\ssa{\aexpr}$, the additional iteration counter is added to its ssa form to support the evaluation, as we have seen in the rule $\textbf{ssa-loop}$ in Figure~\ref{fig:ssa_evaluation}, and is set to $0$. For the variables assigned in the loop body $c$, we leave $ [\bar{\ssa{x'}}, \bar{\ssa{x_1}}, \bar{\ssa{x_2}}]$ in the command, which tracks variables in the loop body whose value may come from two sources: assignment before the loop($\delta$) or assignment from the loop body $\delta_1$. We have two transformation on the body $c$ using the same ssa name environment $\Sigma$ but different translation environments $\delta$ and $\delta_1$. The premise $\Sigma; \delta; c \hookrightarrow \ssa{c_1}; \delta_1 ; \Sigma_1 $ corresponds to the transformation of the loop body in the first iteration with the variables assigned before the loop execution. The second premise $\Sigma; \delta; c \hookrightarrow \ssa{c_2}; \delta_1 ; \Sigma_1 $ corresponds to the transformation in the later iteration with the assigned variables updated by previous execution of the body. Thanks to the extra part $ [\bar{\ssa{x'}}, \bar{\ssa{x_1}}, \bar{\ssa{x_2}}]$ in the ssa loop command, we know that those variables used in the first iteration are stored in $\bar{\ssa{x_1}}$ and those updated by the loop body are stored in $\bar{\ssa{x_2}}$. To finish the ssa transformation, we get the fresh variables $\bar{\ssa{x'}}$ to replace the appearance of $\bar{\ssa{x_1}}$ in  $\ssa{c_1}$ or $\bar{\ssa{x_2}}$ in $\ssa{c_2}$. Finally, we get the loop body $\ssa{c}$ in its ssa form.



% the intuition behind $\delta_1 \bowtie \delta_2 $ is  



% We use a translation environment $\delta$, to map variables $x$ in the low level language to those $\ssa{x}$ in ssa-form language. We use a name environment denoted as $\Sigma$, a set of ssa variables so we can get a fresh variable by $fresh(\Sigma)$. We define $\delta_1 \bowtie \delta_2 $ in a similar way as \cite{vekris2016refinement}.
% \[ \delta_1 \bowtie \delta_2 = \{ ( x, {\ssa{x_1}, \ssa{x_2}} ) \in \mathcal{VAR} \times \mathcal{SV} \times \mathcal{SV} \mid x \mapsto {\ssa{x_1}} \in \delta_1 , x \mapsto {\ssa{x_2} } \in \delta_2, {\ssa{x_1} \not= {\ssa{x_2} }  }  \} \]
% \[ \delta_1 \bowtie \delta_2 / \bar{x} = \{ ( x, {\ssa{x_1}, \ssa{x_2}} ) \in \mathcal{VAR} \times \mathcal{SV} \times \mathcal{SV} \mid x \not\in \bar{x} \land x \mapsto {\ssa{x_1}} \in \delta_1 , x \mapsto {\ssa{x_2} } \in \delta_2, {\ssa{x_1} \not= {\ssa{x_2} }   }  \} \]
% We call a list of variables $\bar{x}$.

% \[
%  [\bar{x}, \bar{\ssa{x_1}}, \bar{\ssa{x_2}}] = \{ (x, x_1,x_2)  | \forall 0 \leq i < |\bar{x}|, x = \bar{x }[i] \land x_1 = \bar{x_1}[i] \land x_2 = \bar{x_2 }[i] \land |\bar{x}| = |\bar{x_1}| = |\bar{x_2}|   \}
% \]

\begin{figure}
\begin{mathpar}
\boxed{ \delta ; e \hookrightarrow \ssa{e} }\\
\inferrule{
}{
 \delta ; x \hookrightarrow \delta(x)
}~{\textbf{S-VAR}}\\
\boxed{ \Sigma; \delta ; c  \hookrightarrow \ssa{c} ; \delta' ; \Sigma' } \\
\inferrule{
  { \delta ; \bexpr \hookrightarrow \ssa{\bexpr} }
  \and
  { \Sigma; \delta ; c_1 \hookrightarrow \ssa{c_1} ; \delta_1;\Sigma_1 }
  \and
  {\Sigma_1; \delta ; c_2 \hookrightarrow \ssa{c_2} ; \delta_2 ; \Sigma_2 }
  \\
  {[\bar{x}, \ssa{\bar{{x_1}}, \bar{{x_2}}}] = \delta_1 \bowtie \delta_2  }
  \and
   {[\bar{y}, \ssa{\bar{{y_1}}, \bar{{y_2}}}] = \delta \bowtie \delta_1 / \bar{x} }
  \and
   {[\bar{z}, \ssa{\bar{{z_1}}, \bar{{z_2}}}] = \delta \bowtie \delta_2 / \bar{x} }
  \\
  { \delta' =\delta[\bar{x} \mapsto \ssa{\bar{{x}}'} ][\bar{y} \mapsto \ssa{\bar{{y}}'} ][\bar{z} \mapsto \ssa{\bar{{z}}'} ]}
  \and 
  {\ssa{\bar{{x}}', \bar{y}', \bar{z}'} \ fresh(\Sigma_2)
  }
  \and{\Sigma' = \Sigma_2 \cup \{ \ssa{ \bar{x}', \bar{y}', \bar{z}' } \} }
}{
 \Sigma; \delta ; [\eif(\bexpr, c_1, c_2)]^l  \hookrightarrow [\ssa{ \eif(\bexpr, [\bar{{x}}', \bar{{x_1}}, \bar{{x_2}}] ,[\bar{{y}}', \bar{{y_1}}, \bar{{y_2}}] ,[\bar{{z}}', \bar{{z_1}}, \bar{{z_2}}] , {c_1}, {c_2})}]^l; \delta';\Sigma'
}~{\textbf{S-IF}}
%
\and
%
\inferrule{
 {\delta ; \expr \hookrightarrow \ssa{\expr} }
 \and
 {\delta' = \delta[x \mapsto \ssa{{x}} ]}
 \and{ \ssa{x} \ fresh(\Sigma) }
 \and { \Sigma' = \Sigma \cup \{ \ssa{x} \} }
}{
 \Sigma;\delta ; [\assign x \expr]^{l} \hookrightarrow [\ssa{\assign {{x}}{ \expr}}]^{l} ; \delta'; \Sigma'
}~{\textbf{S-ASSN}}
%
\and
%
\inferrule{
%  {\delta ; q \hookrightarrow \ssa{q}}
%  \and
 {\delta ; \expr \hookrightarrow \ssa{\expr}}
 \and
 {\delta' = \delta[x \mapsto \ssa{x} ]}
 \and{ \ssa{x} \ fresh(\Sigma) }
  \and { \Sigma' = \Sigma \cup \{ \ssa{x} \} }
}{
 \Sigma;\delta ; [\assign{x}{q(e)}]^{l} \hookrightarrow [\assign {\ssa{x}}{ {q(\ssa{\expr})}}]^{l} ; \delta';\Sigma'
}~{\textbf{S-QUERY}}
% %
% \and
% %
% \inferrule{
%  {\delta ; q_i \hookrightarrow \ssa{q_i}}
%  \and
%  { \delta ; v_k \hookrightarrow \ssa{v_k} }
%  \and
%  { \delta ; v_i \hookrightarrow \ssa{v_i} }
%  \and
%  {\delta' = \delta[x \mapsto \ssa{x} ]}
%  \and{ \ssa{x} \ fresh(\Sigma) }
%   \and { \Sigma' = \Sigma \cup \{ \ssa{x} \} }
% }{
%  \Sigma;\delta ; [ \eswitch({v_k},{x}, ({v_i} \to q_i)) ]^{l} \hookrightarrow [\eswitch(\ssa{v_k},\ssa{x}, (\ssa{v_i} \to \ssa{q_i}))]^{l} ; \delta';\Sigma'
% }~{\textbf{S-SWITCH}}
%
\and
%
\inferrule{
    {\delta ; \aexpr \hookrightarrow \ssa{\aexpr} }
    \and
    { \Sigma; \delta ; c \hookrightarrow \ssa{c_1} ; \delta_1; \Sigma_1 }
    \and 
     { \Sigma; \delta_1 ; c \hookrightarrow \ssa{c_2} ; \delta_1; \Sigma_1 }
     \\
    { [ \bar{x}, \ssa{\bar{{x_1}}}, \ssa{\bar{{x_2}}} ] = \delta \bowtie \delta_1 }
    \and {\delta' = \delta[\bar{x} \mapsto \ssa{\bar{{x}}'}]}
    \and {\ssa{\bar{{x}}'} \ fresh(\Sigma_1 )}
    \and 
    {\ssa{c'= c_1[\bar{x}'/ \bar{x_1}] = c_2[\bar{x}'/ \bar{x_2}]  } }
    % \and{ \delta' ; c \hookrightarrow \ssa{c'} ; \delta'' }
  }{ 
  \Sigma; \delta ;  [\eloop ~ \aexpr ~ \edo ~ c ]^{l} \hookrightarrow [\ssa{\eloop ~ \aexpr, 0, [\bar{{x}}', \bar{{x_1}}, \bar{{x_2}}] ~ \edo ~ {c'} }]^{l} ; \delta_1[\bar{x} \to \ssa{\bar{x}'}]; \Sigma \cup \{\ssa{\bar{x}'}  \}
}~{\textbf{S-LOOP}}
%
\and
%
\inferrule{
 {\Sigma;\delta ; c_1 \hookrightarrow \ssa{c_1} ; \delta_1; \Sigma_1} 
 \and
 {\Sigma_1; \delta_1 ; c_2 \hookrightarrow \ssa{c_2} ; \delta'; \Sigma'} 
}{
\Sigma;\delta ; c_1 ; c_2 \hookrightarrow \ssa{c_1} ; \ssa{c_2} \ ; \delta';\Sigma'
}~{\textbf{S-SEQ}}
\end{mathpar}
 \caption{Transformation rules from loop language to ssa language}
    \label{fig:trans_rules}
\end{figure}

\subsection{The soundness of the transformation}
In this subsection, we show our transformation from the loop language to its ssa form is sound with respect to adaptivity. To be specific, a transformed program $\ssa{c}$ starting with appropriate configuration, generates the same trace as the program before the transformation $c$, in its corresponding configuration.

We first define a good memory in the loop language $m$ or in the ssa language $\ssa{m}$ with respect to a translation environment $\delta$, denoted as $m \vDash \delta$ and $\ssa{m} \vDash \delta$ respectively. 

\begin{defn}[Well defined memory] 
\begin{enumerate}
    % \item $m \vDash c \triangleq \forall x \in \fv{c}, \exists v, (x, v) \in m$.
    \item $ m \vDash \delta  \triangleq \forall x \in \dom(\delta), \exists v, (x,v) \in m$.
    % \item $\ssa{m} \vDash_{ssa} \ssa{c} \triangleq \forall \ssa{x} \in \fvssa{\ssa{c}}, \exists v, (\ssa{x}, v) \in \ssa{m}$.
    \item $ \ssa{m} \vDash_{ssa} \delta  \triangleq \forall \ssa{x} \in \codom(\delta), \exists v, (\ssa{x},v) \in \ssa{m}$.
\end{enumerate}
\end{defn}
The part declared in the translation environment $\delta$ in a ssa memory $\ssa{m}$ can be reverted to corresponding part of the memory $m$ with an inverse of $\delta$ as follows.

\begin{defn}[Inverse of transformed memory]
 $m = \delta^{-1}(\ssa{m}) \triangleq \forall x \in \dom(\delta), (\delta(x), m(x)) \in \ssa{m} $.
\end{defn}

We can also show that the $\expr$ and its translated ssa version $\ssa{\expr}$ through some translation environment $\delta$ evaluates to the same value in Lemma~\ref{same_value}. 
\begin{lem}[Value remains during transformation]
\label{same_value}
Given $\delta; e \hookrightarrow \ssa{e}$, $\forall m. m \vDash \delta. \forall \ssa{m}, \ssa{m} \vDash_{ssa} \delta \land m = \delta^{-1}(\ssa{m})$, $\forall m. m \vDash \delta. \forall \ssa{m}, \ssa{m} \vDash_{ssa} \delta \land m = \delta^{-1}(\ssa{m})$, then $\config{m, e} \to v $ and $\config{
\ssa{m}, \ssa{e}} \to {v}$.  
\end{lem}

Finally, we show the soundness of the transformation. When a program $c$ is transformed to its ssa form $\ssa{c}$, when executing these two programs with corresponding configuration, the newly generated trace from will be the same and the resulting memory $m'$ and $\ssa{m'}$ will also be related. 

\begin{thm}[Soundness of transformation]
Given $\Sigma; \delta ; c \hookrightarrow \ssa{c} ; \delta';\Sigma' $, $\forall m. m \vDash \delta. \forall \ssa{m}, \ssa{m} \vDash_{ssa} \delta \land m = \delta^{-1}(\ssa{m})$, if there exist an execution of $c$ in the loop language, starting with a trace $t$ and while map $w$, $\config{m, c, t, w} \to^{*} \config{m', \eskip, t', w' } $,  then there also exists a corresponding execution of $\ssa{c}$ in the ssa language so that 
  $\config{  {\ssa{m}}, \ssa{c}, t, w} \to^{*} \config{{  \ssa{m'}}, \eskip, t', w' } $ and $ m' = \delta'^{-1}(\ssa{m'}) $.
\end{thm}


% \subsection{The analysis algorithm on ssa programs}
