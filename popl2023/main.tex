%% For double-blind review submission, w/o CCS and ACM Reference (max submission space)
\documentclass[acmsmall,review,anonymous]{acmart}\settopmatter{printfolios=true,printccs=false,printacmref=false}
%% For double-blind review submission, w/ CCS and ACM Reference
%\documentclass[acmsmall,review,anonymous]{acmart}\settopmatter{printfolios=true}
%% For single-blind review submission, w/o CCS and ACM Reference (max submission space)
%\documentclass[acmsmall,review]{acmart}\settopmatter{printfolios=true,printccs=false,printacmref=false}
%% For single-blind review submission, w/ CCS and ACM Reference
%\documentclass[acmsmall,review]{acmart}\settopmatter{printfolios=true}
%% For final camera-ready submission, w/ required CCS and ACM Reference
%\documentclass[acmsmall]{acmart}\settopmatter{}


%% Journal information
%% Supplied to authors by publisher for camera-ready submission;
%% use defaults for review submission.
\acmJournal{PACMPL}
\acmVolume{1}
\acmNumber{CONF} % CONF = POPL or ICFP or OOPSLA
\acmArticle{1}
\acmYear{2018}
\acmMonth{1}
\acmDOI{} % \acmDOI{10.1145/nnnnnnn.nnnnnnn}
\startPage{1}



%% Copyright information
%% Supplied to authors (based on authors' rights management selection;
%% see authors.acm.org) by publisher for camera-ready submission;
%% use 'none' for review submission.
\setcopyright{none}
%\setcopyright{acmcopyright}
%\setcopyright{acmlicensed}
%\setcopyright{rightsretained}
%\copyrightyear{2018}           %% If different from \acmYear

%% Bibliography style
\bibliographystyle{ACM-Reference-Format}
%% Citation style
%% Note: author/year citations are required for papers published as an
%% issue of PACMPL.
\citestyle{acmauthoryear}   %% For author/year citations


%%%%%%%%%%%%%%%%%%%%%%%%%%%%%%%%%%%%%%%%%%%%%%%%%%%%%%%%%%%%%%%%%%%%%%
%% Note: Authors migrating a paper from PACMPL format to traditional
%% SIGPLAN proceedings format must update the '\documentclass' and
%% topmatter commands above; see 'acmart-sigplanproc-template.tex'.
%%%%%%%%%%%%%%%%%%%%%%%%%%%%%%%%%%%%%%%%%%%%%%%%%%%%%%%%%%%%%%%%%%%%%%
\newcommand{\defeq}{\mathrel{\doteq}}

\newcommand{\lzero}{0}

\newcommand{\kw}[1]{\mathtt{#1}}

\newcommand{\expr}{e}
\newcommand{\vall}{w}
\newcommand{\valr}{v}
\newcommand{\eif}{\kw{if}}
\newcommand{\eapp}{\;}
\newcommand{\eprojl}{\kw{fst}}
\newcommand{\eprojr}{\kw{snd}}
%\newcommand{\eprov}[1]{\eta_{#1}}
\newcommand{\etrue}{\kw{true}}
\newcommand{\efalse}{\kw{false}}
\newcommand{\econst}{c}
\newcommand{\eop}{\delta}
\newcommand{\efix}{\mathop{\kw{fix}}}
%\newcommand{\labelA}{\ell}

\newcommand{\tr}{T}
\newcommand{\trift}{\eif^{\kw{t}}}
\newcommand{\triff}{\eif^{\kw{f}}}
\newcommand{\trprojl}{\eprojl}
\newcommand{\trprojr}{\eprojr}
\newcommand{\trtrue}{\etrue}
\newcommand{\trfalse}{\efalse}
\newcommand{\trconst}{\econst}
\newcommand{\trop}{\eop}
\newcommand{\trfix}{\efix}
\newcommand{\trapp}[5]{#1 \; #2 \mathrel{\triangleright} {\efix #3(#4).#5}}

\newcommand{\adap}{\kw{adap}}
\newcommand{\ddep}[1]{\kw{depth}_{#1}}
\newcommand{\nat}{\mathbb{N}}
\newcommand{\natb}{\nat_{\bot}}
\newcommand{\natbi}{\natb^\infty}
\newcommand{\nnatA}{n}
\newcommand{\nnatB}{m}
\newcommand{\nnatbA}{s}
\newcommand{\nnatbB}{t}
\newcommand{\nnatbiA}{q}
\newcommand{\nnatbiB}{r}

\newcommand{\type}{\tau}
\newcommand{\tbase}{\kw{b}}
\newcommand{\tbool}{\kw{bool}}
\newcommand{\tarr}[5]{#1; #3 \xrightarrow{#4; \, #5} #2}
\newcommand{\env}{\theta}

\newcommand{\bigstep}{\mathrel{\Downarrow}}

\newcommand{\dmap}{\rho}
\newcommand{\dmapb}{\bot_\dmap}
\newcommand{\supp}{\kw{supp}}
\newcommand{\dom}{\kw{dom}}

\newcommand{\tvdash}[1]{\vdash_{#1}}

\newcommand{\mg}[1]{\textcolor[rgb]{.90,0.00,0.00}{[MG: #1]}}
\newcommand{\dg}[1]{\textcolor[rgb]{0.00,0.5,0.5}{[DG: #1]}}
\newcommand{\wq}[1]{\textcolor[rgb]{.50,0.0,0.7}{[WQ: #1]}}
\newcommand{\jl}[1]{\textcolor[rgb]{.00,0.8,0.2}{[JL: #1]}}

%%Packages
\usepackage[T1]{fontenc}
\usepackage{fourier} 
\usepackage[english]{babel} 
\usepackage{amsmath,amsfonts} 
\usepackage{amsthm} 
\usepackage{color}   %May be necessary if you want to color links
\usepackage{hyperref}
\usepackage{lscape}
\usepackage{geometry}
\usepackage{amsmath}
\usepackage{algorithm}
\usepackage{algorithmic}
\usepackage{amssymb}
\usepackage{amsfonts}
\usepackage{times}
\usepackage{bm}
\usepackage{ stmaryrd }
\SetSymbolFont{stmry}{bold}{U}{stmry}{m}{n}

\usepackage{ amssymb }
\usepackage{ textcomp }
\usepackage[normalem]{ulem}
% For derivation rules
\usepackage{mathpartir}
\usepackage{color}
\usepackage{a4wide}
\usepackage{caption}
\usepackage{subcaption}
\usepackage{mathpartir}
\usepackage{amsmath,amsfonts}
\usepackage{ amssymb }
\usepackage{color}
\usepackage{algorithm}
\usepackage{algorithmic}
\usepackage{microtype}
\usepackage{eucal}
\usepackage{url}
\usepackage{xspace}
\usepackage{array}
\usepackage{listings}

\usepackage{tikz}
\usetikzlibrary{shapes.geometric}
\usetikzlibrary{arrows.meta,arrows}
\usetikzlibrary{decorations.text}
% % % % 


\usepackage{multirow}


%%%%%%%%%%%%%%%%%%%%%%%%%%%%%%%%%%%%%%%%%%%%%%%%%%%%%Packages And Definitions For Listing the Code%%%%%%%%%%%%%%%%%%%%%%%%%%%%%%%%%%%%%%%%%%%%%%%%%%%%%%%%%%%%%%%%%%%%%%%%
\usepackage{listings}
\usepackage{xcolor}

\definecolor{codegreen}{rgb}{0,0.6,0}
\definecolor{codegray}{rgb}{0.5,0.5,0.5}
\definecolor{codepurple}{rgb}{0.58,0,0.82}
\definecolor{backcolour}{rgb}{0.95,0.95,0.92}

\lstdefinestyle{mystyle}{
    backgroundcolor=\color{backcolour},   
    commentstyle=\color{codegreen},
    keywordstyle=\color{magenta},
    numberstyle=\tiny\color{codegray},
    stringstyle=\color{codepurple},
    basicstyle=\ttfamily\footnotesize,
    breakatwhitespace=false,         
    breaklines=true,                 
    captionpos=b,                    
    keepspaces=true,                 
    numbers=left,                    
    numbersep=5pt,                  
    showspaces=false,                
    showstringspaces=false,
    showtabs=false,                  
    tabsize=2
}

\lstset{style=mystyle}
% %% Some recommended packages.
\usepackage{booktabs}   %% For formal tables:
                        %% http://ctan.org/pkg/booktabs
\usepackage{subcaption} %% For complex figures with subfigures/subcaptions
                        %% http://ctan.org/pkg/subcaption

% \newcommand{\defeq}{\mathrel{\doteq}}

\newcommand{\lzero}{0}

\newcommand{\kw}[1]{\mathtt{#1}}

\newcommand{\expr}{e}
\newcommand{\vall}{w}
\newcommand{\valr}{v}
\newcommand{\eif}{\kw{if}}
\newcommand{\eapp}{\;}
\newcommand{\eprojl}{\kw{fst}}
\newcommand{\eprojr}{\kw{snd}}
%\newcommand{\eprov}[1]{\eta_{#1}}
\newcommand{\etrue}{\kw{true}}
\newcommand{\efalse}{\kw{false}}
\newcommand{\econst}{c}
\newcommand{\eop}{\delta}
\newcommand{\efix}{\mathop{\kw{fix}}}
%\newcommand{\labelA}{\ell}

\newcommand{\tr}{T}
\newcommand{\trift}{\eif^{\kw{t}}}
\newcommand{\triff}{\eif^{\kw{f}}}
\newcommand{\trprojl}{\eprojl}
\newcommand{\trprojr}{\eprojr}
\newcommand{\trtrue}{\etrue}
\newcommand{\trfalse}{\efalse}
\newcommand{\trconst}{\econst}
\newcommand{\trop}{\eop}
\newcommand{\trfix}{\efix}
\newcommand{\trapp}[5]{#1 \; #2 \mathrel{\triangleright} {\efix #3(#4).#5}}

\newcommand{\adap}{\kw{adap}}
\newcommand{\ddep}[1]{\kw{depth}_{#1}}
\newcommand{\nat}{\mathbb{N}}
\newcommand{\natb}{\nat_{\bot}}
\newcommand{\natbi}{\natb^\infty}
\newcommand{\nnatA}{n}
\newcommand{\nnatB}{m}
\newcommand{\nnatbA}{s}
\newcommand{\nnatbB}{t}
\newcommand{\nnatbiA}{q}
\newcommand{\nnatbiB}{r}

\newcommand{\type}{\tau}
\newcommand{\tbase}{\kw{b}}
\newcommand{\tbool}{\kw{bool}}
\newcommand{\tarr}[5]{#1; #3 \xrightarrow{#4; \, #5} #2}
\newcommand{\env}{\theta}

\newcommand{\bigstep}{\mathrel{\Downarrow}}

\newcommand{\dmap}{\rho}
\newcommand{\dmapb}{\bot_\dmap}
\newcommand{\supp}{\kw{supp}}
\newcommand{\dom}{\kw{dom}}

\newcommand{\tvdash}[1]{\vdash_{#1}}

\usepackage{tikz}
\usetikzlibrary{shapes,arrows,snakes,decorations.markings}
\usetikzlibrary{svg.path,arrows.meta, calc} 

\tikzstyle{decision} = [diamond, draw, fill=blue!20, 
    text width=4.5em, text badly centered, node distance=3cm, inner sep=0pt]
% \tikzstyle{block} = [rectangle, draw, fill=blue!20, 
%     text width=5em, text centered, rounded corners, minimum height=4em]
\tikzstyle{block} = [draw, very thick, fill=white, rectangle, 
    minimum height=2.5em, minimum width=6em, text centered]

\tikzstyle{line} = [draw, -latex']
\tikzstyle{cloud} = [draw, ellipse,fill=red!20, node distance=3cm,    minimum height=2em]

\tikzstyle{vecArrow} = [thick, decoration={markings,mark=at position
   1 with {\arrow[semithick]{open triangle 60}}},
   double distance=1.4pt, shorten >= 5.5pt,
   preaction = {decorate},
   postaction = {draw,line width=1.4pt, white,shorten >= 4.5pt}]
\tikzstyle{innerWhite} = [semithick, white,line width=1.4pt, shorten >= 4.5pt]

\newcommand{\THESYSTEM}{\textsf{AdaptFun}}

\begin{document}

%% Title information
\title[Short Title]{Program Analysis for Adaptive Data Analysis}         %% [Short Title] is optional;
                                        %% when present, will be used in
                                        %% header instead of Full Title.
%\titlenote{with title note}             %% \titlenote is optional;
                                        %% can be repeated if necessary;
                                        %% contents suppressed with 'anonymous'
%\subtitle{Subtitle}                     %% \subtitle is optional
%\subtitlenote{with subtitle note}       %% \subtitlenote is optional;
                                        %% can be repeated if necessary;
                                        %% contents suppressed with 'anonymous'


%% Author information
%% Contents and number of authors suppressed with 'anonymous'.
%% Each author should be introduced by \author, followed by
%% \authornote (optional), \orcid (optional), \affiliation, and
%% \email.
%% An author may have multiple affiliations and/or emails; repeat the
%% appropriate command.
%% Many elements are not rendered, but should be provided for metadata
%% extraction tools.

%% Author with single affiliation.
\author{First1 Last1}
\authornote{with author1 note}          %% \authornote is optional;
                                        %% can be repeated if necessary
\orcid{nnnn-nnnn-nnnn-nnnn}             %% \orcid is optional
\affiliation{
  \position{Position1}
  \department{Department1}              %% \department is recommended
  \institution{Institution1}            %% \institution is required
  \streetaddress{Street1 Address1}
  \city{City1}
  \state{State1}
  \postcode{Post-Code1}
  \country{Country1}                    %% \country is recommended
}
\email{first1.last1@inst1.edu}          %% \email is recommended

%% Author with two affiliations and emails.
\author{First2 Last2}
\authornote{with author2 note}          %% \authornote is optional;
                                        %% can be repeated if necessary
\orcid{nnnn-nnnn-nnnn-nnnn}             %% \orcid is optional
\affiliation{
  \position{Position2a}
  \department{Department2a}             %% \department is recommended
  \institution{Institution2a}           %% \institution is required
  \streetaddress{Street2a Address2a}
  \city{City2a}
  \state{State2a}
  \postcode{Post-Code2a}
  \country{Country2a}                   %% \country is recommended
}
\email{first2.last2@inst2a.com}         %% \email is recommended
\affiliation{
  \position{Position2b}
  \department{Department2b}             %% \department is recommended
  \institution{Institution2b}           %% \institution is required
  \streetaddress{Street3b Address2b}
  \city{City2b}
  \state{State2b}
  \postcode{Post-Code2b}
  \country{Country2b}                   %% \country is recommended
}
\email{first2.last2@inst2b.org}         %% \email is recommended


%% Abstract
%% Note: \begin{abstract}...\end{abstract} environment must come
%% before \maketitle command
\begin{abstract}
 Data analyses are usually designed to identify some property of the population from which the data are drawn, generalizing beyond the specific data sample. For this reason, data analyses are often designed in a way that guarantees that they produce a low generalization error. That is, to guarantee that the result of a data analysis run on a sample data does not differ too much from the result one would achieve by running the analysis over the entire population. 

An adaptive data analysis can be seen as a process composed by multiple queries interrogating some data, where the choice of which query to run next may rely on the results of previous queries. The generalization error of each individual query/analysis can be controlled by using an array of well-established statistical techniques. However, when queries are arbitrarily composed, the different errors can propagate through the chain of different queries and bring to high generalization error. To address this issue, data analysts are designing several techniques that not only guarantee bounds on the generalization errors of single queries, but that also guarantee bounds on the generalization error of the composed analyses. 
The total number of queries and the depth of the chain of queries are of great significance to guarantee the generalization error, when the composed data analyses are adaptive. 
The choice of which of these techniques to use, often depends on the process of the chain of queries that an adaptive data analysis can generate.   % Gap
% Unfortunately, this depth which relies on the program(implementation) itself is costly in human efforts, and how to statically obtain this information is not well studied to support data analysts.

In this work, \wq{we propose a new program property 'adaptivity' which measures the depth of chain of queries during the execution of program implementing the target data analysis. In correspondence, we design a program analysis that provides the upper bound on both the total number of queries and this 'adaptivity', when the input program is an implementation of an adaptive data analysis. These two measures can be then used by data analysts to select the right technique to control the generalization error in their adaptive data analysis design.} 
% Given an input program implementing an adaptive data analysis, our program analysis generates an upper bound on the total number of queries that the data analysis will run, and more interestingly also an upper bound on the depth of the chain of queries. 
% These two measures can be used to select the right technique to guarantee a bound on the generalization error of the data analysis. 
\wq{The key novelty of our program analysis lands in the representation of adaptivity in the form of a finite walk through a compact dependency graph between weighted variables in the input program, where the weight relies on the existing work of the reachibility bound. Additionally, the predicted 
 upper bound on adaptivity of our program analysis is based on path searching algorithm, which is proved sound with respect to the finite walk representation. 
 We implement our program analysis and show it can help to analyze the generalization error of several concrete data analyses with different adaptivity structures.
 }

% Our program analysis is based on an analysis of the dependency graph between different queries, representing the potential chain an adaptive data analysis may generate. We show how the proposed program analysis can help to analyze the generalization error of several concrete data analyses with different adaptivity structures.
\end{abstract}


%% 2012 ACM Computing Classification System (CSS) concepts
%% Generate at 'http://dl.acm.org/ccs/ccs.cfm'.
\begin{CCSXML}
<ccs2012>
<concept>
<concept_id>10011007.10011006.10011008</concept_id>
<concept_desc>Software and its engineering~General programming languages</concept_desc>
<concept_significance>500</concept_significance>
</concept>
<concept>
<concept_id>10003456.10003457.10003521.10003525</concept_id>
<concept_desc>Social and professional topics~History of programming languages</concept_desc>
<concept_significance>300</concept_significance>
</concept>
</ccs2012>
\end{CCSXML}

\ccsdesc[500]{Software and its engineering~General programming languages}
\ccsdesc[300]{Social and professional topics~History of programming languages}
%% End of generated code


%% Keywords
%% comma separated list
\keywords{Adaptive data analysis, program analysis, dependency graph}  %% \keywords are mandatory in final camera-ready submission


%% \maketitle
%% Note: \maketitle command must come after title commands, author
%% commands, abstract environment, Computing Classification System
%% environment and commands, and keywords command.
\maketitle


\section{Introduction}

% The topic, motivation, the importance of adaptivity 
Consider a dataset $X$ consisting of $n$ independent samples from some unknown population $\dist$.  How can we ensure that the conclusions drawn from $X$ \emph{generalize} to the population $\dist$?  Despite decades of research in statistics and machine learning on methods for ensuring generalization, there is an increased recognition that many scientific findings generalize poorly (e.g. 
\cite{Ioannidis05,GelmanL13}
).  While there are many reasons a conclusion might fail to generalize, one that is receiving increasing attention is \emph{adaptivity}, which occurs when the choice of method for analyzing the dataset depends on previous interactions with the same dataset~\cite{GelmanL13}.
%
 Adaptivity can arise from many common practices, such as exploratory data analysis, using the same data set for feature selection and regression, and the re-use of datasets across research projects.  Unfortunately, adaptivity invalidates traditional methods for ensuring generalization and statistical validity, which assume that the method is selected independently of the data. The misinterpretation of adaptively selected results has even been blamed for a ``statistical crisis'' in empirical science~\cite{GelmanL13}.
%  ~\cite{GelmanL13}.

\begin{figure}
    \centering
    \includegraphics[width=0.7\columnwidth]{overview.png}
    \caption{Overview of our Adaptive Data Analysis model.
    We have a population that we are interested in studying, and a dataset containing individual samples from this population. 
    The adaptive data analysis we are interested in running has access to the dataset through queries of some pre-determined family (e.g., statistical or linear queries) mediated by a mechanism. 
    This mechanism uses randomization to reduce the generalization error of the queries issued to the data.}
    \label{fig:adaptivity-model-overview}
\vspace{-0.5cm}
\end{figure}

A line of work initiated by \cite{DworkFHPRR15}, \cite{HardtU14} posed the question: Can we design \emph{general-purpose} methods that ensure generalization in the presence of adaptivity, together with guarantees on their accuracy?  
The idea that has emerged in these works is to use randomization to help ensure generalization. 
Specifically, these works have proposed to mediate the access of an adaptive data analysis to the data by means of queries from some pre-determined family (we will consider here a specific family of queries often called "statistical" or "linear" queries) that are sent to a  \emph{mechanism} which uses some randomized process to guarantee that the result of the query does not depend too much on the specific
sampled dataset. 
This guarantees that the result of the queries generalizes well. This approach is described in Fig.~\ref{fig:adaptivity-model-overview}.  
This line of work has identified many new algorithmic techniques for ensuring generalization in adaptive data analysis, leading to algorithms with greater statistical power than all previous approaches. Common methods proposed by these works include, the addition of noise to the result of a query, data splitting, etc. Moreover, these works have also identified problematic strategies for adaptive analysis, showing limitations on the statistical power one can hope to achieve. Subsequent works have then further extended the methods and techniques in this approach and further extended the theoretical underpinning of this approach, e.g.~\cite{dwork2015reusable,dwork2015generalization,BassilyNSSSU16,UllmanSNSS18,FeldmanS17,jung2019new,SteinkeZ20,RogersRSSTW20}.

A key development in this line of work is that the best method for ensuring generalization in an adaptive data analysis depends to a large extent on the number of \emph{rounds of adaptivity}, the depth of the chain of queries. 
As an informal example, the program $x \leftarrow q_1(D);y \leftarrow q_2(D,x);z \leftarrow q_3(D,y)$ has three rounds of adaptivity, since $q_2$  depends on $D$ not only directly because it is one of its input but also via the result of $q_1$, which is also run on $D$, and similarly,  $q_3$ depends on $D$ directly but also via the result of $q_2$, which in turn depends on the result of $q_1$.
The works we discussed above showed that, not only does the analysis of the generalization error depend on the number of rounds, but knowing the number of rounds actually allows one to choose methods that lead to the smallest possible generalization error - we will discuss this further in Section~\ref{sec:overview}. 

For example, these works showed that when an adaptive data analysis uses a large number of rounds of adaptivity then a low generalization error can be achieved by a mechanism  
adding to the result of each query Gaussian noise scaled to the number of rounds. When instead  an adaptive data analysis uses a small number of rounds of adaptivity then a low generalization error can be achieved by using more specialized methods, such as data splitting mechanism or the reusable holdout technique from~\cite{DworkFHPRR15}.
To better understand this idea, we show in Fig.~\ref{fig:generalization_errors} three experiments showcasing these situations.
More precisely, in Fig.~\ref{fig:generalization_errors}(a) we show the results of a specific analysis\footnote{We will use formally a program implementing this analysis (Fig.~\ref{fig:overview-example}) as a running example in the rest of the paper.} with two rounds of adaptivity.
This analysis can be seen as a classifier which first runs 400 non-adaptive queries on the first 400 attributes of the data, looking for correlations between the attributes and a label, and then runs one last query which depends on all these correlations.
Without any mechanism the generalization error of the last query is pretty large, and the lower generalization error is achieved when the data-splitting method is used.
Fig.~\ref{fig:generalization_errors}(c) shows how this situation also change with the number of queries. Specifically, it shows the root mean square error of the last \emph{adaptive} query when the numbers queries varies. This also highlight the fact that different mechanisms, for the same analysis, produce results with very different generalization error.
In Fig.~\ref{fig:generalization_errors}(b), we show the results of a specific analysis\footnote{We will present this analysis formally in Section~\ref{sec:examples}.} with four hundreds rounds of adaptivity.
At each step, this analysis runs an adaptive query based on the results of the previous ones. Without any mechanism, the generalization error of most of the queries is pretty large, and this error can be lowered by using Gaussian noise. 
{\small
\begin{figure}
\centering
\begin{subfigure}{.32\textwidth}
\begin{centering}
\includegraphics[width=1.0\textwidth]{tworound.png}
\caption{}
\end{centering}
\end{subfigure}
\quad
\begin{subfigure}{.32\textwidth}
\begin{centering}
\includegraphics[width=1.0\textwidth]{multipleround.png}
\caption{}
\end{centering}
\end{subfigure}
\begin{subfigure}{.32\textwidth}
\begin{centering}
\includegraphics[width=1.0\textwidth]{twoRounds-rmse-fourmechs.png}
\caption{}
\end{centering}
\end{subfigure}
\vspace{-0.5cm}
 \caption{
 The generalization errors of two adaptive data analysis examples, under different choices of mechanisms.
 (a) Data analysis with 2 rounds adaptivity, 
 (b) Data analysis with 400 rounds adaptivity.
 (c) Same Data analysis as (a) with different query numbers.
}
\label{fig:generalization_errors}
\vspace{-0.6cm}
\end{figure}
}
%gap

This scenario motivates us to explore the design of program analysis techniques that can be used to estimate the number of \emph{rounds of adaptivity} that a program implementing a data analysis can perform. These techniques could be used to help a data analyst in the choice of the mechanism to use,
and they
could ultimately be integrated into a tool for adaptive data analysis such as the \emph{Guess and Check} framework by~\cite{RogersRSSTW20}. 

The first problem we face is \emph{how to formally define} a model for adaptive data analysis which is general enough to support the methods we discussed above and which would permit to formulate the notion of adaptivity these methods use. We take the approach of designing a programming framework for submitting queries to some \emph{mechanism} giving access to the data mediated by one of the techniques we mentioned before, e.g., adding Gaussian noise, randomly selecting a subset of the data, using the reusable holdout technique, etc. In this approach, a program models an \emph{analyst} asking a sequence of queries to the mechanism. The mechanism runs the queries on the data applying one of the methods above and returns the result to the program. The program can then use this result to decide which query to run next. Overall, we are interested in controlling the generalization of the query results returned by the mechanism, by means of the adaptivity. 

The second problem we face is \emph{how to define the adaptivity of a given program}.
Intuitively, a query $Q$ may depend on another query $P$, if there are two values that $P$ can return which affect in different ways the execution of $Q$. 
For example, as shown in \cite{dwork2015reusable}, and as we did in our example in Fig.~\ref{fig:generalization_errors}(a), one can design a machine learning algorithm for constructing a classifier which first computes each feature's correlation with the label via a sequence of queries, and then constructs the classifier based on the correlation values. If one feature's correlation changes, the classifier depending on features is also affected.  
This notion of dependency builds on the execution trace as a \emph{causal history}. In particular, we are interested in the history or provenance of a query up until this is executed, we are not then concerned about how the result is used --- except for tracking whether the result of the query may further cause some other query. This is because we focus on the generalization error of queries and not their post-processing. % 
To formalize this intuition as a quantitative program property,
we use a trace semantics recording the execution history of programs on some given input --- and we create a dependency graph, where the dependency between different variables (queries are also assigned to variables) is explicit and track which variable is associated with a query request. We then enrich this graph with weights describing the number of times each variable is evaluated in a program evaluation starting with an initial state. The adaptivity is then defined as the length of the walk visiting most query-related variables on this graph\footnote{Formally, graphs will be well-defined only for terminating programs, this will guarantee that the longest walk is finite}. In other words, we define adaptivity as a \emph{quantitative form of program dependency}.

% \jl{ 
% To define adaptivity in our programming framework, we consider a weighted dependency graph over variables assigned in the program, where each edge is built by a semantics dependency relation between these variables. The dependency relation relies on the trace semantics of our programming framework which records the execution history of programs implementing adaptive data analysis. 
% %The novelty comes from the definition of relation of dependency between nodes, which consists of the edge in the graph. For now, we can think of each node is associated with a variable, storing the value assigned to its variable 
% }
% \jl{The general idea beneath this dependency relation is that modifying the value of some variable in an execution trace will later affect the following execution trace.
% By tracking if a variable is assigned by a query or not, we are able to distinguish whether one query may depend on the other.}

The third problem we face is \emph{how to estimate the adaptivity of a given program}. 
The adaptive data analysis model we consider and our definition of adaptivity suggest that for this task we can use a  program analysis that is based on some form of dependency analysis. This analysis needs to take into consideration:
1) the fact that, in general, a query $Q$ is not a monolithic block but rather it may depend, through the use of variables and values, on other parts of the program. Hence, it needs to consider some form of data flow analysis. 
2) the fact that, in general, the decision on whether to run a query or not may depend on some other value. Hence, 
 it needs to consider some form of control flow analysis.
 3) the fact that, in general, we are not only interested in whether there is a dependency or not, but in the length of the chain of dependencies. Hence, it needs to consider some quantitative information about the program dependencies. 
 
To address these considerations and be able to estimate a sound upper bound on the adaptivity of a program, 
we develop a static program analysis algorithm, named {\THESYSTEM}, which combines data flow and control flow analysis with reachability bound analysis~\cite{GulwaniZ10}. This combination gives tighter bounds on the adaptivity of a program than the ones one would achieve by directly using the data and control flow analyses or the ones that one would achieve by directly using reachability bound analysis techniques alone. We evaluate {\THESYSTEM} on a number of examples showing that it is able to efficiently estimate precise upper bounds on the adaptivity of different programs. 
All the proofs and extended definitions can be found in the supplementary material.

To summarize, our work aims at the design of a static analysis for programs implementing adaptive analysis that can estimate their rounds of adaptivity. Specifically, our contributions are:
\begin{enumerate}
    \item A programming framework for adaptive data analyses where programs represent analysts that can query generalization-preserving mechanisms mediating the access to some data. 
    \item 
    A formal definition of the notion of adaptivity under the analyst-mechanism model. 
    This definition is built on a variable-based dependency graph that is constructed using sets of program execution traces.
    \item 
    A static program analysis algorithm {\THESYSTEM} combining data flow, control flow and  reachability bound analysis in order to provide tight bounds on the adaptivity of a program.
    \item A soundness proof of the program analysis showing that the adaptivity estimated by {\THESYSTEM} bounds the true adaptivity of the program. 
    \item An implementation of {\THESYSTEM} and an experimental evaluation of the bounds this implementation provides on several examples.
\end{enumerate}

\section{Overview}
\label{sec:overview}
\subsection{Some results in Adaptive Data Analysis}
%\wq{I think we can move this subsection into appendix. Maybe just leave theorm 1.2 and 1.3}
%\jl{I don't agree}
In Adaptive Data Analysis, an \emph{analyst} is interested in studying some distribution $\dist$ over some domain $\univ$.  Following previous works~\cite{DworkFHPRR15,HardtU14,BassilyNSSSU16}, we focus on the setting where the analyst is interested in answers to \emph{statistical queries} (also known as \emph{linear queries}) over the distribution.  A statistical query is usually defined by some function $\qquery \from \univ \to [-1,1]$ (often other codomains such as $[0,1]$ or $[-R,+R]$, for some $R$, are considered).  The analyst wants to learn the \emph{population mean}, which is defined as 
$\qquery(\dist) = \ex{\sample \sim \dist}{\qquery(\sample)}$. 
%
We assume that the distribution $\dist$ can only be accessed via a set of \emph{samples} $\sample_1,\dots,\sample_n$ drawn independently and identically distributed (i.i.d.) from $\dist$.  These samples are held by a mechanism $\mech(\sample_1,\dots,\sample_n)$ who receives the query $\qquery$ and computes an answer 
$\answer \approx \qquery(\dist)$.
%
The na\"ive way to approximate the population mean is to use the \emph{empirical mean}, which (abusing notation) is defined as 
$\qquery(\sample_1,\dots,\sample_n) = \frac{1}{n} \sum_{i=1}^{n} \qquery(X_i)$.
However, the mechanism $M$ can adopt some methods for improving the generalization error $| a- \qquery(\dist)|$.

In this work we consider analysts that ask a sequence of $k$ queries $\qquery_1,\dots,\qquery_k$.  If the queries are all chosen in advance, independently of the answers $a_1,\dots,a_k$ of each other, then we say they are \emph{non-adaptive}.  If the choice of each query $\qquery_j$ depends on the prefix $\qquery_1,\answer_1,\dots,\qquery_{j-1},\answer_{j-1}$ then they are \emph{fully adaptive}.  An important intermediate notion is \emph{$\qrounds$-round adaptive}, where the sequence can be partitioned into $\qrounds$ batches of non-adaptive queries.  Note that non-adaptive queries are $1$-round and fully adaptive queries are $k$-round adaptive.

We now review what is known about the problem of answering $r$-round adaptive queries.  
\begin{thm}[\cite{BassilyNSSSU16}] 
\label{thm:nonadapt-adapt}
\begin{enumerate}

\item For any distribution $\dist$, and any $k$ \emph{non-adaptive} statistical queries, with high probability,
% $$
$
\max_{j=1,\dots,k} | \answer_j - \qquery_j(\dist) | = O\left( \sqrt{\frac{\log k}{n}}  \right)
% $$
$.
%
\item For any distribution $\dist$, and  any $k$  \emph{$\qrounds$-round adaptive} statistical queries, with $\qrounds \geq 2$, with high probability, the empirical mean (rounded to an appropriate number of bits of precision)\footnote{With infinite precision even two queries may give unbounded error, when the first query's result encodes the whole data.} satisfies:\\
% $$
$
\max_{j=1,\dots,k} | \answer_j - \qquery_j(\dist) | = O\left( \sqrt{  \frac{k}{n}}  \right)
% $$
$
\end{enumerate}
\end{thm}
In fact, these bounds are tight (up to constant factors) which means that even allowing one extra round of adaptivity leads to an exponential increase in the generalization error, from $\log k$ to $k$.

\citet{DworkFHPRR15} and \citet{BassilyNSSSU16} showed that by using carefully calibrated Gaussian noise in order to limit the dependency of a single query on the specific data instance, one 
can actually achieve much stronger generalization error as a function of the number of queries, specifically.
\begin{thm}[\cite{DworkFHPRR15, BassilyNSSSU16}] \label{thm:gaussiannoise} For any distribution $\dist$, any $k$, any $\qrounds \geq 2$ and any \emph{$\qrounds$-round adaptive} statistical queries, if we answer queries with carefully calibrated Gaussian noise, with high probability,  we have:
\begin{center}
  $
\max_{j=1,\dots,k} | \answer_j - \qquery_j(\dist) | = O\left( \frac{\sqrt[4]{k}}{\sqrt{n}}  \right)
$  
\end{center}
\end{thm}
% Notice that in order to Theorem~\ref{thm:gaussiannoise} has different quantification in that the optimal choice of mechanism depends on the number of queries.  Thus, we need to know the number of queries \emph{a priori} to choose the best mechanism.
More interestingly, \citet{DworkFHPRR15}
also gave a refined bounds that can be achieved with different mechanisms depending on the number of rounds of adaptivity.   \begin{thm}[\cite{DworkFHPRR15}] \label{thm:gaussiannoise2} For any $r$ and $k$, there exists a mechanism such that for any distribution $\dist$, and any $\qrounds \geq 2$ any \emph{$\qrounds$-round adaptive} statistical queries, with high probability, it satisfies
\begin{center}
  $
\max_{j=1,\dots,k} | \answer_j - \qquery_j(\dist) | = O\left( \frac{r \sqrt{\log k}}{\sqrt{n}}  \right)
$  
\end{center}
\end{thm}
Notice that Theorem~\ref{thm:gaussiannoise2} has different quantification in that the optimal choice of mechanism depends on the number of queries {and number of rounds of adaptivity}.  This suggests that if one knows a good \emph{a priori upper bound on the number of rounds of adaptivity}, one can choose the appropriate mechanism and get a much better guarantee in terms of the generalization error.
As an example, as we can see in Fig.~\ref{fig:generalization_errors}, if we know that an algorithm is 2-rounds adaptive, we can choose data splitting as {the} mechanism, while if we know that an algorithm has many rounds of adaptivity we can choose Gaussian noise. It is worth to stressing that by knowing the number of rounds of adaptivity one can also compute a concrete upper bound on the generalization error of a data analysis. This information allows one to have a quantitative, a priori, estimation of the effectiveness of a data analysis. 
This motivates us to design a static program analysis aimed at giving good \emph{a priori} upper bounds on the number of rounds of adaptivity of a program. 

{\small
\begin{figure}
\centering
\begin{subfigure}{.2\textwidth}
\begin{centering}
$
    \begin{array}{l}
    \kw{towRounds(k)} \triangleq \\
           \clabel{ \assign{a}{0}}^{0} ;
            \clabel{\assign{j}{k} }^{1} ; \\
            \ewhile ~ \clabel{j > 0}^{2} ~ \edo ~ \\
            \Big(
             \clabel{\assign{x}{\query(\chi[j] \cdot \chi[k])} }^{3}  ; \\
             \clabel{\assign{j}{j-1}}^{4} ;\\
            \clabel{\assign{a}{x + a}}^{5}       \Big);\\
            \clabel{\assign{l}{\query(\chi[k]*a)} }^{6}\\
        \end{array}
$
\caption{}
\end{centering}
\end{subfigure}
\begin{subfigure}{.4\textwidth}
%}
\qquad
\begin{centering}
\begin{tikzpicture}[scale=\textwidth/16cm,samples=250]
\draw[] (0, 10) circle (0pt) node
{{ $a^0: {}^{\lambda \trace_0. 1}_{0}$}};
\draw[] (0, 7) circle (0pt) node
{\textbf{$x^3: {}^{\lambda \trace_0. \env(\trace_0) k}_{1}$}};
\draw[] (0, 4) circle (0pt) node {{ $a^5: {}^{\lambda \trace_0. \env(\trace_0) k}_{0}$}};
\draw[] (0, 1) circle (0pt) node
{{ $l^6: {}^{\lambda \trace_0. 1}_{1}$}};
% Counter Variables
\draw[] (8, 9) circle (0pt) node {\textbf{$j^1: {}^{\lambda \trace_0. 1}_{0}$}};
\draw[] (8, 6) circle (0pt) node {{ $j^4: {}^{\lambda \trace_0. \env(\trace_0) k}_{0}$}};
%
% Value Dependency Edges:
\draw[ ultra thick, -latex, densely dotted,] (0, 1.5)  -- (0, 3.5) ;
\draw[ ultra thick, -latex, densely dotted,] (0, 4.5)  -- (0, 6.5) ;
\draw[ thick, -latex] (0, 4.5)  to  [out=-230,in=230]  (0, 9.5) ;
\draw[ thick, -Straight Barb] (1.5, 3.8) arc (120:-200:1);
\draw[ thick, -Straight Barb] (9, 6.5) arc (150:-150:1);
\draw[ thick, -latex] (8, 6.5)  -- (8, 8.5) ;
\draw[ thick, -latex] (0, 1.5)  to  [out=-230,in=230]  (0, 9.5) ;
% Control Dependency
\draw[ thick,-latex] (2, 7)  -- (6, 9) ;
\draw[ thick,-latex] (2, 4.5)  -- (6, 9) ;
\draw[ thick,-latex] (2, 7)  -- (6, 6) ;
\draw[ thick,-latex] (2, 4.5)  -- (6, 6) ;
\end{tikzpicture}
\caption{}
\end{centering}
\end{subfigure}
   \begin{subfigure}{.36\textwidth}
   \begin{centering}
   \begin{tikzpicture}[scale=\textwidth/18cm,samples=200]
\draw[] (0, 10) circle (0pt) node
{{ $a^0: {}^1_{0}$}};
\draw[] (0, 7) circle (0pt) node
{\textbf{$x^3: {}^{k}_{1}$}};
\draw[] (0, 4) circle (0pt) node
{{ $a^5: {}^{k}_{0}$}};
\draw[] (0, 1) circle (0pt) node
{{ $l^6: {}^{1}_{1}$}};
% Counter Variables
\draw[] (5, 9) circle (0pt) node {\textbf{$j^1: {}^{1}_{0}$}};
\draw[] (5, 6) circle (0pt) node {{ $j^4: {}^{k}_{0}$}};
%
% Value Dependency Edges:
\draw[ ultra thick, -latex, densely dotted,] (0, 1.5)  -- (0, 3.5) ;
\draw[ ultra thick, -latex, densely dotted,] (0, 4.5)  -- 
% node [left] {\highlight{$\trace_0 \to \env(\trace_0) k $}}
(0, 6.5) ;
\draw[ thick, -latex] (0, 4.5)  to  [out=-230,in=230]  
% node [left] {\highlight{$\trace_0 \to \env(\trace_0) k $}}
(0, 9.5) ;
\draw[ thick, -Straight Barb] (1.5, 3.5) arc (120:-200:1);
\draw[ thick, -Straight Barb] (6.5, 6.5) arc (150:-150:1);
    % The Weight for this edge
    % \draw[](9, 6) node [] {\highlight{$\trace_0 \to \env(\trace_0) k  $}};
\draw[ thick, -latex] (5, 6.5)  -- (5, 8.5) ;
% Control Dependency
\draw[ thick,-latex] (1.5, 7)  -- (4, 9) ;
\draw[ thick,-latex] (1.5, 4)  -- (4, 9) ;
\draw[ thick,-latex] (1.5, 7)  -- (4, 6) ;
\draw[ thick,-latex] (1.5, 4)  -- (4, 6) ;
\draw[ thick, -latex] (0, 1.5)  to  [out=-230,in=230]  (0, 9.5) ;
\end{tikzpicture}
\caption{}
   \end{centering}
   \end{subfigure}
\vspace{-0.4cm}
 \caption{(a) The program $\kw{towRounds(k)}$, an example 
%  of a program 
with two rounds of adaptivity (b) The corresponding execution-based dependency graph (c) The program-based dependency graph from $\THESYSTEM$.
}
\label{fig:overview-example}
% \vspace{-0.8cm}
\end{figure}
}


\subsection{ {\THESYSTEM} formally through an example.}
We illustrate the key technical components of our framework through a simple adaptive data analysis with two rounds of adaptivity.
% They are 1. the query while language for expressing a data analysis formally, 2. the definition of \emph{adaptivity} (\emph{adaptivity} is the short for \emph{rounds of adaptivity} used in the rest of the paper) based on the language semantics, and 3. the static analysis algorithm providing a sound upper bound on a data analysis' adaptivity.
% }
% \detailed{
% In "two rounds strategy" analysis, the analyst asks in total $k+1$ queries to the mechanism in two phases, the symbol $k$ is an input from the data analyst of this strategy and has no limit on the kind, which can be a constant, or a symbol or even an expression such as $(k+3)*2$.
% } 
%
In this analysis, an analyst asks $k+1$ queries to a mechanism in two phases.
In the first phase, the analyst asks $k$ queries and stores the answers that are provided by the mechanism. In the second phase, the analyst constructs a new query based on the results of the previous $k$ queries and sends this query to the mechanism. 
The mechanism is abstract here and our goal is to use static analysis to provide an upper bound on adaptivity to help choose the mechanism.
This data analysis assumes that the data domain $\univ$ 
contains at least $k$ numeric attributes 
(every query in the first phase focuses on one), which we index just by natural numbers.
The implementation of this data analysis in the language of {\THESYSTEM} is presented in Fig.~\ref{fig:overview-example}(a).

The {\THESYSTEM} language extends a standard while language\footnote{Programs components are labeled, so that we can uniquely identify every component.} with a query request constructor denoted $\query$.
 Queries have the form $\query(\qexpr)$, where $\qexpr$ is a special expression (see syntax in Section~\ref{sec:loop_language}) 
representing a function $\from \univ \to U$ on rows of an hidden database that is only accessible through the mechanisms. The domain $\univ$ of this function is the (arbitrary) domain of rows of the database. The codomain $U$ of this function is the query output space which, depending on the specific program, could be $[-1,1]$, $[0,1]$ or $[-R,+R]$, for some $R$. We use this formalization because we are interested in linear queries which, as we discussed in the previous section, compute the empirical mean of functions on rows.
 As an example, $x \leftarrow \query(\chi[j] \cdot \chi[k])$ computes an approximation, according to the used mechanism, of the empirical mean of the product of the $j^{th}$ attribute and $k^{th}$ attribute, identified by $\chi[j] \cdot \chi[k]$. Notice that we don't materialize the mechanism but we assume that it is implicitly run when we execute the query. 
 In Fig.~\ref{fig:overview-example}(a), the queries inside the while loop correspond to the first phase of the data analysis and compute the sum of the empirical mean of
the product of the $j$th attribute with the $k$th attribute. 
The query outside the loop corresponds to the second phase and computes an approximation of the empirical mean of the last attribute weighted by the sum of the empirical mean of the first $k$ attributes.


This example is intuitively 2-rounds adaptive since we have two clearly distinguished phases, and the queries that we ask in the first phase do not depend on each other (the query $\chi[j] \cdot \chi[k]$ at line $3$ only relies on the counter $j$ and input $k$), while the last query 
(at line 6) depends on the results of all the previous queries. 
However, capturing this concept formally is surprisingly challenging. The difficulty comes from the quantitative nature of this concept and how this quantitative nature interacts with data and control dependency. We describe how we capture it next. 
% \mg{this is weaker than it was in the previous submission.}

%%%%%%%%%%%%%%%%%%%%%%%%%%%%%%%%%%%Some details that might be useful when make passes %%%%%%%%%%%%%%%%%
% \jl{ The $\bullet$ stands for no query, for instance, the second event in the trace $(j, 1, \env(\trace)k , \bullet) $ tells us the assignment at line $1$ does not request a query.} \jl{The third event is a testing event corresponding to the guard of the while loop at line $2$. The evaluation of the query request in the second phase is tracked in }
% % \jl{ 
% The $\bullet$ is a default value for non-query event, 
% for instance, the second event in the trace $(j, 1, K , \bullet) $ tells us the assignment at line $1$ does not request a query.
% The third event is a testing event corresponding to the guard of the while loop at line $2$. The evaluation of the query request in the second phase is tracked in 
% % }
\subsubsection{Adaptivity definition}
\label{sec:adaptivity-informal}
%%%%%%%%%%%%%%%%%%%%%%%%%%%%%%%%%%% Details Below that might be useful when make passes %%%%%%%%%%%%%%%%%
% \detailed{To formally define the adaptivity, we build a directed graph representing the possible dependencies between queries of a program and we call this graph: execution-based dependency graph. The vertices represent the assigned program variables and the edges satisfy the dependency relations between vertices.   Fig.~\ref{fig:overview-example}(b) is the execution dependency graph we build based on the "two rounds strategy program" in Fig.~\ref{fig:overview-example}(a). In brief, the graph is built by collecting the assigned variables with labels of the target program as vertices, which are $a^0$, $j^1$,...$a^5$,$l^6$. We check if there is an edge between two vertices by our dependency relation over two labeled variables (defined in Section~\ref{sec:dep_adaptivity} ). This dependency relation relies on the execution of the program recorded by a trace generated by our trace semantics, which is the reason we call this graph "execution-based". 
% Intuitively from Fig.~\ref{fig:overview-example}(a), the query in the second phase (at line 6) depends on the query results in the first phase stored in $a$ at line 5, and the variable $a$ also relies on the queries at line 3. Correspondingly, we have two edges $(l^6, a^5)$ and $(a^5, x^3)$ in our execution-based dependency graph in Fig.~\ref{fig:overview-example}(b). Besides, we also have special edge which is a circle, to track any variable being updated with its previous value recursively. For instance, the counter $j$ and the variable $a$ are updated based on previous values $k$ times in the first phase and we see two circle edges on $a^5$ and $j^4$.}

The central property we are after in this work is the \emph{adaptivity of a program}. We define formally this notion in three steps (details in Section~\ref{sec:adaptivity}). First, we define a notion of dependency, or better \emph{may-dependency}, between variables. To do this we take inspiration from previous works on dependency analysis and information flow control and we say that a variable \emph{may depend} on another one if changing the execution of the latter can affect the execution of the former. 
We can see in Fig.~\ref{fig:overview-example}(a) that the value of the variable $l$, which corresponds to the result of the execution of the query in the second phase (in the command with label 6), is affected by the value of the variable $x$, which corresponds to the result of the execution of the query at line 3 in the first phase, via the variable $a$.
To formally define this notion of dependency, as in information flow control, we use the execution history of programs recorded by a trace semantics (see Definition~\ref{def:var_dep}).
% \mg{Please, double check that I refer to the right definition. }  

Second, we build an annotated weighted directed graph representing the possible dependencies between labeled variables. We call this graph the \emph{semantics-based dependency graph} to stress that this graph summarizes the dependencies we could see if we knew the overall behavior of the program. 
The vertices of the graph are the assigned program variables with the label of their assignments, edges are pairs of labeled variables which satisfy the dependency relations, weights are functions associated with vertices and describe the number of times the assignment corresponding to the vertex is executed when the program is run in a given starting state\footnote{In our trace semantics the state is recorded in the trace, so an initial state is actually represented by an initial trace. We will use this terminology in later sections.}, and the annotations, which we call \emph{query annotations}, are bits associated with vertices and describe if the corresponding assignment comes from a query (1) or not (0).
The \emph{semantics-based dependency graph} of the $\kw{twoRounds(k)}$ program
we gave in Fig.~\ref{fig:overview-example}(a) is described in Fig.~\ref{fig:overview-example}(b) (we use dashed arrows for two edges that will be highlighted in the next step, for the moment these can be considered similar to the other edges---i.e. solid arrows).
\review{More explanation of the semantics-based dependency graph in Figure 3, including the lambda tau. rho (tau) k notation. Also, consider replacing it with notation that is easier to search using PDF readers such as lastVal(tau, k).}
We have all the variables that are assigned in the program with their labels, and edges representing dependency relations between them. 
For example, we have two edges $(l^6, a^5)$ and $(a^5, x^3)$ describing the dependency between the variables assigned by queries. The vertices $l^6$ and $x^3$ are the only ones with query annotation $1$ (the subscript), since they are the only two variables that are in assignments involving  queries. Notice that the graph contains cycles---in this example it contains two self-loops. These cycles capture the fact that the variables $a^5$ and $j^4$ are updated at every iteration of the loop using their previous values. Cycles are essential to capture mutual dependencies like the ones that are generated in loops. Adaptivity is a quantitative notion, so capturing this form of dependencies is not enough. 
This is why we also use weights. \highlight{The weight of a vertex is a function that given an initial state returns a natural number representing 
the number of times this vertex is visited during the program execution starting in this initial state.  }
% \highlight{To do: more explanation about weight, {$\lambda \trace.1$}, and replacing it}
For example, the vertex $l^{6}$ has weight {$\lambda \trace.1$} since for every initial state {$\trace$} the corresponding statement will be executed one time. The vertex $a^5$ on the other hand has weight {$\lambda \trace. \env(\trace) k$ since the corresponding assignment will be executed a number of times that correspond to the value of $k$ in the initial state $\trace$, and $\env$ is the operator reading value of $k$ from $\trace$.
}

% It is a function which takes an initial state, $\trace_0$ as input,
% then executes the program, and counts the evaluation times of the query request $\clabel{\assign{l}{\query(\chi[k]*a)} }^{6}$ during the execution.
% % returns $1$ for every starting state, since 
% Since this query at line $6$ is outside of any loop, we are expecting this function always return the count $1$ given any initial state.
% The query annotation of this vertex is $1$, which  indicates that 
% $\clabel{\assign{l}{\query(\chi[k] * a)}}^6$ is a query request.
% For another vertex, $a^{5}:{}^{w_{a^{5}}}_0$ in the while loop, we expect its weight function
% returns different counts if the input initial traces have different initial value for $k$.
% Because $\clabel{\assign{a}{x + a}}^{5}$ will be executed different times if the input $k$  is different.
% Its subscript $0$ representing this is a non-query assignment.



% Besides, we also have special edge which is a circle, to track any variable being updated with its previous value recursively. 
% For instance, the loop counter $j$ and the variable $a$ are updated based on previous values $k$ times in the first phase and we see two circle edges on $a^5$ and $j^4$.

%%%%%%%%%%%%%%%%%%%%%%%%%%%%%%%%%%% Details Below that might be useful when make passes %%%%%%%%%%%%%%%%%
% \detailed{The existence of circle edge \jl{(there isn't a name 'circle edge', the terminology is cycle)}
%  allows our graph to express situation when a variable relies on its previous value recursively inside a while loop, but not show how many times of this reliance, which is necessary to define adaptivity. For instance, if we modify our two round example a little bit to make the query $query(\chi[j]\dot \chi[k])$ at line $3$ relies on its previous result to $query(\chi[j]\dot\chi[k] + x)$, then intuitively its adaptivity becomes $k+1$. To this end, we add quantitative information to our graph: weight on every vertex.
% The weight of a vertex is a function that given a starting state returns a natural number representing 
% the number of times the vertex is visited when the program is executed starting from this state.}
% \jl{The existence of cycle
%  allows our graph to handle the while loop.
% When the variable in a while loop relies on its value in the previous iterations, the cycle expresses this reliance.
% But it cannot express the times of this reliance.
% For instance, if we modify the command $3$
% of the $\kw{twoRounds(k)}$ example
% into $\clabel{\assign{x}{\query(\chi[j] \cdot \chi[k] + x)}}^3$. 
% Then $x$ in every iteration relies on the result in the previous iteration
% and the intuitive adaptivity becomes $k+1$. But we don't know the number $k$ by only constructing the edge $x^3 \to x^3$.
% To this end, we add quantitative information to our graph: weight on every vertex.
% The weight of a vertex is a function that given a starting state returns a natural number representing 
% the number of times the vertex is visited during the program execution.
% }
% Each vertex in this graph has a superscript representing its weight, and a subscript $1$ or $0$ telling if the vertex corresponds to a query or not. We will call this subscript a query annotation. 
% For example, in Fig.~\ref{fig:overview-example}(b), the vertex $l^{6}:{}^{w_1}_1$, 
% has weight $w_1$, a constant function which returns $1$ for every starting state, since 
% this query at line $6$ is at most executed once regardless of the initial trace.
% The query annotation of this vertex is $1$, which  indicates that 
% $\clabel{\assign{l}{\query(\chi[k] * a)}}^6$ is a query request.
% Another vertex, $x^{3}:{}^{w_k}_1$, appears in the while loop. 
% It has as weight a function $w_k$ that for every initial state returns the value that $k$ has in this state, since this is also the number the while loop will be iterated. 
% The node $j^{4}:{}^{w_k}_0$ has as a subscript $0$ representing a non-query assignment.
% \jl{
% For example, in Fig.~\ref{fig:overview-example}(b), the vertex $l^{6}:{}^{w_{l^{6}}}_1$, 
% has weight ${w_{l^{6}}}$. It is a function which takes an initial state, $\trace_0$ as input,
% then executes the program, and counts the evaluation times of the query request $\clabel{\assign{l}{\query(\chi[k]*a)} }^{6}$ during the execution.
% % returns $1$ for every starting state, since 
% Since this query at line $6$ is outside of any loop, we are expecting this function always return the count $1$ given any initial state.
% The query annotation of this vertex is $1$, which  indicates that 
% $\clabel{\assign{l}{\query(\chi[k] * a)}}^6$ is a query request.
% For another vertex, $a^{5}:{}^{w_{a^{5}}}_0$ in the while loop, we expect its weight function
% returns different counts if the input initial traces have different initial value for $k$.
% Because $\clabel{\assign{a}{x + a}}^{5}$ will be executed different times if the input $k$  is different.
% Its subscript $0$ representing this is a non-query assignment.
%
%It has as weight a function $w_k$ that for every initial state returns the value that $k$ has in this state, since this is also the number the while loop will be iterated. 
% The node $j^{4}:{}^{w_k}_0$ has as a subscript $0$ representing a non-query assignment.
% }
%%%%%%%%%%%%%%%%%%%%%%%%%%%%%%%%%%% Details Below that might be useful when make passes %%%%%%%%%%%%%%%%%
% \detailed{Since the edges between two vertices represent the fact that one program variable may depend on the other,
% we can define the program adaptivity with respect to a initial trace by means of a walk traversing the graph, visiting each vertex no more than its weight with respect to the initial trace, and visiting as many query nodes as possible.
% Still, look again at our example, we can see that
% in the walk along the dotted arrows,  $l^{6} \to a^5 \to x^3 $, there are $2$ vertices with query annotation $1$ and that this number is maximal, i.e. we cannot find another walk having more than $2$ vertices with query annotation $1$, under the assumption that $k \geq 1$. So the adaptivity of the program in Fig.~\ref{fig:overview-example}(a)  is $2$,
% as expected.
% }
Third, we can finally define adaptivity using the semantics-based dependency graph. We actually define this notion with respect to an initial state $\tau$, since different states can give very different adaptivities.  
\highlight{rephrase the next sentence.}We consider 
% the longest walk  that visits each vertex $v$ of the semantics-based dependency graph no more than the value that the weight $w_v$ assign to $\tau$, and visits as many query nodes as possible. 
any  walk  that visits a vertex $v$ of the semantics-based dependency graph no more than the value that the vertex's weight $w_v$ associates to the initial state $\tau$, and that visits a maximal number of query vertices.
The number of query vertices visited is the adaptivity of the program with respect to $\tau$.
In Fig.~\ref{fig:overview-example}(b), assuming that $\tau(k) \geq 1$, we can see that the 
walk along the dashed arrows,  $l^{6} \to a^5 \to x^3 $ has two vertices with query annotation $1$, and we cannot find another walk having more than $2$ query vertices, although there is another walk, $l^{6} \to x^3 $, which has $2$ query vertices. So the adaptivity of the program in Fig.~\ref{fig:overview-example}(a) with respect to $\tau$ is $2$. If we consider an initial state $\tau$ such that $\tau(k)=0$ we have that the adaptivity with respect to $\tau$ is instead $1$. 
%%%%%%Gap: %%%%%%%%%%%%%%%%%%%%%%%%%%%%%%%%%%%%%%%%%%%%%%%%%%%%%%%%%%%%%%%%%%%%%%%%%%%%%%%%%%%%%%%%%%%%%%%%%%%%%%%%%%%%%%%%%%%%%%%%%%%%%%%%%%%%%%%
% \begin{figure}
%     \centering   
%     \includegraphics[width=1.0\textwidth]{architecture.png}
%     \vspace{-0.8cm}
%   \caption{High level architecture}
%     \label{fig:structure}
%     \vspace{-0.6cm}
% \end{figure}

\subsubsection{Static analysis}
%%%%%%%%%%%%%%%%%%%%%%%%%%%%%%% Previous Version Below that might be useful when make passes %%%%%%%%%%%%%%%%%
%  \detailed{The definition of adaptivity comes from the aforementioned execution-based dependency graph, 
%  our static analysis statically provides a sound upper bound on this adaptivity, via constructing another weighted graph, we call it estimated dependency graph. The upper bound is then found by searching a sound path with respect to our adaptivity in the generated graph. Different from the execution-based dependency which needs the trace from the execution, estimated one is built by our static analysis algorithm which takes the program itself as input. In brief, our algorithm is consist of a graph-generation algorithm, a weight computation algorithm and finally a path searching algorithm in the generated weighted graph.
%  }
%%%%%%%%%%%%%%%%%%%%%%%%%%%%%%% Previous Version Below that might be useful when make passes %%%%%%%%%%%%%%%%%
% \todo{In order to have a sound and accurate upper bound on the  adaptivity of a program $c$,
% we design a program analysis framework named {\THESYSTEM}.
% This framework composes two algorithms as shown in the double-stroke box and the dashed box in Fig.~\ref{fig:adaptfun}.
% The first algorithm in the double-stroke box combines the quantitative and dependency analysis techniques.
% It produces an estimated \emph{dependency graph} for a program.
% The second algorithm in the dashed box is a walk length estimation algorithm.
% It computes the upper bound on the program's \emph{adaptivity} over the estimated graph.}
% \jl{Since the definition of adaptivity comes from the aforementioned execution-based dependency graph, 
%  our static analysis statically provides a sound upper bound on this adaptivity via approximating this graph. The estimated graph is called \emph{estimated dependency graph}. 
%  The upper bound is then computed by searching the walk in this graph such that it can give a sound bound on the adaptivity.
%  Different from the execution-based dependency graph, the estimated one is produced by our static anlaysis algorithm, which only takes the program as input and does not rely on the execution history.
%  In brief, our algorithm is consist of a weighted graph-generation algorithm and a adaptivity computation algorithm over the graph.
%  }
 
 %%%%%%%%%%%%%%%%%%%%%%%%%%%%%%% Previous Version Above for Reference  %%%%%%%%%%%%%%%%%
To compute statically a sound and accurate upper bound on the \emph{adaptivity} of a program $c$,
we design a program analysis framework named {\THESYSTEM} (formally in Section \ref{sec:algorithm}). 
The structure of {\THESYSTEM} (Fig.~\ref{fig:adaptfun}) reflects in part the definition of adaptivity we discussed. Specifically, {\THESYSTEM} is composed by two algorithms (the ones in dashed boxes in the figure), one for building a dependency graph, called \emph{estimated dependency graph}, and the other to estimate the adaptivity from this graph.  
The first algorithm generates the \emph{estimated dependency graph} using several program analysis techniques. Specifically,
 {\THESYSTEM} extracts the vertices and the query annotations by looking at the assigned variables of the program, it estimates the edges by using control flow and data flow analysis, and it estimates the weights by using symbolic reachability-bound analysis---weights in this graph are symbolic expressions over input variables. 
% This combined analysis allow us to obtain more accurate upper bounds than what we would obtain by using any of these single analysis technique in isolation.
The second algorithm estimates the
% longest 
walk which respects the weights and which visits the maximal number of query vertices.
%  as possible. 
The two algorithms together gives us an  upper bound on the program's \emph{adaptivity}.

 \begin{figure}
  \centering    
\includegraphics[width=1.0\columnwidth]{adapfun.png}
  \vspace{-0.8cm}
  \caption{The overview of {\THESYSTEM}}
  \label{fig:adaptfun}
  \vspace{-0.5cm}
\end{figure}

 
%%%%%%%%%%%%%%%%%%%%%%%%%%%%%%%%%%% Details Below that might be useful when others are making passes %%%%%%%%%%%%%%%%%
%   \detailed{Fig.~\ref{fig:overview-example}(c) is the resulting estimated graph of our static analysis algorithm which consumes the program in Fig.~\ref{fig:overview-example}(a).The edges are generated by our graph generation algorithm which combines control flow analysis and data flow analysis, presented in Section~\ref{sec:alg_edgegen}). We can easily see the generated graph in Fig.~\ref{fig:overview-example}(c) is a safe approximation of its execution-based counterpart in Fig.~\ref{fig:overview-example}(b), in the way that we can find a corresponding edge in Fig.~\ref{fig:overview-example}(c) for all the edges in Fig.~\ref{fig:overview-example}(b). We call the weight of every vertex computed by our algorithm as estimated weight,  }
%   estimated by using a reachability-bound estimation algorithm (presented in Section~\ref{sec:alg_weightgen}). \detailed{Different from the execution-based weight $w_1$ or $w_k$ in Fig.~\ref{fig:overview-example}(b) which is a function whose output relies on the initial trace, our estimated weight} can be symbolic and provide a sound upper bound on its execution-based weight of the corresponding vertex in the execution-based dependency graph. For instance, 
%   the estimated weight $k$ of the vertex $x^{3}$ in Fig.~\ref{fig:overview-example}(c) is a sound upper bound on the execution-based weight $w_k$ of vertex $x^{3}$ in Fig.~\ref{fig:overview-example}(b), with the same starting trace $\trace$, $w_k(\trace) \leq\trace(k)$. $\trace(k)$ means getting the value of variable $k$ in the trace $\trace$. The soundness of this step is proved in Theorem~\ref{thm:addweight_soundness}.   
%
We show in Fig.~\ref{fig:overview-example}(c) the estimated dependency graph that our static analysis algorithm returns for the program $\kw{twoRounds(k)}$ in Fig.~\ref{fig:overview-example}(a).
Vertices and query annotations are the same as the ones in Fig.~\ref{fig:overview-example}(b), simply inferred by scanning the program.
 Every edge in Fig.~\ref{fig:overview-example}(b) is precisely inferred by our combined data flow and control flow analysis, this is why Fig.~\ref{fig:overview-example}(c) contains exactly the same edges.
The weight of every vertex is computed using a reachability-bound estimation algorithm which outputs a symbolic expression over the input variables, representing an upper bound on the number of times each assignment is executed.
% \wq{symbolic and provide a sound upper bound on its execution-based weight of the corresponding vertex in the execution-based dependency graph.
% $w_k(\trace) \leq \trace(k)$. $\trace(k)$ means getting the value of variable $k$ in the trace $\trace$. The soundness of this step is proved in Theorem~\ref{thm:addweight_soundness}.}
For example, consider the vertex $x^{3}$, its weight is $k$ and this provides an upper bound on the value returned by the weight function $\lambda \trace. \rho(\trace)k$ associated with vertex $x^{3}$ in Fig.~\ref{fig:overview-example}(b) for any initial state. 
% Indeed, 
% for any initial trace $\trace_0$, when $w_{x^{3}}(\trace_0)$ executes the program and counts the
% execution times of command $3$,
% we expect that this counts is at most the the loop iterations, i.e. $k$'s initial value from $\trace_0$.

The algorithm searching for the walk first finds a path $l^6:{}^1_1 \to a^5: {}^k_0 \to x^3: {}^k_1$, and then constructs a walk based on this path. Every vertex on this walk is visited once, and the number of vertices with query annotation $1$ in this walk is $2$, which is the upper bound we expect.
{It is worth noting here that $x^3$ and $a^5$ can only be visited once because there isn't an edge to go back to them, even though they both have the weight $k$}.  So the algorithm $\pathsearch$ computes the upper bound $2$ instead of $2k+1$. Note that $2$ is not always tight, for example when $k = 0$.
% In this sense, instead of simply computing the weighted length of this path ($2k+1$) as adaptivity, the algorithm $\pathsearch$ computes the upper bound $2$. Note that $2$ is not always tight, for example when $k = 0$.
% \todo{Can you double check if this is clear?}
% \mg{I think we should add a sentence to say that this bound is actually not always tight.}


\section{Loop language }
\label{sec:loop_language}
In this section, we formally introduce the language we will focus on for writing data analyses.  This is a simple loop language with some primitives for calling queries. After defining the syntax of the language and showing an example, we will define its trace-based operational semantics. This is the main technical ingredient we will use to define the program's adaptivity. We will conclude this section by discussing the limitation of this language with respect to static analysis for adaptivity.

\subsection{Syntax}
\label{subsec:loop-syntax}
{\small
\begin{figure}
\[
\begin{array}{llll}
%  \mbox{Arithmatic Operators} & \oplus_a & ::= & + ~|~ - ~|~ \times 
% %
% ~|~ \div \\  
%   \mbox{Boolean Operators} & \oplus_b & ::= & \lor ~|~ \land ~|~ \neg\\
%   %
%   \mbox{Relational Operators} & \sim & ::= & < ~|~ \leq ~|~ == \\  
%  \mbox{Label} & l & := & \mathbb{N} \\ 
%  \mbox{While Map} & w & \in & \mbox{Label} \times \mathbb{N} \\
\mbox{Arithmetic Expressions} & \aexpr & ::= & 
	%
	n ~|~ x ~|~ \aexpr \oplus_a \aexpr ~|~ \\
% \sep \pi (l , \aexpr, \aexpr) \\
    %
\mbox{Boolean Expressions} & \bexpr & ::= & 
	%
	\etrue ~|~ \efalse  ~|~ \neg \bexpr
	 ~|~ \bexpr \oplus_b \bexpr
	%
	~|~ \aexpr \sim \aexpr \\
\mbox{Expressions } & \expr & ::= & \aexpr \sep \bexpr \sep \chi\sep [] ~|~ [\expr, \dots, \expr] ~|~ \chi[\aexpr] ~|~ x[\aexpr]\\
\mbox{Values } & v & ::= & n \sep \etrue \sep \efalse \sep \chi \sep [] ~|~ [v, \dots, v] ~|~ \chi[v] \\
\mbox{Commands} & c & ::= &  \eskip  ~|~  \assign x \expr ~|~  \assign{x}{ q(e)}
%
~|~ \eloop ~ \aexpr  ~ \edo ~ c  ~|~ c;c  ~|~ \eif(\bexpr, c, c) 	 
	\\
%\mbox{Variables} & \mathcal{VAR}  & ::= & \{ {x} \} \\
%
% \mbox{Trace} & t & ::= & [] ~|~ [(q, v)^{(l, w) }] ~|~ t ++ t
\end{array}
\]
    \vspace{-0.3cm}
 \caption{Syntax of Loop language}
    \label{fig:syntax_highlevel}
    \vspace{-0.5cm}
\end{figure}
}
%
We introduce the syntax of the {\tt Loop} language we use to write our data analyses.
%expression
It is standard that expressions can be either arithmetic expressions or boolean expressions.
An arithmetic expression can be a  constant $n$ denoting integer, a variable $x$ from some countable set $\tt Var$, a combination of arithmetic expressions by means of the symbol $\oplus_a$, denoting basic operations including addition, product, subtraction, etc.
%
A boolean expression can be {\tt true} or {\tt false}, the negation of
a boolean expression, or a combination of boolean expressions by means of $\oplus_b$, denoting basic boolean connectives, or the result of some basic comparison $sym$ between arithmetic expressions, e.g., $\leq,=,<,$ etc. 
Besides, the expression also includes the special variable $\chi$ representing a row of the database, and access to values at a certain index in $\chi$, as $\chi[\aexpr]$. Additionally, list over expressions is supported and $[]$ stands for the empty list. The access to elements in the list can be achieved through $x[\aexpr]$ when variable $x$ is referred to a list. The value $v$ now contains the natural number $n$, the boolean primitives $\etrue$ and $\efalse$, the special row $\chi$ and access to it $\chi[v]$, the empty list $[]$ and non-empty list $[v, \dots, v]$.
% 
%

  A command $c$ can either be $\eskip$, an assignment command $\assign{x}{\expr}$, the composition of two commands $c;c$, an if statement $\eif(\bexpr, c, c)$, a loop statement  $\eloop ~ \aexpr ~ \edo ~ c $.
 The main novelty of the syntax is the query request command $\assign{x}{q(\expr)}$. As a reminder, the aforementioned tight bound of adaptive data analysis in Section~\ref{sec:overview} focuses on linear queries, specified by a function from rows to $[0,1]$ or $[-1,+1]$. To express these functions, we introduce the special variable $\chi$ to represent the rows of the database in the arithmetic expressions. In this sense, a simple linear query which returns the first element of the row is written as $q(\chi[1)])$, representing a query $q(\chi) = \chi(1)$. The query can also take variables as input, the aforementioned two round example in Figure~\ref{fig:simpl-two-round-graph}.(a) uses the loop counter $i$ to construct the query $q(\chi[i])$, and a more complicated one $q(\chi[4]+a)$ at the end.  
 
 
% The query $q(\expr)$ in this command is abstract, and the expression $\expr$ inside the query stores the information of the elements used during the construction of the query. This kind of abstraction of query helps in our analysis and still remain expressive enough for adaptive analysis algorithms. 


% \mg{I don't think that this corresponds exactly to our approach. I think that we want to focus on linear queries, these are the ones for which we have the bounds. A linear query is specified by a function from rows to [0,1] or [-1,+1]. So, in some sense, we want to have a language that describes these functions. Cannot we use $q(r)=e$ where $r$ is a special variable denoting the given row, and $e$ is an expression as we have right now? Example: $q_j(x)=x(i)\cdot x(j)$ can be written as $q(\chi (i)\cdot \chi (j))$ which is a notation for
% $q=\lambda \chi \chi (i)\cdot \chi (j)$.}

%   We have seen a simplified version of the two round algorithm in Section~\ref{sec:overview}. We show its complete version $TRC$ expressed in our {\tt Loop} language on the left hand side in Figure~\ref{fig:tworound_complete}.
 
 
 
 
%  \subsection{An example}
%  \mg{I will not change this section because I think we need to think more about what our language of queries is. As I said above, I think we should be more explicit. For example, I would write $q_1$ as $q(x[j]\cdot x[i])$ or something similar. I also don't think it is a good idea to present Algorithm~\ref{alg:two_round} first, and then show how to write this in our language. We can use two-rounds to introduce our language but I would present it directly in our language without the algorithm. I see several issues in using the algorithm: first, you need to present the example twice, first for the algorithm and then the program; second, the algorithms is not much more real world than other examples - we took it from a research paper which actually now put it in the appendix. I suggest we present the example but only as an example, withouth the algorithm.}

%  We go through a real-world adaptive data analysis algorithm in Algorithm~\ref{alg:two_round}, and see how our high level loop language expresses it. 
 

 
%  \begin{algorithm}
% \caption{A two-round analyst strategy for random data}
%  \label{alg:two_round}
% \begin{algorithmic}
% \REQUIRE Mechanism $\mathcal{M}$ with a hidden state $X\in \{-1,+1\}^{n\times (k+1)}$.
% \STATE  {\bf for}\ $j\in [k]$\ {\bf do}.  
% \STATE \qquad {\bf define} $q_j(x)=x(j)\cdot x(k)$ where $x\in \{-1,+1\}^{k+1}$.
% \STATE \qquad {\bf let} $a_j=\mathcal{M}(q_j)$ 
% \STATE \qquad \COMMENT{In the line above, $\mathcal{M}$ computes approx. the exp. value  of $q_j$ over $X$. So, $a_j\in [-1,+1]$.}
% \STATE {\bf define} $q_{k+1}(x)=\mathrm{sign}\big (\sum_{i\in [k]} x(i)\times\ln\frac{1+a_i}{1-a_i} \big )$ where $x\in \{-1,+1\}^{k+1}$.
% \STATE\COMMENT{In the line above,  $\mathrm{sign}(y)=\left \{ \begin{array}{lr} +1 & \mathrm{if}\ y\geq 0\\ -1 &\mathrm{otherwise} \end{array} \right . $.}
% \STATE {\bf let} $a_{k+1}=\mathcal{M}(q_{k+1})$
% \STATE\COMMENT{In the line above,  $\mathcal{M}$ computes approx. the exp. value  of $q_{k+1}$ over $X$. So, $a_{k+1}\in [-1,+1]$.}
% \RETURN $a_{k+1}$.
% \ENSURE $a_{k+1}\in [-1,+1]$
% \end{algorithmic}
% \end{algorithm}

% As described before, the complete version still has two steps. The query asked in the first step now depends on the iteration number so that the query ask at the $j$ iteration is ($q_j(x) = x(j)\cdot x(k)$), expressed as $q(\chi(j)\cdot \chi(k))$. The iteration counter is initialized to 0, $j = 0$. In the second step, the final query is more complicated. It uses an auxiliary function   $\mathrm{sign}(y)=\left \{ \begin{array}{lr} +1 & \mathrm{if}\ y\geq 0\\ -1 &\mathrm{otherwise} \end{array} \right . $ The input of this function is $\sum_{i\in [k]} \chi(i)\times\ln\frac{1+a[i]}{1-a[i]}$, which sums the product of $\chi(i)$ and $\ln\frac{1+a[i]}{1-a[i]}$ that uses the value at index $i$ of the list $a$.
% \begin{figure}
% \[
% {
% \begin{array}{l}
% \bf{TRC}(k) \\
%     % \left[j \leftarrow 0 \right]^1 ; \\
%     \\
%   \clabel{ a \leftarrow []}^{1} ; \\
%     \clabel{\assign{j}{0} }^{2} ; \\
%     \eloop ~ \clabel{k}^{3} ~ \edo ~ \\
%     \Big(
%      \clabel{x \leftarrow q(\chi(j)\cdot \chi(k)) }^{4}  ; \\
%      \clabel{\assign{j}{j+1}}^{5} ;\\
%     \clabel{a \leftarrow x :: a}^{6}       \Big);\\
%     \clabel{l \leftarrow q(\mathrm{sign}\big (\sum_{i\in [k]} \chi(i)\times\ln\frac{1+a[i]}{1-a[i]} \big ))}^{7}\\
% \end{array} ~~
% {
% \begin{array}{l}
% \bf{TRC^{ssa}}(k) \\
%     % \left[j \leftarrow 0 \right]^1 ; \\ 
%     \\
%     \clabel{a_1 \leftarrow []}^{1} ; \\
%     \clabel{\assign{j_1}{0} }^{2} ; \\
%     \eloop ~ \clabel{k}^{3}, 0, ~ \edo [(j_3, j_1,j_2),(a_3, a_1,a_2)]~ \\
%     \Big(
%     \clabel{ x_1 \leftarrow q(\chi(j_3)\cdot \chi(k))}^{4}  ; \\
%     \clabel{ \assign{j_2}{j_3+1} }^{5} ;\\
%     \clabel{a_2 \leftarrow x_1 :: a_3}^{6}       \Big);\\
%     \clabel{l_1 \leftarrow q(\mathrm{sign}\big (\sum_{i\in [k]} \chi(i)\times\ln\frac{1+a_3[i]}{1-a_3[i]} \big ))}^{7}\\
% \end{array}
% }
% }
% \]  
%     \caption{Two round algorithm complete version}
%     \label{fig:tworound_complete}
% \end{figure}
% The second step of the Algorithm~\ref{alg:two_round} is to use the previous results $a_j$ from mechanism $\mathcal{M}$ for $j \in [k]$ to construct a complicated query $q_{k+1}$. In the high level language, we abstract this complex query as $q_2(a)$, $q_2$ representing queries of this complex kind of form and the argument $a$ is what we need for our analysis.

% In a word, we go through a two-round adaptive analysis algorithm and shows how we can represent it in the {\tt Loop} language. It reveals the expressiveness of this language for most of adaptive data analysis algorithms. 

\subsection{ Trace-based Operational Semantics}
 We evaluate programs in our {\tt Loop} language by means of a trace-based operational semantics, to capture the dependency between queries. For distinguishing elements in the the trace, we add a label to commands in the {\tt Loop} language as follow:
% syntax
% \mg{Change "while map" to "Loop maps" everywhere.}
% \dg{Are you sure that a label map is really of type $\mbox{Label} \times \mathbb{N}$ and not $\mbox{Label} \to \mathbb{N}$? I fail to understand how a single pair of label and $\mathbb{N}$ can represent nested loops. } \wq{It is a map}
%
\[
\begin{array}{llll}
     \mbox{Labeled commands} & c & ::= &   [\assign x \expr]^{l} ~|~  [\assign x q(e)]^{l}
 ~|~  \eloop ~ [\aexpr]^{l} ~ \edo ~ c  ~|~ c;c  ~|~ \eif([\bexpr]^l, c, c) 	 ~|~ [\eskip]^{l} \\
\end{array}
\]

Each command is now labeled with a label $l$, a natural number standing for the line of code where the command appears. Notice that we associate the label $l$ to the conditional predicate $\bexpr$ in the if statement, and to the loop counter $\aexpr$ in the loop statement. We will also use  Loop Maps $w$ as defined below.  
% implicitly this gives us a control flow graph representation for the program,
% which is useful when we define adaptivity in the later section. 
% t, m, w, explanation
\[
\begin{array}{llll}
 \mbox{Loop Maps} & w & \in & \mbox{Label} \to \mathbb{N} \\
% \mbox{Labelled commands} & c & ::= &   [\assign x \expr]^{l} ~|~  [\assign x q(e)]^{l}
%  ~|~  \eloop ~ [\aexpr]^{l} ~ \edo ~ c  ~|~ c;c  ~|~ \eif([\bexpr]^l, c, c) 	 ~|~ [\eskip]^{l} \\
% 	\\ ~|~ [\eswitch( \expr, x, v_i \to  q_i)]^{l}
	%
% \mbox{Binary Operation} & \bop & ::= & + ~|~ - ~|~ \times %
% %
% ~|~ \div ~|~ < ~|~ \leq ~|~ = \\
% %
% \mbox{Unary Operation} & \uop & ::= & \ln ~|~ - \\
% %
% \mbox{Memory} & m & ::= & \emptyset ~|~ (x \to v) :: t \\
%
\mbox{Annotated Query} & \mathcal{AQ}  & ::= & \{ q(v)^{(l,w)}  \} \\
\end{array}
\begin{array}{llll}
    \mbox{Memory} & m & ::= & [] ~|~ m[x \to v] \\
\mbox{Trace} & t & ::= & [] ~|~ q(v)^{(l, w) } :: t \\
\end{array}
\]

% \mg{I suggest to move the definitions of memories to the section on semantics.}
  Loop maps are map from labels $l$ to iteration number $n$.
%   Because statements in the loop share the same line number,  varied iterations , the label $l$ is not enough to distinguish statements.
  A  mapping $[k \to n]$ gives accurate information on which loop a statement is in by its key $k$ (label at loop counter), and which iteration $n$ the statement belongs to. For example, the loop maps $w=[3:1, 4:2]$ indicates that the statement is currently in a nested loop, the outer loop starting from label $3$ and in its first iteration, the statement is now in the inner loop starting from label $4$ and in the second iteration. We use $\emptyset$ to represent an empty map, indicating the statement is not in any loop. We define operations on $w$ as follows.
%  We can understand the annotation in such a way. The label $l$ locates arbitrary query request when we look at the execution path of a program with no loop. However, when loop repeats queries in its body, these query requests from varied iterations share the same line number. Hence, the label $l$ is not enough to distinguish queries in loops. For this purpose, another symbol is needed to represent the iteration number of a loop.
% A simplified approach is to use natural number $n$ for the iteration number so that a pair $(l, n)$ can help, but it fails to support nested loop. This accounts for the appearance of loop maps $w$ into the annotation $(l,w)$. As a map from label $l$ to the iteration number n (a natural number), $w$ gives accurate information on which loop specified by its label and which iteration $n$ the query belongs to. 
\[
\begin{array}{lll}
w \setminus l     & = w  & l \not\in Keys(w)   \\
     & = w_l & Otherwise \\
\end{array}
\begin{array}{llll}
   w + l & = w[l \to 1] & l \not \in Keys(w) \\   
     & = w [l \to w(l)+1] & Otherwise
\end{array}
\]
We use $w \setminus l$ to remove the mapping of the key $l$ from the loop maps $w$. This is used when exiting the loop at line $l$. The special loop maps $w_l$ expresses a map identical to $w$, but without the mapping of label $l$. We record in $w$ the first iteration of a loop marked by label $l$ by assigning $l$ with the iteration $1$. The mapped number increase when going into another iteration of the same loop. We use $Keys(w)$ to return all the keys of the loop maps $w$.

%%% trace, queries
 A memory is standard, a map from variables to values. Queries can be uniquely annotated as $\mathcal{AQ}$, and the annotation $(l,w)$ considers the location of the query by line number $l$ and which iteration the query is at when it appears in a loop statement, specified by $w$. A trace $t$ is a list of annotated queries accumulated along the execution of the program. 
 
 %% trace
A trace can be regarded as the program history, where this history consists of the queries asked by the analyst during the execution of the program. We collect the trace with a trace-based small-step operational semantics based on transitions of the form $ \config{m,c, t, w} \to \config{m', \eskip, t', w'} $. It states that a configuration $\config{m, c, t,w}$ evaluates to another configuration with the trace and loop maps updated along with the evaluation of the command $c$ to the normal form of the command $\eskip$.  A configuration contains four elements: a memory $m$, the command $c$ to be evaluated, a starting trace $t$, a starting loop maps $w$. Most of the time, the loop maps remains empty until the evaluation goes into loops.  We also have the evaluation of arithmetic expressions of the form $\config{m,\aexpr} \aarrow \aexpr' $, evaluating an arithmetic expression $\aexpr$ in the memory $m$, and similar for the boolean expressions $\config{m, \bexpr} \barrow \bexpr'$.   

%
% figure, evaluation rules.
{\footnotesize
\begin{figure}
\begin{mathpar}
\boxed{ \config{m, c, t,w} \xrightarrow{} \config{m', c',  t', w'} \; }
\and
%
{\inferrule
{
 \valr_N > 0
}
{
\config{m, \eloop ~ [\valr_N]^{l}  ~ \edo ~ c ,  t, w }
\xrightarrow{} \config{m, c ;  \eloop ~ [(\valr_N-1)]^{l} ~ \edo ~ c ,  t, (w + l) }
}
~\textbf{low-loop}
}
%
\and
%
\inferrule
{
}
{
\config{m, [\eskip]^{l} ; c_2,  t,w} \xrightarrow{} \config{m, c_2,  t,w}
}
~\textbf{low-seq2}
%
\quad
%
{
\inferrule
{
 \valr_N = 0
}
{
\config{m,  \eloop ~ [\valr_N]^{l} ~ \edo ~ c  ,  t, w }
\xrightarrow{} \config{m, [\eskip]^{l} ,  t, (w \setminus l) }
}
~\textbf{low-loop-exit}
}
\and
%
\inferrule
{
}
{
\config{m,  \eif([\efalse]^{l}, c_1, c_2),  t,w} 
\xrightarrow{} \config{m, c_2,  t,w}
}
~\textbf{low-if-f}
%
~~
% {  Memory \times Com  \times Trace \times WhileMap \Rightarrow^{} Memory \times Com  \times Trace \times WhileMap}
\inferrule
{
\config{m,\expr} \to \expr'
}
{
\config{m, [\assign{x}{q(\expr)}]^l, t, w} \xrightarrow{}  \config{m, [\assign{x}{q(\expr')}]^l, t, w}
}
~\textbf{low-query-e}
%
\and
%
%
\inferrule
{
\config{m, c_1,  t,w} \xrightarrow{} \config{m', c_1',  t',w'}
}
{
\config{m, c_1; c_2,  t,w} \xrightarrow{} \config{m', c_1'; c_2, t',w'}
}
~\textbf{low-seq1}
~~
\inferrule
{
q(v) = v_q
}
{
\config{m, [\assign{x}{q(v)}]^l, t, w} \xrightarrow{} \config{m[ v_q/ x], \eskip,  t \mathrel{++} [q(v)^{(l,w )}],w }
}
~\textbf{low-query-v}
%
% \inferrule
% {
% }
% {
% \config{m, [\assign x v]^{l},  t,w} \xrightarrow{} \config{m[v/x], [\eskip]^{l}, t,w}
% }
% ~\textbf{low-assn}
%
%
%
\and
%
\inferrule
{
\config{ m, \bexpr} \barrow \bexpr'
}
{
\config{m, \eif([\bexpr]^{l}, c_1, c_2),  t,w} 
\xrightarrow{} \config{m,  \eif([\bexpr']^{l}, c_1, c_2),  t,w}
}
~\textbf{low-if}
%
~~~~
%
\inferrule
{
}
{
\config{m, \eif([\etrue]^{l}, c_1, c_2),t,w} 
\xrightarrow{} \config{m, c_1,  t,w}
}
~\textbf{low-if-t}
%
% %
%
\end{mathpar}
    \vspace{-0.3cm}
    \caption{Trace-based operational semantics}
    \label{fig:evaluation}
    \vspace{-0.5cm}
\end{figure}
}
%
% explanation of rules
We give a selection of rules of the trace-based operational semantics in Figure~\ref{fig:evaluation}.
The rule $\textbf{low-query-e}$ evaluates the argument of a query request. When the argument is in normal form, this query will be answered. The rule $\textbf{low-query-v}$ modifies the starting memory $m$ to $m[v_q/x]$ using the answer $v_q$ of the query $q(v)$ from the mechanism, with the trace expanded by appending the query $q(v)$ with the current annotation $(l,w)$. The rule for assignment is standard and the trace remains unchanged. The sequence rule keeps tracking the modification of the trace, and the evaluation rule for if conditional goes into one branch based on the result of the conditional predicate $\bexpr$. The rules for loop modify the loop maps $w$. In the rule $\textbf{low-loop}$, the loop maps $w$ is updated by $w + l$ because the execution goes into another iteration when the condition $v_N >0$ is satisfied. When $v_N$ reaches $0$, the loop exits and the loop maps $w$ eliminates the label $l$ of this loop statement by $w \setminus l$ in the rule $\textbf{low-loop-exit}$.     
%
\subsection{ Query-based Dependency Graph}
%
We define adaptivity through a query-based dependency graph. In our model, an \emph{analyst} asks a sequence of queries to the mechanism, and the analyst receives the answers to these queries from the mechanism. A query is adaptively chosen by the analyst when the choice of this query is affected by answers from previous queries. In this model, the adaptivity we are interested in is the length of the longest sequence of such adaptively chosen queries, among all the queries the data analyst asks to the mechanism.  Also, when the analyst asks a query, the only information the analyst will have will be the answers to previous queries and the state of the program. It means that when we want to know if this query is adaptively chosen, we only need to check whether the choice of this query will be affected by changes of answers to previous queries. There are two possible situations that can  affect the choice of a query,  
either the query argument directly uses the results of previous queries (data dependency), or the control flow of the program with respect to a query (whether to ask this query or not) depends on the results of previous queries (control flow dependency).

As a first step, we give a definition of when one query may depend on a previous query, which is supposed to consider both control dependency and data dependency. We first look at two possible candidates:
\begin{enumerate}
    \item One query may depend on a previous query if and only if a change of the answer to the previous query may also change the result of the query.
    \item One query may depend on a previous query if and only if a change of the answer to the previous query may also change the appearance of the query.
\end{enumerate}

   The first candidate works well by witnessing the result of one query according to the change of the answer of another query. We can easily find that the two queries have nothing to do with each other in a simple example   
%   but vulnerable to queries request protected by differential privacy mechanisms. In our loop language, a query $q(e)$ represents a query request to the database through a mechanism, which add random noise to protect the return results. In this setting, the results of one query will be randomized due to the noise attached by the mechanism which fails the first candidate because witnessing the results of one query can no longer tells whether the change of the results comes from another query or the change of noise of the differential privacy mechanism. For example, suppose we have a program $p$ which requests two simple queries $q_1()$ and $q_2()$ with no arguments as follows.
     $ p = \assign{x}{q(\chi(1))} ; \assign{y}{q(\chi(2))}$. This candidate definition works well with respect to data dependency. However, if fails to handle control dependency since it just monitors the changes to the answer of a query when the answer of previous queries returned change. The key point is that this query may also not be asked because of an analyst decision which depend on the answers of previous queries. An example of this situation is shown in program $p_1$ as follows.
      \[
      p_1 = \assign{x}{q(\chi(1))} ; \eif( x > 2 ,\assign{y}{q(\chi(2))}, \eskip )
   \]
%   Follow the first definition, we may conclude that $q_2()$ depends on $q_1()$ because the query $q_2()$ may return a different result. Nevertheless, we know that this change of return result of $q_2()$ when we change the value returned by $q_1()$ comes from the hidden mechanism(the noise). So $q_1$ and $q_2$ are independent. 
   We choose the second candidate, which performs well by witnessing the appearance of one query $q(\chi(2))$ upon the change of the result of one previous query $q(\chi(1))$ in $p_1$. It considers the control dependency, and at the same, does not miss the data dependency. In particular, the arguments of a query characterizes it. In this sense, if the data used in the arguments changes due to a different answer to a certain previous query, the appearance of the query may change as well. This situation is also captured by our definition. Let us look at another variant of program $p$, $p_2$, in which the queries equipped with functions using previously assigned variables storing answer of its previous query.
    \[
      p_2 = \assign{x}{q(\chi(2))} ; \assign{y}{q(x+\chi(3))}
   \]
    As a reminder, in the {\tt Loop} language, the query request is composed by two components: a symbol $q$ representing a linear query type and the argument $\expr$, which represents the function specifying what the query asks. So we do think $q(\chi(1))$ is different from $q(\chi(2))$. Informally, we think $q(x+\chi(3))$ may depend on the query $q(\chi(2))$, because equipped function of the former $x+\chi(3)$ depend on the data assigned with $q(\chi(2))$. We can see the appearance definition catches data dependency in such a way, since $q(x+\chi(2))$ will not be the same query if the value of $x$ is changed.    
   
   We give a formal definition of query may dependency based on the trace-based operational semantics as follows.
  % formal definition of IND
  \begin{defn}[Query may dependency ]
One query $q(v_2)$ may depend on its previous query $q(v_1)$ in a program $c$, with a starting loop maps $w$, denoted as
$\mathsf{DEP}(q(v_1)^{(l_1, w_1)}, q(v_2)^{(l_2, w_2)}, c,w,m,D)$ if: 
% \dg{I think the following definition describes when $q(v_2)$ depends on $q(v_1)$, not the other way around as stated in the previous sentence. Also, couldn't we look for $q(v_2)^{(l_2,w_2)}$ in $(t_3-t_1)$ instead of $(t_3-t)$ and then get rid of the "To" relation in the next definition?}
\[
  \begin{array}{l}
     \forall  t. \exists m_1,m_3,t_1,t_3.
\config{m, c,  t,w} \rightarrow^{*} \config{m_1, [\assign{x}{q(v_1)}]^{l_1} ; c_2,
  t_1,w_1} \rightarrow \\ \config{m_1[q(v_1)(D)/x], c_2,
  t_1++[q(v_1)^{(l_1, w_1)}], w_1} \rightarrow^{*} \config{m_3, \eskip,
  t_3,w_3} \\  
  \land 
\Big( q(v_1)^{(l_1,w_1)} \in (t_3-t) \land q(v_2)^{(l_2,w_2)} \in (t_3-t_1) \implies  \exists v \in \codom(q(v_1)), m_3', t_3', w_3'.  \\
 \config{m_1[v/x], {c_2}, t_1++[q(v_1)^{(l_1,w_1)}], w_1} \rightarrow^{*} \config{m_3', \eskip, t_3', w_3'} \land (q(v_2)^{(l_2,w_2)}) \not \in (t_3'-t_1)
\Big)\\
\land 
\Big(q(v_1)^{(l_1,w_1)} \in (t_3-t) \land q(v_2)^{(l_2,w_2)} \not\in (t_3-t_1) \implies  \exists v \in \codom(q(v_1)),  m_3', t_3', w_3'. \\
 \config{m_1[v/x], {c_2}, t_1++[q(v_1)^{(l_1,w_1)}], w_1} \rightarrow^{*} \config{m_3', \eskip, t_3', w_3'} \land (q(v_2)^{(l_2,w_2)})  \in (t_3'-t_1)
\Big)
\end{array}
\]
\end{defn}
 %TODO: some more explanation on def 1
%
% \dg{I have a feeling that something is very off here. Consider the program $p_1$ above. For this program, the definition above will say that there is a dependency between $q(\chi(1))$ and $q(\chi(2))$. Now consider the program $p_1' ~=~ \assign{x}{q(\chi(1))} ; \assign{z}{q(\chi(2))} ; \eif( x > 2 ,\assign{y}{z}, \eskip )$. This new program $p_1'$ is semantically equal to $p_1$, yet the definition above will say that in $p_1'$ there is no dependency between $q(\chi(1))$ and $q(\chi(2))$. So, the notion of dependency defined here does not respect semantic equivalence of programs, which is weird because adaptivity is fundamentally a semantic property.}
% \jl{I put doubt on the semantic equivalence here. If out output of the semantics is memory, I don't think they are semantically equivalent.}
%
We give a formal definition of the query-based dependency graph with the formal definition of may dependency between the two queries above.  
% graph definition
\begin{defn}[Query-based Dependency Graph]
Given a program $c$, a database $D$, a starting memory $m$, an initial loop maps $w$, the query-based dependency graph $G(c,D,m,w) = (V, E)$ is defined as: \\
$V =\{q(v)^{l,w} \in \mathcal{AQ} \mid \forall t. \exists m',  w', t'.  \config{m ,c, t, w}  \to^{*}  \config{m' , \eskip, t', w' }  \land q(v)^{l,w} \in {(t'-t)}  \}$.
\\
$E = \left\{(q(v)^{(l,w)},q(v')^{(l',w')}) \in \mathcal{AQ} \times \mathcal{AQ} 
~ \left \vert ~ \mathsf{DEP}(q(v')^{(l',w')},q(v)^{(l,w)}, c,w,m,D)
 \right.\right\}$.
\end{defn}
%
% The function $\mathsf{To}(q(v')^{(l',w')}, q(v)^{(l,w)}$ tells that the query request $q(v')^{(l',w')}$ appears after the query request $q(v)^{(l,w)}$ in the trace, by comparing the annotation $(l',w')$ and $(l,w)$. It helps to decide on the direction of one edge.
The edge is directed, when an annotated query $q(v)^{(l,w)}$ may depend on its previous query $q(v')^{(l',w')}$, we have the directed
edge $(q(v)^{(l,w)}, q(v')^{(l'.w')})$, from $q(v)^{(l,w)} $ to $q(v')^{(l'.w')}$.

The query-based dependency graph only considers the newly generated annotated queries during the execution of the program $c$, so we see the nodes coming from the trace $t'-t$. The previous trace before the execution of $c$ is excluded when constructing the graph. To summary, for every execution of a program $c$ staring with different configurations, we can construct a corresponding dependency graph. 

% adaptivity definition - longest path
Finally, we reach the definition of adaptivity, by means of the query-based dependency graph. 

\begin{defn}[Adaptivity in {\tt Loop} language]
Given a program $c$, and a memory $m$, a database $D$, a starting loop maps $w$, the adaptivity of the dependency graph $G(c, D,m,w) = (V, E)$ is the length of the longest path in this graph. We denote the path from $q(v)^{(l,w)}$ to $q(v')^{(l',w')}$ as $p(q(v)^{(l,w)}, q(v')^{(l',w')} )$. The adaptivity denoted as $A(c, D, m, w)$.
%
$$A(c, D, m, w) = \max\limits_{q(v)^{(l,w)},q(v')^{(l',w')} \in V }\{ |p(q(v)^{(l,w)}, q(v')^{(l',w')} )| \}$$
\end{defn}

% \subsection{ Adaptivity through an example }

%  We still use the two round example in Figure~\ref{fig:simpl-two-round-graph} to illustrate the process of collecting the trace, building the dependency graph and reaching the adaptivity. 

% \[
% TRC(k) \triangleq
% {
% \begin{array}{l}
%     % \left[j \leftarrow 0 \right]^1 ; \\
%   \clabel{ a \leftarrow []}^{1} ; \\
%     \clabel{\assign{j}{0} }^{2} ; \\
%     \eloop ~ \clabel{k}^{3} ~ \edo ~ \\
%     \Big(
%      \clabel{x \leftarrow q(\chi(j)\cdot \chi(k)) }^{4}  ; \\
%      \clabel{\assign{j}{j+1}}^{5} ;\\
%     \clabel{a \leftarrow x :: a}^{6}       \Big);\\
%     \clabel{l \leftarrow q(\mathrm{sign}\big (\sum_{i\in [k]} \chi(i)\times\ln\frac{1+a[i]}{1-a[i]} \big ))}^{7}\\
% \end{array} 
% }
% \]
% \\
% Given a specific database $D = [[1, 1], [0, 0], [1, 1], [1, 1]]$, supposing $a= q_1(\chi[0])(D) +q_1(\chi[1])(D) +q_1(\chi[2])(D) = n$, $n$ is a constant variable $a$ stores in the resulting memory, then the execution trace $t$ is generated along with the operational semantics as follows:
% \\
% $\config{[], TRC(3), D, []>} \to^{*}
% \config{[j \to 3, a \to [1, 0, 1], l \to 1], D, \eskip, t>}$\\
% \[t = \left\{
% q_1(\chi(0))^{(4, [3:1])}, \
% q_1(\chi(1))^{(4, [3:2])}, \ 
% q_1(\chi(2))^{(4, [3:3])}, \ 
% q_2(\chi[4]+n)^{(7, \emptyset)}
% \right \}\]
% For the sake of brevity, we use $q_1(0), q_1(1), q_1(2)$ to represent 
% $ q_1(\chi(0)), 
% q_1(\chi(1)),
% q_1(\chi(2))$. The graph is also shown in Figure~\ref{fig:simpl-two-round-graph}.  


% Then we have the graph as:
% \\
% $V = \left\{
% q_2^6, q_1^{(3,1)}, q_1^{(3,2)}, q_1^{(3,3)}
% \right\}$
% \\
% $E = \left \{
% (q_2^6, q_1^{(3,1)}),
% (q_2^6, q_1^{(3,2)}),
% (q_2^6, q_1^{(3,3)})
% \right\}$
% Then we have the dependency graph generated in Figure \ref{fig:two-round-graph}. 
% \\
% $V = \left\{
% q_{1}(0, 3)^{(3, [1])}, \
% q_{1}(1, 3)^{(3, [2])}, \ 
% q_{1}(2, 3)^{(3, [3])}, \ 
% q_{2}([1, 0, 1])^{(6, [])}
% \right \}$

% Todo: A graph for two round algorithm
% \begin{figure}
% %\begin{figure}
% \begin{tikzpicture}[scale=\textwidth/30cm,samples=200]
% %%% The nodes represents the k query in the first round
% \filldraw[black] (0, 4) circle (5pt) node [anchor=south]{$q_0^{(3,[3:1])}$};
% \filldraw[black] (6, 4) circle (5pt) node [anchor=south]{$q_1^{(3,[3:2])}$};
% \filldraw[black] (12, 4) circle (5pt) node [anchor=south]{$q_2^{(3,[3:3])}$};
% \filldraw[black] (6, 0) circle (5pt) node [anchor=north]{$q(v)^{(7,\emptyset)}$};
% \draw[very thick,->, blue] (6, 0)  -- (6, 3.9) ;
% \draw[very thick,->, red] (6, 0)  -- (12, 3.9) ;
% \draw[very thick,->, blue] (6, 0)  -- (0, 3.9) ;
% \end{tikzpicture}
% \caption{A query-based dependency graph for two round algorithm complete version}
% \label{fig:two-round-graph}
% \end{figure}
%\end{figure}
% Even though the high level loop language provides necessary information for static analysis on most real-world data analysis algorithms, it is not suitable to conduct a static analysis directly on. To be specific, it is challenging to achieve a formal definition of the number of rounds of adaptivity for algorithms using the high level language due to the characteristics depicted in its syntax: the execution of one query $q(e)$ in a program $p$ is decided not only by its explicit control flow (e.g. if statement), but also by its argument $e$. 
%
% Then we have the adaptivity calculated from the graph as:
% \[
% \begin{array}{ll}
% A^*(TR, D, m, w) & = \max\limits_{q(v)^{(l,w)},q'(v')^{(l',w')} \in V }\{ |p(q(v)^{(l,w)}, q'(v')^{(l',w')} )| \}\\
% & = |p(q_2^6, q_1^{(3,3)})| = |p(q_2^6, q_1^{(3,1)})| = |p(q_2^6, q_1^{(3,2)})|\\
% & = 1
% \end{array}
% \]
% We can notice the complete version generates the similar query-based dependency graph as its simplified version in Figure~\ref{fig:simpl-two-round-graph}.



\section{Towards Static Single Assignment}
\label{sec:ssa}
In last section, we show the challenges to directly analyze programs in (labelled) loop language and introduce an approach to eliminate - towards single-static-assignment form.

\subsection{The limit of loop language for analysis}
we see the power of the labelled loop language to achieve the adaptivity semantically, from its being capable to express many adaptive data analysis algorithm,  allowing the construction of the query-based dependency graph using traces from the execution, and so on.
However, it is not powerful enough to reach the adaptivity syntactically. The main difficulty is its implicit control flow which raises extra complexity to figure out where some variables used come from. We use three simple but relevant examples to show why the loop language suffers. We use $q_1,q_2,q_3$ to represent three linear queries.

\[
 s_1 = \begin{array}{l}
      \clabel{ \assign{x}{q_1}}^{1} ; \\
      \eif  [(x < 0 )]^{2} \\
      \ethen \clabel{\assign{x}{q_2}}^{3}\\
      \eelse \clabel{\eskip}^{4} ; \\
      \clabel{\assign{y}{q(x+\chi(3))}}^{5}
 \end{array}
 ~~~~~
  s_2 = \begin{array}{l}
      \clabel{ \assign{x}{q_1}}^{1} ; \\
      \eif  [(x < 0 )]^{2} \\
      \ethen \clabel{\assign{x}{q_2}}^{3}\\
      \eelse \clabel{\assign{x}{q_3}}^{4} ; \\
      \clabel{\assign{y}{q(x+\chi(3))}}^{5}
 \end{array}
 ~~~~~~~~
  s_3 = \begin{array}{l}
      \clabel{ \assign{x}{q_1}}^{1} ; \\
      \eif  [(x < 0 )]^{2} \\
      \ethen \clabel{\assign{z}{q_2}}^{3}\\
      \eelse \clabel{\eskip}^{4} ; \\
      \clabel{\assign{y}{q(x+\chi(3)}}^{5}
 \end{array}
\]
In these three examples, the variable $x$ at line $5$ is implicit. In program $s_1$, it refers to the either $x$ at line $1$, or $x$ at line $3$, which means the result of query request $q(x+\chi(3))$ assigned to the variable $y$ may depend on $q_1$(bound to $x$ at line $1$) or $q_2$($x$ at line $3$). When we have a look at the other two programs $s_2$ and $s_3$, it is another talk. We think $q(x+\chi(3))$ may depend on either $q_2$($x$ at line $3$) or $q_3$($x$ at line $4$) in $s_2$,     while $q(x+\chi(3))$ only depends on $q_1$ at line $1$ in program $s_3$. These three examples are structural similar in loop language, however, the dependency between variables are quite dissimilar. We consider variables here because query request is also bound to variables. To solve this dilemma, we move to single static assignment as follows.   
\[
 s_1^{s} = \begin{array}{l}
      \clabel{ \assign{{\ssa{x_1}}}{q_1}}^{1} ; \\
      \eif  [({\ssa{x_1} }< 0 )]^{2}\\
      ([], [{ \ssa{x_3, x_1,x_2} }], []) \\
      \ethen \clabel{\assign{{\ssa{x_2}}}{q_2}}^{3}\\
      \eelse \clabel{\eskip}^{4} ; \\
      \clabel{\assign{{\ssa{y_1}}}{q({\ssa{x_3} + \chi(3)})}}^{5}
 \end{array}
 ~~~~~
  s_2^{s} = \begin{array}{l}
      \clabel{ \assign{{\ssa{x_1}}}{q_1}}^{1} ; \\
      \eif  [({\ssa{x_1}} < 0 )]^{2}, \\
      ( [{\ssa{x_4, x_2,x_3}}], [], [] ) \\
      \ethen \clabel{\assign{{\ssa{x_2}}}{q_2}}^{3}\\
      \eelse \clabel{\assign{{\ssa{x_3}}}{q_3}}^{4} ; \\
      \clabel{\assign{{\ssa{y_1}}}{q({\ssa{x_4}}+\chi(3))}}^{5}
 \end{array}
 ~~~~~~~~
  s_3^{s} = \begin{array}{l}
      \clabel{ \assign{{\ssa{x_1}}}{q_1}}^{1} ; \\
      \eif  [({\ssa{x_1}} < 0 )]^{2} \\
       ( [], [], [] ) \\
      \ethen \clabel{\assign{{\ssa{z_1}}}{q_2}}^{3}\\
      \eelse \clabel{\eskip}^{4} ; \\
      \clabel{\assign{{\ssa{y_1}}}{q({\ssa{x_1}}+\chi(3))}}^{5}
 \end{array}
\]
To distinguish between the loop language and in ssa form, we denote the ssa variable ${\ssa{x_1}}$ in bold. As we can see, the data flow becomes explicit in ssa form and the analysis on the dependency between variables in the program becomes much clear now. Considering this advantage, we aim to estimate the adaptivity through an analysis on program in ssa form. 


% \begin{enumerate}
%     \item Variable may be overwritten. Suppose $p = [\assign{x}{q_1()}]^{1}; [\assign{x}{q_2()}]^{2}; [\assign{y}{q_3(x)}]^{3} $. It increases the difficulty of static analysis to figure out the variable $x$ used at line $3$ inside $q_3(x)$ refers to the one at line $1$ or $2$.
%     \item Dependency of queries inside the loop body is hard to estimate.  
% \end{enumerate}

\subsection{ The ssa loop language }
We present the syntax of the ssa loop language, a language based on the loop language, representing programs in single static assignment form. We omit the standard parts inherited from the loop language and only focus on the syntax relevant to ssa.  
\[
\begin{array}{llll}
%  \mbox{Arithmatic Operators} & \oplus_a & ::= & + ~|~ - ~|~ \times 
% %
% ~|~ \div \\  
%   \mbox{Boolean Operators} & \oplus_b & ::= & \lor ~|~ \land ~|~ \neg\\
%   %
%   \mbox{Relational Operators} & \sim & ::= & < ~|~ \leq ~|~ == \\  
%  \mbox{Label} & l & := & \mathbb{N} \\ 
%  \mbox{While Map} & w & \in & \mbox{Label} \times \mathbb{N} \\
\mbox{SSA Arithmetic Expressions} & \ssa{\aexpr} & ::= & 
	%
	{n} ~|~ {\ssa{x}} ~|~ \chi ~|~ \ssa{\aexpr} \oplus_a \ssa{\aexpr} ~|~ [] ~|~ [\ssa{\aexpr}, \dots, \ssa{\aexpr}] \\
% 	\sep \pi (l , \aexpr, \aexpr) \\
    %
\mbox{SSA Boolean Expressions} & \ssa{\bexpr} & ::= & 
	%
	\textrm{true} ~|~ \textrm{false}  ~|~ \neg \ssa{\bexpr}
	 ~|~ \ssa{\bexpr} \oplus_b \ssa{\bexpr}
	%
	~|~ \ssa{\aexpr} \sim \ssa{\aexpr} \\
\mbox{SSA Expressions} & \ssa{\expr} & ::= & \ssa{\aexpr} \sep \ssa{\bexpr} \\	
\mbox{SSA Labelled commands} & \ssa{c} & ::= &   [\assign {\ssa{x}}{ \ssa{\expr}}]^{l} ~|~  [\assign {{\ssa{x}} } {q({\ssa{e}})}]^{l}
%
% ~|~ [\eswitch( \ssa{\expr}, \ssa{x}, \ssa{v_i} \to \ssa{ q_i})]^{l} 
~|~  {{ifvar(\bar{\ssa{x}}, \bar{\ssa{x}}')}}  ~|~ [\eskip]^{l}  \\ 
&&& [\eloop ~ {\ssa{\aexpr}}, {n},  [\bar{\ssa{x}}, \bar{\ssa{x_1}}, \bar{\ssa{x_2}}] ~ \edo ~ {\ssa{c}} ]^{l} ~|~ \ssa{c};\ssa{c}  ~|~ \\ &&& [\eif(\ssa{\bexpr}, ([\bar{\ssa{x}}, \bar{\ssa{x_1}}, \bar{\ssa{x_2}}] , [\bar{\ssa{y}}, \bar{\ssa{y_1}}, \bar{\ssa{y_2}}],[\bar{\ssa{z}}, \bar{\ssa{z_1}}, \bar{\ssa{z_2}}] ) , \ssa{c}, \ssa{c})]^l 	
	\\
\mbox{SSA Memory} & \ssa{m} & ::= & \emptyset ~|~ { (\ssa{x} \to v) :: \ssa{m} } \\
%
% \mbox{Trace} & t & ::= & [] ~|~ \ssa{({q(v)^{(l, w) }) :: t}} \\
% \mbox{Annotated Query} & \mathcal{AQ}  & ::= & \{ q(v)^{(l,w)}  \} \\
\mbox{SSA Variables} & \mathcal{SV}  & ::= & \{ \ssa{x} \} \\
\mbox{Annotated SSA Variables} & \mathcal{LV}  & ::= & \{ \ssa{x}^{(l,w)}  \}
\end{array}
\]
We use $\ssa{\aexpr}$ to express arithmetic expressions which now contains ssa variable $\ssa{x} \in \mathcal{SV}$, and the boolean expression as $\ssa{\bexpr}$. The ssa expression can be either $\ssa{a}$ and $\ssa{b}$. We also have the ssa variables annotated in a similar way as the annotated queries in the loop language, denoted as $\mathcal{LV}$. The labelled commands $\ssa{c}$ are now in the ssa form. In the assignment $[\assign {\ssa{x}}{ \ssa{\expr}}]^{l}$,  query request $[\assign {{\ssa{x}} } {q({\ssa{e}})}]^{l}$ , the expression ${\ssa{\expr}}$ is now in the ssa form, similar for the $\ssa{\aexpr}$ in the loop and conditional $\ssa{\bexpr}$ in the if command. The if command now contains the extra part $([\bar{\ssa{x}}, \bar{\ssa{x_1}}, \bar{\ssa{x_2}}] , [\bar{\ssa{y}}, \bar{\ssa{y_1}}, \bar{\ssa{y_2}}],[\bar{\ssa{z}}, \bar{\ssa{z_1}}, \bar{\ssa{z_2}}] )$, which helps to track the dependency of new assigned variables in both branches($[\bar{\ssa{x}}, \bar{\ssa{x_1}}, \bar{\ssa{x_2}}]$), then branch $[\bar{\ssa{y}}, \bar{\ssa{y_1}}, \bar{\ssa{y_2}}]$, and else branch $[\bar{\ssa{z}}, \bar{\ssa{z_1}}, \bar{\ssa{z_2}}] $. The $\bar{\ssa{x}}$ is a list of ssa variables, in which every element $\ssa{x}$ may depends on the corresponding element $\ssa{x_1}$ from $\bar{\ssa{x_1}}$ collected in the then branch or the corresponding element $\ssa{x_2}$ from $\bar{\ssa{x_2}}$ collected in the else branch. Every tuple $(\ssa{x,x_1,x_2 })$ from $[\bar{\ssa{x}}, \bar{\ssa{x_1}}, \bar{\ssa{x_2}}]$ can be understood as $\ssa{x} = \phi(\ssa{x_1,x_2})$ in the normal ssa form. The previous example $s_2^{s}$ can be used for reference. The second part $[\bar{\ssa{y}}, \bar{\ssa{y_1}}, \bar{\ssa{y_2}}]$ focuses on the then branch. The list of ssa variables $y_1$ stores the assigned ssa variables before the if command, whose non-ssa version (variables in the loo language) will be modified only in the then branch. We can look at program $s_1$ as a reference,in which $x$ at line $1$ may be modified only in the then branch at line $3$. The list $\bar{\ssa{y_2}}$ tracks the ssa variables assigned only in the then branch. If the variables are assigned in both branches such as in the program $s_2$, they goes into $[\bar{\ssa{x}}, \bar{\ssa{x_1}}, \bar{\ssa{x_2}}]$. Then we think every ssa variable in $\bar{\ssa{y}}$ may come from the corresponding variable $\ssa{y_1}$ in $\bar{\ssa{y_1}}$ before the if command or $\ssa{y_2}$ in $\bar{\ssa{y_2}}$ in the then branch. In this sense, we can also regard every tuple $(\ssa{y,y_1,y_2 })$ from $[\bar{\ssa{y}}, \bar{\ssa{y_1}}, \bar{\ssa{y_2}}]$ as $\ssa{y} = \phi(\ssa{y_1,y_2})$.  The rest part $[\bar{\ssa{z}}, \bar{\ssa{z_1}}, \bar{\ssa{z_2}}]$ focus on the else branch and can be understood similarly. Also, the loop command also has similar part $ [\bar{\ssa{x}}, \bar{\ssa{x_1}}, \bar{\ssa{x_2}}]$, focusing on the loop body. The new command ${{ifvar(\bar{\ssa{x}}, \bar{\ssa{x}}')}}$ does not have explicit label because it is only used for evaluation internally, we will discuss more about it when used in the operational semantics for the ssa loop language. The ssa memory $\ssa{m}$ is a map from ssa variable $\ssa{x}$ to values.

We show the example of the complete version of two round algorithm $TRC^{ssa}$ in the ssa language in Figure~\ref{fig:tworound_complete}.

\subsection{Trace-based Operational Semantics of ssa loop language}
When switching to the ssa loop language, we show that we are still be able to achieve what we can get in Section~\ref{sec:loop_language}. The operational semantics of the ssa loop language mimics its counterpart, of the form $\config{\ssa{m}, \ssa{c}, t, w} \to \config{\ssa{m'}, \eskip, t', w'}$. It still uses a trace to track the query requests during the execution, starting from the ssa configuration with ssa memory $\ssa{m}$ and programs in its ssa form $\ssa{c}$, which allows a similar construction of the query-based dependency graph in the ssa language as its counterpart. We choose selected evaluation rules in Figure~\ref{fig:ssa_evaluation}.

\begin{figure}
    \begin{mathpar}
% \boxed{\config{\ssa{m, a}} \xrightarrow{} \config{\ssa{a}} \;} 
% \inferrule{
%     \ssa{m}(\ssa{x}) = \ssa{v}
% }{ \config{\ssa{m, x }} \to \ssa{v} 
% }~{\textbf{SSA-VAR}}
% \and
\boxed{ \config{\ssa{ m, c}, t,w} \xrightarrow{} \config{\ssa{ m', c'},  t', w'} \; }
\and
% { Memory \times Com  \times Trace \times WhileMap \Rightarrow^{}  Memory \times Com  \times Trace \times WhileMap}
% \and
\inferrule
{
 \config{\ssa{m, \expr} } \to \ssa{\expr'}
}
{
\config{ \ssa{m}, [\assign{{\ssa{x}}}{{q(\ssa{\expr})}}]^l, t, w} \xrightarrow{} \config{ \ssa{m}, [\assign{{\ssa{x}}}{{q(\ssa{\expr'})}}]^l, t, w}
}
~\textbf{ssa-query-arg}
%
\and
%
\inferrule
{
{q(v) = v_q} 
}
{
\config{ \ssa{m}, [\assign{{\ssa{x}}}{{q(v)}}]^l, t, w} \xrightarrow{} \config{\ssa{  m}[ v_q / \ssa{ x} ], \eskip,  t \mathrel{++} [q(v)^{(l,w )}],w  }
}
~\textbf{ssa-query}
% %
% \and
% %
% \inferrule
% {
% }
% {
% \config{\ssa{ m, [\assign x v]^{l},  t,w} } \xrightarrow{} \config{\ssa{m[x \mapsto v], [\eskip]^{l}, t,w} }
% }
% ~\textbf{ssa-assn}
% %
% \and
% %
% \inferrule
% {
% \config{m, c_1,  t,w} \xrightarrow{} \config{m', c_1',  t',w}
% }
% {
% \config{m, c_1; c_2,  t,w} \xrightarrow{} \config{m', c_1'; c_2, t',w}
% }
% ~\textbf{seq1}
% %
% \and
% %
% \inferrule
% {
% }
% {
% \config{m, [\eskip]^{l} ; c_2,  t,w} \xrightarrow{} \config{m, c_2,  t,w}
% }
% ~\textbf{seq2}
% %
% \and
% %
% \inferrule
% {
% \config{ \ssa{m, \bexpr}} \barrow \ssa{\bexpr'}
% }
% {
% \config {\ssa{m, [\eif(\bexpr, [\bar{{x}}, \bar{{x_1}}, \bar{{x_2}}] ,[\bar{\ssa{y}}, \bar{\ssa{y_1}}, \bar{\ssa{y_2}}] ,[\bar{\ssa{z}},\bar{\ssa{z_1}}, \bar{\ssa{z_2}}] , c_1, c_2)]^{l},  t,w} } 
% \xrightarrow{} \config{ \ssa{ m,  [\eif(\ssa{bexpr'},[ \bar{{x}}, \bar{{x_1}}, \bar{{x_2}}] ,[\bar{\ssa{y}}, \bar{\ssa{y_1}}, \bar{\ssa{y_2}}] ,[\bar{\ssa{z}},\bar{\ssa{z_1}}, \bar{\ssa{z_2}}] , c_1, c_2)]^{l},  t, w} }
% }
% ~\textbf{ssa-if}
%
\and
%
\inferrule
{
}
{
\config{\ssa{m, [\eif(\etrue, [\bar{\ssa{x}}, \bar{\ssa{x_1}}, \bar{\ssa{x_2}}] ,[\bar{\ssa{y}}, \bar{\ssa{y_1}}, \bar{\ssa{y_2}}] ,[\bar{\ssa{z}}, \bar{\ssa{z_1}}, \bar{\ssa{z_2}}] , c_1, c_2)]^{l}},t,w} \\
\xrightarrow{} \config{\ssa{m, c_1}; { \ ifvar(\bar{\ssa{x}},\bar{\ssa{x_1}}); ifvar(\bar{\ssa{y}},\bar{\ssa{y_2}});ifvar(\bar{\ssa{z}},\bar{\ssa{z_1}}) }  ,  t,w}
}
~\textbf{ssa-if-t}
%
\and
%
\inferrule
{
}
{
\config{\ssa{m, [\eif(\efalse, [\bar{\ssa{x}}, \bar{\ssa{x_1}}, \bar{\ssa{x_2}}] ,[\bar{\ssa{y}}, \bar{\ssa{y_1}}, \bar{\ssa{y_2}}] ,[\bar{\ssa{z}},\bar{\ssa{z_1}}, \bar{\ssa{z_2}}] , c_1, c_2)]^{l}},t,w} 
\\
\xrightarrow{} \config{\ssa{m, c_2} ; { ifvar(\bar{\ssa{x}},\bar{\ssa{x_2}}); ifvar(\bar{\ssa{y}},\bar{\ssa{y_1}});ifvar(\bar{\ssa{z}},\bar{\ssa{z_2}}) },  t,w}
}
~\textbf{ssa-if-f}
% %
% \and
% %
% {
% \inferrule
% {
% \empty
% }
% {
% \config{\ssa{m}, [ \eswitch(\ssa{v_k},\ssa{x}, (\ssa{v_i} \to q_i))]^{l},  t,w} 
% \xrightarrow{} \config{\ssa{m},  [\assign {\ssa{x}}{ q_k}]^{l},  t, w }
% }
% ~\textbf{ssa-switch-v}
% }
%
\and
%
\inferrule{
}{
 \config{ \ssa{m}, ifvar(\ssa{\bar{x}, \bar{x}'}),{ t,w }} \to \config{ \ssa{m [  m(\bar{x}')/ \bar{x}], \eskip , t,w }  }
}~\textbf{ssa-ifvar}
% %
% \and
% %
% {\inferrule
% {
% }
% {
% \config{m, \ewhile([\bexpr]^l, c),  t,w} 
% \xrightarrow{} \config{m,  \eunfold{[\bexpr^{l}] }{\ewhile([\bexpr]^l,   c)} ,  t,w}
% }
% ~\textbf{while} }
% %
% \and
% %
% {\inferrule
% {
% \config{m, \bexpr} \rightarrow \bexpr'
% }
% {
% \config{m, \eunfold{[\bexpr]^l}{ c}, D, t,w} 
% \xrightarrow{} \config{m, \eunfold{[\bexpr']^l}{ c}, D, t,  w  }
% }
% ~\textbf{unfold}}
% %
% \and
% %
% {\inferrule
% {
% }
% {
% \config{m, \eunfold{[\efalse]^l}{c}, D, t,w} 
% \xrightarrow{} \config{m, [\eskip]^{l}, D, t,  (w \setminus l) }
% }
% ~\textbf{unfold-f}}

% \and
% %
% {\inferrule
% {
% }
% {
% \config{m, \eunfold{[\etrue]^l}{ c}, D, t,w} 
% \xrightarrow{} \config{m, c, D, t, (w+l) }
% }
% ~\textbf{unfold-t} }
% %
% \and
% %
% {
% \inferrule
% {
%   \config{m, \expr } \xrightarrow{} \expr'
% }
% {
% \config{m, [\eswitch(\expr, x, (v_i \to q_i))]^{l},  t,w} 
% \xrightarrow{} \config{m, [ \eswitch(\expr',x, (v_i \to q_i))]^{l},  t, w }
% }
% ~\textbf{switch}
% }
% %
% \and
% %
% {
% \inferrule
% {
% \empty
% }
% {
% \config{m, [ \eswitch(v_k,x, (v_i \to q_i))]^{l},  t,w} 
% \xrightarrow{} \config{m,  [\assign x q_k]^{l},  t, w }
% }
% ~\textbf{switch-v}
% }
%
\and
%
{\inferrule
{
 {{ \valr_N > 0} }\and 
 { {n} = 0 \implies i =1 } \and
 { {n} > 0 \implies i =2 }
}
{
\config{ \ssa{m},  [\eloop ~ \valr_N, n, [\bar{\ssa{x}}, \bar{\ssa{x_1}}, \bar{\ssa{x_2}}] ~ \edo ~ \ssa{c} ]^{l}  ,  t, w } \\
\xrightarrow{} \config{\ssa{ m, c[\bar{x_i} /  \bar{x}   ]};  [\eloop ~ (\valr_N-1), n+1, [\bar{\ssa{x}}, \bar{\ssa{x_1}}, \bar{\ssa{x_2}}] ~ \edo ~ \ssa{c}]^{l} ,  t, (w + l)  }
}
~\textbf{ssa-loop}
}
%
\and
%
{
\inferrule
{
 \valr_N = 0 \and
 { {n} = 0 \implies i =1 } \and
 { {n} > 0 \implies i =2 }
}
{
\config{\ssa{m},  [\eloop ~ \valr_N, n, [\bar{\ssa{x}}, \bar{\ssa{x_1}}, \bar{\ssa{x_2}}] ~ \edo ~ \ssa{c} ]^{l}  ,  t, w }
\xrightarrow{} \config{\ssa{ m[m(\bar{x_i})/\bar{x} ]}, [\eskip]^{l} ,  t, (w \setminus l)  }
}
~\textbf{ssa-loop-exit}
}
%
\end{mathpar}
    \caption{Operational semantics for the ssa loop language}
    \label{fig:ssa_evaluation}
\end{figure}

The key idea underneath the operational semantics is to have the trace and the execution path being constructed in a similar way as in the loop language. Take the query request as an example, the argument $\ssa{e}$ which contains ssa variables will be evaluated to a value $v$ first before the request is sent to the database in rule $\textbf{ssa-query-arg}$. The trace expands in the rule $\textbf{ssa-query}$ likewise in the loop language. The query $q$, a primitive symbol representing the abstract query in both the ssa language and  the loop language, makes no difference in the two languages. Since we add the extra part $[\bar{\ssa{x}}, \bar{\ssa{x_1}}, \bar{\ssa{x_2}}] ,[\bar{\ssa{y}}, \bar{\ssa{y_1}}, \bar{\ssa{y_2}}] ,[\bar{\ssa{z}},\bar{\ssa{z_1}}, \bar{\ssa{z_2}}]  $ in the if command compared to its counterpart in the loop language, the rules relevant to the if condition ($\textbf{ssa-if-t}$ and $\textbf{ssa-if-f}$) use the command $ifvar(\ssa{\bar{x}, \bar{x}'})$ to update the ssa memory $\ssa{m}$ with the mapping from the new generated variable $\ssa{x}$ in $\bar{\ssa{x}}$ to the appropriate value $\ssa{m(x')}$ where $\ssa{x'}$ is the corresponding variable w.r.t $\ssa{x}$ in $\bar{\ssa{x'}}$. The rule $\textbf{ssa-ifvar}$ reflects the usage of $ifvar(\ssa{\bar{x}, \bar{x}'})$. It is easier to understand the usage of $ifvar(\ssa{\bar{x}, \bar{x}'})$ in the rule $\textbf{ssa-if-t}$ when we think about how ssa works: in the ssa form, when a variable to be used may come from two sources (e.g. $\ssa{x_1}$ and $\ssa{x_2}$ in the rule), it generates a new variable $\ssa{x}$, assigning it with $\phi(\ssa{x_1}, \ssa{x_2})$,  and replaces the variable to be used with newly assigned $\ssa{x}$. We know that in the future program after the if command, only the variables $\bar{\ssa{x}}$ will be available instead of   $\bar{\ssa{x_1}}, \bar{\ssa{x_2}}$ from two branches. For the evaluation of the program after the if command, we need to tell the memory the exact value of the newly generated variable $\ssa{x}$, which is the value stored in $\ssa{x_1}$ when the conditional $\ssa{b}$ is true, or the value in $\ssa{x_2}$ when $\ssa{b}$ is false. To this end, the internal command $ifvar(\ssa{\bar{x}, \bar{x}'})$ plays its role. For the if rule, we need to instantiate the variables from $\bar{\ssa{x}}$ whose values come from two branches, $\bar{\ssa{y}}$ whose values from then branch or assignment before the if command, and $\bar{\ssa{z}}$ whose values from else branch or before the if command. Correspondingly, we need to have three ifvar commands.   

The evaluation of loop depends on the loop counter $\ssa{\aexpr}$, which will be evaluated to a value $v_N$. When $v_N$ is greater than 0, the loop is still executing, and all the variables $\ssa{x}$ in $\bar{\ssa{x}}$ of the loop body $\ssa{c}$ are replaced as the corresponding variables in $\bar{\ssa{x_1}}$ in the first iteration($n=0$), or $\bar{\ssa{x_2}}$ in other iterations($n > 0$). The loop turns to an exit when $v_N > 0$, and the memory $\ssa{m}$ updates the mapping of variables in $\bar{\ssa{x}}$ with $\bar{\ssa{x_1}}$ if the iteration counter $n=0$, which means the loop body is not executed once. When the loop enters the exit after executing the body a few times($n$), the variables in $\bar{\ssa{x}}$ is instantiated with the value from the body $\ssa{m}(\bar{\ssa{x_2}})$. 

With the trace-based operational semantics of the ssa language at hand, we are able to provide our query-based dependency graph in the ssa language.

\begin{defn}[Query may dependency in ssa ]
One query ${q(v_1)}$ may depend on another query ${q(v_2)}$ in a program $\ssa{c}$, with a starting while map $w$, denoted as
$\mathsf{DEP_{ssa}}({q(v_1)}^{(l_1, w_1)}, {q(v_2)}^{(l_2, w_2)}, \ssa{c},w, \ssa{m},D)$ is defined as below. 
\[
  \begin{array}{l}
     \forall t. \exists \ssa{m_1,m_3},t_1,t_3. 
\config{\ssa{m, c},  t,w} \rightarrow^{*} \config{\ssa{m_1}, [\assign{\ssa{x}}{q(v_1)}]^{l_1} ; \ssa{c_2},
  t_1,w_1} \rightarrow \\ \config{\ssa{m_1}[q(v_1)(D)/\ssa{x}], \ssa{c_2},
  t_1++[q(v_1)^{(l_1, w_1)}], w_1} \rightarrow^{*} \config{\ssa{m_3}, \eskip,
  t_3,w_3} \\  
  \land 
\Big( q(v_1)^{(l_1,w_1)} \in (t_3-t) \land q(v_2)^{(l_2,w_2)} \in (t_3-t) \implies  \exists v \in \codom(q(v_1)),\ssa{m_3'}, t_3', w_3'. \\
 \config{\ssa{m_1}[v/\ssa{x}], \ssa{c_2}, t_1++[(q(v_1)^{(l_1,w_1)})], w_1} \rightarrow^{*} \config{\ssa{m_3'}, \eskip, t_3', w_3'} \land (q(v_2)^{(l_2,w_2)}) \not \in (t_3'-t)
\Big)\\
\land 
\Big(q(v_1)^{(l_1,w_1)} \in (t_3-t) \land q(v_2)^{(l_2,w_2)} \not\in (t_3-t) \implies  \exists v \in \codom(q(v_1)),\ssa{m_3'}, t_3', w_3'. \\
 \config{\ssa{m_1}[v/\ssa{x}], {c_2}, t_1++[(q(v_1)^{(l_1,w_1)})], w_1} \rightarrow^{*} \config{\ssa{m_3'}, \eskip, t_3', w_3'} \land (q(v_2)^{(l_2,w_2)})  \in (t_3'-t)
\Big)\\
\end{array}
\]
\end{defn}



\begin{defn}
Dependency Graph in ssa.
\\
Given a program $\ssa{c}$, a database $D$, a starting memory $\ssa{m}$, an initial whilemap $w$, the dependency graph $G_{s}(\ssa{c},D,\ssa{m},w) = (V, E)$ is defined as: \\
$V =\{q(v)^{l,w} \in \mathcal{AQ} \mid \forall t. \exists \ssa{m'},  w', t'.  \config{\ssa{m} ,\ssa{c}, t, w}  \to^{*}  \config{\ssa{m'} , \eskip, t', w' }  \land q(v)^{l,w} \in {(t'-t)}  \}$.
\\
$E = \left\{(q(v)^{(l,w)},q(v')^{(l',w')}) \in \mathcal{AQ} \times \mathcal{AQ} 
~ \left \vert ~ \begin{array}{l}
  \mathsf{DEP_{ssa}}(q(v)^{(l,w)},q(v')^{(l',w')}, \ssa{c},w,\ssa{m},D)     \\
  \land \mathsf{To}(q'(v')^{(l',w')}, q(v)^{(l,w)}    
\end{array} \right. 
\right\}$.
\end{defn}

\subsection{Transformation }
We build a bridge between the two languages through a transformation in spirit of the work \cite{vekris2016refinement}. The command transformation of the form $ \Sigma; \delta ; c  \hookrightarrow \ssa{c} ; \delta' ; \Sigma'$ translates the labelled command $c$ in the loop language to its counterpart in ssa language. The key idea beneath single static assignment form is to give every assignment an unique variable. To this end, the SSA name environment $\Sigma$, a set of ssa variables already used before the transformation process, is used to generate a fresh ssa variable via a function $fresh(\Sigma)$. Additionally, translating variables read in the program in the loop language to its unique ssa variable requires a translation environment $\delta$, a map from variable $x \in \mathcal{VAR}$ to its ssa form $\ssa{x} \in \mathcal{SV}$. Also, the translation environment $\delta$ and the ssa name environment $\Sigma$ will be updated to $\delta'$ and $\Sigma'$ respectively, along with the transformation of the target command. The expression is much simpler, of the form $ \delta; \expr \hookrightarrow \ssa{\expr}$, which transforms the variables in $\expr$ to ssa variables stored in the translation environment $\delta$, shown in the rule $\textbf{S-VAR}$.

We present selected transformation rules in Figure~\ref{fig:trans_rules}. The rules $\textbf{S-ASSN}$ and $\textbf{S-QUERY}$ both use $fresh(\Sigma)$ to generate a new fresh variable $\ssa{x}$ to guarantee unique assignment in its ssa form. The translation environment is updated with the mapping from $x$ to the new generated ssa variable $\ssa{x}$ for reference to $x$ used in the future. The ssa name environment is also modified by recording the new variable $\ssa{x}$. The transformation of sequence is standard, with both environments $\delta$ and $\Sigma$ updated during the transformation procedure.   

We look at the rule $\textbf{S-IF}$ by first introducing the binary operation $\bowtie$ on two translation environments $\delta_1$ and $\delta_2$. 
\[ \delta_1 \bowtie \delta_2 = \{ ( x, {\ssa{x_1}, \ssa{x_2}} ) \in \mathcal{VAR} \times \mathcal{SV} \times \mathcal{SV} \mid x \mapsto {\ssa{x_1}} \in \delta_1 , x \mapsto {\ssa{x_2} } \in \delta_2, {\ssa{x_1} \not= {\ssa{x_2} }  }  \} \]
\[ \delta_1 \bowtie \delta_2 / \bar{x} = \{ ( x, {\ssa{x_1}, \ssa{x_2}} ) \in \mathcal{VAR} \times \mathcal{SV} \times \mathcal{SV} \mid x \not\in \bar{x} \land x \mapsto {\ssa{x_1}} \in \delta_1 , x \mapsto {\ssa{x_2} } \in \delta_2, {\ssa{x_1} \not= {\ssa{x_2} }   }  \} \]
This definition of $\delta_1 \bowtie \delta_2$ combines two translation environments by only keeping mapping who share the same key in both environments. It returns a set of tuples with three elements $(x,\ssa{x_1}, \ssa{x_2})$ to show that a variable $x$ in the loop language may be translated to either $\ssa{x_1}$ or $\ssa{x_2}$, depending on the control flow. We use $\bar{x}$ to represent a list of variables $x$, in this sense, the results of  $\delta_1 \bowtie \delta_2$ is denoted as $ [\bar{x}, \bar{\ssa{x_1}}, \bar{\ssa{x_2}}]$.
\[
 [\bar{x}, \bar{\ssa{x_1}}, \bar{\ssa{x_2}}] = \{ (x, x_1,x_2)  | \forall 0 \leq i < |\bar{x}|, x = \bar{x }[i] \land x_1 = \bar{x_1}[i] \land x_2 = \bar{x_2 }[i] \land |\bar{x}| = |\bar{x_1}| = |\bar{x_2}|   \}
\]

When it comes to the if command, a variable in loop language may be translated to two possible ssa variables in three cases: (1) the variable $x$ is assigned in both two branches, whose mapping of $x$ is stored in $\delta_1$(then branch) and $\delta_2$(else branch). (2) the variable $x$ is assigned before the if command (this mapping in $\delta$) and only assigned in the then branch $\delta_1$ (3) the variable $x$ is assigned before the if command (this mapping in $\delta$) and only assigned in the else branch $\delta_2$. This corresponds to the aforementioned discussion of the if command in ssa language. We leave these mapping explicitly in the if command of the ssa language. We use the variant of $\delta \bowtie \delta_1$, $\delta \bowtie \delta_1 / \bar{x}$ to guarantee that the variables stored in $ [\bar{y}, \bar{\ssa{y_1}}, \bar{\ssa{y_2}}]$ only appears in the then branch, not in the else branch. Similarly for $ [\bar{z}, \bar{\ssa{z_1}}, \bar{\ssa{z_2}}]$. In the translated if command, the variable in $\bar{x}, \bar{y}, \bar{z}$ is replaced with the fresh ssa variables stored in $\bar{\ssa{x}},\bar{\ssa{y}},\bar{\ssa{z}}$.    

The loop transformation rule also deserves a deep discussion. Except the normal transformation of the loop counter $\aexpr$ to $\ssa{\aexpr}$, the additional iteration counter is added to its ssa form to support the evaluation, as we have seen in the rule $\textbf{ssa-loop}$ in Figure~\ref{fig:ssa_evaluation}, and is set to $0$. For the variables assigned in the loop body $c$, we leave $ [\bar{\ssa{x'}}, \bar{\ssa{x_1}}, \bar{\ssa{x_2}}]$ in the command, which tracks variables in the loop body whose value may come from two sources: assignment before the loop($\delta$) or assignment from the loop body $\delta_1$. We have two transformation on the body $c$ using the same ssa name environment $\Sigma$ but different translation environments $\delta$ and $\delta_1$. The premise $\Sigma; \delta; c \hookrightarrow \ssa{c_1}; \delta_1 ; \Sigma_1 $ corresponds to the transformation of the loop body in the first iteration with the variables assigned before the loop execution. The second premise $\Sigma; \delta; c \hookrightarrow \ssa{c_2}; \delta_1 ; \Sigma_1 $ corresponds to the transformation in the later iteration with the assigned variables updated by previous execution of the body. Thanks to the extra part $ [\bar{\ssa{x'}}, \bar{\ssa{x_1}}, \bar{\ssa{x_2}}]$ in the ssa loop command, we know that those variables used in the first iteration are stored in $\bar{\ssa{x_1}}$ and those updated by the loop body are stored in $\bar{\ssa{x_2}}$. To finish the ssa transformation, we get the fresh variables $\bar{\ssa{x'}}$ to replace the appearance of $\bar{\ssa{x_1}}$ in  $\ssa{c_1}$ or $\bar{\ssa{x_2}}$ in $\ssa{c_2}$. Finally, we get the loop body $\ssa{c}$ in its ssa form.



% the intuition behind $\delta_1 \bowtie \delta_2 $ is  



% We use a translation environment $\delta$, to map variables $x$ in the low level language to those $\ssa{x}$ in ssa-form language. We use a name environment denoted as $\Sigma$, a set of ssa variables so we can get a fresh variable by $fresh(\Sigma)$. We define $\delta_1 \bowtie \delta_2 $ in a similar way as \cite{vekris2016refinement}.
% \[ \delta_1 \bowtie \delta_2 = \{ ( x, {\ssa{x_1}, \ssa{x_2}} ) \in \mathcal{VAR} \times \mathcal{SV} \times \mathcal{SV} \mid x \mapsto {\ssa{x_1}} \in \delta_1 , x \mapsto {\ssa{x_2} } \in \delta_2, {\ssa{x_1} \not= {\ssa{x_2} }  }  \} \]
% \[ \delta_1 \bowtie \delta_2 / \bar{x} = \{ ( x, {\ssa{x_1}, \ssa{x_2}} ) \in \mathcal{VAR} \times \mathcal{SV} \times \mathcal{SV} \mid x \not\in \bar{x} \land x \mapsto {\ssa{x_1}} \in \delta_1 , x \mapsto {\ssa{x_2} } \in \delta_2, {\ssa{x_1} \not= {\ssa{x_2} }   }  \} \]
% We call a list of variables $\bar{x}$.

% \[
%  [\bar{x}, \bar{\ssa{x_1}}, \bar{\ssa{x_2}}] = \{ (x, x_1,x_2)  | \forall 0 \leq i < |\bar{x}|, x = \bar{x }[i] \land x_1 = \bar{x_1}[i] \land x_2 = \bar{x_2 }[i] \land |\bar{x}| = |\bar{x_1}| = |\bar{x_2}|   \}
% \]

\begin{figure}
\begin{mathpar}
\boxed{ \delta ; e \hookrightarrow \ssa{e} }\\
\inferrule{
}{
 \delta ; x \hookrightarrow \delta(x)
}~{\textbf{S-VAR}}\\
\boxed{ \Sigma; \delta ; c  \hookrightarrow \ssa{c} ; \delta' ; \Sigma' } \\
\inferrule{
  { \delta ; \bexpr \hookrightarrow \ssa{\bexpr} }
  \and
  { \Sigma; \delta ; c_1 \hookrightarrow \ssa{c_1} ; \delta_1;\Sigma_1 }
  \and
  {\Sigma_1; \delta ; c_2 \hookrightarrow \ssa{c_2} ; \delta_2 ; \Sigma_2 }
  \\
  {[\bar{x}, \ssa{\bar{{x_1}}, \bar{{x_2}}}] = \delta_1 \bowtie \delta_2  }
  \and
   {[\bar{y}, \ssa{\bar{{y_1}}, \bar{{y_2}}}] = \delta \bowtie \delta_1 / \bar{x} }
  \and
   {[\bar{z}, \ssa{\bar{{z_1}}, \bar{{z_2}}}] = \delta \bowtie \delta_2 / \bar{x} }
  \\
  { \delta' =\delta[\bar{x} \mapsto \ssa{\bar{{x}}'} ][\bar{y} \mapsto \ssa{\bar{{y}}'} ][\bar{z} \mapsto \ssa{\bar{{z}}'} ]}
  \and 
  {\ssa{\bar{{x}}', \bar{y}', \bar{z}'} \ fresh(\Sigma_2)
  }
  \and{\Sigma' = \Sigma_2 \cup \{ \ssa{ \bar{x}', \bar{y}', \bar{z}' } \} }
}{
 \Sigma; \delta ; [\eif(\bexpr, c_1, c_2)]^l  \hookrightarrow [\ssa{ \eif(\bexpr, [\bar{{x}}', \bar{{x_1}}, \bar{{x_2}}] ,[\bar{{y}}', \bar{{y_1}}, \bar{{y_2}}] ,[\bar{{z}}', \bar{{z_1}}, \bar{{z_2}}] , {c_1}, {c_2})}]^l; \delta';\Sigma'
}~{\textbf{S-IF}}
%
\and
%
\inferrule{
 {\delta ; \expr \hookrightarrow \ssa{\expr} }
 \and
 {\delta' = \delta[x \mapsto \ssa{{x}} ]}
 \and{ \ssa{x} \ fresh(\Sigma) }
 \and { \Sigma' = \Sigma \cup \{ \ssa{x} \} }
}{
 \Sigma;\delta ; [\assign x \expr]^{l} \hookrightarrow [\ssa{\assign {{x}}{ \expr}}]^{l} ; \delta'; \Sigma'
}~{\textbf{S-ASSN}}
%
\and
%
\inferrule{
%  {\delta ; q \hookrightarrow \ssa{q}}
%  \and
 {\delta ; \expr \hookrightarrow \ssa{\expr}}
 \and
 {\delta' = \delta[x \mapsto \ssa{x} ]}
 \and{ \ssa{x} \ fresh(\Sigma) }
  \and { \Sigma' = \Sigma \cup \{ \ssa{x} \} }
}{
 \Sigma;\delta ; [\assign{x}{q(e)}]^{l} \hookrightarrow [\assign {\ssa{x}}{ {q(\ssa{\expr})}}]^{l} ; \delta';\Sigma'
}~{\textbf{S-QUERY}}
% %
% \and
% %
% \inferrule{
%  {\delta ; q_i \hookrightarrow \ssa{q_i}}
%  \and
%  { \delta ; v_k \hookrightarrow \ssa{v_k} }
%  \and
%  { \delta ; v_i \hookrightarrow \ssa{v_i} }
%  \and
%  {\delta' = \delta[x \mapsto \ssa{x} ]}
%  \and{ \ssa{x} \ fresh(\Sigma) }
%   \and { \Sigma' = \Sigma \cup \{ \ssa{x} \} }
% }{
%  \Sigma;\delta ; [ \eswitch({v_k},{x}, ({v_i} \to q_i)) ]^{l} \hookrightarrow [\eswitch(\ssa{v_k},\ssa{x}, (\ssa{v_i} \to \ssa{q_i}))]^{l} ; \delta';\Sigma'
% }~{\textbf{S-SWITCH}}
%
\and
%
\inferrule{
    {\delta ; \aexpr \hookrightarrow \ssa{\aexpr} }
    \and
    { \Sigma; \delta ; c \hookrightarrow \ssa{c_1} ; \delta_1; \Sigma_1 }
    \and 
     { \Sigma; \delta_1 ; c \hookrightarrow \ssa{c_2} ; \delta_1; \Sigma_1 }
     \\
    { [ \bar{x}, \ssa{\bar{{x_1}}}, \ssa{\bar{{x_2}}} ] = \delta \bowtie \delta_1 }
    \and {\delta' = \delta[\bar{x} \mapsto \ssa{\bar{{x}}'}]}
    \and {\ssa{\bar{{x}}'} \ fresh(\Sigma_1 )}
    \and 
    {\ssa{c'= c_1[\bar{x}'/ \bar{x_1}] = c_2[\bar{x}'/ \bar{x_2}]  } }
    % \and{ \delta' ; c \hookrightarrow \ssa{c'} ; \delta'' }
  }{ 
  \Sigma; \delta ;  [\eloop ~ \aexpr ~ \edo ~ c ]^{l} \hookrightarrow [\ssa{\eloop ~ \aexpr, 0, [\bar{{x}}', \bar{{x_1}}, \bar{{x_2}}] ~ \edo ~ {c'} }]^{l} ; \delta_1[\bar{x} \to \ssa{\bar{x}'}]; \Sigma \cup \{\ssa{\bar{x}'}  \}
}~{\textbf{S-LOOP}}
%
\and
%
\inferrule{
 {\Sigma;\delta ; c_1 \hookrightarrow \ssa{c_1} ; \delta_1; \Sigma_1} 
 \and
 {\Sigma_1; \delta_1 ; c_2 \hookrightarrow \ssa{c_2} ; \delta'; \Sigma'} 
}{
\Sigma;\delta ; c_1 ; c_2 \hookrightarrow \ssa{c_1} ; \ssa{c_2} \ ; \delta';\Sigma'
}~{\textbf{S-SEQ}}
\end{mathpar}
 \caption{Transformation rules from loop language to ssa language}
    \label{fig:trans_rules}
\end{figure}

\subsection{The soundness of the transformation}
In this subsection, we show our transformation from the loop language to its ssa form is sound with respect to adaptivity. To be specific, a transformed program $\ssa{c}$ starting with appropriate configuration, generates the same trace as the program before the transformation $c$, in its corresponding configuration.

We first define a good memory in the loop language $m$ or in the ssa language $\ssa{m}$ with respect to a translation environment $\delta$, denoted as $m \vDash \delta$ and $\ssa{m} \vDash \delta$ respectively. 

\begin{defn}[Well defined memory] 
\begin{enumerate}
    % \item $m \vDash c \triangleq \forall x \in \fv{c}, \exists v, (x, v) \in m$.
    \item $ m \vDash \delta  \triangleq \forall x \in \dom(\delta), \exists v, (x,v) \in m$.
    % \item $\ssa{m} \vDash_{ssa} \ssa{c} \triangleq \forall \ssa{x} \in \fvssa{\ssa{c}}, \exists v, (\ssa{x}, v) \in \ssa{m}$.
    \item $ \ssa{m} \vDash_{ssa} \delta  \triangleq \forall \ssa{x} \in \codom(\delta), \exists v, (\ssa{x},v) \in \ssa{m}$.
\end{enumerate}
\end{defn}
The part declared in the translation environment $\delta$ in a ssa memory $\ssa{m}$ can be reverted to corresponding part of the memory $m$ with an inverse of $\delta$ as follows.

\begin{defn}[Inverse of transformed memory]
 $m = \delta^{-1}(\ssa{m}) \triangleq \forall x \in \dom(\delta), (\delta(x), m(x)) \in \ssa{m} $.
\end{defn}

We can also show that the $\expr$ and its translated ssa version $\ssa{\expr}$ through some translation environment $\delta$ evaluates to the same value in Lemma~\ref{same_value}. 
\begin{lem}[Value remains during transformation]
\label{same_value}
Given $\delta; e \hookrightarrow \ssa{e}$, $\forall m. m \vDash \delta. \forall \ssa{m}, \ssa{m} \vDash_{ssa} \delta \land m = \delta^{-1}(\ssa{m})$, $\forall m. m \vDash \delta. \forall \ssa{m}, \ssa{m} \vDash_{ssa} \delta \land m = \delta^{-1}(\ssa{m})$, then $\config{m, e} \to v $ and $\config{
\ssa{m}, \ssa{e}} \to {v}$.  
\end{lem}

Finally, we show the soundness of the transformation. When a program $c$ is transformed to its ssa form $\ssa{c}$, when executing these two programs with corresponding configuration, the newly generated trace from will be the same and the resulting memory $m'$ and $\ssa{m'}$ will also be related. 

\begin{thm}[Soundness of transformation]
Given $\Sigma; \delta ; c \hookrightarrow \ssa{c} ; \delta';\Sigma' $, $\forall m. m \vDash \delta. \forall \ssa{m}, \ssa{m} \vDash_{ssa} \delta \land m = \delta^{-1}(\ssa{m})$, if there exist an execution of $c$ in the loop language, starting with a trace $t$ and while map $w$, $\config{m, c, t, w} \to^{*} \config{m', \eskip, t', w' } $,  then there also exists a corresponding execution of $\ssa{c}$ in the ssa language so that 
  $\config{  {\ssa{m}}, \ssa{c}, t, w} \to^{*} \config{{  \ssa{m'}}, \eskip, t', w' } $ and $ m' = \delta'^{-1}(\ssa{m'}) $.
\end{thm}


% \subsection{The analysis algorithm on ssa programs}


\section{The analysis algorithm on ssa programs}
\label{sec:algorithm}
In this section, we clarify the algorithm {\THESYSTEM} that analyzes the adpativity of a target program in ssa language, which consists of three auxiliary algorithms: $\mathsf{AG}$ and $\mathsf{AD}$ for generating a weighted variable-based dependency graph and $\mathsf{AP}$ to find the most weighted path in the graph. We do not show details of  $\mathsf{AP}$, it is quite standard path-finding algorithm in a weighted graph. We go through the two round example to illustrate the algorithm. Finally, we show the soundness of our algorithm with respect to the adaptivity.  


\subsection{The ideas behind the algorithm}
In consideration of the definition of adaptivity from the query-based dependency graph, our analysis whose goal is a sound upper bound on the adaptivity, is supposed to take care of paths(possible adaptivity candidates) in all the possible dependency graphs (per configuration). To this end, our algorithm aims to syntactically construct a dependency graph, in which the nodes are annotated ssa variables and the directed edges showing one annotated variable may depend on the other if there exists an edge between them. Intuitively, query requests in the query-based dependency graph is assigned to  variables that appears in the  predicted ssa-variable-based dependency graph. 

The algorithm {\THESYSTEM} estimates the adaptivity from the weighted variable-based dependency graph, whose structure is shown in Figure~\ref{fig:adaptfun}. We have a look at the structure of {\THESYSTEM}. We start with the dependency graph, represented in the form of a matrix $M$. To know which variable is associated with a query request in the matrix, an extra vector $V$ is produced in the algorithm, of the size of all estimated assigned variables in the target program $\ssa{c}$. Naturally, the first question comes to us, what is the size of the matrix $M$ and $V$, or the estimated assigned variables? To solve this,    
the analysis goes through the ssa program $\ssa{c}$ twice, estimating the assigned variables to be tracked and adding the appropriate annotation $(l,w)$, in the first time scan. Those annotated variables are collected in a global list $G$, whose size determines the size of the matrix and vector. This first scan generating $G$ is conducted by the algorithm $\mathsf{AG}$ in Figure~\ref{fig:adaptfun}, which takes the target program $\ssa{c}$ as input. The global list $G$ is fed into the second scan, which decides the size of the matrix and vector.
For example, if $G$ of size $N$, then the matrix used in the second round is of size $N \times N$, and vector of size $N$. It maintains an unique mapping from variables in $G$ to the matrix $M$ and the vector $V$. For example,  the $i$th row, $j$th column of the matrix $M[i][j] =1 $ represents the may-dependency from variable $ G[i]$ to $G[j]$, $M[i][j] =0$ means no dependency. In a similar way, $V[i]=1$ means the variable $G[i]$ is assigned to a query request. The second scan of $\ssa{c}$, implemented by the algorithm $\mathsf{AD}$, then records the may-dependency in $M$, the weight is tracked in $V$. We think variable associated with a query request of weight $1$ in the graph and others of weight $0$. After the analysis of $\mathsf{AG}$ and $\mathsf{AD}$, a ssa-variable-based weighted dependency graph is constructed. We use another auxiliary algorithm $AP$ to find the path with the most weights in the graph. Then the weight is the adaptivity $A$ {\THESYSTEM} estimates.    

\begin{figure}
    \centering
    \includegraphics[width=0.7\columnwidth]{adapfun.png}
    \caption{The overview of {\THESYSTEM}}
    \label{fig:adaptfun}
\end{figure}


\subsection{Variable Estimation algorithm}
We first show how $G$ will be collected, through a variable estimating algorithm $\mathsf{AG}$ of the form $\ag{G; w; \ssa{c}}{ G'; w'} $ in Figure~\ref{fig:ag}. The input of $\mathsf{AG}$ is a list of estimated annotated variables $G$ collected before the program $\ssa{c}$, a while map $w$ consistent with previous estimation, a program $\ssa{c}$. The output of the algorithm is the updated global list $G'$, along with the updated while map $w$ for later estimation.   
\begin{figure}
 \begin{mathpar}
\inferrule
{
}
{ \ag{G ;w; \ssa{[\assign {x}{\expr}]^{l}}}{G ++ [\ssa{x}^{(l,w)}];w}
% G ;w; \ssa{[\assign {x}{\expr}]^{l}} \to G ++ [x^{(l,w)}];w 
}
~\textbf{ag-asgn}
\and
\inferrule
{
}
{ \ag{G ;w;  [ \assign{\ssa{x}}{q(\ssa{\expr})}]^{l}}{  G ++ [\ssa{x}^{(l,w)}] ; w} 
}~\textbf{ag-query}
%
\and 
%
\inferrule
{
\ag{G; w; \ssa{c_1}}{  G_1;w_1}
\and 
 \ag{G_1;w ; \ssa{c_2}}{  G_2; w_2}
 \\
 {G_3 = G_2 ++ \ssa{[\bar{x}^{(l,w)}]++ \ssa{[\bar{y}^{(l,w)}]}++ \ssa{[\bar{z}^{(l,w)}]} }}
}
{
\ag{G; w;
[\eif(\ssa{\bexpr},[ \bar{\ssa{x}}, \bar{\ssa{x_1}}, \bar{\ssa{x_2}}] ,[ \bar{\ssa{y}}, \bar{\ssa{y_1}}, \bar{\ssa{y_2}}],[ \bar{\ssa{z}}, \bar{\ssa{z_1}}, \bar{\ssa{z_2}}], \ssa{ c_1, c_2)}]^{l} }{ G_3 ;w}
}~\textbf{ag-if}
%
%
%
\and 
%
\inferrule
{
\ag{G; w; \ssa{c_1}}{ G_1; w_1}
\and 
\ag{G_1;w_1; \ssa{c_2}}{ G_2; w_2}
}
{
\ag{G; w;
\ssa{(c_1 ; c_2)}}{  G_2 ; w_2}
}
~\textbf{ag-seq}
\and 
\inferrule
{
{G_0 = G \quad w_0 =w }
\and
\forall 0 \leq z < N. 
{ \ag{ G_z ++ \ssa{[\bar{x}^{(l, {w_z}+l)}]} ; (w_z+l); \ssa{c}}{ G_{z+1} ; w_{z+1}}  }
\\
{G_f = G_N ++ \ssa{[\bar{x}^{(l, w_N \setminus l)}]} }
\and
{ \ssa{\aexpr} =  {N}  }
}
{\ag{G; w; [\eloop ~ \ssa{\aexpr}, n, [\bar{\ssa{x}}, \bar{\ssa{x_1}}, \bar{\ssa{x_2}}] ~ \edo ~ \ssa{c}]^{l} }{ G_f; w_N\setminus l }
}~\textbf{ag-loop}
% \and 
% \inferrule
% {
% \Gamma \vdash^{(i,i+a )}_{M, V} c 
% }
% {\Gamma \vdash^{(i, i+ N*a)}_{M_{i,a}^N(f), V_{i, a}^N} 
% \ewhile([\bexpr]^l,   c) : \phi \Rightarrow \psi
% }~\textbf{while}
% %
% \and
% %
% \inferrule
% {
% }
% { \ag{ G;w; [ \eswitch(\ssa{\expr}, \ssa{x},(\ssa{v_j} \rightarrow \ssa{q_j} ) )]^{l} } { G ++ [\ssa{x}^{(l, w)}] ; w} }
% ~\textbf{switch}
\end{mathpar}
 \caption{The algorithm $\mathsf{AG}$ }
    \label{fig:ag}
\end{figure}
The assignment is the source of variables $\mathsf{AG}$ estimates, in the case $\textbf{ag-asgn}$ and $\textbf{ag-query}$, the output global list is extended by $\ssa{x}^{(l,w)}$. When it comes to the if command in the rule $\textbf{ag-if}$, variables assigned in the then branch $\ssa{c_1}$, as well as the variables assigned in the else branch $\ssa{c_2}$, and the new generated variables $\bar{\ssa{x}},\bar{\ssa{y}},\bar{\ssa{z}}$ in $ [ \bar{\ssa{x}}, \bar{\ssa{x_1}}, \bar{\ssa{x_2}}] ,[ \bar{\ssa{y}}, \bar{\ssa{y_1}}, \bar{\ssa{y_2}}],[ \bar{\ssa{z}}, \bar{\ssa{z_1}}, \bar{\ssa{z_2}}]$. The sequence is standard by accumulating the predicted variables in the two commands $\ssa{c_1}$ and $\ssa{c_2}$ in a sequence $\ssa{c_1;c_2}$. The loop considers the loop iterations as well by assuming the loop counter $\ssa{\aexpr}$ to be certain natural number $N$ in the rule $\textbf{ag-loop}$. The algorithm counts the assigned variables in every iteration, including those new assigned variables in $\bar{\ssa{x}}$, those variables assigned in the body $\ssa{c}$, with the appropriate annotation showing the iteration number of the variables.      


\subsection{Matrix and Vector based algorithm}
We have a global list $G$ after the first scan of the ssa programs $\ssa{c}$. We develop a matrix and vector based approach based on the global list $G$ to get an estimated upper bound on the adaptivity of the program.  In this approach, a matrix is constructed according to those estimated annotated variables from the global list, in which every row and every column corresponds to the unique variable in $G$ by position. To be precise, in this matrix, $0$ means no dependency while non-zero value in the matrix shows may-dependency between corresponding variables. If the value of $M[i][j]$ in the matrix $M$ is greater than zero, we know annotated variable $G[i]$ may depend on $G[j]$. A vector $V$ has the same size as $G$ and it records whether the corresponding variable $G[i]$ is assigned with a query request. $V[i] =1$ means assigned with a query request and  otherwise $V[i]=0$. We call this algorithm which tracks may-dependency in the ssa programs and query information, $\mathsf{AD}$. The algorithm of the form $ \ad{\Gamma; \ssa{c} ;i }{ M; V;  i' } $, the input is a tuple consisting of three elements: (1) a one-row-N-column matrix $\Gamma$ storing the dependency from previous program, it is used when handling the if command. (2) the ssa program $\ssa{c}$ to be analyzed (3) an index $i$ specifying the location of the first assigned variable of the program $\ssa{c}$ in the global list $G$. The output of $\mathsf{AD}$ also consists of three elements ; (1) A matrix $M$ showing the may-dependency in $\ssa{c}$ (2) A vector $V$ records the query requests in $\ssa{c}$ (3) the index $i'$ that refers to the next position of the last assigned variable in $\ssa{c}$, if exists. The existence of the index $i'$ helps to locate the first assigned variable if we need to analyze programs after $\ssa{c}$.  

We first define some functions which use the indices in $G$. 
The function $\mathsf{L(i)}$ generates one-column-N-rows  matrix, where only the $i-th$ row is $1$ and all the other rows are $0$. This function is used to locate the right row when calculate the matrix when analyzing assignment and query. 

The function $\mathsf{R(e, i)}$ generates a one-row-N-column matrix. For every variable used in $e$, it finds the corresponding index $i$ in $G$ so that $G[i]$ maps to the variable and mark the $i$th column as $1$. If it is not found, we do not mark. When we say $G[i]$ maps to a target variable, we take off the annotation of $G[i]$ and check if the left variable is the same as the target variable. To handle loop, for instance, a variable $y$ appears many times in $G$, but with different annotations(iteration numbers), the argument $i$ helps to find most recent assignment variable $y$ before the index $i$ in $G$. It is still used when analyzing assignment and query. Thanks to our ssa language, our choice of the most recent assigned variable is reasonable because the variable used in the loop refers to the most recent assignment and every variable is uniquely assigned in its ssa form. 

We define $M_1;M_2$ to combine two matrix, where $M_1 + M_2$ is the standard sum of two matrix.
\[
M_1 ; M_2  :=  M_2 \cdot M_1 + M_1 + M_2 
\]
And define the operator $\uplus$ to combine two vectors.
\[
V_1 \uplus V_2  :=  \left\{
\begin{array}{ll}
1 & (V_1[i] = 1 \lor V_2[i] = 1) \land i = 1, \cdots, N \land |V_1| = |V_2|\\
0 & o.w.
\end{array}\right.
\]
For the sake of brevity, we also define some annotations as follows. We show how the algorithm $\mathsf{AD}$ handles the extra part $[ \bar{\ssa{x}}, \bar{\ssa{x_1}},\bar{\ssa{x_2}} ] $ in the if and loop commands. First, we give a unique name for variables in lists $\bar{\ssa{x}}, \bar{\ssa{x_1}}, \bar{\ssa{x_2}}$ respectively, as follows: {$ \forall 0 \leq z < |\bar{\ssa{x}}|. \bar{\ssa{x}}(z) = \ssa{x_z}, \bar{\ssa{x_1}}(z) = \ssa{x_{1z}}, \bar{\ssa{x_2}}(z) = \ssa{x_{2z}} $ }. And then we treat every tuple $(\ssa{x_z},\ssa{x_{1z}},\ssa{x_{2z}} )$ in $[\bar{\ssa{x}}, \bar{\ssa{x_1}}, \bar{\ssa{x_2}} ]$ as the simple may dependency case : $\ssa{x_z}$ may depend on both $ \ssa{x_{1z}}$ and $\ssa{x_{2z}}$, just like $ \assign{\ssa{x_z} }{\ssa{x_{1z}} + \ssa{x_{2z}} }$, defined as follows. 
$
 \ad{\Gamma; [\bar{\ssa{x}}, \bar{\ssa{x_1}}, \bar{\ssa{x_2}} ] ; i }{ M; V_{\emptyset}; i + |\bar{\ssa{x}}| } 
  \triangleq { \forall 0 \leq z < |\bar{\ssa{x}}|.
  \ad{\Gamma;\assign{\ssa{x_z} }{\ssa{x_{1z}} + \ssa{x_{2z}} }; i+z }{ M_{x_z};  V_{\emptyset}; i+z+1} }$
   where $M = \sum_{o \leq z < |\bar{\ssa{x}}| }M_{x_z} $.
% \framebox{$ \ad{\Gamma; \ssa{c} ; i_1){M;V;i_2} $}
%
\begin{figure}
\begin{mathpar}
\inferrule
{M = \mathsf{L}(i) * ( \mathsf{R}(\ssa{\expr},i) + \Gamma )
}
{
 \ad{\Gamma;[\assign {\ssa{x}}{\ssa{\expr}} ]^{l}; i }{M; V_{\emptyset}; i+1 }
% \Gamma \vdash_{M, V_{\emptyset}}^{(i, i+1)} [\assign {\ssa{x}}{\ssa{\expr}} ]^{l}
}
~\textbf{ad-asgn}
\and
\inferrule
{M = \mathsf{L}(i) * ( \mathsf{R}(\ssa{\expr},i) + \Gamma )
\\
V= \mathsf{L}(i)
}
{ 
\ad{\Gamma;[ \assign{\ssa{x}}{q(\ssa{\expr})} ]^{l} ; i }{M;V;i+1}
%  \vdash^{(i, i+1)}_{M, V} [ \assign{\ssa{x}}{q(\ssa{\expr})} ]^{l} 
}~\textbf{ad-query}
%
\and 
%
\inferrule
{
{\ad{\Gamma + \mathsf{R}(\ssa{\bexpr}, i_1); \ssa{c_1} ; i_1 }{ M_1;V_1;i_2 }}
% \Gamma + \mathsf{R}(\bexpr, i_1) \vdash^{(i_1, i_2)}_{M_1, V_1} \ssa{c_1} 
% : \Phi \land \bexpr \Rightarrow \Psi
\and 
{\ad{\Gamma + \mathsf{R}(\ssa{\bexpr}, i_1);\ssa{c_2} ; i_2 }{ M_2; V_2 ;i_3}}
% \Gamma + \mathsf{R}(\ssa{\bexpr}, i_1) \vdash^{(i_2, i_3)}_{M_2, V_2} \ssa{c_2} 
% : \Phi \land \neg \bexpr \Rightarrow \Psi
\\
% { \forall 0 \leq j < |\bar{x}|. \bar{x}(j) = x_j, \bar{x_1}(j) = x_{1j}, \bar{x_2}(j) = x_{2j}  }
{\ad{\Gamma; [ \bar{\ssa{x}}, \bar{\ssa{x_1}}, \bar{\ssa{x_2}}]; i_3 }{ M_x; V_{\emptyset}; i_3+|\bar{\ssa{x}}| }}
%
\and
%
{\ad{\Gamma; [ \bar{\ssa{y}}, \bar{\ssa{y_1}}, \bar{\ssa{y_2}}]; i_3+|\bar{\ssa{x}}| }{ M_y; V_{\emptyset}; i_3+|\bar{\ssa{x}}|+|\bar{\ssa{y}}| }}
%
\\
%
{\ad{\Gamma; [ \bar{\ssa{z}}, \bar{\ssa{z_1}}, \bar{\ssa{z_2}}]; i_3+|\bar{\ssa{x}}|+ |\bar{\ssa{y}}|}{ M_y; V_{\emptyset}; i_3+|\bar{\ssa{x}}|+|\bar{\ssa{y}}| + |\bar{\ssa{z}}| }}
% { \forall 0 \leq j < |\bar{x}|.  \Gamma \vdash_{M_{x_j}, V_{\emptyset}}^{i_3+j, i_3+j+1 } x_j \leftarrow x_{1j} + x_{2j} }
% \and
% { \forall 0 \leq j < |\bar{y}|.  \Gamma \vdash_{M_{y_j}, V_{\emptyset}}^{i_3+|\bar{x}|+j, i_3+|\bar{x}|+j+1 } y_j \leftarrow y_{1j} + y_{2j} }
% \\
% { \forall 0 \leq j < |\bar{z}|.  \Gamma \vdash_{M_{z_j}, V_{\emptyset}}^{i_3+|\bar{x}|+|\bar{y}|+j, i_3+|\bar{x}|+|\bar{y}|+j+1 } z_j \leftarrow z_{1j} + z_{2j} }
\\
{M = (M_1+M_2)+ M_x+M_y +M_z }
}
{
\Gamma \vdash^{(i_1, i_3+|\bar{x}|+|\bar{y}|+|\bar{z}|)}_{M, V_1 \uplus V_2 } 
[\eif(\ssa{\bexpr},[ \bar{\ssa{x}}, \bar{\ssa{x_1}}, \bar{\ssa{x_2}}] ,[ \bar{\ssa{y}}, \bar{\ssa{y_1}}, \bar{\ssa{y_2}}] , [ \bar{\ssa{z}}, \bar{\ssa{z_1}}, \bar{\ssa{z_2}}] , \ssa{ c_1, c_2)}]^{l}
}~\textbf{ad-if}
%
%
%
\and 
%
\inferrule
{
{\ad{\Gamma; \ssa{c_1} ; i_1 }{ M_1 ; V_1; i_2 }  }
% \Gamma \vdash^{(i_1, i_2)}_{M_1, V_1} \ssa{c_1} 
% : \Phi \Rightarrow \Phi_1
\and 
{\ad{\Gamma;\ssa{c_2}; i_2}{M_2; V_2 ;i_3 }}
% \Gamma \vdash^{(i_2, i_3)}_{M_2, V_2} \ssa{c_2} 
% : \Phi_1 \Rightarrow \Psi 
}
{
\ad{\Gamma ; (\ssa{c_1 ; c_2} ) ; i_1}{(M_1 {;} M_2) ; V_1 \uplus V_2 ; i_3  }
% \Gamma \vdash^{(i_1, i_3)}_{M_1 {;} M_2, V_1 \uplus V_2}
% \ssa{c_1 ; c_2} 
% : \Phi \Rightarrow \Psi
}
~\textbf{ad-seq}
\and 
\inferrule
{
B= |\ssa{\bar{x}}| \and {A = |\ssa{c}|}
% \and
% {\Gamma \vdash^{(i, i+B)}_{M_{10}, V_{10}} [\bar{\ssa{x}}, \bar{\ssa{x_1}}, \bar{\ssa{x_2}}] }
% \and
% {\Gamma \vdash^{(i+B,i+B+A )}_{M_{20}, V_{20}} \ssa{c} 
% }
\\
\forall 0 \leq j < N. 
{\ad{\Gamma;[\bar{\ssa{x}}, \bar{\ssa{x_1}}, \bar{\ssa{x_2}}]; i+ j*(B+A) }{M_{1j};V_{1j}; i+B+j*(B+A) }}
% {\Gamma \vdash^{(i+j*(B+A), i+B+j*(B+A))}_{M_{1j}, V_{1j}}  } [\bar{\ssa{x}}, \bar{\ssa{x_1}}, \bar{\ssa{x_2}}]
\\
{
\ad{\Gamma;\ssa{c} ; i+B+j*(B+A)  }{M_{2j}; V_{2j}; i+B+A+j*(B+A) }
% \Gamma \vdash^{(i+B+j*(B+A),i+B+A+j*(B+A) )}_{M_{2j}, V_{2j}} \ssa{c} 
% : \Phi \land e_n = \lceil{z+1}\rceil \Rightarrow \Psi 
}
\\
{
\ad{\Gamma ; [\bar{\ssa{x}}, \bar{\ssa{x_1}}, \bar{\ssa{x_2}}] ; i+N*(B+A) }{M; V ;i+N*(B+A)+B}
% \Gamma \vdash^{(i+N*(B+A) ,i+N*(B+A)+B )}_{M, V} [\bar{\ssa{x}}, \bar{\ssa{x_1}}, \bar{\ssa{x_2}}]
% : \Psi \Rightarrow \Phi \land e_N = \lceil{z}\rceil 
}
\\
{ \ssa{\aexpr} =  {N}  }
\and
{M' = M+ \sum_{0 \leq j <N}( M_{1j}+M_{2j})  }
\and
{V' = V \uplus \sum_{0 \leq j <N}( V_{1j} \uplus V_{2j})  }
}
{\Gamma \vdash^{(i, i+N*(B+A)+B   )}_{M', V'} 
[\eloop ~ \ssa{\aexpr}, 0, [\bar{\ssa{x}}, \bar{\ssa{x_1}}, \bar{\ssa{x_2}}] ~ \edo ~ \ssa{c}]^{l}
% : \Phi \land \expr_N = \lceil { N} \rceil \Rightarrow \Phi \land \expr_N = \lceil{0}\rceil
}~\textbf{ad-loop}
% \and 
% \inferrule
% {
% \Gamma \vdash^{(i,i+a )}_{M, V} c 
% }
% {\Gamma \vdash^{(i, i+ N*a)}_{M_{i,a}^N(f), V_{i, a}^N} 
% \ewhile([\bexpr]^l,   c) : \phi \Rightarrow \psi
% }~\textbf{while}
% %
% \and
% %
% \inferrule
% { \Gamma + \mathsf{R}(\expr,i) \vdash^{(i, i+1)}_{M, V} \assign{ x}{q_j} 
% % : \Phi \Rightarrow \Psi
% \\
% j \in \{1, \dots, N\}     }
% {\Gamma \vdash^{(i, i+1)}_{ M,V } 
% [\eswitch(\ssa{\expr}, \ssa{x},(v_j \rightarrow q_j ) ]^{l}
% % : \Phi \Rightarrow \Psi 
% }
% ~\textbf{switch}
% %
% \and
% %
% \inferrule
% { 
% \vDash 
% \Phi \Rightarrow \Phi'  
% \and
% \Gamma \vdash^{(i_1, i_2)}_{(M',V')} c : \Phi' \Rightarrow \Psi'
% \and
% \vDash \Psi' \Rightarrow \Psi
% \and 
% \Phi \vDash M' \leq M
% \and 
% \Phi \vDash V' \leq V
% }
% {\Gamma \vdash^{(i_1, i_2)}_{(M,V)} c 
% : \Phi \Rightarrow \Psi
% }
% ~\textbf{conseq}
\end{mathpar}
    \caption{The algorithm AD}
    \label{fig:algo_ad}
\end{figure}
One of the key idea under algorithm $\mathsf{AD}$ is to track the indices $i$,$i'$ both in the input and output to synchronize with its previous algorithm $\mathsf{AG}$. The index in $\mathsf{AD}$ increases as the same way as the global list expands after the analysis of a program $\ssa{c}$, which helps $\mathsf{AD}$ record the dependency relation from the program $\ssa{c}$ in the right place of the matrix. For example, in the case $\textbf{ad-asgn}$ and $\textbf{ad-query}$, the index increases by 1, which corresponds to their counterparts of algorithm $\textbf{AG}$. The if and loop commands have the extra part $ [\bar{\ssa{x}}, \bar{\ssa{x_1}}, \bar{\ssa{x_2}}] $ and we find that the output index increases by also considering this part as we do in collecting the global list.  

Another interesting point is the construction of the matrix. The fundamental case is the assignment and query cases. We use a function $L(i)$ to generate a N-row-one-column matrix $L$ to guarantee the resulting matrix only has non-zeros at row $i$. The intuition behind is that one single assignment or query request can only reveal the dependency of its assigned variable (corresponding to one row of the matrix) to the variables used on the right hand sides. Thanks to the index $i$, we know which row this assignment should be in the matrix. The function $\mathsf{R}(\ssa{\expr},i)$ gets a one-row-N-column matrix marking the variables used in the right hand side. The $\Gamma$ is designed for the if command and we will discuss it later. We can see one simple example $sa$ to get a taste.     
\[
sa \triangleq
\begin{array}{l}
    \left[x_1 \leftarrow 2 \right]^1; \\
    \left[x_2 \leftarrow x_1 + 2 \right]^2 ; \\
    \left[x_3 \leftarrow x_1 + x_2 \right]^3
\end{array}
\]
In the program $sa$, only simple assignment is involved. When we assume $\Gamma$ is empty, for the assignment at line $3$, the matrix is built as follows.
\[
\textbf{line3:} ~~
 \left[x_3 \leftarrow x_1 + x_2 \right]^3 :
 ~~~
\begin{blockarray}{cc}
\begin{block}{c[c]}
 x_1 & 0   \\
 x_2 & 0 \\
 x_3 & 1 \\
\end{block}
\end{blockarray}
*
\begin{blockarray}{ccc}
x_1 & x_2 & x_3 \\
\begin{block}{[ccc]}
1 & 1 & 0 \\
\end{block}
\end{blockarray}
= 
\begin{blockarray}{cccc}
& x_1 & x_2 & x_3\\
\begin{block}{c[ccc]}
x_1 & 0 & 0 & 0 \\
x_2 & 1 & 0 & 0 \\
x_3 & 1 & 1 & 0 \\
\end{block}
\end{blockarray}
\]
% a simple example of assignment
% \begin{example}[Simple Assignment]
% \[
% SA \triangleq
% \begin{array}{l}
%     \left[x_1 \leftarrow 2 \right]^1; \\
%     \left[x_2 \leftarrow x_1 + 2 \right]^2 ; \\
%     \left[x_3 \leftarrow x_1 + x_2 \right]^3
% \end{array}
% \]
% %
% %
% \[
%  \textbf{line2:} ~~
%  \left[x_2 \leftarrow x_1 + 2 \right]^2 :
%  ~~~
% \begin{blockarray}{cc}
% \begin{block}{c[c]}
%  x_1 & 0   \\
%  x_2 & 1 \\
%  x_3 & 0 \\
% \end{block}
% \end{blockarray}
% *
% \begin{blockarray}{ccc}
% x_1 & x_2 & x_3 \\
% \begin{block}{[ccc]}
% 1 & 0 & 0 \\
% \end{block}
% \end{blockarray}
% = 
% \begin{blockarray}{cccc}
% & x_1 & x_2 & x_3\\
% \begin{block}{c[ccc]}
% x_1 & 0 & 0 & 0 \\
% x_2 & 1 & 0 & 0 \\
% x_3 & 0 & 0 & 0 \\
% \end{block}
% \end{blockarray}
% \]
% %
% %
% %
% \[
% \textbf{line3:} ~~
%  \left[x_3 \leftarrow x_1 + x_2 \right]^3 :
%  ~~~
% \begin{blockarray}{cc}
% \begin{block}{c[c]}
%  x_1 & 0   \\
%  x_2 & 0 \\
%  x_3 & 1 \\
% \end{block}
% \end{blockarray}
% *
% \begin{blockarray}{ccc}
% x_1 & x_2 & x_3 \\
% \begin{block}{[ccc]}
% 1 & 1 & 0 \\
% \end{block}
% \end{blockarray}
% = 
% \begin{blockarray}{cccc}
% & x_1 & x_2 & x_3\\
% \begin{block}{c[ccc]}
% x_1 & 0 & 0 & 0 \\
% x_2 & 1 & 0 & 0 \\
% x_3 & 1 & 1 & 0 \\
% \end{block}
% \end{blockarray}
% \]
% %
% \end{example}
 We use a one-row-N-column matrix $\Gamma$ as one of the input of $\mathsf{AD}$, to handle the cases when the control flow diverges in the labelled if command $[\eif(\ssa{\bexpr},[ \bar{\ssa{x}}, \bar{\ssa{x_1}}, \bar{\ssa{x_2}}] ,[ \bar{\ssa{y}}, \bar{\ssa{y_1}}, \bar{\ssa{y_2}}] , [ \bar{\ssa{z}}, \bar{\ssa{z_1}}, \bar{\ssa{z_2}}] , \ssa{ c_1, c_2)}]^{l} $, where the execution of either branch may depend on the conditional guard $\ssa{\bexpr}$. Follow this intuition, the analysis of either branch is supposed to consider the variables used in the conditional $\ssa{\bexpr}$, tracked in $\Gamma$. In the case $\textbf{ad-if}$, we can see the analysis of the two branches $\ssa{c_1}$ and $\ssa{c_2}$ share the same input $\Gamma + \mathsf{R}(\ssa{\bexpr}, i)$ and $\mathsf{R}(\ssa{\bexpr}, i) $ tells the variables assigned before the if command and used in the conditional $\ssa{\bexpr}$.   

We compose the matrix and vectors in the case of sequence in $\textbf{ad-seq}$. The non-zeros or we call it effect range of the matrix and vector is decided by its input and output indices. From the case $\textbf{ad-seq}$, two programs $\ssa{c_1}$ and $\ssa{c_2}$ have disjoint effect ranges $[i_1, i_2)$ and $[i_1,i_3)$, it is safe to combine them without lose information. 

The loop case is handled in a well organized way. We use $B= |\bar{\ssa{x}}|$ and $A= |{\ssa{c}}|$ to estimate the size of variables assigned in $\bar{\ssa{x}}$ and $\ssa{c}$. And  $|\ssa{c}|$ is defined by the help of the algorithm $\mathsf{AG}$, defined as $|\ssa{c}|= |G|$ when $\ag{[];\ssa{c};\emptyset }{ G; \emptyset }$. The algorithm then gets how many iterations $N$ the loop may executes from the loop counter $\ssa{\aexpr}$. For every iteration, it first records the dependency relations between variables in $ [\bar{\ssa{x}}, \bar{\ssa{x_1}}, \bar{\ssa{x_2}}]$ by constructing a corresponding matrix $M_{1j}$ ($j$ is the iteration number) and an empty vector $V_{1j}$, and analyze the loop body $\ssa{c}$ with a resulting matrix $M_{2j}$ and vector $V_{2j}$. We give an extra analysis of those new assigned variables as what $\mathsf{AG}$ does, it works well when the loop is executed ($N = 0$) or not. We know that for all the possible iteration number $j$, $M_{1j}$ and $M_{2j}$ have disjoint effect ranges so we combine them, similar as the vectors $V_{1j}$ and $V_{2j}$.   

Finally, we are able to construct a variable-based weighted dependency graph based on $G$,$M$ and $V$ generated by the framework. The definition of the estimated adaptivity is the weight of the most weighted path in the graph defined as follows. 

\begin{defn}
[Estimated Adaptivity]
Given a program $\ssa{c}$, the global list $G$, and $\ad{\Gamma; \ssa{c}; i_1}{M, V, i_2}$, the weighted dependency graph $G_{ssa}(M, V,G,i_1,i_2) = (Nodes, Edges, Weights)$ is defined as:
\\
Nodes $Vt = \{ G(j) \in \mathcal{LV} \mid i_1 \leq j < i_2 \}$
\\
Edges $E = \{ (G(j_1), G(j_2)) \in \mathcal{LV} \times \mathcal{LV} \mid M[j_1][j_2] \geq 1 \land  i_1 \leq j_1,j_2 < i_2   \}$
\\
 Weights $Wt = \{ (  G(j), 1 ) \in \mathcal{LV} \times \mathcal{N} | i_1 \leq j < i_2 \land V[j] = 1\}
        \cup \{ (  G(j), 0 ) \in \mathcal{LV} \times \mathcal{N} | i_1 \leq j < i_2 \land V[j] = 0 \} $.
        
Adaptivity of the program defined according to the graph is as:
\[
Adapt(M, V,i_1,i_2) := \max_{vt_1, vt_2 \in Vt}\{ \mathsf{Weight}( p(vt_1, vt_2), Wt) \},
\]
where $p(k, l)$ is the path in graph $G_{ssa}(M, V, i_1,i_2)$ starting from $k$ to $l$, $\mathsf{Weight}(p(vt_1,vt_2), Wt)$ get the total sum of weights along the path $p(vt_1,vt_2)$.
\end{defn}        

\subsection{Analysis on two round algorithm}
We show how {\THESYSTEM} analyze the two round algorithm. For the sake of brevity, we conduct the analysis on the simplified two round algorithm $TR^{ssa}$.

% \[
% TR^{ssa}(k) \triangleq
% {
% \begin{array}{l}
%     % \left[j \leftarrow 0 \right]^1 ; \\
%     \clabel{a_1 \leftarrow []}^{1} ; \\
%     \clabel{\assign{j_1}{0} }^{2} ; \\
%     \eloop ~ \clabel{k}^{3} ~ \edo [(j_3, j_1,j_2),(a_3, a_1,a_2)]~ \\
%     \Big(
%     \clabel{ x_1 \leftarrow q(\chi(j_3)\cdot \chi(k))}^{4}  ; \\
%     \clabel{ \assign{j_2}{j_3+1} }^{5} ;\\
%     \clabel{a_2 \leftarrow x_1 :: a_3}^{6}       \Big);\\
%     \clabel{l_1 \leftarrow q(\mathrm{sign}\big (\sum_{i\in [k]} \chi(i)\times\ln\frac{1+a_3[i]}{1-a_3[i]} \big ))}^{7}\\
% \end{array}
% }
% \]
{\THESYSTEM} first runs the algorithm $\mathsf{AG}$ to generate the global list $G$. We assume the input $k=2$ and have the following.

\[G_{k=2} = \left[
  {a_1}^{(1,\emptyset)} , {a_3}^{(2,[2:1])} , {x_1}^{(3,[2:1])} , {a_2}^{(4,[2:1])} ,  {a_3}^{(2,[2:2])} , {x_1}^{(3,[2:2])} , {a_2}^{(4,[2:2])} , {a_3}^{(2,\emptyset)} , {l_1}^{(5,\emptyset)}   \right] \]
% \[G_{k=2} = \left[ \begin{array}{l}
%      {a_1}^{(1,\emptyset)} , {j_1}^{(2,\emptyset)}, {j_3}^{(3,[2:1])} , {a_3}^{(3,[2:1])} , {x_1}^{(4,[2:1])} ,{j_2}^{(5,[2:1])} ,
%   {a_2}^{(6,[2:1])},  \\
%     {j_3}^{(3,[2:2])} , {a_3}^{(3,[2:2])} , {x_1}^{(4,[2:2])} ,{j_2}^{(5,[2:2])} ,
%   {a_2}^{(6,[2:2])},
%   {j_3}^{(3,\emptyset)} , {a_3}^{(3,\emptyset)} ,
%   {l_1}^{(7,\emptyset)} \\  
% \end{array}
%      \right] \]
 We denote $a_1^{1}$ short for ${a_1}^{(1,\emptyset)}$ and ${a_3}^{(2,1)}$ short for ${a_3}^{(2,[2:1])}$. Then the resulting matrix $M_{tr}$ and $V_{tr}$ of the algorithm $\mathsf{AD}$ as follows.
 
{ \tiny
 \[
M_{tr} =  \left[ \begin{matrix}
 & a_1^{1} & a_3^{(2,1)} & x_1^{(3,1)} & a_2^{(4,1)}  & a_3^{(2,2)} & x_1^{(3,2)} & a_2^{(4,2)} & a_3^{2} & l_1^{5}\\
 a_1^{1} & 0 & 0 & 0 & 0 & 0 & 0 & 0 &0 &0 \\
a_3^{(2,1)} & 1 & 0 & 0 & 0 & 0 & 0 & 0&0&0\\
x_1^{(3,1)} & 0 & 0 & 0 & 0 & 0 & 0& 0& 0 &0\\
a_2^{(4,1)} & 0 & 1 & 1 & 0 & 0 & 0 & 0& 0&0\\
a_3^{(2,2)} & 1 & 0 & 0 & 1 & 0 & 0 & 0 & 0&0 \\
x_1^{(3,2)} & 0 & 0 & 0 & 0 & 0 & 0 & 0& 0&0\\
a_2^{(4,2)} & 0 & 0 & 0 & 0 & 1 & 1 & 0& 0&0\\
a_3^{2} & 1 & 0 & 0 & 0 & 0 & 0 & 1& 0&0\\
l_1^{5} & 0 & 0 & 0 & 0 & 0 & 0 & 0 & 1 &0 \\
 \end{matrix} \right] 
~ , V_{tr} = \left [ \begin{matrix}
a_1^{1} &  0 \\
a_3^{(2,1)} & 0 \\
x_1^{(3,1)} & 1 \\
a_2^{(4,1)} &  0 \\
a_3^{(2,2)} & 0 \\
x_1^{(3,2)} & 1 \\
a_2^{(4,2)} &  0 \\
a_3^{2} &  0 \\
l_1^{5} &  1 \\
\end{matrix} \right ]
\]
}
%% a graph is better here

\subsection{ Soundness of {\THESYSTEM}}
We would like to show that the query-based dependency graph generated from the trace of the execution of the target ssa program is a subgraph of the variable-based dependency graph predicted from our algorithm ${\THESYSTEM}$, and the query requested during the execution is also bounded by an estimation from our algorithm.

We first give a definition of subgraph of a query-based dependency graph with respect to a variable-based dependency graph.
\begin{defn}
[Subgraph]
Given a query-based dependency graph $G_{s} = (V_1, E_1)$, a variable-based dependency graph $G_{ssa} = (V_2, E_2)$, $G_{s} \subseteq G_{ssa}$ iff:\\
$\exists f$ so that \\
1. for every $v \in V_1$, $f(v) \in V_2$. 
\\
2. $\forall e=(v_i, v_j) \in E_1$, there exists a path 
% $g(e)$ 
from $f(v_i)$ to $f(v_2)$ in $G_{ssa}$.
\end{defn}

Then we show the soundness of {\THESYSTEM}. In the theorem, we use some definition. $G \vDash M, V$ says that $G$ and $M$, $V$ have the corresponding size. $G; w \vDash (\ssa{c}, i_1,i_2)$ checks if the variables assigned in $\ssa{c}$ calculated by $\mathsf{AG}$ matches the variables in $G$ from index $i_1$ to $i_2$.

\begin{thm}
[Soundness of {\THESYSTEM}]
Given $ \ad{\Gamma; \ssa{c}; i_1 }{M; V;i_2}$,  for any global list $G$,  loop maps $w$ such that $G ;w \vDash (\ssa{c}, i_1, i_2) \land G \vDash (M,V)$. $K$ is the number of queries inquired during the execution of the piece of program $\ssa{c}$ and |V| gives the number of non-zeros in $V$. 
% $|.|_{low} $ is the annotation erasure, which turns a ssa form program $\ssa{c}$ to its low-level version.
Then,
\[
K \leq |V| \land \forall D, \ssa{m}. G_{s}(\ssa{c},D,\ssa{m},w) \subseteq G_{ssa}(M, V,G,i_1, i_2)
\]      
\end{thm}


\section{More examples}
\label{sec:examples}
%
\label{ex:multipleRounds}
\label{ex:multiRoundsS}
We illustrate here how our analysis work on two different examples.
%\begin{example}[Multiple Rounds Algorithm]
%\label{ex:multipleRounds}
%
%%%%%%%%%%%%%%%%%%%%%%%%%%%%%%%%%% Previous Version For Reference %%%%%%%%%%%%%%%%%%%%%%%%%%%%%%%%%%
% We look at an advanced adaptive data analysis algorithm - $\kw{multipleRounds}$ algorithm in Fig.~\ref{fig:multipleRounds}(a).
% This is a simplified version of the \emph{Monitor Augment} from \cite{RogersRSSTW20} with complete program in Apdix.
% It takes the user input $k$ which decides the 
% number of iterations.
% It starts from an initialized empty tracking list $I$,
% goes $k$ rounds and at every round, tracking list $I$ is updated by a query result of $\query(\chi[I])$.
% After $r$ rounds, the algorithm returns the columns of the hidden database $D$ not specified in the tracking list $I$.
% The functions $\kw{updnscore}(p,a)$,
% $\kw{updcscore}(p,a)$, $\kw{update}(I,ns,cs)$ simplify the computations of updating $ns$, $cs$ and $I$.%
Our first example, Algorithm $\kw{multipleRounds}$ in Fig.~\ref{fig:multipleRounds}(a), is a simplified form of the \emph{monitor argument} by \citet{RogersRSSTW20}.
The input $k$ is the number of iterations.
It uses a list $I$ to track queries. Specifically at each iteration it updates $I$ by using the result of a query which relies on $I$:  $\query(\chi[I])$.
After $k$ iterations, the algorithm returns the columns of the hidden database $D$ which are not contained in the  tracking list $I$.
The functions $\kw{updnscore}(p,a)$,
$\kw{updcscore}(p,a)$, $\kw{update}(I,ns,cs)$ simplify the computations of updating $ns$, $cs$ and $I$. They depends on the result of the query but they do not perform queries themselves%
%
%
%%%%%%%%%%%%%%%%%%%%%%%%%%%%%%%%%% Previous Version For Reference %%%%%%%%%%%%%%%%%%%%%%%%%%%%%%%%%%
% {Different from $\kw{twoRounds(k)}$ in Fig.~\ref{fig:overview-example},
% the query request, in each loop iteration is not independent. 
% The query in the $j^{th}$ iteration now depends on the tracking list $I$ from the previous  $(j - 1)^{th}$ iteration, 
% $I$ is updated by all the query results in the previous $j-1$ iterations. 
% In this sense, all these $k$ queries are adaptively chosen according to our discussion in overview.
% }
Different from the code in the example $\kw{twoRounds(k)}$,
the query request, $\clabel{\assign{a}{\query(I)}}^6$, in each loop iteration
depends on the tracking list $I$, which in turn depends on  all the querieas from previous iterations. 
%In particular, $I$ depends updated by all the query results in the previous iterations as well. 
In this sense, all these $k$ queries are fully adaptively chosen, and so the adaptivity is $k$.
%%%%%%%%%%%%%%%%%%%%%%%%%%%%%%%%%% Previous Version For Reference %%%%%%%%%%%%%%%%%%%%%%%%%%%%%%%%%%
% The program-based dependency graph is presented  in Fig.~\ref{fig:multi_graphs}(b). 
% We omitted its execution-based dependency graph $\traceG(\kw{multipleRounds(k)})$ because they have the same graph topology and only differ in weights.
% For the vertices in Fig.~\ref{fig:multi_graphs}(b) that have the weight $k$,
% their weights in $\traceG(\kw{multipleRounds(k)})$ are 
% the function $f_k$. $f_k$ takes an initial trace as input and returns the value of $k$ from the initial trace. 
% And the vertices with weight $1$ have the constant function $f_1 : \tdom_0(\kw{twoRounds(k)}) \to \{1\}$ 
% in  $\traceG(\kw{multipleRounds(k)})$. 
% For simplicity, we abuse the same symbols, $f_k$ and $f_1$ for all the following examples in their execution-based dependency graph
% to denote the weight function of a vertex. $f_k$ and $f_1$ compute the same value as defined above with the input initial trace w.r.t. different examples.
The estimated dependency graph $\progG(\kw{multipleRounds(k)})$ is presented in Fig.~\ref{fig:multipleRounds}(b) and we omitted the semantics-based dependency graph $\traceG(\kw{multipleRounds(k)})$ because it has the same topology and only differ in weights.
%
%%%%%%%%%%%%%%%%%%%%%%%%%%%%%%%%%% Previous Version For Reference %%%%%%%%%%%%%%%%%%%%%%%%%%%%%%%%%%
% As the adaptivity definition in our formal adaptivity model in Def.~\ref{def:trace_adapt},
% there is a finite walk along the dashed arrows,
% $a^{6} \to I^9 \to ns^{7} \to  \cdots \to ns^7$ , 
% where every vertex is visited $f_k(\trace_0)$ times given input $\trace_0$.
% The vertex $a^{6}$ has query annotation $1$, and it is visited $f_k(\trace_0)$ times.
% In this sense, the adaptivity of this program is
% $f_k(\trace_0)$ given input $\trace_0$.
% Since $f_k(\trace_0)$ computes the value of input variable $k$ from $\trace_0$, we have
% \begin{equation}
%     \label{eq:adapt_multipleRounds}
%     \forall \trace_0 \in \tdom_0(\kw{multipleRounds(k)}) \st  A(\kw{multipleRounds(k)})(\trace_0) = \env(\trace_0) k
% \end{equation} 
% where $\env$ is the environment operator.
% \jl{
% By the adaptivity definition in Def.~\ref{def:trace_adapt},
% there is a finite walk along the dashed arrows,
% $a^{6} \to I^9 \to ns^{7} \to  \cdots \to ns^7$ , 
% where $a^{6}$, $I^9$ and $ns^{7}$ are visited $w_{a^{6}}(\trace_0)$,
% $w_{I^9}(\trace_0)$ and $w_{ns^{7}}(\trace_0)$
% times respectively with input $\trace_0$.
% The vertex $a^{6}$ has query annotation $1$, and it is visited $w_{a^{6}}(\trace_0)$ times.
% In this sense, the adaptivity of this program is
% $w_{a^{6}}(\trace_0)$ given input $\trace_0$, i.e., $A(\kw{multipleRounds(k)})(\trace_0) = w_{a^{6}}(\trace_0)$.
% Since $w_{a^{6}}(\trace_0)$
% counts the execution times of command $\clabel{\assign{a}{\query(I)}}^6;$,
% this count is at most the loop iteration numbers, i.e., $k$'s initial value, $\env(\trace_0) k$ from the initial trace $\trace_0$.
% }
{
Our program analysis {$\THESYSTEM$} provides a tight upper bound for this example using $\pathsearch(\kw{multipleRounds(k)})$.
It first finds a path on the graph $\progG(\kw{multipleRounds(k)})$
$a^{6}: {}^k_1 \to I^9:{}^k_0 \to ns^7:{}^k_0$ with three weighted vertices. 
Then $\pathsearch$ algorithm transforms this path into a walk $a^{6}: {}^k_1 \to I^9:{}^k_0 \to ns^7:{}^k_0 \to a^{6}: {}^k_1 \cdots$, where $a^6, I^9, ns^{7}$ are all visited $k$ times respectively. 
So $\progA(\kw{multipleRounds(k)}) = k$.
We know for any initial trace $\trace_0$, $\config{\trace_0, k} \earrow \env(\trace_0)k$, i.e., $A(\kw{multipleRounds(k)})(\trace_0) \leq \env(\trace_0)k$ for any $\trace_0$, 
and so what we have produced is a tight and sound bound.
}
% \end{example}
%
\begin{figure}
\centering
\begin{subfigure}[t]{0.3\textwidth}
    \small{
    $
\begin{array}{l}
\kw{multipleRounds(k)} \triangleq\\
    \clabel{\assign{j}{k}}^0;
    \clabel{\assign{I}{[]}}^1; \\
    \clabel{\assign{ns}{0}}^2; 
    \clabel{\assign{cs}{0}}^3; \\
    \ewhile ~ \clabel{j > 0}^{4} ~ \edo ~ \\
    \Big(
    \clabel{\assign{j}{j-1}}^{5} ;\\
    \clabel{\assign{a}{\query(I)}}^6; \\
    \clabel{\assign{ns}{\kw{updnscore}(ns, a)}}^7; \\
    \clabel{\assign{cs}{\kw{updcscore}(cs, a)}}^8; \\
    \clabel{\assign{I}{\kw{updI}(I, ns, cs)}}^9
    \Big) 
\end{array}
    $
    }
    \caption{}
\end{subfigure}
        \begin{subfigure}{.45\textwidth}
        \begin{centering}
        \begin{tikzpicture}[scale=\textwidth/15cm,samples=200]
    % Variables Initialization
     \draw[] (-7, 1) circle (0pt) node{{ $I^1: {}^1_{0}$}};
     \draw[] (-7, 7) circle (0pt) node{{$ns^2: {}^{1}_{0}$}};
     \draw[] (-7, 4) circle (0pt) node{{ $cs^3: {}^{1}_{0}$}};
     % Variables Inside the Loop
     \draw[] (0, 10) circle (0pt) node{{ $a^6: {}^{k}_{1}$}};
     \draw[] (0, 7) circle (0pt) node{{ $ns^7: {}^{k}_{0}$}};
     \draw[] (0, 4) circle (0pt) node{{ $cs^8: {}^{k}_{0}$}};
     \draw[] (0, 1) circle (0pt) node{{ $I^9: {}^{k}_{0}$}};
     % Counter Variables
     \draw[] (7, 9) circle (0pt) node {{$j^0: {}^{1}_{0}$}};
     \draw[] (7, 6) circle (0pt) node {{ $j^5: {}^{k}_{0}$}};
     %
     % Value Dependency Edges:
     \draw[  -latex,] (0, 1.5)  -- (0, 3.5) ;
     \draw[ ultra thick, -latex, densely dotted,] (0, 7.5)  -- (0, 9.5) ;
     \draw[  -Straight Barb] (1.4, 4) arc (120:-200:1);
     \draw[  -Straight Barb] (8.5, 6.5) arc (150:-150:1);
     \draw[  -Straight Barb] (1, 7.5) arc (220:-100:1);
     \draw[  -latex] (7, 6.5)  -- (7, 8.5) ;
     % Value Dependency Edges on Initial Values:
     \draw[  -latex,] (-1.5, 1)  -- (-5.5, 1) ;
     \draw[  -latex,] (-1.5, 4)  -- (-5.5, 4) ;
     \draw[  -latex,] (-1.5, 7)  -- (-5.5, 7) ;
     %
     \draw[ ultra thick, -latex, densely dotted,] (-1, 9.5)  to  [out=-130,in=130]  (-1, 1.5);
     \draw[ ultra thick, -latex, densely dotted,] (-0.8, 1.7)  to  [out=-230,in=230]  (-0.5, 6.5);
     % Value Dependency from cs8 -> a6
     \draw[  -latex, ] (-0.8, 4.0)  to  [out=-230,in=230]  (-0.5, 9.5);
     % Value Dependency from a6 -> I1
     \draw[  -latex,] (-1.2, 9.7)  -- (-5.5, 1);
     \draw[  -Straight Barb] (1.7, 1.5) arc (120:-200:1);
     % Control Dependency
     \draw[  -latex] (1.5, 7)  -- (5.8, 6) ;
     \draw[  -latex] (1.5, 4)  -- (5.8, 6) ;
     \draw[  -latex] (1.5, 1)  -- (5.8, 6) ;
     \draw[  -latex] (1.5, 10)  -- (5.8, 6) ;
     % Edges Produced by Transitivity by Control Dependency
     \draw[  -latex] (1.5, 7)  -- (5.8, 9) ;
     \draw[  -latex] (1.5, 4)  -- (5.8, 9) ;
     \draw[  -latex] (1.5, 1)  -- (5.8, 9) ;
     \draw[  -latex] (1.5, 10)  -- (5.8, 9) ;
     % Edges Produced by Transitivity from vertext a6 Dependency
     \draw[  -latex,] (-1.2, 9.7)  -- (-5.5, 4);
     \draw[  -latex,] (-1.2, 9.7)  -- (-5.5, 7);
     \draw[ -latex] (-1, 9.5)  to  [out=-130,in=130]  (-1, 7.5);
     \draw[ -latex] (0.5, 9.5)  to  [out=-50,in=50]  (0.5, 4);
     \draw[  -Straight Barb] (0.5, 10.5) arc (150:-150:1);
     % Edges Produced by Transitivity from vertext cs8 Dependency
     \draw[  -latex,] (-1.2, 4)  -- (-5.5, 1);
     \draw[  -latex,] (-1.2, 4)  -- (-5.5, 7);
     \draw[  -latex,] (0, 4.5)  -- (0, 6.5) ;
     % Edges Produced by Transitivity from vertext I9 Dependency
     \draw[  -latex,] (-1.2, 1)  -- (-5.5, 4);
     \draw[  -latex,] (-1.2, 1)  -- (-5.5, 7);
     \draw[  -latex,] (0.5, 1.0)  to  [out=50,in=-50]  (0.5, 9.5);
     % Edges Produced by Transitivity from vertext ns7 Dependency
     \draw[  -latex,] (-1.2, 7)  -- (-5.5, 1);
     \draw[  -latex,] (-1.2, 7)  -- (-5.5, 4);
     \draw[ -latex] (0.5, 6.5)  to  [out=-50,in=50]  (0.5, 1.5);
     \draw[ -latex] (0.5, 6.5)  to  [out=-50,in=50]  (0.5, 4.5);
     \end{tikzpicture}
     \caption{}
        \end{centering}
        \end{subfigure}
    \vspace{-0.4cm}
    \caption{(a) The simplified multiple rounds example (b) The estimated dependency graph from $\THESYSTEM$}
    \vspace{-0.5cm}
    \label{fig:multipleRounds}
\end{figure}
%
% %
\begin{example}[Linear Regression Algorithm with Gradient Decent Optimization]
\label{ex:linearregression}
    The linear regression algorithm with gradient decent Optimization works well 
    in our $\THESYSTEM$ as well.
            %   \[
            %   %
            %   \begin{array}{l}
            %   \kw{linearRegression(step, rate)} \triangleq \\
            %          \clabel{ a \leftarrow 0}^{0} ; \\
            %          \clabel{ c \leftarrow 0}^{1} ; \\
            %           \clabel{\assign{j}{\kw{step}} }^{2} ; \\
            %         %   \clabel{\assign{d}{10000000} }^{2} ; \\
            %           \ewhile ~ \clabel{j > 0}^{3} ~ \edo ~ \\
            %           \Big(
            %               \clabel{\assign{da}{\query(-2 * (\chi[1] - (\chi[0]\times a + c)) \times (\chi[0]))} }^{4}  ; \\
            %               \clabel{\assign{dc}{\query(-2 * (\chi[1] - (\chi[0]\times a + c)))} }^{5}  ; \\
            %               \clabel{\assign{a}{a - \kw{rate} * da} }^{6}  ; \\
            %               \clabel{\assign{c}{c - \kw{rate} * dc} }^{7}  ; \\
            %            \clabel{\assign{j}{j-1}}^{8} 
            %         %   \clabel{a \leftarrow x :: a}^{6} 
            %           \Big);
            %       \end{array}
            %   \]
              %
              %
                   %
\begin{figure}
\centering
\begin{subfigure}{0.45\textwidth}
    \centering
    {\small
        \[
        \begin{array}{l}
            \kw{linearRegressionGD(k, rate)} \triangleq \\
                   \clabel{ a \leftarrow 0}^{0} ; 
                   \clabel{ c \leftarrow 0}^{1} ; 
                    \clabel{\assign{j}{\kw{k}} }^{2} ; \\
                  %   \clabel{\assign{d}{10000000} }^{2} ; \\
                    \ewhile ~ \clabel{j > 0}^{3} ~ \edo ~ \\
                    \Big(
                        \clabel{\assign{da}{\query(-2 * (\chi[1] - (\chi[0]\times a + c)) \times (\chi[0]))} }^{4}  ; \\
                        \clabel{\assign{dc}{\query(-2 * (\chi[1] - (\chi[0]\times a + c)))} }^{5}  ; \\
                        \clabel{\assign{a}{a - \kw{rate} * da} }^{6}  ; 
                        \clabel{\assign{c}{c - \kw{rate} * dc} }^{7}  ; \\
                     \clabel{\assign{j}{j-1}}^{8} 
                  %   \clabel{a \leftarrow x :: a}^{6} 
                    \Big);
                \end{array}
        \]
        }
     \caption{}
        \end{subfigure}
      \begin{subfigure}{.45\textwidth}
          \begin{centering}
          \begin{tikzpicture}[scale=\textwidth/20cm,samples=200]
    % Variables Initialization
    \draw[] (-6, 1) circle (0pt) node{{ $a^0: {}^1_{0}$}};
    \draw[] (-6, 4) circle (0pt) node{{ $c^1: {}^{1}_{0}$}};
    % Variables Inside the Loop
       \draw[] (0, 10) circle (0pt) node{{ $da^4: {}^{k}_{1}$}};
       \draw[] (0, 7) circle (0pt) node{{ $dc^5: {}^{k}_{0}$}};
       \draw[] (0, 4) circle (0pt) node{{ $a^6: {}^{k}_{0}$}};
       \draw[] (0, 1) circle (0pt) node{{ $c^7: {}^{k}_{0}$}};
       % Counter Variables
       \draw[] (7, 9) circle (0pt) node {{$j^0: {}^{1}_{0}$}};
       \draw[] (7, 6) circle (0pt) node {{ $j^8: {}^{k}_{0}$}};
       %
       % Value Dependency Edges:
       \draw[ thick, -latex,] (0, 1.5)  -- (0, 3.5) ;
       \draw[ thick, -Straight Barb] (1.8, 4.2) arc (220:-100:1);
       \draw[ thick, -Straight Barb] (7.5, 6.5) arc (150:-150:1);
       \draw[](10, 6) node[] {\highlight{$k$}} ;
       \draw[ thick, -Straight Barb] (1.7, 1.) arc (120:-200:1);
       \draw[](4, 0) node[] {\highlight{$k$}} ;
       \draw[ thick, -latex] (6, 6.5)  -- 
       node [right] {\highlight{$k$}}(6, 8.5) ;
       % Value Dependency Edges on Initial Values:
       \draw[ thick, -latex,] (-2, 1)  -- 
       node [above] {\highlight{$k$}}(-4.5, 1) ;
       \draw[ thick, -latex,] (-2, 4)  -- 
       node [above] {\highlight{$k$}}(-4.5, 4) ;
       %
       \draw[ ultra thick, -latex, densely dotted,] (-1, 1.5)  to  [out=-220,in=220]  
       node [below] {\highlight{$k$}}(-1, 6.5);
       \draw[ ultra thick, -latex, densely dotted,] (-1, 4.5)  to  [out=-220,in=220]  
       node [above] {\highlight{$k$}}(-1, 9.5);
       \draw[ ultra thick, -latex, densely dotted,]  (1, 6.2) to  [out=-60,in=60] 
       node [below] {\highlight{$k$}}(0.5, 1.5) ;
       \draw[ ultra thick, -latex, densely dotted,]  (1.2, 9.2)  to  [out=-50,in=50] 
       node [above] {\highlight{$k$}}(0.5, 4.5);
       % Control Dependency
      %  \draw[ thick,-latex] (1.5, 7)  -- (4, 9) ;
      %  \draw[ thick,-latex] (1.5, 4)  -- (4, 9) ;
       \draw[ thick,-latex] (1.8, 10)  -- 
       node [above] {\highlight{$k$}}(5.5, 6) ;
       \draw[ thick,-latex] (1.8, 7)  -- (5.5, 6) ;
       \draw[ thick,-latex] (1.8, 4)  -- 
       node [above] {\highlight{$k$}}(5.5, 6) ;
       \draw[ thick,-latex] (1.8, 1)  -- 
       node [below] {\highlight{$k$}}(5.5, 6) ;
       \end{tikzpicture}
       \caption{}
          \end{centering}
          \end{subfigure}
          %
        \begin{subfigure}{.8\textwidth}
            \begin{centering}
            \begin{tikzpicture}[scale=\textwidth/20cm,samples=200]
      % Variables Initialization
      \draw[] (-6, 1) circle (0pt) node{{ $a^0: {}^1_{0}$}};
      \draw[] (-6, 4) circle (0pt) node{{ $c^1: {}^{1}_{0}$}};
      % Variables Inside the Loop
         \draw[] (0, 10) circle (0pt) node{{ $da^4: {}^{k}_{1}$}};
         \draw[] (0, 7) circle (0pt) node{{ $dc^5: {}^{k}_{0}$}};
         \draw[] (0, 4) circle (0pt) node{{ $a^6: {}^{k}_{0}$}};
         \draw[] (0, 1) circle (0pt) node{{ $c^7: {}^{k}_{0}$}};
         % Counter Variables
         \draw[] (7, 9) circle (0pt) node {{$j^0: {}^{1}_{0}$}};
         \draw[] (7, 6) circle (0pt) node {{ $j^8: {}^{k}_{0}$}};
         %
         % Value Dependency Edges:
         \draw[ thick, -latex,] (0, 1.5)  -- (0, 3.5) ;
         \draw[ thick, -Straight Barb] (1.8, 4.2) arc (220:-100:1);
         \draw[ thick, -Straight Barb] (7.5, 6.5) arc (150:-150:1);
         \draw[](10, 6) node[] {\highlight{$\trace_0 \to \env(\trace_0) k $}} ;
         \draw[ thick, -Straight Barb] (1.7, 1.) arc (120:-200:1);
         \draw[](4, 0) node[] {\highlight{$\trace_0 \to \env(\trace_0) k $}} ;
         \draw[ thick, -latex] (6, 6.5)  -- 
         node [right] {\highlight{$\trace_0 \to \env(\trace_0) k $}}(6, 8.5) ;
         % Value Dependency Edges on Initial Values:
         \draw[ thick, -latex,] (-2, 1)  -- 
         node [above] {\highlight{$\trace_0 \to \env(\trace_0) k $}}(-4.5, 1) ;
         \draw[ thick, -latex,] (-2, 4)  -- 
         node [above] {\highlight{$\trace_0 \to \env(\trace_0) k $}}(-4.5, 4) ;
         %
         \draw[ ultra thick, -latex, densely dotted,] (-1, 1.5)  to  [out=-220,in=220]  
         node [below] {\highlight{$\trace_0 \to \env(\trace_0) k $}}(-1, 6.5);
         \draw[ ultra thick, -latex, densely dotted,] (-1, 4.5)  to  [out=-220,in=220]  
         node [above] {\highlight{$\trace_0 \to \env(\trace_0) k $}}(-1, 9.5);
         \draw[ ultra thick, -latex, densely dotted,]  (1, 6.2) to  [out=-60,in=60] 
         node [below] {\highlight{$\trace_0 \to \env(\trace_0) k $}}(0.5, 1.5) ;
         \draw[ ultra thick, -latex, densely dotted,]  (1.2, 9.2)  to  [out=-50,in=50] 
         node [above] {\highlight{$\trace_0 \to \env(\trace_0) k $}}(0.5, 4.5);
         % Control Dependency
        %  \draw[ thick,-latex] (1.5, 7)  -- (4, 9) ;
        %  \draw[ thick,-latex] (1.5, 4)  -- (4, 9) ;
         \draw[ thick,-latex] (1.8, 10)  -- 
         node [above] {\highlight{$\trace_0 \to \env(\trace_0) k $}}(5.5, 6) ;
         \draw[ thick,-latex] (1.8, 7)  -- (5.5, 6) ;
         \draw[ thick,-latex] (1.8, 4)  -- 
         node [above] {\highlight{$\trace_0 \to \env(\trace_0) k $}}(5.5, 6) ;
         \draw[ thick,-latex] (1.8, 1)  -- 
         node [below] {\highlight{$\trace_0 \to \env(\trace_0) k $}}(5.5, 6) ;
         \end{tikzpicture}
         \caption{}
            \end{centering}
            \end{subfigure}
    \vspace{-0.5cm}
    \caption{(a) The linear regression algorithm 
    (b) The program-based dependency graph from $\THESYSTEM$
    (c) The execution-based dependency graph.}
    \vspace{-0.5cm}
    \label{fig:linear_regression}
\end{figure}
%
Analysis Result: $ \progA(\kw{linearRegressionGD(k, rate)}) = k$
\end{example} 
%
 
This linear regression algorithm 
% in order to
aims to
model a linear relationship between a dependent variable $y$,
% corresponding to the observed value in the column $\chi[1]$ in database, 
and an independent variable $x$, $y = a \times x + c$, specifically approximating the 
model parameter $a$ and $c$.
In order to have a good approximation on the model parameter 
$a$ and $c$, 
% corresponding to the observed value in the column $\chi[0]$ in database, 
it sends query to a training data set adaptively in every iteration.
This training data set contains two columns (can extend to higher dimensional data sets), first column is used as the observed value for the independent variable $x$,
second column is used as the observed label value for the dependent variable $y$.
This algorithm is written in our {\tt Query While} language in Figure~\ref{fig:linear_regression}(a) as $\kw{linearRegressionGD(k, rate)}$.
% taking the iteration number $\kw{step}$ 

This linear regression algorithm starts from initializing the linear model parameters and the counter variable,
and then goes into the training iterations.
In each iteration, it computes the differential value w.r.t. parameter
$a$ and $c$ respectively,
through requesting two queries, $\query(-2 * (\chi[1] - (\chi[0]\times a + c)) \times (\chi[0]))$ and 
$\query(-2 * (\chi[1] - (\chi[0]\times a + c)))$
at line 4 and 5.
Then, it uses these two differential values stored in variable $da$ and $dc$ to update the linear model parameters $a$ and $c$.
%
Its the program-based dependency graph is shown in Figure~\ref{fig:linear_regression}(b). Its execution-based dependency graph share the same graph, only needs to change the weight, $k$ into $w_k$ and $1$ for $w_1$ as we do in the previous example.
% We omit the detail of how to 
% generate this graph, which is similar to the generation procedure in 
% Example~\ref{alg:multiRound}.
In the execution-based dependency graph, there are multiple walks having the same longest query length.
For example, the walk $c^7 \to dc^6 : \to c^7 \to \cdots \to dc^6$ along the 
dotted arrows, where each vertex is visited $w_k(\trace_0)$ times for an initial trace $\trace_0$.
% By counting the total occurrence time of vertices with annotation $1$ in this walk, we have this program's adaptivity $k$.
There is actually other walks having the same query length $k$, the 
walk $a^7 \to da^6  \to a^7 \to \cdots \to da^6 $ along the 
dotted arrows, where each vertex is visited $w_k(\trace_0)$ times.
% the dotted path corresponds to a finite walk with the longest query length and its adaptivity on this walk is $k$.
But it doesn't affect the adaptivity for this program, which is still the maximal query length $w_k(\trace_0)$ with respect to initial trace $\trace_0$.
Also, $\THESYSTEM$, estimates the adaptivity $k$ for this example. Similarly as the multiple round example, we can show it is a tight bound.
%
% \begin{example}
[Multiple Rounds Odds Algorithm]
\label{ex:overapproximate}
The $\THESYSTEM$ comes across an over-approximation due to its path-insensitive nature. 
It occurs when the control flow can be decided in a particular way in front of conditional branches,
while the static analysis fails to witness. 

As in Figure~\ref{fig:overappr_example}(a), $\kw{multipleRoundsOdd}(k)$
is an example program with $1 + k$ adaptivity rounds and two paths while loop.
% we call it a multiple rounds odd iteration algorithm. 
In each iteration, 
the query $\query(\chi[x])$ in command $5$ is based on previous query results stored in $x$, which is similar to Example~\ref{ex:multipleRounds}.
The difference is that, only the query answers from the even iterations ($i = 0, 2, \cdots $) are 
% used to $b$. 
used in the query 
in command $7$, $\query(\chi[\ln(y)])$.
  Because the execution trace only updates 
%   $b$ using the query answers at odd iterations, so the answers from even iterations do not affect the queries at odd iterations. From the query-based dependency graph in Figure~\ref{fig:overappr_example}(b), we can see that there is no edge from queries at odd iterations (such as $q_1,q_3,q_5$) to queries at even iteration(such as $q_2,q_4$). The longest path is dashed with a length $3$.  However, {\THESYSTEM} fails to realize that odd iteration will always execute then branch and even iteration means else branch, so its dependency graph considers both branches for every iteration. In this sense, the dependency graph by {\THESYSTEM} is similar to the one in the multiple rounds strategy. We show the estimated graph in Figure~\ref{fig:overappr_example}(c). The estimated upper bound is then, $5$, instead of $3$. 
$x$ using the query answers in even iterations, so the answers from odd iterations do not affect the queries in even iterations. 
From the execution-based dependency graph in Figure~\ref{fig:overappr_example}(b), 
we can see that the weight for the vertex $y^5$ is 
$f_{k/2}$.
$f_{k/2} : \trace_0 \to (\env(\trace_0) k) / 2$, computes return the value of $k/2$ from input initial $\trace_0$.  
However, {\THESYSTEM} fails to realize that all the odd iterations only execute the first branch
and only even iterations execute the second branch. 
% its dependency 
So it considers both branches for every iteration when estimating the adaptivity. 
In this sense, the weight estimated for $y^5$ and $p^6$ are both 
$k$ as in Figure~\ref{fig:overappr_example}(c).
As a result, {\THESYSTEM}  estimates the longest walk from Figure~\ref{fig:overappr_example}(c) as
$y^5  \to x^7  \to y^5  \to \cdots \to x^7  $ with each vertex being visited $k$ times.
And the computed adaptivity  
% estimated from the program-based dependency graph graph from by finding the walk with the longest query length 
is $1 + 2 * k$, instead of $1 + k$. 
% We omitted the estimated graph, which is identical to the graph in Figure~\ref{fig:overappr_example}(b). 
%
{ \small
\begin{figure}
\centering
    \begin{subfigure}{0.33\textwidth}
\centering
\small{
    \[
    %
    \begin{array}{l}
        \kw{multipleRoundsOdd}(k) \triangleq \\
        \clabel{ \assign{j}{k}}^{0} ; 
        \clabel{ \assign{x}{\query(\chi[0])} }^{1} ; \\
            \ewhile ~ \clabel{j > 0}^{2} ~ \edo ~ 
            \Big(
             \clabel{\assign{j}{j-1}}^{3} ;\\
             \eif(\clabel{j \% 2 == 0}^{4}, \\
             \clabel{\assign{y}{\chi[x]}}^{5}, 
             \clabel{\assign{p}{\chi[x]}}^{6});\\                            
             \clabel{\assign{x}{\query(\chi(\ln(y)))} }^{7} \Big)
        \end{array}
    \]
}
 \caption{}
    \end{subfigure}
%
\begin{subfigure}{.31\textwidth}
    \begin{centering}
    \begin{tikzpicture}[scale=\textwidth/11cm,samples=200]
% Variables Initialization
\draw[] (5, 1) circle (0pt) node{{ $x^1: {}^{f_1}_{1}$}};
% Variables Inside the Loop
 \draw[] (0, 7) circle (0pt) node{{ $y^5: {}^{f_k/2}_{1}$}};
 \draw[] (0, 4) circle (0pt) node{{ $p^6: {}^{f_k/2}_{1}$}};
 \draw[] (0, 1) circle (0pt) node{{ $x^7: {}^{f_k}_{1}$}};
 % Counter Variables
 \draw[] (5, 7) circle (0pt) node {{$j^0: {}^{f_1}_{0}$}};
 \draw[] (5, 4) circle (0pt) node {{ $j^3: {}^{f_k}_{0}$}};
 %
 % Value Dependency Edges:
 \draw[ thick, -latex,]  (0, 3.5) -- (0, 1.5) ;
%  \draw[ thick, -Straight Barb] (1, 4.2) arc (220:-100:1);
 \draw[ thick, -Straight Barb] (6.5, 4.5) arc (150:-150:1);
 \draw[ thick, -latex] (5, 4.5)  -- (5, 6.5) ;
%  \draw[ thick, -Straight Barb] (1., 1.5) arc (120:-200:1);
 % Value Dependency Edges on Initial Values:
 \draw[ thick, -latex,] (1.5, 1)  -- (4, 1) ;
 %
 \draw[ ultra thick, -latex, densely dotted,] (-0.6, 1.5)  to  [out=-220,in=220]  (-0.5, 6.5);
\draw[ ultra thick, -latex, densely dotted,]  (0.5, 6.5) to  [out=-30,in=30] (0.6, 1.6) ;
%  \draw[ ultra thick, -latex, densely dotted,]  (0.5, 10)  to  [out=-50,in=50] (0.5, 4);
 % Control Dependency
 \draw[ thick,-latex] (1.5, 7)  -- (4, 6) ;
 \draw[ thick,-latex] (1.5, 4)  -- (4, 6) ;
 \draw[ thick,-latex] (1.5, 1)  -- (4, 6) ;
%  \draw[ thick,-latex] (1.5, 10)  -- (4, 6) ;
 \end{tikzpicture}
 \caption{}
    \end{centering}
    \end{subfigure}
    \begin{subfigure}{.31\textwidth}
        \begin{centering}
        \begin{tikzpicture}[scale=\textwidth/11cm,samples=200]
    % Variables Initialization
    \draw[] (5, 1) circle (0pt) node{{ $x^1: {}^1_{1}$}};
    % Variables Inside the Loop
     \draw[] (0, 7) circle (0pt) node{{ $y^5: {}^{k}_{1}$}};
     \draw[] (0, 4) circle (0pt) node{{ ${p^6: {}^{k}_{1}}$}};
     \draw[] (0, 1) circle (0pt) node{{ ${x^7: {}^{k}_{1}}$}};
     % Counter Variables
     \draw[] (5, 7) circle (0pt) node {{$j^0: {}^{1}_{0}$}};
     \draw[] (5, 4) circle (0pt) node {{ $j^3: {}^{k}_{0}$}};
     %
% Value Dependency Edges:
 \draw[ thick, -latex,]  (0, 3.5) -- (0, 1.5) ;
%  \draw[ thick, -Straight Barb] (1, 4.2) arc (220:-100:1);
 \draw[ thick, -Straight Barb] (6.5, 4.5) arc (150:-150:1);
 \draw[ thick, -latex] (5, 4.5)  -- (5, 6.5) ;
%  \draw[ thick, -Straight Barb] (1., 1.5) arc (120:-200:1);
 % Value Dependency Edges on Initial Values:
 \draw[ thick, -latex,] (1.5, 1)  -- (4, 1) ;
 %
 \draw[ ultra thick, -latex, densely dotted,] (-0.6, 1.5)  to  [out=-220,in=220]  (-0.5, 6.5);
\draw[ ultra thick, -latex, densely dotted,]  (0.5, 6.5) to  [out=-30,in=30] (0.6, 1.6) ;
%  \draw[ ultra thick, -latex, densely dotted,]  (0.5, 10)  to  [out=-50,in=50] (0.5, 4);
 % Control Dependency
 \draw[ thick,-latex] (1.5, 7)  -- (4, 6) ;
 \draw[ thick,-latex] (1.5, 4)  -- (4, 6) ;
 \draw[ thick,-latex] (1.5, 1)  -- (4, 6) ;
%  \draw[ thick,-latex] (1.5, 10)  -- (4, 6) ;
     \end{tikzpicture}
     \caption{}
        \end{centering}
        \end{subfigure}
        \vspace{-0.4cm}
\caption{(a) The multiple rounds odd example 
(b) The execution-based dependency graph
(c) The program-based dependency graph graph from $\THESYSTEM$.}
    \label{fig:overappr_example}
    \vspace{-0.5cm}
\end{figure}
}
%
\end{example}
\begin{example}[Single Adaptivity Round Example]
    \label{ex:multiRoundsS}
    The program's adaptivity definition in our formal model,
    (in Definition~\ref{def:trace_adapt})
    comes across an over-approximation when capturing the program's intuitive adaptivity rounds.
    It results from the difference between its weight calculation and the \emph{variable may-dependency} definition.
    It occurs when the weight is computed over the traces different from the traces used in 
    witnessing the \emph{variable may-dependency} relation.
    
    The program $\kw{multiRoundsS(k)}$ in Figure~\ref{fig:multiRoundsS}(a) demonstrates this over-approximation.
    It is a variant of the multiple rounds strategy with input $k$.
    In each iteration, the query request $\clabel{\assign{p}{\query(\chi[y]+p)} }^{7}$ is based on value stored in $p$ and $y$ from previous iteration.
    Differ from Example~\ref{ex:multipleRounds},
    only the query answer from the $(k - 2)^{th}$ iteration is used in the query request, $\clabel{\assign{p}{\query(\chi[y]+p)} }^{7}$ of the next $(k - 1)^{th}$ iteration.
    % $\clabel{\assign{p}{\query(\chi[y]+p)} }^{7}$.
    In all the other iterations, $j \neq (k - 2)$, the if-control goes to the first branch
    % Because the execution will reset
    $p$'s value is reset by the constant $0$ in command $ \clabel{\assign{p}{0}}^{9}$.
    % in all the other iterations
    % at line $10$ after this query request.
    In this way, all the query answers stored in $p$ are erased and are not used
    in the query request at the next iteration, except the one at the $(k - 2)^{th}$ iteration.
    Intuitively, when $k \geq 2$, only the $\clabel{\assign{p}{\query(\chi[y]+p)} }^{7}$ in the $(k - 1)^{th}$ iteration
    depends on the query in $(k - 2)^{th}$ iteration and the \emph{adaptivity} round is $2$.
    When $k = 0$, the program does not go into the loop and there is no dependency between any query request.
    When $k = 1$, the program goes to the first branch in the first if-control with guard $\clabel{ k = 1}^{3}$.
    In this case, the second query request $ \clabel{ \assign{y}{\query(z)}}^{4}$ depends on the first one,
    $\clabel{\assign{z}{\query(0)} }^{1}$ and then the next query in the loop, $\clabel{\assign{p}{\query(\chi[y]+p)} }^{7}$ depends on the second one. Intuitively, the adaptivity is $3$.
    % In this sense, the intuitive \emph{adaptivity} rounds for this example is $2$. 
    However, our adaptivity definition fails to realize that there is only a dependency relation 
    between $p^7$ to itself at the $(k - 2)^{th}$ iteration.
    % but not in all the others. 
    As shown in the semantics-based dependency graph in Figure~\ref{fig:multiRoundsS}(b), 
    there is a cycle on $p^7$ representing the existence of the \emph{Variable May-Dependency} from $p^7$ on itself.
    % and the visiting times of labeled variable $p^7$ is 
    % $\lambda \trace_0 \st k$. 
    Weight of this vertex is $\lambda \trace_0 \st \env(\trace_0) k$,
    % The function $\lambda \trace_0 \st \env(\trace_0) k$ 
    which returns the evaluation times of the command $\clabel{\assign{p}{\query(\chi[y]+p)} }^{7}$ during the program execution under the initial trace $\trace_0$.
    % , which is expected to be equal to the loop iteration numbers, i.e., an initial value of input $k$ from the initial trace $\trace_0$.
    Since the command $\clabel{\assign{p}{\query(\chi[y]+p)} }^{7}$  will always be evaluated the same time as the loop iteration numbers, i.e. $k$,
    the weight function returns $\env(\trace_0) k$.
    However, $\env(\trace_0) k$ is the total number that this command is evaluated, rather than the number of the evaluations in which this command depends on other query requests.
    As a result, the walk with the longest query length 
    is
    $p^7 \to \cdots \to p^7 \to y^4 \to z^1 $ with the vertex $p^7$ visited $\env(\trace_0) k$ times, as the dotted arrows. 
    The adaptivity based on this walk
    is $\lambda \trace_0 \st \env(\trace_0) k + 2$,
    which is expected to be $0$ when $k = 0$ and $2$ when $k \geq 2$.
    %  instead of $\max\{0, 2, 3\}$. 
    % Though the $\THESYSTEM$ is able to give us $2 + k$, as an accurate bound w.r.t this definition.
    {
    \begin{figure}
    \centering
    %}
    \quad
    \begin{subfigure}{.8\textwidth}
    \begin{centering}
    {
    $ \begin{array}{l}
    \kw{multiRoundsS(k)} \triangleq \\
    \clabel{ \assign{j}{0}}^{0} ; 
    \clabel{\assign{z}{\query(0)} }^{1} ; 
    \clabel{\assign{p}{0} }^{2} ; \\
    \eif(\clabel{ k = 1}^{3},
    \clabel{ \assign{y}{\query(z)}}^{4},
    \clabel{\eskip}^5);\\
    \ewhile ~ \clabel{j \neq k}^{6} ~ \edo ~ \Big(
    \\
    \qquad \clabel{\assign{p}{\query(\chi[y]+p)} }^{7} ; \\
    \qquad 
    \eif(\clabel{ j \neq k - 2}^{8}, 
    \clabel{ \assign{p}{0}}^{9} ,
    \clabel{\eskip}^{10})  \\ 
    \qquad \clabel{\assign{j}{j + 1}}^{11} ; 
    \Big)
    \end{array}
    $ 
    }
    \caption{}
    \end{centering}
    \end{subfigure}
    \begin{subfigure}{.8\textwidth}
    \begin{centering}
    \begin{tikzpicture}[scale=\textwidth/15cm,samples=200]
    % Variables Initialization
    \draw[] (-5, 2) circle (0pt) node{{ $z^1: {}^{\lambda \trace_0 \st 1}$}};
    \draw[] (-5, 7) circle (0pt) node{{$p^2: {}^{\lambda \trace_0 \st 1}$}};
    \draw[] (-5, 4) circle (0pt) node{{ $y^4: {}^{\lambda \trace_0 \st 1}$}};
    % Variables Inside the Loop
    \draw[] (0, 6) circle (0pt) node{ $p^7: {}^{\lambda \trace_0 \st \env(\trace_0) k}$};
    \draw[] (0, 2) circle (0pt) node{ $p^{10}: {}^{\lambda \trace_0 \st \env(\trace_0) k}$};
    % Counter Variables
    \draw[] (5, 6) circle (0pt) node {$j^0: {}^{\lambda \trace_0 \st 1}$};
    \draw[] (5, 2) circle (0pt) node { $j^8: {}^{\lambda \trace_0 \st \env(\trace_0) k}$};
    %
    % Value Dependency Edges:
    \draw[ thick, -Straight Barb, densely dotted,] (0.8, 7) arc (220:-100:1);
    \draw[ -latex] (-1.5, 5.5) to [out=-130,in=130] (-1.5, 2);
    % Value Dependency Edges on Initial Values:
    \draw[ thick, -latex, densely dotted,] (-5, 3.5) -- (-5, 2.5) ;
    \draw[ -latex,] (-1.5, 5.5) -- (-4, 7) ;
    \draw[ thick, -latex, densely dotted,] (-1.5, 5.5) -- (-4, 4.7) ;
    \draw[ -latex,] (-1.5, 5.5) -- (-4, 2) ;
    %
    % Value Dependency For Control Variables:
    \draw[ -Straight Barb] (6.5, 2.5) arc (150:-150:1);
    % Control Dependency
    \draw[ -latex] (5, 2.5) -- (5, 5.5) ;
    \draw[ -latex] (1.2, 6) -- (3.5, 6) ;
    \draw[ -latex] (1.2, 6) -- (3.5, 2) ;
    \draw[ -latex] (1.5, 1.8) -- (3.5, 2) ; 
    % Edges Produced by Transitivity
    \draw[ -latex] (1.5, 1.8) -- (3.5, 6) ; 
    \end{tikzpicture}
    \caption{}
    \end{centering}
    \end{subfigure}
    \caption{(a) The loop example with single adaptivity rounds.
    (b) The corresponding semantics-based dependency graph.}
    \label{fig:multiRoundsS}
    \end{figure}
    }
    \end{example}


\section{Related Work}

% \dg{Please cite adaptive Fuzz, and explain how its adaptivity analysis differs from ours.}

%In terms of techniques, our work relies on ideas from both static analysis and dynamic analysis. We discuss closely related work in both areas.


\paragraph{Dependency Definitions and Analysis} 
There is a vast literature on dependency definitions and dependency analysis. 
We consider a semantics definition of dependencies which consider (intraprocedural) data and control dependency~\cite{bilardi1996framework,cytron1991efficiently,pollock1989incremental}.    
Our definition is inspired by classical works on traditional dependency analysis~\cite{DenningD77} and noninterference~\cite{GoguenM82a}.
Formally, our definition is similar to the one by \citet{Cousot19a}, which also identifies dependencies by considering differences in two execution traces. 
However, Cousot excludes some forms of implicit dependencies, e.g. the ones generated by empty observations,  which instead we consider. 
%
Common tools to study dependencies are dependency graphs~\cite{ferrante1987program}. We use here a semantics-based approach to dependency graph similar, for example, to works by \citet{austin1992dynamic}, \citet{hammer2006dynamic} and \cite{mastroeni2008data}.
%propose ways of constructing different kinds of program slices, by choose different program 
%DDGs have been used in many other domains. \citet{nagar2018automated} use DDGs to find serializability violations. dependency. 
% For example, in either syntactic or semantics sense.
% This abstract dependency is based on properties rather than exact data.
% Aims to give finer and smaller program slice. 
%They actually use a combination of  
%static and dynamic dependency graphs but in a manner that is different from how we use the two. Their slicing uses both static and dynamic dependency graphs, while we use the dynamic dependency graph as the basis of a definition, which is then soundly approximated by an analysis based on the static dependency graph.}
%\paragraph{Static program analysis} 
%Our algorithm in Section~\ref{sec:algorithm} is influenced by previous works in static analysis related to effect systems, control-flow analysis, and data-flow analysis. 
%The idea of statically estimating a sound upper bound for the adaptivity from the semantics is indirectly inspired from prior work on cost analysis via effect systems~\cite{cciccek2017relational,radivcek2017monadic,qu2019relational}. The idea of defining adaptivity using data flow is inspired by the work of graded Hoare logic~\cite{gaboardi2021graded}, which reasons about data flows as a resource. 
%
Our approach shares some similarities with the use of dependency graphs in works analyzing dependencies between events, e.g. in event programming. \citet{memon2007event} uses an event-flow graph, representing all the possible event interactions, where vertices are GUI event edges represent pairs of events that can be performed immediately one after the other. In a similar way, we use edges to track the may-dependence between variables looking at all the possible interactions. 
% of one variable with respect to another variable. The main difference is in the way the graph is constructed. {\THESYSTEM} relies on the structure of the target program, while the event-flow model only considers the event type.
\citet{arlt2012lightweight} use a weighted edges indicating a dependency between two events, e.g. one event possibly reads data written by the other event, with the weight showing the intensity of the dependency (the quantity of data involved). We also use weights but on vertices and with different meaning, they are functions describing the number of times the vertices can be visited given an initial state.
% WCET on systems: \cite{} 
% [GustafssonEL05]Towards a Flow Analysis for Embedded System C Programs
% --> abstract interpretation.
% --> on embedded system of c program
% [AlbertAGP08] Automatic Inference of Upper Bounds for Recurrence Relations in Cost Analysis
% --> invariant generation through ranking functions
%
% General While langue:
% [BrockschmidtEFFG16]
% Analyzing Runtime and Size Complexity of Integer Programs
% --> invariant generation through ranking functions
% [AliasDFG10] Multi-dimensional Rankings, Program Termination, and Complexity Bounds of Flowchart Programs
% --> invariant generation through ranking functions
% [Flores-MontoyaH14]Resource Analysis of Complex Programs with Cost Equations
% --> invariant generation through cost equations or ranking functions
%
% [GulwaniJK09]Control-flow Refinement and Progress Invariants for Bound Analysis
% --> program abstraction and invariant inference
% []Bound Analysis using Backward Symbolic Execution
% --> program abstraction and invariant inference
%
% [CicekBG0H17]relational Cost Analysis 0
% Monadic refinements for relational cost analysis
% [RajaniG0021]A unifying type-theory for higher-order (amortized) cost analysis
% --> type-system
Differently from all these previous works, we use a dependency graph with quantitative information needed to identify the length of chain of dependencies. Our weight estimation is inspired by  works in complexity analysis and WCET. 
Specifically, it is inspired by works on  reachability-bound analysis using program abstraction and invariant inference~\cite{GulwaniZ10, SinnZV17,GulwaniJK09} and work on invariant inference through cost equations and ranking functions~\cite{BrockschmidtEFFG16,AlbertAGP08,AliasDFG10,Flores-MontoyaH14}.
% The techniques are based on
% type system~\cite{CicekBG0H17, RajaniG0021}, Hoare logic~\cite{CarbonneauxHS15}, abstract interpretation~\cite{GustafssonEL05, HumenbergerJK18},
% i
% or a combination of
% In general, these techniques give the approximated upper bound of the program's total running time or resource cost.
% However, they failed to consider the case where the cost -- the adaptivity-- could decrease when there isn't a dependency relation between variables.


\paragraph{Generalization in Adaptive Data Analysis}
Starting from the works by \citet{DworkFHPRR15} and \citet{HardtU14}, several works have designed methods that ensure generalization for adaptive data analyses~\cite{dwork2015reusable,dwork2015generalization,BassilyNSSSU16,UllmanSNSS18,FeldmanS17,jung2019new,SteinkeZ20,RogersRSSTW20}.
Several of these works drew inspiration from differential privacy, a notion of formal data privacy. By limiting the influence that an individual can have on the result of a data analysis, even in adaptive settings, differential privacy can also be used to limit the influence that a specific data sample can have on the statistical validity of a data analysis. This connection is actually in two directions, as discussed for example by \citet{YeomGFJ18}.
%
Considering this connection between generalization and privacy, it is not surprising that some of the works on programming language techniques for privacy-preserving data analysis are related to our work. 
Adaptive Fuzz~\cite{Winograd-CortHR17} is a programming framework for differential privacy that is designed around the concept of adaptivity. 
This framework is based on a typed functional language that distinguish between several forms of adaptive and non-adaptive composition theorem with the goal of achieving better upper bounds on the privacy cost. Adaptive Fuzz uses a type system and some partial evaluation to guarantee that the programs respect differential privacy. However, it does not include any technique to bound the number of rounds of adaptivity. 
\citet{lobo2021programming} propose a language for differential privacy where one can reason about the accuracy of programs in terms of confidence intervals on the error that the use of differential privacy can generate. These are akin to bounds on the generalization error. This language is based on a static analysis which however cannot handle adaptivity. 
%
The way we formalize the access to the data mediated by a mechanism is a reminiscence of how the interaction with an oracle is modeled in the verification of security properties. As an example, the recent works by \citet{BarbosaBGKS21} and \citet{AguirreBGGKS21} use different techniques to track the number of accesses to an oracle. However, reasoning about the number of accesses is easier than estimating the adaptivity of these calls, as we do instead here.

%\cite{SatoABGGH19}

% together with guarantees on their accuracy? 
% {The first important application of the differential privacy concept started from the work by \cite{DworkFHPRR15}}, which applied this concept into guaranteeing the generalization error of adaptive data analysis. 

% Previous works on reducing the risk of spurious scientific discoveries are under the assumption that a fixed collection of learning algorithms to be applied are selected non-adaptively before seeing the data. In contrast with them, they developed this work under the adaptive data analysis settings. They formalized the generalization error for adaptive data analysis and then presented their validation guarantees.
% Concretely, they proposed the famous transfer theorem.
% And based on this theorem, they proved high probabilistic bounds on the generalization error for $\epsilon$- and $(\epsilon,\delta)$-differentially private adaptive data analysis as well as adaptive analysis with statistic and non-statistic queries.
% These works connected the theory with the practice of data analysis, which in my perspective, is the most significant and interesting contribution of this paper. 
% %
% % At the end, they presented the application of applying concrete differentially private techniques into adaptive data analysis.

% {This extension from differential privacy to adaptive data analysis made significant progress in reducing the overfitting risks in practical works (i.e. the data analysis in adaptive setting). Further works on improving these probabilistic bounds, guaranteeing the generalization error such as \cite{dwork2015generalization}, \cite{BassilyNSSSU16}, \cite{dwork2015reusable}, \cite{jung2019new} etc. are all influenced by this work.}

% {Following all previous works on preserving the statistical validity in adaptive data analysis, \cite{smith2017information} made a survey.}
% This survey started from giving formal and clear introduction to the concept of adaptive data analysis.
%  % by giving formal definitions of the concepts and clear representations and notations. 
% Then, it summarized the probability bounds on the generalization error w.r.t. the true population when applying different mechanisms in numeric adaptive data analysis.
% The mechanisms includes split data with adaptivity, adding Gaussian noise with specific standard derivation, adopting differentially private algorithms etc.
% Next, he extended the scope onto the non-numeric adaptive data analysis and presented the corresponding probabilistic bounds based on the information measures. 

% {This survey was developed in an easy-to-understand way and included a thorough knowledge on state-of-the-art works on adaptive data analysis, which helped me to sort out the results from existing works and relations between them.}

% {Following the same line of work, \cite{jung2019new} gave a new analysis on the role of differential privacy in adaptive data analysis.}
% They gave a substantially better probability bounds on differential privacy's generalization guarantee based on a new proof technique of the transfer theorem (initially from \cite{dwork2015generalization}). 

% The key point in their proof technique is looking into the posterior distributions, which is an insightful new perspective on proving the generalization error bound. This new perspective also brought a better understanding in the specific reason of analysis overfitting and the role of differential privacy in adaptive data analysis. This new technique I think will be fruitful in future work.
% %  based on my personal interests on the posterior distribution analysis
% % Another very meaningful structural insight inspired by this paper is the role of differential privacy and sample accuracy. This is pointed to the end of the paper that the sample accuracy serves to guarantee that the reported answers are close to their posterior means and differential privacy serves to guarantee that the posterior means are close to their true answers.

% % There are also some interesting works on further improving the accuracy bound unresolved in this paper, such as replace the Markov-like dependence with a Chernoff-like dependence. I'm deeply interested in making contributions on it.

% % {Based on all the excellent theory works on guaranteeing the statistical validity of adaptive data analysis, I started to think from the perspective of the programming language.}
% {Existing works on adaptive data analysis are trying to improve the probabilistic bounds for generalization error in terms of the adaptive and non-adaptive queries numbers and size of the data sample on pen-and-paper proofs.
% However, we still cannot guarantee implementations of the corresponding algorithms adhere to this generalization error bounds.
% Given an arbitrary data analysis program, we are unable to tell its generalization error. So in my consideration, verifying the programs' generalization error would be a possible interesting research direction. Furthermore, since the size of the data sample can be determined by the input or the users, then the most interesting and challenging point would be figuring out the program's adaptive query numbers. 
% This motivated us to look into the verification of algorithms' adaptivity, in order to formally verifiy their generalization error.}

% \paragraph{Data analysis} There is a significant amount of work on programming for data analysis. Many popular platforms are based on the R language\cite{ihaka1996r, marcon2021orchestrating}. Jaql is a declarative scripting language for large-scale data analysis\cite{beyer2011jaql}. 


% Program Analysis in terms of dependency graph:


% \subsection{Dynamic Program Graph Analysis}

% \cite{sinha2001interprocedural}: 
% Support Interprocedural Control dependence analyzing, semantically.
% \\
% They identified the dependence information between the interactions of among procedures, specifically the control dependence between procedures.
% Their analysis support the relationship of control and data dependence to semantics dependence.

% % \cite{austin1992dynamic}: Dynamic Program Dependency Graph.
% % \\
% % They gave the dynamic analysis for the program's dependency, by producing 3 different kinds of graph, 
% % including the data flow graph, storage dependence and control dependence graph from program's execution traces. 
% % \\
% % Then, they constructing dynamic execution graphs by adopting the 3 graph, aims to expose the parallization of the programs

% \cite{hammer2006dynamic}: dynamic path conditions in dependence graphs.
% They adopting the dynamic information from program trace to the path condition in dependency graph. Then based on these information, 
% they present new approach combining dynamic slicing, which could reveal both dependences holding during program execution as well as why these dependences are holding. 
% Aims to have a finer and preciser analysis of the program.

% % \subsection{Utilization of the Dynamic Program Dependency}

% % \cite{nagar2018automated}: Utilize dependency graph for finding serializability violation. 
% % \\
% % Combine with the dependency graph of serialization and abstract execution, to statically finding bounded serializability violation. 
% % Then reduce the problem of serializability to satisfiability of a formula in FOL.
% % Also reason about unbounded executions.

% \subsection{Static Program Dependency}

% \cite{mastroeni2008data}: They propose ways of constructing different kinds of program slices, by choose different program dependency. For example, in either syntactic or semantics sense.
% This abstract dependency is based on properties rather than exact data.
% Aims to give finer and smaller program slice. 

% \subsection{Utilization Static Program Flow Graph}

% \cite{arlt2012lightweight}: Lightweight Static Analysis for GUI Testing. They give the relevant event graph based on black and white Box.
% To construct finer Event Sequence Graph, 
% they propose new approach to select relevant event sequences among the event sequences generated by black box.
% This new approach based on static analysis on bytecode of the applications, 
% giving a precisely defined dependency between a fixed number of events in event sequence.
% Then, they inferred a finer Event Dependency graph, aims to give a better lightweight static analysis on applications.




\section{Conclusion and future works}
We presented {\THESYSTEM}, a program analysis useful to provide an upper bound on the adaptivity of a data analysis under a specific data analysis model. This estimation can help data analysts to control the generalization errors of their analyses by choosing different algorithmic techniques based on the adaptivity. Besides, a key contribution of our works is the formalization of the notion of adaptivity for adaptive data analysis. 

In future work, we plan to address some of the limitations of {\THESYSTEM}. Our algorithm may over-estimate the adaptivity of a program, as shown in Section~\ref{sec:examples}, due to its path-insensitive nature. We plan in future work to explore the possibility of making {\THESYSTEM} path-sensitive. While we believe that in many concrete situations in data analysis requiring a concrete bound for loops is not a strong limitation, we also plan to explore how to add support for dynamic or unbounded loops. To extend our work in this direction we plan to use classical abstraction techniques, at the cost of a more imprecise estimation.  


%% Acknowledgments
\begin{acks}                            %% acks environment is optional
                                        %% contents suppressed with 'anonymous'
  %% Commands \grantsponsor{<sponsorID>}{<name>}{<url>} and
  %% \grantnum[<url>]{<sponsorID>}{<number>} should be used to
  %% acknowledge financial support and will be used by metadata
  %% extraction tools.
  This material is based upon work supported by the
  \grantsponsor{GS100000001}{National Science
    Foundation}{http://dx.doi.org/10.13039/100000001} under Grant
  No.~\grantnum{GS100000001}{nnnnnnn} and Grant
  No.~\grantnum{GS100000001}{mmmmmmm}.  Any opinions, findings, and
  conclusions or recommendations expressed in this material are those
  of the author and do not necessarily reflect the views of the
  National Science Foundation.
\end{acks}


% Bibliography
\bibliography{main}


%% Appendix
% \appendix
% \section{Appendix}

% Text of appendix \ldots

\end{document}
