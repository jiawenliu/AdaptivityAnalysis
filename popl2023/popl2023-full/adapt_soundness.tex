  \begin{thm}[Soundness of the \THESYSTEM]
    \label{thm:adaptfun_soundness}
  Given a program ${c}$, we have:
  %
  \[
    \forall \trace \in \mathcal{T} \st 
    \config{\progA(c), \trace} \earrow n \implies n \geq A(c)(\trace)
  \]
  \end{thm}
  %
Proof Summary:
\\
construct the
program-based graph $\progG({c}) = (\progV, \progE, \progW, \progF)$
\\
and trace-based graph $\traceG({c}) = (\traceV, \traceE, \traceW, \traceF)$ 
\\
1. prove the one-on-one mapping from $\progV$ to $\traceV$, in Lemma~\ref{lem:vertex_map};
\\
2. prove the total map from $\traceE$ to $\progE$, in Lemma~\ref{lem:edge_map};
\\
3. prove that the weight of every vertex in $\traceG$ is bounded by the weight of the same vertex in $\progG$, in 
Lemma~\ref{lem:weights_map};
\\
4. prove the one-on-one mapping from $\progF$ to $\traceF$, in Lemma~\ref{lem:queryvertex_map};
\\
5. show every walk in $\walks(\traceG)$ is bounded by a walk in $\walks(\progG)$ of the same $\qlen$.
\\
6. get the conclusion that $A(c)$ is bounded by 
the $\progA(c)$.
%
\begin{proof}
Given a program ${c}$, 
we construct its 
\\
program-based graph $\progG({c}) = (\progV, \progE, \progW, \progF)$
by Definition~\ref{def:prog_graph}
\\ and 
trace-based graph $\traceG({c}) = (\traceV, \traceE, \traceW, \traceF)$  by Definition~\ref{def:trace_graph}.
\\
The parameter $(c)$ for the components in the two graphs are omitted for concise.
\\
%
According to the Definition \ref{def:prog_adapt} and Definition~\ref{def:trace_adapt}, it is sufficient to show:
%
$$
\forall \trace \in \mathcal{T} \st
\config{\max\left\{ \qlen(k) \ \mid \  k\in \walks(\progG({c}))\right \}
, \trace } \earrow n \implies
n \geq
\max \big 
\{ \qlen(k)(\trace) \mid k \in \walks(\traceG(c)) \big \} 
$$
%
%
Then it is sufficient to show that:
\[
  \forall 
  k_t \in \walks(\traceG({c}),
  \exists k_p \in \walks(\progG({c})) 
  \st \forall \trace \in \mathcal{T} \st
  \qlen(k_p), \trace \earrow n
   \implies 
  n \geq \qlen(k_t(\trace))
  %  \leq \qlen(k_p)
\]
%
Let $k_t\in \walks(\traceG(c))$ be an arbitrary walk in $\traceG(c)$, 
and $\trace \in \mathcal{T}$ be arbitrary trace.
\\
Then, 
let $(e_{p1}, \cdots, e_{p(n-1)}) $ and
$(v_1, \cdots, v_n)$ be the edges and vertices sequence  for $k_t(\trace)$.
\\
%
% By Lemma~\ref{lem:vertex_map},
% we know 
% % for every $v_1$, there exists a unique image of $v_1$ in vertices set of $\progG(c)$.
% % Let $f_v : \mathcal{LV} \to \mathcal{LV}$ be this mapping function, we have:
% \[
%   \forall v_i \in (v_1, \cdots, v_n) \st v_i \in \progV
%   \]
%
% By 3 and 4, we know:
% $   
% \forall v_i \in k_t \st \traceF(v_i) = \progF(f_v(v_i))
% \land 
% \traceW(v_i) \leq \progW(f_v(v_i))$.
% \\
By Lemma~\ref{lem:vertex_map} and Lemma~\ref{lem:edge_map}, we know
%
\[
  \forall e_i \in k_t \st e_i = (v_i, v_{i + 1}) \implies
  \exists e_{pi} \st e_{pi} = (v_i, v_{i + 1}) \land e_{pi} \in \progE
  \]
  %
Then we construct a walk $k_p$ with an edge sequence $(e_{p1}, \cdots, e_{p(n-1)}) $ 
with a vertices sequence $(v_1, \cdots, v_n)$ where 
$e_{pi} = (v_i, v_{i + 1}) \in \progE$ for all $e_{pi} \in (e_{p1}, \cdots, e_{p(n-1)})$.
\\
Let $n \in \mathbb{N}$ such that 
% \[ 
$\config{\qlen(k_p), \trace} \earrow n$,
% \]
% to show $n \geq A(c)(\trace)$, 
then, it is sufficient to show
\[
  k_p \in \progG(c) \land n \geq \qlen(k_t)(\trace)
  \] 
  % following two, 
% \\
% (1) 
To show $k_p \in \progG(c)$, by Definition~\ref{def:finitewalk} for finite walk, 
we know
% \[
%   \forall (v_i, n_i) \in \traceW \st v_i \in (v_1, \cdots, v_n)
%   \implies \visit((v_1, \cdots, v_n), (v_i)) \leq n_i
% \]
\[
  \forall v_i \in (v_1, \cdots, v_n), (v_i, w_i) \in \traceW(c) 
  \st
  \visit((v_1, \cdots, v_n), (v_i)) \leq w_i(\trace)
\]
%
By Lemma~\ref{lem:weights_map}, we know for every 
\[
  \forall v_i \in (v_1, \cdots, v_n), (v_i, w_i) \in \progW(c), n_i \in \mathbb{N} 
  \st
  \config{w_i, \trace} \earrow n_i
  \implies
   w_i(\trace) \leq n_i ~ (\star)
\]
% $ v_i \in (v_1, \cdots, v_n)$,
% $\visit((v_1, \cdots, v_n), (v_i)) \leq \progW(v_i)$.
% \\
Then, by Definition~\ref{def:prog_finitewalk}, we know
the occurrence times for every $v_i \in (v_1, \cdots, v_n)$ 
is bound by the arithmetic expression $w_i$ where $(v_i, w_i) \in \progW(c)$.
\\
So we have $k_p \in \walks(\progG)$ proved.
\\
In order to show $ n \geq \qlen(k_t)(\trace) $, it is sufficient to show
\[
  \begin{array}{l}
  \forall v_i \in (v_1, \cdots, v_n),
  (v_i, w_i) \in \progW(c), (v_i, w'_i) \in \traceW(c), n_i \in \mathbb{N} 
  \st
  \config{w_i, \trace} \earrow n_i
  \\
  \implies
   \sum\limits_{\traceF(c)(v_i) = 1}
   w'_i(\trace) 
   \leq 
   \sum\limits_{\progF(c)(v_i) = 1}n_i 
  \end{array}
  \]
By Lemma~\ref{lem:queryvertex_map} and Definition~\ref{def:qlen}, we know for every $v_i$, $\traceF(c)(v_i) = \progF(c)(v_i) $ 
\\
Then by $(\star)$, we know $  \sum\limits_{\traceF(c)(v_i) = 1}
w'_i(\trace) 
\leq 
\sum\limits_{\progF(c)(v_i) = 1}n_i $.
\\
Then we have $ n \geq \qlen(k_t)(\trace) $ proved.
% $\qlen(k_t) = \qlen(k_p)$.
% Taking an arbitrary starting memory $m$ and an arbitrary underlying database $D$,
% we construct a trace-based graph $\traceG({c}, \text{D}, {m}) = (\vertxs, \edges)$ by the definition \ref{def:trace-based_graph}.
% %
% \\
% %
% Let $\midG({c},{m},\text{D}) = \{\midV, \midE, \midF\}$ be the intermediate graph by Definition~\ref{def:midgraph}.
% \\
% By Lemma~\ref{lem:bie_trace_to_mid}, we know:
% \[
%   \forall p, {m}, D, ~ s.t., ~ p \in \paths(\traceG({c}, \text{D}, {m}),
%   \exists p' \in \paths(\midG({c},{m},\text{D})) \land 
%   \len(p) = \len_q(p')
% \]
% %
% Then it is sufficient to show that:
% %
% \[
%   \forall p, {m}, D, ~ s.t., ~ p \in \paths(\midG({c}, \text{D}, {m}),
%   \exists k \in \walks(\progG({c})) \land 
%   \qlen(p) \leq \qlen(k)
% \]
% %
% We prove a stronger statement instead:
% \[
%   \forall p, {m}, D, ~ s.t., ~ p \in \paths(\midG({c}, \text{D}, {m}),
%   \exists k \in \walks(\progG({c})) \land 
%   \qlen(p) = \qlen(k) 
% \]
%
%
% By Lemma~\ref{lem:sujv_mid_to_prog}, let $g$ be the surjective function $g: \progV \to \midV$ s.t.:
% %
% $$
% \forall \av \in \midV. ~ \progF(f(\av)) = \midF(\av) 
% \land |\kw{image}(f(\av))| \leq W(f(\av)).
% $$
%
%
% \item(1) $\len(p_{\av_1 \to \av_2}) = \len(k_{f(\av_1) \to f(\av_2)})$
% %
% \item(2) $\forall \av \in p_{\av_1 \to \av_2}. ~ f(\av) \in k_{f(\av_1) \to f(\av_2)}$
% %
% \item(3) $\forall \av \in p_{\av_1 \to \av_2}. ~ 
% \kw{image}(f(\av)) \cap {p_{\av_1 \to \av_2}}| = \# \{f(\av) \mid f(\av) \in k_{f(\av_1) \to f(\av_2)}\}$
\\
This theorem is proved.
\end{proof}
The following are the four lemmas used in the proof of Theorem~\ref{thm:adaptfun_soundness} above,
showing the correspondence properties between the program based graph and trace based graph.
	\begin{lem}[One-on-One Mapping of vertices from $\traceG$ to $\progG$]
	\label{lem:vertex_map}
	Given a program $c$ with its
	program-based graph $\progG({c}) = (\progV, \progE, \progW, \progF)$
	and 
	trace-based graph $\traceG({c}) = (\traceV, \traceE, \traceW, \traceF)$,
	then for every $v \in \mathcal{VAR} \times \mathbb{N}$,
	$v \in \progV$ if and only if $v \in \traceG$.
	\[
	\begin{array}{l}
	\forall c \in \cdom , v \in \mathcal{VAR} \times \mathbb{N} \st 
	\progG({c}) = (\progV, \progE, \progW, \progF)
	\land 
	\traceG({c}) = (\traceV, \traceE, \traceW, \traceF)
	\\ \quad
	\implies
	v \in \progV \Longleftrightarrow v \in \traceV
	\end{array}
	\]
	%
	\end{lem}
\begin{proof}
Proof Summary: Proving by Definition~\ref{def:prog_graph} and Definition~\ref{def:trace_graph}.
\\
Taking arbitrary program $c$,
by Definition~\ref{def:prog_graph} and Definition~\ref{def:trace_graph}, 
we have   
\\
its program-based graph $\progG({c}) = (\progV, \progE, \progW, \progF)$ 
\\
and 
trace-based graph $\traceG({c}) = (\traceV, \traceE, \traceW, \traceF)$.
\\
By the two definitions, we also know 
$\traceV  = \lvar_c$ and $\progV = \lvar_c$.
\\
Then we know $\traceV  = \progV$, i.e., 
for arbitrary $v \in \mathcal{VAR} \times \mathbb{N}$, $v \in \progV \Longleftrightarrow v \in \traceV$.
\end{proof}
%
	\begin{lem}[Mapping from Egdes of $\traceG$ to $\progG$]
	\label{lem:edge_map}
	Given a program $c$ with its
	program-based graph $\progG({c}) = (\progV, \progE, \progW, \progF)$
	and 
	trace-based graph $\traceG({c}) = (\traceV, \traceE, \traceW, \traceF)$,
	then for every $e = (v_1, v_2) \in \traceE$, there exists an edge 
	$e' = (v_1', v_2') \in \progE$ with 
	$v_1 = v_1' \land v_2 = v_2'$.
	\[
	\begin{array}{l}
	\forall c \in \cdom \st
	 \progG({c}) = (\progV, \progE, \progW, \progF)
	\land 
	\traceG({c}) = (\traceV, \traceE, \traceW, \traceF)
	\\ \quad
	\implies
	\forall e = (v_1, v_2) \in \traceE
	\st 
	% \exists e' \in \mathcal{VAR} \times \mathbb{N} \times (\mathcal{VAR} \times \mathbb{N})\st 
	\exists e' \in \progE \st e' = (v_1, v_2)
	\end{array}
	\]
	\end{lem}
\begin{proof}
Proof Summary: Proving by Lemma~\ref{lem:vertex_map}, Lemma~\ref{thm:flowsto_soundness} Definition~\ref{def:prog_graph} and Definition~\ref{def:trace_graph}
\\
Taking arbitrary program $c$,
by Definition~\ref{def:prog_graph} and Definition~\ref{def:trace_graph}, 
we have   
\\
its program-based graph $\progG({c}) = (\progV, \progE, \progW, \progF)$ 
\\
and 
trace-based graph $\traceG({c}) = (\traceV, \traceE, \traceW, \traceF)$.
\\
Taking arbitrary edge $e = (x^i, y^j) \in \traceE$, it is sufficient to show $(x^i, y^j) \in \progE$.
\\
By Lemma~\ref{lem:vertex_map}, we know $x^i, y^j \in \progV$.
\\
By definition of $\traceE$, we know $\vardep(x^i, y^j, c)$.
\\
By Theorem~\ref{thm:flowsto_soundness}, we know $ \exists n \in \mathbb{N}, z_1^{r_1}, \cdots, z_n^{r_n} \in \lvar_{{c}} \st 
n \geq 0 \land
\flowsto(x^i,  z_1^{r_1}, c) 
\land \cdots \land \flowsto(z_n^{r_n}, y^j, c) $.
\\
Then by definition of $\progE$, we know $(x^i, y^j) \in \progE$. This Lemma is proved.
\end{proof}
%
\begin{lem}[Weights are bounded]
	\label{lem:weights_map}
	Given a program $c$ with its
	program-based graph $\progG({c}) = (\progV, \progE, \progW, \progF)$
	and 
	trace-based graph $\traceG({c}) = (\traceV, \traceE, \traceW, \traceF)$,
	for every $v \in \traceV$, there is $v \in \progV$ and $\traceW(v) \leq \progW(v)$.
%
	% then for every $(v, n) \in \traceW$, 
	% there exists $(v, n') \in \progW$ with $n \leq n'$.
	% \[
	% \begin{array}{l}
	% \forall c \in \cdom , (v, n) \in \mathcal{VAR} \times \mathbb{N} \times \mathbb{N}
	%  \st 
	%  \\ \quad
	%  \progG({c}) = (\progV, \progE, \progW, \progF)
	% \implies 
	% \traceG({c}) = (\traceV, \traceE, \traceW, \traceF)
	% \\ \quad
	% \implies
	% (v, n) \in \traceW
	% \implies 
	% \exists n' \in \mathbb{N}\st 
	% (v, n') \in \progW  \land n \leq n'
	% \end{array}
	% \]
%
\[
	\begin{array}{l}
	\forall c \in \cdom 
	% , (v, n) \in \mathcal{VAR} \times \mathbb{N} \times \mathbb{N}
	 \st 
	%  \\ \quad
	 \progG({c}) = (\progV, \progE, \progW, \progF)
	\land 
	\traceG({c}) = (\traceV, \traceE, \traceW, \traceF)
	\\ \quad
	\implies
	\forall v \in \traceV \st 
	v \in \progV \land
	\traceW(v) \leq \progW(v)
	\end{array}
	\]
	\[
		\begin{array}{l}
			\forall c \in \cdom 
			% , (v, n) \in \mathcal{VAR} \times \mathbb{N} \times \mathbb{N}
			 \st 
			%  \\ \quad
			 \progG({c}) = (\progV, \progE, \progW, \progF)
			\land 
			\traceG({c}) = (\traceV, \traceE, \traceW, \traceF)
			\\ \quad
			\implies
			% \forall v \in \traceV \st 
			% v \in \progV \land
			% \traceW(v) \leq \progW(v)
			\forall (x^l, w_{t}) \in \traceW,
			(x^l, w_{p}) \in \progW, \vtrace, \trace' \in \mathcal{T}, v \in \mathbb{N} \st
			% \lvar_c \st 
			%  \vcounter(\vtrace') l ~ \middle\vert~
			% \forall \vtrace \in \mathcal{T} \st 
			% \config{{c}, \trace} \to^{*} \config{\eskip, \trace\tracecat\vtrace'} 
			% \land 
			\config{w_{p}, \trace} \earrow v
			\implies
			% \right\} 
			w_{t}(\trace) \leq v
		\end{array}
		% 		\forall (x^l, w_{t}) \in \traceW,
		% (x^l, w_{p}) \in \progW, \vtrace, \trace' \in \mathcal{T} \st
		% % \lvar_c \st 
		% %  \vcounter(\vtrace') l ~ \middle\vert~
		% % \forall \vtrace \in \mathcal{T} \st 
		% \config{{c}, \trace} \to^{*} \config{\eskip, \trace\tracecat\vtrace'} 
		% \land 
		% \config{w_{p}, \trace} \earrow v
		% \implies
		% % \right\} 
		% \leq 
		% w_{t}(\trace) \leq v
		\]
	\end{lem}
\begin{proof}
Proof Summary: Proving by Definition~\ref{def:prog_graph}, Definition~\ref{def:trace_graph} and relying on the soundness of Reachability Bound 
Analysis.
\\
Taking arbitrary program $c$,
by Definition~\ref{def:prog_graph} and Definition~\ref{def:trace_graph}, 
we have   
\\
its program-based graph $\progG({c}) = (\progV, \progE, \progW, \progF)$ 
\\
and 
trace-based graph $\traceG({c}) = (\traceV, \traceE, \traceW, \traceF)$.
\\
Taking arbitrary 
% $v \in \traceV$, by Lemma~\ref{lem:vertex_map}, we know $v \in \progV$.
$(x^l, w_{t}) \in \traceW, (x^l, w_{p}) \in \progW, \vtrace, \trace' \in \mathcal{T}$, satisfying:
\\
$\config{{c}, \trace} \to^{*} \config{\eskip, \trace\tracecat\vtrace'} 
\land 
\config{w_{p}, \trace} \earrow v$
\\
By soundness of reachability bound analysis in Theorem~\ref{thm:addweight_soundness}, we know 
% $\traceW(v) \leq \progW(v)$. This lemma is proved.
$\vcounter(\vtrace', l) \leq v$
\\
By definition~\ref{def:trace_graph}, we know $w_t(\trace) = \vcounter(\vtrace', l)$,
then we have $w_t(\trace) \leq v$ and this is proved.
\end{proof}
%
\begin{lem}[One-on-One Mapping for Query Vertices]
	\label{lem:queryvertex_map}
	Given a program $c$ with its
	program-based graph $\progG({c}) = (\progV, \progE, \progW, \progF)$
	and 
	trace-based graph $\traceG({c}) = (\traceV, \traceE, \traceW, \traceF)$,
	then for every $(x^i, n) \in \mathcal{VAR} \times \mathbb{N}  \times \{0, 1\} $,
	 $(x^i, n) \in \traceF$ if and only if $ (x^i, n) \in \progF$.
	\[
	\begin{array}{l}
	\forall c \in \cdom , (x^i, n) \in \mathcal{VAR} \times \mathbb{N}  \times \{0, 1\} 
	 \st 
	 \\ \quad
	 \progG({c}) = (\progV, \progE, \progW, \progF)
	\land 
	\traceG({c}) = (\traceV, \traceE, \traceW, \traceF)
	\\ \quad
	\implies
	(x^i, n) \in \traceF \Longleftrightarrow  (x^i, n) \in \progF
	\end{array}
	\]
	\end{lem}
\begin{subproof}
Proving by Definition~\ref{def:prog_graph}, Definition~\ref{def:trace_graph}.
\\
Taking arbitrary program $c$,
by Definition~\ref{def:prog_graph} and Definition~\ref{def:trace_graph}, 
we have   
\\
its program-based graph $\progG({c}) = (\progV, \progE, \progW, \progF)$ 
\\
and 
trace-based graph $\traceG({c}) = (\traceV, \traceE, \traceW, \traceF)$.
\\
By the two definitions, we also know $\traceF  = \progF$, 
i.e., 
for arbitrary $ (x^i, n) \in \mathcal{VAR} \times \mathbb{N}  \times \{0, 1\} $,
 $(x^i, n) \in \traceF \Longleftrightarrow  (x^i, n) \in \progF$.
 \\
 This lemma is proved.
\end{subproof}
