\begin{thm}[Conditional Completeness of $\pathsearch$]
  \label{thm:adaptalg_pcomplete}
  For every program $c$, given its \emph{Program-Based Dependency Graph} $\progG$,
  if $\progG(c)$ is acyclic directed, then
   $$\pathsearch(\progG) = \progA(\progG).$$
\end{thm}
proof Summary:
\\
1. for every two vertices $x, y$ with a walk $k_{x,y}$ from $x$ to $y$ on $\progG$, 
\\
2 since $\progG$ is acyclic directed,
then this walk corresponds to a path $p_{x,y}$
where every vertex is visited exactly once.
\\
3. the query length is sum of the query annotation.
\\
From Algorithm~\ref{alg:adaptscc}, every vertex is a SCC with only one vertex and zeor edge,
its adaptivity is exactly its query annotation.
\\
=> $\qlen(k_{x,y}) = \sum\limits_{v_i \in ssc_i}\kw{Adapt[scc_i]}$
\\
This is proved.
\begin{proof}
  Taking arbitrary program $c \in \cdom$, let $\progG(c) = (\progV, \progE, \progW, \progF)$ be its 
  program based dependency graph.
  \\
  Let the walk $k_{max} \in \walks{(\progG(c))}$ be the finite walk with the longest query length, and the vertices sequence
  $(s_1, \cdots, s_n)$, 
  it is sufficient to show:
  \[
    \qlen(k_{max}) = \len(s | s \in (s_1, \cdots, s_n) \land \progF(s) = 1) = \pathsearch(\progG(c))
  \]
  In order to show the completeness, it is sufficient to show two following items,
  \\
  1. By line: 15, $\pathsearch(\progG(c))$ can find a path $p_{max}$ such that $\kw{adapt_{p_{max}}} = \qlen(k_{max})$ 
  \\
  2. This $p_{x,y}$ is the longest weighted path found by $\pathsearch(\progG(c))$, and $\kw{adapt_{p_{max}}}$ is returned as the final output.
  \\
  By the property of ACG, we know
  every $s_i \in (s_1, \cdots, s_n)$ shows up exactly once. 
  Then we know this walk is a path and
  \[
    \qlen(k_{max}) = \sum_{s_i \in (s_1, \cdots, s_n)} \progF(s_i)
  \]
  %
  By line: 13, through searching on all the vertices connected on $\progG(c)$ from the starting node $s_i$,
  we know that $\pathsearch(\progG(c))$ finds this path $p_{max} = (s_1, \cdots, s_n)$.
  \\
  Then, it is sufficient to show 
  $$
  \kw{adapt_{p_{max}}} = \sum_{s_i \in (s_1, \cdots, s_n)} \progF(s_i).
  $$
  By line: 15, let $\sccgraph_1, \cdots, \sccgraph_m$ be all the SCC, where each vertex in 
  $(s_1, \cdots, s_n) $ belongs to, it is sufficient to show:
  \[
    \sum_{\sccgraph_i \in (\sccgraph_1, \cdots, \sccgraph_m)} \kw{adapt_{scc}[\sccgraph_i]} 
    = \sum_{s_i \in (s_1, \cdots, s_n)} \progF(s_i).
    \]
  \\
  By line:3 in Algorithm~\ref{alg:adpt_alg}, 
  % let $\kw{\sccgraph_1}, \cdots, \kw{\sccgraph_m}$ be the strong connected components on $\progG$ with $0 \leq n \leq |\vertxs|$,
  let $\kw{\sccgraph_i} = (\vertxs_i, \edges_i, \weights_i, \qflag_i)$ for $\sccgraph_i \in (\sccgraph_1, \cdots, \sccgraph_m)$
  be the SCC found by the standard Algorithm.,
  \\
  % By line:5-6 in Algorithm~\ref{alg:adpt_alg}, let $\kw{adapt_{scc}[\sccgraph_i]}$ be the result of $\pathsearch_{\kw{scc}}(\sccgraph_i)$ for each $\sccgraph_i$.
  % \\
  Then, by the property of ACG, we know every $\sccgraph_i$ is a single vertex $v_i$ without edge and $\qflag_i$ is the 
  query annotation of $v_i$, i.e., $\vertxs_i = \{s_i\}$ and $\qflag_i = \{(s_i, \progF(s_i))\}$.
  \\
  So we know $n = m$.
  \\
  Also by Algorithm~\ref{alg:adaptscc} line: 4-5, we know $\kw{adapt_{scc}[\sccgraph_i]} = \progF(s_i)$.
  \\
  Then we can conclude:
  \[
    \sum_{\sccgraph_i \in (\sccgraph_1, \cdots, \sccgraph_m)} \kw{adapt_{scc}[\sccgraph_i]} = 
    \sum_{\sccgraph_i \in (\sccgraph_1, \cdots, \sccgraph_n)} \progF(s_i)
    = \sum_{s_i \in (s_1, \cdots, s_n)} \progF(s_i).
  \]
%   Then by line: 15 in Algorithm~\ref{alg:adpt_alg}, we know 
%   \\
% Let $\sccgraph_x, \sccgraph_1, \cdots, \sccgraph_m, \sccgraph_y, 0 \leq m$ be all the SCC this walk pass by, where each vertex in 
% $(x, s_1, \cdots, s_n, y) $ belongs to a single SCC number, then we have:
% \\
So we have (1). "the existence" proved.
In order to show $p_{max}$ is the longest path found and $\kw{adapt_{p_{max}}}$ is returned by $\pathsearch(\progG(c))$,
% \[ \qlen(k_i) \leq \max\{\qlen(k_i) | k_i \in \walks(\sccgraph_i)\} \leq \pathsearch_{scc}(\sccgraph_i) = \kw{adapt_{scc}[\sccgraph_i]} \]
by line: 18, it is sufficient to show $\kw{adapt = adapt_{p_{max}}}$.
\\
It is sufficient to show a contradiction if $\kw{adapt \neq adapt_{p_{max}}}$ in following two cases:
\caseL{{$\kw{adapt < adapt_{p_{max}}}$}}, it is easy to show the contradiction by line: 18 where 
$\kw{adapt = \max(adapt, adapt_{p_{max}}) \geq adapt_{p_{max}} }$.
%
\caseL{$\kw{adapt > adapt_{p_{max}}}$},
Let $p_{max}'$ be the path such that $\kw{adapt = adapt_{p_{max}'} > adapt_{p_{max}}}$ with vertices sequence $(s_1', \cdots, s_n')$.
Then we know $p_{max}'$ corresponds to a walk $k_{max}'$ with the same vertices sequence.
\\
Then by the same proof above, we know $\qlen(k_{max}') = adapt_{p_{max}'}$
and $\qlen(k_{max}') > \qlen(k_{max})$.
Then there is a contradiction that $k_{max}'$ is the walk with the longest query length rather than $k_{max} $.
% \[\]%
% Then we have:
% \[ 
%   \qlen(k_{x,y}) = \qlen(k_x) + \qlen(k_1) + \cdots + \qlen(k_y) \leq 
%   \kw{adapt_{scc}[\sccgraph_1]} + \kw{adapt_{scc}[\sccgraph_1]}  + \cdots + \kw{adapt_{scc}[\sccgraph_y]}
%   \]
% , where $\kw{adapt}$ is the output of $\pathsearch(\progG)$.
% This case is proved.
Then, we have (2) proved.
\end{proof}