\begin{subproof}[Proof of the Basecase: Case 2]
%
\label{pf:eventdep_base_ctl}
We have the following by the definition $\eventdep(\event_1, \event_2, [\event_1; \event_2], c, D)$ of case 2:
\begin{equation}
  \label{eq:eventdep_def_base_ctl}
  \begin{array}{ll}   
    & \exists \vtrace_0,
    \vtrace_1, \vtrace', \vtrace_3, \vtrace_3' \in \mathcal{T},\event_1' \in \eventset^{\asn}, {c}_1, {c}_2  \in \cdom, 
    \event_b \in \eventset^{\test}
    \st
    \\ 
   &   \qquad    \diff(\event_1, \event_1') 
\land
   \Big(
  \config{{c}, \vtrace_0} \rightarrow^{*} 
      \config{{c}_1, \vtrace_1 \tracecat [\event_1]}  \rightarrow^{*} 
      \config{c_2,  \trace_1 \tracecat [\event_1;\event_b] \tracecat  \trace_3} 
    \\   
   & \qquad \bigwedge 
    \config{{c}_1, \vtrace_1 \tracecat [\event_1']}  \rightarrow^{*} 
    \config{c_2,  \vtrace_1 \tracecat [\event_1'] \tracecat \trace' \tracecat [(\neg \event_b)] \tracecat \trace_3'} 
    \\
    & \qquad \bigwedge  \tlabel_{\trace_3} \cap \tlabel_{\trace_3'} = \emptyset
     \land \vcounter(\trace') ~  \pi_2(\event_b) = \vcounter(\trace) ~  \pi_2(\event_b)
      \land \event_2 \eventin \trace_3
    \land \event_2 \not\eventin \trace_3'
   \Big)
 \end{array}
  \end{equation}
  %
%
Let $\vtrace_0,
\vtrace_1, \vtrace', \vtrace_3, \vtrace_3' \in \mathcal{T}, 
\event_2' \in \eventset, \event_1' \in \eventset^{\asn}, \event_b, {c}_1, {c}_2$ be the traces, events and commands satisfying the executions,
by Inversion Lemma~\ref{lem:inv_event} on 
$\event_1$, $\event_2$, and $\event_b$,
we have the following instance of the first execution in Eq.~\ref{eq:eventdep_def_base_ctl},
 %
%
% Let $\event_{ih} = (b, l_b, n_b, v_b)$, by Eq.~\ref{eq:ctldep_inv1} and {Inversion Lemma~\ref{lem:inv_test}}, we have:
\begin{equation}
\label{eq:ctldep_inv1}
  \begin{array}{l}   
\config{{c}, \vtrace_0} \rightarrow^{*} 
\config{[\assign{{x}_1}{\expr_1 / \query(\qexpr_1)}]^{l_1} ; {c}_1, \vtrace_1}  
\rightarrow^{assn/query}
 \config{c_1, \vtrace_1 \tracecat [\event_1]} 
 \\
  \qquad \rightarrow^{*} 
  \config{\eif ([b]^{l_b}, c_t, c_f) / \ewhile [b]^{l_b} \edo c_w;{c}_3', 
  \vtrace_1 \tracecat [\event_1]} 
  \\
  \qquad 
   \rightarrow^{\rname{if-b / while-b}} 
  \config{(c_t;c_3' / c_f;c_3') /(c_3' / c_w; \ewhile [b]^{l_b} \edo c_w;{c}_3'), 
  \trace_1 \tracecat [\event_1;\event_b]} 
  \\
  \qquad   \rightarrow^{*} 
  \config{c_3, 
  \trace_1 \tracecat [\event_1;\event_b] \tracecat  \trace_3}
  % 
\end{array}
\end{equation}
%  %
% \begin{equation}
% \label{eq:ctldep_inv1}
%   \begin{array}{l}   
% \config{{c}, \vtrace_0} \rightarrow^{*} 
% \config{[\assign{x_1}{\expr_1 / \query(\qexpr_1)}]^{\pi_2(\event_1)} ; {c}_1
% \footnote{
% $\assign{x}{\expr / \query(\qexpr)}$ denotes variable $x$ is assigned by either an expression $\expr$ or query $\query(\qexpr)$
% }, 
% \vtrace_1}  
% \rightarrow^\rname{assn/query}
%  \config{c_1, \vtrace_1 \tracecat [\event_1]} \\
%   \qquad \rightarrow^{*} 
%   \config{[\assign{{x}_2}{\expr_2 / \query(\qexpr_2)}]^{l_2};{c}_2, 
%   \vtrace_1 \tracecat [\event_1]} 
%   \rightarrow^\rname{assn/query} 
%   \config{{c}_2,  \vtrace_1 \tracecat [\event_1; \event_2]} 
%   % 
% \end{array}
% \end{equation}
% %
% % \wqside{Some typo in equation 4, but I can follow:-)}
% % \jl{thanks}
, where $x_1 = \pi_1(\event_1)$, $l_1 = {\pi_2(\event_1)}$, 
% $x_2 = \pi_1(\event_2)$, $l_2 = \pi_2(\event_2)$, 
and $\eif ([b]^{l_b}, c_t, c_f) / \ewhile [b]^{l_b} \edo c_w$ 
is the conditional command of the assignment commands 
associated to the $\event_b$ from Inversion Lemma~\ref{lem:inv_event} of testing event.
\\
%
By the command label consistency,
we also have the instance of second execution in Eq.~\ref{eq:eventdep_def_base_ctl} as follows:
% \[
% \config{{c}_1, \vtrace_1 \tracecat [\event_1']}  \rightarrow^{*} 
%   \config{{c}_2',  \vtrace_1 \tracecat [\event_1']\cdot \vtrace_2' \tracecat [\event_2'] } 
%   \], 
% we know there exists $\expr_2'$ or $\qexpr_2'$ and following execution instance,
%  \[
%   \begin{array}{l}   
%   \config{c_1, \vtrace_1 \tracecat [\event_1']} 
%   \rightarrow^{*} 
%   \config{[\assign{{x}_2'}{\expr_2' / \query(\qexpr_2')}]^{l_2'} ; {c}_2', \vtrace_1 \tracecat [\event_1']\tracecat \vtrace'} 
%   \rightarrow^\rname{assn/query} 
%   \config{{c}_2',   \vtrace_1 \tracecat [\event_1'] \tracecat \vtrace' \tracecat [\event_2']} 
%   % 
% \end{array}
%  \]
%  , where  $x_2' = \pi_1(\event_2')$ and $l_2' = \pi_2(\event_2')$.
% %
% Unfolding $\diff(\event_2,\event_2')$, we have:
% \[
%   x_2 = x_2' \land l_2 = l_2' 
% \] 
% %
% Since each command in $c$ has a unique label, we have $\expr_2' = \expr_2$, $\qexpr_2 = \qexpr_2'$, and following execution instance:
\begin{equation}
\label{eq:ctldep_inv2}
\begin{array}{l}   
  \config{{c}, \vtrace_0} \rightarrow^{*} 
  \config{[\assign{{x}_1}{\expr_1 / \query(\qexpr_1)}]^{l_1} ; {c}_1, \vtrace_1}  
  \rightarrow^{assn/query}
   \config{c_1, \vtrace_1 \tracecat [\event_1]} 
   \\
    \qquad \rightarrow^{*} 
    \config{\eif ([b]^{l_b}, c_t, c_f) / \ewhile [b]^{l_b} \edo c_w;{c}_3', 
    \vtrace_1 \tracecat [\event_1] \tracecat \trace'} 
    \\
    \qquad 
     \rightarrow^{\rname{if-b / while-b}} 
    \config{(c_f;c_3' / c_t;c_3') /(c_w; \ewhile [b]^{l_b} \edo c_w;{c}_3' / c_3'), 
    \trace_1 \tracecat [\event_1]  \tracecat \trace' \tracecat [\neg \event_b]} 
    \\
    \qquad   \rightarrow^{*} 
    \config{c_3, 
    \trace_1 \tracecat [\event_1]  \tracecat \trace' \tracecat [\neg \event_b] \tracecat  \trace_3'}
    % 
  \end{array}
\end{equation}
%
From Eq.~\ref{eq:eventdep_def_base_ctl}, we also have
  $\vcounter(\vtrace') l_b = \vcounter( [] ) l_b = 0$.
\\
%
%
By the same proof steps from case 1 in Subproof~\ref{pf:eventdep_base_val}, we have
\[
  x_1 \in VAR(b)  \land x_1^{l_1} \in \live(l_b, c)
\]
%
By Lemma~\ref{lem:ctldep_inv}, we also know:
\[
  \forall z \in VAR(\pi_1(\event_b)) \st \exists i \in \mathbb{N} \st
\flowsto(z^i, \pi_1(\event)^{\pi_2(\event)}, c)
\]
%
Then by $\flowsto$ definition, we have $\flowsto(x_1^{l_1}, {x}_2^{l_2}, c)$
%
i.e.,
%
\[
\flowsto(\pi_1(\event_1)^{\pi_2(\event_1)}, \pi_1(\event_2)^{\pi_2(\event_2)}, c)
 \]
%
This case is proved.
\end{subproof}