\documentclass[a4paper,11pt]{article}

\usepackage{mathpartir}
\usepackage{amsmath,amsthm,amsfonts}
\usepackage{ amssymb }
\usepackage{color}
\usepackage{algorithm}
\usepackage{algorithmic}
\usepackage{microtype}


\newcommand{\defeq}{\mathrel{\doteq}}

\newcommand{\lzero}{0}

\newcommand{\kw}[1]{\mathtt{#1}}

\newcommand{\expr}{e}
\newcommand{\vall}{w}
\newcommand{\valr}{v}
\newcommand{\eif}{\kw{if}}
\newcommand{\eapp}{\;}
\newcommand{\eprojl}{\kw{fst}}
\newcommand{\eprojr}{\kw{snd}}
%\newcommand{\eprov}[1]{\eta_{#1}}
\newcommand{\etrue}{\kw{true}}
\newcommand{\efalse}{\kw{false}}
\newcommand{\econst}{c}
\newcommand{\eop}{\delta}
\newcommand{\efix}{\mathop{\kw{fix}}}
%\newcommand{\labelA}{\ell}

\newcommand{\tr}{T}
\newcommand{\trift}{\eif^{\kw{t}}}
\newcommand{\triff}{\eif^{\kw{f}}}
\newcommand{\trprojl}{\eprojl}
\newcommand{\trprojr}{\eprojr}
\newcommand{\trtrue}{\etrue}
\newcommand{\trfalse}{\efalse}
\newcommand{\trconst}{\econst}
\newcommand{\trop}{\eop}
\newcommand{\trfix}{\efix}
\newcommand{\trapp}[5]{#1 \; #2 \mathrel{\triangleright} {\efix #3(#4).#5}}

\newcommand{\adap}{\kw{adap}}
\newcommand{\ddep}[1]{\kw{depth}_{#1}}
\newcommand{\nat}{\mathbb{N}}
\newcommand{\natb}{\nat_{\bot}}
\newcommand{\natbi}{\natb^\infty}
\newcommand{\nnatA}{n}
\newcommand{\nnatB}{m}
\newcommand{\nnatbA}{s}
\newcommand{\nnatbB}{t}
\newcommand{\nnatbiA}{q}
\newcommand{\nnatbiB}{r}

\newcommand{\type}{\tau}
\newcommand{\tbase}{\kw{b}}
\newcommand{\tbool}{\kw{bool}}
\newcommand{\tarr}[5]{#1; #3 \xrightarrow{#4; \, #5} #2}
\newcommand{\env}{\theta}

\newcommand{\bigstep}{\mathrel{\Downarrow}}

\newcommand{\dmap}{\rho}
\newcommand{\dmapb}{\bot_\dmap}
\newcommand{\supp}{\kw{supp}}
\newcommand{\dom}{\kw{dom}}

\newcommand{\tvdash}[1]{\vdash_{#1}}

%Packages
\usepackage[T1]{fontenc}
\usepackage{fourier} 
\usepackage[english]{babel} 
\usepackage{amsmath,amsfonts} 
\usepackage{amsthm} 
\usepackage{color}   %May be necessary if you want to color links
\usepackage{hyperref}
\usepackage{lscape}
\usepackage{geometry}
\usepackage{amsmath}
\usepackage{algorithm}
\usepackage{algorithmic}
\usepackage{amssymb}
\usepackage{amsfonts}
\usepackage{times}
\usepackage{bm}
\usepackage{ stmaryrd }
\SetSymbolFont{stmry}{bold}{U}{stmry}{m}{n}

\usepackage{ amssymb }
\usepackage{ textcomp }
\usepackage[normalem]{ulem}
% For derivation rules
\usepackage{mathpartir}
\usepackage{color}
\usepackage{a4wide}
\usepackage{caption}
\usepackage{subcaption}
\usepackage{mathpartir}
\usepackage{amsmath,amsfonts}
\usepackage{ amssymb }
\usepackage{color}
\usepackage{algorithm}
\usepackage{algorithmic}
\usepackage{microtype}
\usepackage{eucal}
\usepackage{url}
\usepackage{xspace}
\usepackage{array}
\usepackage{listings}

\usepackage{tikz}
\usetikzlibrary{shapes.geometric}
\usetikzlibrary{arrows.meta,arrows}
\usetikzlibrary{decorations.text}
% % % % 


\usepackage{multirow}


%%%%%%%%%%%%%%%%%%%%%%%%%%%%%%%%%%%%%%%%%%%%%%%%%%%%%Packages And Definitions For Listing the Code%%%%%%%%%%%%%%%%%%%%%%%%%%%%%%%%%%%%%%%%%%%%%%%%%%%%%%%%%%%%%%%%%%%%%%%%
\usepackage{listings}
\usepackage{xcolor}

\definecolor{codegreen}{rgb}{0,0.6,0}
\definecolor{codegray}{rgb}{0.5,0.5,0.5}
\definecolor{codepurple}{rgb}{0.58,0,0.82}
\definecolor{backcolour}{rgb}{0.95,0.95,0.92}

\lstdefinestyle{mystyle}{
    backgroundcolor=\color{backcolour},   
    commentstyle=\color{codegreen},
    keywordstyle=\color{magenta},
    numberstyle=\tiny\color{codegray},
    stringstyle=\color{codepurple},
    basicstyle=\ttfamily\footnotesize,
    breakatwhitespace=false,         
    breaklines=true,                 
    captionpos=b,                    
    keepspaces=true,                 
    numbers=left,                    
    numbersep=5pt,                  
    showspaces=false,                
    showstringspaces=false,
    showtabs=false,                  
    tabsize=2
}

\lstset{style=mystyle}
\newcommand{\aexpr}{a}
\newcommand{\bexpr}{b}
\newcommand{\cmd}{c}
\newcommand{\node}{N}
\newcommand{\assign}[2]{ \mathrel{ #1  \leftarrow #2 } }
\newcommand{\fassign}[3]{ \mathrel{ #1  \leftarrow^{#3}  \delta^{#3}(
    #2 ) } }
\newcommand{\impif}[3]{\mathrel{\eif \eapp #1\eapp \ethen \eapp #2 \eapp
    \eelse \eapp #3 }}
\newcommand{\impwhile}[2]{\mathrel{ \kw{while} (#1) \eapp #2 } }
\newcommand{\labl}{l}

\let\originalleft\left
\let\originalright\right
\renewcommand{\left}{\mathopen{}\mathclose\bgroup\originalleft}
\renewcommand{\right}{\aftergroup\egroup\originalright}

\theoremstyle{definition}

\title{Revisit of Adaptivity analysis}

\author{}

\date{}

\begin{document}

\maketitle

\section{Attempt 1: linear-type based}

\begin{figure}[h]
 $$
 \begin{array}{rcl}
     \text{Types} & \quad & \tau ::= b \sep \tau \multimap \tau' \sep !_n \tau \sep
     \tau \times \tau \sep \tforallN{i}{\tau} \sep \query \\[2mm]

     \text{Term} & \quad & t ::= c \sep \fix{t} \sep \app{t}{t} \sep !t \sep (t_1,t_2) \sep  \letx{!x}{t_1}{t_2} \sep \Lambda.t \sep t[] \sep \abs{x}{t} \sep  M(t) \sep x \sep q \sep\\
     && \quad   \tcaseof{t}\ \{c_i \Rightarrow t_i\}_{c_i \in b}  \sep \letx{(x_1,x_2)}{t_1}{t_2} \\[2mm]
      
     \text{Normal Form} &\quad & v ::=  c \sep \fix{t} \sep !t \sep (v_1, v_2) \sep \Lambda. t \sep \abs{x}{t}  \sep x \sep q \sep \tcaseof{v}\ \{c_i \Rightarrow v_i\}_{c_i \in b_i} \sep\\
     && \enil \sep \econs(v_1,v_2)  \\[2mm]

     \text{Mechanisms} &\quad & M ::=  {\tt gauss} \sep {\tt thdt} \\[2mm]


	\text{Tree} &\quad& T_b :: = c \sep M(T_{query}) \sep \tcaseof{T_b}\ \{ c_i \Rightarrow T_{b_i}\}_{c_i \in b} \\

	\text{} &\quad& T_{query} :: = q \sep \tcaseof{T_b}\ \{ c_i \Rightarrow T_{query_i}\}_{c_i \in b} \\[2mm]
     \text{Depth} &\quad&   \depth(c) = 0 \\
       &\quad& \depth(!t) = \depth(t) \\
           &\quad&      \depth( \app{t_1}{t_2} ) = \max(\depth(t_1), \depth(t_2)) \\
            &\quad&  \depth(M(t)) = 1 + \depth(t) \\
             &\quad&  \depth(\abs{x}{t}) = \depth(t) \\
              &\quad& \depth(x) = 0 \\
              & \quad & \depth(q) = 0 \\
              & \quad & \depth((t_1, t_2)) = \max(\depth(t_1), \depth(t_2))\\
              & \quad & \depth(\letx{(x_1, x_2)}{t}{t'}) = \max(\depth(t), \depth(t'))\\
              & \quad & \depth{}(\letx{!x}{t}{t'}) = \max(\depth(t), \depth(t'))\\
              & \quad & \depth{}(\tcaseof{t}\ \{c_i \Rightarrow t_i\}_{c_i \in b}) = \max(\depth(t), \depth(t_i))\\
            & \quad & \depth(\Lambda.t) = \depth(t)\\
              & \quad & \depth(t\, []) = \depth(t)\\
\end{array}
$$
\caption{syntax}
\end{figure}

%
%
\begin{figure}[h]
\boxed{  \Gamma \jtype{n,m}{}{t}{\tau}    }\\
\boxed{  \Gamma :: = \emptyset \sep \Gamma, x : \tau \sep \Gamma, x : [\tau]_p   }

\begin{mathpar}
  \inferrule*[right = const]
   {\empty}
   {\Gamma \jtype{n,m}{}{c}{b}  }
   
   \and
    \inferrule*[right = abs]
   {\Gamma, x: \tau_1 \jtype{n}{}{t}{\tau_2}}
   { \Gamma \jtype{n,m}{}{\abs{x}{t}}{\tau_1 \multimap \tau_2}  }
   
   \and
   \inferrule*[right = pr]
   {[\Gamma] \jtype{n}{}{t}{\tau}}
   {\Delta, p + [\Gamma] \jtype{n+p}{}{!t}{!_p \tau}  }
%   \boxed{
%   \inferrule*[right = pr]
%   {[\Gamma] \jtype{n}{}{t}{\tau}}
%   {p + [\Gamma]\textbf{} \jtype{n}{}{!t}{!_p \tau}  }}
   
   \and
    \inferrule*[ right = var]
   {\empty}
   {\Gamma, x:\tau \jtype{n}{}{x}{\tau}  } 
   
   \and
   \inferrule*[ right = MT ]
   {[\Gamma] \jtype{n}{}{t}{query}}
   {\Delta, 1 + [\Gamma] \jtype{n+1}{}{M(t)}{b}  }
   
   \and
    \inferrule*[right = query]
   {\empty}
   {\Gamma \jtype{n}{}{q}{query}  }
   
   \and
%     \inferrule*[ right = app ]
%   {\Gamma_1 \jtype{n_1}{}{t_1}{\tau_1 \rightarrow \tau_2} \\ \Gamma \jtype{n_2}{}{t_2}{\tau_1}}
%   { \Gamma_2 \jtype{\max(n_1,n_2)}{}{\app{t_1}{t_2}}{\tau_2}  }
  \inferrule*[ right = app ]
   {\Gamma_1 \jtype{n_1}{}{t_1}{\tau_1 \multimap \tau_2} \\ \Gamma_2 \jtype{n_2}{}{t_2}{\tau_1}}
   { \max(\Gamma_1, \Gamma_2) \jtype{\max(n_1,n_2)}{}{\app{t_1}{t_2}}{\tau_2}  }
   
   \and 
   
%   \inferrule*[ right = der ]
%   {\Gamma, x: \tau \jtype{n}{}{t}{ \tau }  }
%   { \Gamma , x: [\tau]_p \jtype{n }{}{t }{\tau }  }
   \boxed{
   \inferrule*[ right = der ]
   {\Gamma, x: \tau \jtype{n}{}{t}{ \tau }  }
   { \Gamma , x: [\tau]_0 \jtype{n }{}{t }{\tau }  }
 }
   
   \and 
   
   \inferrule*[ right = let-b ]
   {\Gamma_1 \jtype{n_1}{}{t}{!_p \tau} \\ \Gamma_2, x: [\tau]_p \jtype{n_2}{}{t'}{\tau'}}
   { \max(\Gamma_1, \Gamma_2)  \jtype{\max(n_1,n_2)}{}{\letx{!x}{t}{t'}}{\tau'}  }
   
   \and 
   \inferrule*[ right = let-p ]
   {\Gamma_1 \jtype{n_1}{}{t}{\tau_1 \times \tau_2 } \\ \Gamma_2, x_1: \tau_1, x_2 : \tau_2 \jtype{n_2}{}{t'}{\tau'}}
   { \max(\Gamma_1, \Gamma_2)  \jtype{\max(n_1,n_2)}{}{ \letx{(x_1,x_2)}{t}{t'} }{\tau'}  }
   
   \and
     \inferrule*[right = pair]
   {\Gamma_1 \jtype{n_1}{}{t_1}{\tau_1} \\ \Gamma_2 \jtype{n_2}{}{t_2}{\tau_2}}
   { \max(\Gamma_1, \Gamma_2)  \jtype{\max(n_1,n_2)}{}{(t_1, t_2)}{\tau_1 \times \tau_2}  }
 
   \and
    \inferrule*[ right = case-const ]
   {\Gamma_1 \jtype{n_1}{}{t}{b} \\ \Gamma_2 \jtype{n_2}{}{t_i}{b} }
   {\max(n_2 + \Gamma_1, \Gamma_2)  \jtype{ (n_1+n_2)}{}{\tcaseof{t}\ \{c_i \Rightarrow t_i\}_{c_i \in b} } {b} }
   
   \and
    \inferrule*[ right = case-query]
   {\Gamma_1 \jtype{n_1}{}{t}{b} \\ \Gamma_2 \jtype{n_2}{}{t_i}{query} }
   {\max(\Gamma_1, \Gamma_2) \jtype{ (n_1+n_2)}{}{\tcaseof{t}\ \{c_i \Rightarrow t_i\}_{c_i \in b} } {query} }
   
   \inferrule*[right = iabs]
  { 
    \inferrule*[]
    {}
    {i::\mathbb{N};\Gamma \jtype{n}{}{t}{ \tau } }
    \and
    \inferrule*[]
    {}
    { i \notin \fiv{\Gamma }  } 
  }
  { \Gamma \jtype{n}{ }{  \Lambda.t  }{ \tforallN{i}{\tau}  } }
  
   \inferrule*[ right =  iapp]
  { 
    \inferrule*[]
    {}
    { \Gamma  \jtype{n}{}{t}{ \tforallN{i}{\tau}   } }
    \and
    \inferrule*[]
    {}
    { \jiterm{I}{ \mathbb{N} } } 
  }
  {\Gamma \jtype{n }{ }{t\, [] }{ \tau \{ I/i \}  } }

  \inferrule*[right = sub]
  { 
   { \Gamma \jtype{n}{}{t}{\tau} } \\
   {  \Gamma' \subseteq \Gamma } \\
   { \vDash n \leq n' } \\
   { \tau \subseteq \tau' }
  }
  { \Gamma' \jtype{n'}{}{t}{\tau'} }
\end{mathpar}
\caption{Typing judgment}
\end{figure}

\clearpage

\begin{figure}[h]
 \begin{mathpar}
  
  \inferrule*[right= S-ID]
  { }
  { \tau <: \tau  }
  \and
  \inferrule*[right = S-B]
  { 
   {A <: B}
   \\
   { q \leq p }
  }
  { !_p A <: !_q B  }
  \and
  \inferrule*[right =  S-ARROW]
  { {A' <: A}
    \\
    {B <: B'}
  }
  { A \multimap B <: A' \multimap B' }
  \and
  \inferrule*[right = S-D ]
  {
    { A \subseteq B }\\
    { q \leq p }
  }
  { [A]_p \subseteq [B]_q }
  
  \and
  \inferrule*[right = S-IDC]
  { }
  { \Gamma \subseteq \Gamma }
  
   \and
  \inferrule*[right = S-empty]
  { }
  { \Gamma \subseteq \emptyset }
  
  \and
  \inferrule*[right = S-Ctx]
  {
  {A \subseteq B}\\
  { \Gamma \subseteq \Delta }
  }
  { \Gamma, x: A \subseteq \Delta, x: B }
  
  \and
  \inferrule*[right = S-xctx1]
  {
  { \Delta \subseteq \Gamma }
  }
  { x : \tau, \Delta \subseteq \Gamma }
  
  \and
  \inferrule*[right = S-xctx2]
  {
  { \Delta \subseteq \Gamma }
  }
  { x : [\tau]_p, \Delta \subseteq \Gamma }
  
 \end{mathpar}
 \caption{sub typing}
\end{figure}



\begin{figure}[h]
\boxed{\eval{t }{v }{m}}
\begin{mathpar}
%  \inferrule*[ right=E-values]
%   { }
%   { \eval{ F   }{ F  }{0}}   
%   \and
 \inferrule*[ right=E-const]
  { }
  { \eval{ c   }{ c  }{0}}   
  
  \and

 \inferrule*[ right=E-query]
  { }
  { \eval{  q  }{ q  }{0}}   
  
  \and

 \inferrule*[ right=E-ABS]
  { }
  { \eval{ \abs{x}{t}   }{ \abs{x}{t}  }{0}}   
  
  \and
  
  \inferrule*[ right=E-bang]
  { }
  { \eval{ ! t   }{ ! t  }{0}}   
  
  \and
  
   \inferrule*[ right=E-pair]
  {   
    { \eval{ t_1  }{ v_1  }{m_1} }
    \\
    { \eval{ t_2  }{ v_2  }{m_2} } 
  }
  { \eval{  (t_1,t_2)  }{ (v_1,v_2)  }{ \max(m_1, m_2) %m_1+m_2
  } }  
  
  \and

   \inferrule*[ right=E-app]
  {   
    { \eval{ t_1  }{ \abs{x}{t}  }{m_1} }
    \\
    { \eval{ t_2  }{ v  }{m_2} } 
    \\
    { \eval{t[v/x] }{ v'}{m_3 } }
  }
  { \eval{ \app{t_1}{t_2}  }{ v'  }{ \max(m_1, m_2 ) + m_3  } }  
 
 \boxed{ 
   \inferrule*[ right=E-let-bang]
  {   
    { \eval{ t_1  }{ !t_3  }{m_1} } 
    \\
    {\eval{t_3}{v'}{m_2}}
    \\
    { \eval{t_2[v'/x] }{v}{m_3 } }
  }
  { \eval{  \letx{!x}{t_1}{t_2}  }{ v  }{ \max(m_1+ m_2, m_3)  } }  
}

%  \inferrule*[ right=E-let-bang]
%   {   
%     { \eval{ t_1  }{ !t_3  }{m_1} } 
%     \\
%     { \eval{t_2[!t_3 /x] }{ F}{m_3 } }
%   }
%   { \eval{  \letx{!x}{t_1}{t_2}  }{ F  }{ m_1+m_3  } } 
  
  \inferrule*[ right=E-let-p]
  {   
    { \eval{ t  }{ (v_1,v_2)  }{m_1} } 
    \\
    { \eval{t'[v_1/x_1][v_2/x_2] }{v}{m2 } }
  }
  { \eval{  \letx{(x_1,x_2)}{t}{t'}  }{ v  }{ \max(m_1,m_2)  } } 
  
 

  \inferrule*[ right=E-case]
  { 
    \inferrule*[]
    {}
    {\eval{  t  }{ v }{m }  }
    \\
    \inferrule*[]
    {}
    { \eval{ t_i  }{ v_i   }{ m_i }  }
  }
  { \eval{ \tcaseof{t}\ \{c_i \Rightarrow t_i\}_{c_i \in b}  }{ \tcaseof{v}\ \{c_i \Rightarrow v_i\}_{c_i \in b}  }{  m + \max(m_i) } }
  
    \inferrule*[ right=E-fix]
  { 
  }
  { \eval{  \fix{t}  }{ \fix{t}  }{ 0 } }
  
      \inferrule*[ right=E-x]
  { 
  \empty
  }
  { \eval{  x  }{ x  }{ 0 } }
  
      \inferrule*[ right=E-ILAM]
  { 
    \empty
  }
  { \eval{  \Lambda. t  }{  \Lambda. t }{ 0 } }
  
      \inferrule*[ right=E-iapp]
  { 
    \inferrule*[]
    {}
    {\eval{  t  }{ \Lambda. t' }{m }  }
  }
  { \eval{  t[]  }{  t' }{  m } }
  
      \inferrule*[ right=E-mech]
  { 
    \inferrule*[]
    {}
    {\eval{  t  }{ v }{m }
    \and 
    \eval{M(v)}{v'}{1}
    }
  }
  { \eval{  M(t)  }{ v'  }{  m + 1 } }
  
\end{mathpar}
 \caption{Evaluation Rules}
\end{figure}

\begin{figure}[h]
\begin{mathpar}

\inferrule*[right = E-if-true]
{
 \eval{b}{true}{0}
 \and 
 \eval{t_1}{v_1}{m} 
}
{
\eval{\tif{b}{t_1}{t_2}}{v_1}{m}
}

\inferrule*[right = E-if-false]
{
 \eval{b}{false}{0}
 \and 
 \eval{t_2}{v_2}{m} 
}
{
\eval{\tif{b}{t_1}{t_2}}{v_2}{m}
}

\inferrule*[right = E-nil]
{
 \empty
}
{
\eval{\enil}{\enil}{0}
}

\inferrule*[right = E-cons]
{
\eval{t_1}{v_1}{m_1}
\and 
\eval{t_2}{v_2}{m_2}
}
{
\eval{\econs(t_1,t_2)}{\econs(v_1,v_2)}{\max(m_1,m_2)}
}


\inferrule*[right = E-let]
{
\eval{t_2}{v_2}{m_2}
\and
\eval{t[v_2/x]}{v}{m}
}
{
\eval{\letx{x}{t_2}{t}}{v}{\max(m_2,m)}
}

\end{mathpar}
 \caption{New Added Evaluation Rules}
\end{figure}


\begin{figure*}[h]
$$
\begin{array}{rcl}
      \llu{\tau}{\epsilon}  &\quad &  = \{ \, e \, | \, \exists v. e \Downarrow v \land v \in   \llu{\tau}{v} \,  \}  \\[2mm]
      \llu{b}{v} &\quad &  = \{ \,  v  \, | \, v = T_b \}  \\[2mm]
      \llu{query}{v} &\quad &  = \{ \,  v \, |  \, v = T_{query} \}  \\[2mm]
      \llu{\tau_1 \rightarrow \tau_2}{v} & \quad & = \{\, \abs{x}{t} \, | \, \forall v \in\llu{\tau}{v}. t[v/x] \in \llu{\tau_2}{\epsilon} \, \} \\[2mm]
      \llu{ \ !_n \tau}{v} & \quad & = \{\, !t \, | \, t \in \llu{\tau}{\epsilon} \, \} \\[2mm]
      \llu{\tforallN{i}{\tau}}{v}  & \quad & = \{  \Lambda. t \, | \, \forall I. \vdash i :: \mathbb{N}. t[I/i] \in \llu{\tau}{\epsilon}   \}  \\[2mm]
      \llu{\tau_1 * \tau_2}{v}  & \quad & = \{  (v_1, v_2) \, | \, v_1 \in \llu{\tau_1}{v} \land v_2 \in \llu{\tau_2}{v}     \} \\[2mm]
      \llu{\cdot}{} &\quad & = \{ \emptyset \} \\[2mm]
      \llu{\Gamma, x : [\tau]_p}{} & \quad & = \{ \gamma[x \rightarrow v] | v \in \llu{\tau}{v} \land \gamma \in \llu{\Gamma}{}   \}  \\ [2mm]
      \boxed{\llu{\Gamma, x : [\tau]_p}{}}  & \quad & = \{ \gamma[x \rightarrow v] | v \in \llu{!_p \tau}{v}  \land \gamma \in \llu{\Gamma}{}   \}  \\ [2mm]
      \llu{\Gamma, x : \tau}{} & \quad & = \{ \gamma[x \rightarrow v] | v \in \llu{\tau}{v} \land \gamma \in \llu{\Gamma}{}   \}  \\ [2mm]
      \gamma \vDash \Gamma &\quad & \triangleq dom(\gamma) = dom(\Gamma) \land \forall x \in dom(\Gamma). \gamma(x) \in \llu{\Gamma(x)}{v}
\end{array}
$$
\caption{denotations}
\end{figure*}


\clearpage
\begin{lem} $ $
	\label{lem:1}
    \begin{enumerate}
\item If $\jtype{n,m}{}{v}{b} $ then $ \exists T_{b} : v = T_{b}$.\\
\item If $\jtype{n,m}{}{v}{query} $ then $ \exists T_{query} : v = T_{query}$
\end{enumerate}
	
	
\end{lem}

\begin{lem}[Depth Definition] 
	\label{lem:2}
	If $\Gamma \jtype{n,m}{}{t}{\tau} $ then $\depth(t) \leq n$\\
\end{lem}
%
%
%
\begin{lem}[Depth Weakening1] 
	\label{lem:deweaken1}
	$\Gamma \jtype{n_1,m}{}{t}{\tau} \land n_1 \leq n_2 \implies \Gamma \jtype{n_2,m}{}{t}{\tau}$\\
\end{lem}
\begin{proof}
  By induction on $\Gamma \jtype{n_1}{}{t}{\tau}  $.
\end{proof}

\begin{lem}[Depth Weakening2] 
	\label{lem:deweaken2}
	$\Gamma, x:[\tau]_{p_1} \jtype{n}{}{t}{\tau} \land p_1 \leq p_2 \implies \exists m. n \leq m$ s.t. $\Gamma, x:[\tau]_{p_2} \jtype{n}{}{t}{\tau} $\\
% 	$\Gamma, x:[\tau]_{p_1} \jtype{n}{}{e}{\tau} \land p_1 \leq p_2 \land \Gamma \subseteq \Gamma' \land n \leq n' \implies \Gamma', x:[\tau]_{p_2} \jtype{n'}{}{e}{\tau} $\\
\end{lem}
%
%
%
\begin{lem}[Context weakening - 1]
    \label{lem:coweaken1}
    $\Gamma \jtype{n,m}{}{t}{\tau}  \implies \Gamma, x:\tau \jtype{n,m}{}{t}{\tau} $\\
\end{lem}

\begin{lem}[Context weakening - 2]
    \label{lem:coweaken2}
    $\Gamma \jtype{n,m}{}{t}{\tau}  \implies \Gamma, x:[\tau]_p \jtype{n,m}{}{t}{\tau} $\\
\end{lem}



\begin{lem}[Context exchange]
    \label{lem:coex}
    $\Gamma, x : \tau_1, \Delta, y : \tau_2 \jtype{n,m}{}{t}{\tau}  \implies \Gamma, y : \tau_2, \Delta, x : \tau_1 \jtype{n,m}{}{t}{\tau} $\\
\end{lem}

\begin{lem}
	\label{lem:sub}
	If $\Gamma \jtype{n}{}{t}{\tau}$ and $\gamma \vDash \Gamma$, then $ \cdot \jtype{n}{}{\gamma(t)}{\tau} $\\
\end{lem}

\begin{lem}
	\label{lem:subext}
	If $\Gamma \subseteq \Gamma'$,  and $\Gamma' \jtype{n}{}{t}{\tau}$, then $\exists m. n \leq m$ s.t.  $\Gamma  \jtype{m}{}{t}{\tau} $. \\
\end{lem}
%
%
%
%%%%%%%%%%%%%%%%%%%%%%%%%%
\begin{thm}[Type Safety]
	If $\cdot \jtype{n,m}{}{t}{\tau} $ then $ \exists F. t \Downarrow F \land \jtype{n,m}{}{F}{\tau}$
\end{thm}
%
\begin{coro}
\label{cor:typesafety}
	If $ \cdot\jtype{n,m}{}{t}{b} $ then $ \exists T_b. t \Downarrow T_b \land \depth(T_b) \leq n$
\end{coro}
%

\begin{thm}[Normalization] 
	If $\cdot\jtype{n,m}{}{t}{\tau} $ then $ \exists F: t \Downarrow F $
\end{thm}
We prove two theorems instead.
\begin{thm}
If $\gamma(t) \in \llu{\tau}{\epsilon} $, then $\exists F.\eval{ \gamma(t)}{ F}{}$.
\end{thm}
%
%
\begin{thm}
 If $\Gamma \jtype{n}{}{t}{\tau}$ and $\gamma \vDash{\Gamma}$, then $\gamma(t) \in \llu{\tau}{\epsilon} $.
\end{thm} 

\begin{thm}[Preservation]
	If $\cdot\jtype{n}{}{t}{\tau} \land t \Downarrow F$ then $ \jtype{n}{}{F}{\tau} $
\end{thm}
%
%
\begin{thm} [Substitution]
	If $\Gamma \jtype{n_1}{}{t_1}{\tau_1}$  and   $\Delta, x: [\tau_1]_p \jtype{n_2}{}{t_2}{\tau_2}$ then $\max( \Gamma,\Delta) \jtype{\max{(p + n_1, n_2)}}{}{t_2[t_1/x]}{\tau_2} $
\end{thm}
%
%
\begin{thm} [Substitution] $ $
\label{thm:sub}
\begin{enumerate}
    \item If $\Gamma \jtype{n_1}{}{t_1}{\tau_1}$  and   $\Delta, x: \tau_1 \jtype{n_2}{}{t_2}{\tau_2}$ then $max(\Gamma, \Delta) \jtype{\max{(n_1, n_2)}}{}{t_2[t_1/x]}{\tau_2} $
    \item If $\Gamma \jtype{n_1}{}{ t_1}{!_p \tau_1}$  and   $\Delta, x: [\tau_1]_p \jtype{n_2}{}{t_2}{\tau_2}$ then $max(\Gamma, \Delta) \jtype{\max{(n_1 + p, n_2)}}{}{t_2[t_1/x]}{\tau_2} $
\end{enumerate}
\end{thm}

%
%
\fail{multi-Round case doesn't work under this semantics}

%%%%%%%%%%%%%%%%%%%%%%%%%%%%%%%%%%%%%%%%%%%%%%%%%%%%%%%%%%%%%%%%%%%%%%%%%%%%%%%%%%%%%%%%%%%%%%%%%%%%%%
%%%%%%%%%%%%%%%%%%%%%%%%%%%%%%%%%%%%%%%%%%%%%%%%%%%%%%%%%%%%%%%%%%%%%%%%%%%%%%%%%%%%%%%%%%%%%%%%%%%%

\section{Attempt 2: Trace-based effect system}
\[\begin{array}{llll}
\mbox{Expr.} & \expr & ::= & x ~|~ \expr_1 \eapp \expr_2 ~|~ {\efix f(x:\type).\expr}
 ~|~ (\expr_1, \expr_2) ~|~ \eprojl(\expr) ~|~ \eprojr(\expr) ~| \\
%
& & & \etrue ~|~ \efalse ~|~ \eif(\expr_1, \expr_2, \expr_3) ~|~
\econst ~|~ \eop(\expr)  ~|~  {\eilam \expr ~|~ \expr \eapp [] } \\
& & & ~|~  {\elet  x:q = \expr_1 \ein \expr_2 } ~|~ \enil ~|~  \econs (
      \expr_1, \expr_2) \\
& & & ~|~ { ~~~~~~~
 \bernoulli \eapp \expr ~|~ \uniform \eapp \expr_1 \eapp
      \expr_2 } \\
%
\mbox{Value} & \valr & ::= & \etrue ~|~ \efalse ~|~ \econst ~|~
(\efix f(x:\type).\expr, \env) ~|~ (\valr_1, \valr_2) 
    ~|~ \enil ~|~ \econs (\valr_1, \valr_2) | \\
& & &  {(\eilam \expr , \env) } \\ 
%
\mbox{Environment} & \env & ::= & x_1 \mapsto \valr_1, \ldots, x_n \mapsto \valr_n
\end{array}\]


%%%%%%%%%%%%%%%%%%%%%%%%%%%%%%%%%%%%%%%%%%%%%%%%%%%%%

%%%%%%%%%%%%%%%%%%%%%%%%%%%%%%%%%%%%%%%%%%%%%%%%%%%%%


\subsection{Tracing operational semantics and adaptivity}

\paragraph{Traces}
A trace $\tr$ is a representation of the big-step derivation of an
expression's evaluation. Our big-step semantics output a trace. We use
traces to define the adaptivity of a run. Our notion of traces and the
tracing semantics is taken from~\cite[Section 4]{perera:dep}, but we
omit their ``holes'' for which we have no need. The construct
$\trapp{\tr_1}{\tr_2}{f}{x}{\tr_3}$ records a trace of function
application. $\tr_1$ is the trace of the head, $\tr_2$ the trace of
the argument and $\tr_3$ is the trace of the function body. $f$ and
$x$ are bound in $\tr_3$.
%
\[\begin{array}{llll}
\mbox{Trace} & \tr & ::= & {(x, \env)} ~|~ \trapp{\tr_1}{\tr_2}{f}{x}{\tr_3} ~|~
{ (\trfix f(x:\type).e, \env) } ~|~ (\tr_1, \tr_2) ~|~ \trprojl(\tr) ~|\\ 
%
& & & \trprojr(\tr) ~|~ \trtrue ~|~ \trfalse ~|~ \trift(\tr_b, \tr_t)
~|~ \triff(\tr_b, \tr_f) ~|~ \trconst ~|~ \trop(\tr) \\
%
& & & \trnil ~|~ \trcons (\tr_1, \tr_2) ~|~ \triapp{\tr_1}{\tr_2} ~|~
       {(\eilam \expr, \env)}
\end{array}\]


\paragraph{Big-step tracing semantics}
The big-step, tracing semantics $\env, \expr \bigstep \valr, \tr$
computes a value $\valr$ and a trace $\tr$ from an expression $\expr$
and an enviroment $\env$ which maps the free variables of $\expr$ to
\emph{closed} values. The rules, taken from~\cite{perera:dep}, are
shown in Figure~\ref{fig:big-step}. Some salient points:
\begin{itemize}
\item[-] Erasing the traces from the semantics yields a standard
  big-step semantics.
\item[-] The trace of a primitive application $\eop(\expr)$
  records that $\eop$ was applied to the trace of
  $\expr$. This enables us to define adaptivity from a trace later.
\item[-] The trace of a variable $x$ is $x$. This way traces record
  where substitutions occur and, hence, variable dependencies. This is
  also needed for defining adaptivity.
\end{itemize}

\begin{figure}
\begin{mathpar}
   { \inferrule{ }{\env, x \bigstep \env(x), (x, \env ) }  }
  %
  \and
  %
  \inferrule{ }{\env, \econst \bigstep \econst, \trconst}
  %
  \and
  %
  \inferrule{ }{\env, \etrue \bigstep \etrue, \trtrue}
  %
  \and
  %
  \inferrule{ }{\env, \efalse \bigstep \efalse, \trfalse}
  %
  \and
  { \inferrule{  \env, \expr \bigstep \econst, \tr }{\env, \bernoulli \eapp \expr \bigstep \econst,
      \bernoulli (\tr)
    } }
  \and
 \inferrule{ \env, \expr_1 \bigstep \econst, \tr_1 \\ \env, \expr_2 \bigstep \econst, \tr_2  }{\env, \uniform \eapp \expr_1 \eapp
      \expr_2\bigstep \econst, \uniform(\tr_1,\tr_2)  } 
  \and
  %
  { \inferrule{
  }{
    \env, \efix f(x:\type). \expr \bigstep (\efix f(:\type).\expr, \env),
    (\trfix f(x:\type).\expr, \env)
  }
}
  %
  \and
  %
  \inferrule{
    \env, \expr_1 \bigstep \valr_1, \tr_1 \\
    { \valr_1 = (\efix f(x:\type).\expr, \env')} \\\\
    \env, \expr_2 \bigstep \valr_2, \tr_2 \\
    \env'[f \mapsto \valr_1, x \mapsto \valr_2], \expr \bigstep \valr, \tr
  }{
    \env, \expr_1 \eapp \expr_2 \bigstep \valr, \trapp{\tr_1}{\tr_2}{f}{x}{\tr}
  }
  %
  \and
  %
  \inferrule{
    \env, \expr_1 \bigstep \valr_1, \tr_1 \\
    \env, \expr_2 \bigstep \valr_2, \tr_2
  }{
    \env, (\expr_1, \expr_2) \bigstep (\valr_1, \valr_2), (\tr_1, \tr_2)
  }
  %
  \and
  %
  \inferrule{
    \env, \expr \bigstep (\valr_1, \valr_2), \tr
  }{
    \env, \eprojl(\expr) \bigstep \valr_1, \trprojl(\tr)
  }
  %
  \and
  %
  \inferrule{
    \env, \expr \bigstep (\valr_1, \valr_2), \tr
  }{
    \env, \eprojr(\expr) \bigstep \valr_2, \trprojr(\tr)
  }
  %
  \and
  %
  \inferrule{
    \env, \expr \bigstep \etrue, \tr \\
    \env, \expr_1 \bigstep \valr, \tr_1
  }{
    \env, \eif(\expr, \expr_1, \expr_2) \bigstep \valr, \trift(\tr, \tr_1)
  }
  %
  \and
  %
  \inferrule{
    \env, \expr \bigstep \efalse, \tr \\
    \env, \expr_2 \bigstep \valr, \tr_2
  }{
    \env, \eif(\expr, \expr_1, \expr_2) \bigstep \valr, \triff(\tr, \tr_2)
  }
  %
  \and
  %
  \inferrule{
    \env, \expr \bigstep \valr, \tr \\
    \eop{}(\valr) = \valr'
  }{
    \env, \eop(\expr) \bigstep \valr', \trop(\tr)
  }
%
\and
%
  \inferrule{
}
{ \env, \enil \bigstep \enil, \trnil }
%
\and
%
\inferrule{
\env, \expr_1 \bigstep \valr_1, \tr_1 \\
\env, \expr_2 \bigstep \valr_2, \tr_2
}
{ \env, \econs (\expr_1, \expr_2)  \bigstep \econs (\valr_1, \valr_2),
  \trcons(\tr_1, \tr_2)
}
%
\and
%
\inferrule{
  \env, \expr_1 \bigstep \valr_1, \tr_1 \\
  \env[x \mapsto \valr_1] , \expr_2 \bigstep \valr, \tr_2
}
{\env, \elet x;q = \expr_1 \ein \expr_2 \bigstep \valr, \trlet (x,
  \tr_1, \tr_2) }
%
\\\\
%
\boxed{\color{red}
\inferrule
{
  \empty
}
{
  \env, \eilam \expr \bigstep (\eilam \expr, \env), (\eilam \expr , \env)
}
}
%
\and
%
\boxed{\color{red}
\inferrule{
  \env, \expr \bigstep (\eilam \expr', \env'), \tr_1 \\
  \env, \expr' \bigstep \valr, \tr_2
}
{\env, \expr [] \bigstep \valr, \triapp{\tr_1}{\tr_2} }

}
\end{mathpar}
  \caption{Big-step semantics with provenance}
  \label{fig:big-step}
\end{figure}


\paragraph{Adaptivity of a trace}
We define the \emph{adaptivity} of a trace $\tr$, $\adap(\tr)$, which
means the maximum number of nested $\eop$s in $\tr$, taking variable
and control dependencies into account. To define this, we need an
auxiliary notion called the \emph{depth of variable $x$} in trace
$\tr$, written $\ddep{x}(\tr)$, which is the maximum number of
$\eop{}$s in any path leading from the root of $\tr$ to an occurence
of $x$ (at a leaf), again taking variable and control dependencies
into account. Technically, $\adap: \mbox{Traces} \to \nat$ and
$\ddep{x}: \mbox{Traces} \to \natb$. If $x$ does not appear free in
$\tr$, $\ddep{x}(\tr)$ is $\bot$.

The functions $\adap$ and $\ddep{x}$ are defined by mutual induction
in Figure~\ref{fig:adap}. 

\begin{figure}
  \framebox{$\adap: \mbox{Traces} \to \nat$}
  \begin{mathpar}
    \begin{array}{lcl}
       { \adap( (x,\env) )} & = & 0 \\
      %
      \adap(\trapp{\tr_1}{\tr_2}{f}{x}{\tr_3}) & = &
      \adap(\tr_1) + \max (\adap(\tr_3), \adap(\tr_2) + \ddep{x}(\tr_3))\\
      %
       {\adap( (\trfix f(x:\type).\expr, \env)  ) } & = & 0 \\
      %
      \adap((\tr_1, \tr_2)) & = & \max(\adap(\tr_1), \adap(\tr_2)) \\
      %
      \adap(\trprojl(\tr)) & = & \adap(\tr) \\
      %
      \adap(\trprojr(\tr)) & = & \adap(\tr) \\
      %
      \adap(\trtrue) & = & 0 \\
      %
      \adap(\trfalse) & = & 0 \\
      %
      \adap(\trift(\tr_b, \tr_t)) & = & \adap(\tr_b) + \adap(\tr_t) \\
      %
      \adap(\triff(\tr_b, \tr_f)) & = & \adap(\tr_b) + \adap(\tr_f) \\
      %
      \adap(\trconst) & = & 0 \\
      %
      \adap(\trop(\tr)) & = & { 1 + \adap(\tr) } \\
           & &       {  +  \textsf{MAX}_{\valr \in \type} \Big(
                              \max \big(\adap(\tr_3 (\valr) ),
                              \ddep{x}(\tr_3(\valr)) \big) \Big) } \\
      &\mathsf{where}&  { \valr_1 = (\efix f(x: \type). \expr, \env ) =
                       \mathsf{extract}(\tr) } \\
 & & { \conj  \env[f \mapsto
                       \valr_1, x \mapsto \valr], \expr \bigstep
                       \valr', \tr_3(\valr) } \\ 
      %
     \adap(\trnil) & = & 0 \\
     %
     \adap(\trcons(\tr_1,\tr_2) ) & = &  \max(\adap(\tr_1),
                                        \adap(\tr_2)) \\
     %
    \adap( \trlet (x, \tr_1,\tr_2) ) & = & \max (\adap(\tr_2),
                                           \adap(\tr_1)+\ddep{x}(\tr_2)  )
                                           \\
     \adap(\triapp{\tr_1}{\tr_2}) & = & \adap(\tr_1) + \adap(\tr_2)\\
    %
     { \adap( (\eilam \expr, \env) ) } & = & 0 \\
     { \adap( \bernoulli (\tr)  ) } & = & \adap(\tr) \\
      { \adap( \uniform  (\tr_1, \tr_2)  ) } & = & \max (\adap(\tr_1),
                                                      \adap(\tr_2) ) \\
      \end{array}
  \end{mathpar}
  %
  \framebox{$\ddep{x}: \mbox{Traces} \to \natb$}
  \begin{mathpar}
    \begin{array}{lcl}
       { \ddep{x}( ( y, \env )) } & = &
      \left\lbrace
      \begin{array}{ll}
        0 & \mbox{if } x = y \\
        \bot & \mbox{if } x \neq y
      \end{array}
      \right.\\
      %
      \ddep{x}(\trapp{\tr_1}{\tr_2}{f}{y}{\tr_3}) & = & \max(\ddep{x}(\tr_1), \\
      & & \adap(\tr_1) + \max(\ddep{x}(\tr_3), \ddep{x}(\tr_2) + \ddep{y}(\tr_3))) \\
      %
      { \ddep{x}(  (\trfix f(y:\type).\expr,\env)  )  }& = & \bot \\
      %
      \ddep{x}((\tr_1, \tr_2)) & = & \max(\ddep{x}(\tr_1), \ddep{x}(\tr_2)) \\
      %
      \ddep{x}(\trprojl(\tr)) & = & \ddep{x}(\tr) \\
      %
      \ddep{x}(\trprojr(\tr)) & = & \ddep{x}(\tr) \\
      %
      \ddep{x}(\trtrue) & = & \bot \\
      %
      \ddep{x}(\trfalse) & = & \bot \\
      %
      \ddep{x}(\trift(\tr_b, \tr_t)) & = & \max(\ddep{x}(\tr_b), \adap(\tr_b) + \ddep{x}(\tr_t)) \\
      %
      \ddep{x}(\trift(\tr_b, \tr_f)) & = & \max(\ddep{x}(\tr_b), \adap(\tr_b) + \ddep{x}(\tr_f)) \\
      %
      \ddep{x}(\trconst) & = & \bot \\
      %
      \ddep{x}(\trop(\tr)) & = & 1 +  \max(\ddep{x}(\tr),  \\
      & &  \adap(\tr) + \textsf{MAX}_{\valr \in \type} \Big(
          \max(\ddep{x}(\tr_3(\valr)), \bot )   \Big ) ) \\  
 &\mathsf{where}&  { \valr_1 = (\efix f(x: \type). \expr, \env ) =
                       \mathsf{extract}(\tr) } \\
 & & { \conj  \env[f \mapsto
                       \valr_1, x \mapsto \valr], \expr \bigstep
                       \valr', \tr_3(\valr) } \\ 
       %
      \ddep{x}(\trnil) & = & \bot \\
      %
      \ddep{x}(\trcons(\tr_1,\tr_2) ) & = & \max(\ddep{x}(\tr_1),
                                            \ddep{x}(\tr_2)) \\
      %
      \ddep{x}( \trlet(y, \tr_1, \tr_2) ) & = & \max( \ddep{x}(\tr_2),
                                                \ddep{x}(\tr_1)+\ddep{y}(\tr_2)  )\\
       \ddep{x}(\triapp{\tr_1}{\tr_2})  & = & 
                                                    \max(\ddep{x}(\tr_1), \adap(\tr_1) + \ddep{x}(\tr_2))\\
    %
     { \ddep{x}( (\eilam \expr, \env) ) } & = & \bot \\
    \ddep{x}(\uniform (\tr_1,\tr_2) ) & = & \max(\ddep{x}(\tr_1),
                                            \ddep{x}(\tr_2)) \\
  \ddep{x}(\bernoulli (\tr)) & = & \ddep{x}(\tr)
    \end{array}
  \end{mathpar}
  \caption{Adaptivity of a trace and depth of variable $x$ in a trace}
  \label{fig:adap}
\end{figure}

\subsection{Challenge (Couterexample)}



\clearpage
\section{Attempt 3: Linear type based 2}

\[\begin{array}{llll}
\mbox{Expr.} & \expr & ::= & x ~|~ \expr_1 \eapp \expr_2 
 ~|~ \lambda x. \expr 
    \\
             & & &  \etrue ~|~ \efalse ~|~
  \eif  \expr \ethen \expr_2 \eelse \expr_3 ~|~
\econst ~|~ \eop(\expr)  \\
\mbox{Environment} & \env & ::= & x_1 \mapsto (\valr_1, \adapt_1), \ldots, x_n \mapsto (\valr_n,\adapt_n)
\end{array}\]

\[
\begin{array}{llll}
  \mbox{Index Term} & \idx, \nnatA & ::= &     i ~|~ n \\
 %                                  - \idx_2 ~|~ \smax{\idx_1}{\idx_2}\\
%                                  \mbox{Sort} & S & ::= & \nat \\
  \mbox{Linear type} & \type &::=  &  \ltype \lto{\nnatA} \type ~|~
                                     \tbase ~|~ \tbool \\
  \mbox{Nonlinear Type} & \ltype & ::= & \bang{\idx} \type   \\
  \mbox{Typing context } & \Gamma & ::= & x_1 : \ltype_1, \ldots,
                                          x_n : \ltype_n
\end{array}
\]


\begin{figure}[h]
  \begin{mathpar}
    \inferrule{
    }{
     \valr, \env \bigstep{0} \valr, \env} ~\textsf{val}
   %  \and
   % \inferrule{  \mathsf{fetch} (\env,x)  =  (\valr, \adapt)  }{x, \env
   %   \bigstep{\adapt} \valr, \env }~\textsf{var}
   \and
   %
     \inferrule{  \env(x)  =  (\valr, \env_1,  \adapt)  }{x,
       \env  \bigstep{\adapt} \valr, \env_1 }~\textsf{var}
     %
   \and
  %
   \inferrule{ }{\econst , \env \bigstep{0} \econst, \env}~\textsf{const}
   %
   \and
   %
 \inferrule{
  }{
    \lambda x. \expr, \env
    \bigstep{0} \lambda x.\expr, \env
  }~\textsf{lambda}
  %
  \and
  %
  \inferrule{
    \expr_1, \env_1 \bigstep{\adapt_1} \lambda x.\expr , \env_1' \\
    %\forall x_i \in \dom(\env_1 \cap \env_2).  \fresh \eapp x_i' \\
     \expr_2, \env_2 \bigstep{\adapt_2} \valr_2 , \env_2' \\
    \fresh \eapp x' \\
    \expr[x'/x], \env_1'[ x'  \to (\valr_2, \env_2', \adapt_2  ) ] 
    \bigstep{\adapt_3} \valr, \env_3
  }{
     \expr_1 \eapp \expr_2 , (\env_1 \uplus \env_2)\bigstep{{\max(\adapt_1, \adapt_3)}} \valr, \env_3
  }
 %
  \and
  %
{
  \inferrule{
     \expr , \env \bigstep{\adapt} \valr , \env_1 
   }{
     \eop(\expr), \env \bigstep{\adapt +1} \eop(\valr),  \env_1
   }~\textsf{delta}
 }%
\and
 %
 \inferrule{
   \expr, \env \bigstep{\adapt} \efalse, \env'
   \\
   \expr_2 , \env \bigstep{\adapt_2} \valr_2, \env_2
  }{
    \eif \expr \ethen \expr_1 \eelse \expr_2 , \env \bigstep{\adapt +
      \adapt_2 } \valr_2, \env_2
  }~\textsf{if-f}
%
\and
%
\inferrule{
   \expr , \env \bigstep{\adapt} \etrue, \env'
   \\
    \expr_1, \env \bigstep{\adapt_1} \valr_1, \env_1
  }{
    \eif \expr \ethen \expr_1 \eelse \expr_2 , \env \bigstep{\adapt +
      \adapt_1 } \valr_1, \env_1
  }~\textsf{if-t}
  %
  \and
  %
{  
\inferrule{
    \expr, \env \bigstep{\adapt} (\expr_1, \expr_2) , \env_1 \\
    %\forall x_i \in \dom(\env_1 \cap \env_2).  \fresh \eapp x_i' \\
     \expr_1, \env_1 \bigstep{\adapt_1} \valr_1 , \env_1' \\
     \expr_2, \env_1 \bigstep{\adapt_2} \valr_2 , \env_2' \\
    \fresh \eapp x', y' \\
    \expr'[x'/x][y'/y], 
    \env_1[ x'  \to (\valr_1, \env_1', \adapt_1), 
    y' \to (\valr_2, \env_2', \adapt_2) ] 
    \bigstep{\adapt'} \valr, \env'
  }{
     \elet \eapp (x, y) = \expr \ein \expr' , \env \bigstep{{\max(\adapt, \adapt')}} \valr, \env'
  }~\textsf{bind}
  }
  \\\\
  \begin{array}{llll}
    \env_1 \uplus \emptyset & \triangleq & \env_1 &\\
     \emptyset \uplus \env_2 & \triangleq & \env_2 &\\
  \end{array}
\end{mathpar}
  \caption{Big-step semantics }
  \label{fig:semantics1}
\end{figure}


\begin{figure}[h]
  \begin{mathpar}
    \inferrule{
    }{
      \ictx \tctx , x: \bang{1} \type \tvdash{0} x: \type
    }~\textbf{Ax}
    %
    \and
    %
    \inferrule{
    }{
      \ictx \Gamma \tvdash{0} c : \tbase 
    }~\textbf{const}
    %
    %
    \and
    %
    \inferrule{
      \ictx \Gamma, x: \ltype
      \tvdash{\nnatA }
      \expr: \type
    }{
      \ictx \Gamma \tvdash{0} \lambda x. \expr : \ltype
      \lto{\nnatA} \type
    }~\textbf{lambda}
    \and
    %
    \inferrule{
      \ictx \Gamma_1  \tvdash{\nnatA_1} \expr_1:  \bang{\idx} \type_1
      \lto{\nnatA} \type_2      \\
      \ictx \Gamma_2 \tvdash{\nnatA_2} \expr_2: \type_1 
    }{
      \ictx   \Gamma_1 + \idx \times \Gamma_2  
      \tvdash{    \nnatA_1 +
        \idx \times \nnatA_2 + \nnatA    } 
        \expr_1 \eapp \expr_2 : \type_2
    }~\textbf{app}
    %
    \and
    %
    \inferrule{
      \ictx \Gamma \tvdash{\nnatA} \expr:  \tbase 
    }{
      \ictx \Gamma  \tvdash{1+\nnatA} \delta(\expr): \tbase
    }~\textbf{delta}
     %
    \and
    %
    \inferrule{
      \ictx \Gamma'  \tvdash{\nnatA'} \expr: \type' \\
      \Gamma' \leqslant \Gamma \\
      \nnatA' \leq \nnatA\\
      \sub{\type'}{\type} 
    }{
      \ictx \Gamma  \tvdash{\nnatA} \expr: \type 
    }~\textbf{subtype}
      %
    \and
    %
    \inferrule{
      \ictx \Gamma, y: \type', x: \type ,\Gamma'  \tvdash{\nnatA} \expr: \type 
    }{
      \ictx \Gamma, x: \type, y: \type' ,\Gamma'  \tvdash{\nnatA} \expr: \type 
    }~\textbf{exchange}
    %
    \and
    %
    {
    \inferrule{
      \ictx \Gamma \tvdash{\nnatA} \expr: \tbool
      \\
      \ictx \Gamma_1 \tvdash{\nnatA_1} \expr_1: \type
      \\
      \ictx \Gamma_2 \tvdash{\nnatA_2} \expr_2: \type
      \\
      \Gamma' = \Gamma + \Gamma_1 + \Gamma_2
      \\
      \nnatA' = \nnatA + \max(\nnatA_1, \nnatA_2)
    }{
      \ictx \Gamma' \tvdash{\nnatA'}  \eif \expr  \ethen \expr_1 \eelse \expr_2 : \type 
    }~\textbf{if}
    }
     \\\\
\begin{array}{llll}
  \idx \times \Gamma &\triangleq  &  \Gamma  & \idx =1  \\
                     &\triangleq  &  \bang{0} \Gamma & \idx =0 \\
  \bang{\idx_1} \type + \bang{\idx_2} \type  &\triangleq  
  & \bang{ \max(\idx_1,\idx_2)} \type &  \\
  \Gamma + \emptyset & \triangleq & \Gamma & \\
  \emptyset+ \Gamma  & \triangleq & \Gamma & \\
  ( [x : \ltype ],\Gamma) +  ([x: \ltype'],\Delta)  & \triangleq
                            & [x: \ltype + \ltype' ], \Gamma +
                              \Delta &   \\
   \sub{\Gamma}{\Delta} & \triangleq &  \dom(\Gamma) = \dom(\Delta) & \\
    & &                                    \land \forall x \in
                                      \dom(\Gamma),
        \sub{\Delta(x)}{\Gamma(x)} &  
\end{array}
  \end{mathpar}
  \caption{Typing rules, first version}
  \label{fig:type-rules1}
\end{figure}

\begin{figure}[h]
  \begin{mathpar}
    \inferrule{
      \idx_1 \leq \idx \\
      \sub{\ltype}{\ltype_1}
    }{
      \sub{\bang{\idx} \ltype}{\bang{\idx_1} \ltype_1}
    }~\textsf{bang}
    %
    \and
    %
    \inferrule{
        \nnatA \leq \nnatA' \\
        \sub{\type_1}{\type}   \\
      \sub{\type'}{\type_1'}
    }{
      \sub{\type \lto{\nnatA} \type' }{\type_1 \lto{\nnatA'} \ltype_1'}
    }~\textsf{arrow}
    %
    \and
    %
    \inferrule{
    }{
    \sub{\tbase}{\tbase}
    }~\textsf{base}
  \end{mathpar}
  \caption{subtyping}
 \end{figure}

 \clearpage

 \begin{figure}[h]
  \begin{mathpar}
    \inferrule{
     \env ( x ) = (\valr, \env', \adapt)
      \\
      \tvdash{\nnatA} ( \valr, \env') : \type
          }{
     \tvdash{\adapt + \nnatA}   ( x, \env):  \type
    }~\textbf{C-Ax}
    %
    \and
    %
    \inferrule{
    }{
      \tvdash{0} (  c, \env) : \tbase
    }~\textbf{C-const}
   
    \and
    %
    \inferrule{
      \tvdash{\nnatA' } ( \valr', \theta') : \type_1
      \\
      \fresh\eapp  x' ~~ \forall \adapt'
      \\
      \tvdash{ S+ \idx \times (\adapt' + \nnatA' ) +\nnatA }
     ( \expr[x'/x], \env[x' \to (\valr', \theta', R')]      ) :
     \type_2
    }{
     \tvdash{S} (  \lambda x. \expr, \env )  : \bang{\idx} \type_1
      \lto{\nnatA} \type_2
    }~\textbf{C-lambda}
    \and
    %
    \inferrule{
       \tvdash{\nnatA_1} ( \expr_1, \env_1) :  \bang{\idx} \type_1
      \lto{\nnatA} \type_2      \\
      \tvdash{\nnatA_2} ( \expr_2, \env_2 ): \type_1
    }{
       \tvdash{    \nnatA_1 +
        \idx \times \nnatA_2 + \nnatA    } (  \expr_1 \eapp \expr_2, \env_1 \uplus \env_2   ) : \type_2
    }~\textbf{C-app}
    %
    \and
    %
    \inferrule{
      \tvdash{\nnatA} (\expr, \env) :  \tbase
   }{  \tvdash{1+\nnatA} (\delta(\expr) , \env ) : \tbase
    }~\textbf{C-delta}
    \\\\
    \begin{array}{lll}
       \theta  & \triangleq (x_i \to (\valr_i, \env_i, R_i)) & i \in 
       \mathbb{N}\\
      (x_i : \bang{ \idx }\type_i), \Gamma \vDash (x_i \to (\valr_i, \env_i, R_i))
      \uplus \theta & \triangleq ~~~\tvdash { \_ } (\valr_i, \env_i)
                                          :  \type_i  &\conj
                                   \Gamma \vDash \theta
      \end{array}
  \end{mathpar}
  \caption{Typing rules, configure}
  \label{fig:configure-rules}
\end{figure}

\begin{figure}[h]
  \begin{mathpar}
    \begin{array}{lll}
      \lrv{\tbase} & = & \{  ( \econst, \env,  \nnatA)  \} \\
      %
      % \lrv{\type_1 \times \type_2} & = & \{(\valr_1, \valr_2) ~|~ \valr_1 \in \lrv{\type_1} \conj \valr_2 \in \lrv{\type_2} \}\\
      %
      \lrv{\bang{k} \type } & = & \{  ( \valr, \env,   \nnatA) |  (\valr, \env,
                                   \nnatA ) \in \lrv{\type}  \} \\
      %
      \lrv{ \bang{k} \type_1 \lto{\nnatA} \type_2    } & = &
      \{( \lambda x.\expr, \env,  \nnatA_1) ~|~ \forall \valr', \env',
      \nnatA'. ( \valr',\env',  \nnatA') \in
      \lrv{ \bang{k} \type_1 }.\\
      & & 
          \implies   \fresh \eapp x' \land \\
      & & \forall \adapt. ( \expr[x'/x], \env[x' \mapsto (\valr', \env', \adapt )] ) \in
          \lre{    }{ \nnatA_1+\nnatA+ \idx \times (\adapt + \nnatA') }{\type_2}     \} \\
      %
      \\
      %
      \lre{}{\nnatA}{\type} & = & \{  ( \expr, \env) ~|~  ( \expr , \env
                                  \bigstep{\adapt}  \valr, \env' ) \\
      & & ~~~~~~~~~~~~~\implies \adapt \leq \nnatA \conj 
     ( \valr, \env', \nnatA- \adapt) \in \lrv{\type})
      \}
    \end{array}
  \end{mathpar}
  \caption{Logical relation without step-indexing}
  \label{fig:lr:non-step}
\end{figure}

\paragraph{Typable Approach By Weihao}
\[
\begin{array}{ll}
 F(\expr, \phi ) & where \eapp ~~ \phi(x_i) = (\idx_i, \adapt_i, \nnatA_i ) \\
   F(x,\phi) & = \sum_{x_i \in \fv{x}  } \idx_i \times (\adapt_i+ \nnatA_i)  \\
F(\lambda x. \expr ,  \phi  ) & =  \sum_{x_i \in \fv{\lambda x.\expr}  } \idx_i \times (\adapt_i+ \nnatA_i)   \\  %\sum_{x_i \in \fv{\lambda x.\expr} } k_i \times R_i  \\
F(\delta(\expr) , \phi ) & = \sum_{x_i \in \fv{\delta(\expr)} } \idx_i \times (\adapt_i+ \nnatA_i)  \\
F(c, \phi ) & = 0  \\
F(\expr_1 \eapp \expr_2, \phi ) & = F(\expr_1, \phi ) +
                                  F(\expr_2,\phi ) \\
F(\eif \expr \ethen \expr_1 \eelse \expr_2, \phi ) & = F(\expr, \phi) + \max(F(\expr_1, \phi),  F(\expr_2, \phi)   )  
\end{array} 
\]

\begin{defn}[Typable]
  \label{typable}
  A closure $( \expr, [ x_1 \to (\valr_1 ,  \env_1 , \adapt_1 ) , \ldots, x_i \to (\valr_i, \env_i, \adapt_i )] )$ is typable with type $\type$ and adaptivity $J$ if exists $k_i$\\
  \[
     x_1 : \bang{\idx_1} \type_1, \ldots, \bang{\idx_i} \type_i 
     \tvdash{\nnatA}  \expr : \type  \]
   and each closure $(\valr_i, \env_i)$  is also typable with type $\bang{\idx_i} \type_i$ and adaptivity $\nnatA_i$, $ \phi = [x_1
     \to (\idx_1, \adapt_1, \nnatA_1), \ldots,  x_i \to (\idx_i, \adapt_i,
     \nnatA_i)  ] $,  $J = \nnatA + F( \expr, \phi ) $.
 \end{defn}

 \begin{defn}[ClosedClosure]
  \label{closure}
   A closure $(\expr, \env)$ is closed if $\fv{\expr} \subseteq \dom(\env)$. 
 \end{defn}

% \begin{lem}[ClosureTypable ]
%   \label{ct}
%    If a closure $(\expr, \env)$ is closed, then there exists $\type$ and $J$ so that $(\expr, \env)$ is typable with $\type$ and $J$.
%    \end{lem}
 

\begin{lem}[programTypable]
  \label{proglemma}
   If $ \tvdash{\nnatA}   \expr : \type $, then $(
     \expr, \emptyset ) $ is typable with $\type$ and adaptivity $\nnatA$. 
   \end{lem}

   \begin{lem}[TypableMono]
     \label{tmono}
     If a closure is $D$ is typable with $\type$ and $\nnatA$, and $\nnatA \leq \nnatA'$, then
     D is typable with $\type$ and $\nnatA'$.
    \end{lem} 

   
\begin{lem}[TypableSoundness]
  \label{tsound}
  If a closure $D$ is typable with $\type$ and $J$, and $D \bigstep{\adapt} E$, then
    closure $E$ is typable with $\type$ and adaptivity $J - \adapt$. 
   \end{lem}
\paragraph{Typable Approach By Marco}
\begin{defn}[Typable Closures]
  \label{def:typable}
Let $\env=[ x_1 \to (\valr_1 ,  \env_1 , R_1 ) ,
  \ldots, x_n \to (\valr_n, \env_n, R_n )]$. 
  The closure $( \expr, \env)$ is typable with
  type $\type$ and adaptivity $J$ if:
\begin{enumerate}\item  
     $x_1 : \bang{k_1} \type_1, \ldots, \bang{k_i} \type_i 
     \tvdash{Z}  \expr : \type$, for some types $\bang{k_i}
   \type_i$ for $(1\leq i\leq n)$, 
\item each closure $(\valr_i, \env_i)$ for $(1\leq i\leq n)$ is typable with type
  $\bang{k_i} \type_i$ and adaptivity $Z_i$,
\item $J = Z + \sum_{(v_i,\theta_i,S_i)\in\theta} k_i \times (R_i
  +Z_i)$.
\end{enumerate}
 \end{defn}
To justify why we chose $\sum$ in the third clause above it is worth
to consider the following configuration:
$$
[x\mapsto (\lambda u.\lambda w.\delta(u)+\delta(w),[\,],0)
,y:\mapsto (v,[\,],2) ], x\, y\, y
$$
   
\begin{lem}[Soundness]
  \label{tsound}
  If a closure $D$ is typable with type $\type$ and adaptivity $J$, and $D \bigstep{R} E$, then
    the closure $E$ is typable with type $\type$ and adaptivity $I$,
    where $I+R\leq J$. 
   \end{lem}
\fail{Soundness is unable to be proven}

\section{dependency graph}



\[\begin{array}{llll}
\mbox{AExpr.} & \aexpr & ::= & n~|~ x ~|~ \aexpr_1 + \aexpr_2  ~|~  \aexpr_1 -
                            \aexpr_2 ~|~ \aexpr_1 * \aexpr_2 \\
%
\mbox{BExpr} & \bexpr & ::= & v ~|~  \aexpr_1 < \aexpr_2 ~|~ \aexpr_1
                              = \aexpr_2 ~|~ \neg \bexpr ~|~ \bexpr_1
                              \land \bexpr_2 ~|~ \bexpr_1 \lor \bexpr_2
\\
  \mbox{Command} & \cmd& ::= & Skip ~|~ \cmd_1 ; \cmd_2 ~|~ \impif{\bexpr}{\cmd_1}{\cmd_2}
                               ~|~             \impwhile{\bexpr}{\cmd}  \\
              &&& ~|~\assign{x}{\aexpr} ~|~ \fassign{x}{\aexpr}{l}  
    \\
    \mbox{Label}  & \labl & \in &  \mathbb{Z} \\
    \mbox{Trace} & T & ::= &   \{  [(x_1,\labl_1),\dots, (x_i,
                             \labl_i)],  \dots,  [(y_1,\labl_1),\dots, (y_i,
                             \labl_i)]  \}    \\
\mbox{Environment} & \env & ::= & x_1 \mapsto (n_1, T_1),
                                  \ldots, x_n \mapsto
                                  (\valr_n,T_n) \\
    \mbox{Node}  & \node & ::=&  Empty ~|~D_1(x, T) ~|~ D_2(x,T) ~|~ IFT(T_b, \node) ~|~
                                IFF(T_b, \node) ~|~ W (T_b, \node)
                                ~|~ \node_1 ; \node_2\\
                                  \mbox{Trace} & T & \in  & Set <
                                                            List<Var
                                                            \times Label> >
\end{array}\]





%%%%%%%%%%%%%%%%%%%%%%%%%%%%%%%%%%%%%%%%%%%%%%%%%%%% sementics

%%%%%%%%%%%%%%%%%%%%%%%%%%%%%%%%%%%%%%%%%%%%%%%%%%%%% 

\begin{figure}
  \boxed{ \env, \aexpr \bigstep{T} n  }
  \begin{mathpar}
   \inferrule{ \env(x) =  (n, T)  }{\env , x \bigstep{T} n
   }~\textsf{var}
   %
   \and
   %
   \inferrule{  }{\env , n \bigstep{()} n
   }~\textsf{const}
  \and
  %
  \inferrule{
    \env, \aexpr_1 \bigstep{T_1} n_1\\
    \env, \aexpr_2 \bigstep{T_2} n_2
  }{\env, \aexpr_1 + \aexpr_2 \bigstep{ T_1 \cup T_2 }
     n_1 + n_2}~\textsf{sum} 
\end{mathpar}
\boxed{\env, \bexpr \bigstep{T} v }
\begin{mathpar}
   \inferrule{
    \env, \aexpr_1 \bigstep{T_1} n_1\\
    \env, \aexpr_2 \bigstep{T_2} n_2
  }{\env, \aexpr_1 + \aexpr_2 \bigstep{ T_1 \cup T_2 }
    n_1 < n_2}~\textsf{les}
  %
  \and
  %
  \inferrule{
    \env, \bexpr_1 \bigstep{T_1} v_1\\
    \env, \bexpr_2 \bigstep{T_2} v_2
  }{\env, \bexpr_1 \land \bexpr_2 \bigstep{ T_1 \cup T_2 }
    v_1 \land v_2}~\textsf{land}
\end{mathpar}
\boxed{ \env, \cmd \bigstep{\node} \env'  }
\begin{mathpar}
  \inferrule{ }{\env, Skip \bigstep{} \env
  }~\textsf{skip}
  %
  \and
  %
  \inferrule{
    \env, \cmd_1 \bigstep{\node_1} \env_1\\
    \env_1, \cmd_2 \bigstep{\node_2} \env_2
  }{\env, \cmd_1 ; \cmd_2 \bigstep{ \node_1 ; \node_2 }
    \env_2}~\textsf{Seq}
  %
  \and
  %
  \inferrule{
    \env, \aexpr \bigstep{ T } n 
  }{\env, \assign{x}{\aexpr} \bigstep{ D_1(x, T)  }
    \env[x \to (n, T)]  }~\textsf{Assign}
  %
  \and
  %
  \inferrule{
    \env, \aexpr \bigstep{ T } n \\
    \eop(n) = n_1
    }{\env, \fassign{x}{\aexpr}{\labl} \bigstep{ D_2(x,T+(x,l))}
    \env[x \to (n_1, T +(x,l) )] }~\textsf{FAssign}
  %
  \and
  %
  \inferrule{
    \env, \bexpr \bigstep{ T_b } false \\
    \env, \cmd_2 \bigstep{\node_2} \env_2
    }{\env, \impif{\bexpr}{\cmd_1}{\cmd_2}  \bigstep{ IFF(T_b, \node_2)}
    \env_2 }~\textsf{IFF}
 %
  \and
  %
  \inferrule{
    \env, \bexpr \bigstep{ T_b } true \\
    \env, \cmd_1 \bigstep{\node_1} \env_1
    }{\env, \impif{\bexpr}{\cmd_1}{\cmd_2}  \bigstep{ IFT(T_b, \node_1)}
      \env_1 }~\textsf{IFT}
    %
    \and
    %
    \inferrule{
    \env, \bexpr \bigstep{ T_b } false 
    }{\env, \impwhile{\bexpr}{\cmd}  \bigstep{ WF(T_b)}
      \env }~\textsf{WHILEF}
    %
    \and
    %
    \inferrule{
      \env, \bexpr \bigstep{ T_b } true\\
      \env, \cmd \bigstep{\node_1} \env_1  \\
      \env_1, \impwhile{\bexpr}{\cmd} \bigstep{\node_2} \env_2
    }{\env, \impwhile{\bexpr}{\cmd}  \bigstep{ WT(T_b,\node_1, \node_2 )}
      \env_2 }~\textsf{WHILET}
 \\\\
  \begin{array}{lll}
    \{ \} + (x,l) & \triangleq & \{ (x,l)  \}   \\
     \{  [(x_1, l_1), \dots, (x_i, l_i)] , \dots, [(y_1, l_1'), \dots, (y_i,
    l_i')]  \} + (x,l)   & \triangleq & 
    \\
     \{  [(x_1, l_1), \dots, (x_i,
                                       l_i),(x,l)] , \dots, [ (y_1,
                                        l_1'), \dots, (y_i, l_i'),
    (x,l) ]\}  & &
  \end{array}
\end{mathpar}
  \caption{Big-step semantics}
  \label{fig:semantics1}
\end{figure}





%%%%%%%%%%%%%%%%%%%%%%%%%%%%%%%%%%%%%%%%%%%%%%%%%%%%%

%%%%%%%%%%%%%%%%%%%%%%%%%%%%%%%%%%%%%%%%%%%%%%%%%%%%%


% \[
% \begin{array}{llll}
%  % \mbox{Index Term} & \idx, \nnatA & ::= &     i ~|~ n ~|~ \idx_1 + \idx_2 ~|~  \idx_1
%  %                                  - \idx_2 ~|~ \smax{\idx_1}{\idx_2}\\
% %                                  \mbox{Sort} & S & ::= & \nat \\
%   \mbox{Linear type} & \ltype &::=  &  \type \lto \type ~|~ \tbase \\
%   \mbox{Nonlinear Type} & \type & ::= & \bang{\idx} \ltype   \\
% \end{array}
% \]

% \begin{figure}
%   \begin{mathpar}
%     \inferrule{
%     }{
%       \ictx \tctx , x: \bang{\nnatA}\ltype, \Gamma' \tvdash{\nnatA} x: \bang{\nnatA}\ltype
%     }~\textbf{Ax}
%     %
%     \and
%     %
%     \inferrule{
%     }{
%       \ictx \Gamma \tvdash{\nnatA} c : \bang{\nnatA}\tbase 
%     }~\textbf{const}
%     %
%     % \and
%     % %
%     % \inferrule{
%     % }{
%     %   \ictx \Gamma \tvdash{\nnatA} \evec : \bang{\nnatA}\tbase 
%     % }~\textbf{Dict}
%     %
%     \and
%     %
%     \inferrule{
%       \ictx \Gamma, x: \type_1
%       \tvdash{\nnatA }
%       \expr: \type_2
%     }{
%       \ictx k+\Gamma \tvdash{k+\nnatA} \lambda x. \expr : \bang{k}  ( \type_1
%       \lto \type_2)
%     }~\textbf{lambda}
%     \and
%     %
%     \inferrule{
%       \ictx \Gamma_1  \tvdash{\nnatA_1} \expr_1:  \bang{0} ( \type_1
%       \lto \type_2      ) \\
%       \ictx \Gamma_2 \tvdash{\nnatA_2} \expr_2: \type_1 
%     }{
%       \ictx \max (\Gamma_1, \Gamma_2 ) \tvdash{\max( \nnatA_1,\nnatA_2) } \expr_1 \eapp \expr_2 : \type_2
%     }~\textbf{app}
%     %
%     \and
%     %
%     \inferrule{
%       \ictx \Gamma \tvdash{\nnatA} \expr: \bang{k} \ltype 
%     }{
%       \ictx \Gamma' ,1+\Gamma  \tvdash{1+\nnatA} \delta(\expr): \bang{k} \ltype 
%     }~\textbf{delta}
%      %
%     \and
%     %
%     \inferrule{
%       \ictx \Gamma'  \tvdash{\nnatA'} \expr: \type' \\
%       \Gamma' \leqslant \Gamma \\
%       \nnatA' \leq \nnatA\\
%       \sub{\type'}{\type} \\
%       \ictx \Gamma \tvdash{\nnatA} \expr: \bang{k} \ltype 
%     }{
%       \ictx \Gamma  \tvdash{\nnatA} \expr: \type 
%     }~\textbf{subtype}
%       %
%     \and
%     %
%     \inferrule{
%       \ictx \Gamma, y: \type', x: \type ,\Gamma'  \tvdash{\nnatA} \expr: \type 
%     }{
%       \ictx \Gamma, x: \type, y: \type' ,\Gamma'  \tvdash{\nnatA} \expr: \type 
%     }~\textbf{exchange}
%     \\\\
%     \boxed{
%  \inferrule{
%       \ictx \Gamma, x: \type_1
%       \tvdash{\nnatA }
%       \expr: \type_2
%     }{
%       \ictx k+\Gamma \tvdash{k} \lambda x. \expr : \bang{k}  ( \type_1
%       \lto^{\nnatA} \type_2)
%     }~\textbf{lambda}
%     \and
%     %
%     \inferrule{
%       \ictx \Gamma  \tvdash{\nnatA_1} \expr_1:  \bang{0} ( \type_1
%       \lto^{\nnatA} \type_2      ) \\
%       \ictx \Gamma \tvdash{\nnatA_2} \expr_2: \type_1 
%     }{
%       \ictx \Gamma  \tvdash{ \nnatA_1 + \max( \nnatA,\nnatA_2) } \expr_1 \eapp \expr_2 : \type_2
%     }~\textbf{app}
%     }
%     \\\\
% \begin{array}{lll}
%    k+\bang{r} \ltype  &\triangleq  &  \bang{k+r} \ltype  \\
%   k + \emptyset   & \triangleq & \emptyset   \\
%   k + ( [x : \type], \Gamma) & \triangleq &  [x : k+\type], k+\Gamma   
%   \\
%   \max(\bang{k_1} \ltype, \bang{k_2} \ltype) & \triangleq& \bang{ \max(k_1,
%                                                     k_2) } \ltype \\
%   \max(\Gamma, \emptyset) & \triangleq & \Gamma \\
%   \max(\emptyset, \Gamma) & \triangleq & \Gamma \\
%   \max\Big(  ([x : \type ],\Gamma),  ([x: \type'],\Delta)  \Big) & \triangleq
%                             & [x: \max(\type, \type')], \max(\Gamma,
%                               \Delta )\\
%   \sub{\Gamma}{\Delta} & \triangleq &  \dom(\Gamma) = \dom(\Delta)
%                                       \land \forall x \in
%                                       \dom(\Gamma), \sub{\Delta(x)}{\Gamma(x)}  
% \end{array}
%   \end{mathpar}
%   \caption{Typing rules, first version}
%   \label{fig:type-rules1}
% \end{figure}

% \begin{figure}
%   \begin{mathpar}
%     \inferrule{
%       k_1 \leq k \\
%       \sub{\ltype}{\ltype_1}
%     }{
%       \sub{\bang{k} \ltype}{\bang{k_1} \ltype_1}
%     }~\textsf{bang}
%     %
%     \and
%     %
%      \inferrule{
%         \sub{\type_1}{\type}   \\
%       \sub{\type'}{\type_1'}
%     }{
%       \sub{\type \lto \type' }{\type_1 \lto \ltype_1'}
%     }~\textsf{arrow}
%     %
%     \and
%     %
%     \inferrule{
%     }{
%     \sub{\tbase}{\tbase}
%     }~\textsf{base}
%   \end{mathpar}
%   \caption{subtyping}
%  \end{figure}

%  \clearpage

%  \begin{thm}[Weaking]
%   \label{sub}
%   \begin{enumerate} 
%    \item If $ \Gamma,x : \type' \tvdash{ \nnatA} \expr : \type $ and $
%   x \not \in \fv{\expr}  $ ,  then  $ \Gamma \tvdash{ \nnatA} \expr : \type $.
%   \end{enumerate}
% \end{thm}

% \begin{thm}[Value Adaptivity]
%   \label{sub}
%   \begin{enumerate} 
%    \item for all type $\bang{k} \ltype$,  exist value $\valr$, then  $
%      \empty \tvdash{ k} \valr : \bang{k} \ltype $.
%   \end{enumerate}
% \end{thm}

% \begin{thm}[Substitution]
%   \label{sub}
%   \begin{enumerate} 
%    \item If $ \Gamma,x : \type' \tvdash{ \nnatA} \expr : \type $ and $
%   \empty \tvdash{\nnatA'} \valr : \type'  $ , then  $\Gamma
%   \tvdash{\max(\nnatA,\nnatA' )} \expr[\valr/x]  : \type$. 
%   \end{enumerate}
% \end{thm}

% \begin{proof}
%   By induction on the typing derivation.\\
% \caseL{
%   $   \inferrule{
%     }{
%       \ictx \tctx , x: \bang{\nnatA}\ltype \tvdash{\nnatA} x: \bang{\nnatA}\ltype
%     }~\textbf{Ax}  $
%   }
% Assume $\empty \tvdash{\nnatA'} \valr : \bang{\nnatA}\ltype $, TS:  $\Gamma
%   \tvdash{\max(\nnatA,\nnatA' )} x[\valr/x]  : \type$. proved by
%   subtype rule on the assumption.
% \caseL{
%  $   \inferrule{
%     }{
%       \ictx \tctx ,y:\type', x: \bang{\nnatA}\ltype \tvdash{\nnatA} x: \bang{\nnatA}\ltype
%     }~\textbf{Ax2}  $
%   }
%   Assume $\empty \tvdash{\nnatA'} \valr : \bang{\nnatA}\ltype $, TS:
%   $\Gamma,   x: \bang{\nnatA}\ltype
%   \tvdash{\max(\nnatA,\nnatA' )} x[\valr/y]  : \type$. proved by rule
%   AX and then subtype.
%   \caseL{
%    \inferrule{
%       \ictx \Gamma, x: \type_1, y:\type'
%       \tvdash{\nnatA }
%       \expr: \type_2
%     }{
%       \ictx k+\Gamma, y: k + \type' \tvdash{k+\nnatA} \lambda x. \expr : \bang{k}  ( \type_1
%       \lto \type_2)
%     }~\textbf{lambda}
%   }
%    Assume $\empty \tvdash{k+\nnatA'} \valr : k+\type' $, TS:
%   $k+\Gamma
%   \tvdash{\max(k+\nnatA,k+\nnatA' )} (\lambda x. \expr)[\valr/y]  : \type$. From the
%   Lemma~\ref{para-dec} on the assumption, we know: $\empty
%   \tvdash{\nnatA'} \valr : \type' ~(1)$.\\
%   By Induction hypothesis on the premise, we get: $ \Gamma, x:\type_1
%   \tvdash{\max( \nnatA, \nnatA' )}
%       \expr[\valr/y]: \type_2 ~(2)$. By rule lambda, we conclude that
%       $k+\Gamma \tvdash{ k+ ( \max(\nnatA,\nnatA ) }
%       \lambda x.\expr[\valr/y]: \type_2 $.
%       \caseL{
%       \inferrule{
%       \ictx \Gamma_1,x:\type'  \tvdash{\nnatA_1} \expr_1:  \bang{0} ( \type_1
%       \lto \type_2      ) \\
%       \ictx \Gamma_2 ,x: \type'', \tvdash{\nnatA_2} \expr_2: \type_1 
%     }{
%       \ictx \max (\Gamma_1, \Gamma_2 ), x:\max(\type',\type'') \tvdash{\max( \nnatA_1,\nnatA_2) } \expr_1 \eapp \expr_2 : \type_2
%     }~\textbf{app}
%   }
%   Assume $\empty \tvdash{\nnatA'} \valr : \max(\type',\type'')$, TS: $\max (\Gamma_1, \Gamma_2 )
%   \tvdash{\max(\nnatA_1,\nnatA_2, \nnatA' )} (\expr_1 \eapp
%   \expr_2)[\valr/x]  : \type_2$. From the definition of $\max(\type',
%   \type'')$, we know that $\type'$ and $\type''$ have similar
%   form. Let us assume $\type'= \bang{k_1} \ltype$ and $\type'' =
%   \bang{k_2} \ltype$ so that $\max(\type',\type'') = \bang{\max(k_1,k_2)}
%   \ltype$.\\
%   From the Lemma~\ref{para-dec} on the assumption, we have $\empty
%   \tvdash{\nnatA' - (\max(k_1,k_2)-k_1) } \valr : \bang{k_1}
%   \ltype~(1)$ and $\empty
%   \tvdash{\nnatA' - (\max(k_1,k_2)-k_2) } \valr : \bang{k_2}
%   \ltype~(2)$.\\ By induction hypothesis on $(1)$ and $(2)$ respctively,
%   we know that:  $ \Gamma_1  \tvdash{ \max( \nnatA_1, \nnatA' - (\max(k_1,k_2)-k_1) ) } \expr_1[\valr/x]:  \bang{0} ( \type_1
%   \lto \type_2   ) ~(3)$  and $ \Gamma_2  \tvdash{\max(\nnatA_2 ,
%     \nnatA' - (\max(k_1,k_2)-k_2)   )} \expr_2[\valr/x]: \type_1 ~(4)$.  By the
%   rule app and $(3)$, $(4)$, we conclude that $$\max (\Gamma_1, \Gamma_2 )
%   \tvdash{\max(  \max( \nnatA_1, \nnatA' - (\max(k_1,k_2)-k_1) )  , \max(\nnatA_2 ,
%     \nnatA' - (\max(k_1,k_2)-k_2)   )  )} \expr_1[\valr/x] \eapp
%   \expr_2[\valr/x]  : \type_2 ~(5).$$
%   Because $\max(\nnatA' - (\max(k_1,k_2)-k_1) ) , \nnatA' -
%   (\max(k_1,k_2)-k_2)   ) \leq \nnatA' $, by subtype, we raise the
%   adaptivity to  $\max(\nnatA_1,\nnatA_2, \nnatA' ) $ from $(5)$.
%    \caseL{
%       \inferrule{
%       \ictx \Gamma_1,x:\type'  \tvdash{\nnatA_1} \expr_1:  \bang{0} ( \type_1
%       \lto \type_2      ) \\
%       \ictx \Gamma_2  \tvdash{\nnatA_2} \expr_2: \type_1 
%     }{
%       \ictx \max (\Gamma_1, \Gamma_2 ), x:\type' \tvdash{\max( \nnatA_1,\nnatA_2) } \expr_1 \eapp \expr_2 : \type_2
%     }~\textbf{app2}
%   }
%   It is another case for application when x only appear in the first
%   premise. In this case, $\expr_2[\valr/x] = \expr_2$. Another case
%   when variable x only appears in the second premise can be proved in
%   a similar way.\\
%   Assume $\empty \tvdash{\nnatA'} \valr :\type'$. TS:$\max (\Gamma_1, \Gamma_2 )
%   \tvdash{\max(\nnatA_1,\nnatA_2, \nnatA' )} (\expr_1 \eapp
%   \expr_2)[\valr/x]  : \type_2$.  By Induction Hypothesis on the first
%   premise using the assumption, we get: $\Gamma_1
%   \tvdash{\max(\nnatA_1, \nnatA')} \expr_1[\valr/x]:  \bang{0} ( \type_1
%       \lto \type_2  )  ~(1)$. By the rule app using (1) and the second
%       premise, we conclude that $$ \max (\Gamma_1, \Gamma_2 )
%       \tvdash{\max( \max(\nnatA_1,\nnatA'),\nnatA_2) }
%       \expr_1[\valr/x] \eapp \expr_2 : \type_2$$
%       \caseL{
%  \inferrule{
%       \ictx \Gamma, x:\type' \tvdash{\nnatA} \expr: \bang{k} \ltype 
%     }{
%       \ictx \Gamma' ,1+\Gamma, x:1+\type'  \tvdash{1+\nnatA} \delta(\expr): \bang{k} \ltype 
%     }~\textbf{delta}
%   }
%   Assume $\empty \tvdash{\nnatA'+1} \valr : 1+\type' $, TS: $ \Gamma'
%   ,1+\Gamma \tvdash{\max(1+\nnatA, 1+\nnatA')} \delta(\expr)
%   [\valr/x]: \bang{k} \ltype $.
%   By Lemma~\ref{para-dec} on the assumption, we have $\empty
%   \tvdash{\nnatA'} \valr : \type'~(1) $. By IH on the first premise
%   along with (1), we have: $\Gamma \tvdash{\max(\nnatA, \nnatA')}
%   \expr[\valr/x]: \bang{k} \ltype~ (2)$.
%    By the rule delta using (2), we conclude that $\Gamma' ,1+\Gamma  \tvdash{1+(\nnatA,\nnatA')} \delta(\expr[\valr/x]): \bang{k} \ltype$.
% \end{proof}



% \begin{proof}
%   By Induction on the typing derivation.
%   \caseL{
%      $   \inferrule{
%     }{
%       \ictx \tctx , x: \bang{\nnatA}\ltype \tvdash{\nnatA} x: \bang{\nnatA}\ltype
%     }~\textbf{Ax}  $
%   }
%   Assume $\env= \Big( \env_1, [x \to (\valr,\adapt
%   )] , \Big) \vDash (\tctx , x: \bang{\nnatA}\ltype  )$ where $\env_1 \vDash \Gamma$. We know that $
%   \empty \tvdash{\adapt} \valr : \bang{\nnatA}\ltype $.
%   From the evaluation rule var, we know $\env , x \bigstep{\adapt} \valr,
%   \env  $.
%   TS:  $ \adapt + adap(\valr, \env)  \leq  \nnatA +
%   F(\env) \implies \adapt + 0 \leq \nnatA + \max( \adapt, F(\env_1))
%   $.It is trivially true.
% \caseL{
%   $
%     \inferrule{
%       \ictx \Gamma, x: \type_1
%       \tvdash{\nnatA }
%       \expr: \type_2
%     }{
%       \ictx k+\Gamma \tvdash{k+\nnatA} \lambda x. \expr : \bang{k}  ( \type_1
%       \lto \type_2)
%     }~\textbf{lambda}
%   $
% }
% Choose $\env \vDash  (k+\Gamma)$ so that $\forall x_i \in
% (\Gamma). \env(x_1) =(\valr_i, \adapt_i ) \land \empty
% \tvdash{\adapt_i } \valr_i: k+\Gamma(x_i) $.  By the evaluation rule
% we know $\env, \lambda x. \expr \bigstep{0}
%                                        \lambda x.\expr, \env $, TS: $0
%                                        + \adap(\lambda x.\expr, \env)
%                                        \leq  k+\nnatA + F(\env)$, which is trivially
%                                        true because $ \adap(\lambda
%                                        x.\expr, \env) \leq F(\env) $.
                                       
% \caseL{
%     $  \inferrule{
%       \ictx \Gamma_1  \tvdash{\nnatA_1} \expr_1:  \bang{0} ( \type_1
%       \lto \type_2      ) \\
%       \ictx \Gamma_2 \tvdash{\nnatA_2} \expr_2: \type_1 
%     }{
%       \ictx \max (\Gamma_1, \Gamma_2 ) \tvdash{\max( \nnatA_1,\nnatA_2) } \expr_1 \eapp \expr_2 : \type_2
%     }~\textbf{app}  $
%   }
%   Choose $\env = [x_i \to (\valr_i,0)] $ for all $x_i$ in
%   $\dom(\max(\Gamma_1,\Gamma_2))$
%   so that  $\empty \tvdash{\nnatA_i} \valr_i  : (\max(\Gamma_1,
%   \Gamma_2)(x_i) $.
%   From the definition, we know that $\env \vDash \Gamma_1$ and $\env
%   \vDash \Gamma_2$. Because $\expr_1$ has the arrow type and will be
%   evaluated to a function, assume exists $\env_1$ so that $\env,
%   \expr_1 \bigstep{\adapt_1} \lambda x.\expr , \env_1 $.  By induction
%   hypothesis on the first premise, we know that: $\adapt_1 +
%   \adap(\lambda x. \expr, \env_1) \leq \nnatA_1 + F(\env,
%   \Gamma_1)~(1)$.Assume exists $\env_2$ so that $\expr_2$ is evaluated
%   to an arbitrary value $\valr_2$ : $ \env, \expr_2 \bigstep{\adapt_2}
%   \valr_2 , \env_2$, by induction hypothesis, we conclude that :  $\adapt_2 +
%   \adap(\valr , \env_2) \leq \nnatA_2 + F(\env,
%   \Gamma_2)~(2)$.
                            


% \[
% \inferrule{
%     \env, \expr_1 \bigstep{\adapt_1} \lambda x.\expr , \env_1 \\
%     \env, \expr_2 \bigstep{\adapt_2} \valr_2 , \env_2 \\
%     (\env_1 \uplus \env_2)[ x  \to (\valr_2,   \adapt_2  ) ], \expr
%     \bigstep{\adapt_3} \valr, \env_3
%   }{
%     \env, \expr_1 \eapp \expr_2 \bigstep{\adapt_1+\adapt_3} \valr, \env_3
%   }~\textsf{app}
% \]
%  \end{proof} 


\end{document}



