\[\begin{array}{llll}
\mbox{Expr.} & \expr & ::= & x ~|~ \expr_1 \eapp \expr_2 ~|~ {\efix f(x:\type).\expr}
 ~|~ (\expr_1, \expr_2) ~|~ \eprojl(\expr) ~|~ \eprojr(\expr) ~| \\
%
& & & \etrue ~|~ \efalse ~|~ \eif(\expr_1, \expr_2, \expr_3) ~|~
\econst ~|~ \eop(\expr)  ~|~  {\eilam \expr ~|~ \expr \eapp [] } \\
& & & ~|~  {\elet  x:q = \expr_1 \ein \expr_2 } ~|~ \enil ~|~  \econs (
      \expr_1, \expr_2) \\
& & & ~|~ { ~~~~~~~
 \bernoulli \eapp \expr ~|~ \uniform \eapp \expr_1 \eapp
      \expr_2 } \\
%
\mbox{Value} & \valr & ::= & \etrue ~|~ \efalse ~|~ \econst ~|~
(\efix f(x:\type).\expr, \env) ~|~ (\valr_1, \valr_2) 
    ~|~ \enil ~|~ \econs (\valr_1, \valr_2) | \\
& & &  {(\eilam \expr , \env) } \\ 
%
\mbox{Environment} & \env & ::= & x_1 \mapsto \valr_1, \ldots, x_n \mapsto \valr_n
\end{array}\]


%%%%%%%%%%%%%%%%%%%%%%%%%%%%%%%%%%%%%%%%%%%%%%%%%%%%%

%%%%%%%%%%%%%%%%%%%%%%%%%%%%%%%%%%%%%%%%%%%%%%%%%%%%%


\subsection{Tracing operational semantics and adaptivity}

\paragraph{Traces}
A trace $\tr$ is a representation of the big-step derivation of an
expression's evaluation. Our big-step semantics output a trace. We use
traces to define the adaptivity of a run. Our notion of traces and the
tracing semantics is taken from~\cite[Section 4]{perera:dep}, but we
omit their ``holes'' for which we have no need. The construct
$\trapp{\tr_1}{\tr_2}{f}{x}{\tr_3}$ records a trace of function
application. $\tr_1$ is the trace of the head, $\tr_2$ the trace of
the argument and $\tr_3$ is the trace of the function body. $f$ and
$x$ are bound in $\tr_3$.
%
\[\begin{array}{llll}
\mbox{Trace} & \tr & ::= & {(x, \env)} ~|~ \trapp{\tr_1}{\tr_2}{f}{x}{\tr_3} ~|~
{ (\trfix f(x:\type).e, \env) } ~|~ (\tr_1, \tr_2) ~|~ \trprojl(\tr) ~|\\ 
%
& & & \trprojr(\tr) ~|~ \trtrue ~|~ \trfalse ~|~ \trift(\tr_b, \tr_t)
~|~ \triff(\tr_b, \tr_f) ~|~ \trconst ~|~ \trop(\tr) \\
%
& & & \trnil ~|~ \trcons (\tr_1, \tr_2) ~|~ \triapp{\tr_1}{\tr_2} ~|~
       {(\eilam \expr, \env)}
\end{array}\]


\paragraph{Big-step tracing semantics}
The big-step, tracing semantics $\env, \expr \bigstep \valr, \tr$
computes a value $\valr$ and a trace $\tr$ from an expression $\expr$
and an enviroment $\env$ which maps the free variables of $\expr$ to
\emph{closed} values. The rules, taken from~\cite{perera:dep}, are
shown in Figure~\ref{fig:big-step}. Some salient points:
\begin{itemize}
\item[-] Erasing the traces from the semantics yields a standard
  big-step semantics.
\item[-] The trace of a primitive application $\eop(\expr)$
  records that $\eop$ was applied to the trace of
  $\expr$. This enables us to define adaptivity from a trace later.
\item[-] The trace of a variable $x$ is $x$. This way traces record
  where substitutions occur and, hence, variable dependencies. This is
  also needed for defining adaptivity.
\end{itemize}

\begin{figure}
\begin{mathpar}
   { \inferrule{ }{\env, x \bigstep \env(x), (x, \env ) }  }
  %
  \and
  %
  \inferrule{ }{\env, \econst \bigstep \econst, \trconst}
  %
  \and
  %
  \inferrule{ }{\env, \etrue \bigstep \etrue, \trtrue}
  %
  \and
  %
  \inferrule{ }{\env, \efalse \bigstep \efalse, \trfalse}
  %
  \and
  { \inferrule{  \env, \expr \bigstep \econst, \tr }{\env, \bernoulli \eapp \expr \bigstep \econst,
      \bernoulli (\tr)
    } }
  \and
 \inferrule{ \env, \expr_1 \bigstep \econst, \tr_1 \\ \env, \expr_2 \bigstep \econst, \tr_2  }{\env, \uniform \eapp \expr_1 \eapp
      \expr_2\bigstep \econst, \uniform(\tr_1,\tr_2)  } 
  \and
  %
  { \inferrule{
  }{
    \env, \efix f(x:\type). \expr \bigstep (\efix f(:\type).\expr, \env),
    (\trfix f(x:\type).\expr, \env)
  }
}
  %
  \and
  %
  \inferrule{
    \env, \expr_1 \bigstep \valr_1, \tr_1 \\
    { \valr_1 = (\efix f(x:\type).\expr, \env')} \\\\
    \env, \expr_2 \bigstep \valr_2, \tr_2 \\
    \env'[f \mapsto \valr_1, x \mapsto \valr_2], \expr \bigstep \valr, \tr
  }{
    \env, \expr_1 \eapp \expr_2 \bigstep \valr, \trapp{\tr_1}{\tr_2}{f}{x}{\tr}
  }
  %
  \and
  %
  \inferrule{
    \env, \expr_1 \bigstep \valr_1, \tr_1 \\
    \env, \expr_2 \bigstep \valr_2, \tr_2
  }{
    \env, (\expr_1, \expr_2) \bigstep (\valr_1, \valr_2), (\tr_1, \tr_2)
  }
  %
  \and
  %
  \inferrule{
    \env, \expr \bigstep (\valr_1, \valr_2), \tr
  }{
    \env, \eprojl(\expr) \bigstep \valr_1, \trprojl(\tr)
  }
  %
  \and
  %
  \inferrule{
    \env, \expr \bigstep (\valr_1, \valr_2), \tr
  }{
    \env, \eprojr(\expr) \bigstep \valr_2, \trprojr(\tr)
  }
  %
  \and
  %
  \inferrule{
    \env, \expr \bigstep \etrue, \tr \\
    \env, \expr_1 \bigstep \valr, \tr_1
  }{
    \env, \eif(\expr, \expr_1, \expr_2) \bigstep \valr, \trift(\tr, \tr_1)
  }
  %
  \and
  %
  \inferrule{
    \env, \expr \bigstep \efalse, \tr \\
    \env, \expr_2 \bigstep \valr, \tr_2
  }{
    \env, \eif(\expr, \expr_1, \expr_2) \bigstep \valr, \triff(\tr, \tr_2)
  }
  %
  \and
  %
  \inferrule{
    \env, \expr \bigstep \valr, \tr \\
    \eop{}(\valr) = \valr'
  }{
    \env, \eop(\expr) \bigstep \valr', \trop(\tr)
  }
%
\and
%
  \inferrule{
}
{ \env, \enil \bigstep \enil, \trnil }
%
\and
%
\inferrule{
\env, \expr_1 \bigstep \valr_1, \tr_1 \\
\env, \expr_2 \bigstep \valr_2, \tr_2
}
{ \env, \econs (\expr_1, \expr_2)  \bigstep \econs (\valr_1, \valr_2),
  \trcons(\tr_1, \tr_2)
}
%
\and
%
\inferrule{
  \env, \expr_1 \bigstep \valr_1, \tr_1 \\
  \env[x \mapsto \valr_1] , \expr_2 \bigstep \valr, \tr_2
}
{\env, \elet x;q = \expr_1 \ein \expr_2 \bigstep \valr, \trlet (x,
  \tr_1, \tr_2) }
%
\\\\
%
\boxed{\color{red}
\inferrule
{
  \empty
}
{
  \env, \eilam \expr \bigstep (\eilam \expr, \env), (\eilam \expr , \env)
}
}
%
\and
%
\boxed{\color{red}
\inferrule{
  \env, \expr \bigstep (\eilam \expr', \env'), \tr_1 \\
  \env, \expr' \bigstep \valr, \tr_2
}
{\env, \expr [] \bigstep \valr, \triapp{\tr_1}{\tr_2} }

}
\end{mathpar}
  \caption{Big-step semantics with provenance}
  \label{fig:big-step}
\end{figure}


\paragraph{Adaptivity of a trace}
We define the \emph{adaptivity} of a trace $\tr$, $\adap(\tr)$, which
means the maximum number of nested $\eop$s in $\tr$, taking variable
and control dependencies into account. To define this, we need an
auxiliary notion called the \emph{depth of variable $x$} in trace
$\tr$, written $\ddep{x}(\tr)$, which is the maximum number of
$\eop{}$s in any path leading from the root of $\tr$ to an occurence
of $x$ (at a leaf), again taking variable and control dependencies
into account. Technically, $\adap: \mbox{Traces} \to \nat$ and
$\ddep{x}: \mbox{Traces} \to \natb$. If $x$ does not appear free in
$\tr$, $\ddep{x}(\tr)$ is $\bot$.

The functions $\adap$ and $\ddep{x}$ are defined by mutual induction
in Figure~\ref{fig:adap}. 

\begin{figure}
  \framebox{$\adap: \mbox{Traces} \to \nat$}
  \begin{mathpar}
    \begin{array}{lcl}
       { \adap( (x,\env) )} & = & 0 \\
      %
      \adap(\trapp{\tr_1}{\tr_2}{f}{x}{\tr_3}) & = &
      \adap(\tr_1) + \max (\adap(\tr_3), \adap(\tr_2) + \ddep{x}(\tr_3))\\
      %
       {\adap( (\trfix f(x:\type).\expr, \env)  ) } & = & 0 \\
      %
      \adap((\tr_1, \tr_2)) & = & \max(\adap(\tr_1), \adap(\tr_2)) \\
      %
      \adap(\trprojl(\tr)) & = & \adap(\tr) \\
      %
      \adap(\trprojr(\tr)) & = & \adap(\tr) \\
      %
      \adap(\trtrue) & = & 0 \\
      %
      \adap(\trfalse) & = & 0 \\
      %
      \adap(\trift(\tr_b, \tr_t)) & = & \adap(\tr_b) + \adap(\tr_t) \\
      %
      \adap(\triff(\tr_b, \tr_f)) & = & \adap(\tr_b) + \adap(\tr_f) \\
      %
      \adap(\trconst) & = & 0 \\
      %
      \adap(\trop(\tr)) & = & { 1 + \adap(\tr) } \\
           & &       {  +  \textsf{MAX}_{\valr \in \type} \Big(
                              \max \big(\adap(\tr_3 (\valr) ),
                              \ddep{x}(\tr_3(\valr)) \big) \Big) } \\
      &\mathsf{where}&  { \valr_1 = (\efix f(x: \type). \expr, \env ) =
                       \mathsf{extract}(\tr) } \\
 & & { \conj  \env[f \mapsto
                       \valr_1, x \mapsto \valr], \expr \bigstep
                       \valr', \tr_3(\valr) } \\ 
      %
     \adap(\trnil) & = & 0 \\
     %
     \adap(\trcons(\tr_1,\tr_2) ) & = &  \max(\adap(\tr_1),
                                        \adap(\tr_2)) \\
     %
    \adap( \trlet (x, \tr_1,\tr_2) ) & = & \max (\adap(\tr_2),
                                           \adap(\tr_1)+\ddep{x}(\tr_2)  )
                                           \\
     \adap(\triapp{\tr_1}{\tr_2}) & = & \adap(\tr_1) + \adap(\tr_2)\\
    %
     { \adap( (\eilam \expr, \env) ) } & = & 0 \\
     { \adap( \bernoulli (\tr)  ) } & = & \adap(\tr) \\
      { \adap( \uniform  (\tr_1, \tr_2)  ) } & = & \max (\adap(\tr_1),
                                                      \adap(\tr_2) ) \\
      \end{array}
  \end{mathpar}
  %
  \framebox{$\ddep{x}: \mbox{Traces} \to \natb$}
  \begin{mathpar}
    \begin{array}{lcl}
       { \ddep{x}( ( y, \env )) } & = &
      \left\lbrace
      \begin{array}{ll}
        0 & \mbox{if } x = y \\
        \bot & \mbox{if } x \neq y
      \end{array}
      \right.\\
      %
      \ddep{x}(\trapp{\tr_1}{\tr_2}{f}{y}{\tr_3}) & = & \max(\ddep{x}(\tr_1), \\
      & & \adap(\tr_1) + \max(\ddep{x}(\tr_3), \ddep{x}(\tr_2) + \ddep{y}(\tr_3))) \\
      %
      { \ddep{x}(  (\trfix f(y:\type).\expr,\env)  )  }& = & \bot \\
      %
      \ddep{x}((\tr_1, \tr_2)) & = & \max(\ddep{x}(\tr_1), \ddep{x}(\tr_2)) \\
      %
      \ddep{x}(\trprojl(\tr)) & = & \ddep{x}(\tr) \\
      %
      \ddep{x}(\trprojr(\tr)) & = & \ddep{x}(\tr) \\
      %
      \ddep{x}(\trtrue) & = & \bot \\
      %
      \ddep{x}(\trfalse) & = & \bot \\
      %
      \ddep{x}(\trift(\tr_b, \tr_t)) & = & \max(\ddep{x}(\tr_b), \adap(\tr_b) + \ddep{x}(\tr_t)) \\
      %
      \ddep{x}(\trift(\tr_b, \tr_f)) & = & \max(\ddep{x}(\tr_b), \adap(\tr_b) + \ddep{x}(\tr_f)) \\
      %
      \ddep{x}(\trconst) & = & \bot \\
      %
      \ddep{x}(\trop(\tr)) & = & 1 +  \max(\ddep{x}(\tr),  \\
      & &  \adap(\tr) + \textsf{MAX}_{\valr \in \type} \Big(
          \max(\ddep{x}(\tr_3(\valr)), \bot )   \Big ) ) \\  
 &\mathsf{where}&  { \valr_1 = (\efix f(x: \type). \expr, \env ) =
                       \mathsf{extract}(\tr) } \\
 & & { \conj  \env[f \mapsto
                       \valr_1, x \mapsto \valr], \expr \bigstep
                       \valr', \tr_3(\valr) } \\ 
       %
      \ddep{x}(\trnil) & = & \bot \\
      %
      \ddep{x}(\trcons(\tr_1,\tr_2) ) & = & \max(\ddep{x}(\tr_1),
                                            \ddep{x}(\tr_2)) \\
      %
      \ddep{x}( \trlet(y, \tr_1, \tr_2) ) & = & \max( \ddep{x}(\tr_2),
                                                \ddep{x}(\tr_1)+\ddep{y}(\tr_2)  )\\
       \ddep{x}(\triapp{\tr_1}{\tr_2})  & = & 
                                                    \max(\ddep{x}(\tr_1), \adap(\tr_1) + \ddep{x}(\tr_2))\\
    %
     { \ddep{x}( (\eilam \expr, \env) ) } & = & \bot \\
    \ddep{x}(\uniform (\tr_1,\tr_2) ) & = & \max(\ddep{x}(\tr_1),
                                            \ddep{x}(\tr_2)) \\
  \ddep{x}(\bernoulli (\tr)) & = & \ddep{x}(\tr)
    \end{array}
  \end{mathpar}
  \caption{Adaptivity of a trace and depth of variable $x$ in a trace}
  \label{fig:adap}
\end{figure}

\subsection{Challenge (Couterexample)}
