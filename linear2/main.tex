\documentclass[a4paper,11pt]{article}

\usepackage{mathpartir}
\usepackage{amsmath,amsthm,amsfonts}
\usepackage{ amssymb }
\usepackage{color}
\usepackage{algorithm}
\usepackage{algorithmic}
\usepackage{microtype}


\newcommand{\defeq}{\mathrel{\doteq}}

\newcommand{\lzero}{0}

\newcommand{\kw}[1]{\mathtt{#1}}

\newcommand{\expr}{e}
\newcommand{\vall}{w}
\newcommand{\valr}{v}
\newcommand{\eif}{\kw{if}}
\newcommand{\eapp}{\;}
\newcommand{\eprojl}{\kw{fst}}
\newcommand{\eprojr}{\kw{snd}}
%\newcommand{\eprov}[1]{\eta_{#1}}
\newcommand{\etrue}{\kw{true}}
\newcommand{\efalse}{\kw{false}}
\newcommand{\econst}{c}
\newcommand{\eop}{\delta}
\newcommand{\efix}{\mathop{\kw{fix}}}
%\newcommand{\labelA}{\ell}

\newcommand{\tr}{T}
\newcommand{\trift}{\eif^{\kw{t}}}
\newcommand{\triff}{\eif^{\kw{f}}}
\newcommand{\trprojl}{\eprojl}
\newcommand{\trprojr}{\eprojr}
\newcommand{\trtrue}{\etrue}
\newcommand{\trfalse}{\efalse}
\newcommand{\trconst}{\econst}
\newcommand{\trop}{\eop}
\newcommand{\trfix}{\efix}
\newcommand{\trapp}[5]{#1 \; #2 \mathrel{\triangleright} {\efix #3(#4).#5}}

\newcommand{\adap}{\kw{adap}}
\newcommand{\ddep}[1]{\kw{depth}_{#1}}
\newcommand{\nat}{\mathbb{N}}
\newcommand{\natb}{\nat_{\bot}}
\newcommand{\natbi}{\natb^\infty}
\newcommand{\nnatA}{n}
\newcommand{\nnatB}{m}
\newcommand{\nnatbA}{s}
\newcommand{\nnatbB}{t}
\newcommand{\nnatbiA}{q}
\newcommand{\nnatbiB}{r}

\newcommand{\type}{\tau}
\newcommand{\tbase}{\kw{b}}
\newcommand{\tbool}{\kw{bool}}
\newcommand{\tarr}[5]{#1; #3 \xrightarrow{#4; \, #5} #2}
\newcommand{\env}{\theta}

\newcommand{\bigstep}{\mathrel{\Downarrow}}

\newcommand{\dmap}{\rho}
\newcommand{\dmapb}{\bot_\dmap}
\newcommand{\supp}{\kw{supp}}
\newcommand{\dom}{\kw{dom}}

\newcommand{\tvdash}[1]{\vdash_{#1}}

\newcommand{\mg}[1]{\textcolor[rgb]{.90,0.00,0.00}{[MG: #1]}}
\newcommand{\dg}[1]{\textcolor[rgb]{0.00,0.5,0.5}{[DG: #1]}}
\newcommand{\wq}[1]{\textcolor[rgb]{.50,0.0,0.7}{ #1}}

\let\originalleft\left
\let\originalright\right
\renewcommand{\left}{\mathopen{}\mathclose\bgroup\originalleft}
\renewcommand{\right}{\aftergroup\egroup\originalright}

\theoremstyle{definition}
\newtheorem{thm}{Theorem}
\newtheorem{lem}[thm]{Lemma}
\newtheorem{cor}[thm]{Corollary}
\newtheorem{prop}[thm]{Proposition}
\newtheorem{defn}[thm]{Definition}

\title{Adaptivity analysis}

\author{}

\date{}

\begin{document}

\maketitle

% \begin{abstract}
% An adaptive data analysis is based on multiple queries over a data set, in which some queries rely on the results of some other queries. The error of each query is usually controllable and bound independently, but the error can propagate through the chain of different queries and bring to high generalization error. To address this issue, data analysts are adopting different mechanisms in their algorithms, such as Gaussian mechanism, etc. To utilize these mechanisms in the best way one needs to understand the depth of chain of queries that one can generate in a data analysis. In this work, we define a programming language which can provide, through its type system, an upper bound on the adaptivity  depth (the length of the longest chain of queries) of a program implementing an adaptive data analysis. We show how this language can help to analyze the generalization error of two data analyses with different adaptivity structures.
% \end{abstract}


% \section{Everything Else}

% \paragraph{Adaptivity}
% Adaptivity is a measure of the nesting depth of a mechanism. To
% represent this depth, we use extended natural numbers. Define $\natb =
% \nat \cup \{\bot\}$, where $\bot$ is a special symbol and $\natbi =
% \natb \cup \{\infty\}$. We use $\nnatA, \nnatB$ to range over $\nat$,
% $\nnatbA, \nnatbB$ to range over $\natb$, and $\nnatbiA, \nnatbiB$ to
% range over $\natbi$.

% The functions $\max$ and $+$, and the order $\leq$ on natural numbers
% extend to $\natbi$ in the natural way:
% \[\begin{array}{lcl}
% \max(\bot, \nnatbiA) & = & \nnatbiA \\
% \max(\nnatbiA, \bot) & = & \nnatbiA \\
% \max(\infty, \nnatbiA) & = & \infty \\
% \max(\nnatbiA, \infty) & = & \infty \\
% \\
% %
% \bot + \nnatbiA & = & \bot \\
% \nnatbiA + \bot & = & \bot \\
% \infty + \nnatbiA & = & \infty ~~~~ \mbox{if } \nnatbiA \neq \bot \\
% \nnatbiA + \infty & = & \infty ~~~~ \mbox{if } \nnatbiA \neq \bot \\
% \\
% %
% \bot \leq \nnatbiA \\
% \nnatbiA \leq \infty
% \end{array}
% \]
% One can think of $\bot$ as $-\infty$, with the special proviso that,
% here, $-\infty + \infty$ is specifically defined to be $-\infty$.

% \paragraph{Language}
% Expressions are shown below. $\econst$ denotes constants (of some base
% type $\tbase$, which may, for example, be reals or rational
% numbers). $\eop$ represents a primitive operation (such as a
% mechanism), which determines adaptivity. For simplicity, we assume
% that $\eop$ can only have type $\tbase \to \tbool$. We make
% environments explicit in closures. This is needed for the tracing
% semantics later.
\[\begin{array}{llll}
\mbox{Expr.} & \expr & ::= & x ~|~ \expr_1 \eapp \expr_2 
 ~|~ \lambda x. \expr% ~|~ \eprojl(\expr) ~|~ \eprojr(\expr) ~|
    \\
%
             & & & % \etrue ~|~ \efalse ~|~
  \eif(\expr_1, \expr_2, \expr_3) ~|~
\econst ~|~ \eop(\expr)  % ~|~ \wq {\eilam \expr ~|~ \expr \eapp [] }
    \\
% & & & ~|~ \wq {\elet  x = \expr_1 \ein \expr_2 } ~|~ \enil ~|~  \econs (
%       \expr_1, \expr_2) \\
% & & & ~|~ \wq{ ~~~~~~~
%  \bernoulli \eapp \expr ~|~ \uniform \eapp \expr_1 \eapp
%       \expr_2 } ~|~  \wq{ \evec({\attr_i \to \expr_i'}^{ i \in 1\dots n})    }  \\
%
\mbox{Value} & \valr & ::= & \econst ~|~ \lambda x. \expr
% (\efix f(x:\type).\expr, \env) ~|~ (\valr_1, \valr_2) 
%     ~|~ \enil ~|~ \econs (\valr_1, \valr_2) |
    \\
% & & & \wq {(\eilam \expr , \env) } ~|~  \wq{ \evec({\attr_i \to \valr_i'}^{ i \in 1\dots n})    } \ \\ 
%
 % \mbox{Adaptivity} & \adapt& ::= & n\\
\mbox{Environment} & \env & ::= & x_1 \mapsto (\valr_1, \adapt_1), \ldots, x_n \mapsto (\valr_n,\adapt_n)
\end{array}\]





%%%%%%%%%%%%%%%%%%%%%%%%%%%%%%%%%%%%%%%%%%%%%%%%%%%% sementics

%%%%%%%%%%%%%%%%%%%%%%%%%%%%%%%%%%%%%%%%%%%%%%%%%%%%% 

\begin{figure}
  \begin{mathpar}
    \inferrule{
    }{
     \valr, \env \bigstep{0} \valr, \env} ~\textsf{val}
    \and
   \inferrule{  \mathsf{fetch} (\env,x)  =  (\valr, \adapt)  }{x, \env
     \bigstep{\adapt} \valr, \env }~\textsf{var}
   \and
   %
     \inferrule{  \env(x)  =  (\valr, \env_1,  \adapt)  }{x,
       \env  \bigstep{\adapt} \valr, \env_1 }~\textsf{var2}
     %
  %
  % \and
  % %
  % \inferrule{ }{\env, \etrue \bigstep{0} \etrue}
  % %
  % \and
  % %
  % \inferrule{ }{\env, \efalse \bigstep{0} \efalse}
  % %
 %  \and
 % \inferrule{  \env, \expr \bigstep{K} \econst }{\env, \bernoulli \eapp \expr \bigstep{K} \econst
 %    }~\textsf{bernoulli} 
 %  \and
 % \inferrule{ \env, \expr_1 \bigstep{R} \econst \\ \env, \expr_2 \bigstep{S} \econst  }{\env, \uniform \eapp \expr_1 \eapp
 %      \expr_2\bigstep{R+S} \econst  } ~\textsf{uniform}
 %  \and
 %
   \and
  %
   \inferrule{ }{\econst , \env \bigstep{0} \econst, \env}~\textsf{const}
   %
   \and
   %
 \inferrule{
  }{
    \lambda x. \expr, \env
    \bigstep{0} \lambda x.\expr, \env
  }~\textsf{lambda}
  %
  \and
  %
  \inferrule{
    \expr_1, \env_1 \bigstep{\adapt_1} \lambda x.\expr , \env_1' \\
    %\forall x_i \in \dom(\env_1 \cap \env_2).  \fresh \eapp x_i' \\
     \expr_2, \env_2 \bigstep{\adapt_2} \valr_2 , \env_2' \\
    \fresh \eapp x' \\
    \expr[x'/x], \env_1'[ x'  \to (\valr_2, \env_2', \adapt_2  ) ] 
    \bigstep{\adapt_3} \valr, \env_3
  }{
     \expr_1 \eapp \expr_2 , (\env_1 \uplus \env_2)\bigstep{\adapt_1+\adapt_3} \valr, \env_3
  }~\textsf{app}
 %
  \and
  % %
  % \wq{
  % \inferrule{
  %   \env, \expr_1 \bigstep{R} \valr_1 \\
  %   \env, \expr_2 \bigstep{S} \valr_2  }
  % {
  %   \env, (\expr_1, \expr_2) \bigstep{(R,S)} (\valr_1, \valr_2)
  % }~\textsf{pair}
  % }
  % %
  % \and
  % %
  % \wq{
  % \inferrule{
  %   \env, \expr \bigstep{(R_1,R_2)} (\valr_1, \valr_2)
  % }{
  %   \env, \eprojl(\expr) \bigstep{R_1} \valr_1
  % }~\textsf{fst}
  % }
  % %
  % \and
  % %
  % \inferrule{
  %   \env, \expr \bigstep{(R_1,R_2)} (\valr_1, \valr_2), \tr
  % }{
  %   \env, \eprojr(\expr) \bigstep{(R_2)} \valr_2, \trprojr(\tr)
  % }~\textsf{snd}
  % %
  % \and
  % %
  % \inferrule{
  %   \env, \expr \bigstep{R} \etrue\\
  %   \env, \expr_1 \bigstep{S} \valr, \tr_1
  % }{
  %   \env, \eif(\expr, \expr_1, \expr_2) \bigstep{R+S} \valr
  % }~\textsf{if-true}
  % %
  % \and
  % %
  % \inferrule{
  %   \env, \expr \bigstep{R} \efalse \\
  %   \env, \expr_2 \bigstep{S} \valr
  % }{
  %   \env, \eif(\expr, \expr_1, \expr_2) \bigstep{R+S } \valr
  % }~\textsf{if-false}
  % %
  \and
  %
  \inferrule{
    \expr , \env \bigstep{\adapt} \valr' , \env_1 \\
    \eop{}(\valr') = \valr
  }{
    \eop(\expr), \env \bigstep{\adapt +1} \valr,  \env_1
  }~\textsf{delta}
  %
   % \and
% %
%   \inferrule{
% }
% { \env, \enil \bigstep{0} \enil, \trnil }~\textsf{nil}
% %
% \and
% %
% \inferrule{
% \env, \expr_1 \bigstep{R} \valr_1 \\
% \env, \expr_2 \bigstep{S} \valr_2
% }
% { \env, \econs (\expr_1, \expr_2)  \bigstep{  \max(R,S)} \econs (\valr_1, \valr_2)
% }~\textsf{cons}
% %
% \and
% %
% \inferrule{
%   \env, \expr_1 \bigstep{R} \valr_1 \\
%   \env[x \mapsto (\valr_1, R  )] , \expr_2 \bigstep{S} \valr
% }
% {\env, \elet x = \expr_1 \ein \expr_2 \bigstep{S} \valr }~\textsf{let}
% %
% \\\\
% %
% \inferrule
% {
%   \env, \expr \bigstep{R} \valr
% }
% {
%   \env, \eilam \expr \bigstep{0} \eilam \valr,
% }~\textsf{eilam}
% %
% \and
% %
% \inferrule{
%   \env, \expr \bigstep{K} (\eilam \expr') \\
%   \env, \expr' \bigstep{S} \valr
% }
% {\env, \expr [] \bigstep{K+S} \valr, \triapp{\tr_1}{\tr_2}
% }~\textsf{eiapp1}
% %
% \and
% %
% \inferrule{
%   \env, \expr \bigstep{K} \valr \not\equiv (\eilam \expr') \\
% }
% {\env, \expr [] \bigstep{K} \valr []
% }~\textsf{eiapp2}
% %
% \and
% %
% \wq{
%  \inferrule{
%     \env, \expr \bigstep{R} \valr \not\equiv \mathbb{B} 
%   }{
%     \env, \eif(\expr, \expr_1, \expr_2) \bigstep{R } \eif(\valr, \expr_1, \expr_2)
%   }~\textsf{if}
% }
% %
% \and
% %
% \wq{
%  \inferrule{
%     \env, \expr \bigstep{R} \valr
%   }{
%     \env, \eprojl(\expr) \bigstep{R} \eprojl(\valr)
%   }~\textsf{fst1}
% }
% %
% \and
% %
% \wq{
%  \inferrule{
%     \env, \expr \bigstep{R} \valr
%   }{
%     \env, \eprojr(\expr) \bigstep{R} \eprojr(\valr)
%   }~\textsf{snd1}
%   }
  \\\\
  \begin{array}{llll}
    \env_1 \uplus \emptyset & \triangleq & \env_1 &\\
     \emptyset \uplus \env_2 & \triangleq & \env_2 &\\
    % (\env_1,[x \to (\valr, \adapt_1)] )\uplus (\env_2, [x \to (\valr,
    % \adapt_2)] )  &  \triangleq & (\env_1 \uplus \env_2),[x \to
    %                               (\valr, \max(\adapt_1, \adapt_2))] & \\       
    
  \end{array}
  \\\\
  \begin{array}{lllll}
    \mathsf{fetch} (\env, x) & ::=  &  \mathsf{Some} ( \Pi_1
                                      (\env(x)), \Pi_3(\env(x)) ) &   &  x \in  \dom(\env) \\
            &  & \mathsf{match} \eapp \env  \eapp \mathsf{with} & & x \not\in  \dom(\env) \\
             &    &     | (y \to (\valr', \env', R)) :: \env_t \to & 
                    \mathsf{match} \eapp \mathsf{fetch} (\env', x) \eapp
                    \mathsf{with} & \\                             
              &  &   & \mathsf{None}  \to \mathsf{fetch}(\env_t, x ) &
    \\
             &   &   & \mathsf{Some} (\valr, r) \to \mathsf{Some} (\valr,
                   r)  & \\ 
             & & | [ ] \to  \mathsf{None} & &                     
  \end{array}  
  
\end{mathpar}
  \caption{Big-step semantics}
  \label{fig:semantics1}
\end{figure}





%%%%%%%%%%%%%%%%%%%%%%%%%%%%%%%%%%%%%%%%%%%%%%%%%%%%%

%%%%%%%%%%%%%%%%%%%%%%%%%%%%%%%%%%%%%%%%%%%%%%%%%%%%%


\[
\begin{array}{llll}
  \mbox{Index Term} & \idx, \nnatA & ::= &     i ~|~ n \\
 %                                  - \idx_2 ~|~ \smax{\idx_1}{\idx_2}\\
%                                  \mbox{Sort} & S & ::= & \nat \\
  \mbox{Linear type} & \type &::=  &  \ltype \lto{\nnatA} \type ~|~ \tbase \\
  \mbox{Nonlinear Type} & \ltype & ::= & \bang{\idx} \type   \\
  \mbox{Typing context } & \Gamma & ::= & x_1 : \ltype_1, \ldots,
                                          x_n : \ltype_n
\end{array}
\]

\begin{figure}
  \begin{mathpar}
    \inferrule{
    }{
      \ictx \tctx , x: \bang{1} \type \tvdash{0} x: \type
    }~\textbf{Ax}
    %
    \and
    %
    \inferrule{
    }{
      \ictx \Gamma \tvdash{0} c : \tbase 
    }~\textbf{const}
    %
    % \and
    % %
    % \inferrule{
    % }{
    %   \ictx \Gamma \tvdash{\nnatA} \evec : \bang{\nnatA}\tbase 
    % }~\textbf{Dict}
    %
    \and
    %
    \inferrule{
      \ictx \Gamma, x: \ltype
      \tvdash{\nnatA }
      \expr: \type
    }{
      \ictx \Gamma \tvdash{0} \lambda x. \expr : \ltype
      \lto{\nnatA} \type
    }~\textbf{lambda}
    \and
    %
    \inferrule{
      \ictx \Gamma_1  \tvdash{\nnatA_1} \expr_1:  \bang{\idx} \type_1
      \lto{\nnatA} \type_2      \\
      \ictx \Gamma_2 \tvdash{\nnatA_2} \expr_2: \type_1 
    }{
      \ictx   \Gamma_1 + \idx \times \Gamma_2  \tvdash{    \nnatA_1 +
        \idx \times \nnatA_2 + \nnatA    } \expr_1 \eapp \expr_2 : \type_2
    }~\textbf{app}
    %
    \and
    %
    \inferrule{
      \ictx \Gamma \tvdash{\nnatA} \expr:  \tbase 
    }{
      \ictx \Gamma  \tvdash{1+\nnatA} \delta(\expr): \tbase
    }~\textbf{delta}
     %
    \and
    %
    \inferrule{
      \ictx \Gamma'  \tvdash{\nnatA'} \expr: \type' \\
      \Gamma' \leqslant \Gamma \\
      \nnatA' \leq \nnatA\\
      \sub{\type'}{\type} 
    }{
      \ictx \Gamma  \tvdash{\nnatA} \expr: \type 
    }~\textbf{subtype}
      %
    \and
    %
    \inferrule{
      \ictx \Gamma, y: \type', x: \type ,\Gamma'  \tvdash{\nnatA} \expr: \type 
    }{
      \ictx \Gamma, x: \type, y: \type' ,\Gamma'  \tvdash{\nnatA} \expr: \type 
    }~\textbf{exchange}
   %  \\\\
 %    \boxed{
 % \inferrule{
 %      \ictx \Gamma, x: \type_1
 %      \tvdash{\nnatA }
 %      \expr: \type_2
 %    }{
 %      \ictx k+\Gamma \tvdash{k} \lambda x. \expr : \bang{k}  ( \type_1
 %      \lto^{\nnatA} \type_2)
 %    }~\textbf{lambda}
 %    \and
 %    %
 %    \inferrule{
 %      \ictx \Gamma  \tvdash{\nnatA_1} \expr_1:  \bang{0} ( \type_1
 %      \lto^{\nnatA} \type_2      ) \\
 %      \ictx \Gamma \tvdash{\nnatA_2} \expr_2: \type_1 
 %    }{
 %      \ictx \Gamma  \tvdash{ \nnatA_1 + \max( \nnatA,\nnatA_2) } \expr_1 \eapp \expr_2 : \type_2
 %    }~\textbf{app}
 %    }
     \\\\
\begin{array}{llll}
  \idx \times \Gamma &\triangleq  &  \Gamma  & \idx =1  \\
                     &\triangleq  &  \bang{0} \Gamma & \idx =0 \\
  \bang{\idx_1} \type + \bang{\idx_2} \type  &\triangleq  & \bang{
                                                          \max(\idx_1,\idx_2)
                                                          } \type &  \\
  \Gamma + \emptyset & \triangleq & \Gamma & \\
  \emptyset+ \Gamma  & \triangleq & \Gamma & \\
  ( [x : \ltype ],\Gamma) +  ([x: \ltype'],\Delta)  & \triangleq
                            & [x: \ltype + \ltype' ], \Gamma +
                              \Delta &   \\
   \sub{\Gamma}{\Delta} & \triangleq &  \dom(\Gamma) = \dom(\Delta) & \\
    & &                                    \land \forall x \in
                                      \dom(\Gamma),
        \sub{\Delta(x)}{\Gamma(x)} &  
\end{array}
  \end{mathpar}
  \caption{Typing rules, first version}
  \label{fig:type-rules1}
\end{figure}

\begin{figure}
  \begin{mathpar}
    \inferrule{
      \idx_1 \leq \idx \\
      \sub{\ltype}{\ltype_1}
    }{
      \sub{\bang{\idx} \ltype}{\bang{\idx_1} \ltype_1}
    }~\textsf{bang}
    %
    \and
    %
    \inferrule{
        \nnatA \leq \nnatA' \\
        \sub{\type_1}{\type}   \\
      \sub{\type'}{\type_1'}
    }{
      \sub{\type \lto{\nnatA} \type' }{\type_1 \lto{\nnatA'} \ltype_1'}
    }~\textsf{arrow}
    %
    \and
    %
    \inferrule{
    }{
    \sub{\tbase}{\tbase}
    }~\textsf{base}
  \end{mathpar}
  \caption{subtyping}
 \end{figure}

 \clearpage

 \begin{figure}
  \begin{mathpar}
    \inferrule{
     \env ( x ) = (\valr, \env', \adapt)
      \\
      \tvdash{\nnatA} ( \valr, \env') : \type
          }{
     \tvdash{\adapt + \nnatA}   ( x, \env):  \type
    }~\textbf{C-Ax}
    %
    \and
    %
    \inferrule{
    }{
      \tvdash{0} (  c, \env) : \tbase
    }~\textbf{C-const}
   
    \and
    %
    \inferrule{
      \tvdash{\nnatA' } ( \valr', \theta') : \type_1
      \\
      \fresh\eapp  x' ~~ \forall \adapt'
      \\
      \tvdash{ S+ \idx \times \adapt' +\nnatA }
     ( \expr[x'/x], \env[x' \to (\valr', \theta', R')]      ) :
     \type_2
    }{
     \tvdash{S} (  \lambda x. \expr, \env )  : \bang{\idx} \type_1
      \lto{\nnatA} \type_2
    }~\textbf{C-lambda}
    \and
    %
    \inferrule{
       \tvdash{\nnatA_1} ( \expr_1, \env_1) :  \bang{\idx} \type_1
      \lto{\nnatA} \type_2      \\
      \tvdash{\nnatA_2} ( \expr_2, \env_2 ): \type_1
    }{
       \tvdash{    \nnatA_1 +
        \idx \times \nnatA_2 + \nnatA    } (  \expr_1 \eapp \expr_2, \env_1 \uplus \env_2   ) : \type_2
    }~\textbf{C-app}
    %
    \and
    %
    \inferrule{
      \tvdash{\nnatA} (\expr, \env) :  \tbase
   }{  \tvdash{1+\nnatA} (\delta(\expr) , \env ) : \tbase
    }~\textbf{C-delta}
    \\\\
    \begin{array}{lll}
       \theta  & \triangleq (x_i \to (\valr_i, \env_i, R_i)) & i \in
                                                               \mathbb{N}\\
      (x_i : \bang{ \idx }\type_i), \Gamma \vDash (x_i \to (\valr_i, \env_i, R_i))
      \uplus \theta & \triangleq ~~~\tvdash { \_ } (\valr_i, \env_i)
                                          :  \type_i  &\conj
                                   \Gamma \vDash \theta
      \end{array}
  \end{mathpar}
  \caption{Typing rules, configure}
  \label{fig:configure-rules}
\end{figure}

\begin{figure}
  \begin{mathpar}
    \begin{array}{lll}
      \lrv{\tbase} & = & \{  ( \econst, \env,  \nnatA)  \} \\
      %
      % \lrv{\type_1 \times \type_2} & = & \{(\valr_1, \valr_2) ~|~ \valr_1 \in \lrv{\type_1} \conj \valr_2 \in \lrv{\type_2} \}\\
      %
      \lrv{\bang{k} \type } & = & \{  ( \valr, \env,   \nnatA) |  (\valr, \env,
                                   \nnatA ) \in \lrv{\type}  \} \\
      %
      \lrv{ \bang{k} \type_1 \lto{\nnatA} \type_2    } & = &
      \{( \lambda x.\expr, \env,  \nnatA_1) ~|~ \forall \valr', \env',
                                                             \nnatA'. (
                                                             \valr',\env',                                   
                                                             \nnatA') \in
                                                             \lrv{
                                                             \bang{k} \type_1}.\\
      & & 
          \implies   \fresh \eapp x' \land \\
      & & \forall \adapt. ( \expr[x'/x], \env[x' \mapsto (\valr', \env', \adapt )] ) \in
          \lre{    }{ \nnatA_1+\nnatA+ \idx \times \adapt }{\type_2}     \} \\
      %
      \\
      %
      \lre{}{\nnatA}{\type} & = & \{  ( \expr, \env) ~|~  ( \expr , \env
                                  \bigstep{R}  \valr, \env' ) \\
      & & ~~~~~~~~~~~~~\implies R \leq \nnatA \conj 
     ( \valr, \env', \nnatA- \adapt) \in \lrv{\type})
      \}
    \end{array}
  \end{mathpar}
  \caption{Logical relation without step-indexing}
  \label{fig:lr:non-step}
\end{figure}


\clearpage

 \begin{thm}[Monotonicity]
  \label{mono}
  \begin{enumerate} 
   \item If  $(
     \expr, \env) \in  \lre{}{\nnatA}{ \type} $ and $\nnatA' \geq \nnatA$,  then  $  (
     \expr, \env) \in  \lre{}{\nnatA'}{ \type} $.
   \item   If  $(
     \valr,\env,  R) \in  \lrv{\type} $ and $R' \geq R$,  then  $ (
     \valr,\env, R') \in  \lrv{\type} $.
     \item If $ \tvdash{\nnatA} (\expr, \env) : \type$ and $\nnatA
       \leq \nnatA'$, then $\tvdash{\nnatA'} (\expr, \env) : \type $.
     \item If $\Gamma \tvdash{\nnatA} \expr: \type$ and $\nnatA
       \leq \nnatA'$,   then  $\Gamma \tvdash{\nnatA'} \expr: \type$.
  \end{enumerate}
\end{thm}

\clearpage
\[
  \begin{array}{ll}
    F_{c2t} (\env) = \sum_{x_i \in \dom(\env) } R_i &~~where ~~  \env(x_i) =
                                                      (\valr_i,
                                                      \env_i, R_i) 
    \end{array}
  \]
  
\begin{thm}[ConfigurationToTyping]
  \label{sound}
  \begin{enumerate} 
   \item If $ \tvdash{\nnatA}  ( \expr, \env) : \type $ and $\Gamma \vDash
     \env $, then $ \Gamma \tvdash{\max(0, \nnatA- F_{c2t}(\env) ) } \expr: \type $.
  \end{enumerate}
\end{thm}
 \begin{proof}
  By induction on the configuration derivation.\\
   
 \caseL{ Case$     \inferrule{
     \env ( x ) = (\valr, \env', R) ~ (\star)
      \\
      \tvdash{0} ( \valr, \env') : \type ~(\star)
    }{
     \tvdash{R}   ( x, \env):  \type
    }~\textbf{C-Ax}
    $
  }

  Asssume $\Gamma \vDash \env$,  TS: $\Gamma \tvdash{R} x: \type $.
  
  From $(\star)$ and the definition of $\Gamma \vDash \env$, we know $\Gamma(x) = \bang{1} \type$,
  So we conclude $\Gamma \tvdash{0} x: \type ~(1)$ by the typing rule Ax. This case is proved by subtyping on $(1)$.

    \caseL{ Case
    $
       \inferrule{
      \tvdash{\_} ( \valr', \theta') : \type_1 ~(\star)
      \\
      \fresh\eapp  x'
      \\
      \tvdash{  \idx \times R' +\nnatA }
     ( \expr[x'/x], \env[x' \to (\valr', \theta', R')]      ) :
     \type_2~(\diamond)
     \\
       \Gamma \vDash \env
      \\
      \Gamma \tvdash{0} \lambda x. \expr :  \bang{\idx} \type_1
      \lto{\nnatA} \type_2
    }{
     \tvdash{0} (  \lambda x. \expr, \env )  : \bang{\idx} \type_1
      \lto{\nnatA} \type_2
    }~\textbf{C-lambda}
    $
  }

\end{proof}


 \clearpage

 \[
\begin{array}{ll}
 F(\expr, \env, \Gamma) & where \eapp ~~ \Gamma(x_i) = \bang{k_i}
                                           \type_i \land \env(x_i) =
                                        (\valr_i, \env_i, R_i)  \\
   F(x, \env, \Gamma) & = \sum_{x_i \in \fv{x}  } k_i \times R_i  \\
F(\lambda x. \expr , \env,\Gamma) & = 0 \\  %\sum_{x_i \in \fv{\lambda x.\expr} } k_i \times R_i  \\
F(\delta(\expr) , \env,\Gamma) & = \sum_{x_i \in \fv{\delta(\expr)} } k_i \times R_i  \\
F(c, \env,\Gamma) & = 0  \\
F(\expr_1 \eapp \expr_2, \env,\Gamma) & = F(\expr_1, \env, \Gamma) +
                                        F(\expr_2,\env, \Gamma)
\end{array} 
\]

\begin{thm}[TypingtoConfiguration]
  \label{sound}
  \begin{enumerate} 
   \item If $ x_1: \bang{\idx_1} \type_1, \ldots ,  x_i : \bang{\idx_i} \type_i
     \tvdash{\nnatA} \expr: \type$,  let $  (\valr_i, \env_i,
     R_i) \in \lrv{ \bang{\idx_i} \type_i }   $,  let $\env = [ x_1
     \to (\valr_1, \env_1, R_1), \ldots,  x_i \to (\valr_i, \env_i,
     R_i)   ]$, $ \Gamma =  x_1: \bang{\idx_1} \type_1, \ldots ,  x_i
     : \bang{\idx_i} \type_i $ , then   $
     \tvdash{\nnatA +  F( \expr, \env, \Gamma)  }
     ( \expr, \env ) : \type $ .
  \end{enumerate}
\end{thm}
\begin{proof}
  By induction on the typing derivation.\\

  \caseL{
    Case $
      \inferrule{
    }{
      \ictx \tctx , x: \bang{1} \type \tvdash{0} x: \type
    }~\textbf{Ax}
    $
  }

  let $(\valr, \env', R) \in \lrv{\bang{1} \type}$ and $(\valr_i,
  \env_i, R_i) \in \lrv{\Gamma(x_i)}$,
 

  let $\env =  [x \to (\valr, \env', R)] \uplus [ x_1 \to (\valr_1,
  \env_1, R_1), \ldots,   x_i \to (\valr_i,
  \env_i, R_i)]   $.

  $F\big(x,\env, (\Gamma, x: \bang{1} \type) \big) = 1 \times R $.

  TS:$\tvdash{0 +1 \times R} ( x,   \env  ) $.

   
  
  We conclude from the configuratio rule C-Ax.
  
  \[ \inferrule{
     \env ( x ) = (\valr, \env', R)
      \\
      \tvdash{0} ( \valr, \env') : \type
    }{
     \tvdash{R}   ( x, \env):  \type
    }~\textbf{C-Ax}
  \]

  \caseL{
    Case $
       \inferrule{
      \ictx \Gamma, x: \ltype
      \tvdash{\nnatA }
      \expr: \type
    }{
      \ictx \Gamma \tvdash{0} \lambda x. \expr : \ltype
      \lto{\nnatA} \type
    }~\textbf{lambda}
    $
  }

  let $\Gamma = x_i : \bang{k_i} \type_i$ and  $\ltype = \bang{k} \type_1$.

  Assume $(\valr', \env', R') \in \lrv{ \bang{k} \type_1}~(1)$ and $ (\valr_i, \env_i, R_i) \in \lrv{\bang{k_i} \type_i } $.

Let  $ \env =  [x_1 \to (\valr_1, \env_1, R_1), \ldots,   x_i \to (\valr_i, \env_i, R_i)]  $.
  
  TS: $ \tvdash{0 + F(\lambda x. \expr, \env, \Gamma)  }   (\lambda
  x.\expr,  \env) $.
  

  Let $S = \sum_{x_i \in \fv{\lambda x. \expr} } k_i \times R_i $ .

  From assumption $(1)$, we know :  $\tvdash{\_} ( \valr', \theta') : \type_1~(\star)$.

  Take a fresh variable x', doing alpha renaming on the premise,then
  by induction hypothesis on the premise , we know: $\tvdash{ \nnatA+
 F(\expr[x'/x], [x' \to (\valr', \env', R')] \uplus \env),  (\Gamma,x:
 \ltype) }   ( \expr[x'/x],  [x' \to (\valr', \env', R')] \uplus \env
) ~(\diamond) $.

 $F(\expr[x'/x], [x' \to (\valr', \env', R')] \uplus \env),  (\Gamma,x:
 \ltype)  = \sum_{x_i \in \fv{\lambda x. \expr} } k_i \times R_i  + k
 \times R'  =  S + k \times R'$.


 We can conclude the following by the configuration rule. 
    
  
  \[
       \inferrule{
      \tvdash{\_} ( \valr', \theta') : \type_1 ~(\star)
      \\
      \fresh\eapp  x'
      \\
      \tvdash{ S+  \idx \times R' +\nnatA }
     ( \expr[x'/x], \env[x' \to (\valr', \theta', R')]      ) :
     \type_2~(\diamond)
    }{
     \tvdash{S} (  \lambda x. \expr, \env )  : \bang{\idx} \type_1
      \lto{\nnatA} \type_2
    }~\textbf{C-lambda}
  \]


  \caseL{Case
    $
        \inferrule{
      \ictx \Gamma_1  \tvdash{\nnatA_1} \expr_1:  \bang{\idx} \type_1
      \lto{\nnatA} \type_2      \\
      \ictx \Gamma_2 \tvdash{\nnatA_2} \expr_2: \type_1 
    }{
      \ictx   \Gamma_1 + \idx \times \Gamma_2  \tvdash{    \nnatA_1 +
        \idx \times \nnatA_2 + \nnatA    } \expr_1 \eapp \expr_2 : \type_2
    }~\textbf{app}
    $
  }
  
  Let us assume $\Gamma_1 = x_i : \bang{k_i} \type_i $ and $\Gamma_2 = x_i' : \bang{k_i'} \type_i'$,($\Gamma_1$ and $\Gamma_2$ may overleap.).
  
  forall the variables $x_i''$ in $\dom (\Gamma_1 + k \times \Gamma_2
  )$, we assume: $(\valr_i'', \env'', R'') \in \lrv{ (\Gamma_1 + k
    \times \Gamma_2)(x_i'')  }$ and set $ (\Gamma_1 + k \times
  \Gamma_2 )(x_i'') = \bang{k_i''} \type_i'' $.

  let $\env = [x_1'' \to (\valr_1'', \env_1'', R_1''), \ldots,   x_i'' \to (\valr_i'', \env_i'', R_i'')]$.

  TS: $ \tvdash{\nnatA_1 + \idx \times \nnatA_2 + \nnatA   +   F(\expr_1 \eapp \expr_2,\env, \Gamma_1 + k
    \times \Gamma_2 )  }   (\expr_1 \eapp \expr_2 , \env) $.\\

  let $\env_1 = \{  [x_i'' \to (\valr_i'', \env_i'', R_i'') ]   |     x_i'' \in \dom(\Gamma_1)  \}   $.

  let $\env_2 = \{  [x_i'' \to (\valr_i'', \env_i'', R_i'') ]   |
  x_i'' \in \dom(\Gamma_2)  \}   $.

  We know  $ \dom(\env) = \dom(\Gamma_1) \uplus \dom(\Gamma_2) \implies   \env = \env_1 \uplus \env_2 $

  By induction hypothesis on the first premise, we have:
  $ \tvdash{ \nnatA_1 + F(\expr_1, \env_1, \Gamma_1) } (\expr_1, \env_1) : \bang{\idx} \type_1 \lto{\nnatA} \type_2 ~(\star)$ where $x_i'' \in dom(\Gamma_1) $.

  By induction hypothesis on the second premise, we have:
  $ \tvdash{ \nnatA_2 + F(\expr_2, \env_2, \Gamma_2)  } (\expr_2, \env_2) : \type_1 $ where $x_i'' \in dom(\Gamma_2)~(\diamond) $.

  
  
  

  From the configuration rule C-app, we get:

  
  \[
 \inferrule{
       \tvdash{\nnatA_1 + F(\expr_1, \env_1, \Gamma_1) } ( \expr_1, \env_1) :  \bang{\idx} \type_1
      \lto{\nnatA} \type_2    ~(\star)  \\
      \tvdash{\nnatA_2 +  F(\expr_2, \env_2, \Gamma_2)} ( \expr_2, \env_2 ): \type_1 ~(\diamond)
        }{
       \tvdash{    \nnatA_1 + F(\expr_1, \env_1, \Gamma_1) + \idx
         \times (\nnatA_2 +  F(\expr_2, \env_2, \Gamma_2)) + \nnatA
       } (  \expr_1 \eapp \expr_2, \env_1 \uplus \env_2   ) : \type_2 ~(\clubsuit)
    }~\textbf{C-app} 
  \]

  Because $ F(\expr_1, \env_1, \Gamma_1)  + \idx \times F(\expr_2,
  \env_2, \Gamma_2) \leq F(\expr_1 \eapp \expr_2,  \env, \Gamma_1 +
  \idx \times \Gamma_2 ) $, so we conclude :$  \nnatA_1 + F(\expr_1, \env_1, \Gamma_1) + \idx \times
  (\nnatA_2 +  F(\expr_2, \env_2, \Gamma_2)) + \nnatA  \leq \nnatA_1 + \idx \times \nnatA_2 + \nnatA   +   F(\expr_1 \eapp \expr_2,\env, \Gamma_1 + k
  \times \Gamma_2 )  $ .

  The TS can be shown using Lemma~\ref{mono} on the $\clubsuit$.
  
  
 \end{proof} 


\clearpage

\begin{thm}[ConfigurationSoundness]
  \label{sound}
  \begin{enumerate} 
   \item $ \tvdash{\nnatA}  ( \expr, \env) : \type $, then $(
     \expr, \env) \in  \lre{}{\nnatA}{ \type} $
  \end{enumerate}
\end{thm}

\begin{proof}
  By induction on the configuration derivation.\\
   
 \caseL{ Case$     \inferrule{
     \env ( x ) = (\valr, \env', R)
      \\
      \tvdash{0} ( \valr, \env') : \type ~(\star)
      \\
      \Gamma \vDash \env
      \\
      \Gamma \tvdash{0} x: \type
    }{
     \tvdash{R}   ( x, \env):  \type
    }~\textbf{C-Ax}
    $
  }

  TS: $(x, \env) \in  \lre{}{\nnatA}{ \type}$.
  
  Unfold the definition, let us first assume: \[   \inferrule{  \env(x)  =  (\valr, \env',  \adapt)  }{x,
      \env  \bigstep{\adapt} \valr, \env' }~\textsf{var2}  \]
  
  STS: $R \leq R $  and $ (\valr, \env', R  ) \in \lrv{\type} $.

  By induction hypothesis on $(\star)$, we know $ (\valr, \env') \in \lre{}{0}{\type} $, by unfolding its definition and $ \valr, \env \bigstep{0} \valr, \env $,  we know that : $ (\valr, \env' ,0) \in \lrv{\type}~(1)$.

  This case is proved by Lemma~\ref{mono} on $(1)$.\\

  \caseL{ Case
    $
       \inferrule{
      \tvdash{\_} ( \valr', \theta') : \type_1 ~(\star)
      \\
      \fresh\eapp  x'
      \\
      \tvdash{  \idx \times R' +\nnatA }
     ( \expr[x'/x], \env[x' \to (\valr', \theta', R')]      ) :
     \type_2~(\diamond)
     \\
       \Gamma \vDash \env
      \\
      \Gamma \tvdash{0} \lambda x. \expr :  \bang{\idx} \type_1
      \lto{\nnatA} \type_2
    }{
     \tvdash{0} (  \lambda x. \expr, \env )  : \bang{\idx} \type_1
      \lto{\nnatA} \type_2
    }~\textbf{C-lambda}
    $
  }

   TS: $(\lambda x. \expr , \env) \in  \lre{}{0}{  \bang{\idx} \type_1\lto{\nnatA} \type_2 }$.
  
  Unfold the definition, becasue $\lambda x. \expr$ is value, we know :$ \valr, \env \bigstep{0} \valr, \env$.
  
  STS: $0 \leq 0 $  and $ (\lambda x.\expr, \env, 0  ) \in \lrv{ \bang{\idx} \type_1\lto{\nnatA} \type_2 } $.

  By induction hypothesis on $(\star)$,  we know : $ (\valr', \env' ,0) \in \lrv{\bang{k} \type_1} ~(1) $.

  Unfold the definition of $\lrv{ \bang{\idx} \type_1\lto{\nnatA} \type_2 }$, Pick any $R' \geq 0$, we conclude from (1) from Lemma~\ref{mono}: $  (\valr', \env' ,R') \in \lrv{\bang{k} \type_1} $.

  STS: $ \fresh \eapp  x' \land (\expr[x'/x] , \env[x' \to (\valr', \theta', R')] )  \in \lre{}{\nnatA+\idx \times R' }{\type_2} $.

  It is proved by Induction hypothesis on $(\diamond)$. \\

  \caseL{
  $
     \inferrule{
       \tvdash{\nnatA_1} ( \expr_1, \env_1) :  \bang{\idx} \type_1
      \lto{\nnatA} \type_2    ~(\star)  \\
      \tvdash{\nnatA_2} ( \expr_2, \env_2 ): \type_1 ~(\diamond)
      \\
        \Gamma \vDash \env
      \\
      \Gamma \tvdash{  \nnatA_1 +\idx \times \nnatA_2 + \nnatA  }  \expr_1 \eapp \expr_2 : \type_2
    }{
       \tvdash{    \nnatA_1 +\idx \times \nnatA_2 + \nnatA    } (  \expr_1 \eapp \expr_2, \env_1 \uplus \env_2   ) : \type_2
    }~\textbf{C-app}
  $
  }

     TS: $(\expr_1 \eapp  \expr_2, \env_1 \uplus \env_2) \in  \lre{}{ \nnatA_1 +\idx \times \nnatA_2 + \nnatA  }{ \type_2 }$.

  Unfold the definition, let us first assume: \[    \inferrule{
     \expr_1, \env_1 \bigstep{\adapt_1} \lambda x.\expr , \env_1' ~(a)\\
     \expr_2, \env_2 \bigstep{\adapt_2} \valr_2 , \env_2' ~(b) \\
    \fresh \eapp x' \\
    \expr[x'/x], \env_1'[ x'  \to (\valr_2, \env_2', \adapt_2  ) ] 
    \bigstep{\adapt_3} \valr, \env_3~(c)
  }{
     \expr_1 \eapp \expr_2 , (\env_1 \uplus \env_2)\bigstep{\adapt_1+\adapt_3} \valr, \env_3
  }~\textsf{app}
 \]
  
 STS1: $ \adapt_1+\adapt_3 \leq \nnatA_1 +\idx \times \nnatA_2 + \nnatA   $ .

 STS2:  $ (\valr, \env_3, \adapt_1+\adapt_3   ) \in \lrv{\type_2} $.

By Induction hypothesis on $(\star)$,  we get: $ (\expr_1, \env_1) \in \lre{}{ \nnatA_1 }{ \bang{\idx} \type_1
  \lto{\nnatA} \type_2   } ~(1)$.

Unfold $(1)$, from the assumption $(a)$, we know: $\adapt_1 \leq \nnatA_1 \land (\lambda x.\expr, \env_1', \adapt_1) \in \lrv{\bang{\idx} \type_1
  \lto{\nnatA} \type_2} ~(2)$.

By Induction hypothesis on $(\diamond)$,  we get: $ (\expr_2, \env_2) \in \lre{}{ \nnatA_2 }{ \type_1 } ~(3)$.

Unfold $(3)$, from the assumption $(b)$, we know: $\adapt_2 \leq \nnatA_2 \land (\valr_2, \env_2', \adapt_2) \in \lrv{\type_1} ~(4)$.

Unfold $(2)$, pick $(\valr', \env', R') = (\valr_2, \env_2', \adapt_2) \in \lrv{\type_1} $,  we know: $ \fresh \eapp x' \land  (  \expr[x'/x], \env_1'[ x'  \to (\valr_2, \env_2', \adapt_2  ) ]  ) \in \lre{}{\nnatA+\idx\times R_2 }{\type_2} ~(5) $.

Unfold $(5)$, from the assumption $(c)$, we conclude that: $ \adapt_3 \leq \nnatA+\idx\times R_2~(6)$ and $(\valr, \env_3, \adapt_3) \in \lrv{\type_2} ~(7) $ .

STS1 is proved by $(1), (6)$ and STS2 is proved by using Lemma~\ref{mono} on $(7)$. \\

\caseL{
  $
 \inferrule{
      \tvdash{\nnatA} (\expr, \env) :  \tbase ~(\star)
      \\
       \Gamma \vDash \env
      \\
      \Gamma \tvdash{\nnatA} \expr: \tbase
    }{  \tvdash{1+\nnatA} (\delta(\expr) , \env ) : \tbase
    }~\textbf{C-delta}
  $
}

 TS: $(\delta(\expr), \env ) \in  \lre{}{ 1+ \nnatA}{ \tbase }$.
 Unfold the definition, we first assume:
 \[ \inferrule{
    \expr , \env \bigstep{\adapt} \valr' , \env_1 ~(a) \\
    \eop{}(\valr') = \valr
  }{
    \eop(\expr), \env \bigstep{\adapt +1} \valr,  \env_1
  }~\textsf{delta}
  \]

  STS1: $\adapt+1 \leq \nnatA +1$.
  
STS2: $ (\valr, \env_1, \adapt+1) \in \lrv{\tbase} $.

By induction hypothesis on $(\star)$, we get: $ (\expr, \env) \in \lre{}{\nnatA}{\tbase} $.
Unfold this statement, from the assumption $(a)$, we get: $ \adapt \leq \nnatA~(1) $ and $ (\valr', \env_1, \adapt)  \in \lrv{\tbase}~(2)$.

STS1 is proved by $(1)$,  STS2 is proved by $(2)$ as well as the definition of $\lrv{\tbase}$.





 \end{proof} 

% \begin{thm}[Substitution]
%   \label{sub}
%   \begin{enumerate} 
%    \item If $ \Gamma,x : \type' \tvdash{ \nnatA} \expr : \type $ and $
%   \empty \tvdash{\nnatA'} \valr : \type'  $ , then  $\Gamma
%   \tvdash{\max(\nnatA,\nnatA' )} \expr[\valr/x]  : \type$. 
%   \end{enumerate}
% \end{thm}

% \begin{proof}
%   By induction on the typing derivation.\\
% \caseL{
%   $   \inferrule{
%     }{
%       \ictx \tctx , x: \bang{\nnatA}\ltype \tvdash{\nnatA} x: \bang{\nnatA}\ltype
%     }~\textbf{Ax}  $
%   }
% Assume $\empty \tvdash{\nnatA'} \valr : \bang{\nnatA}\ltype $, TS:  $\Gamma
%   \tvdash{\max(\nnatA,\nnatA' )} x[\valr/x]  : \type$. proved by
%   subtype rule on the assumption.
% \caseL{
%  $   \inferrule{
%     }{
%       \ictx \tctx ,y:\type', x: \bang{\nnatA}\ltype \tvdash{\nnatA} x: \bang{\nnatA}\ltype
%     }~\textbf{Ax2}  $
%   }
%   Assume $\empty \tvdash{\nnatA'} \valr : \bang{\nnatA}\ltype $, TS:
%   $\Gamma,   x: \bang{\nnatA}\ltype
%   \tvdash{\max(\nnatA,\nnatA' )} x[\valr/y]  : \type$. proved by rule
%   AX and then subtype.
%   \caseL{
%    \inferrule{
%       \ictx \Gamma, x: \type_1, y:\type'
%       \tvdash{\nnatA }
%       \expr: \type_2
%     }{
%       \ictx k+\Gamma, y: k + \type' \tvdash{k+\nnatA} \lambda x. \expr : \bang{k}  ( \type_1
%       \lto \type_2)
%     }~\textbf{lambda}
%   }
%    Assume $\empty \tvdash{k+\nnatA'} \valr : k+\type' $, TS:
%   $k+\Gamma
%   \tvdash{\max(k+\nnatA,k+\nnatA' )} (\lambda x. \expr)[\valr/y]  : \type$. From the
%   Lemma~\ref{para-dec} on the assumption, we know: $\empty
%   \tvdash{\nnatA'} \valr : \type' ~(1)$.\\
%   By Induction hypothesis on the premise, we get: $ \Gamma, x:\type_1
%   \tvdash{\max( \nnatA, \nnatA' )}
%       \expr[\valr/y]: \type_2 ~(2)$. By rule lambda, we conclude that
%       $k+\Gamma \tvdash{ k+ ( \max(\nnatA,\nnatA ) }
%       \lambda x.\expr[\valr/y]: \type_2 $.
%       \caseL{
%       \inferrule{
%       \ictx \Gamma_1,x:\type'  \tvdash{\nnatA_1} \expr_1:  \bang{0} ( \type_1
%       \lto \type_2      ) \\
%       \ictx \Gamma_2 ,x: \type'', \tvdash{\nnatA_2} \expr_2: \type_1 
%     }{
%       \ictx \max (\Gamma_1, \Gamma_2 ), x:\max(\type',\type'') \tvdash{\max( \nnatA_1,\nnatA_2) } \expr_1 \eapp \expr_2 : \type_2
%     }~\textbf{app}
%   }
%   Assume $\empty \tvdash{\nnatA'} \valr : \max(\type',\type'')$, TS: $\max (\Gamma_1, \Gamma_2 )
%   \tvdash{\max(\nnatA_1,\nnatA_2, \nnatA' )} (\expr_1 \eapp
%   \expr_2)[\valr/x]  : \type_2$. From the definition of $\max(\type',
%   \type'')$, we know that $\type'$ and $\type''$ have similar
%   form. Let us assume $\type'= \bang{k_1} \ltype$ and $\type'' =
%   \bang{k_2} \ltype$ so that $\max(\type',\type'') = \bang{\max(k_1,k_2)}
%   \ltype$.\\
%   From the Lemma~\ref{para-dec} on the assumption, we have $\empty
%   \tvdash{\nnatA' - (\max(k_1,k_2)-k_1) } \valr : \bang{k_1}
%   \ltype~(1)$ and $\empty
%   \tvdash{\nnatA' - (\max(k_1,k_2)-k_2) } \valr : \bang{k_2}
%   \ltype~(2)$.\\ By induction hypothesis on $(1)$ and $(2)$ respctively,
%   we know that:  $ \Gamma_1  \tvdash{ \max( \nnatA_1, \nnatA' - (\max(k_1,k_2)-k_1) ) } \expr_1[\valr/x]:  \bang{0} ( \type_1
%   \lto \type_2   ) ~(3)$  and $ \Gamma_2  \tvdash{\max(\nnatA_2 ,
%     \nnatA' - (\max(k_1,k_2)-k_2)   )} \expr_2[\valr/x]: \type_1 ~(4)$.  By the
%   rule app and $(3)$, $(4)$, we conclude that $$\max (\Gamma_1, \Gamma_2 )
%   \tvdash{\max(  \max( \nnatA_1, \nnatA' - (\max(k_1,k_2)-k_1) )  , \max(\nnatA_2 ,
%     \nnatA' - (\max(k_1,k_2)-k_2)   )  )} \expr_1[\valr/x] \eapp
%   \expr_2[\valr/x]  : \type_2 ~(5).$$
%   Because $\max(\nnatA' - (\max(k_1,k_2)-k_1) ) , \nnatA' -
%   (\max(k_1,k_2)-k_2)   ) \leq \nnatA' $, by subtype, we raise the
%   adaptivity to  $\max(\nnatA_1,\nnatA_2, \nnatA' ) $ from $(5)$.
%    \caseL{
%       \inferrule{
%       \ictx \Gamma_1,x:\type'  \tvdash{\nnatA_1} \expr_1:  \bang{0} ( \type_1
%       \lto \type_2      ) \\
%       \ictx \Gamma_2  \tvdash{\nnatA_2} \expr_2: \type_1 
%     }{
%       \ictx \max (\Gamma_1, \Gamma_2 ), x:\type' \tvdash{\max( \nnatA_1,\nnatA_2) } \expr_1 \eapp \expr_2 : \type_2
%     }~\textbf{app2}
%   }
%   It is another case for application when x only appear in the first
%   premise. In this case, $\expr_2[\valr/x] = \expr_2$. Another case
%   when variable x only appears in the second premise can be proved in
%   a similar way.\\
%   Assume $\empty \tvdash{\nnatA'} \valr :\type'$. TS:$\max (\Gamma_1, \Gamma_2 )
%   \tvdash{\max(\nnatA_1,\nnatA_2, \nnatA' )} (\expr_1 \eapp
%   \expr_2)[\valr/x]  : \type_2$.  By Induction Hypothesis on the first
%   premise using the assumption, we get: $\Gamma_1
%   \tvdash{\max(\nnatA_1, \nnatA')} \expr_1[\valr/x]:  \bang{0} ( \type_1
%       \lto \type_2  )  ~(1)$. By the rule app using (1) and the second
%       premise, we conclude that $$ \max (\Gamma_1, \Gamma_2 )
%       \tvdash{\max( \max(\nnatA_1,\nnatA'),\nnatA_2) }
%       \expr_1[\valr/x] \eapp \expr_2 : \type_2$$
%       \caseL{
%  \inferrule{
%       \ictx \Gamma, x:\type' \tvdash{\nnatA} \expr: \bang{k} \ltype 
%     }{
%       \ictx \Gamma' ,1+\Gamma, x:1+\type'  \tvdash{1+\nnatA} \delta(\expr): \bang{k} \ltype 
%     }~\textbf{delta}
%   }
%   Assume $\empty \tvdash{\nnatA'+1} \valr : 1+\type' $, TS: $ \Gamma'
%   ,1+\Gamma \tvdash{\max(1+\nnatA, 1+\nnatA')} \delta(\expr)
%   [\valr/x]: \bang{k} \ltype $.
%   By Lemma~\ref{para-dec} on the assumption, we have $\empty
%   \tvdash{\nnatA'} \valr : \type'~(1) $. By IH on the first premise
%   along with (1), we have: $\Gamma \tvdash{\max(\nnatA, \nnatA')}
%   \expr[\valr/x]: \bang{k} \ltype~ (2)$.
%    By the rule delta using (2), we conclude that $\Gamma' ,1+\Gamma  \tvdash{1+(\nnatA,\nnatA')} \delta(\expr[\valr/x]): \bang{k} \ltype$.
% \end{proof}

% \begin{thm}[Substitution]
%   \label{sub}
%   \begin{enumerate} 
%    \item If $ \Gamma,x : \bang{k} \ltype \tvdash{ \nnatA} \expr : \type $ and $
%   \empty \tvdash{k} \valr : \bang{k} \ltype  $ , then  $\Gamma
%   \tvdash{ \nnatA } \expr[\valr/x]  : \type$. 
%   \end{enumerate}
% \end{thm}

% \begin{proof}
%   By induction on the typing derivation.\\
%%%%%%%%%%%%%%%%%%%%%
% \caseL{
%   $   \inferrule{
%     }{
%       \ictx \tctx , x: \bang{\nnatA}\ltype \tvdash{\nnatA} x: \bang{\nnatA}\ltype
%     }~\textbf{Ax}  $
%   }
  
% Assume $\empty \tvdash{\nnatA} \valr : \bang{\nnatA}\ltype $ TS:  $\Gamma
% \tvdash{\nnatA } x[\valr/x]  : \type$. proved by the assumption.\\

% \caseL{
%  $   \inferrule{
%     }{
%       \ictx \tctx ,y:\type', x: \bang{\nnatA}\ltype \tvdash{\nnatA} x: \bang{\nnatA}\ltype
%     }~\textbf{Ax2}  $
%   }
  
% Assume $\empty \tvdash{\nnatA} \valr : \bang{\nnatA}\ltype $, TS:
%    $\Gamma,   x: \bang{\nnatA}\ltype
%    \tvdash{\nnatA } x[\valr/y]  : \type$. proved by the assumption.\\

% %%%%%%%%%%%%%%%%%%%%%%%%%%%%%%%
%   \caseL{
%    \inferrule{
%       \ictx \Gamma, x: \type_1, y: \bang{k_1} \ltype
%       \tvdash{\nnatA }
%       \expr: \type_2
%     }{
%       \ictx k+\Gamma, y: k + \bang{k_1} \ltype \tvdash{k} \lambda x. \expr : \bang{k}  ( \type_1
%       \lto^{\nnatA} \type_2)
%     }~\textbf{lambda}
%   }

%    Assume $\empty \tvdash{k+k_1 } \valr : k+ \bang{k_1} \ltype $, TS:
%   $k+\Gamma
%   \tvdash{ k } (\lambda x. \expr)[\valr/y]  : \bang{k}  ( \type_1
%       \lto^{\nnatA} \type_2)$.

%  From the  Lemma~\ref{para-dec} on the assumption, we know: $\empty
%   \tvdash{ k_1} \valr : \bang{k_1} \ltype ~(1)$.

%   By Induction hypothesis on the premise: $ \Gamma, x:\type_1
%   \tvdash{\nnatA }
%       \expr[\valr/y]: \type_2 ~(2)$. 

% By rule lambda, we conclude that
%       $k+\Gamma \tvdash{ k }
%       \lambda x.\expr[\valr/y]: \bang{k}  ( \type_1
%       \lto^{\nnatA} \type_2) $.\\
      
%       %%%%%%%%%%%%%%%%%%%%%%%%%%%%%%%%%
      
%       \caseL{
%       \inferrule{
%       \ictx \Gamma ,x:\bang{k} \ltype  \tvdash{\nnatA_1} \expr_1:  \bang{0} ( \type_1
%       \lto^{\nnatA} \type_2      ) \\
%       \ictx \Gamma ,x:\bang{k} \ltype  \tvdash{\nnatA_2} \expr_2: \type_1 
%     }{
%       \ictx \Gamma, x:\bang{k} \ltype \tvdash{ \nnatA+ \max( \nnatA,\nnatA_2) } \expr_1 \eapp \expr_2 : \type_2
%     }~\textbf{app}
%   }
  
%   Assume $\empty \tvdash{k} \valr : \bang{k} \ltype$, TS: $\Gamma
%   \tvdash{ \nnatA_1+ \max(\nnatA,\nnatA_2)  } (\expr_1 \eapp
%   \expr_2)[\valr/x]  : \type_2$.
  
% By induction hypothesis on the first and second premise, we conclude
% : $\ictx \Gamma \tvdash{\nnatA_1}  \expr_1[\valr/x]  :  \bang{0}
% ( \type_1 \lto^{\nnatA} \type_2 )~(1)$ and $ \ictx \Gamma \tvdash{\nnatA_2} \expr_2[\valr/x]: \type_1(2)$.

%  By the rule app and $(1)$, $(2)$,  we prove that $$ \Gamma
%   \tvdash{  \nnatA_1 + \max(\nnatA, \nnatA_2)  }  \expr_1[\valr/x] \eapp
%   \expr_2[\valr/x]  : \type_2$$

% \end{proof}

% \begin{lem}[Parameter Decreasing]
%   \label{para-dec}
%   if  $k+\Gamma \tvdash{\nnatA} \valr : k+ \type  $, then exists 
%   $\nnatA'$ so
%   that   $\Gamma \tvdash{\nnatA' } \valr :
%   \type$ and  $\nnatA' \leq \nnatA-k $.
% \end{lem}
% \begin{proof}
  
%   If $\valr$ is a constant, then it is trivial, assume $\type =
%   \bang{r} \tbase$, then $\nnatA = r+k$,  choose
%   $\nnatA' = r$, proved from the rule $const$.
  
%   If $\valr$ is an abstraction term, assuming $\valr = \lambda
%   x. \expr$. Correspondingly, the type of $\valr$ is an arrow type,
%   assuming $\Gamma = r+ \Gamma_1 $ and   $\type =\bang{r} (\type_1
%   \lto^{\nnatA} \ltype_2)$, from the typing derivation, we know : 
%   \[
%   \inferrule{
%       \ictx \Gamma_1, x: \type_1
%       \tvdash{\nnatA }
%       \expr: \type_2  ~(1)
%     }{
%       \ictx k+r+\Gamma_1  \tvdash{k+r} \lambda x. \expr : k + \bang{r}  ( \type_1
%       \lto^{\nnatA} \type_2)
%     }~\textbf{lambda}
% \]

% Use $(1)$ as premise, we use lambda rule again.

%  \[
%   \inferrule{
%       \ictx \Gamma_1, x: \type_1
%       \tvdash{\nnatA }
%       \expr: \type_2  ~(1)
%     }{
%       \ictx r+\Gamma_1  \tvdash{r} \lambda x. \expr :  \bang{r}  ( \type_1
%       \lto^{\nnatA} \type_2)
%     }~\textbf{lambda}
% \]

% \end{proof}

% \[
%   \begin{array}{lll}
%      \env \vDash \Gamma &\triangleq  & 
%                                         \forall x_i \in \dom( \Gamma) . \env(x_i) =
%                                        (\valr_i, \adapt_i) \land
%                                        \empty \tvdash{\adapt_i} \valr_i
%                                        : \Gamma(x_i)   \\
%     F(\env, \expr) & ::= &  \max(\max(\adapt_i) , 0 )    \\
%     & where &   \forall x_i \in \fv{\expr} . \env(x_i) = (\valr_i, \adapt_i).
%                                           \\
%     \end{array}
% \]

% \begin{thm}[Soundness- one attempt]
% \label{soundness}
% If $\Gamma \tvdash{\nnatA} \expr : \bang{k} \ltype$, $ \forall \env$ that $\env
% \vDash \Gamma$, exists $\env'$ and $\valr$ so that $\env , \expr \bigstep{\adapt} \valr,
% \env'  $, then  $ \adapt + k \leq  \nnatA + F(\env, \expr)$.  
% \end{thm}
% \begin{proof}
%   By Induction on the typing derivation.
%   \caseL{
%      $   \inferrule{
%     }{
%       \ictx \tctx , x: \bang{\nnatA}\ltype \tvdash{\nnatA} x: \bang{\nnatA}\ltype
%     }~\textbf{Ax}  $
%   }
  
%   Assume $\env= \Big( \env_1, [x \to (\valr,\adapt
%   )]  \Big) \vDash (\tctx , x: \bang{\nnatA}\ltype  )$ where $\env_1
%   \vDash \Gamma$.
  
%   We know from the evaluation rule var: $\Big( \env_1, [x \to (\valr,\adapt
%   )]  \Big) , x \bigstep{\adapt} \valr,
%   \env  $.
  
%   TS:  $ \adapt +\nnatA  \leq  \nnatA +
%   F(\env,x) \implies \adapt + \nnatA \leq \nnatA +  R
%   $. It is trivially true.\\

% \caseL{
%  $
% \inferrule{
%       \ictx \Gamma, x: \type_1
%       \tvdash{\nnatA }
%       \expr: \type_2
%     }{
%       \ictx k+\Gamma \tvdash{k} \lambda x. \expr : \bang{k}  ( \type_1
%       \lto^{\nnatA} \type_2)
%     }~\textbf{lambda}
%     \and
%     %
%  $
% }

% Assume $\env \vDash k+\Gamma $,  from the evaluation rule lambda:
% $\env , \lambda x. \expr \bigstep{0}   \lambda x. \expr ,
%   \env  $.

% TS: $ 0 + k  \leq  k  +
%   F(\env, \lambda x. \expr)  
%   $, which is trivially true. \\
  
% \caseL{
% $
%     \inferrule{
%       \ictx \Gamma  \tvdash{\nnatA_1} \expr_1:  \bang{0} ( \type_1
%       \lto^{\nnatA} \type_2      ) \\
%       \ictx \Gamma \tvdash{\nnatA_2} \expr_2: \type_1 
%     }{
%       \ictx \Gamma  \tvdash{ \nnatA_1 + \max( \nnatA,\nnatA_2) } \expr_1 \eapp \expr_2 : \type_2
%     }~\textbf{app}
% $
% }

% \end{proof}



% \begin{thm}[Soundness-original]
% \label{soundness}
% If $\Gamma \tvdash{\nnatA} \expr : \type$, $ \forall \env$ that $\env
% \vDash \Gamma$, exists $\env'$ and $\valr$ so that $\env , \expr \bigstep{\adapt} \valr,
% \env'  $, then  $ \adapt + adap(\valr, \env')  \leq  \nnatA + F(\env, \expr)$.  
% \end{thm}


% \begin{proof}
%   By Induction on the typing derivation.
%   \caseL{
%      $   \inferrule{
%     }{
%       \ictx \tctx , x: \bang{\nnatA}\ltype \tvdash{\nnatA} x: \bang{\nnatA}\ltype
%     }~\textbf{Ax}  $
%   }
%   Assume $\env= \Big( \env_1, [x \to (\valr,\adapt
%   )] , \Big) \vDash (\tctx , x: \bang{\nnatA}\ltype  )$ where $\env_1 \vDash \Gamma$. We know that $
%   \empty \tvdash{\adapt} \valr : \bang{\nnatA}\ltype $.
%   From the evaluation rule var, we know $\env , x \bigstep{\adapt} \valr,
%   \env  $.
%   TS:  $ \adapt + adap(\valr, \env)  \leq  \nnatA +
%   F(\env) \implies \adapt + 0 \leq \nnatA + \max( \adapt, F(\env_1))
%   $.It is trivially true.
% \caseL{
%   $
%     \inferrule{
%       \ictx \Gamma, x: \type_1
%       \tvdash{\nnatA }
%       \expr: \type_2
%     }{
%       \ictx k+\Gamma \tvdash{k+\nnatA} \lambda x. \expr : \bang{k}  ( \type_1
%       \lto \type_2)
%     }~\textbf{lambda}
%   $
% }
% Choose $\env \vDash  (k+\Gamma)$ so that $\forall x_i \in
% (\Gamma). \env(x_1) =(\valr_i, \adapt_i ) \land \empty
% \tvdash{\adapt_i } \valr_i: k+\Gamma(x_i) $.  By the evaluation rule
% we know $\env, \lambda x. \expr \bigstep{0}
%                                        \lambda x.\expr, \env $, TS: $0
%                                        + \adap(\lambda x.\expr, \env)
%                                        \leq  k+\nnatA + F(\env)$, which is trivially
%                                        true because $ \adap(\lambda
%                                        x.\expr, \env) \leq F(\env) $.
                                       
% \caseL{
%     $  \inferrule{
%       \ictx \Gamma_1  \tvdash{\nnatA_1} \expr_1:  \bang{0} ( \type_1
%       \lto \type_2      ) \\
%       \ictx \Gamma_2 \tvdash{\nnatA_2} \expr_2: \type_1 
%     }{
%       \ictx \max (\Gamma_1, \Gamma_2 ) \tvdash{\max( \nnatA_1,\nnatA_2) } \expr_1 \eapp \expr_2 : \type_2
%     }~\textbf{app}  $
%   }
%   Choose $\env = [x_i \to (\valr_i,0)] $ for all $x_i$ in
%   $\dom(\max(\Gamma_1,\Gamma_2))$
%   so that  $\empty \tvdash{\nnatA_i} \valr_i  : (\max(\Gamma_1,
%   \Gamma_2)(x_i) $.
%   From the definition, we know that $\env \vDash \Gamma_1$ and $\env
%   \vDash \Gamma_2$. Because $\expr_1$ has the arrow type and will be
%   evaluated to a function, assume exists $\env_1$ so that $\env,
%   \expr_1 \bigstep{\adapt_1} \lambda x.\expr , \env_1 $.  By induction
%   hypothesis on the first premise, we know that: $\adapt_1 +
%   \adap(\lambda x. \expr, \env_1) \leq \nnatA_1 + F(\env,
%   \Gamma_1)~(1)$.Assume exists $\env_2$ so that $\expr_2$ is evaluated
%   to an arbitrary value $\valr_2$ : $ \env, \expr_2 \bigstep{\adapt_2}
%   \valr_2 , \env_2$, by induction hypothesis, we conclude that :  $\adapt_2 +
%   \adap(\valr , \env_2) \leq \nnatA_2 + F(\env,
%   \Gamma_2)~(2)$.
                            


% \[
% \inferrule{
%     \env, \expr_1 \bigstep{\adapt_1} \lambda x.\expr , \env_1 \\
%     \env, \expr_2 \bigstep{\adapt_2} \valr_2 , \env_2 \\
%     (\env_1 \uplus \env_2)[ x  \to (\valr_2,   \adapt_2  ) ], \expr
%     \bigstep{\adapt_3} \valr, \env_3
%   }{
%     \env, \expr_1 \eapp \expr_2 \bigstep{\adapt_1+\adapt_3} \valr, \env_3
%   }~\textsf{app}
% \]
%  \end{proof} 


% \begin{thm}[Subject Reduction]
% \label{sub-red}
% If $\Gamma \tvdash{\nnatA} \expr : \bang{k} \ltype$, $\forall \env . \env
% \vDash \Gamma$,   exists $\env'$ and $\valr$, $\env , \expr \bigstep{\adapt} \valr,
% \env'  $, then $ \Gamma  \tvdash{ k} \valr :\bang{k} \ltype $.  
% \end{thm}
% By induction on the typing derivation.

\end{document}



