%\begin{example}[Multiple Rounds Algorithm]
%\label{ex:multipleRounds}
%
%%%%%%%%%%%%%%%%%%%%%%%%%%%%%%%%%% Previous Version For Reference %%%%%%%%%%%%%%%%%%%%%%%%%%%%%%%%%%
% We look at an advanced adaptive data analysis algorithm - $\kw{multipleRounds}$ algorithm in Fig.~\ref{fig:multipleRounds}(a).
% This is a simplified version of the \emph{Monitor Augment} from \cite{RogersRSSTW20} with complete program in Apdix.
% It takes the user input $k$ which decides the 
% number of iterations.
% It starts from an initialized empty tracking list $I$,
% goes $k$ rounds and at every round, tracking list $I$ is updated by a query result of $\query(\chi[I])$.
% After $r$ rounds, the algorithm returns the columns of the hidden database $D$ not specified in the tracking list $I$.
% The functions $\kw{updnscore}(p,a)$,
% $\kw{updcscore}(p,a)$, $\kw{update}(I,ns,cs)$ simplify the computations of updating $ns$, $cs$ and $I$.%
Our first example, Algorithm $\kw{multipleRounds}$ in Fig.~\ref{fig:multipleRounds}(a), is a simplified form of the \emph{monitor argument} by \citet{RogersRSSTW20}.
The input $k$ is the number of iterations.
It uses a list $I$ to track queries. Specifically at each iteration it updates $I$ by using the result of a query which relies on $I$:  $\query(\chi[I])$.
After $k$ iterations, the algorithm returns the columns of the hidden database $D$ which are not contained in the  tracking list $I$.
The functions $\kw{updnscore}(p,a)$,
$\kw{updcscore}(p,a)$, $\kw{update}(I,ns,cs)$ simplify the computations of updating $ns$, $cs$ and $I$. They depends on the result of the query but they do not perform queries themselves%
%
%
%%%%%%%%%%%%%%%%%%%%%%%%%%%%%%%%%% Previous Version For Reference %%%%%%%%%%%%%%%%%%%%%%%%%%%%%%%%%%
% {Different from $\kw{twoRounds(k)}$ in Fig.~\ref{fig:overview-example},
% the query request, in each loop iteration is not independent. 
% The query in the $j^{th}$ iteration now depends on the tracking list $I$ from the previous  $(j - 1)^{th}$ iteration, 
% $I$ is updated by all the query results in the previous $j-1$ iterations. 
% In this sense, all these $k$ queries are adaptively chosen according to our discussion in overview.
% }
Different from the code in the example $\kw{twoRounds(k)}$,
the query request, $\clabel{\assign{a}{\query(I)}}^6$, in each loop iteration
depends on the tracking list $I$, which in turn depends on  all the querieas from previous iterations. 
%In particular, $I$ depends updated by all the query results in the previous iterations as well. 
In this sense, all these $k$ queries are fully adaptively chosen, and so the adaptivity is $k$.
%%%%%%%%%%%%%%%%%%%%%%%%%%%%%%%%%% Previous Version For Reference %%%%%%%%%%%%%%%%%%%%%%%%%%%%%%%%%%
% The program-based dependency graph is presented  in Fig.~\ref{fig:multi_graphs}(b). 
% We omitted its execution-based dependency graph $\traceG(\kw{multipleRounds(k)})$ because they have the same graph topology and only differ in weights.
% For the vertices in Fig.~\ref{fig:multi_graphs}(b) that have the weight $k$,
% their weights in $\traceG(\kw{multipleRounds(k)})$ are 
% the function $f_k$. $f_k$ takes an initial trace as input and returns the value of $k$ from the initial trace. 
% And the vertices with weight $1$ have the constant function $f_1 : \tdom_0(\kw{twoRounds(k)}) \to \{1\}$ 
% in  $\traceG(\kw{multipleRounds(k)})$. 
% For simplicity, we abuse the same symbols, $f_k$ and $f_1$ for all the following examples in their execution-based dependency graph
% to denote the weight function of a vertex. $f_k$ and $f_1$ compute the same value as defined above with the input initial trace w.r.t. different examples.
The estimated dependency graph $\progG(\kw{multipleRounds(k)})$ is presented in Fig.~\ref{fig:multipleRounds}(b) and we omitted the semantics-based dependency graph $\traceG(\kw{multipleRounds(k)})$ because it has the same topology and only differ in weights.
%
%%%%%%%%%%%%%%%%%%%%%%%%%%%%%%%%%% Previous Version For Reference %%%%%%%%%%%%%%%%%%%%%%%%%%%%%%%%%%
% As the adaptivity definition in our formal adaptivity model in Def.~\ref{def:trace_adapt},
% there is a finite walk along the dashed arrows,
% $a^{6} \to I^9 \to ns^{7} \to  \cdots \to ns^7$ , 
% where every vertex is visited $f_k(\trace_0)$ times given input $\trace_0$.
% The vertex $a^{6}$ has query annotation $1$, and it is visited $f_k(\trace_0)$ times.
% In this sense, the adaptivity of this program is
% $f_k(\trace_0)$ given input $\trace_0$.
% Since $f_k(\trace_0)$ computes the value of input variable $k$ from $\trace_0$, we have
% \begin{equation}
%     \label{eq:adapt_multipleRounds}
%     \forall \trace_0 \in \tdom_0(\kw{multipleRounds(k)}) \st  A(\kw{multipleRounds(k)})(\trace_0) = \env(\trace_0) k
% \end{equation} 
% where $\env$ is the environment operator.
% \jl{
% By the adaptivity definition in Def.~\ref{def:trace_adapt},
% there is a finite walk along the dashed arrows,
% $a^{6} \to I^9 \to ns^{7} \to  \cdots \to ns^7$ , 
% where $a^{6}$, $I^9$ and $ns^{7}$ are visited $w_{a^{6}}(\trace_0)$,
% $w_{I^9}(\trace_0)$ and $w_{ns^{7}}(\trace_0)$
% times respectively with input $\trace_0$.
% The vertex $a^{6}$ has query annotation $1$, and it is visited $w_{a^{6}}(\trace_0)$ times.
% In this sense, the adaptivity of this program is
% $w_{a^{6}}(\trace_0)$ given input $\trace_0$, i.e., $A(\kw{multipleRounds(k)})(\trace_0) = w_{a^{6}}(\trace_0)$.
% Since $w_{a^{6}}(\trace_0)$
% counts the execution times of command $\clabel{\assign{a}{\query(I)}}^6;$,
% this count is at most the loop iteration numbers, i.e., $k$'s initial value, $\env(\trace_0) k$ from the initial trace $\trace_0$.
% }
{
Our program analysis {$\THESYSTEM$} provides a tight upper bound for this example using $\pathsearch(\kw{multipleRounds(k)})$.
It first finds a path on the graph $\progG(\kw{multipleRounds(k)})$
$a^{6}: {}^k_1 \to I^9:{}^k_0 \to ns^7:{}^k_0$ with three weighted vertices. 
Then $\pathsearch$ algorithm transforms this path into a walk $a^{6}: {}^k_1 \to I^9:{}^k_0 \to ns^7:{}^k_0 \to a^{6}: {}^k_1 \cdots$, where $a^6, I^9, ns^{7}$ are all visited $k$ times respectively. 
So $\progA(\kw{multipleRounds(k)}) = k$.
We know for any initial trace $\trace_0$, $\config{\trace_0, k} \earrow \env(\trace_0)k$, i.e., $A(\kw{multipleRounds(k)})(\trace_0) \leq \env(\trace_0)k$ for any $\trace_0$, 
and so what we have produced is a tight and sound bound.
}
% \end{example}
%
\begin{figure}
\centering
\begin{subfigure}{0.25\textwidth}
    \footnotesize{
    $
\begin{array}{l}
\kw{multipleRounds(k)} \triangleq\\
    \clabel{\assign{j}{k}}^0;
    \clabel{\assign{I}{[]}}^1; \\
    \clabel{\assign{ns}{0}}^2; 
    \clabel{\assign{cs}{0}}^3; \\
    \ewhile ~ \clabel{j > 0}^{4} ~ \edo ~ \\
    \Big(
    \clabel{\assign{j}{j-1}}^{5} ;
    \clabel{\assign{a}{\query(I)}}^6; \\
    \clabel{\assign{ns}{\kw{updnscore}(ns, a)}}^7; \\
    \clabel{\assign{cs}{\kw{updcscore}(cs, a)}}^8; \\
    \clabel{\assign{I}{\kw{updI}(I, ns, cs)}}^9
    \Big) 
\end{array}
    $
    }
    \caption{}
\end{subfigure}
        \begin{subfigure}{.65\textwidth}
        \begin{centering}
        \begin{tikzpicture}[scale=\textwidth/25cm,samples=200]
    % Variables Initialization
     \draw[] (-7, 1) circle (0pt) node{{ $I^1: {}^1_{0}$}};
     \draw[] (-7, 7) circle (0pt) node{{$ns^2: {}^{1}_{0}$}};
     \draw[] (-7, 4) circle (0pt) node{{ $cs^3: {}^{1}_{0}$}};
     % Variables Inside the Loop
     \draw[] (0, 10) circle (0pt) node{{ $a^6: {}^{k}_{1}$}};
     \draw[] (0, 7) circle (0pt) node{{ $ns^7: {}^{k}_{0}$}};
     \draw[] (0, 4) circle (0pt) node{{ $cs^8: {}^{k}_{0}$}};
     \draw[] (0, 1) circle (0pt) node{{ $I^9: {}^{k}_{0}$}};
     % Counter Variables
     \draw[] (7, 9) circle (0pt) node {{$j^0: {}^{1}_{0}$}};
     \draw[] (7, 6) circle (0pt) node {{ $j^5: {}^{k}_{0}$}};
     %
     % Value Dependency Edges:
     \draw[  -latex,] (0, 1.5)  -- (0, 3.5) ;
     \draw[ ultra thick, -latex, densely dotted,] (0, 7.5)  -- (0, 9.5) ;
     \draw[  -Straight Barb] (1.4, 4) arc (120:-200:1);
     \draw[  -Straight Barb] (8.5, 6.5) arc (150:-150:1);
     \draw[  -Straight Barb] (1, 7.5) arc (220:-100:1);
     \draw[  -latex] (7, 6.5)  -- (7, 8.5) ;
     % Value Dependency Edges on Initial Values:
     \draw[  -latex,] (-1.5, 1)  -- (-5.5, 1) ;
     \draw[  -latex,] (-1.5, 4)  -- (-5.5, 4) ;
     \draw[  -latex,] (-1.5, 7)  -- (-5.5, 7) ;
     %
     \draw[ ultra thick, -latex, densely dotted,] (-1, 9.5)  to  [out=-130,in=130]  (-1, 1.5);
     \draw[ ultra thick, -latex, densely dotted,] (-0.8, 1.7)  to  [out=-230,in=230]  (-0.5, 6.5);
     % Value Dependency from cs8 -> a6
     \draw[  -latex, ] (-0.8, 4.0)  to  [out=-230,in=230]  (-0.5, 9.5);
     % Value Dependency from a6 -> I1
     \draw[  -latex,] (-1.2, 9.7)  -- (-5.5, 1);
     \draw[  -Straight Barb] (1.7, 1.5) arc (120:-200:1);
     % Control Dependency
     \draw[  -latex] (1.5, 7)  -- (5.8, 6) ;
     \draw[  -latex] (1.5, 4)  -- (5.8, 6) ;
     \draw[  -latex] (1.5, 1)  -- (5.8, 6) ;
     \draw[  -latex] (1.5, 10)  -- (5.8, 6) ;
     % Edges Produced by Transitivity by Control Dependency
     \draw[  -latex] (1.5, 7)  -- (5.8, 9) ;
     \draw[  -latex] (1.5, 4)  -- (5.8, 9) ;
     \draw[  -latex] (1.5, 1)  -- (5.8, 9) ;
     \draw[  -latex] (1.5, 10)  -- (5.8, 9) ;
     % Edges Produced by Transitivity from vertext a6 Dependency
     \draw[  -latex,] (-1.2, 9.7)  -- (-5.5, 4);
     \draw[  -latex,] (-1.2, 9.7)  -- (-5.5, 7);
     \draw[ -latex] (-1, 9.5)  to  [out=-130,in=130]  (-1, 7.5);
     \draw[ -latex] (0.5, 9.5)  to  [out=-50,in=50]  (0.5, 4);
     \draw[  -Straight Barb] (0.5, 10.5) arc (150:-150:1);
     % Edges Produced by Transitivity from vertext cs8 Dependency
     \draw[  -latex,] (-1.2, 4)  -- (-5.5, 1);
     \draw[  -latex,] (-1.2, 4)  -- (-5.5, 7);
     \draw[  -latex,] (0, 4.5)  -- (0, 6.5) ;
     % Edges Produced by Transitivity from vertext I9 Dependency
     \draw[  -latex,] (-1.2, 1)  -- (-5.5, 4);
     \draw[  -latex,] (-1.2, 1)  -- (-5.5, 7);
     \draw[  -latex,] (0.5, 1.0)  to  [out=50,in=-50]  (0.5, 9.5);
     % Edges Produced by Transitivity from vertext ns7 Dependency
     \draw[  -latex,] (-1.2, 7)  -- (-5.5, 1);
     \draw[  -latex,] (-1.2, 7)  -- (-5.5, 4);
     \draw[ -latex] (0.5, 6.5)  to  [out=-50,in=50]  (0.5, 1.5);
     \draw[ -latex] (0.5, 6.5)  to  [out=-50,in=50]  (0.5, 4.5);
     \end{tikzpicture}
     \caption{}
        \end{centering}
        \end{subfigure}
    \vspace{-0.4cm}
    \caption{(a) The simplified multiple rounds example (b) The estimated dependency graph from $\THESYSTEM$}
    \vspace{-0.5cm}
    \label{fig:multipleRounds}
\end{figure}
%