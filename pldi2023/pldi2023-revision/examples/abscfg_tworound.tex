{\footnotesize
\begin{figure} 
    \centering
    \begin{subfigure}{.2\textwidth}
    \begin{centering}
    {\footnotesize
    $
        \begin{array}{l}
              \clabel{ \assign{a}{0}}^{0} ;   
                \clabel{\assign{j}{k} }^{1} ; \\
                \ewhile ~ \clabel{j > 0}^{2} ~ \edo ~ \\
                \Big(
                 \clabel{\assign{x}{\query(\chi[j])} }^{3}  ; \\
                 \clabel{\assign{j}{j-1}}^{4} ;\\
                \clabel{\assign{a}{x + a}}^{5}       \Big);\\
                \clabel{\assign{l}{\query(\chi[k]*a)} }^{6}\\
            \end{array}
    $
    }
    \caption{}
    \end{centering}
    \end{subfigure}
\begin{subfigure}{.35\textwidth}
        \begin{centering}
      \begin{tikzpicture}[scale=\textwidth/22cm,samples=200]
      \draw[] (-7, 10) circle (0pt) node{{ $0$}};
      \draw[] (0, 10) circle (0pt) node{{ $1$}};
      \draw[] (0, 7) circle (0pt) node{\textbf{$2$}};
      \draw[] (0, 4) circle (0pt) node{{ $3$}};
      \draw[] (0, 1) circle (0pt) node{{ $4$}};
      \draw[] (-7, 1) circle (0pt) node{{ $5$}};
      % Counter Variables
      \draw[] (6, 7) circle (0pt) node {\textbf{$6$}};
      \draw[] (6, 4) circle (0pt) node {{ $\lex$}};
      %
      % Control Flow Edges:
      \draw[  -latex] (-6, 10)  -- node [above] {$\top$}(-1.5, 10);
      \draw[ -latex] (0, 9.5)  -- node [left] {$j' \leq k$} (0, 7.5) ;
      \draw[ -latex] (0, 6.5)  -- node [right] {$j > 0 $}  (0, 4.5);
      \draw[ -latex] (0, 3.5)  -- node [right] {$\top $} (0, 1.5) ;
      \draw[ -latex] (-0.5, 1)  -- node [above] {$j' \leq j - 1$} (-6, 1) ;
      \draw[ -latex] (-6, 1.5)  -- node [left] {$\top$} (-0.5, 7)  ;
      \draw[ -latex] (0.5, 7)  -- node [above] {$ j \leq 0 $}  (5.5, 7);
      \draw[ -latex] (6, 6.5)  -- node [right] {$\top$} (6, 4.5) ;
      \end{tikzpicture}
      \caption{}
        \end{centering}
        \end{subfigure}
        \begin{subfigure}{.35\textwidth}
          \begin{centering}
        %   \todo{abstract-cfg for two round}
        \begin{tikzpicture}[scale=\textwidth/22cm,samples=200]
        \draw[] (-10, 10) circle (0pt) node{{ $0: 1$}};
        \draw[] (0, 10) circle (0pt) node{{ $1: 1$}};
        \draw[] (0, 7) circle (0pt) node{\textbf{$2: k$}};
        \draw[] (0, 4) circle (0pt) node{{ $3: k$}};
        \draw[] (0, 1) circle (0pt) node{{ $4: k$}};
        \draw[] (-10, 1) circle (0pt) node{{ $5: k$}};
        % Counter Variables
        \draw[] (6, 7) circle (0pt) node {\textbf{$6: 1$}};
        \draw[] (6, 4) circle (0pt) node {{ $\lex: 1$}};
        %
        % Control Flow Edges:
      \draw[  -latex] (-8, 10)  -- node [above] {$\top$}(-1.5, 10);
      \draw[ -latex] (0, 9.5)  -- node [left] {$j' \leq k$} (0, 7.5) ;
      \draw[ -latex] (0, 6.5)  -- node [right] {$j > 0 $}  (0, 4.5);
      \draw[ -latex] (0, 3.5)  -- node [right] {$\top $} (0, 1.5) ;
      \draw[ -latex] (-1.5, 1)  -- node [above] {$j' \leq j - 1$} (-8, 1) ;
      \draw[ -latex] (-8, 1.5)  -- node [left] {$\top$} (-1.5, 7)  ;
      \draw[ -latex] (1.5, 7)  -- node [above] {$j \leq 0 $}  (4.5, 7);
      \draw[ -latex] (6, 6.5)  -- node [right] {$\top$} (6, 4.5) ;
        \end{tikzpicture}
        \caption{}
          \end{centering}
          \end{subfigure}
      \vspace{-0.5cm}
      \caption{(a) The same $\kw{towRounds(k)}$ program as Figure~\ref{fig:overview-example}
      (b) The abstract control flow graph, $\absG(\kw{twoRounds(k)})$  (c) $\absG(\kw{twoRounds(k)})$ with the reachability bound.}
      \label{fig:abscfg_tworound}
      \vspace{-0.5cm}
    \end{figure}
    }