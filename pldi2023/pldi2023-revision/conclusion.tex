We presented {\THESYSTEM}, a program analysis useful to provide an upper bound on the adaptivity of a data analysis, as well as on the total number of queries asked. This estimation can help data analysts to control the generalization errors of their analyses by choosing different algorithmic techniques based on the adaptivity. Besides, a key contribution of our works is the formalization of the notion of adaptivity for adaptive data analysis. We showed the applicability of our approach by implementing and experimentally evaluating our program analysis.

As future work, we plan to investigate the potential integration of  {\THESYSTEM} in an adaptive data analysis framework like Guess and check by Rogers at al.~\cite{RogersRSSTW20}. As we discussed, this framework is  designed to support adaptive data analyses with limited generalization error. As our experiments show, this framework could benefit from the information provided by {\THESYSTEM} to provide more precise estimate and improved confidence intervals. Another direction we will explore is to make 
the uppper bounds provided by {\THESYSTEM} more precise by integrating our algorithm with a path-sensitive approach.


% estimate address the over-approximation of . Our algorithm may over-estimate the adaptivity of a program, as shown in Section~\ref{sec:examples}, due to its path-insensitive nature. We plan to explore the possibility of making {\THESYSTEM} path-sensitive. We also see the over-approximation when we estimate weight in Section~\ref{sec:alg_weightgen} in some complicated examples with nested while loops, the corresponding improvement is also in our plan.