%
    \begin{example}[The Complete Gradient Decent Optimization Algorithm]
        This example is the gradient decent algorithm example is a generalization of the linear regression on a higher degree data relation.
        It uses gradient decent algorithm to minimize 
        the mean square loss function
        for a two-degree relation
         $y = a_1 \times x_1^2 + a_2 \times x_2 + c$
        on the dataset of two feature columns and one indicator column.
     \[
     %
     \begin{array}{l}
     \kw{gradientDecent(step, rate, t, n)} \triangleq \\
        \clabel{ a_1 \leftarrow 0}^{0} ; \\
        \clabel{ a_2 \leftarrow 0}^{1} ; \\
        \clabel{ c \leftarrow 0}^{2} ; \\
        \clabel{\assign{j}{\kw{step}} }^{3} ; \\
        \ewhile ~ \clabel{j > 0}^{4} ~ \edo ~ \\
      \Big(
          \clabel{\assign{da1}{\query(-2 * (\chi[2] - (\chi[0]^2 \times a_1 + \chi[1] \times a_2 + c)) \times (\chi[0]))} }^{5}  ; \\
          \clabel{\assign{da2}{\query(-2 * (\chi[2] - (\chi[0]^2 \times a_1 + \chi[1] \times a_2 + c)) \times (\chi[1]))} }^{6}  ; \\  \clabel{\assign{dc}{\query(-2 * (\chi[2] - (\chi[0]^2 \times a_1 + \chi[1] \times a_2 + c)))} }^{5}  ; \\
          \clabel{\assign{a_1}{a_1 - \kw{rate} * da1} }^{7}  ; \\
          \clabel{\assign{a_2}{a_2 - \kw{rate} * da2} }^{8}  ; \\
          \clabel{\assign{c}{c - \kw{rate} * dc} }^{9}  ; \\
       \clabel{\assign{j}{j-1}}^{10} 
      \Big);
  \end{array}
     \]
     %
     %
        This approach can be generalized to the regression of a variety of 
        relations in machine learning area.
   %
     \end{example}
%

    \begin{example}[Sequence with Linear Query Dependency]
        \label{ex:seq}
        This example algorithm contains only sequence of four query commands.
        Each of them depends on a previous query.
        The longest dependency depth, i.e., the adaptivity is expectation to be $4$.
        %
        %
        \[
        %
            \kw{seq()} \triangleq 
        \begin{array}{l} 
               \clabel{ \assign{x}{\chi[0]}}^{0} ; 
   \clabel{\assign{y}{\chi[x + 1]} }^{1} ; \\
   \clabel{\assign{z}{\chi[y + 1]}}^{2}; 
    \clabel{\assign{w}{\chi[z + 1]} }^{3}
            \end{array}
        \]
        Evaluation Result: $ \progA( \kw{seq()}) = 4$
        \end{example}
    %
    \begin{example}[Sequence with Query Dependency between Related Variables]
        \label{ex:seqRV}
        %
        This example algorithm contains a sequence of four query commands.
        Each of them depends on one or more of the previous queries.
        The longest dependency depth, i.e., the adaptivity is expectation to be $4$.
        %
        \[
        %
            \kw{seqRV()} \triangleq 
        \begin{array}{l} 
               \clabel{ \assign{x}{\chi[0]}}^{0} ;
   \clabel{\assign{y}{\chi[x + 1]} }^{1} ; \\
   \clabel{\assign{z}{\chi[y + x]}}^{2}; 
    \clabel{\assign{w}{\chi[z + 1] \cdot \chi[y]} }^{3}
            \end{array}
        \]
        Evaluation Result: $ \progA(\kw{seqMultiVar()}) = 4$
    \end{example}
    %
        \begin{example}[If with Data-Value Dependency Separated]
            \label{ex:ifVD}
            This example algorithm contains a $\eif$ command and a query requests
            in each branch.
            Only the query in the first branch depend on the query in the command $0$,
            and the variable in the guard is not assigned by a query request.
            % Each of them depends on one or more of the previous queries.   %
            The longest dependency depth, i.e., the adaptivity is expectation to be $3$.
            \[
            %
            \kw{ifVD}(k) \triangleq 
            \begin{array}{l}
               \quad \clabel{ \assign{z}{\query(\chi[0])}}^{0} ; 
               \quad \clabel{\assign{x}{k / 2} }^{1} ; \\
               \quad \eif(\clabel{x < 0}^2,
               \quad \clabel{\assign{y}{\query(\chi[z])}}^{3},
               \quad \clabel{\assign{y}{\query(\chi[0])}}^{4})
   \end{array}
            \]
            Evaluation Result: $ \progA( \kw{ifVD()}) = 3$
        \end{example}
    
            \begin{example}[If with Data-Control Dependency Overlapped]
   \label{ex:ifCD}
   %
   This example algorithm contains a $\eif$ command and a query requests
   in each branch.
   The variable in the guard is assigned by a query request in command $1$.
   The two queries in the branches depend on the second query in command $1$
   but not depend on the query in the command $0$.
   Even though the variable $x$ isn't used in the query expression in the query $3$ and $4$,
   there are still dependency relation because $x$ is in the guard.
%
The longest dependency depth, i.e., the adaptivity is expectation to be $3$.
   \[
   %
   \kw{ifCD()} \triangleq 
   \begin{array}{l}
\clabel{ \assign{z}{\query(\chi[0])}}^{0} ;
\clabel{\assign{x}{\query(\chi[z])} }^{1} ; \\
\eif(\clabel{x < 0}^{2}, 
\clabel{\assign{y}{\query(\chi[0] + \chi[1])}}^{3}, 
\clabel{\assign{y}\query{(\chi[0])}}^{4})
   \end{array}
   \]
   %
   Evaluation Result: $ \progA( \kw{ifCD()}) = 3$
   \end{example}
    
    
\begin{example}[While with Nested Query Dependency]
\label{ex:whileNested}
This example algorithm contains a simple while loop.
There is one query requests in the loop body at command $3$.
In each iteration, the query request depend on the query result from previous iteration.
The longest dependency depth, i.e., the adaptivity is expectation to be $k$.
%
\[
%
\kw{whileNested}(k) \triangleq
\begin{array}{l}
    \clabel{ \assign{j}{k} }^{0} ; 
    \clabel{ \assign{a}{\query(\chi[0])} }^{1} ; \\
        \ewhile ~ \clabel{j > 0}^{2} ~ \edo ~ \\
        \Big(
         \clabel{\assign{x}{\query(\chi[a]) }}^{3}  ; 
         \clabel{\assign{a}{x + a}}^{4} ;
        \clabel{\assign{j}{j-1}}^{5}       \Big)
    \end{array}
\]
The Evaluation Result: $ \progA(\kw{whileRec}(k)) = 1 + k$
   \end{example}
    %
            \begin{example}[While with Multi-Path Query Dependency]
   \label{ex:whileM}
   %
   This example algorithm contains a simple while loop and a $\eif$ command in the loop body.
% There is one query requests in the loop body at command $3$.
Each branch  has a query request (in the commands $5$ and $6$)
depend on the query at command $1$ and the query at command $7$.
Among the $\frac{k}{2}$ iterations,
% the query at command $7$ depend on the query at line $5$, otherwise not.
 result from previous iteration.
The longest dependency depth, i.e., the adaptivity is expectation to be $1 +2 * \lfloor \frac{k}{2} \rfloor$.
            %
            \[
            %
            \kw{whileM}(k) \triangleq 
            \begin{array}{l}
   \clabel{ \assign{j}{k}}^{0} ; 
   \clabel{ \assign{x}{\query(\chi[0])} }^{1} ; \\
\ewhile ~ \clabel{j > 0}^{2} ~ \edo ~ \\
\Big(
 \clabel{\assign{j}{j-1}}^{3} ;\\
 \eif(\clabel{j \% 2 == 0}^{4}, 
 \clabel{\assign{y}{\chi[x]}}^{5}, 
 \clabel{\assign{w}{\chi[x]}}^{6});\\        
 \clabel{\assign{x}{\query(\chi(\ln(y)))} }^{7} \Big)
   \end{array}
            \]
            The Evaluation Result: $ \progA(\kw{whileM}(k)) = 1 +2 * \lfloor \frac{k}{2} \rfloor $
        \end{example}
    %
            \begin{example}[While with Query Dependency through Related Variables]
   \label{ex:whileRV}
   This example algorithm contains a simple while loop
    and a sequence of three query requests in the loop body.
% There is one query requests in the loop body at command $3$.
In each iteration, every query request depend on one or more
query results from previous iteration.
% the query at command $7$ depend on the query at line $5$, otherwise not.
The longest dependency depth, i.e., the adaptivity is expectation to be $1 +2 * k$.
   \[
   %
   \kw{whileRV}(k) \triangleq 
   \begin{array}{l}
   \clabel{\assign{j}{k} }^{0} ; 
   \clabel{ \assign{x}{\query(\chi[0])}}^{1} ; 
\clabel{ \assign{y}{\query(\chi[1])}}^{2} ; \\
    \ewhile ~ \clabel{j > 0}^{3} ~ \edo ~ \\
    \Big(
     \clabel{\assign{j}{j-1}}^{4} ;
     \clabel{\assign{z}{\query(\chi(x + \ln(y)))} }^{5}  ; 
     \clabel{ \assign{x}{\query(\chi[z])}}^{6} ; 
     \clabel{ \assign{y}{\query(\chi[z])}}^{7} 
    \Big)
\end{array}
   \]
   The Evaluation Result: $ \progA(\kw{whileRV}(k)) = 1 + 2 * k $
            \end{example}
   %
   %
   \begin{example}[While with Query Dependency trhough Control Flow and Data Flow]
\label{ex:whileVCD}
%
This example algorithm contains a simple while loop
and a sequence of three query requests in the loop body.
The variable in the guard is assigned by a query request in command $0$.
In each iteration, the query at $3$ depends on either the query at line $1$, and the query result at line $4$ from the previous iteration.
%  in the branches depend on the second query in command $1$
In each iteration, the query at $4$ depends on either the query at line $0$ and the query at line $3$ in the same iteration.
% Even though the variable $x$ isn't used in the query expression in the query $3$ and $4$,
The longest dependency depth, i.e., the adaptivity is expectation to be $1 +2 * k$.
\[
\kw{whileVCD}() \triangleq
\begin{array}{l}
    \clabel{ \assign{x}{\query(\chi[0])} }^{0} ; 
    \clabel{ \assign{z}{\query(\chi[0])} }^{1} ; \\
        \ewhile ~ \clabel{x > 0}^{2} ~ \edo ~ \\
        \Big(
        \clabel{\assign{x}{\query(\chi(z))} }^{3}  ; 
        \clabel{\assign{z}{\query(\chi(x))}}^{4}
      \Big)
    \end{array}
\]
The Evaluation Result: $ \progA(\kw{whileVCD}(k)) = 1 + 2 * k $
   \end{example}
    %
   \begin{example}[While with Multiple Path Query Dependency Dependency]
\label{ex:whileMPVCD}
%
This example algorithm contains a simple while loop and a $\eif$ command in the loop body.
% There is one query requests in the loop body at command $3$.
Each branch  has a query request (in the commands $5$ and $6$)
depend on either the query at command $1$ or the query at command $7$.
% Among the $\frac{k}{2}$ iterations,
% % the query at command $7$ depend on the query at line $5$, otherwise not.
%  result from previous iteration.
The longest dependency depth, i.e., the adaptivity is expectation to be $2 + k$.
\[
    %
    \kw{whileMPVCD}(k) \triangleq
    \begin{array}{l}
        \clabel{ \assign{x}{\query(k)}}^{0} ; 
        \clabel{\assign{y}{0} }^{1} ; 
            \ewhile ~ \clabel{x > 0}^{2} ~ \edo ~ \\
            \Big(
             \eif(\clabel{y > 0}^{3}, 
             \clabel{\assign{y}{\query(\chi[12])}}^{4}, 
             \clabel{\assign{w}{\query(\chi[9])}}^{5});        
             \\
             \clabel{\assign{x}{x-1}}^{6}\Big);\\
             \clabel{\assign{y}{\query(\chi(\ln(y)))} }^{7} 
        \end{array}
    \]
    The Evaluation Result: $ \progA(\kw{whileMPVCD}(k)) = 2 + k $
\end{example}
   %
\begin{example}[Nested While with Nested Query Dependency]
    \label{ex:nestWhileVD}
    %
    This example algorithm contains two nested while loops.
    The query in the outer loop at line $5$ depends on either the query at line $1$ or
    the query results at line $8$ from the previous iteration of the inner loop.
    The longest dependency depth, i.e., the adaptivity is expectation to be $2 + k^2$.
        %
    \[
    %
    \kw{nestWhileVD}(k) \triangleq 
    \begin{array}{l}
        \clabel{ \assign{i}{k} }^{0} ; 
        \clabel{\assign{x}{\query(\chi[0])}}^{1} ; \\
            \ewhile ~ \clabel{i > 0}^{2} ~ \edo ~ 
            \Big(
             \clabel{\assign{i}{i-1}}^{3} ;
             \clabel{\assign{j}{k}}^{4} ;
             \clabel{\assign{y}{\query(\chi(\ln(x)))} }^{5}  ; \\
             \ewhile ~ \clabel{j > 0}^{6} ~ \edo ~ 
             \Big(
              \clabel{\assign{j}{j-1}}^{7};
              \clabel{\assign{x}{\query(\chi(\ln(x)))} }^{8}
              \Big) \Big)
        \end{array}
    \]
    The Evaluation Result: $ \progA(\kw{nestWhileVD}(k)) = 2 + k^2 $
\end{example}
    
    \begin{example}[Nested While with Query Dependency through Related Variables]
        \label{ex:nestedWhileRV}
        %
        This example algorithm contains two nested while loops, one query in the outer loop, and one query in the inner loop.
        The query in the outer loop at line $8$ depends on only the query result at line $7$
        from the last iteration of the inner loop.
        %  either the query at line $1$ or
        However, the query at line $7$ depends on  either the query at line $1$ 
        the query results at line $8$ from the previous iteration.
        The longest dependency depth, i.e., the adaptivity is expectation to be $1 + 2 * k $.
            %
        \[
        %
            \kw{nestWhileRV}(k) \triangleq 
        \begin{array}{l}
            \clabel{ \assign{i}{k} }^{0} ; 
            \clabel{\assign{x}{\query(\chi[0])}}^{1} ; \\
   \ewhile ~ \clabel{i > 0}^{2} ~ \edo ~ 
   \Big(
    \clabel{\assign{i}{i-1}}^{3} ;
    \clabel{\assign{j}{k}}^{4} ;\\
    \ewhile ~ \clabel{j > 0}^{5} ~ \edo ~ 
    \Big(
     \clabel{\assign{j}{j-1}}^{6};
     \clabel{\assign{y}{\query(\chi(x) + \chi(1))} }^{7}
     \Big); \\
    \clabel{\assign{x}{\query(\chi(\ln(y)))} }^{8}
     \Big)
            \end{array}
        \]
        The Evaluation Result: $ \progA(\kw{nestWhileRV}(k)) = 1 + 2 * k $
    \end{example}
%
   
        \begin{example}[Nested While with Nest Query Dependency and Related Variable Accross Outer and Inner Loop]
            \label{ex:nestedWhileMR}
            %
            This example algorithm contains two nested while loops, one query in the outer loop, and one query in the inner loop as well.
            The two queries depend on both the query results assigned to themselves in previous iteration.
            The longest dependency depth, i.e., the adaptivity is expectation to be $1 + k + k^2 $.
            \[
            %
            \kw{nestWhileMR}(k) \triangleq 
            \begin{array}{l}
                \clabel{\assign{i}{k} }^{0} ; 
                \clabel{ \assign{x}{\query(\chi[0])}}^{1} ; 
                \clabel{ \assign{y}{\query(\chi[1])}}^{2} ; 
                \ewhile ~ \clabel{i > 0}^{3} ~ \edo ~ \\
                \Big(
                \clabel{\assign{i}{i-1}}^{4} ;
                \clabel{\assign{j}{k}}^{5} ;
                \clabel{\assign{y}{\query(\chi(\ln(x) + y))} }^{6}  ; \\
                \ewhile ~ \clabel{j > 0}^{7} ~ \edo ~ 
                \Big(
                \clabel{\assign{j}{j-1}}^{8};
                \clabel{\assign{x}{\query(\chi(\ln(y))+\chi[x])} }^{9}
                \Big) \Big)
            \end{array}
            \]
            The Evaluation Result: 
            $ \progA(\kw{nestWhileMR}(k)) = 1 + k + k^2$
            \\
            Reachability Bound The Evaluation Result: \\
            weight for Variable: j of label 6 is: 0 + 0 + 1 * k * k\\
            weight for Variable: y of label 7 is: 0 + 0 + 1 * k * k\\
            weight for Variable: j of label 4 is: 0 + 1 * k\\
            weight for Variable: i of label 3 is: 0 + 1 * k\\
            weight for Variable: x of label 8 is: 0 + 1 * k\\
            weight for Variable: x of label 1 is: 1\\
            weight for Variable: i of label 0 is: 1\\
            \end{example}
            \begin{example}[Nested While with MultiplePath and Nested Recursive Multiple Variable 
   Data-Value Dependency Across Outer and Inner Loop]
   \label{ex:nestedWhileMPRV}
   %
   We then show a more complex example with nested while command and nested data-flow across the outer and inner while loop through multiple variables.
   This example also contains the if command with data dependency occurred through the if guard.
   The longest dependency depth, i.e., the adaptivity is expectation to be $1 + k + k^2 $.
   %
   \[
   %
   \kw{nestWhileMPRV}(k) \triangleq 
   \begin{array}{l}
\clabel{\assign{i}{k} }^{0} ; 
\clabel{ \assign{x}{\query(\chi[0])}}^{1} ; 
\clabel{ \assign{y}{\query(\chi[1])}}^{2} ; \\
    \ewhile ~ \clabel{i > 0}^{3} ~ \edo ~ 
    \Big(
     \clabel{\assign{i}{i-1}}^{4} ;
     \clabel{\assign{j}{k}}^{5} ;\\
     \eif(\clabel{x > 0}^6, \clabel{\assign{y}{\query(\chi(\ln(x) + y))} }^{7},
     \clabel{\assign{y}{\query(\chi(x))} }^{8} )
      ; \\
     \ewhile ~ \clabel{j > 0}^{9} ~ \edo ~ 
     \Big(
      \clabel{\assign{j}{j-1}}^{10};
      \clabel{\assign{x}{\query(\chi(\ln(y))+\chi[x])} }^{11}
      \Big) \Big)
\end{array}
   \]
   \end{example}
   The Evaluation Result: $ \progA(\kw{nestWhileMPRV}(k)) = 1 + k + k^2$
   \\
   Reachability Bound The Evaluation Result: \\
            weight for Variable: j of label 10 is: 0 + 0 + 1 * k * k \\
   weight for Variable: x of label 11 is: 0 + 0 + 1 * k * k \\
   weight for Variable: y of label 7 is: 0 + 1 * k \\
   weight for Variable: y of label 8 is: 0 + 1 * k \\
   weight for Variable: j of label 5 is: 0 + 1 * k \\
   weight for Variable: i of label 4 is: 0 + 1 * k \\
   weight for Variable: y of label 2 is: 1 \\
   weight for Variable: x of label 1 is: 1 \\
   weight for Variable: i of label 0 is: 1 \\